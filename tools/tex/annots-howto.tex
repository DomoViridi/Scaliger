% XeLaTeX can use any Mac OS X font. See the setromanfont command below.
% Input to XeLaTeX is full Unicode, so Unicode characters can be typed
% directly into the source.

% The next lines tell TeXShop to typeset with xelatex, and to open and
% save the source with Unicode encoding.

%!TEX TS-program = xelatex
%!TEX encoding = UTF-8 Unicode

%%% Count out columns for fixed-width source font
% 000000011111111112222222222333333333344444444445555555555666666666677777777778
% 345678901234567890123456789012345678901234567890123456789012345678901234567890

\documentclass{article}

\usepackage{geometry}
%% See geometry.pdf to learn the layout options. There are lots.

\geometry{a4paper}         %% ... or a4paper or a5paper or ...
%\geometry{landscape}      %% Activate for rotated page geometry

%\usepackage[parfill]{parskip}
%% Activate to begin paragraphs with an empty line rather than an indent
%\usepackage{graphicx}
%\usepackage{amssymb}

% Will Robertson's fontspec.sty can be used to simplify font choices.
% To experiment, open /Applications/Font Book to examine the fonts provided
% on Mac OS X,
% and change "Hoefler Text" to any of these choices.

\usepackage{fontspec,xltxtra,xunicode}
%\setmainfont{Hoefler Text}

\setmainfont{Times New Roman}
% Chosen font needs to be able to show Greek characters koppa and sampi.
% Preferably should also have small caps.

%\defaultfontfeatures{Mapping=tex-text}
%\setromanfont[Mapping=tex-text]{Hoefler Text}
%\setsansfont[Scale=MatchLowercase,Mapping=tex-text]{Gill Sans}
%\setmonofont[Scale=MatchLowercase]{Andale Mono}

% Use color to mark the hyperlinks
\usepackage{color}

%%% Booktabs package for nice looking tables
\usepackage{booktabs}

% Allow hypertext links to external documents/websites
% Hyperref manual advises to load this last
% Bidi insists you load hyperref *before* bidi (which is part of polyglossia)
\usepackage[colorlinks]{hyperref}

%% Command to create a hyperlink to an Issue on GitHub
%% Parameter: the issue number on GitHub
\newcommand{\SeeIssue}[1]{%
See \textcolor{blue}{%
\href{https://github.com/DomoViridi/Scaliger/issues/#1}%
{Issue \#{#1}}%
} on GitHub%
}

\title{How to add line number annotations to a PDF}
\author{Erik Groenhuis}
\date{\today} % Activate to display a given date or no date

\begin{document}
\maketitle

\tableofcontents{}

% For many users, the previous commands will be enough.
% If you want to directly input Unicode, add an Input Menu or Keyboard
% to the menu bar using the International Panel in System Preferences.
% Unicode must be typeset using a font containing the appropriate characters.
% Remove the comment signs below for examples.

% \newfontfamily{\A}{Geeza Pro}
% \newfontfamily{\H}[Scale=0.9]{Lucida Grande}
% \newfontfamily{\J}[Scale=0.85]{Osaka}

% Here are some multilingual Unicode fonts: this is Arabic text:
% {\A السلام عليكم}, this is Hebrew: {\H שלום},
% and here's some Japanese: {\J 今日は}.

%%=====
%\chapter{First Chapter}



%%---
\section{Summary}
\begin{enumerate}
\item Generate line number annotations in XFDF form with the xfdfnotes program.
\item Load the generated annotations into Adobe Acrobat,
 together with the source PDF.
\item Manually adjust the annotations to the desired position.
\item Export the annotations as a new XFDF file.
\item Clean up the XFDF with the xsltproc command.
\item Import the clean XFDF into the source PDF.
\item Export the modified PDF
\end{enumerate}


%%---
\section{Detailed steps}

%--
\subsection{Generate line number annotations}
\texttt{xfdfnotes} is a program written in \texttt{C}
(source in \texttt{xfdfgen.c}) by domoviridi
which generates an xfdf file to stdout.
This XFDF file describes all the annotations to be added to one or more
chapters of the PDF original.
These annotations are of the "freetext" type,
with the simple "content" subtype.

The program uses a configuration file. The name of this configuration file
(usually called 'annots.cfg') is passed to the program as a parameter.
Example invocation:
  
\begin{verb}
         xfdfnotes annots.cfg >notes.xfdf
\end{verb}

The values in the configuration file can be tweaked to make the line numbers
appear more or less in the correct position in the PDF. 

%--
\subsection{Load annotations into Adobe Acrobat}
\label{subsec:load}
Adobe Acrobat is used to combine the annotations with the source PDF.
Preferably use a "clean" version of the PDF, i.e. one without any annotations
in it.
Acrobat is very slow while handling files with many annotations.
The fewer annotations there are, the quicker it responds.

The PDF original and the XFDF file can be combined in two ways:
\begin{itemize}
\item Load the XFDF file into Adobe Acrobat.
The PDF will then be automatically loaded too.
The name and path of the PDF is in the XFDF file
(as an \verb+<f>+ tag, near the end of the file)
\item First load the PDF into Adobe Acrobat and then import the XFDF file.
\end{itemize}

    
%--
\subsection{Adjust the annotations}
The annotations need to be manually adjusted to match the scanned text.
Most commonly the line numbers need to be moved up or down slightly.
  Sometimes tables, images or section headings make it necessary to remove
  some of the higher line numbers.
Remember that only body text lines should be inluded in the line count.
  
This adjustment needs to be done page-by-page, by hand.
It is advisable to start with the last page and work backwards.
That way Adobe Acrobat will run more smoothly.

%--
\subsection{Export the annotations}
The annotations can then be exported again from Adobe Acrobat as an xfdf file.
\begin{itemize}
\item Select \textbf{Comment} in the toolbar on the right.
\item Select the \textbf{...} for a menu of more options,
and select \textbf{Export All To Data File ...} from that menu.
\item In the dialogue window that opens
look for \textbf{Format:} and select \textbf{Acrobat XFDF Files}.
\item At \textbf{Save As:} set at suitable name, and the directory you want.
\end{itemize}

Note: do not simply save the PDF from Adobe Acrobat, as this will give a
very bloated version of the file (800 MB rather than 78 MB).

%--
\subsection{Clean the XFDF}
The annotations exported in the previous step
 wil be of the \emph{freetext} type,
 but with the \emph{contents-richtext} subtype.
These can be converted back to the much simpler and compact form,
 using the \emph{freetext/content} type again.
The script \texttt{'annotform.xslt'}
 was developed to perform this transformation.
Command:
\begin{quote}
  \texttt{xsltproc tools/annotform.xslt 1629-export.xfdf >nuts.xfdf}
\end{quote}
Where:
\begin{description}
\item[\texttt{xsltproc}] is a command  which perform XSL transformations.
It can transform XML documents into other XML documents.
The command is part of the XSLT library for GNOME, and can be found
on \texttt{xmlsoft.org}.
\item[\texttt{tools/annotform.xslt}] is a specialy crafted script,
which is included in this package.
\item[\texttt{1629-export.xfdf}] is the XML file describing the annotations
which was produced in the previous step.
\item[\texttt{>nuts.xfdf}] redirects the output of the command to a file,
in this case called \texttt{nuts.xfdf}
\end{description}

%--
\subsection{Import the clean XFDF}
Adobe Acrobat can then be used to load the source PDF and import the
simplified xfdf, similarly to the description in section \ref{subsec:load}.
  
  
%--
\subsection{Export the new PDF}
Without making any changes, save the PDF.
If all went well, you now have a PDF with the new line numbers added,
which is still less than 100 MB in size.
\end{document}
