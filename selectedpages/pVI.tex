% VI
% {PDF page nr}{source page nr}{line nr}
\Rplnr{33}{6}{2}Continuantur autem saepernumero in aliquo reliquorum mensium
duo Sabbata: idque fit, quando solenne est aut feria prima, aut feria
sexta.
\lnr{4}Quorum alterutrum quotannis incidere, nisi quando Tisri
incipit feria tertia, Doctor ignoravit.
\lnr{5}In primam feriam incidunt
haec solennia, \rnum{xxv} Casleu, et \rnum{x} Tebeth in anno defectivo tam
communi, quam embolimaeao, quotiescunque Tisri incipit feria secunda:
\rnum{vi} Sivvan; quando Nisan incipit feria septima:
\rnum{xv} Nisan, \rnum{xvii} Tamuz,
\rnum{ix} Ab, quando Nisan incipit feria prima.
\lnr{9}In feriam autem sextam
convenit solenne \rnum{xxv} Casleu et
 \rnum{x} Tebeth, quando Tisri est feria
septima in anno tam communi, quam embolimaeo.
\rnum{xiiii} Adar, quando
Nisan sequens est feria prima: \rnum{vi} Sivvan, quando Nisan feria quinta.
\lnr{13}Vides, quot Sabbata quotannis, nisi quando Tisri incipit
 feria tertia, Iudaei
continuent in aliquo mensium, praeterquam in solo Tisri, cuius
unius gratia illa cautio instituta.
% No period at end of sentence
\lnr{15}Itaque doctor tam frustra, quam ridicule
Iosephi testimonium adduxit de sexta Sivvan, id est, Pentecoste
feria prima; cum illo anno neomenia Nisan fuerit Sabbatum.
\lnr{17}Atqui
nihil superesse putavit, quam ut Vaticani montis imago redderet
\textgreek{ἰὴ παιαν[?]}.
\lnr{19}Sed ipse valde ignarus est harum rerum, ut reliqui omnes,
qui contendunt novitium esse Iudaeorum commentum.
\lnr{20}Nos
validissime demonstravimus, et saeculo Christi, et retro sub Seleucidis
translationes in usu fuisse.
\lnr{22}Et sane res peruetusta est.
\lnr{22}Quae tamen
non minus ignorata, quam periodus Calippica, qua Seleucidae, et
Seleucidarum edicto Iudaei usi.
\lnr{24}Quod non solum ex Nisan anni excidii
Hierosolymorum a nobis demonstratum est, sed etiam patet
ex definitione Rabbi Adda.
\lnr{26}Is annum definit dierum \rnum{ccclxv},
horarum 5, \myfrac{997}{1080}. \myfrac{48}{76}.
\lnr{27}Quid hac definitione aliud vult, quam periodum
Iudaicam fuisse annorum 76?
\lnr{28}Cum Meto definit annum dierum
365. hor. 5. \myfrac{1}{19}. ex eo coniiciendum relinquit,
 se uti periodo annorum
19.
\lnr{30}Utebantur igitur periodo 76 annorum, id est, Calippica:
et tamen in omnibus neomeniis Lunae \textgreek{φάσιν} observabant, non
quod eam ex praescripto periodi non indicerent, sed ideo, ut eam
sanctificarent.
\lnr{33}Nam et hodie quoque observant \textgreek{φάσιν}, non ut ex ea
neomeniam indicant, sed ut eam sanctificent.
\lnr{34}Itaque Luna statim
visa dicunt: \texthebrew{[Hebrew]}.
\lnr{35}\textgreek{ἀγαθὸν τέρας ἔστω ἡμῖν καὶ παυτὶ Ισραήλ.}
\lnr{36}Idem faciunt et Muhammedani, quamuis neomenias ex
scripto indicere soleant.
\lnr{37}Neque aliud intellexit fabulosus quidem,
sed tatem vetus auctor \textgreek{περιόδῳν Πέτρου ἀποστόλου[?]} apud Clementem:
\textgreek{μηδὲ κατα Ιουδαίους σἔβεαθε (τὸν θεὸν.)[?]}.
\lnr{39}\textgreek{Καὶ γὰρ ἐκεῖνοι μόνοι οἰόμηνοι τὸν θεὸν
γινώσκειν, οὐκ ἐπίστανται λατρέυοντες ἀγγέλοις κὰι ἀρχαγγελοις, μηνὶ κὰι σελένῃ.[?]}
\lnr{41}\textgreek{Καὶ ἐὰν μὴ ἡ σελήνη φανῇ,
 σάββατον οὐκ ἄγουσι τὸ λεγόμεν πρῶτον, ὀυδὲ νεομηνίαν
ἄγουσιν, ὄυτε ἄζυμα, ὄυτε έορτὴν, ὄυτε μεγάλην ἡμέραν.[?]}
\lnr{42}Praeclara quidem ista: sed nescit, quid dicit.
%
