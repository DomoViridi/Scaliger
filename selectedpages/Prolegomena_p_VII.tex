\plnr{VII}{1}Nam in Iudaeorum potestate
nunquam fuit, ut exspectarent \textgreek{φάσιν}:
% Greek: phase
 quia raro Luna se ostendit,
nisi secundo post coitum die.
\lnr{3}Quod si expectandum ipsis esset,
res ridicula accideret, ut Elul, qui semper est cavus mensis, non solum
plenus, sed etiam aliqando unius et triginta dierum esset.
\lnr{5}Sine dubio translationem feriae intelligit, cuius caussam ignorat.
\lnr{6}\textgreek{πρῶτον σάββατον} vocat
 \texthebrew{רֹאשׁ הַשָּׁנָה‎}
% Rosh Hashanah
 caput anni.
\lnr{7}Nam Sabbatum vocat, quia Festus
dies, \textgreek{κὶα᾽εργός}[?].
\lnr{8}Ita etiam vocatur Levitici \textsc{xxiii}, 24.
% Leviticus 23:24: “Tell the people of Isra’el, ‘In the seventh month, the
% first of the month is to be for you a day of complete rest for remembering,
% a holy convocation announced with blasts on the shofar.'"
% λάλησον [Speak] τοις [to the] υιοίς [sons] Ισραήλ [of Israel,] λέγων [saying!]
% του [The] μηνός [(²month] του [] εβδόμου [¹seventh),] μία [day one]
% του [of the] μηνός [month] έσται [will be] υμίν [to you] ανάπαυσις [a rest,]
% μνημόσυνον [a memorial] σαλπίγγων [of trumpets,] κλητή [(²convocation]
% αγία [¹a holy)] τω [to the] κύριος [LORD.]
\lnr{8}\textgreek{ἑορτὴν}[?]
% feast
 intellige
\textgreek{κατ᾽ ἐξοχήν τὴν πεντηκοστήν}:
% eminently the Pentekost [?]
 quod ita Hebraice vocetur, nempe \texthebrew{עֲצֶרֶת}[?] ([sh'miní] 'atséret).
% "The eighth [day] of assembly".
\lnr{10}Vide in Computo Iudaico.
\lnr{10}At \textgreek{μεγάλην ἡμέραν}
% great day
 vocat \textgreek{τὴν σκηνοπηγιαν, κατ᾽ ἐξοχήν}
 quoque, id est \texthebrew{הַנ}[?].
\lnr{11}Nam aliae erant \textgreek{μεγάλαι ἡμέραι},
% Great Days
proinde ut et \texthebrew{חַנִּים}[?].
\lnr{12}Sic Tertullianus magnos dies dixit, quos
Hebraei \texthebrew{[Hebrew]} vel \texthebrew{[Hebrew]}.
\lnr{13}Eius verba sunt ex v in Marcionem:
\textit{Dies observatis, et menses, et tempora, et annos, et Sabbata, ut opinor,
et cenas puras, et ieiunia, et dies magnos.}
% Tertullianus: De Adversus Marcionem, Book 5, chapter 4, section 6.
\lnr{15}Sed quid Tertullianum
advoco?
\lnr{16}Ecce Biblia Graeca ita vertunt ex primo caput Isaiae:
\textgreek{τὰς νουμἠνίας ὑμῶν, καὶ τὰ σάββατα,
 καὶ ἡμέραν μεγάλην οὐκ ἀνέχομαι}[?].
% Isaiah 1:13 ?: I cannot bear your new moons, and your sabbaths,
% and the great day;
\lnr{18}Quod Hebraice est \texthebrew{[Hebrew]},
 vertunt \textgreek{[Greek]}, quod idem
est quod \texthebrew{[Hebrew]}: et quidem manifesto Sabbata distinguuntur a
magnis diebus. 
\lnr{20}Quare perperam quidam \textgreek{[Greek]} interpretantur
Sabbatum apud Iohannem, \textgreek{[Greek]}.
\lnr{22}De quo infra.
\lnr{22}Quin et Tertullianus ipse \textgreek{[Greek]},
quas cenas puras vocat, a diebus magnis, et a ieiuniis, et a
Sabbatis distinguit.
\lnr{24}De Cena pura, praeter id quod diximus ad
Festum, ita reperi in veteri et peroptimo Glossario Latinoarabico:
\textit{Parasceue, cena pura, id est, praparatio, que fit prosabbato.}
\lnr{27}Conditor Annalium Ecclesiasticorum turbat de cena
pura, et negat esse parasceuen, quia cena pura apud Festum
habeat offam suillam.
\lnr{29}Sed ipse, (pace docti viri dixerim) non
aduertit Puram dici, non quia careat carnibus, sed quia religionis
et dicis caussa fit.
\lnr{31}Nam et parasceuae Iudaicae habent carnes,
et nihilominus dicuntur cenae purae, quod dicis caussa coquebantur,
coquunturque hodie prosabbato, quia in Sabbato
coqui non liceat.
\lnr{34}Non negabis, candide Lector, haec vulgo non intelligi.
\lnr{35}Itaque locus ille est nobilissimus. 
\lnr{35}Tamen quotus quisque est ex tot Lectoribus, qui non haec aut praeteribit,
aut calumniabitur?
\lnr{37}Sequuntur periodi magnae Hagerenorum,
ex quibus ratio anni soluti Indorum, et Muhammedanorum
tota pendet.
\lnr{39}Omnia nunc primum ex Arabum scriptis
prodeunt: atque adeo omnis tractatio nostris hominibus
nova est.
%\end{parnumbers}
%\clearpage
\pnr{p. VIII [pdf 35]}
%\begin{parnumbers}
\lnr{41}Excipit hanc doctrina anni Iudaici hodierni, res, quod
saepe diximus, artificiosissima, ideoque eximia, quia melior
anni Lunaris forma constitui non potest.
