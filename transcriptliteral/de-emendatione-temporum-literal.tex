% !TEX TS-program = xelatex
% !TEX encoding = UTF-8 Unicode
% this template is specifically designed to be typeset with XeLaTeX;
% it will not work with other engines, such as pdfLaTeX

%%% Count out columns for fixed-width source font
% 000000011111111112222222222333333333344444444445555555555666666666677777777778
% 345678901234567890123456789012345678901234567890123456789012345678901234567890

%% use [draft] for a draft version with boxes instead of images
%% Paper size: a4paper (modern size) or legalpaper (close to original)
%% Original size = 217.6 x 338.3 mm
%% Legal size    = 215.9 x 355.6 mm
\documentclass[12pt,twoside,legalpaper]{book}

%%% Geometry package to make the margins better match the original
%% use [showframe] to show the frames
\usepackage[top=18mm, bottom=55mm, inner=24mm, outer=38mm]{geometry}

%%% Allow generation of pseudo-latin body text for testing
%% Optional parameter [x-y] where x and y are numbers: generate paragraph
%% number x through y (150 paragraphs available).
\usepackage{lipsum}

%%% Array package so we can use m{'width'} and adjust the row spacing in tables
\usepackage{array}

%%% Control the layout of chapter headers
\usepackage{titlesec}

%%% Disable chapter/section numbering.
%%  Put Chapters/Sections/Subsections in the Table of Contents
\setcounter{secnumdepth}{0}
\setcounter{tocdepth}{3}

%%% Allow the text to flow around tables and figures.
%% (Must load before bidi package, i.e. before polyglossia package)
\usepackage{wrapfig}

\usepackage{fontspec} % Openfont specifications for XeTeX;
% fontspec automatically loads xltxtra and xunicode, both of which are needed.
% Though using \vfrac{}{} (part of xltxtra) without explicit loading of
% xltxtra does not work.
\usepackage{xltxtra}

%%% Main body text is in latin. We want:
%% The long s (automaticaly convert 's' in the text to long s, except at 
%% the end of words): Contextuals = {Inner}
%% 'ct' ligatures (not 'st'; use 'ſt' instead)
%% A capital 'Q' with a swish: Alternate = 1,
%%  but this makes the wrong '&' and too fancy italics. We suppress this
%%  by adding 'ItalicFeatures={Alternative=0}'
%% Common ligatures for 'fi' 'ffi' etcetera
%% Similar ligatures for long s
%% Allow (or convert to) 'ae' and 'oe' ligatures.
%% Old style numbers

\setmainfont{Hoefler Text}[
    Alternate   = 1,
    Ligatures   = {Common},
    Contextuals = {Inner},
    ItalicFeatures = {Alternate=0},
    SmallCapsFeatures = {Color=FF4422,LetterSpace=30.0},
    LetterSpace = 1.0
]

%%% Add a font that supports Greek
\newfontfamily\greekfont{Arial Unicode MS}

%%% Add a font that supports Hebrew
\newfontfamily\hebrewfont{Arial Unicode MS}

%%% Add a font that supports Arabic
\newfontfamily\arabicfont{Arial Unicode MS}

%%% Add a font that has Astrological symbols
\newfontfamily\astrofont{Menlo}
%% Command to allow "\astro{some text}"
\newcommand\astro[1]{{\astrofont #1}}

%%% Add a font for dropcaps. The same as mainfont, but
%%% without the swash on the Q (which sadly runs into the main text)
%%% The Q is required for e.g. PDF p. 111
\newfontfamily\dropcapfont{Hoefler Text}[Color=880088]

%%% Command to make header of any size
%% use: head{<scalesize>}{<spacing>}{<text>}
%% <scalesize> is a sizefactor relative to normal size
%% <spacing> is the LetterSpace parameter
\newcommand\head[3]{{\normalsize\fontspec{Hoefler Text}[
    Scale=#1,
    LetterSpace=#2,
    Color=4422FF
] #3}}

%%% Commands to number paragraphs
%% http://tex.stackexchange.com/questions/10513/automatically-assign-a-number-to-every-paragraph
%% get marginpars to always show up on the correct side (need to compile twice)
\usepackage{mparhack}

%% No indent on paragraphs, because we make a paragraph out of each sentence.
\setlength\parindent{0cm}

%% Command \parnum to format the paragraph counter number
%% (in bold, arabic numbers)
\newcommand\parnum{%
    \bfseries\footnotesize\arabic{parcount}%
}

%% Special paragraph counter which resets with every 'subject'
%% (represented as a subsection)
\newcounter{parcount}[subsection]

%% Command for an isolated paragraph counter mark
%% Useful e.g. for big, multi-paragraph spanning Initials
\newcommand\p{%
    \stepcounter{parcount}%
    \leavevmode\marginpar[\hfill\parnum]{\parnum}%
}

%% Define an environment to use in the source.
\newenvironment{parnumbers}{%
    \par%
    \everypar{%
        \stepcounter{parcount}%
        \leavevmode\marginpar[\hfill\parnum]{\parnum}%
    }%
}{}

%% Our own cookup to put letters in the margin to indicate quarters of a
%% page (A-D) as used in the original.
%% Uses mparhack.
%% Because of limitations of \marginpar{} the margin letters must go
%% on the same side as the paragraph numbers. We put them a bit further
%% away from the body text to make them more distinct.
\newcommand\mletter[1]{%
    \leavevmode\marginpar[\hfill\textbf{#1}\hspace*{24pt}]%
    {\hspace*{24pt}\textbf{#1}}%
}

%%% Make drop caps (illuminated initial letters)

%% Add fancy font for Initials
%% https://www.lemald.org/blog/?p=108
\input GoudyIn.fd
\newcommand*\initfamily{\usefont{U}{GoudyIn}{xl}{n}}

%% Variables used by the dropcap commands to calculate the scaling required
%% to make the initial the right hight
\newlength{\DCunit}  % Height unit for font factor
\newlength{\DCscale} % Font scaling value

%% Define a command for a dropcap where multiple paragraphs automatically
%% wrap around the text (unlike the \lettrine command)
%% This works for all upper case characters that do not have a descender
%% (i.e. all letters except Q)
%% Syntax: \dropcap{lines}{initial}{caps}
\newcommand\dropcap[3]{%
    \setlength{\intextsep}{-0.2ex}%
    \setlength{\DCunit}{1.4\baselineskip}%
    \setlength{\DCscale}{#1\DCunit}%
    \addtolength{\DCscale}{-0.5\DCunit}%
    \setlength{\columnsep}{1pt}\begin{wrapfigure}[#1]{l}{0mm}{\dropcapfont\fontsize{\DCscale}{1em}\selectfont #2}\end{wrapfigure}%
    \textsc{#3}%
}

%% Special version of the dropcap command for the letter Q
%%
%% Syntax: \dropcapQ{lines}{initial}{caps}
%% (Note: the 'initial' parameter is kept to make this command conform
%% to the other versions. Normally the user would enter a 'Q' for this
%% parameter.)
\newcommand\dropcapQ[3]{%
    \setlength{\DCunit}{1.1\baselineskip}%
    \setlength{\DCscale}{#1\DCunit}%
    \addtolength{\DCscale}{-0.5\DCunit}%
    \setlength{\columnsep}{1pt}\begin{wrapfigure}[#1]{l}{0mm}{\dropcapfont\fontsize{\DCscale}{1em}\selectfont #2}\end{wrapfigure}%
    \textsc{#3}%
}
%% Define a command for illuminated dropcaps where multiple paragraphs
%% automatically wrap around the text (unlike the \lettrine command).
%%
%% Syntax: \dropcapil{lines}{initial}{caps}
\newcommand\dropcapil[3]{%
    \setlength{\intextsep}{-0.2ex}%
    \setlength{\DCunit}{1.25\baselineskip}%
    \setlength{\DCscale}{#1\DCunit}%
    \addtolength{\DCscale}{-0.5\baselineskip}%
    \setlength{\columnsep}{1pt}\begin{wrapfigure}[#1]{l}{#1\baselineskip}{\initfamily\fontsize{\DCscale}{1em}\selectfont\uppercase{#2}}\end{wrapfigure}%
    \textsc{#3}%
}

%%% Use polyglossia to handle multiple languages
%% We have (at least): latin, polytonic greek, hebrew, arabic, persian
%% (Contains bidi package, and must be loaded *after*:
%%  wrapfig, lettrine)
\usepackage[quiet]{polyglossia}
\setmainlanguage{latin}
\setotherlanguage{greek}
\setotherlanguage{hebrew}
\setotherlanguage{arabic}
\setotherlanguage{english}

%%% Reduce the white space above and below a chapter title
%%% The word "Chapter" (or "Caput") and the number are removed by making
%%% the third parameter empty.
%% Note: these commands must be given *after* plyglossia is loaded and
%% its commands (\setotherlanguage{}) are given
\titleformat{\chapter}[hang]
  {\normalfont\tiny\bfseries}{}{0em}{\Huge}
\titlespacing*{\chapter}{0pt}{-10pt}{0.5cm}

%%% Reduce the white space above and below figures
\setlength\intextsep{0pt}

%% Prevent text from going outside the margins, by making
%% the word spacing sloppy
\sloppy

%%% Tell XeTeX where to look for graphics files
%% Note: all file paths are relative to the root document, i.e.
%% the tex file that is passed as an argument to the compiler.
%% In other words: this file.
\graphicspath{{./img/}}

%%%========================================================%%%

%% list the parts we want to compile
\includeonly{./tex/prolegomena}

\begin{document}

%% Source file for testing. Comment out when not testing
%% Does not need to be mentioned in the above \includeonly{}
%% !TEX TS-program = xelatex
% !TEX encoding = UTF-8 Unicode
% this template is specifically designed to be typeset with XeLaTeX;
% it will not work with other engines, such as pdfLaTeX

%%% Count out columns for fixed-width source font
% 000000011111111112222222222333333333344444444445555555555666666666677777777778
% 345678901234567890123456789012345678901234567890123456789012345678901234567890

\chapter[Dropcap]{Dropper Capper}

\setcounter{parcount}{0}
\begin{parnumbers}
\dropcap{12}{Q}{Vintusdecimus} hic annus agitur, candide Lector, postquam opus nostrum de Emendatione Temporum emisimus.
\\ \p
Persuaseram mihi, homines studiosos aliquam nobis gratiam habituros tot rerum, quas \& scitu dignas, \& a nobis primum indicatas negare non poterant.
\\ \p
Sed longe aliter animatos experti sumus: atque adeo rem potius inuidiosam atque obtrectationi opportunam, quam illis gratam me suscepisse intellexi.

Denique nihil aliud quam significarunt, quiduis potius se ignorare malle, quam a nobis aliquid discere.

In quibusdam candorem, in aliis studium, in omnibus sensum bonarum rerum \framebox[1.0\width]{desideraui}.

Normal font:\\
ABCDEFGHIJKLMNOPQRSTUVWXYZ\\
abcdefghijklmnopqrstuvwxyz\\
Quistactum solissima

Dropcap font:\\
{\dropcapfont
ABCDEFGHIJKLMNOPQRSTUVWXYZ\\
abcdefghijklmnopqrstuvwxyz\\
Quistactum solissima
}

\end{parnumbers}
Columnsep: \the\columnsep
\setlength{\columnsep}{30pt} \\
Columnsep: \the\columnsep

\dropcapw{8}{Q}{Vintusdecimus} hic annus agitur, candide Lector, postquam opus nostrum de Emendatione Temporum emisimus.
Persuaseram mihi, homines studiosos aliquam nobis gratiam habituros tot rerum, quas \& scitu dignas, \& a nobis primum indicatas negare non poterant.
Sed longe aliter animatos experti sumus: atque adeo rem potius inuidiosam atque obtrectationi opportunam, quam illis gratam me suscepisse intellexi.
Denique nihil aliud quam significarunt, quiduis potius se ignorare malle, quam a nobis aliquid discere.
In quibusdam candorem, in aliis studium, in omnibus sensum bonarum rerum desideraui.
Columnsep: \the\columnsep

\setcounter{parcount}{0}
\begin{parnumbers}

\newlength{\dcunit}\setlength{\dcunit}{1.4\baselineskip}
\newlength{\dcscale}\setlength{\dcscale}{12\dcunit}
\addtolength{\dcscale}{-0.5\dcunit}
Dcheight: \the\dcscale

\setlength{\columnsep}{1pt}\begin{wrapfigure}[12]{l}{0mm}{\fontsize{\dcscale}{1em}\selectfont Q}\end{wrapfigure}
\textsc{Vintusdecimus} hic annus agitur, candide Lector, postquam opus nostrum de Emendatione Temporum emisimus.

Persuaseram mihi, homines studiosos aliquam nobis gratiam habituros tot rerum, quas \& scitu dignas, \& a nobis primum indicatas negare non poterant.

Sed longe aliter animatos experti sumus: atque adeo rem potius inuidiosam atque obtrectationi opportunam, quam illis gratam me suscepisse intellexi.

Denique nihil aliud quam significarunt, quiduis potius se ignorare malle, quam a nobis aliquid discere.

In quibusdam candorem, in aliis studium, in omnibus sensum bonarum rerum desideraui.
\end{parnumbers}

\setcounter{parcount}{0}
\begin{parnumbers}

\setlength{\dcunit}{1.1\baselineskip}
\setlength{\dcscale}{3\dcunit}
\addtolength{\dcscale}{-0.5\dcunit}
DCscale: \the\dcscale

\setlength{\columnsep}{1pt}\begin{wrapfigure}[3]{l}{0mm}{\dropcapfont\fontsize{\dcscale}{1em}\selectfont Q}\end{wrapfigure}
\textsc{Vintusdecimus} hic annus agitur, candide Lector, postquam opus nostrum de Emendatione Temporum emisimus.

Persuaseram mihi, homines studiosos aliquam nobis gratiam habituros tot rerum, quas \& scitu dignas, \& a nobis primum indicatas negare non poterant.

Sed longe aliter animatos experti sumus: atque adeo rem potius inuidiosam atque obtrectationi opportunam, quam illis gratam me suscepisse intellexi.

Denique nihil aliud quam significarunt, quiduis potius se ignorare malle, quam a nobis aliquid discere.

In quibusdam candorem, in aliis studium, in omnibus sensum bonarum rerum desideraui.
\end{parnumbers}

Columnsep: \the\columnsep\\
\setlength{\columnsep}{1pt}
Columnsep: \the\columnsep\\

\dropcapQ{9}{Vamquam} \lipsum[1]


\frontmatter
\include{./tex/title}
\include{./tex/dedication}
\include{./tex/greek-quotes}
% !TEX TS-program = xelatex
% !TEX encoding = UTF-8 Unicode
% this template is specifically designed to be typeset with XeLaTeX;
% it will not work with other engines, such as pdfLaTeX

%%% Count out columns for fixed-width source font
% 000000011111111112222222222333333333344444444445555555555666666666677777777778
% 345678901234567890123456789012345678901234567890123456789012345678901234567890

\chapter[Prolegomena]{}
\begin{center} \vspace{-18mm}
{\scshape
\head{3.0}{35}{PROLEGOMENA}\\ \vspace{6mm}
\head{1.5}{60}{IN}\\ \vspace{5mm}
\head{2.0}{50}{LIBROS}\\ \vspace{9mm}
\head{1.5}{60}{DE}\\ \vspace{6mm}
\head{3.3}{25}{EMENDATIONE}\\ \vspace{7mm}
\head{2.0}{40}{TEMPORVM}\\ \vspace{7mm}
} % scshape
\em{Ad candidum Lectorem.}
\end{center}
\normalsize

\setcounter{parcount}{0}
\begin{parnumbers}
\dropcapil{8}{Q}{Vintusdecimus} hic annus agitur, candide
Lector, postquam opus nostrum de
Emendatione Temporum emisimus.
\\ \p
Persuaseram
mihi, homines studiosos aliquam nobis
gratiam habituros tot rerum, quas \& scitu
dignas, \& a nobis primum indicatas negare
non poterant.
\\ \p
Sed longe aliter animatos experti
sumus: atque adeo rem potius inuidiosam
atque obtrectationi opportunam, quam illis gratam me suscepisse
intellexi.

Denique nihil aliud quam significarunt, quiduis potius
se ignorare malle, quam a nobis aliquid discere.

In quibusdam
candorem, in aliis studium, in omnibus sensum bonarum rerum desideraui.

Nos vero, qui nihil unquam prius habuimus, quam vt horum
orationes sinamus praeterfluere, modo verum eruere, \& inimicos
nostros etiam inuitos iuuare possimus, opus nostrum iterum in
manus sumptum auximus, illustrauimus, emendauimus, vt, quanuis
idem sit, aliud tamen a noua cultura videri possit.

Quæ huic editioni
accesserunt, haud promptum est dicere.

Sed in quibus a priore demutat,
postea intelliges, siquidem instituti nostri rationem aperuero.

Subiectum operis nostri est ratio Temporum civilium, \& eorum,
quæ in vetustatis cognitione versantur: finis, Emendatio: quod quidem
me tacente, \& Titulus ipse promittit.

Ciuilium temporum cognitio,
eorumque historia, vertitur in multiplici diversorum annorum
forma \& eorum methodis vulgaribus, quos Computos posterior
ætas vocauit.

\textgreek{Τα ιςορομγυα[?]} civilium temporum habes in primoribus
tribus libris, \& maiore parte quarti: methodum autem in septimo.

A emendationis duæ partes sunt.

%\end{parnumbers}
\clearpage
p. II [pdf 29]
%\begin{parnumbers}
Prior versatur circa epocharum
inuestigationem, posterior circa verum annum tropicum, 
\& periodos Lunares: quam materiam posterior pars libri quarti,
item toti quintus \& sextus sibi vindicant.

Iam quemadmodum Epochæ
sunt notationes, \& tituli temporum, ita ipsarum epocharum
quædam debent esse propria \textgreek{γνωρίσματα} \& characteres: quorum
characterum alij sunt naturales, alij ciuiles. 

Naturales quidem a rationibus
utriusque sideris, unde nati cycli Solaris, \& Lunaris: civiles
ab instituto, cuiusmodi indictiones \& anni Sabbatici: sine quibus in
harum rerum tractatione omnis conatus irritus. 

Rursus \& eorum
quoque fallax vsus est, nisi quædam annorum ex illis periodus instituatur.

Sed eæ sunt totidem, quot aut formæ annorum, aut civilia
initia.

Nam in anno Ægyptiaco Nabonassari alia opus est, ac in anno
Solari, quia diversa forma: item in anno Actiaco siue Diocletianeo
alia, ac in Iuliano, propter diversa initia.

In anno Ægyptiaco vago
naturales characteres sunt \textgreek{εἰκοσιπεν σαετηρις[?]} Lunaris, \&
\textgreek{έπταετηρις[?]} Solaris:
civilis autem character est quadriennium, quem canicularem
annum minorem vocabant Ægyptij.

Hi tres characteres in se ducti
producunt periodum magnam annorum 700 Ægyptiacorum: qua
vti debet disputator temporum, siquidem rationes suas ad annos
Nabonassari, Armeniorum, aut Persarum exigit.

At qui anno Iuliano,
quæ omnium formarum temporibus est conuenientissima, vti
volet, is cyclo vtriusque fideris quindecies ducto componet elegantissimam
periodum annorum 7980, cuius initium in cyclo Solari,
\& Indictione Romana, a Kal. Ianuarij, in cyclo Lunari a Martio, in
anno Sabbatico ab autumno.

Itaque non minus utilis, quam necessaria
est.

Sine ea nihil agit Chronologus: cum ea tempori, \& sæculis
imperat.

Quam enim lubricum sit retro ab aliqua epocha notare tēpora,
quod maior pars doctorum virorū facit, satis nos vsus docuit.

His ita positis, ad singula huius operis membra venio.

%% == Libro primo
Libro primo
præter divisionem temporum, \& iucundissimam mensium, \&
annorum historiam, de antiquissima anni forma disputatur, quæ in
menses æquabiles annum describit, qua pleraque omnes Græcia vsa
est, \& ab ea omnis ratio Olympiadum pendet: nisi potius eam e ratione
Olympiadum propagatā dicas: quod sine cognitione Olympiadum
numquam tam eximium vetustatis \& temporum monimentum
in lucem eruissemus. 

Ex tanta autem Græcorū scriptorum
copia vnicus Pindarus nobis facē alluxit, qui solus nos docuit tempus
ludicri Olympici.

Aliter, quæ paucitas est bonorum scriptorum,
nulla erat via ad hæc interiora perueniendi.

Huius anni Græci
formæ doctrina tanto acceptior esse debet, quanto obscurior eius
rei apud maiores nostros scientia fuit: cum ante hos mille quadringentos
plus minus annos eius rei neque volam, neque vestigium
vetustas retinuerit.

%\end{parnumbers}
\clearpage
p. III [pdf 30]
%\begin{parnumbers}
Nam falso veteres multi, ac post eos infamæ antiquitatis
scriptores, Macrobius ac Solinus, atque proauorum memoria
summus vir Theodorus Gaza, annum Græcorum statim
ab initio merum Lunarem fuisse prodiderunt.

Quamuis enim in
Panegyribus suis, ac nobilioribus sacris, quæ certo annorum circuitu
redibant, vnius Lunæ rationem habebant, tamen, vt vno verbo
dicam, eorum anni forma Lunaris non erat.

Olympicum enim ludicrum
ipsa Lunæ plena lampade celebrabatur, vt solus veterum nos
docet Pindarus. 

Prætera Laconibus ante plenilunium, aut nouilunium
aliquid incipere religio erat.

Vnde \textgreek{Λαχώνιχας ςελωοαζ}[?] vicinorum
prouerbio iactatas, \& contra Arcadibus prouerbiali conuicio
neglectum religionis obiectum legimus. 

Quod enim ante nouilunium,
aut plenilunium vt plurimim bella aut alia seriora aggrederentur,
ob eam rem a finitimis nationibus \textgreek{[Greek]} vocabātur:
quæ conuicij caussa ab ipsis Arcadibus interpretatione elusa est,
probrum in laudem conuersum ad vetustatem originis suæ referentibus,
\& antiquiorem sidere gētem suam gloriantibus. 

Quod igitur
nouilunij ac plenilunij tempora Panegyricis ludicris deligebant,
propterea sacra trieterica instituta: cuiusmodi erant orgia Bacchi,
Nemea, Isthmia, alia.

Ea enim est anni Græcanici forma, vt si, verbi
gratia, nouilunium in neomeniam Gamelionis incurrat, plenilunium
in eandem neomeniam incidat anno tertio redeunte.

Itaque
cum in Tetraeteride orgia Bacchi trieterica celebrabantur, tertio
anno redibant in eum sistum Lunæ, qui priorum orgiorum situi oppositus
erat.

Quare elegantissime Statius trieterida vocat alternam:
quia alternis in nouilunium, \& plenilunium incurreret.

At sacra,
quæ necessario eodem Lunæ tempore obibantur, ea semper erant
tetraeterica: vt in Attica Panathenaica maiuscula, in Elide Olympias,
vt iam tetigimus, plenilunio.

quod sane fieri non poterat, nisi absoluta
Tetraeteride, \& Pentaeteride ineunte.

Atq; ita Tetraeterides
in idem \textgreek{χῆμα} Lunæ, non vtiq; in idem tempus Solis redibant.

Vt
enim in orbem Solis \& Lunæ redirent, non aliter putabant fieri,
quam octaeteride confecta, eneaeteride ineunte.

Ex quo quædam
eneaeterica sacra eo nomine instituta: cuiusmodi ab initio Pythia
fuerunt: \& quidem merito.

Apollini enim, quem eundem cum Sole
faciebant, erant attributa.

Hinc colligimus, non solum Olymiadis
interuallum annis quatuor solidis explicatum fuisse; sed etiam puerliter
peccare eos, qui annorum quinque solidorum fuisse putant.

Neq; vero quibusdam recētioribus succensendum, qui ita censent,
ita scribunt, sed \& Ausonium nostratem culpa liberat Ouidius, scriptor
longe antiquior, \& nobilior, qui ætatem suam quinquaginta annorum
decem Olympiadibus definit: quo magis mirum Pausaniam
hominem Græcum in ea hæresi fuisse, vt suo loco a nobis relatum est.

%\end{parnumbers}
\clearpage
p. IV [pdf 31]
%\begin{parnumbers}
Nam minus mirandum de Solino, qui cap. \textsc{xiii} Isthmia vocat
quinquennalia, quæ erant tantum triennalia, quod certamen a Cypselo
tyrāno intermissum, anno primo Olympiadis 49 instauratum
fuisse dicit.

Horum igitur omnium caussæ ad typum anni Græci referendæ
sunt.

In quo argumento nihil eorum prætermisimus, quæ
ei rei illustrandæ faciebant, quanquam pene omnibus præsidiis
destituti.

Et quidem primum in genere, quod semper solemus, deinde
priuatim multarum Græciæ nationum periodos proposuimus,
quæ quidem non anni forma, sed situ \& capite inter se differunt: in
qua tractatione diu nobis res fuit cum præstantissimo viro Theodoro
Gaza, vel potius cum eius sequacibus, a quibus extorqueri non
potest doctrina \& situs mensium, ab illo primum proditus. 

quæ quidem
velitatio nobilioribus ingeniis, \& ab omni inuidia remotis, vt
spero, iucunda erit.

Quid enim est toto libro primo, cuius vel minima
pars, nō dicam istis querulis, qui nihil sciunt, sed etiam doctioribus,
hoc sæculo, \& ante multa retro sæcula oboluerit?

Quid dicam \textgreek{[Greek]}?

quis illarū caussas, \& vsum sciebat?

quis
locum nobilem de illis in Verrina Ciceronis intelligebat?

quis locum
\textgreek{Εξαιρέσεως[?]} in secunda Boedromionis?

quis Posideonem intercalarem
mensem fuisse?

Huic materiæ accessit \textgreek{ἐποχη χέντςα[?] θερινᾶ}
in \textsc{viii} Iulij, quæ in priori editione omissa erat.

Id erat \textgreek{χάντσον[?]}
populare, quod nomine \textgreek{τςοπων[?] θερινῶν[?]} Aristoteles, Thophrastus,
Plutarchus, \& omnes veteres intelligunt, non autem ipsum verum
Solstitium: quæ rei pulcherrimæ notatio nobis viam ad illustriora
præiuit.

Quod Solstitiorum, \& Æquinoctiorum puncta \textgreek{χέντρα} vocentur,
satis sciunt, qui veterum Græcorum libros legerunt.

Columella
cardines vocauit.

In præstantissimo Parapegmate, quod falso
Ptolemæo attribuitor (est enim antiquius Ptolemæo) ad \textsc{viii} Kal.
Iulij (quod est Solstitium Sosigenis) annotatum est: \textit{Æstiuus cardo,
\& momentanea aeris perturbatio}.

In Græco (vtinam haberemus!)
sine dubio fuit: \textgreek{Θερινὸν χέντρον, χαὶ ςιγμιαία αἔσος Ιασοιχή[?]}.

Igitur \textgreek{χέντρον
θερινὸν} nihil aliud, quam \textgreek{τροπαὶ θεριναί}.

Cur \textsc{viii} dies Iulij erat
epocha æestiua in vsu ciuilis anni, non semel caussam reddidimus. 

Adiecta etiam pernecessaria neomeniarū Atticarum Tabula: quæ
non solum priori editioni, sed etiam doctrinæ anni Attici deerat.

Liber secundus anno Lunari dicatus est ideo, quia is annus ex illo
Græco æquabili manasse videtur.

Ibi aperitur omnis antiquitas \textgreek{ἔτοις[?]
πρυτανείας[?]}, Octaeteridum Cleostrati, Harpali, \& Eudoxi: quæ omnia
hodie nomine tenus nota erant.

Eudoxea Octaeteris numquam
in vsus ciuiles admissa est.

Anni vero \textgreek{πρυτανείας[?]} in vetustissimis Psephismasin
Atheniensium primo quidem ex Cleostratea, deinde, illa
abrogata, ex Harpalea petiti sunt.

%\end{parnumbers}
\clearpage
p. V [pdf 32]
%\begin{parnumbers}
Sequitur magnus annus Metonicus
ambarum, \& Calippicus Metonici castigator.

Et quidem hi
ambo nomine noti tantum: caussarum autem, \& omnium, quæ ad
illa pertinent, mira ignoratio hactenus fuit.

Accesserunt huic editioni
Tabulæ operosissimæ dispensationum neomeniarum Metonicarum,
\& Calippicarum: cuiusmodi etiam in Harpalea Octaeteride
exhibuimus. 

Quod de Eudoxea Octaeteride diximus, idem de
periodo Chaldæorum dicendum, eam nunquam ad ciuilia tempora,
sed ad Genethliacorum themata vsurpatam fuisse.

Id quod tum
multa argumenta, tum vnicum certissimum illud est, quod eorum
menses appellationibus Macedonicis, nō vero Chaldaicis fuerunt.

Propterea recte cum illius anni diatriba doctrinam dodecaeteridis
Chaldaicæ Genethliacorum coniunximus, cuius nomen quidem
solum notum erat ex Censorino: cognitio autem nobis ex Arabum,
\& Orentalium vsu repetenda fuit.

An aliquis Græcorum \textgreek{δωδεχαετρίδος Χαλδαιχῆς}
meminerit, haud promptum est dicere.

Vnum tantum Orpheum siue Onomacritum eius meminisse scimus. 

\textgreek{ὀρφδὶςὀν[?] ταῖς[?] δωδεχαετνρίσιν[?]:}

\begin{greek}
ἔςαι δ᾽ αὖθις ἀνὴρ, ἢ χοίραν[G-circ], ἠὲ τύρανν[G-circ],

ἢ βαοιλδὶς, ὂς τῆμ[G-circ] ἐς οὐρανὸν ἴξε[right curl] αἰπιιύ [all doubtful].
\end{greek}

Est apotelesma cuiusdā Genethliaci consulti super alicuius genesi,
de quo ipse respondit, eum fore magnum regem aut Dynastam, \&c.

Citat Tzetzes. 

Hæc multum illustrant doctrinam Dodecaeteridos
genethliacæ parum antehac notæ.

Itaque quemadmodum \textgreek{τελετὰς}
ita etiam \textgreek{δωδεκαετνρίδας[?]} scripserat Onomacritus sub nomine
Orphei.

Qualis fuerit Iudæorum annus sub Seleucidis, quibus parebant,
multis exemplis testatum reliquimus: in quibus etiam translationis
feriarum in capite anni antiquitatem asseruimus aduersus homines
nostrorum temporum, qui nugantur commentum nuperum
Iudæorum esse.

In illis Doctor Theologus ingenti commentario
suo in Euangelium secundum Iohannem exultabundus ait illam
translationem confutari ex loco Iosephi, in quo scribit, quo anno
Hyrcanus fœdus icit cum Antiocho Sidete, Pentecosten fuisse feria
prima.

Hunc locū Iosephi nos olim in priore editione produximus,
vnde is, aut qui illi indicauit, accepit.

En, inquit, duo Sabbata continua.

Si propter continuationem duorum Sabbatorum, feria transfertur,
ergo vbi sunt duo continua Sabbata, non transfertur. 

In quibus aperte ostendit se ignorare caussam feriæ transferendæ, quæ fiebat
propter solum Tisri, non autem propter alios menses; propterea
quod ille mesis multa solennia habet, adeo vt si non habeatur
ratio translationis, aliquando non solum duo, sed entiam tria continua
sabbata concurrere necesse sit.

%\end{parnumbers}
\clearpage
p. VI [pdf 33]
%\begin{parnumbers}
Si enim feria sexta inciperet
neomenia Tisri, omnino tria sabbata continuarentur, neomenia,
siue clangor tubæ, sabbatum ordinarium, \& ieiunium Godoliæ.

Continuantur autem sæpernumero in aliquo reliquorum mensium
duo Sabbata: idque fit, quando solenne est aut feria prima, aut feria
sexta.

quorum alterutrum quotannis incidere, nisi quando Tisri
incipit feria tertia, Doctor ignorauit.

In primam feriam incidunt
hæc solennia, \textsc{xxv} Casleu, \& \textsc{x} Tebeth in anno defectiuo tam
cōmuni, quam embolimæao, quotiescunq; Tisri incipit feria secunda:
\textsc{vi} Sivvan; quando Nisan incipit feria septima:
\textsc{xv} Nisan, \textsc{xvii} Tamuz,
\textsc{ix} Ab, quando Nisan incipit feria prima.

In feriā autem sextā
conuenit solenne \textsc{xxv} Casleu \& \textsc{x} Tebeth, quādo Tisri est feria
septima in anno tā communi, quā embolimæo.

\textsc{xiiii} Adar, quando
Nisan sequens est feria prima: \textsc{vi} Sivvan, quādo Nisan feria quinta.

Vides, quot Sabbata quotānis, nisi quādo Tisri incipit feria tertia, Iudæi
continuent in aliquo mensium, præterquam in solo Tisri, cuius
vnius gratia illa cautio instituta
% No period at end of sentence

Itaq; doctor tā frustra, quam ridicule
Iosephi testimonium adduxit de sexta Sivvan, id est, Pentecoste
feria prima; cum illo anno neomenia Nisan fuerit Sabbatum.

Atqui
nihil superesse putauit, quam vt Vaticani montis imago redderet
\textgreek{ἰὴ παιαή[?]}.

Sed ipse valde ignarus est harum rerum, vt reliqui omnes,
qui contendunt nouitium esse Iudæorum commentum.

Nos
validissime demonstrauimus, \& sæculo Christi, \& retro sub Seleucidis
translationes in vsu fuisse.

\& sane res peruetusta est.

quæ tamen
non minus ignorata, quam periodus Calippica, qua Seleucidæ, \&
Seleucidarum edicto Iudæi vsi.

quod non solum ex Nisan anni excidij
Hierosolymorum a nobis demonstratum est, sed etiam patet
ex definitione Rabbi Adda.

Is annum definit dierum \textsc{ccclxv},
horarum 5, 997/1080. 48/76.

Quid hac definitione aliud vult, quam periodum
Iudaicam fuisse annorum 76?

Cum Meto definit annum dierum
365. hor. 5. 1/19. ex eo coniiciendum relinquit, se vti periodo annorum
19.

Vtebantur igitur periodo 76 annorum, id est, Calippica:
\& tamen in omnibus neomeniis Lunæ \textgreek{φάσιν} obseruabant, non
quod eam ex præscripto periodi non indicerent, sed ideo, vt eam
sanctificarent.

nam \& hodie quoque obseruant \textgreek{φάσιν}, non vt ex ea
neomeniam indicant, sed vt eam sanctificent.

Itaque Luna statim
visa dicunt: \texthebrew{[Hebrew]}.

\textgreek{ἀγαθὸν τέρας ἔςω ἡμῖν ης[??] παντὶ Ισραήλ.}

Idem faciunt \& Muhammedani, quamuis neomenias ex
scripto indicere soleant.

Neque aliud intellexit fabulosus quidem,
sed tatem vetus auctor \textgreek{[Greek]} apud Clementem:
\textgreek{[Greek][Lots of Greek]}.

%\end{parnumbers}
\clearpage
p. VII [pdf 34]
%\begin{parnumbers}
Præclara quidem ista: sed nescit, quid dicit.

Nam in Iudæorum potestate
nunquam fuit, vt exspectarent \textgreek{φάσιν}: quia raro Luna se ostendit,
nisi secundo post coitum die.
% Greek: phase

Quod si expectandum ipsis esset,
res ridicula accideret, vt Elul, qui semper est cauus mensis, non solum
plenus, sed etiam aliqando vnius \& triginta dierum esset.

Sine dubio translationem feriæ intelligit, cuius caussam ignorat.

\textgreek{[Greek]} vocat \texthebrew{[Hebrew]} caput anni.

Nam Sabbatū vocat, quia Festus
dies, \textgreek{[Greek]}.

Ita etiā vocatur Leuitici \textsc{xxiii}, 24.

\textgreek{[Greek]} intellige
\textgreek{[Greek]}: quod ita Hebraice vocetur, nēpe \texthebrew{[Hebrew]}.

Vide in Computo Iudaico.

At \textgreek{[Greek]} vocat \textgreek{[Greek],
[Greek]} quoq;, id est \texthebrew{[Hebrew]}.

Nam aliæ erant \textgreek{[Greek]},
proinde vt \& \texthebrew{[Hebrew]}.

Sic Tertullianus magnos dies dixit, quos
Hebræi \texthebrew{[Hebrew]} vel \texthebrew{[Hebrew]}.

Eius verba sunt ex v in Marcionem:
\textit{Dies obseruatis, \& menses, \& tempora, \& annos, \& Sabbata, vt opinor,
\& cenas puras, \& ieiunia, \& dies magnos.}

Sed quid Tertullianum
aduoco?

Ecce Biblia Græca ita vertunt ex primo cap.

Isaiæ:
\textgreek{[Greek], [Greek], [Greek]}.

Quod Hebraice est \texthebrew{[Hebrew]}, vertunt \textgreek{[Greek]}, quod idem
est quod \texthebrew{[Hebrew]}: \& quidem manifesto Sabbata distinguuntur a
magnis diebus. 

Quare perperam quidam \textgreek{[Greek]} interpretantur
Sabbatum apud Iohannem, \textgreek{[Greek]}. de quo infra.

Quin \& Tertullianus ipse \textgreek{[Greek]},
quas cenas puras vocat, a diebus magnis, \& a ieiuniis, \& a
Sabbatis distinguit.

De Cena pura, præter id quod diximus ad
Festum, ita reperi in veteri \& peroptimo Glossario Latinoarabico:
\textit{Parasceue, cena pura, id est, praparatio, que fit prosabbato.}

Conditor Annalium Ecclesiasticorum turbat de cena
pura, \& negat esse parasceuen, quia cena pura apud Festum
habeat offam suillam.

Sed ipse, (pace docti viri dixerim) non
aduertit Puram dici, non quia careat carnibus, sed quia religionis
\& dicis caussa fit.

Nam \& parasceuæ Iudaicæ habent carnes,
\& nihilominus dicuntur cenæ puræ, quod dicis caussa coquebantur,
coquunturque hodie prosabbato, quia in Sabbato
coqui non liceat.

Non negabis, candide Lector, hæc vulgo non intelligi.

Itaque locus ille est nobilissimus. 

Tamen quotus quisque est ex tot Lectoribus, qui non hæc aut præteribit,
aut calumniabitur?

Sequuntur periodi magnæ Hagerenorum,
ex quibus ratio anni soluti Indorum, \& Muhammedanorum
tota pendet.

Omnia nunc primum ex Arabum scriptis
prodeunt: atque adeo omnis tractatio nostris hominibus
noua est.

%\end{parnumbers}
\clearpage
p. VIII [pdf 35]
%\begin{parnumbers}
Excipit hanc doctrina anni Iudaici hodierni, res, quod
sæpe diximus, artificiosissima, ideoque eximia, quia melior
anni Lunaris forma constitui non potest.

Docemus præterea, vnde
natus sit ille annorum computus, quo vtuntur hodie, a \textsc{vii} Octobris:
quem inepte putant a conditu rerum.

Post multarum Periodorum,
Cyclorum, Octaeteridum, Paschalium historias, in locum vltimum
\textgreek{[Greek]} veteris anni Romanorum coniecimus, ideo
quod ea forma proxime abesset a Lunari: vbi de sæculo Romano,
\& capite veteris anni Romani, temporibus vltimis C. Iulij Cæsaris,
multa accuratissime disputata.

Itaque ex singulis rebus singula capita
confecimus, cum potius singuli libri \& quidem ingentes confieri
possent, si, quæ hominum hodiernorum est ambitio, eadem nobis
incessisset.

Tertio libro opportune annus æquabilis datus est,
cum annus Solaris Ægyptiacus, adscitis diebus quinque, ex Græco
propagatus sit: (quemadmodum annus Lunaris ex eodem Græco
manauit, abiectis ab eo totidem diebus cum horis \textsc{xv}, paulo amplius)
quod, metacente, Plutarchus docuit in libro \textgreek{[Greek]}.

Adeo inter se libri nostri mutuo conspirant, neq; ab eis ratio,
methodus, \& ordo abest.

In eo libro de Neuruz antiqui Persarum
periodo annorum \textsc{cxx}, deq; cognominibus dierum Persicorum,
de translatione \textgreek{[Greek]} in enthronismis nouorum Regnum,
item de caussis anni Iezdegird, de annis Armeniorum, \& eorum
mensibus, omnia noua protulimus. 

Sed hæc non expergefacient animos
hominum, nisi forte ad obtrectandum.

Quartus liber est emendatio
tertij, vt secundus primo erat subsidiarius: qua methodo imperfectus
Lunaris Græcus libro primo disputatur, vt perfectus secundo.

Sic etiam perfectus Solaris, \& siqui alij naturam perfecti immitantur,
supplent id, quod æquabili Ægyptiaco, Persico, \& Armeniaco
vetuitas detraxerat.

Itaq; in quatuor partes tribuendus fuit.

In
prima continentur anni, quibus quarto anno exeunte dies ex quatuor
quadrantibus conflatus accrescit.

Ex illis nobiliores selegimus,
Iulianum, Actiacum, Antiochenum, Samaritanum, \& alios. 

Nam \& alios quoq; eius formæ habebamus, vt Tyriorum, quorum menses
appelationib. Macedonicis, diuersa initia a Iulianis habent.
% "appelationb." interpreted as abbriviation for "appelationibus"

Sic
etiam Gazensium annus mere Actiacus fuit, appellationibus mensiū 
Macedonicis, mensibus tricenariis. 

Marcus Ecclesiæ Gazensis Diaconus,
in actis Porphyrij Gazensis Episcopi vocat \textgreek{Διον} Nouembrem
\textgreek{Απελλαιον} Decembrem quæ nomina habent a Macedonibus. 

Sed
idem scribit Gazenses celebrasse Theophaniorum diem vndecima
Audynæi, quæ est sexta Ianuarij Iuliani, se autē redisse Constantinopoli,
Xanthici vicesima tertia, quam ait fuisse decimam octauam
Aprilis secundum Romanos: quibus ostēdit formam illius anni mere
Actiacam fuisse, mensibus tricenariis, appellationibus Macedonicis. 

%\end{parnumbers}
\clearpage
p. IX [pdf 36]
%\begin{parnumbers}
Secunda pars annos emendatos, eorumque emendandorum
rationem complectitur: tertia periodos multiplices, quarum finis
conciliatio anni ciuilis cum Solari, cui dies quinto quoque anno
ineunte accrescit.

Quarta pars agit de vera emendatione anni, \&
de anno cælesti instituendo, qui pertinet ad methodum epochæ
mundi.

Quemadmodum autem nulla Lunaris anni ciuilis ratio
recta iniri potest, præter eam, qua Iudæi vtuntur: ita nullus annus
cælestis Tropicus recte institui potest, nisi ex forma, quam edidimus,
quam nemo vituperabit, nisi qui ignorauerit; omnis laudabit,
qui intellexerit.

Alioquin scio \& malignos \& obtrectatores non defuturos. 

Annus tam noster, quam Iudaicus ciuilis quidem, sed naturalis,
vterq; ad motum quisq; sui sideris descriptus. 

Ideo eius saltem
in scriptis vsus esse debet, qualis olim Philadelphi Dionysianus,
Chaldæorū Calippicus, hodie Persarum Gelaleus. 

Tres igitur libri
primi, \& prima pars quarti pertinent ad \textgreek{[Greek]} temporum ciuilium
cum septimo.

At reliquæ tres partes quarti cum duobus libris
sequentibus pertinent ad ipsam emendationem temporum.

Atque
vt a mundi primordiis omnes res deducuntur, ita mundi epocham
primā ordine posuimus: qua in re quam pueriliter hallucinati sint
omnes, nō sine admiratione tam imperitiæ quam pertinaciæ eorum
dicere possum.

Non loquor de iis, qui sæculo vno, aut pluribus altius
originem rei repetunt.

Nam quemadmodum ij nullam rationem
sibi proposuerunt, quam sequerentur, ita nullos lectores nancisci
possunt, nisi imperitos. 

Qui intra sæculum maiorem mundi
epocham faciunt, eorū duo genera reperio.

Prius genus est eorum,
qui solutionem captiuitatis in primum annum Olympiadis \textsc{lv} conferunt:
alterum eorum, qui tempus illud \textsc{xviii} aut \textsc{xix} annis ante
\textsc{lxiiii} Olympiadem definiunt.

In priori hæresi fuerunt \&
quidam veterum Ecclesiasticorum, vt alicubi indicauimus. 

Aiunt
Cyrum cæpisse imperare primo anno Olymiadis \textsc{lv}, hoc est 217
anno Iphiti, quod verū est: de quibus deductis septuaginta, relinquitur
annus excidij Hierosolymorū, \& casus Sedekiæ 147 a primo ludicro
Olympico.

Sed puerilis sentētia multis absurditatibus eluditur.

Primo computatione non recta annorum \textsc{lxx} a capto Sedekia.

Deinde quod Cyrum statim initio regni sui Regem Mediæ, Persidos,
Susidos, Assyriæ, Babyloniæ, totius Asiæ minoris, Indiæ, totius
Syriæ constituunt, qui vnius Persidos Rex fuerit aliquot annis ante
casum Astyagis[?], \& post illud tempus pauculis annis ante obitum
Babylone potitus sit.
% Astyagis or Aftyagis?

Hæc sola absurditas facit, vt non solum eorum
nulla ratio habeatur, sed vt ludibriū quoq; debeāt.

Tertio 147 annus
Iphiti est 118 Nabonassari: qui erat annus quintus ante initiū Nabopollassari
patris Nabuchodonosori.

%\end{parnumbers}
\clearpage
p. X [pdf 37]
%\begin{parnumbers}
Ergo Nabuchodonosor anno
decimono regni sui templum \& Hiersolyma euertit annis quinq;
ante quam pater ipsius, cui ipse successit, regnaret.

Digna profecto
talibus doctoribus sententia.

Tamen tantum abest, vt hac tam insigni
absurditate a sententia desistant, vt animos ab eiusmodi portentis
opinionum sumant.

Postremo ignorant diuersa esse initia Regnum,
vt ipsius Nabuchodonosori, cum patre, \& solius: Alexandri,
ab excessu Philippi patris, \& ab initio Seleuci: Diocletiani, ab æra
martyrum, \& a primo anno imperij.

Sic etiam Cyri, apud Græcos, ab
initio regni Persidis: apud Babylonios, vel a subacto toto Babyloniæ
imperio, vel ab aliquo insigni facto, quodcunque illud fuerit, siue
ex edicto ipsius Cyri, siue translatione \textgreek{πδν ἐπα γο υ[?]ων},
vt solebat fieri.

Qui tantam inscitiam sequi noluerunt, non tamen rectam viam
institerunt, quia quindecim aut amplius annis ante \textsc{xlvi} Olymiadem
casum Sedekiæ coniiciunt.

Nos ante annum quartum illius
Olympiadis id non potuisse accidere ita demonstramus. 

Ezekias
Rex Iuda, postquam singulari Dei beneficio ab ancipiti morbo conualuisset,
anno \textsc{xiiii} regni sui, accepit Legatos \& \textgreek{ςωτήρια[?]}, a
Merodach Rege Chaldæorum.

Ponamus \textsc{xiiii} annum Ezekiæ in
primo anno Merodach, hoc est, in \textsc{xxvii} Nabonassari.

Nam is est
annus primus Merodach apud Ptolemæum ex Chaldaicis obseruationibus. 

Hoc modo annus primus Ezekiæ conuenerit in annum
\textsc{xiiii} Nabonassari. Ab initio Ezekiæ, ad excidium templi, anni
sunt absoluti 138.

Hoc est, annus ipsius excidij est 139 labens ab initio
Ezekiæ.

quod ita demonstramus. 

Annus primus Sedekiæ est
quartus Hebdomadis, teste Ierimia, initio cap. \textsc{xxviii}: \& proinde
vndecimus, qui \& vltimus, est Sabbaticus. 

de quo extat testimonium
apud Ieremiam, \& nemo dubitat.

Rursus annus tertius decimus
Ezekiæ erat Sabbaticus. 

auctor Isaias \textsc{xxxvii}, 30.

ex quibus
manifesto colligitur, \textsc{xiiii} Ezekiæ esse primū Hebdomadis, \& primum
Ezekiæ esse sextum Hebdomadis. 

Ergo annis ab initio Ezekiæ vnitas addenda, ad methodum anni Sabbatici.

Addita vnitate annis
139, numerus erit septenarius. 

Quare annus labens 139 est verus
annus ab initio Ezekiæ.

Quibus additis 13 annis Nabonassari præteris
(quia posuimus 14 Nabonassari primum Ezekiæ) componitur
annus Nabonassari 152, in quo casus Sedekiæ ex hac hypothesi
locādus est, hoc est, in anno periodi Iulianę 4118: de quibus deductis
907 absolutis ab Exodo, remanet annus Exodi 3211 in periodo Iuliana.

Porro Nisan Exodi cæpit feria quinta, vt toties diximus, \& ex
Mose rectissime ante nos Iudęi docuerūt.

At in anno periodi Iulianę
3211 Nisan nō cæpit feria quinta, sed feria tertia, Martij \textsc{xi}, cyclo tā
Solis, quā Lunæ \textsc{xix}.

%\end{parnumbers}
\clearpage
p. XI [pdf 38]
%\begin{parnumbers}
Ergo annus proximus, quo Nisan cæpit feria
quinta, is debuit saltē esse annus Exodi: atq; adeo is fuerit annus periodi
Iulianæ 3214: in quo sane nisan cæpit feria quinta, Aprilis \textsc{vi},
cyclo Solis \textsc{xxii}, Lunæ \textsc{iii}.

Additis 907 annis absolutis ab Exodo,
annus 4121 periodi Iulianæ suerit is, in quo excidium templi contigit:
qui est quartus Olympiadis 46, vt erat propositum.

Sed \& post
Olympiadem 46 ponendum esse casum Sedekiæ ita probabimus.

Amasis rex Ægypti, postquam regnasset annos 55, obiit circiter annum
7 Cambysis, anno ante excessum ipsius Cambysis, hoc est, anno
225  Nabonassari.

Nechao intersectus est a Nabuchodonosoro anno
quarto Ioiakim regis Iuda.

Ieremias \textsc{xlvi}, 2.

Post eum regnauit
Psammitichus annos \textsc{vi}.

Cui Aprias, cuius meminit Ieremias
\textsc{xliiii}, 30, succedit.

Is post \textsc{xxv} annos relinquit regnum Amasi.

Summa annorum a cæde Nechao ad obitum Amasis anni 86, qui
deducti de 225, relinquunt annum Nabonassari 139, quartum Ioiakim
Regis Iudæ, primum Nabuchodonosori.

Ergo Sedekias captus
anno 158 Nabonassari, qui erat tertius 47 Olympiadis.

Idque verum
esse postea validissime demonstrabimus.

Diodorus Siculus,
auctor omnium Græcorum certissimus, attribuit, \textsc{lv} annos Amasidi.
reliquos Apriæ \& Psammatichi habemus ex Herodoto.

Temere
igitur, \& imperite faciunt, qui casum Sedekiæ antiquiorem illo
tempore constituunt: neque his cassibus sese explicare poterunt,
quantumuis sua commoueant sacra, vt Plautus loquitur.

His valide
demonstratis, \& licentia chronologorum intra aliquos fines summota
quos amplius migrare non possunt, ad originas ipsas penetremus.

Sed prius vt in Mathematicis cōcessa quædam, aut quæ negari
non possunt, assumuntur, ita \& nobis quoque faciendum.

Tempora \& initia Regum Babyloniæ a Chaldæis notata in obseruationibus
eclipticis, quæ reiicere \& damnare extremæ impudentiæ \&
inscitiæ est: item, eorundem regum initia \& tempora a Beroso Chaldæo,
qui minus quam tribus sæculis post illos vixit, \& qui quæ Actis
ac fastis Babyloniorum publicis continebantur, ignorare nō potuit,
hæc inquam, non tantum tāquam vera haberi postulamus, sed etiā
qui aliter putant, tanquam indignos censeri, qui aut audiri a nobis
mereantur, aut vllas literas attingant, aut aliquem locum inter
doctos habeant.

Tricesimum annum, cuius initio Prophatiæ suæ
meminit Ezekiel, quique capti Iechoniæ quintus erat, Iudæi inepti
deducunt a libro legis reperto, anno \textsc{xviii} Iosiæ Regis.

Quis vnquam a libro reperto vllam æram, aut edicto Iosiæ institutam, aut a
Prophetis vsurpatam legit?

Si tanti erat illa temporis nota, quare
eam nō vsurpat Ieremias, qui tā accurate annos Regum Iuda Iosiæ,
Ioiakim, Iechoniæ, Sedekiæ notare solet?

Capite \textsc{xxv}, quare dicit
anno quarto Ioiakim, cum dicendū esset vicesimo secundo a libro
inuento?

Esto, cur Ezekiel dixit tricesimo, non tricesimo a libro inuento?

qui tamen dixit anno quinto deportationis Regis Ioachin.

%\end{parnumbers}
\clearpage
p. XII [pdf 39]
%\begin{parnumbers}
Certe mos est vti epocha, quæ omnibus \& nota \& in vsu sut.

Quare
igitur epocham producit, neque plebi notam, neque in vsu positam?

Sed quid ea epocha opus in Babylonia, inter deportatos?

Nugæ Iudæorum,
nugæ sunt istæ, \& halluciationes doctorum, qui eos sequūtur.

Quare eruditiores Iudæorum, huius absurditatis \& nugatoriæ
caussæ conscij, his ineptiis explosis, dicunt, illum annum non a
libro inuento, sed Iubilei fuisse tricesimum.

At hoc est litem lite decidere.

Nam, quomodo Iudæi annos a Iubileo putarent, qui Iubilea
numquam vsurparunt?

Annos quidem Hebdomadis notant, vtinitio
\textsc{xxviii} Ieremiæ mentio anni quarti septimanæ: \textit{Initio regni
Sedekia, anno quarto.}

Rursus mentio anni primi, \& secundi in annis
\textsc{xiiii}, \& \textsc{xv} Ezekiæ, apud Isaiam \textsc{xxxvii}, 30.

Sed notationem
per Iubilea, imo ne Iubilei quidem mentionem, nusquam, nisi
in lege, reperies.

Præcepta fuit tantum, non recepta Iubilei obseruatio.

Sed quæ hæc plumbea Iudæorum sententia a \textsc{xviii} Iosiæ
Iubileum putare?

Iubilea putantur a primo anno hebdomadis, non
a septimo.

At \textsc{xviii} Iosiæ suit septimus septimanæ, non primus.

Quare, si a Iubileis annos putare mos esset, suerit hic annus non vtique
tricesimus, sed vndetricesimus Iubilei, a \textsc{xviiii}, non a
\textsc{xviii} Iosiæ.

Denique is erat annus 862 ab excessu Mosis, 855 a
diuisione terræ siue \textgreek{[Greek]}.

Ergo suit vicesimus secundus, non
vndetricesimus Iubilei.

En quot errores locus præpostere sumptus
nobis peperit.

Cum igitur neque a libro legis inuento, quod est absurdissimum,
neque a Iubileo, quod est falsum dupliciter, ille tricesimus
annus putandus sit; sequitur, quod negari non potest, a
quodam rege tunc imperante putandum esse.

Nam deportati \& captiui
inter victores, qua epocha vti possunt, nisi victoris?

In Palæstina,
cum aliqua esset Iudæorum Respublica, \& Ecclesia bene constituta,
Iudæ cogebantur vti anno Alexandreo dominorum Seleucidarum:
quanto magis Chaldæorum, in media Chaldæa, nullis legibus,
nulla Republica, nulla Ecclesia.

Nehemias initio libri sui ita
scribit: \textit{Accidit mense Casleu, anno vicesimo, cùm eßem in castro Susan.}

Si alibi non expressisset se de vicesimo anno Artaxerxis loqui, haud
dubie aliquod Iubileum hic commenti essent inepti Iudæi, \& inepti
quidam hominum nostrorum sequuti essent.

Eodem quoq; modo
loquitur Ezekiel \textit{anno tricesimo}, non adiecto regis nomine.

Quid enim opus erat in Chaldæa?

Duo ergo Reges simul imperabant,
Nabuchodonosor, \& ille, qui iam tricesimum annum currentem
imperabat.

Quisnam Rex, obsecro, potuit trecesimum annum in regno
agere, cum iam Nabuchodonosor tertium decimum regnaret?

Non alius igitur suerit, præter Nabopollassarum patrem Nabuchodonosori,
quod verum est.

%\end{parnumbers}
\clearpage
p. XIII [pdf 40]
%\begin{parnumbers}
Nam \textsc{xxix} solidos annos imperauit,
teste Beroso.

Quod si filius eius anno \textsc{xxx} partis iam duodecimum
absoluerat, profecto imperare cæperit anno partis decimooctauo,
qui erat Nabonassari 140.

Nam primus Nabopollassari est 123 Nabonassari,
testibus Chaldæis apud Ptolemæum, ex defectibus Lunaribus
obseruatis.

Et proinde Sedekias captus fuit anno 158 Nabonassari,
tertio autem Olympiadis 47.

Vide locū Berosi apud Iosephum.

Nabopollassarus audita rebellione Ægypti misit filium eo
cum regio imperio, \& regio exercitu: a quo tempore consurgit initium
Nabuchodonosori cum patre regnātis.

Mos erat Regum Babyloniæ
\& Persidis, vt aut prosecturi in expeditionem, filios reges declarent,
aut in expeditionem mitterent cum regio nomine, tanquam
designatos, si contigisset ipsum patrem mori, absente filio, ne
vllus de rege futuro tumultus oriretur.

Exemplum habemus apud
Herodotum de Cyro Cambysen in solium suum collocante in expeditione
in Scythas.

Hinc Ctesias Cambysi attribuit annos 18,
cum tamen solus regnarit octo annos, testibus omnibus veteribus
Græcis, \& Chaldæis ipsis apud Ptolemæum.

Dario vero Notho annos
idem attribuit 35, cum tantum 19 solus imperarit.

Rursus Berosus
\textsc{xxxxiii} annos ait Nabuchodonosorum imperasse, comprehensis
nimirum 13 annis, quos cum patre communicauit, cum
illis quos solus in imperio transegit.

Quare Nabuchodonosori regnum
dixit non Satrapian, tanquam a patre non vt Satrapes, sed Rex
\& socius imperij in rebelles missus.

Verba eius sunt hæc: \textgreek{[Greek]}.

\textit{Victo rebelli, eius regionem regno suo subiecit.}

Mox subiicit, Nabopollassari patris morto[?]
audita, qui \textsc{xxix} annos solidos regnauerat, ipsum Babylonem se
contulisse: quod accidit proculdubio aliquot diebus post illud tēpus
ab Ezekiele designatum.

Obiit enim Nabopollassarus anno regni
sui \textsc{xxx}.

\textgreek{[Greek]}.

Pulcherrima hæc est obseruatio, quam Beroso vernaculo
Babylonicarū rerum scriptori debemus.

Eadem verba repetit Eusebius
De pręparatione euangelica, vbi plane \textgreek{[Greek]}, quemadmodum
est apud Ptolemęum, nominat, nō \textgreek{[Greek]}, vt perperam
est editū in Iosepho: ex quo ineptus quidā duos esse coniecit
Nabulassarum \& Nabopollassarum; cum tamen eadē verba sint, ne
vna quidem syllaba minus, præter illud nomen.

Rursus apud Iosephum
lib.\textsc{x} ca.\textsc{ii}.eadem verba Berosi repetuntur.

Sed vbi hic est \textgreek{[Greek]},
ibi est bis \textgreek{[Greek]}, vtrobique male pro \textgreek{[Greek]}.

Quā bene hæc diuinis scripturis conueniunt?

%\end{parnumbers}
\clearpage
p. XIV [pdf 41]
%\begin{parnumbers}
Vnde etiam sequitur, mortuo Nabopollassaro, non tricesimum annum Nabuchodonosori
dici cæptum in Chaldæa, sed primum quæ res obseruatione
digna.

Iudæi primum annum putarunt ab eo tempore, quo
cum imperio missus est. Sed in Chaldæa primus eius annus consurgit
ab obitu patris.

Itaque Danielis 11, annus secundus Nabuchodonosori
est sine dubio secundus ab obitu Nabopollassari, tricesimus
primus ab initio eiusdem, 152 ab initio Nabonassari, sextus
Sedekiæ.

Vnde indubitata eruintur temporis nota illius Capitis secundi
apud Danielem, qui erat quartusdecimus nnus capti Danielis,
\& sociorum cum rege Ioiakim, sextus autem regni Sedekiæ.

Proinde annus ille erat \textsc{xiiii} Nabuchodonosori in Syria, secundus
autem in Babylonia: non autem \textsc{xxv}, vt coniicit Hieronymus
ex quadam victoria Nabuchodonosori de Syria, \& Arabia, cuius
meminerit Borosus.

At Berosus loquitur tantum vsque ad obitum
Nabopollassari, qui erat \textsc{xiii} Nabuchodonosori eius filij.

His tam illustribus demonstrationibus sua somnia præserant, quibus antiquius
est somniare, quam vera dicere, aut nosse.

Nos ad reliqua
pergamus.

Annus capti Sedekiæ est 158 Nabonassari, 4124 in periodo Iuliana.

Deductis annis 907 solidis, relinquitur annus 3217
Exodi, qui est 2264 Iudaici Computi in quo sane Neomenia Nisan
habuit characterem feriam quintam, secunda Aprilis, Cyclo
Solis \textsc{xxv}, Lunæa \textsc{vi}.

Sed quadragesimus annus, \& quadragesimus
septimus, hoc est 2303, \& 2310 Iudaicus fuit sabbaticus.

Iosuæ
\textsc{xiiii}, 7, 10.

Iudæi dicunt septenarios annorum Computi siue æræ
suæ esse Sabbaticos.

Atqui 2303, \& 2310 sunt septenarij.

Ergo recte
Sabbaticos annos putant Iudæi; vt apud illos post legem nihil
hac obseruatione vetustius sit; res prosecto, quæ firmissimum
minimentum futura sit harum rerum inuestigatoribus.

Neomenia Nisan Exodi conueniebat cum neomenia Krionos.

Ita vere naturalis suit illa neomenia.

Præterea quadragesimus septimus
annus conuenit sabbatico Iudaico: 902 autem annus est tricesimus
Nabopollassari conueniens cum testimonio Ezekielis.

Deniq; anni 86 a septimo Cambysæ retro putati desinunt in anno
cædis Nechao Ægyptij, eodemque 139 Nabonassari: quod conuenit
eidem computationi.

Negari igitur non potest, hanc esse veram
Exodi epocham, quam \& verbum diuinum, \& vsus anni Sabbatici,
\& historiæ fides penes eximium scriptorem Chaldæum
Berosum, \& naturales neomeniæ vtriusq; sideris in vnum conuenientes
confirmant.

Quid postulamus præterea?

An vt tam certis,
tam egregiis, tam firmis argumentis somnia Corybantum anteponamus?

Quis vnquam ita hæc demonstrauit?

Quid demonstrauit?

%\end{parnumbers}
\clearpage
p. XV [pdf 42]
%\begin{parnumbers}
Quis aliter potest demonstrare?

Iam a conditu rerum, ad exodum,
anni sunt absoluti 2452 cum mensibus sex ab autumno, anni vero
absoluti 2453 a vere.

Sed ante Exodum initium anni putabatur ab
autumno, \& eodem initio in tempus veris translato, tekupha tamen,
hoc est, finis anni Solaris mansit in autumno, circa quam tekupham
Deus \textgreek{[Greek]} celebrari præcepit.

Igitur vbi initium anni
ab vltima antiquitate suit, inde \& rerum quoq; initium repetendum.

quod quidem a nobis factum, damnata priori sententia, quæ
initium rerum statuebat in vere.

Reliqua pete ex capite de conditu
rerum.

Præterea, quibus annus Lunaris in vsu est, illis commodius
initium, \& rationibus Tropicis conuenientius ab autumno, quam
a vere, vt Iudæis propter \textgreek{[Greek]}, \& Pascha.

Nam si annum
nostrum cælestem admitterent, \& hoc vnum cauerent, vt \textgreek{[Greek]}
citima sit in secunda Zygonos, semper citimum Pascha esset in neomenia
Krionos.

quia interuallum a neomenia Zygonos, ad neomeniam
Krionos, est semper 178 dierum, vno die plus, quam a scenopegia
ad Pascha.

Anni Sabbatici caussas iam reddidimus, \& verum
annum sabbaticum a Iudæis hactenus obseruari demonstrauimus,
initio hebdomadum sumpto, non vtique a defectu Mannæ,
quod fanatici quidam, \& veritatis hostes faciūt, sed a 48 anno Exodi,
ex capite \textsc{xiiii} Iosue, \& rationibus doctorum Habræorum, qui
dicunt septem annos \texthebrew{[Hebrew]}, id est, subiugationis terræ,
septem \texthebrew{[Hebrew]}
fuisse, id est, diuisionis.

quod rectissimum est: ideoq; hebdomadem
primam diuisionis, non subiugationis procedere in numerum.

An
potuit annus sabbaticus esse ante agrorum culturam?

Furor est aliter putare.

Tamen non desunt, non deerunt, qui solo contradicendi
studio, vt sapere videantur, aliter staduent: quibus per me non solum
hoc facere, sed etiam nos irridere licet; quandoquidem veritas apud
illos nullo in precio est.

Vnde nata sit diuersitas epochæ excidij Ilij,
cum alij 407 annis, alij 405, eum casum antiquiorem prima Olympiade
statuant, aperuimus ex doctrina anni Attici, cui acceptum
referimus quicquid eximium ex alta obliuione eruimus.

Veram sententiam
Eratosthenis esse deprehendimus, quæ illam cladem coniicit
in annum 407 ante caput primæ Olymiadis: eiusque veram
diem in anno Iuliano ostendimus.

Primam autem Olympiadem
ex doctrina itidem anni Græci \textsc{xxiii} die Iulij celebratam fuisse ante
nos aperuerat nemo.

Et tamen quidam Simioli tanquam rem
vulgatam in suis vanidicis Chronologiis retulerunt: cuius rei cognitionem
vnus Pindarus, quem illi neq; viderunt, neq; norunt, nos
docuit.


%\end{parnumbers}
\clearpage
p. XVI [pdf 43]
%\begin{parnumbers}
Quemadmodum autem Olympia, ita etiam Karnia plenilunio
celebrata fuisse, libro primo, capite de periodo Laconum
ostendimus. neque solum plenilunio, sed etiam eodem anno, quo
Olympia.

Itaq; Herodotus libro \textsc{viii} Olympia \& Karnia anno primo
Olympiadis 75 celebrata suisse scribit, pag. 307 editionis Henrici
Stephani nostri.

Cum multi eruditissimi viri, \& quidem in iis
Onufrius Panuinius Pater historiæ, multa accurate de Palilibus Vrbis
disseruerint, vt ei doctrinæ nihil ad perfectionem deesse videatur,
tamen \& plura deesse ex nostris disputationibus colligi potest.

Monere vero debent Annalium \& Fastorum scriptores, qui tempora
sua ad annos Vrbis dirigunt, vtra Palilia sequantur, Varroniana,
an Catoniana.

Nam certe Onufrius noster, tametsi Catonem sequitur,
tamen quibusdam imprudens ad Varronem transfugit.
% "transfugit" should not be rendered with a long s

Nisi
hæc distinctio adhibeatur, ridicula multa consequi necesse est.

Exemplum habemus in annis Christi per annos Vrbis eruendis,
quod hactenus ab omnibus factitatum.

Christus in annis Varronianus
vno anno maior est apud aliquem, quam in Catonianis apud alium.

Quare, vt dixi, ridicula sunt.

In sequentibus epochis quanuis
non ea occurrit obscuritas, quæ in prioribus: tamen semper aliquid
noue demonstratur, præter superiorum scriptorum consuetudinem:
in quibus sunt quædam de vero die \& anno natalis Alexandri, eiusque
obitus: de Encæniis Machabæi, de initio Simonis Iudæorum
Ethnarchæ, quem Iudæi Iohannem vocant, de æra Hispanica.

De quibus omnibus pluria noua disseruntur, quam trita \& vulgaria.

Iam
excessum Herodis ad suum verum annum ex Iosepho retulimus,
qui ad epocham Actiacam illud tempus diligenter exigit, \& præterea
notationem, cui contradici non possit, adducit, defectum Lunarem,
qui contigit \textsc{ix} Ianuarij, anno 45 Iuliano ineunte, in cuius
anni sequenti Decembri Dionysius Exiguus imperite statuit natalem
Christi, nouem solidis mensibus scilicet post excessum Herodis.

Itaq; diligentissimus \textgreek{[Greek]} omnium scriptorum Iosephus
recte ait decessisse \textsc{xxxv} anno labente regni eius a captis a Sofio[?]
Hirosolymis. in quo tamen interpretatio adhibenda.
% Sofio or Sosio

Nam reuera Herodes
obiit anno tricesimo sexto ex diebus æstiuis noni anni Iuliani.

Ergo tricesimus sextus annus Herodis iniuit ex diebus æstiuis anni
Iuliani \textsc{xliiii}.

Obiit autem initio Nisan.

Igitur sine dubio decessit
anno Iuliano \textsc{xlv}, qui erat tricesimus sextus iniens ex diebus æstiuis,
vt diximus.

Sed ex computatione ciuili Iudæorum, nondum
\textsc{xxxvi} annus iniuerat.

Iosephus enim, \& Iudæi eo sæculo putabant
omnia tempora a \textsc{xxiii} Ijar, vt albi ostendimus: cuius consuetudinis
ignoratio multos decepit.

Ab Ijar igitur Hyrcani, siue, vt Iudæi
vocant, Iohannis Hasmunai, tricesimus sextus annus Herodis inibat,
qui tamen iam nouem mēsibus ante ex consuetudine Romana iniuisset.

%\end{parnumbers}
\clearpage
p. XVII [pdf 44]
%\begin{parnumbers}

Itaq; eius decessus confirmatur primum accurata putatione
diligentissimi scriptoris, deinde notatione eclipsis, quæ omnem contradictionem
excludit.

At ex epilogismis Eusebij Herodes obierit
anno Iuliano \textsc{lii}, septem annis solidis post illum defectum.

qui stupor non meret castigationem, cum tanquam sorex indicio suo perierit.

Nam statim ab eius decessu tetrarchiam suam Archelaus eius filius
iniuit: quod quidem, si huic oraculo Eusebiano credimus, contigerit
anno Christi Dionysiano septimo labente.

Ergo Christus fuerit
annorum septem, cum ex Ægypto monitu Angeli reuoctus est.

Quod est ridiculum.

Rursus anno decimo regni, aut tetrarchiæ suæ
Archelaus ab Augusto relegatus est Viennam Allobrogum.

Secundum tempus ab Eusebio determinatum, hoc contigerit anno Iuliano
\textsc{lxi}, qui erat annus Tiberij tertius currens, biennio absoluto
post excessum Augusti.

Hoc modo anno tertio excessus sui Augustus
Archelaum relegauerit.

Vides \textgreek{[Greek]}.

Atqui innumeros videas,
quibus hoc somnium placet.

Nam sane omnes fere Chronologiæ
\& Annales hoc stigmate inusta sunt.

Atque vtinam in illis hominibus
non esset vir eximia doctrina præditus Dominus Cæsar Baronius,
Annalium Ecclesiasticorum scriptor, cuius operis copia nobis
facta est ab amicis, cum hæc \textgreek{[Greek]} scriberemus.

Is eruditissimus
vir ex hoc loco Eusebij Iosephum exagitat, tanquam imperitum
temporum: cum Eusebius potius ex Iosepho castigādus fuisset.

Nam absque Iosepho esset, quid certi de Herode haberemus?

Quis hæc tractauit, præter illum?

Qui fieri potuit, vt scriptor, cuius diligentia
\& fides in notatione temporum spectatissima, in iis peccauerit,
quæ sine illo Eusebius \& alij ignorassent?

Sed ipse doctus Annalium
conditor potest iam videre, vtri fides de hac re habenda, Iosepho,
cuius ratiocinia cum motibus cælestibus congruunt, an Eusebio,
cuius sententia \& historiæ, \& rationi aduersatur?

Sed de Iosepho
nos hoc audacter dicimus, non solum in rebus Iudaicis, sed etiam
in externis tutius illi credi, quam omnibus Græcis, \& Latinis.

Itaque
definat mirari doctus vir, cur tot eruditi, \& nos quoq; qui non in illis
eruditis, sed in huius scriptoris lectione peregrini non sumus, tantum
illi deseramus, cuius fides \& eruditio in omnibus elucet.

Cæterum de Eusebij anilibus hallucinationibus, præter hanc, quam
modo protulimus, satis libro sexto differuimus.

Sed ad Epochas
nostras venio: quarum omnium rationem reddere longum esset.

De Epocha Martyrum Diocletianea non possumus tacere, eam hactenus
etiam doctissimis imposuisse, quod eam ab initio Diocletiani
incipere omnes credunt.

Hinc prodigiosi errores, \& magna Consulum
confusio in Annales \& Fastos deriuata sunt, præsertim in annis.

%\end{parnumbers}
\clearpage
p. XVIII [pdf 45]
%\begin{parnumbers}
Nam initio Diocletiani perperam sumpto, perperam quoque
persecutionis Epocha initur.

Ea semper antiquitus a solis Ægyptiis
Christianis hactenus vsurpata fuit.

Itaque Historici \& Chronologi,
qui temporibus Caroli Magni dicunt cæptum putari ab annis
Christi, cum antea mos esset annis Diocletiani vti, errant.

Nam
nullis nationibus in vsu fuit.

Vnica autem Ecclesia duntaxat Alexandrina,
\& quæ illi subditæ sunt, hac Epocha vsa est semper, vtirurque
hactenus, \& vocatur ab Ægyptiis, qui Elkupt dicuntur,
\textarabic{[Arabic]} \textit{Æra Martyrum sanctorum.}

Nam
hallucinatus est ille, qui nuper \textarabic{[Arabic]}
\textit{Captiuitatem} vertit in literis
Alexandrinæ Ecclesiæ Romam missis, anno Martyrum 1310, qui
erat Christi 1593.

Epocha igitur Martyrum iniuit \textsc{xxix} Augusti,
id est, neomenia Thoth Actiaci, vel Mascarā Habesseni, anno Christi
Dionysiano 284.

Initium autem imperij Diocletiani a Palilibus
anni 287.

Differentia anni duo, menses octo.

Perturbatio, quæ est in
Consulibus a temporibus Maximinorum, vsq; ad filios Constantini,
ea vtique ab antiquo est.

Sed \& non minor confusio in annis persecutionis:
vbi magnæ sunt \textgreek{[Greek]} apud Eusebium: quanuis
recte sentit de initio Diocletiani, \& primo anno persecutionis.

Tamē
omnium Chronologorum fides hac in parte nutat.

Nam edictū
Diocletiani de tradendis codicibus prius est Ecclesiarum euersione,
euersio Ecclesiarum prior cæde Martyrum.

Felix Africanus Episcopus
\& socij eius supplicio in Campania affecti ideo, quod codices
Deificos, id est, sacram scripturam tradere noluissent.

Itaque in
Actis illorum scriptū fuit: \textit{Et ductus est ad paßionis locum, cum etiam
ipsa Luna in sanguinem conuersa est, die tertio Kalen. Sept.}

De Eclipsi
Lunari loqui manifestum est, cuius is color fuerit, quem sanguineum
astrologi vocant: cuiusmodi proculdubio accidit anno Christi
301, cyclo lunæ 17, annis quatuor solidis ante edictum de euertendis
Ecclesiis, idque \textsc{iii} Nonas Septembris, non autem \textsc{iii} Kal.
Sept. diebus quatuor post passionem Martyrum.

Itaq; perturbatus
est ordo verborum.

Legendum enim videtur: \textit{Et ductus est ad paßionis
locum, die tertio Kal. Sept. cum etiam ipsa Luna in sanguinem conuersa
est.}

id est, quo tempore Luna defecit, proximo nimirum nouilunio.

Nam cum constet passos \textsc{iii} Kal. Septembris, \& ita habeat
Kalendarium, non videtur esse error in notatione temporis.

At Dominus Baronius hæc gesta confert in annum 302, tribus annis ante
persecutionem: \& tamen putat eum esse secundum annum persecutionis,
qui erat decimus nonus Æræ Martyrum, decimus autem
septimus currens ab imperio Diocletiani.

%\end{parnumbers}
\clearpage
p. XIX [pdf 46]
%\begin{parnumbers}
Sed \textgreek{[Greek]} illorum
Annalium propagati sunt partim ex erroribus aliorum Chronologorum,
quos auctor sequitur, partim ex annis Christi male ad
suam \& veram epocham reductis.

Vnde factum, vt ap initio operis,
ad tempora Nicenæ synodi, ne vnus quidem annus Christi
veræ epochæ suæ redditus sit.

Itaque triennio aliquando, aliquando
quadriennio, vt plurimum autem biennio erratum est.

Exempli
gratia: Excidium Hierosolymorum contigit anno Christi
Dionysiano \textsc{lxx}, quo neomenia Nisan conueniebat cum neomenia
Xanthici, teste Iosepho.

In Annalibus refertur ad annum
72: qui est error Eusebij, sed alibi ab eodem castigatus.

Certum est, Fructuosum Episcopum, Christi Martyrem, cum fociis
passum anno antequam pax \& interspiratio data esset Ecclesiis
sub Marco Aurelio Antonino, \& L. Ælio Vero.

quod tempus Eusebius confert in annum quartum Olympiadis \textsc{ccxxxiiii},
id est Christi Dionysianum 160.

Ergo passus est Fructuosus anno Christi
159.

Hoc aliter demonstrabimus.

In Actis agonis Fructuosi \&
sociorum legitur: \textit{Producti sunt duodecimo Kalend. Februarii, feria
sexta.}

Ergo litera Dominicalis erat B.

Proinde hoc accidit anno
159, triennio citius, quam notatum in Annalibus.

In Actis Andreæ
militis \& sociorum scriptum extat, eos necatos fuisse decimoquarto
Kalendas Septembris, Dominico die, hora secunda.

Igitur litera Dominicalis erat G.

Hoc necessario contigit anno 305,
qui erat primus persecutionis a Pascha illius anni antecedente, post
euersas Ecclesias: quod quidem Pascha celebratum 25 Martij, ipso
die termini.

At in Annalibus hoc refertur in annum 301, quadriennio
ante rem gestam.

Rursus in Epistola Vigilij Episcopi Tridentini
de Passione Sanctorum Sisinnij, Martyrij, \& Alexandri,
ita legitur: \textit{Die paßionis Sanctorum, quarto Kalendas lunias, feria
sexta, nascente luce.}

Passi ergo sunt anno 403, cyclo Solis \textsc{xx}, quando
\textsc{xxix} Maij erat feria \textsc{vi}.

At in Annalibus dicitur scripta
anno 400 Christi.

Scripta ergo fuisset triennio ante cædem
ipsorum Martyrum.

Cui absurditati ipse non adscribet, certo scio.

In iisdem
Annalibus ex codice Antonij Augustini mentio fit Homiliæ
Cyrilli Episcopi dictæ in natiuitate Ioannis Baptistæ, Pharmuthi
vicesima octaua, indictione prima, sub Theodosio iuniore \& Valentiniano.

Ergo dicta fuit Homilia anno Christi 433, April. vicesima
tertia.

At in Annalibus refertur in annum 432, April. 29. S. Benedictus
Monachorum Occidentis Pater, obiit \textsc{xi} Kal. Aprilis, Sabbato
sancto, vt refert Aimoinus monachus ex Actis S. Mauri ipsius
Benedicti discipuli.

Toto illo sæculo hoc non potuit contingere, nisi
anno 536.

%\end{parnumbers}
\clearpage
p. XX [pdf 47]
%\begin{parnumbers}
Tamē in Annalibus Ecclesiasticis obitus Benedicti cōfertur
in annum 542, sex annis serius.

Multa igitur peccari necesse est
in Gestis Benedicti, quæ in illis Annalibus referuntur.

In Encyclica
epistola Vigilij Papæ scriptum fuit: \textit{Piißimus atque clementißimus
Imperator Dominico die, id est, Kalendis Februarij, gloriosos Iudices suos
ad nos destinare dignatus est.}

Anno 554 Kalendis Februarij fuit dies
Dominica.

At in Annalibus hoc confertur in annum 552, duobus
annis citius.

Anno 546 turbatio facta in Pascha, vt ex Cendreno docuimus,
capite de periodo Dionysiana, libro \textsc{iiii}.

In Annalibus referetur
sub anno 545.

Martinus Episcopus Turonensis obiit anno
395, vt accurate a nobis disputatum est.

Auctor Annalium Sigebertum
sequutus coniicit in annum 402.

Ex eo errore multū peccatum
est in temporibus Regum Francorum.

de quibus consulatur vltima
diatriba libri sexti huius operis nostri.

Non semel monuimus magnam
perturbationem esse in initiis Imperatorum, a Maximinis
ad Valentinianum.

Vt alios taceam, Constantini initium ab aliis in
305, ab aliis in 306 annum coniicitur.

At Constantinus iniuit imperium
post obitum patris sui Chlori.

Obiit autem Chlorus in Britannia
anno primo Olympiadis 271, vt inquit Socrates.

Nos ostendimus,
apud Socratem, Hieronymi Supplementū, Ausonium, \& alios,
semper Olympiadem sumi pro lustro Iuliano, non pro lustro Olympico
Elidensium, idq; lustrum Iulianum biennio posterius esse Elidensi,
cum incipiat ab anno Iuliano bisextili.

Itaq; is fuit annus bisextilis,
quo obiit Chlorus, \& imperium iniuit Constantinus.

Sed duæ
cautiones adhibendæ.

Prior est, vt scias annum Constantinopolitanum,
siue Nicenum hic intelligi, qui incipiebat a \textsc{xxiiii} Septembris.

Altera, vt prolepsis vsurpata intelligatur in anno mortis Chlori.

Nam obiit \textsc{xxv} Iulij, \textsc{lxi} diebus ante
\textsc{xxiiii} Septembris, \&
tamen obitus eius ad eundem annum refertur quo iniuit imperium
eius filius, \textgreek{[Greek]}, vt dixi.

Omnino igitur iniuit imperium anno
303, aut 307.

Nam primus annus Olympiadis Iulianæ incipit semper
diebus 153 ante bisextum.

Sed nemo concedet Chlorum obiisse
anno 303.

Obiit ergo 307.

Et proinde anno 307 iniuit imperium
Constantinus, ex ante diem \textsc{viii} Kal. Octobr. eiusdem anni 307.

In his prouocamur a docto Annalium scriptore, \& rem absurdissimam
prodidisse nos dicit, Constantini imperium iniisse ex anno
308, cum, vt inquit ipse, iniuerit anno primo Olympiadis 271,
Christi vero 306[?].
% 300 or 306 ?

Nos vero negamus vllam culpam aut absurditatem
in nobis admissam.

Nam annus Christi 308 Constantinopolitanus
incipit a Septembri anni 307, vt iam dictum est.

Et proinde ipsum, \& alios errare, qui annum Christi 306 a Kalendis Ianuarij
dicunt esse annum labentem Constantini.

%\end{parnumbers}
\clearpage
p. XXI [pdf 48]
%\begin{parnumbers}
Hoc enim volunt,
cum putant primum 271 Olympiadis Elidensis annum esse primum
Constantini.

Olympias enim illa Iphitea cæpit ex diebus æstiuis
anni 305, qui fuit annus primus presecutionis.

Quare in annis
Constantini, vt in aliis, insigniter peccatum est a viro docto.

His
postis, quinquennalia Constantini data sunt anno 312: vicennalia
autem anno 327.

Interuallum inter illas duas celebritates interiectum
haud dubie vocatur Indictio, iniens a datis quinquennalibus,
desinens[?] in vicennalibus, quibus concilium Nicenum dimissum.
% desinens or definens?

Sed neq; hoc placet Domino Baronio: neque caussam appellationis
Indictionum admittit.

At nos dicimus, non minus iniuste nos
hic, quam in initio imperij Constantiniani reprehendi.

An negat
Indictiones in quinquennia indici, \& in quinquennalibus Principum
panegyribus remitti?

Si non credit, legat \& quæ priore, \& quæ
hac editione ad eam rem collegimus.

Quinquennalia illa dicuntur
\textgreek{[Greek]}, hoc est ad verbum, sparsiones, largitiones, profusiones, in
quibus liberalitas Principis ad remissionem vsq; tributorum, \& indictionum,
editiones munerum \& spectaculorum, congiaria, \& donatiua
extendebatur.

Inde \textgreek{[Greek]} non solum pro illa largitione
sumitur, sed \& pro ipsa indictionis temporalis nota.

Nam quod Latini
dicunt, Indictione prima, secunda, tertia hoc factum est, Græci
dicunt, \textgreek{[Greek]}.

Non ergo nos, sed ipse fallitur.

Quid?

si initium Constantini a nobis ignoraretur, tamē quinquennalia
eius nos manu ad illud deducerent.

Itaque ignorari n n [?]
potest.
% Probable printing error. "n n" should read "non".

Neq; minus errat, cum cladem Maxentij coniicit in annum
312.

Quot modis enim hoc refelli potest?

Sed de eo suo loco.

Nam
Maxientius anno 313, non 312 extinctus est, vt recte Panuinius notat,
sed male inde Indictionum initia \& caussas repetit: quod a nobis
olim diligenter discussum fuit.

%%% === Sextus Liber
Sextus liber continet residuum Epocharū,
in quo nobiliores quæstiones de Natali die, \& Passione Christi,
de Hebdomadibus Danielis, quæ breuibus diatribis explicari
non possunt, presequimur.

Ne autem aut rudiores, aut refractarij auctoritate
veterum scriptorum nobis præscribere possent, pauca de
Eusebij erroribus in antecessum delibauimus, in quibus, præter frequentes
\textgreek{[Greek]}, puerile illud deliramentū de Effenis confutauimus,
quos Christianos fuisse hoc vnico argumento probat, quod
\textgreek{[Greek]} essent, \& solitarie viuerent, \& monasteria haberent:

quasi Bonzios
Iapanensiū Christianos esse censeamus, quia \& cœnobitæ sunt,
\& Psalmos quosdam instar monachorum Europæorum alternis modulantur,
\& horas Canonicales eorum exemplo habent.

Eorum Essenorum alij \textgreek{[Greek]}, alij \textgreek{[Greek]} fuerunt.

Sed horum non videtur
secta diuturna fuisse.

%\end{parnumbers}
\clearpage
p. XXII [pdf 49]
%\begin{parnumbers}
Ast \textgreek{[Greek]}, aut eorum non dissimilium
synagogæ fuerunt ad tempora Iustiniani.

Sunt enim ij, qui Cælicolæ
vocantur.

Nam \& nomen id indicat.

Cælicolæ enim sunt
Angeli.

Ita vocari volebant, propter sanctum, \& cæleste, vt ipsis videbatur,
vitæ institutum.

In perueteri Glossario Latinoarabico \textit{Cælicola}
[Greek][Arabic][?]. id est, Angelus.

Præterea quia erant \textgreek{[Greek]}, noui
baptismi auctores Donatistis fuerunt.

Princeps eorum vocatur
Maior, vt \& aliorum Iudæorum.

Hoc enim est \texthebrew{[Hebrew]}.

Philo dubitans
quare Esseni illi dicti sint \textgreek{[Greek]},
 vtrum quia medicinam profiterentur,

an quia Deum colerent, ex eo coniiciendum relinquit,
eos non dictos esse quasi \texthebrew{[Hebrew]} \textgreek{[Greek]},
 vt volebat quidam Lunaticus
literarum Hebraicarum professor, sed quia \textgreek{[Greek]} vocat, eo ostendit
\texthebrew{[Hebrew]} dictos, hoc est, \textgreek{[Greek]}.

Quod Christiani non essent, sed
mere Esseni, statim initio libri ostendit Philo.

sed \& Sabbati summus
cultus, \& reliqua, quæ a Philone de ipsis narrantur, satis leuitatis
damnant Eusebium, \& reliquos veteres, qui Eusebium sequuti,
idem hariolati sunt.

Sed in Annalium tomo primo tacite perstringitur
sententia nostra ab auctore, qui tamen fatetur veros Essenos Iudæos
fuisse.

Mirati sumus, quomodo ille putauit in vnum hæc bene
conuenire posse, Iudaismum \& Christianismum.

Vt hoc probet, ait
veteres patres idem scribere, quod Eusebium.

Atqui ex Eusebio
hoc desumpserunt, \& eius auctoritate contenti Philonem non consuluerunt.

quem si legissent, nunquam tam ridiculæ sententiæ assensum
accommodassent.

Hæc vero puerilia sunt.

Venio nunc ad natalem
Christi, quem vetustas Christianismi ad \textsc{xxviii} annum Actiacum
retulit, recte.

Nam Christus iniens annum vnum a tricesimo
ætatis suæ accessit ad baptismum, vt omnes vetustissimi Patres ex
Luca retulerunt, \& post eos eruditus Annalium scriptor.

Baptizatus est anno \textsc{xv} Tiberij, duobus Geminis \textsc{coss}. anno
Iuliano 74.

Ergo \textsc{xxv} Decembris anni 73 illi inibat annus primus a tricesimo.

Deductis 30 annis absolutis de 73, remanet annus Iulianus
43, in cuius \textsc{xxv} Decembris natus fuerit Dominus, cyclo Lunæ
\textsc{xviii}, anno Actiaco \textsc{xxviii},
 vt illi vetustissimi partes crediderunt,
duobus annis solidis ante epocham hodiernam Dionysianam,
anno solido cum diebus aliquot ante excessum Herodis.

Hoc proculdubio
verum est.

Sed in Annalibus peccatur ab auctore in anno
\textsc{xv} Tiberij.

Quem enim putat \textsc{xv}, is est \textsc{xvi}, \& magno errore illi
attribuit Consules duos Geminos, quibus Consulibus annus \textsc{xvi}
Tiberij iniit ex \textsc{xix} Augusti, cyclo Lunæ vndecimo, anno Iuliano
74.

Nisan igitur is, qui proxime sectus est baptismum Christi,
Consulibus duobus Geminis, antecessit annum \textsc{xvi} Tiberij ineuntem,
mensibus quinque.

%\end{parnumbers}
\clearpage
p. XXIII [pdf 50]
%\begin{parnumbers}
At scriptor Annalium putat duos Geminos
Consulatum gessisse cyclo Lunæ \textsc{xvi}: in quo ne sic quidem
sibi constat.

Nam is fuerit annus 75 Iulianus iniens.

Hoc modo Decembri anni 74 Christus iniuerit annum primum a tricesimo: \&
deductis 30 absolutis, remanebit annus 44 Iulianus, in quo natus
Christus fuerit, tribus circiter mensibus ante excessum Herodis, anno
solido ante epocham Dionysianam, qua hodie Ecclesia vtitur.

quæ sane multorum veterum, inque illis Eusebij fuit opinio.

Sed
Christus baptizatus anno 74 Iuliano: passus 78.

Differentia, anni
quatuor solidi, paschata quinque.

Quorum nullum vestigium in illis
Annalibus extat.

Quinetiam auctor, quando numerus annorum
non succedit ex voto, culpam in Iosephum reiicit, mendacem multis
modis arguens: inter alia, quod scripserit \textgreek{[Greek]} factam post
Archelai relegationem, cum, inquit, ea \textgreek{[Greek]} Christo nascente
contigerit, \& aperte Eusebius id indicauerit.

Nos hallucinationem
Eusebij loco suo confutauimus, in quo descriptionem patrimonij
Archelai cum descriptione totius orbis Romani confundit more
suo, neq; meminit verbis illis, \textgreek{[Greek]}, designari non
vnicam fuisse illam descriptionem, cum \textgreek{[Greek]} mentio fiat.
% Final period not visible in original.

Quare
idem Euangelistes quemadmodum prioris meminit in Euangelio,
ita alterius mentionē facit in Actis.

vt nō sit audiendus doctus Annalium
scriptor, qui non solum hac in parte Eusebij auctoritatem
Iosepho opponit, sed etiam adiicit descriptionem illam[?] eandem esse,
de qua Æthicus statim initio libri sui loquitur: cum tamen neque
tempus, neque res conueniat[?] descriptioni nascente Christo factæ.

Nam descriptio, de qua intelligit Æthicus, cæpit ab anno cædis
Cæsaris, desiuit[?] in anno \textsc{xxxiii},
 qui erat tricesimus quartus a primis
Kalendis Ianuariis Iulianis, decem annis absolutis ante verum
natalem Christi, duodecim ante epocham Christi hodiernam Dionysianam.

Res autem eadem non est, imo longe diuersa: atq; adeo
tantem differt[?] descriptio, de qua Æthicus loquitur, a descriptione,
quæ facta Christo nascente, quantum decempeda, \& tabulæ [censuales][?].

Nam illa descriptio Æthici mandata est agrimensoribus, \&
Geometris, hæc Rationalibus.

Illa orbis mensura, \textgreek{[Greek]},
hac census \& facultates in Tabulas relatæ.

Sed neq; recte concludit,
Iosephum hallucinatum, quod paulo ante initia belli Iudaici
auditam ex adytis templi vocem scripserit, quæ diceret \textsc{hinc
migremvs}: cum, inquit, Eusebius id in passionis Dominicæ tempus
referat.

Quomodo Eusebius melius scire potuit ea, quæ contigerunt
Christi \& belli Iudaici tempore, quam Iosephu? aut vnde,
quam ex Iosepho? de illis dico, quæ non pertinent ad historiam euangelicam.



%\end{parnumbers}
\clearpage
p. XXIV [pdf 51]
%\begin{parnumbers}
Sed tam friuolum argumentum eluditur iis, quæ aduersus
hanc Eusebij hallucinationem libro sexto decimus.

Denique iniuste
vbique Iosephum reprehendit, omnium scriptorum veracissimum
\& religiosissimum, quod quidem ipsius scripta loquuntur.

quem
auctorem si non tam contempsisset, nunquā eos
 \textgreek{ανἀχρονισμους [Greek:anachronism]} commisisset,
quibus totus contextus temporum primi tomi perturbatus
est.

Sed antequam ex hac velitatione facessimus, qua \& nos \& cognominem
nostrum scriptorem ab animaduersione docti viri vindicamus,
nos homines Aquitani expostulamus cum eo, quod a nobis
tres summos viros abdixit, Paulinum, Phœbadium, \& Sulpitium
Seuerum:

qui cum suerint natione, \& domo Aquitani, tamen
Paulinum \& Sulpitium Romæ natos scribit, Phœbadium in Hispania.

Quis illum docuit Paulinum non esse natum Burdigalæ, vbi
antiquitus Paulina gens, hodieque quædam regio vrbis Burdigalensis
Paulino cognominis est?

Phœbadium autem Aginni Nitiobrigum
Episcopum quare in Hispania natum dicit, aut quo auctore?

Apud Hieronymum male excusum est Sœbadius, qui error irrepsit
ex Sophronio, vbi legitur \textgreek{[Greek]}.

Sed liber manu scriptus
Sanctæ Mariæ de Granateria liquido habet Febadium.

Apud Sulpitium
Seuerum deprauatum quoq; est, vbi legitur Fegadius, pro
Febadius, vt quidem librarij scribunt.

nam orthographia est \textgreek{[Greek]},
Phœbadius: satis hodie notus erudita sua in Arrianos Epistola,
quæ ante \textsc{xxv} annos primum edita.

Mei municipes Fiarium vocant,
cuius memoriam bis quotannis instaurant, ineunte ieiunio
quadragesimæ, \& die Marci Euangelistæ, mense Aprili, si bene
memini.

Huic successit Gauidius in episcopatu.

Sulpitium Seuerum
nemo hactenus Aquitanum fuisse dubitauit: sed patria ignoratur,
cum tamen ipse Nitiobrigem sese manifesto prodat, cum Seruationem
Tungrorum, Phœbadium autem suum Episcopum fuisse scribit.

Phœbadius autem erat Nitiobrigum Episcopus.

Iste Sulpitius
Ecclasiasticorum purissimus scriptor, post transitum Martini recepit
sese Elusonem, quo tempore ad eum scribebat Paulinus.

Id oppidum est cum arce veteri in finibus Nitiobrigum, qua amni Draguto
a Petrocoriis diuiduntur.

Vulgo \textit{Lausun}.

Sed de hoc satis.

Mei
Nitiobriges pro Sulpitio Supplicium dicunt, quomodo \& Bituriges
suum illum vocant, quem eundem cum hoc faciunt perperam,
cum inter transitum Martini, cuius noster Sulpitius discipulus fuit,
\& ordinationem Sulpitij Episcopi Bituricensis sub Guntchramno
Rege, intercedant plus minus anni 190.

Non iniuriam facimus
docto viro, si cum bona eius venia doctissimos viros Aquitanos,
\& Christianissimos originibus suis vindicamus.

%\end{parnumbers}
\clearpage
p. XXV [pdf 52]
%\begin{parnumbers}
Sed quemadmodum tribus viris Aquitaniam orbauerat, ita eandem duabus
alienis ciuitatibus donauit, Reiensi, \& Vasensi.

Prosperum non vno
loco dicit Regiensium in Aquitania fuisse Episcopum, cum dicendum
fuerit, Prosperum Aquitanum fuisse Episcopum Reiensium,
aut Regiensium in secunda prouincia Narbonensi.

Hodie \textit{Ries} vocatur.

Nugantur qui eum Regij Lepidi Episcopum \& scripserunt,
\& in fronte eius sacrorum poematum apponi curarunt: quasi Reienses,
in secunda prouincia Narbonensi, iidem sint cum Regio Lepidi
in Æmilia.

Vasense autem consilium idiotismus illorum temporum
vocauit, quod potius Vasionense dicendum erat.

Vasio Vocontiorum hodie \textit{Vaison} dicitur.

Est Episcopatus Auenioni metropoli
attributus.

Imperite quidam cum foro Vocontiorum confundunt.

Itaque Vasense, vel Vasionense, in Vasatense mutandum non
erat.

quemadmodum in anno Christi 552 perperam Firminum
Vticensem mutat in Venciensem.

Vticenses, vulgo dicuntur \textit{Vsetz}.

Est Episcopatus in prima Narbonensi.

Dicuntur etiam Vcetenses,
\& Vcetiæ Episcopus.

Apud Gregorium Turonensem libro \textsc{vi},
mentio est Ferreoli Episcopi Vcetensis: vbi vulgo male Vcecensis.

Sed tam imperite vulgus Vticenses deprauauit in Vcetenses, quam
Arausio in Aurasio: Vasensis dixit, pro Vasionensis.

At ciuitas siue
Episcopatus Venciensis, est in secunda Narbonensi. Vulgo S. Paulus
de Venciis.

Scribendum vero per t.[?] Ventiensis, \textgreek{[Greek]} enim dicitur
Ptolemæo.

Fuitque Nerusiorum in Alpibus Graiis Metropolis.

Sequuntur in sexto libro illa quinque Paschata a baptismo
ad resurrectionem, fuis temporibus, Consulibus, \& cyclis notata.

In tertio Paschate quid fit \textgreek{[Greek]}, explicamus,
quæ verissima interpretatio adhuc assensum vel mereri, vel
exprimere a doctis hominibus non potuit: quod valde miror,
cum absurdissima sit ea, quam sequuntur ipsi.

Omnes igitur vno
ore putant \textgreek{[Greek]}, pro \textgreek{[Greek]} dictum esse.

Id ad verbum
Hebraice esset \texthebrew{[Hebrew]}:
 aliter \textgreek{[Greek]}, Latine Præposterum.

quo nihil præposterius dici potuit.

Nam quid est præposterum Sabbatum?

Non pudet iocularis interpretationis?

Sed ita est.

Alius fortasse assensum extorsisset.

Sed quia a nobis, ideo
minus acceptum.

Theophylactus post Epiphanium, \& alios veteres,
interpretatur \textgreek{[Greek]}.

Itaque verum est, quod diximus, omnes tam veteres, quam
recentiores \textgreek{[Greek]} interpretari
 \textgreek{[Greek]}, id est \textgreek{[Greek]},
præposterum.

Vt illud probet, idem Theophylactus
subiicit: \textgreek{[Greek]}.

%\end{parnumbers}
\clearpage
p. XXVI [pdf 53]
%\begin{parnumbers}
quod falsum est,
propter translationes, quas imperiti negant sæculo Christi vsurpatas,
cum tamen longe ante Christum in vsu fuisse demonstrauerimus,
vt locus non sit pertinaciæ.

[Prolegomena continues up to page LII]

\end{parnumbers}


%% ToC generation gives obscure errors; solution: delete the .aux files
%% and re-compile (twice)
\tableofcontents{}

\mainmatter
% !TEX TS-program = xelatex
% !TEX encoding = UTF-8 Unicode
% this template is specifically designed to be typeset with XeLaTeX;
% it will not work with other engines, such as pdfLaTeX

%%% Count out columns for fixed-width source font
% 000000011111111112222222222333333333344444444445555555555666666666677777777778
% 345678901234567890123456789012345678901234567890123456789012345678901234567890

\setheaders{\shorttitle{} Liber I}{\shortauthor{}}
\chapter{De Anno Aequabili Minore}

% 1
% {PDF page nr}{source page nr}{line nr}
\plnr{84}{1}{1}Si verum est, quod sciscit Stoicorum
schola, Tempus esse normam rerum, et
custodiam, quia veritatis index atque examen
est, et rerum gestarum memoriam, ac
diuturnitatem posteritati tuetur: ii non vulgari
laude digni sunt, qui temporum rationes
conscribere, atque fugitivam antiquitatem
retrahere conantur.
\lnr{8}Qua in re cum
tam priscis scriptoribus, quam aequalibus
temporum nostrum opera egregie navata sit, dolendum tamen, aut
ferius, quam oportebat, antiquos sese ad id studium contulisse, aut pauciora
ea de re monumenta, quam ab ipsis auctoribus relicta sunt, ad
nos pervenisse.
\lnr{13}Nam ut omnia extent veterum Graecorum scripta, ea
tamen paucorum temporum intervallum complectebantur.
\lnr{14}Graecis
enim ante initia Olymiadum suarum nihil plane exploratum est: et,
quod dolendum est, de illorum scriptis, quae ad Chronologiam spectabant,
nihil nobis praeter desiderium relictum est.
\lnr{17}Nam quae Eusebii exstant,
quamuis ex Graecorum monumentis hausta sunt, et multa egregia
% è -> ex
ac cognitu digna nobis conservarunt: tamen dissimulandum non est,
multa in illis reperiri, quae castigatioribus iudiciis non satisfaciant.
\lnr{21}Quod si Thalli, Castoris, Phlegontis,
 Eratosthenis canones exstarent,
perparua, aut nulla potius ratio haberetur librorum quorundam, qui
hodie in penuria meliorum nobis in pretio sunt.
\lnr{23}Apud Romanos vero,
ea scriptio infeliciter cessit, quod eam cognitionem ferius amplexi sint.
\lnr{25}Nam ante Consulatum Bruti nihil certi apud illos: omnia fabulosa: et,
si rem propius spectemus, ne ipsius quidem Bruti Consulatum, ac tempus
Regifugii satis exploratum habent.
\lnr{27}Quamuis, ut prodidit Censorinus,
Varro collatis diversarum civitatum temporibus, et intervalla retexens,
verum in lucem protulerit, et viam reperit, qua certus
annorum Urbis conditae numerus iniri posset.

% 2
% {PDF page nr}{source page nr}{line nr}
\plnr{85}{2}{2}Sed, ut suo loco disputabitur,
non magis constabat Varroni de initiis Urbis, quam Graecis de
anno excidii Troiae.
\lnr{4}Nam ea demum est vera demonstratio, quae cogit,
non quae persuadet.
\lnr{5}Soli sacri libri supersunt, ex quorum fontibus
certa temporum ratio hauriri possit.
\lnr{6}Sed omnis temporum cognitio
inutilis est, nisi certa epocha in illis deprehendatur, ad quam omnium
temporum contextus, tam antecedentium, quam consequentium referri
possit.
\lnr{9}Nam, ut praeclare dixit vetus inter Christianos scriptor
Tatianus, apud quos temporum notatio non cohaeret, apud illos neque
veritatis et fidei historicae ratio ulla constare potest.
\lnr{11}Quod si aliquis
sacrae historiae peritissimus, hoc est, qui intervalla rerum gestarum
nobilissima certissimis ratiociniis ex Mose, et
 reliquis sacris Bibliis explorata
habeat, nihil tamen ex illis ad certam epocham historiae Graecae,
aut Romanae referre possit: quodnam adiumentum is ex eiusmodi
diligentia adferre potest aut sibi, aut studiosis rerum antiquarum?
\lnr{17}Nam omnis cognitionis finis ad usum aliquem spectat, quem si ex medio
literarum sustuleris, ingratus est omnis labor et opera, quaecunque
in omne studium impenditur.
\lnr{19}Eiusmodi est Iudaeorum scientia, qui
in ratiociniis quidem sacrorum temporum colligendis tantum studio
et diligentia consecuti sunt, ut proxime ad veritate abesse dici possint: sed
% à -> ad
dum nullam aut saltem depravatam rerum extrarum cognitionem
tenent, multum errant, quod sine externa historia sacram tractare
aggrediuntur.
\lnr{24}Venio ad nostros, recentiores dico, qui hodie summo
cum fructu, sacrae, Graecae, et Romanae historiae tempora digesserunt.
\lnr{26}Ii heroica virtute chronologiam negligentia et contemtu maiorum
intermortuam ac sepultam, ex tenebris et oblivionis silentio quotidie
% è -> ex
eruere conantur.
\lnr{28}Certe meum semper iudicium fuit, eam rem maiore
cum laude ab illis restitutam, quam ab antiquis proditam fuisse.
\lnr{29}Nam
non solum pleraque in ratione temporum pristinae integritati reddiderunt,
sed et longe meliora effecerunt.
\lnr{31}In multis tamen iudicium, in quibusdam
etiam diligentiam requiro.
\lnr{32}Neque enim dum verum adepti sunt.
\lnr{33}Argumento suerint omnium, quotquot de his rebus tractarunt,
 dissensiones:
ut inter tot millia Chronologorum vix inter duos de eadem re
conveniat.
\lnr{35}Quanta adhuc contentione de Septimanis Danielis, de initio,
medio, et fine earum velitantur?
\lnr{36}Tamen nihil plane eorum, quae volunt,
assecuti sunt.
\lnr{37}Ab eorum lectione incertior atque indoctior sum,
quam dudum.
\lnr{38}Quis unquam eorum veram epocham Exodi Habraeorum;
quis, quod pudendum est, verum annum natalis Dominici odoratus
est?
\lnr{40}Ecce trita, obvia, vulgaria, ut nobis videtur, ignoramus, et remotiorum
ac reconditiorum indicium promittimus!
\lnr{41}Quis eorum Danielis
Hebdomadas interpretandas suscepit, qui inscitiae suae latebram
non quaesiverit, et reges Persidis, qui nunquam in rerum natura fuerunt,
non commentus sit?

% 3
% {PDF page nr}{source page nr}{line nr}
\plnr{86}{3}{3}Quod si Danielem accuratissime legissent,
eis ad negotium explicandum non aliis regibus Persidis opus fuisset,
quam iis, quos Herodotus, Diodorus, et omnis Graecorum antiquitas
novit.
\lnr{6}Sed quo non progressa est \textgreek{ἀμηχανία[?]}?
\lnr{6}Berosos, Metasthenes, et
nescio quos Catones, ac Philones consulunt, qui ante hos centum annos
ex officina nescio cuius indocti et impudentis prodierunt.
\lnr{8}Et sese
Criticos in temporum notatione profitentur, quibus tam facili genere,
tam pueriliter unus homo otiosus in tanta luce literarum quotidie imoponit.
\lnr{11}Cuius hominis inscitiam si nihil aliud, certe illud arguere
 possit, quod
Metasthenem pro Megasthene posuit.
\lnr{12}Si Iosephum Graece, aut Strabonem,
aut Athenaeum legisset, is Megasthenem vocari deprehendisset,
quem Metasthenem vocat.
\lnr{14}Si Graece scisset, nunquam \textgreek{μεταοθένη[?]} in illa
lingua reperiri, neque hanc compositionem in eadem probari intellexisset.
\lnr{16}Ut igitur ii resipiscant, qui et novos reges in Perside crearunt,
et Assueros Priscos, Assueros Longimanos, Assueros Pios, duos Cyros,
et nescio quae alia somnia Annii Viterbiensis in medium producunt,
primum uno verbo indicabo fontem erroris eorum: deinde qui medicina
huic morbo fieri possit, docebo.
\lnr{20}Quod igitur in veri investigatione
eos ratio fugerit, duas summas causas reperio: unam, quod veterum
tempora civilia, annorum, mensium formas, status, ac genera ignorarunt:
alteram, quod characterem, et notationem ei anno, quem sibi
proposuerant, non adhibuerunt.
\lnr{24}Ex utraque quidem causa temporum
confusio manavit, sed diverso genere.
\lnr{25}Ex priore causa ignoratus est
annus, mensis et dies multarum nobilium epocharum.
\lnr{26}Huius enim
rei cognitio pertinet ad tempus civile nationum.
\lnr{27}Ex altera causa Palilia
urbis Romae nunc tertio anno Olympiadis, nunc quarto attribuuntur.
\lnr{29}Item Consulatus Bruti nunc in hunc, nunc in illum annum
Olympiadis confertur.
\lnr{30}
Ut igitur novam rationem emendationis temporum
ineamus, duo illa praecipue nobis discutienda sunt: sed prius
de omnium nationum temporibus civilibus: quam assequi perdifficile
est, nisi prius tempore in sua principia, hoc est ab annis, periodis,
mensibus in ultimum terminum, dies, horas ac scrupula resoluto.
\lnr{35}Nam qui ante nos hanc provinciam aggressi sunt, si modo hanc nostram,
non aliam aggressi sunt, ii satis de tempore, et eius natura
disputarunt.
\lnr{37}Sed hanc disputationem melius interpres
 \textgreek{φυσικῆς ἀκροάσεως[?]}
sibi vindicasset.
\lnr{38}Neque vero nos id agimus, ut difiniamus
tempus esse hoc secundum Peripateticos, aut illud secundum Stoicos,
aut Academicos.
\lnr{40}Qui istis definitionibus diu immorati sunt, et hac
sola scientia Chronologiae scribendae modum terminarunt, illi fatis
verborum quiedem, sed rerum nihil definiverunt.

% 4
% {PDF page nr}{source page nr}{line nr}
\plnr{87}{4}{1}Nequid tamen
\textgreek{ἀμεθοδέυτως[?]} transigatur, decrevi singularum, vel
 minimarum temporis
partium prius conspectum aliquem dare, quam ad descriptionem
\textgreek{ἱστορικὴν[?]} temporum civilium, et eorum methodum aggrediar.
\lnr{4}Incipiam igitur ab ultimo termino, a die scilicet, et eius partibus,
hoc est hora, et scrupulis.
\lnr{6}Ab hora igitur, si libet, principium esto.

\section{De Horis et partibus diei reliquis}

\lnr{7}Veteribus statim ab initio has diei partes, quas \textsc{Horas}
vocamus, in usu non fuisse, argumento fuerint priscae locutiones,
quibus dies non in partes secatur, sed actionibus quotidianis
distiguitur: ut cum \textgreek{βουλυτὸν[?]} vesperam vocabant, 
nimirum, ut poëta
% Source: poëta; Rare occurence of a diaeresis. Not sure if it should be
% left there or removed.
inquit, \textit{Demeret emeritis cum iuga Phoebus equis}.
\lnr{11}Item quod tempus
antemeridianum disignantes dicebant \textgreek{πληθυόυσης[?]}
 vel \textgreek{πληθόυσης ἀγορᾶς[Greek]},
convenientibus scilicet eo tempore in Comitium viris: ut Hesiodus dicit,
\textgreek{εὖτ᾽ ἂν ἀληθείην λαοὶ κρίνοντες ἄγωσιν[?]}.
\lnr{14}Quod tamen longe aliter interpretes
Graeci illius poëtae exponunt.
% Source: poëtae
\lnr{15}Aiunt enim Hesiodum intellexisse
de tricesima mensis Lunaris: et sensum loci Hesiodei esse perinde
ac si dixisset, Quando homines veram \textgreek{τριακάδα[?]} Lunarem agunt, et
non secundum usum politicum, sed secundum motum Lunae.
\lnr{18}Quod
tamen nobis valde coactum videtur: et mentem Hesiodi hanc fuisse dicimus:
\textgreek{τριακάδα[?]} esse valde idoneam rebus gerendis ea hora,
 qua homines
ad ius in forum conveniunt.
\lnr{21}
Homerus Odyss. \textgreek{\gnum{μ}}. % 40
% Expand "Odyss." in the propper way, with the correct declension.
\begin{verse}
\textgreek{--- Η῏μος δ᾽ ἐπὶ δόρπον ἀνὴρ ἀγορῆθεν ἀνέστη}\\
\textgreek{Κρίνων νείκεα πολλὰ δικαζομένων αἰζηῶν}.
\end{verse}
% Homer: Odyssey 12. 440-441
% Spelling and diacriticals verified.
% At the time of day when a man who is judging many disputes [νείκεα] of men
% in their prime [αἰζηῶν] seeking justice in lawsuits [δικαζομένων] gets up
% from the court [ἀγορῆθεν] [to go] to dinner

\lnr{24}Quae sane interpretatio melior vulgari.
\lnr{24}Sic etiam paulo post dicit,
\textgreek{ἤματος ἐκ πλείου[?]}, loquens de undecima: cuius partem designat,
 cum dicit
\textgreek{ἤματος ἐκ πλείου[?]}.
% Hesiod: Works and Days, 778
\lnr{26}Quod nos interpretamur iam adulto die.
\lnr{26}Sic Homerus
meridiem designat, \textgreek{ὅταν δρυτόμος ἀνὴρ δόρπον ὁπλίσσατο[?]}.
\lnr{27}Porro neque
hoc verbum \textgreek{ὥρα[?]} id, quod nunc, valebat.
\lnr{28}Sed tempus actuum quotidianorum
illo notabatur: ut cum dicebant \textgreek{ὥρα δόρπου, ὥρα δείπνου[?]}.
\lnr{29}Latinis
vero Tempestas dicebatur.
\lnr{30}In Legibus Decemvirum Atticis fuit:
\textsc{Sol occasus suprema tempestas esto}.
\lnr{31}Neque recte
quidam hinc expungunt \textsc{tempestas}.
\lnr{32}Quod \textsc{suprema} absolute
diceretur, ut apud Plautum.
\lnr{33}Nam plane in legibus Solonis, unde illud
caput traductum, scriptum fuit,
 \textgreek{ὁ ἥλιος ἐπὶ τῶν ὀρῶν ἐσχάτη ὥρα ἔστω[?]}.
\lnr{34}Stoicus
scriptor apud Stobaeum loquens de Socratis iudicio capitali: 
\textgreek{καὶ
τριῶν ἡμερῶν ἀυτῷ δοθέισῶν, τῇ πρώτῃ ἔπιεν,
 καὶ ὀυ προσέμεινεν τὴς τρίτης ἡμέρας τὴν
ΕΣΧΑΤΗΝ ΩΡΑΝ παρατηρεῖν, ἐι ἐστὶν ΗΛΙΟΣ ΕΠΙ ΤΩΝ
ΟΡΩΝ, ἀλλ᾽ ἐυθαρσῷς τῇ πρώτῃ[?]}.
\lnr{38}Idem censeas de veteribus Hebraeis,
qui diei nullas alias partes, quam mane, meridiem, et vesperam norant.
et ita dies dividitur Psalmo \rnum{lv}, commate \rnum{xviii}.
% Psalm 55:17 (KJV) "Evening (εσπέρας), and morning (πρωϊ),
% and at noon (μεσημβρίας), will I pray, and cry aloud:
% and he shall hear my voice."

% 5
% {PDF page nr}{source page nr}{line nr}
\plnr{88}{5}{2}Sic Homero,
\textgreek{ἠὼς, ἢ δείλη, ἢμέσον ἦμαρ[?]}.
\lnr{3}Sed hic dies intelligitur Lux, exclusa nocte.
\lnr{4}Nam totum \textgreek{νυχθήμερον[?]} Hebraei in quatuor partes
 dividebant, quas vigilias
vocabant.
\lnr{5}Prima vigilia erat ab vespere: secunda ab media nocte:
tertia ab mane: quarta ab meridie.
% à -> ab (4x)
\lnr{6}Alioqui nomen hoc \texthebrew{[Hebrew]} quo hodie
horam designant, ne notum quidem illis erat: atque apud Danielem
aliud significat.
\lnr{8}Posterorum inventum est Horologium, et \textgreek{ἡλιοτρόπια[?]},
%[Greek: heliotropia; probably: sundails]
quibus dies per lineas, et intervalla umbrarum distinguebatur.
\lnr{10}Unde prodiit locutio \textgreek{ἑνδεκάπους σκιὰ[?]}, pro hora coenae.
\lnr{10}Vel \textgreek{ἑνδεκάπουν στοιχεῖον[?]}:
quia notis literarum singularium horae distinguebantur.
\lnr{11}Testatur et Epigrammatium de Horologio:
\begin{verse}
\textgreek{ἓξ ὧραι μόχθοις ἱκανώταται. αἱ δὲ μετ᾽ αὐτὰς}\\
\textgreek{γράμμασι δεικνύμεναι ΖΗΘΙ λέγουσι βροτοῖς}
\end{verse}
% Greek spelling and diacriticals verified.
% Anonymous poem from "Greek Anthology" 10.43, tr. W.R. Paton
% Six hours are most suitable for labour, and the four that follow,
% when set forth in letters, say to men "Live".
\lnr{14}Nam ante
\textgreek{Ζ, Η, Θ, Ι,} erat \textgreek{Α, Β, Γ, Δ, Ε, ϛ.}
\lnr{15}Arabibus, Persis, et reliquis Orientis
gentibus non horologiis, sed
naturalibus matutini, meridiani,
et vespertini temporis
intervallis diem notare,
etiam hodie consuetudo manet.
\lnr{21}Astronomis propria
est divisio diei in sexagesimas
primas, secundas, tertias,
et sic deinceps.
\lnr{24}Artificibus
computi annalis in
horas, puncta, ostena, minuta,
partes.
\lnr{27}Hora est punctorum
4. minutorum 40.
partium 480. momentorum
1760.
\lnr{30}Ostenta autem sunt arbitraria,
quibuslibet aliarum
divisionum in illa resolutis.
\lnr{33}Orientalibus vero Computatoribus
compendiosa horarum
resolutio est.
\lnr{35}
Non
enim in sexagesimas assem
dividunt, sed in 1080 partes
ita ut 18 particulae uni minuto
horario respondeant.
\lnr{40}Hac divisione hodie Iudaei,
Samaritani, Arabes, Persae,
et aliae Orientis nationes utuntur.

% 6
% {PDF page nr}{source page nr}{line nr}
\plnr{89}{6}{1}Quorum in sexagesimas, et
contra, sexagesimarum in haec convertendarum, Tabellas duas posuimus
 (p. \pageref{tab:p006}).
\begin{table}
  %% Tabula convertendi ostenta in sexagesimas et vice versa
%% Liber Primus, p.6, PDF 89
%%
%%% Count out columns for fixed-width source font
% 000000011111111112222222222333333333344444444445555555555666666666677777777778
% 345678901234567890123456789012345678901234567890123456789012345678901234567890
%
%% Select a general font size (uncomment one from the list)
%\tiny
%\scriptsize
%\footnotesize
%\small
\normalsize
%% Center the whole table left-right
\centering
%% Modify separation between columns
\setlength{\tabcolsep}{3pt}
%% Modify distance between rows
\renewcommand{\arraystretch}{1.1}
%% Define column header format
\newcommand{\colhead}[1]{\multicolumn{1}{c}{\footnotesize #1}}
%
\begin{tabular}{@{} r r r r  c r r r r }
\toprule
  \multicolumn{9}{c}{\Large\scshape Tabula convertendi ostenta} \\
  \multicolumn{9}{c}{\Large\scshape in sexagesimas et v.v.} \\
\toprule
  \multicolumn{4}{c}{\normalsize\scshape Ostenta in sexagesimas} & &
  \multicolumn{4}{c}{\normalsize\scshape Sexagesimas in ostenta}
\\
\cmidrule[\heavyrulewidth]{1-4} \cmidrule[\heavyrulewidth]{6-9}
\colhead{Ostenta} &
\colhead{Sexag.} &
\colhead{Sexag.} &
\colhead{Sexag.} &
\hspace{8mm} &
\colhead{Sexag.} &
\colhead{Sexag.} &
\colhead{Ostenta} &
\colhead{Ostenta}
\\
\cmidrule{1-4} \cmidrule{6-9}
   1 &  0' &  3'' & 20''' & &  0' &  1'' &    0' & 324'' \\
   2 &  0' &  6'' & 40''' & &  0' &  2'' &    0' & 648'' \\
   3 &  0' & 10'' &  0''' & &  0' &  3'' &    0' & 972'' \\
   4 &  0' & 13'' & 20''' & &  0' &  4'' &    1' & 210'' \\
   5 &  0' & 16'' & 40''' & &  0' &  5'' &    1' & 540'' \\
   6 &  0' & 20'' &  0''' & &  0' &  6'' &    1' & 864'' \\
   7 &  0' & 23'' & 20''' & &  0' &  7'' &    2' & 108'' \\
   8 &  0' & 26'' & 40''' & &  0' &  8'' &    2' & 432'' \\
   9 &  0' & 30'' &  0''' & &  0' &  9'' &    2' & 756'' \\
  10 &  0' & 33'' & 20''' & &  0' & 10'' &    3' &   0'' \\
  20 &  1' &  6'' & 40''' & &  0' & 20'' &    6' &   0'' \\
  30 &  1' & 40'' &  0''' & &  0' & 30'' &    9' &   0'' \\
  40 &  2' & 13'' & 20''' & &  0' & 40'' &   12' &   0'' \\
  50 &  2' & 46'' & 40''' & &  0' & 50'' &   15' &   0'' \\
  60 &  3' & 20'' &  0''' & &  1' & 60'' &   18' &   0'' \\
  70 &  3' & 53'' & 20''' & &  2' &  0'' &   36' &   0'' \\
  80 &  4' & 26'' & 40''' & &  3' &  0'' &   54' &   0'' \\
  90 &  5' &  0'' &  0''' & &  4' &  0'' &   72' &   0'' \\
 100 &  5' & 33'' & 20''' & &  5' &  0'' &   90' &   0'' \\
 200 & 11' &  6'' & 40''' & &  6' &  0'' &  108' &   0'' \\
 300 & 16' & 40'' &  0''' & &  7' &  0'' &  126' &   0'' \\
 400 & 22' & 13'' & 20''' & &  8' &  0'' &  144' &   0'' \\
 500 & 27' & 46'' & 40''' & &  9' &  0'' &  162' &   0'' \\
 600 & 33' & 20'' &  0''' & & 10' &  0'' &  180' &   0'' \\
 700 & 38' & 53'' & 20''' & & 20' &  0'' &  360' &   0'' \\
 800 & 44' & 26'' & 40''' & & 30' &  0'' &  540' &   0'' \\
 900 & 50' &  0'' &  0''' & & 40' &  0'' &  720' &   0'' \\
1000 & 55' & 33'' & 20''' & & 50' &  0'' &  900' &   0'' \\
\multicolumn{4}{c}{}      & & 60' &  0'' & 1080' &   0'' \\
\bottomrule
\end{tabular}
%
\caption{Convertendi ostenta in sexagesimas et vice versa}
\label{tab:p006}
%

\end{table}
% Suggested improvements for tables:
% - Split in two tables
% - Bigger letters/numbers
% - Remove Smallcaps latin titles from top. Put as caption to each table

\section{De Diebus}

\lnr{4}\textgreek{Το νυχθήμερον},
%[Greek: the day and night, i.e. a full 24 hour cycle]
quod est spatium viginti quatuor horarum, Daniel
eleganter vocat \texthebrew{[Hebrew]} quasi dicas
 \textgreek{ὀψιπρώϊον[?]}, initio diei civilis
sumto Iudiace ab eo tempore, quod proxime Solem occasum
sequitur.
\lnr{7}Nam illud intervallum, quatenus vigintiquatuor horarum est,
naturale est: quatenus aliud atque aliud initium habet, dicitur civile,
Atticis et Iudaeis ab occasu Solis: Aegyptiis et Romanis ab media nocte:
Chaldaeis Genethliacis ab ortu Solis: Umbris ab meridie initium
sumentibus.
% à -> ab (2x)
\lnr{11}Dierum notationes duplices: aut secundum numerum, et
ordinem: ut prima, secunda, tertia mensis.
\lnr{12}Aut secudum \textgreek{ἐπωνυμίαν[?]},
qua dies alicui rei cognomines.
\lnr{13}Ut dies mensis Persici sunt cognomines
regum priscorum: et dies mensis Mexicanorum, animalium, aut aliarum
rerum: et \textgreek{ἐπαγόμεναι[?]} Aegyptiorum nominibus singulorum Deorum
vocatae.
\lnr{16}Et dies festi, ut quinquatrus, \textgreek{κρόνια},
%[Greek: of Kronos, i.e. Saturn]
\textgreek{ϑαργήλια[Greek]}, Quirinalia.
\lnr{17}Et ab eventu, dies Alliensis, Regifugium.
% - Alliensis: Possibly:
% (Dies) (June 16th, BCE 390), when the Romans were cut to pieces by the
% Gauls near the banks of the river Allia; and ever after held to be a dies
% nefastus , or unlucky day. 
% Or: July 18th of 390 BCE (Dies quartus decimus ante Kalendas Augustas) 
% - Regifugium: Roman feast day, celebrating the eviction of the last king,
% Tarquinius Superbus
\lnr{17}Ab stellis, dies Septimanae.
% à -> Ab
\lnr{18}Ecclesia Romana vocat ferias.
\lnr{18}Quia veteris anni Ecclesiastici initium
ab Pascha.
% à -> ab
\lnr{19}Et Pascha dicebatur annus novus, ut etiam hodie ab Ecclesia
Antiochena: ab Constantinopolitana autem \textgreek{διακαινίσιμος ἑβδομὰς[?]},
% à -> ab
ab eadem mente.
\lnr{21}Illius autem Hebdomadis dies omnes septem erant
feriati, ut testis est Hieronymus, et alii veteres.
\lnr{22}Hinc obtinuit, ut reliquarum
hebdomadum dies etiam Feriae vocarentur, praecipuo quodam
principis septimanae Paschalis auspicio et omine.
\lnr{24}Solon autem
primus omnium \textgreek{τὲν τριακάδα ἔνην καὶ νέαν[?]} vocavit,
 cum antea \textgreek{ἔνη[?]} esset
prima mensis.
\lnr{26}Hesiodus: \textgreek{Πρῶτον ἔνη τετράς τε καὶ ἑβδόμη ἰερὸν ἦμαρ[?]}.
\lnr{27}Diei divisio summa ab actibus quotidianis, in fastos, nefastos, atros,
religiosos, intercisos, iustos: ut Graecis
 \textgreek{εἰς ἐνεργοὺς, καὶ ἀέργους[?]}, vel, ut alii,
\textgreek{ἀνεσίμους ἡμέρας καὶ ἀποφράδας, καὶ ἑορτασίμους[?]}.
\lnr{29}Aut ab aequatione annui
temporis, Solaris, et Lunaris, in
 \textgreek{προσθετὰς ἡμέρας, ἐπακτὰς, ἐξαιρεσίμους,
ὑπερβάτους, ἐμβολίμους, ἐπαγομένας, περιττάς[?]}.
\lnr{31}\textgreek{Προσθεταὶ ἧμέραι[?]} Computatoribus
Graecis dicuntur, quae Latinis Regulares, quae cum Concurrentibus,
id est Epactis Solaribus compositae dant characterem Kalendarum,
aut alius diei mensis.
\lnr{34}\textgreek{Ε᾽πακταὶ[Greek]} sunt duplicis generis, Solares, et
Lunares.
\lnr{35}Solares fiunt abiectis septenariis ex cyclo Solari, addito praeterea
% fiunt <-> siunt?
die bisextili.
\lnr{36}
Lunares producuntur, excessu Solis, qui est \rnum{xi} dierum,
in numerum aureum ducto, abiectis tricenariis.
\lnr{37}Praeterea utrarumque
Epactarum sua methodus: Solarium ad characterem dierum:
Lunarium ad aetatem Lunae, ut Computatores Latini loquuntur, ut
Graeci autem, \textgreek{εἰς ποστιαίαν σελήνης.[?]}.

% 7
% {PDF page nr}{source page nr}{line nr}
\plnr{90}{7}{1}\textgreek{Εξαιρέσιμοι[?]} sunt, quae eximuntur de
mense, duplici ex causa: aut ut rationes Solis cum Lunaribus congruant,
ut in anno veteri Graecorum: et in enneadecaeteride Paschali
Saltus Lunae Latinis dictus, Graecis \textgreek{ὑποτομὴ σελήνης[?]}.
\lnr{4}Aut ut solennia
festa cum feria Septimanae, ut in anno Iudaico.
\lnr{5}\textgreek{Υπέρθετοι[?]}, vel \textgreek{ὑπέρβατοι[?]},
sunt, quae ex caussa religionis, transferuntur, et dissimulantur per speciem
comperendinationis, ut in anno Iudaico, et olim in prisco Romano.
\lnr{8}In Iudaico enim \textgreek{ὑπερθέσεις[?]} et comperendinationes
 institutae, ne
feria secunda, quarta, sexta in caput anni incurrat.
\lnr{9}In Romano prisco
comperendinabantur Nundinae, ut ab religiosis diebus summoveientur,
% à -> ab
auctore Macrobio.
\lnr{11}\textgreek{Εμβόλίμοι [Greek]} sunt, ut notio verbi declarat, insititii
dies: et erant naturales, aut civiles.
\lnr{12}Naturales, qui ex scrupulis, et
horis appendicibus colliguntur, ut quatro quoque anno exeunte unus
dies ex quadrantibus anni Iuliani, quod \textsc{Bisextum} vocatur: item
in periodo Arabica undecies unus dies intercalatur in fine Dulhagiathi,
% fine or sine? Fine -> end of the month. Sounds good.
qui est ultimus mensis anni Hagareni Mohamedici.
\lnr{16}Civiles sunt,
qui praeter naturalem anni rationem et modum inseruntur, ut unus
dies in fine Marcheschuvan Iudaici, anno qui dicitur superfluus, aut
% fine or sine? Again: Fine -> end of the month.
abundans.
\lnr{19}\textgreek{Επαγόμεναι[?]},
 quae explendis spatiis anni adiiciuntur potius,
quam inseruntur, ut quinque, quae anno aequabili extra ordinem mensium
adiectae Aegyptiis dicuntur \textsc{nisi}, Persis, et Armeniis
 \textsc{musteraka}: 
\lnr{22}item duae, quae extra modum anni Attici in calce Posideonis
appensae, \textgreek{ἄναρχοι ῾ξμέραι[?]} dicebantur,
 aut \textgreek{ὑπερβάλλουσαι[?]}, aut \textgreek{ἀρχαιρέσιαι[?]}.
\lnr{24}At \textgreek{περιτταὶ ἡμέραι[?]} locum habent in anno mobili.
\lnr{24}Est autem intervallum
inter epocham et caput anni, utroque termino excluso.
\lnr{25}Hoc
constat semper in annis, quorum caput nunquam epocham antevertebat.
\lnr{27}
Ut in anno Attico caput Hecatombaeonis nunquam ante Solstitii
veterem epocham statuebatur.
\lnr{28}Itaque quod inter Solstitium, et
propositum Hecatombaeonem interiacet spatii, utroque termino excluso,
dicebantur \textgreek{περιτταὶ ἡμέραι[?]}.
\lnr{30}Idem observabatur in annis magnis
Metonis et Calippi.
\lnr{31}Rursus Romanorum sacri dies Kalendae, Nonae,
Eidus: Graecorum autem \textgreek{ἔνη, τετρὰς, ἑβδόμη [?]}.
\lnr{32}Quod ex versu Hesiodi ab
% à -> ab
nobis adducto constat.
\lnr{33}Sunt praeterea nomina imposita diebus mensium
singulis, ut suo loco referetur.
\lnr{34}Sunt et secundum hebdomadas
ut infra subiecimus.
% Insert table
\begin{table}[hbtp]
  %% Dies Hebdomadis
%% Liber Primus, p.8, PDF 91
%%
%%% Count out columns for fixed-width source font
% 000000011111111112222222222333333333344444444445555555555666666666677777777778
% 345678901234567890123456789012345678901234567890123456789012345678901234567890
%
%% Select a general font size (uncomment one from the list)
%\tiny
%\scriptsize
%\footnotesize
%\small
\normalsize
%% Center the whole table left-right
\centering
%% Modify separation between columns
%\setlength{\tabcolsep}{0.5em}
%% Modify distance between rows
%\renewcommand{\arraystretch}{0.85}
%%
\begin{tabular*}%
{\textwidth}{%
@{\extracolsep{\fill} } r r r @{\hspace{4pt}} || r @{\hspace{4pt}} | @{} l 
}
\multicolumn{3}{c}{\textsc{DIES HEBDOMADIS}} &
\multicolumn{2}{c}{\textsc{ALITER PERSICE.}}
\\
\multicolumn{3}{c}{\textsc{persicae.}} & \multicolumn{2}{c}{}
\\
\hline
\texthebrew{שנב} % some random characters as filler text
& \textarabic{شزذيثب} % some random characters as filler text
& \textarabic{ل}
& 1
& \textit{Ruz iache}
\\
\texthebrew{[Hebrew]}
& \textarabic{[Persian]}
& \textarabic{ب}
& 2
& \textit{Ruz duiemi}
\\
\texthebrew{[Hebrew]}
& \textarabic{[Persian]}
& \textarabic{ج}
& 3
& \textit{Ruz siumi}
\\
\texthebrew{[Hebrew]}
& \textarabic{[Persian]}
& \textarabic{ﺩ}
& 4
& \textit{Ruz tzeharmi}
\\
\texthebrew{[Hebrew]}
& \textarabic{[Persian]}
& \textarabic{م}
& 5
& \textit{Ruz pengemin}
\\
\texthebrew{[Hebrew]}
& \textarabic{[Persian]}
& \textarabic{و}
& 6
& \textit{Ruz schesmin}
\\
\texthebrew{[Hebrew]}
& \textarabic{[Persian]}
& \textarabic{ز}
& 7
& \textit{Ruz haphthemi}
\end{tabular*}

\vspace{\baselineskip}

\begin{tabular*}
{\textwidth}{%
    @{\extracolsep{\fill} } r r @{\hspace{4pt}} || r @{\hspace{4pt}} r c
}
\multicolumn{2}{c}{\textsc{TURCIAE HEBDOMADIS}} & \multicolumn{3}{c}{\textsc{SECUNDUM PLANETAS.}}
\\
\multicolumn{2}{c}{\textsc{dies.}} & \multicolumn{3}{c}{}
\\
\texthebrew{[Hebrew]}
& \textarabic{[Arabic]}
& \texthebrew{[Hebrew]}
& \textarabic{[Arabic]}
& \astro{♄}
\\
\texthebrew{[Hebrew]}
& \textarabic{[Arabic]}
& \texthebrew{[Hebrew]}
& \textarabic{[Arabic]}
& \astro{♃}
\\
\texthebrew{[Hebrew]}
& \textarabic{[Arabic]}
& \texthebrew{[Hebrew]}
& \textarabic{[Arabic]}
& \astro{♂}
\\
\texthebrew{[Hebrew]}
& \textarabic{[Arabic]}
& \texthebrew{[Hebrew]}
& \textarabic{[Arabic]}
& \astro{☉}
\\
\texthebrew{[Hebrew]}
& \textarabic{[Arabic]}
& \texthebrew{[Hebrew]}
& \textarabic{[Arabic]}
& \astro{♀}
\\
\texthebrew{[Hebrew]}
& \textarabic{[Arabic]}
& \texthebrew{[Hebrew]}
& \textarabic{[Arabic]}
& \astro{☿}
\\
\texthebrew{[Hebrew]}
& \textarabic{[Arabic]}
& \texthebrew{[Hebrew]}
& \textarabic{[Arabic]}
& \astro{☾}
\end{tabular*}
%
\caption{Dies Hebdomadis}
\label{tab:p008}
%

\end{table}

% 8
% {PDF page nr}{source page nr}{line nr}
\plnr{91}{8}{1}Cur autem dies cognomines Planetarum non sequuntur ordinem et
situm siderum, quorum cognomines sunt, ut scilicet post diem Saturni
non sequatur dies Iovis, sed dies Solis, haec caussa est.
% Diagram: circle with heptagram, with planets at the points:
% Moon ☾, mercury ☿, venus ♀, sun ☉,
% mars ♂, jupiter ♃, saturn ♄
\begin{figure}[hbtp]
  \centering
  \def\svgwidth{9\baselineskip}
  {\astrofont\input{./img/008_planets.pdf_tex}}
  \caption{Septem Planetae}
  \label{fig:p008}
\end{figure}
\lnr{3}Septem Planetae
per circulum secumdum ordinem suum
dispositae, aequabili intervallo constituunt septem
Triangula isoscele ad peripheriam, quorum
bases sunt latera Heptagoni circulo inscripti,
ut habes in circulo proposito, ad cuius
peripheriam septem errantes sunt secundum
feriem suam sitae, constituentes triangula
isoscele \astro{♄♀♃}, \astro{♃☿♂}, \astro{♂☽☉},
 \astro{☉♄♀}, \astro{♀♃☿}, \astro{☿♂☽}, \astro{☽☉♄}.
\lnr{12}In quibus Triangulis dexter angulus ad basim
est prima stella Trianguli, secunda in angulo ad verticem, tertia angulus
sinister ad basim: ita ut omnis stella anguli dextri habeat oppositam
stellam anguli in vertice, stella autem anguli ab vertice stellae
% à -> ab (accent better visible in other copies)
anguli sinistri ad basim sit opposita.

% 9
% {PDF page nr}{source page nr}{line nr}
\plnr{92}{9}{2}Sequentur igitur sese omnes septem
Planetae non per seriem suam, sed per intervalla laterum, quae
verae sunt oppositiones.
\lnr{4}Sit igitur Triangulum \astro{☉☽♂} primum ordine.
\lnr{5}\astro{☉} in angulo basis dextro praeibit.
\lnr{5}Sequetur Luna ei opposita in vertice,
eam oppositus Mars in angulo sinistro basis.
\lnr{6}Qui quidem Mars cum in
Tiangulo \astro{☉☽♂}, sinistrum angulum basis occupet,
 in triangulo \astro{♂☿♃} occupabit
dextrum basis angulum, habens oppositum Mercurium,
Mercurius autem oppositum Iovem in angulo sinistro.
\lnr{9}Qui Iuppiter
faciet angulum dextrum in Triangulo \astro{♃♀♄}, habens oppositam in vertice
Venerem, ut ea opposita est Saturno in angulo sinistro.
\lnr{11}Sed angulus
ille rursus erit dexter in Triangulo \astro{♄☉☽}.
\lnr{12}Et sic erogati sunt septem
planetae in totidem dies, quas Ecclesia Romana vocat ferias.
\lnr{14}Haec est vera harum appelationum ratio.

\section{De Mensibus}

\lnr{15}Ex diebus fiunt \textgreek{συστήματα καὶ ὁμάδες[?]},
 quae notationes et epochas
temporum constituunt.
\lnr{16}Primum \textgreek{σύστεμα[?]} ex diebus dicitur Septimana,
res omnibus quidem Orientis populis ab ultima usque
antiquitate usitata,
 nobis autem Europaeis vix tandem post Christianismum
recepta.
\lnr{19}De ea iam dictum est.
\lnr{19}Tum Romanorum \textgreek{ὀγδοὰς[?]}: cui
successit hebdomas nostra.
\lnr{20}Nam nono quoque die Nundinae erant.
et spatium illud in Kalendario vetere Romano notatum est literis ab
\textsc{a} ad \textsc{h}, ut in nostro Kalendario Hebdomas
 notata est ab \textsc{a} ad \textsc{g}, inclusive,
ut loquuntur.
\lnr{23}Mexicanorum \textgreek{τριοκαιδεκὰσ[?]} sequitur.
% trio kai decas = 13
\lnr{23}Quod
enim spatium nobis septenis diebus, illis finitur ternis denis.
\lnr{24}Ita Iudaeorum
est \textgreek{ἑπταήμερον[?]} veterum Romanorum \textgreek{ὀκταήμερον[?]},
% hepta emeron = 7 days
% okta emeron = 8 days
 Mexicanorum
\textgreek{τριοκαιδεκαήμερον[?]}.
% trio kai deka emeron = 13 days
\lnr{26}Proximum ab hoc \textgreek{σύστημα[?]} dierum est Mensis:
qui et naturaliter, et civiliter sumitur.
\lnr{27}Naturalis mensis et ipse duplex.
\lnr{28}Aut enim Lunaris, aut Solaris.
\lnr{28}Rursus Lunaris triplicis generis:
aut quatenus Luna ab eodem puncto Zodiaci profecta, ad idem
revertitur: qui dicitur \textgreek{περίπατος[?]},
 item \textgreek{περίοδος σελήνης[?]}.
\lnr{30}
Quod intervallum
minus est, quam viginti octo dierum: maius quam viginti septem.
\lnr{32}Secundum genus est eiusdem sideris ab Sole profecti ad eundem
% à -> ab
reditus.
\lnr{33}Haec dicitur \textgreek{σύνοδος σελήνης[?]}.
\lnr{33}Tertii generis mensis est secundus
dies \textgreek{ἀπὸ τὴς συνόδου[?]},
 quae dicitur \textgreek{φάσις, φεγγάριον[?]},
 et \textgreek{ἀπόκρουσις σελήνης[?]}.
\lnr{35}Secundum et tertium genus in temporibus civilibus locum habent.
\lnr{36}Nam Athenienses \textgreek{ἀπὸ τὴς συνόδου[?]} neomenias suas putabant:
 hodie vero
Hagareni \textgreek{ἀπὸ τὴς φάσεως[?]}.
\lnr{37}Graecorum enim neomenias ab ipso iugo
Lunae putari solitas testis Vitruvius ex Aristarcho Samio, his verbis,
loquensde Luna:

% 10
% {PDF page nr}{source page nr}{line nr}
\plnr{93}{10}{1}\textit{Quot mensibus sub rotam Solis radiosque primo die
antequam praeterit, latens obscuratur.}
 \textit{Et, cum est sub Sole, nova vocatur.}
\lnr{2}\textit{Postero autem die, quo numeratur secunda,
 praeteriens ab Sole, visitationem
% à -> ab
facit tenuem extremae rotundationis.}
%[Vitruvius, De architectura libri decem, Liber IX, Capitulum II, Sect. 3:
% "Ita quot mensibus sub rotam solis radiosque uno die, antequam praeterit, 
% latens obscuratur. Cum est cum sole, nova vocatur. Postero autem die, quo 
% numeratur secunda, praeteriens ab sole visitationem facit tenuem extremae 
% rotundationis." 
% Transl.: "whence, on the first day of its [the moon's] monthly 
% course, hiding itself under the sun, it is invisible; and when thus in 
% conjunction with the sun, it is called the new moon. The following day, which 
% is called the second, removing a little from the sun, it receives a small 
% portion of light on its disc."]
\lnr{4}Ubi etiam dixit visitationem
extremae rotundationis, quam ille Samius sine ullo dubio
 \textgreek{φαίσιν μηνοειδῆ} vocabat.
% Greek: phase crescent
\lnr{6}Sed et Onomacritus, qui sub nomine Orphei
 \textgreek{τελετὰς}
% Greek: ceremony
scripsit, in opere, quod
 \textgreek{ἡμέρας}
% Greek: day, hours of daylight
 vocavit, mensem Lunarem ab iugo Lunae
% à -> ab
incipit.
\lnr{8}Cuius versus apposui:
\begin{verse}
\lnr{9}\begin{greek}
Πάντ᾽ ἐδάης Μουσαῖε ϑεοφραδἐς. εἰδέ σ᾽ ἀνώγει[?]\\
ϑυμὸς ἐπωνυμίας μήνης κατὰ μοῖραν ἀκοῦσαι,[?]\\
ῥεῖά τοι ἐξερέω, σύ δ᾽ ἐνὶ φρεσὶ βάλλεο σῆσιν,[?]\\
οἵην τάξιν ἔχοντα κυρεῖ. μάλα γὰρ χρέος ἐστὶν[?]\\
ἴδμεναι, ῶς αὕτη παρέχει κλέος ἄντυγι[?] μηνός.[?]\\
πρῶτα μὲν εἰ πρώτῳ ἐνὶ ἤματι φαίνεται, ἄρες,[?]\\
μήνη δ᾽ ἐστ᾽ ἄρην ἐπιτέλλεται, ἴσχεο δ᾽ ἔργων[?]\\
τὸν δὲ παρεξανύασα φύσιν δίκερων ἀναφαίνει.[?]\\
αὐτὰρ ἀπὶ τρίτον ἦμαρ ἀπόπροθεν ἠελίοιο[?]\\
πᾶσιν ἐπιχθονίοισι φυτοσπόρου αἰτίη ἀλκῆς.[?]\\
τετράδι δ᾽ αἰξομέυη πολυφεγγέα λαμπάδα τείνει.[?]
\end{greek}
\end{verse}
\lnr{20}Sed Neomenia Arabica, excedit modum
 \textgreek{φάσεως} ut plurimum.
% Greek: phase
\lnr{20}
Ita ut
civiles neomeniae mensium Lunarium sint non unius generis: Atticae
\textgreek{ἀπὸ τὴς συνόδου[?]}: Iudaicae saepe
 \textgreek{ἀπὸ τὴς ἀποκρούσεως[?]}.
\lnr{22}Arabicae semper \textgreek{ἀπὸ τοῦ
μηνοειδοῦς σχήματος[?]},
ab tertia, inquam, die.
% à-> ab
\lnr{23}Mensis Solis naturalis est,
qui naturalibus circuli coelestis segmentis definitur, qualis est transitus
Solis ab signo ad signum.
% à-> ab
\lnr{25}Hi, et Lunares, sunt vere coelestes menses.
\lnr{26}Mensis civilis Solis est, qui non naturali modo,
 sed aequaliter tributus
est.
\lnr{27}Ut in anno Aegyptiaco et Graeco omnes aequaliter sunt
 \textgreek{τριακονθήμεροι[?]}:
et in Lunari alternis pleni, et cavi.
\lnr{28}In anno Mexicano \textgreek{εἰκοσαήμεροι[?]},
cum ex \rnum{xviii} mensibus eorum annus constituatur.
% period removed after xviii
\lnr{29}Apud Albanos
Martius erat sex et triginta dierum, Maius viginti duum, Sextilis
duodeviginti, September sedecim.
\lnr{31}Tusculanorum Quintilis habuit
triginta sex, October triginta duos, Aricinorum October trigintanovem.
\lnr{33}At rationes Lunae non patiuntur, ut menses sint alternis
perpetuo pleni, et cavi.
\lnr{34}Sed hoc ad methodum civilis temporis institutum.
\lnr{35}Sunt et alii menses ex superfluis diebus collecti, qui Embolimi
dicuntur: iique aut naturales, aut civiles: ambo autem ad aequationem
Solis directi.
\lnr{37}Naturales embolimi sunt, qui ex Solis excessu collecti
ad spatia Lunae complenda adhibentur.
\lnr{38}Cuiusmodi est Iudaicus
Adar prior, et Samaritanus Adar alter.
\lnr{39}Isque mensis est semper tricenum dierum.
\lnr{40}Civilis embolimus, qui ex diebus Solis superfluis consurgens
fulciendo anno cavo adiictur.
\lnr{41}Eiusmodi erat Merkendonius
prisci anni Romani alternis binum et vicenum, item trinum et vicenum
dierum.

% 11
% {PDF page nr}{source page nr}{line nr}
\plnr{94}{11}{2}Eiusmodi et Posideon Atticus.
\lnr{2}Neque enim Posideon
naturalis esse potest, quamuis triginta dierum, cum neque Lunaris
esset, quod eius neomenia longe ab lunari discederet: neque Solaris,
% à -> ab
quod pars esset illius anni, qui ad Solis cursum descriptus non esset.
\lnr{6}Idem de Merkedonio dicas, qui neque ad Solarem annum, neque ad
Lunarem pertineret, neque modum eum haberet, qui iusto mensi
competit, cum esset tantum \rnum{xxii}, aut ad summum \rnum{xxiii} dierum.
\lnr{9}Mensis divisio Atticis in \textgreek{δεκάδασ[?]}.
\lnr{9}Prima \textgreek{δεκὰσ[?]} dicebatur \textgreek{μὴν ἱστάμευσο[?]},
secunda \textgreek{μὴν μεσεύων[?]}, tertia \textgreek{μὴν φθίνων[?]}.
\lnr{10}Idque factum, quia
illorum menses omnes erant \textgreek{τριακονθήμεροι[?]}.
% triakonthemeroi = thirty days.
\lnr{11}Persae vero in \textgreek{πεμπτάδας[?]},
non solum, quia eorum menses omnes \textgreek{τριακονθήμεροι[?]},
 sed etiam, quia
totus annus constat ex quinariis tribus et septuaginta.
\lnr{13}In mense \textgreek{ἐξαιρεσιμαίῳ[?]}
Athenienses pro \textgreek{δευτέρα ἱσταμεύου[?]} dicebant
 \textgreek{τρίτη ἰσταμεύου[?]}.
\lnr{14}Quamuis
enim mensem uno die mutilabant, tamen cum tertia mensis
pro secunda dicebant, non videbantur mensem mutilare, cuius
\textgreek{τριακάδα[?]} numerabant.
\lnr{17}Meton vero et Calippus eam diem eximunt,
quae post duas syzygias et dies quatuor succedebat.
\lnr{18}Mensium nomina
in antiqua Hebraici anni forma nulla fuerunt, neque in hodierna
Sinarum, Iaponensium, et Indorum.
\lnr{20}Menses enim illis ab ordine
primi, secundi, tertii dicuntur.
\lnr{21}In anno Romano mistae sunt appellationes,
ex cognominibus, et ordine numerario.
\lnr{22}Quidam etiam cognomines
imperatorum Romanorum, ut Cypriis \textgreek{καισάρειοσ, Σεβαστὸς,
Αυτοκρατορικὸς[?]}.
\lnr{24}Romanis ipsus Iulius, Augustus: et temporibus Domitiani
Germanicus pro Septembri, Domitianus pro Octobri.
% Insert: image of latin text "M. AVR. AVG. LIB."
\begin{figure}[t]
  \centering
  \includegraphics[scale=0.125]{011_lapis_lavinii}
  \caption{Lapis Lavinii}
  \label{fig:lapis_lavinii}
\end{figure}
\lnr{25}Martialis:
 \textit{Dum Ianus hiemes, Domitianus
autumnos}, et cetera.
\lnr{27}Sed Statius omnes
Kalendas vindicat Domitiano,
praeter Iulium, et Augustum,
– \textit{Nondum omnis honorem
Annus habet, cupiuntque decem tua
nomina menses.}
\lnr{32}Insania quoque
Commodiidem consecuta esset, si
longior vita monstro illi data fuisset.
% ō -> on (2x)
Augustum enim Commodum,
% ō -> om
Septembrem Herculeum, Octobrem
Invictum, Novembrem
Exuperatorium, Decembrem
Amazonium vocari edicit.
\lnr{39}Extat
quoque lapis Lavinii, in quo mentio
Iduum Commodarum.
% There is also a stone in Lavinus, which mentions "the ides of Commodus"
\lnr{41}Ubi et
nomen Commodi Senatusconsulto prius derasum, postea alia manu
incisum.
% "And where the name of Commodus was earlier erased by decree of the senate,
% a later hand inscribed it (again)."

% 12
% {PDF page nr}{source page nr}{line nr}
\plnr{95}{12}{3}Quaedam nationes etiam geminos menses cognomines habent.
\lnr{4}Annus Syrochaldaicus habet geminum Tisrin, item geminum Conum.
\lnr{5}Annus Hagarenus geminum Regiab, et geminum Giumadi.
\lnr{6}Annus Saxonicus geminum Giuli, et geminum Lida.
\lnr{6}Sed in
anno embolimaeo Lida est tergeminus.
\lnr{7}Et tunc annus ille dicebatur
Trilida.
\lnr{8}Item, diversarum nationum iidem menses communes.
\lnr{8}Nam
Panemus in anno Macedonico fuit, item Corinthiaco, et Thebano.
\lnr{10}Artemisius communis fuit Laconum, et Macedonum: Carneus Syracusanis,
et Cyrenensibus usitatus.
\lnr{11}Sed differbant situ anni et tempore:
ut suo loco disputabitur.
\lnr{12}Sic Martius primus erat Romanorum:
tertius Albanorum, Aricinorum, Formianorum: quartus Forensium,
Pelignorum, Sabinorum: quintus Faliscorum, Laurentum:
sextus Hernicorum: decimus Aequicolorum.
\lnr{15}Haec in genere
de mensibus.

\section{De Anno}

\lnr{17}Maximum \textgreek{σύστημα[?]}
 dierum annus, sed qui multipliciter dictus
sit.
\lnr{18}Tot enim constitui possunt, quot sunt siderum errantium
periodi.
\lnr{19}Est enim annus circuitus eius periodi, cuius cognominis
ipse est.
\lnr{20}Ut annus Solaris est cognominis circuitus eius sideris,
qui quidem circuitus dupliciter sumitur.
% Started new sentence here. Original has comma. New sentence fits better
% with comming sentences.
\lnr{21}Aut ab Solstitio ad Solstitium,
% à -> ab
ab bruma ad brumam: et est minor anno Iuliano.
% à -> ab
\lnr{22}Aut ab puncto Zodiaci,
% à -> ab
ad idem punctum Zodiaci, qui est maior anno Iuliano.
% Changed point to comma, to get same sentence structure as previous one.
\lnr{23}Hoc est maior 365~\myfrac{1}{4} diei.
\lnr{24}Quo ad id punctum Zodiaci redit, unde profectum
erat.
\lnr{25}Eadem fere quantitas quae et Soli, attribuitur Veneri et Mercurio.
\lnr{26}Saturni periodus est dierum 10747.18'.59''.13'''.
\lnr{26}Hoc est annorum
Aegyptiorum 29. dierum 162.
\lnr{27}Iovis annus dierum 4330. horarum 17.14'.
\lnr{28}Id est annorum Aegyptiorum 11.315.
\lnr{28}Martis annus dierum
686. horarum 22.24'.
\lnr{29}Annorum Aegyptiorum 1.321 dierum.
\lnr{29}Lunae,
dierum 29.31'50''.8'''.
\lnr{30}Obtinuit tamen vulgo, ut duorum siderum,
Solis et Lunae, labentem coelo qui ducunt annum, ratio in temporibus
civilibus haberetur.
\lnr{32}Et Lunae quidem primum unus circuitus
pro anno habebatur, ut apud Aegyptios.
\lnr{33}Deinde tres, ut apud eosdem
Aegyptios et Arcades.
\lnr{34}Tandem duodecim periodi Lunares annum
civilem constituerunt dierum 354 cum triente, et paulo plus quam
duum trientum horariorum.
\lnr{36}Duodecim quoque segmenta Zodiaci
componunt annum Solarem tantum, quantum diximus.
\lnr{38}Sed ignoratio
motuum utriusque sideris alias atque alias anni formas veteribus
peperit:

% 13
% {PDF page nr}{source page nr}{line nr}
\plnr{96}{13}{1}quarum vetustissima est ea, quae annum quidem ad cursum
Lunae describebat:
\lnr{2}sed incertis neomeniis, quae non prodeunt ex observatione
motus Lunae, quales vulgus rusticorum observare solet, et
quae proprie civilem mensem constituere non possunt.
\lnr{4}Cum igitur
hoc modo incertae essent neomeniae, convenit primum, ut menses omnes
tricenis diebus explicarent, annumque dierum sexaginta et trecentum
constituerent.
\lnr{7}Quod genus longe desciscebat ab modo anni
% à -> ab
Lunaris.
\lnr{8}Haec diu seruata fuit apud Graecos anni forma.
\lnr{8}In Oriente
septuagesima secunda pars illius anni, hoc est quinque dies, accesserunt
anno Graeco: ut anni modus fuerit dierum trecentorum sexaginta quinque:
qua ratione ab anno solari se minimum discedere arbitrati sunt.
\lnr{12}Unde duo praecipua genera anni apud veteres suerunt neque Lunaria,
neque Solaria, sed ambigui inter utrumque generis.
\lnr{13}Prior forma in
Graecia resedit: altera in Oriente.
\lnr{14}Graeci vero non una via ad emendationem
suae aggressi sunt.
\lnr{15}Difficile erat menses plenos omnes ad
Lunae rationes exigere: et tamen in quibusdam actibus civilibus opus
habebant motu Lunae.
\lnr{17}Nam semper Olympias plenilunio, et \rnum{xv}
die mensis celebrabatur.
\lnr{18}Ut igitur annus Graecus aequabilis Olympiadem
deprehenderet in \rnum{xv} mensis, hoc difficile non erat.
\lnr{19}Ut autem
\rnum{xv} mensis in \rnum{xv}
 Lunae incidat in mensibus aequabilibus, hoc fieri non
potest, nisi post fingula quadriennia, adiectis unicuique anno singulis
biduis, quas \textgreek{ἀνάρχους ἡμέρας[?]} vocabant.
\lnr{22}Haec Tetraeteris Elidensibus
vocata est Olympias, Delphis Pythias.
\lnr{23}Eiusque mensis primus duantaxat
erat Lunaris: reliquorum ratio claudicabat.
% written as: claudicābat. Probably a smudge. Other copies don't have the bar.
\lnr{24}Primus Cleostratus
eum annum in Lunarem modum reformare conatus est, excogitata
octaeteride dierum 2922, cuius menses alternis pleni et cavi: anni vero
singuli communes 354 dierum: embolimaei 384. communes quidem
quinque, embolimaeitres.
\lnr{28}Syzygiae autem novem et nonaginta.
\lnr{28}Octaeteridum
vitio deprehenso, Meton enneadecaeterida excogitavit dierum
solidorum 6940.
\lnr{30}Cui castigandae periodus Calippica successit dierum
27759, sine ullis scrupulis appendicibus, anno ab editione Metonica
centesimo tertio.
\lnr{32}Hanc excepit ultimus, tanquam secutor quidam,
Hipparchus, annis circiter centum octoginta octo ab epocha Calippica,
periodo publicata dierum 111035: quae minor est Calippicis rationibus
die uno, Metonicis autem quinque.
\lnr{35}Quare duae castigationes adhibitae
anno aequabili Graeco.
\lnr{36}Altera est coniugatio alterna vel interrupta
mensium plenorum et cavorum, ut cum ipsa Luna congruerent, quod
annus Graecus maior esset Lunari.
\lnr{38}Altera est embolismus mensium, ut
cum sole aequaretur, quod annus Lunaris minor est Solari.
\lnr{39}Sed alternatio
plenorum et cavorum mensium aliquando variat: idque fit aut
naturaliter, aut civiliter.
\lnr{41}Naturalis varietas committitur propter embolismum
aut mensis, aut diei.

% 14
% {PDF page nr}{source page nr}{line nr}
\plnr{97}{14}{1}Utroque enim modo duo menses pleni continuantur.
\lnr{2}Ut in anno Iudaico cum intercalatur mensis Adar, tunc
Schebat, et Adar embolimus ambo sunt pleni.
\lnr{3}In anno vero Arabico
cum accedit dies mensi ultimo, qui Dulhagiathi dicitur, tunc et ipse
Dulhagiathi, et antecedens Dulkaadathi ambo fiunt tricenum dierum.
\lnr{6}Sed in Samaritano saepe continuantur tricenarii menses, et in antiquo
Iudaico, ut ex Talmud et Iad Mosis cognoscimus: et menses Harpali,
Metonis, et Calippi non semper alternis continuati sunt.
\lnr{8}Sed saepe bini
pleni continuati, nunquam autem bini cavi.
\lnr{9}Quin etiam cum dies accedit
ultimo mensi Arabico, tres continui menses sunt pleni, Dulkaadathi,
Dulhagiathi, et Muharam sequentis anni.
\lnr{11}Isque annus ab Arabibus
dicitur \textarabic{[Arabic]} hoc est embolimaeus.
\lnr{12}Sic etiam anno Iudaico pleno
tres menses continui sunt pleni, Tisri, Marchesuvan, Casleu.
\lnr{13}Civilis
varietas accidit anno Iudaico tantum, accrescente mensi Marcheschuvan
die uno: et Marchesuvan ex cavo sit plenus.
\lnr{15}Rursus et in embolismo
mensium differentia situ, et tempore.
\lnr{16}Situ, si aut in medio, aut in calce
intercalatio fiat.
\lnr{17}Ut in anno Attico ultimus mensis intercalabatur, qui
dicebatur \textgreek{πορειδεὼν πρότερος[?]}.
\lnr{18}In Iudaico sextus mensis intercalatur, et
dicitur Adar prior.
\lnr{19}In anno Hagereno mensis embolimus erat desultor,
qui omnes menses anni percurrebat in annis 228, quae sunt enneadecaeterides
duodecim.
\lnr{21}Qua intercalatione memoria proavorum nostrorum
utebantur Turcae Cilices, donec annum Hegirae simplicem
Muhamedicum usurpare coeperunt.
\lnr{23}At in anno prisco Romanorum
situs embolismi longe diversus ab aliis.
\lnr{24}Non enim is inter duos
menses interiiciebatur, ut alias solet: sed in mensem ipsum, tanquam
surculus in truncum infindebatur.
\lnr{26}Inter \rnum{xxiii} enim, aut \rnum{xxiiii},
aut inter \rnum{xxii}, et \rnum{xxiii} Februarii inserebatur.
neque vero sine caussa.
\lnr{28}Hoc enim semper observabant, ut mensis proximus Martio semper esset
dierum \rnum{xxviii}.
\lnr{29}Eratque Februarius ordinarius.
\lnr{29}At intervallum inter exitum
Ianuarii, et Kalendas Februarii ordinarii imputabatur Merkedonio.
\lnr{31}Et Kelendae Februarii ordinarii in anno embolimaeo nunc in Regifugium,
nunc in Terminalia, incurrebant.
\lnr{32}Neque enim semper inter
Terminalia, et Regifugium intercalabantur, ut vult Censorinus.
\lnr{34}
Quia hoc pacto Februarius ordinarius nunc viginti octo; nunc undetricenum
dierum fuisset.
\lnr{35}Quod tamen falsum ex Varrone convicitur.
% conuicitur: better visable in other copy
\lnr{36}Tempore differt intercalatio, quatenus Iudaei nunquam intercalant,
priusquam \textgreek{ὑπεροχὴ ἡλιακὴ[?]}, qui sunt dies decem cum horis paulo
magis quam una et viginti, eo rationes Solis deduxerint, ut commode
mensis Lunaris conflari possit.
\lnr{39}Quod spatium numquam maius est
triennio, nunquam minus biennio: et in \rnum{xix}. annis semper septies fit.
\lnr{41}At in Calippico et Metonico anno aliquando citius, aliquando ferius
intercalabatur, quam ratiocinia \textgreek{ὑπεροχῆς ἡλιακῆς[?]}
 postulare videntur.

% 15
% {PDF page nr}{source page nr}{line nr}
\plnr{98}{15}{2}Quandoquidem hoc unum cavent praecipue Athenienses,
 ne Hecatombaeonis
neomenia Solstitii priscam epocham antevertat: cum in
% neomenia: better visable in other copy
anno Iudaico ut plurimum neomenia Tisri aequinoctium autumnale,
neomenia vero Nisan aequinoctium veris antiquum, si ratio Iuliani
anni habeatur, antevertat.
\lnr{6}Anni Lunaris non unum genus est: sed
summa divisio in duo fastigia discedit: in annos periodicos, et simplices.
\lnr{8}Anni periodici dicuntur, qui certo annorum orbe, interventu
embolismorum, recurrunt.
\lnr{9}Huius intervalli modum veteres certo
definire non potuerunt.
\lnr{10}Quippe Cleostratus dierum 2922, Harpalus
2924, Eudoxus plusquam 2922, minus quam 2924: Meton aliter:
et ab omnibus diverse Calippus, et denique ab eo discedens Hipparchus.
\lnr{13}Cuius sententia, sed caelestibus rationibus leviter castigata,
 enneadecaeterida
% έννεακαιδεκα-ετηρίδα: Meton cycle, or "Great Year". Nine-and-ten anniversary
% ετηρίδα: anniversary
% Search "Enneadecaeteris" on Wikipedia to give the "Metonic cycle" page.
Lunarem minorem Iuliana statuit, hora una cum scrup. [?] paulo
% What is "scrup." short for? scrupula? See also p.17, 18
plus quam viginti septem.
\lnr{15}Simplices anni et ipsi quidem sine remedio
intercalationis in pristinam epocham recurrunt, sed longo intervallo,
annorum scilicet Iulianorum 228, qui sunt anni simplices Arabici 235,
scrupuli diurni quinquaginta.
\lnr{18}Sunt et in annis Lunaribus cavi, superflui,
aequabiles.
\lnr{19}Annus cavus is est, cui competit \textgreek{ἐξαίρεσις ἡμέρας[?]}.
\lnr{20}Ideo ab nobis \textgreek{ἐξαιρεσιμαῖον ἔτος[?]} vocabitur.
% à -> ab
\lnr{20}Ex eo enim eximitur dies
vel propter civile institutum, cuiusmodi est annus Iudaicus,
quem defectivum
Computatores Iudaeorum vocant.
\lnr{22}(In eo quippe Casleu, qui natura est plenus, instituto fit cavus.)
% Casleu: Hebrew כִּסְלֵו, Kislev, the third month of the civil year
% (9th of ecclesiastical year) in the Hebrew calendar.
\lnr{23}Vel naturali de caussa: ut anno
decimonono Cycli Paschalis Dionysius diem unum eximit, quem
vocavit Saltum Lunae: Graeci vero Computatores
 \textgreek{ὑποτομὴν σελήνης[?]}.
\lnr{26}Quamquam inepte annum ultimum enneadecaeteridis constituit dierum
duntaxat 353, cum eiusmodi annus natura nullus fit.
\lnr{27}Superfluus
annus vocetur ab nobis \textgreek{ἔτος ὑπερήμερον[?]}.
% à -> ab
\lnr{28}Accedit enim illi \textgreek{ἡμέρα ἐμβόλιμος[?]}
tam ex caussa civili, ut in anno Iudaico Marcheschuvan naturaliter
cavus, civiliter fit plenus: quam e caussa naturali: ut undecim anni
in Triacontaeteride Arabica augentur singulis diebus ex ratiociniis
Lunae collectis.
\lnr{32}Annus aequabilis vocetur \textgreek{ἔτος ὁμαλόν[?]}.
\lnr{32}Iudaeis computatoribus
dicitur annus ordinarius.
\lnr{33}Is est, cui nihil accedit, nihil decedit.
\lnr{34}Huc usque ad annum Lunarem deduxit nos aequabilis minoris
disputatio.
\lnr{35}Nunc de altero aequabili maiore disputandum, quo Aegyptii,
Persae, et Armenii, Mexicani, et Perusiani usi.
\lnr{36}Hic antiquitus
Orientis nationibus unus idemque fuit: praeter quam si quando
 \textgreek{ἐπαγόμεναι[?]}
% Epagomenal days are days within a solar calendar that are outside
% any regular month.
quinque in alium locum traductae, diversum anni caput constituebant.
\lnr{39}Qua \textgreek{ἐπαγομένων [?]} tralatione utebantur ii,
 qui post annos 120
aequabiles mensem solidum intercalabant, ut Persae: qui quidem
 \textgreek{ἐπαγόμένας[?]}
suas in aequinoctium vernum semper reiiciebant.
\lnr{41}Terminum autem vocabant \textsc{nevruz}.
% https://en.wikipedia.org/wiki/Nowruz
% Persian: نوروز‎‎ litterary "New Day"

% 16
% {PDF page nr}{source page nr}{line nr}
\plnr{99}{16}{1}Et habebant mensem desultorem
% mensem desultorem = leap month?
\textgreek{ἐμβόλιμον[?]}, omnes menses anni pervagantem, donec in primum
% Embolismic month: 13th intercalary month inserted in the year.
mensem recurreret.
\lnr{3}Qui orbis non redibat, nisi anno aequabili 1461
vertente, qui sunt anni Iuliani perfecti 1460.
\lnr{4}Hic est magnus annus,
cuius menses sunt annorum aequabilium tricenum, quot dierum simplex
mensis.
\lnr{6}\textgreek{ἐπαγόμεναι [?]} autem sunt quinquies quatuor annorum, ut
illae simplices quinque dierum.
\lnr{7}Quod autem illa anni forma retenta
fit, in caussa fuit non tam ignoratio annis solaris,
 quam facilis, et tractabilis,
ac vere popularis eius usus.
\lnr{9}Alioqui nulla fere natio fuit, quae
quadrantem anni Solaris ignorarit: sed modum illius dispensandi
nesciebant.
\lnr{11}Praeterea ab mensibus superfluis, qui sunt maiores tricenis
% à -> ab
diebus, refugiebant, quos necesse est retincri,
 quadrante illo retento.
\lnr{13}Aegyptii singulis quadrienniis exactis diem intercalabant
 in ortu Caniculae,
et quadriennium illud exactum \textgreek{ἔτος ἡλιακὸν, ἔτος θεοῦ, ἔτος
κυνικὸν[?]}, vocabant.
\lnr{15}Attici diem quarto quoque anno exacto intercalabant
inter septimum et octavum diem Ianuarii.
\lnr{16}Elidenses inter
octavum, et nonum Iulii.
\lnr{17}Syromacedones, Chaldaei, et Iudaei inter
septimum et octavum Octobris.
\lnr{18}Eamque diei intercalationem ab Seleucidarum
% à -> ab
temporibus usque ad imperium Constantini et infra retinuerunt
Iudaei: quam utique simul cum anni Calippici forma ab victoribus
% à -> ab
Syromacedonibus acceperant.
\lnr{21}Romani Atticos secuti brumae
sidere confecto intercalabant; quae ipsis Olympiadum mysteria vocabantur.
\lnr{23}Nam et Attici et reliqui omnes Graeci annum Solarem in
quatuor quadrantes dividebant, quae \textgreek{κέντρα[?]}
 vocabant, singulis dies 91.
hor. 7~\myfrac{1}{2} attribuentes.
\lnr{25}Quod ab temporibus Seleucidarum, ad hanc usque
% à -> ab
diem, Iudaei constanter observant.
\lnr{26}Itaque \rnum{viii} Iulii erant \textgreek{τροπαὶ θεριναὶ[?]},
\rnum{vii} Octobris \textgreek{ἰσημερία ὀπωρινὴ[?]}:
 \rnum{vii} Ianuarii \textgreek{τροπαὶ χειμεριναὶ[?]}, \rnum{viii}
Aprilis \textgreek{ἰσημερία ἐαρινέ[?]}.
\lnr{28}Quare cum legis \textgreek{τροπὰς θερινὰς[?]},
 et \textgreek{χειμερινὰς[?]},
nullas alias intellige, praeter has.
\lnr{29}Quod et \textgreek{περὶ ἰσημεριῶν[?]} quoque intelligendum.
\lnr{30}Haec \textgreek{κέντρα[?]} Iudaei Tekuphoth vocant.
\lnr{30}Germani, Celtae,
Saxones inter \rnum{xxv} et \rnum{xxvi} Decembris intercalabant:
 quam noctem
vocabant \textsc{mudranecht}.
\lnr{32}Tartari hodie inter ultimam Ianuarii,
et Kalendas Februarii, quas Kalendas patrio sermone Festum Alborum
% comma for period
vocant, quia albis vestibus eam diem colunt.
% comma for period
\lnr{34}Denique quanuis
Lunari anno, aut alio longe diverso ab Solari uterentur, tamen tacita
% à -> ab
quadam observatione post dies 1460 unum diem intercalandum esse
sentiebant.
\lnr{37}Neque enim aliter Habraei quatuor Tekuphas suas tueri
potuissent, nisi quadrante post quartum quemque annum rationibus accedente.
\lnr{39}Et sane unaquaeque Tekupha est dierum 91, horarum 7~\myfrac{1}{2}.
% Period inserted
\lnr{39}Unde
quatuor tantae Tekuphae fiunt dies 365~\myfrac{1}{4}.
\lnr{40}Displicuit tamen haec quadrantis
observatio Graecis Astronomis, propter causam admodum futilem
et puerilem, qua Solis quantitatem ad Lunae ratiocinia exigebant,
et cum utriusque sideris exactum modum adhuc non tenerent,
ex Lunae comparatione Solares rationes eliciebant.

% 17
% {PDF page nr}{source page nr}{line nr}
\plnr{100}{17}{3}Itaque tantam
censuerunt Solis quantitatem, quantam summam dies periodi in annos
periodi distributae relinquebant.
\lnr{5}Metonis periodus est dierum
6940.
\lnr{6}Divisa per 19 annos relinquit quantitatem anni Solaris Metonici
dierum 365. scrup. diurnorum 15~\myfrac{5}{19} Calippi periodus dierum
% Again "scrup.". See also p.15, 18
27759 per 76 annos divisa relinquit modum anni Calippici Solaris
dierum 365~\myfrac{1}{4}.
\lnr{9}Qualis est annus noster Iulianus.
\lnr{9}Periodus Hipparchi
est dierum 111035, annorum 304.
\lnr{10}Sed neglectis illis 4,
trecentesima pars diei detrahitur de quantitate anni Calippici Solaris,
ut fiat annus Solaris Hipparcheus
 dierum 365. hor. 5. 55.' 15.'' \myfrac{15}{19}
\lnr{13}Detractis ex quadrante hor. 0. 4.' 44.'' \myfrac{4}{19}
 quae etiam fuit sententia
Ptolemaei.
\lnr{14}Itaque ex sententia Hipparchi et Ptolemaei annus
Tropicus, est annus Iulianus, vel Calippicus nonadecima parte
differentiae enneadecaeteridis Lunaris et Iulianae diminutus: qui
est verus annus Rabbi Ada: de quo alibi.
\lnr{17}Philolai Pythagorei magnus
annus dierum 21505~\myfrac{1}{2} per 59 annos divisus constituit modum
Solarem dierum 365.
\lnr{19}Oenopidae annus magnus dierum 21557
itidem per 59 annos divisus dat modum anni Solaris dierum 365 cum
parte dierum duum et viginti undesexagesima.
\lnr{21}Harpali octaeteride per
8 annos divisa remanet modus anni Solaris dierum 365~\myfrac{1}{2}.
\lnr{22}Annus magnus
Democriti dierum 29950~\myfrac{1}{2} per 82 annos divisus relinquit annum
Solarem dierum 365, cum quadrante et centesima sexagesimaquatra
parte unius diei.
\lnr{25}Denique nullus veterum non putavit rationes
Solis ad Lunam exigendas esse.
\lnr{26}Et quotiescunque ex certa collectione
dierum utriusque sideris rationes congruerent, dies illi per tot
annos divisi, quot ex illa summa dierum constitui poterant, visi sunt
illis certam anni Solaris quantitatem definire posse.
\lnr{29}Sapientiores vero,
quanuis incomprehensibilem illam existimarent, tamen pro vero quod
proximum putabant amplexi sunt, dies trecentos sexaginta quinque
cum quadrante, qui est modus anni Iuliani.
\lnr{32}Cui singulis quadrienniis
exactis unus dies accrescit.
\lnr{33}Sed hic annus comparatione Aegyptiaci
est Solaris: comparatione autem Tropici est aequabilis.
\lnr{34}Maior
enim est vera anni ratione scrup. horariis 11.' 6.'' 40.'' secundum
% Again "scrup.". See also p.15, 18
Gelalaeam formam, aut 10.' 48.'' fere, ut Alfonsini docent.
\lnr{36}Neque
Prutenicae tabulae multum abludunt, quae constituunt motum
aequalem Solis ab aequinoctio dierum 365. Hor. 5. 49.' 15.'' 46.'''
\lnr{39}Itaque hinc nasci possunt aliquot genera anni Solaris.
\lnr{39}Aequabilis,
ut Iulianus.
\lnr{40}Tropicus, ut Persarum Gelalaeus.
\lnr{40}Rursus Tropicus
aut aequabilis, aut caelestis.
\lnr{41}Aequabilis
Tropicus, cuius quantitas
Tropica est, partes autem, hoc est menses, aequales et civiles: ut is,
quem modo dixi, Galelaeus.

% 18
% {PDF page nr}{source page nr}{line nr}
\plnr{101}{18}{2}Descriptus est enim mensibus aequalibus,
omnibus tricenum dierum, cum epagomenis appendicibus, quae
in communi anno sunt quinque, in embolimaeo sex.
\lnr{4}Caelestis Tropicus,
cuius partes in naturalia Zodiaci segmenta tributae sunt.
\lnr{5}Rursus
et annus Solis aequabilis in civilem et caelestem dividi potest.
\lnr{6}Civilis,
ut Iulianus Romanorum, Syrograecorum, Graecorum Elkupti.
\lnr{7}Caelestis,
ut Dionysianus Prolemaei Philadelphi.
\lnr{8}Nam et is quoque quadrantem
Canicularem quadriennio exacto accipiebat.
\lnr{9}Finis vero
omnis periodi is est, ut caput recurrat et revoluatur in idem principium,
quam \textgreek{ἐποχὴν[?]} Graeci vocant: quae quidem pessum iverit tandem,
non seruata veri anni Tropici mensura.
\lnr{12}Et quia annus Iulianus
suam tueri non potuit, manifestum est Kalendas Ianuarias ab \rnum{viii}
parte Capricorni, in qua statuerat eas Caesar, in vicesimam primam
fere traductas esse hodie.
\lnr{15}Sed nihilo commodius epocha in enneadecaeteride
seruari potest.
\lnr{16}Nam enneadecaeteris Tropica est velocior
Lunari horis plusquam duabus.
\lnr{17}Contra enneadecaeteris Iuliana
maior Lunari hora una, et scrup. plusquam 26.
% Again "scrup.". Scrupulus? See also p.15, 17.
\lnr{18}Cum vero peccatur
utraque ratione, Tropica et Iuliana, Luna, cuius rationes mediae sunt
inter illas duas, fines epochae suae tueri non potest: ut in cyclo Dionysii
Paschali accidit, cuius neque rationes ad enneadecaeterida Lunarem
collectae sunt, neque epocha ad Solis motum castigata: sed eius
forma potius tota mere Calippica est.
\lnr{23}Ita ut eius statum post trecentos
4 annos variare necesse sit.
% 4. -> 4
\lnr{24}Quare ut epochat suas servarent illi veteres,
immanes periodos excogitaverunt, quales illae Calippi, Philolai, Democriti,
Oenopidae.
\lnr{26}Sunt etiam periodi, quae omnem modum excedebant.
\lnr{27}Et cum in omnibus illis orbibus annorum praecipuam
utriusque sideris rationem haberent, tamen nescio quae confidens eos
incessebat opinio, non solum utriusque sideris, sed etiam omnium
\textgreek{ἀπλανῶν ἀποκαράστασιν[?]} illo circuitu fieri.
\lnr{30}Sic Harpalus et Eudoxus putarunt
in sua Octaeteride omnes \textgreek{ἀνατολὰς[?]}
 et \textgreek{δύσεις[?]} in orbem redire.
\lnr{32}Idem etiam censet fieri Aratus in Metonica enneadecaeteride, Eudoxum
suum sectus, qui in fabrica Sphaerae suae eam planetarum et inerrantium
harmoniam in eorum orbibus ostendit esse, ut sequente
restitutione utriusque sideris, necessario et omnium inerrantium reditum
contingere concluderet.
\lnr{36}Propterea tot Sphaeras \textgreek{ἄστρων[?]} commentus
est, quot narat Aristoteles libro \rnum{xi}
 \textgreek{τῶν μετὰ τὰ φυσικὰ[?]} quem
consulas licet.
\lnr{38}Quin etiam Calippus alios orbes praeter Eudoxum
addidit, ea ratione, ut \textgreek{ἀποκατάστασιν τῶν φαινομένων[?]} adstrueret,
 \textgreek{τὰ φαινόμενα εἰ μέλλοι τις ἀποδώσειν[?]},
ut Aristoteles de ea re scribens pronunciavit.
\lnr{41}Itaque \textgreek{τῶν φαινομένων[?]} nomine intelligendum ortus,
 et occasus \textgreek{τῶν ἀπλανῶν[?]},
non autem \textgreek{τῶν πλανητῶν καὶ τὰς ἐπισημασίας[?]}, hoc est significationes
eorum: quas in orbem redire cum Luna et Sole in enneadecaeteride
Meto quidem, Calippus, et Hipparchus putarunt, et aliis
persuaserunt, donec deprehenso vero anni Tropici modulo vitium
harum periodorum castigatum est.

% 19
% {PDF page nr}{source page nr}{line nr}
\plnr{102}{19}{5}Cicero quoque apud Macrobium,
sexto de republica, annum illum immanem, quem ex tot millibus
annorum simplicium componit, non aliter in orbem rediturum
cum omnibus errantibus et inerrantibus censet, quam si eadem defectio
Solis in eodem loco, eodem tempore fiat: quanuis defectiones
cyclo enneadecaeterico recurrant non raro.
\lnr{10}Et tamen ea eclipsi putat
non tantum Solis et Lunae, sed etiam quinque errantium ad eandem
inter se comparationem, confectis omnium spatiis, reditum fieri, quo
eadem caeli positio, siderumque, quae ab initio maxime fuit, rursus existit.
\lnr{14}Quare eclipses ad eam rem notabant veteres, ut etiam
 \textgreek{ἐκλειπτικάς
περιόδους[?]} excogitarint.
\lnr{15}\textgreek{ἐξελιγμοὺς[?]} vocabant.
\lnr{15}Eorum vetustissimus fuit
dierum 6585~\myfrac{1}{3}, qui sunt anni Arabici 18, syzygiae 7.
\lnr{16}In genere vero
sunt syzygiae 223.
\lnr{17}Quamobrem in secundo libro Plinii perparem legitur
sive culpa ipsius Plinii, sive librarii, defectus luminum ducentis
viginti duobus mensibus redire.
\lnr{19}Hipparchus alium \textgreek{ἐξελιγμὸν[?]} longe
maiorem excogitavit dierum 126007, syzygiarum 4267, annorum
Arabicorum 355 cum syzygiis 7: annorum Iulianorum 344 cum
diebus 361.
\lnr{22}Quae sunt tolerabiles periodi.
\lnr{22}Nam ab caussis naturalibus,
% à -> ab
nempe ab defectionibus luminum proficiscuntur.
% à -> ab
\lnr{23}Quemadmodum
etiam enneadecaeteris Lunaris, et Cyclus Solis: quorum illa Lunam
Soli restituit, hic Solem Septimanae, et praeterea periodus Mexicanorum
constans annis \rnum{lii}, quae restituit
 \textgreek{τὴν τρισκαιδεκαήμερον[?]}, quae ist ipsis
vicem nostrae Hebdomadis.
\lnr{27}Neque alia fuit periodus magna Persarum
veterum, quam Salchodai vocabant.
\lnr{28}Sunt et aliae, sed civiles, et Indictio;
Aliae inanibus coniecturis insistunt, ut Dodecaeteris Chaldaica
Genethliacorum, item Heracliti, Lini, Orphei, Dionis, et Magorum:
quorum periodus ad modum octavae sphaerae composita est annorum
360000 ab conditu Mundi, ut ipsi putant.
% à -> ab
\lnr{32}Quorum annorum hic est
centies octagies quater millesimus, sexcentesimus nonagesimus quartus.
\lnr{34}Sed longe illa Sinarum prodigiosior, iuxta quam hic annus Christi
1594 est ab conditu rerum octigenties octagies quater millesimus,
% à -> ab
septingentesimus septuagesimus tertius.
\lnr{36}Bonziorum vero Iaponensium
periodus annorum 470 desivit cum anno Christi 1561. et 1562
coepit sequens.
\lnr{38}Eiusque hic est vicesimus currens.
\lnr{38}Ea vertente scelera
extirpatum iri: reliquum tempus omnia pacata fore credunt.
\lnr{39}Taceo
diversas Christianorum, Iudaeorum, Samaritanorum de conditu rerum
opiniones: item Romanorum lustrum quinque annorum, saeculum
centum et decem.

% 20
% {PDF page nr}{source page nr}{line nr}
\plnr{103}{20}{1}Sunt et periodi Computatorum: ut Iudaea
annorum 6916, quae constat cyclis Lunaribus 364, Solaribus 247, periodis
magnis Dionysianis 13.
\lnr{3}Habetque tot cyclorum septimanas,
quot dierum septimanae sunt in anno Solari: tot periodos Dionysianas,
quot menses annus embolimaeus: tot cyclos Solares, quot cyclos
Lunares magnus cyclus Iudaicus.
\lnr{6}Itaque elegantissima est, et artificiosissima.
\lnr{7}Eiusque hic agitur annus 5354, anno Christi vulgari 1594.
\lnr{8}Et inibit 1595 annus eiusdem proximo autumno, unde omnes epilogismi
neomeniarum Iudaicarum.
\lnr{9}Periodus Dionysiana et ipsa ad
annalem computum pertinet, annis constans 532, ducto in sese utroque
cyclo.
\lnr{11}Verae quidem periodi magnae caput incurrit in annum
primum utriusque cycli, pertinetque ad methodum Lunae et Solis.
\lnr{12}Et
locum habet dumtaxat in anno Iuliano, hoc est in eo, cui praeter 365
dies quadrans attibuitur.
\lnr{14}Itaque eius initium est ab Kal. Ianuariis in
anno Romano: in anno Constantinopolitano ab Kal. Septembris. in
Antiocheno ab Kal. Octobris. in Alexandrino et Samaritano ab a. d.
\rnum{iiii}. Kal. Septemb.
% à Kal. -> ab Kal. (3x)
\lnr{17}Periodus vero Dionysii pertinet ad methodum
neomeniae Paschalis, initio sumto ab anno primo natalis Christi, ut
ipse quidem putabat: item ab anno decimo cycli Solis Iuliani, et ab
ea neomenia, cuius quartadecima dies proxime post
 \rnum{xxi}, aut in \rnum{xxii}
Martii conficeretur.
\lnr{21}Hactenus ab minimis initiis ad summa temporum
% à -> ab
incrementa, quam \textgreek{ὁμάδαχρόνων[?]} Graeci vocant, Chronologum
perduximus, et eum in conspectu totius antiquitatis collocavimus.
\lnr{24}Superest nunc, ut quae carptim et obiter perstrinximus, ea uberius
suis locis expicentur.
\lnr{25}Resumamus igitur eos annos, ex quibus tanquam
elementis, ad tot tamque diversa genera annorum progressus
factus est.
\lnr{27}Ex anno Graeco, qui est aequabilis minor, omnes anni, Lunaris
formas propagatas esse vidimus: ut ex Aegyptiaco, qui est aequabilis
maior, omnes Solares.
\lnr{29}Non igitur confuse, et per saturam haec
tractanda, sed suo quaeque et loco et ordine.
\lnr{30}Quatuor igitur libris
quatuor genera anni summa explicare decrevimus.
\lnr{31}Primus erit de
anno aequabili minore.
\lnr{32}Eo enim omnis Graecia usa tam diversis generibus,
quam multae fuerunt eius terrae nationes, et \textgreek{πολιτεῖαι[?]}.
\lnr{33}Itaque
ea erit reliqua pars huius libri.
\lnr{34}Secundum locum sibi vindicat annus
Lunaris, quia ex illo priore derivatus.
\lnr{35}Tertius liber complectetur anni
aequabilis maioris formas, \textgreek{ἰδιότητας[?]}, et differentias.
\lnr{36}Quartus illius anni
traduces et propagines persequetur, diversa nempe anni Solaris genera,
et mutationes.
\lnr{38}Haec est pars prior, quam initio huius diatribae.
\lnr{39}Chronologo promisimus, de annorum et temporum Civilium generibus.
\lnr{40}Altera pars est de charactere, qui necessarius est notandis temporum
intervallis, quae sequentibus libris tractabimus, item diversis
computis nationum annalibus, de quibus librum singularem ad calcem
operis adiiciemus, non tanquam appendicem, sed partem unam
operis nostri.

% 21
% {PDF page nr}{source page nr}{line nr}
\plnr{104}{21}{3}Quis igitur sit usus characteris temporum, docet nos
Dionysius ex Ephoro, qui cum annum excidii Troiae ex Olympiadum
epocha notare non posset, cum is casus aliquot seculis antiquior
sit prima Olympiade, dixit id accidisse eo anno Attico, quo viginti
\textgreek{περιτταὶ ἡμέραι[?]} annum explebant.
\lnr{7}Statim peritis anni Attici subolebat,
quo anno id accidere potuerit.
\lnr{8}Sciebant enim quoties in quanto
intervallo annorum id fieri posset.
\lnr{9}Exemplo Ephori aut Dionysii
erit nobis character excogitandus, quo animus anceps in trivio constitutus
quaesitum ad fontem manu deducatur.
\lnr{11}Erit igitur primum
totius instituti nostri fundamentum annus Iulianus, quem fingimus
ante multa millia annorum fuisse.
\lnr{13}Characteres vero illi duos dabimus,
cyclum Lunae Dionysianum, cuius hic est annus \rnum{xviii}.
\lnr{14}Et cyclum
Solis Iulianum, cuius hodie annus \rnum{vii} currit.
\lnr{15}Tertium etiam,
ubi ratio temporum patietur, Indictiones non aspernabimur.
\lnr{16}Nam
qui his characteribus semel uti institerint, illi,
 quae sit constantia, et fides
illius methodi pulcherrimae in ratione temporum, experientur.
\lnr{18}Si
quis hoc anno Christi 1594 incertus, quot annos natus sit, tamen et
maiorem se quadraginta novem annorum, et minorem quinquaginta
sex sciat, is imitatur imperitiam Chronologorum Graecorum, qui
circiter illius, et illius regis tempora illud,
 et illud accidisse dicunt, annum
vero certum non difiniunt.
\lnr{23}Sed cum idem adiicit natum se Nonis
Augusti, feria quinta, is addit characterem certum et indubitatum,
quales sunt viginti \textgreek{περιτταὶ ἡμέραι[?]} Ephori.
\lnr{25}Nam feria quinta non
potuit incurrere in Nonas Augusti, nisi cum litera Dominicalis est C.
Ante 49 autem annos id accidit anno Domini 1540, cyclo Solis nono.
\lnr{28}Itaque hoc characterismo constantissime affirmanus eo anno hominem
natum, et proximis Nonis Augusti Iulianis illi quinquagesimum
quintum natelem initurum.
\lnr{30}Idem usus cycli Lunaris, adhibita
castigatione, ut ab prima Olympiade, ad annum Domini 1400, tot
% à -> ab
dies neomeniis adhibeas, quoties 304 annos reperies.
\lnr{32}Exemplum.
% Period after Exemplum: New sentence or different punctuation mark?
\lnr{33}Hic est annus ab prima Olympiade 2370.
% à -> ab
\lnr{33}In quibus annis septies reperitur
numerus 304.
\lnr{34}Septem igitur dies neomeniis hodiernis adiiciendi.
\lnr{35}Verbi gratia.
% Periods after numbers that do not indicate the end of a sentence
\lnr{35}Anno primo cycli epactae sunt \rnum{xi}. novilunium
Martii \rnum{xviii}. additis
 \rnum{vii}. diebus, novilunium, vel potius coniunctio
luminarium erat in \rnum{xxv}.
\lnr{37}Martii anno quarto ante primam Olympiadem,
aut quintodecimo post eandem primam Olympiadem, et deinceps
ad 304 annos.
\lnr{39}Sed ab hoc saeculo nostro post 150 annos minuendae
erunt neominiae totidem diebus, quoties 304 anni reperientur
post annum Christi 1700. et fortasse citius.
\lnr{41}Sed quia nullam epocham
veterem certiorem Olympiadum capite habemus: illud autem
cum vetustate comparatum novitium esse videtur: inutiles erunt characteres
cyclorum et Indictionis, nisi ab quadam remotissima epocha
% à -> ab
initium temporum instituamus.

% 22
% {PDF page nr}{source page nr}{line nr}
\plnr{105}{22}{4}Excogitemus igitur periodum,
quae et utrunque cyclum, et Indictionem contineat: quod fiet, si periodum
Dionysii Exigui quindecies multiplicemus: qui fient anni
7980.
\lnr{7}Ita periodus illa incipiet ab anno primo tum utriusque cycli,
tum Indictionis: et proinde eiusdem ultimus annus definit in ultimis
utriusque cycli, et Indictionis.
\lnr{9}Sed annus Christi, ut vulgo putamus,
3267 desinet in ultimum utriusque cycli, et Indictionis.
\lnr{10}Ergo deductis
3267 de 7980 annis, relinquetur epocha anni ante vulgarem
Christi, nempe 4713.
\lnr{12}Ita ut 4714 sit primus annus Christi vulgaris cyclo
Solis \rnum{x}, Lunae 2, Indictionis 4, ab Kal. Ianuarii: quamuis et Indictio
% à -> ab
autumno proxime antecedenti, Cyclus autem Lunae Martio sequenti
caeperit.
\lnr{15}Quare annus iste, qui ex errore vulgi putatur 1594, est 6307.
periodi huius, quam Iulianam vocamus, quod ad Iulianam anni formam
accommodata sit.
\lnr{17}Ideo 6307 divisis per 28, per 19, per 15 habebimus
huius anni 6307 periodi Iulianae, vel vulgaris Christi 1594, cyclum
Solis septimum ab Kal. Ianuarii:
\lnr{19}Lunae decimumoctavum ab
% à -> ab (2x)
Martio sequente:
\lnr{20}Indictionis septimum Caesarianae quidem ab ante d.
\rnum{viii} Kal. Octobris antecedentis anni 6306:
\lnr{21}Pontificiae vero ab Kalendis
% à -> ab
Ianuarii anni propositi 6307.
\lnr{22}Non praedicabo laudes huiusce periodi:
Chronologi et astrologi, qui omnia \textgreek{ἐπιστημονικῶς[?]}
 disputare volunt,
non poterunt eam satis laudare.
\lnr{24}Qui igitur eclipses ex Tabulis
% Tabulis: clearer in other copy.
Prutenicis putare volent, ex anno periodi Iulianae auferant 2408.
\lnr{25}Et
cum residuo toto excerperant tempora epochae diluvii.
\lnr{26}Exemplum: Eclipsis
Lunaris accidit in Septembri anno Olympiadico 446, qui est annus
periodi Iulianae 4383.
\lnr{28}Deductis 2408, remanent 1975.
\lnr{28}Excerpo
primum 1900 ex epocha Diluvii: deinde 75, ex filo annorum expansorum.
\lnr{30}Postremo menses usque ad Septembrem.
\lnr{30}Et reliqua ut ex methodo
Prutenica.
\lnr{31}Qui omne dubium ex temporum ratione tollere
volet, uti debet hac periodo, sine qua nihil unquam certi in natione
temporum adferre poterit.
% === End of the part entered as litteral transcript.

\section{De Anno aequabili Minore Graecorum}
\lnr{34}Cum quidam veterum, ut Macrobius et Solinus, annum Graecorum
merum Lunarem fuisse prodiderint:
\lnr{35}neque solum in ea
haeresi fuerit vir eruditissimus Theodorus Gaza, sed et vetustissimum
scriptorem Herodotum opinionis suae testem adhibeat:
\lnr{37}equidem non
temere ab eius auctoritate discedendum esse censuissem, nisi hominem
clarissimum, atque utriusque linguae vindicem, in re manifesta
pueriliter erasse deprehendissem.

% 23
% {PDF page nr}{source page nr}{line nr}
\plnr{106}{23}{3}Is igitur ut probet menses Graecorum
Lunares, et alternis plenos et cavos fuisse, haec verba ex Herodoto
producit: \textgreek{ἐς τὴς ἑβδομήκοντα ἔτεα ὀῦον τὴς ζοῆς ἀνθρώπῳ προτίθημι[?]}.
\lnr{5}\textgreek{ὀῦτοι
ἐόντες ἐνιαυτοὶ ἑβδομήκοντα παρέχονται ἡμέρας διηκοσίας καὶ πεντακισχιλίας καὶ
δισμυρίας, ἐμβολίμου μηνὸς μὴ γενομένου[?]}.
\lnr{7}Videamus, an vera sit summi
viri sententia: et dies vicesies quinquies mille ac ducentos per septuaginta
annos partiamur.
\lnr{9}Prodit modus unius anni, dies trecenti sexaginta.
\lnr{10}Perperam igitur Lunarem annum definit, cuius menses omnes
fuerunt solidi.
\lnr{11}Duodecim enim menses omnes \textgreek{τριακονθημέρους[?]}
annum habuisse, prodit Herodotus, non, ut ipse vult, alternis plenos
et cavos.
\lnr{13}Sed cum ea fuerit Gazae sententia, mirum, non contentum
fuisse hominem, unum Herodoti testimonium contra se produxisse,
nisi et Aristotelis altero ex libris \textgreek{ζώων ἱστρίας[?]}
 loco magnam iniuriam
existimationi suae fecisset.
\lnr{16}Scribit enim Aristoteles eo, quem ipse
Gaza adducit, loco,
 \textgreek{ἔνιαι τῶν κυνῶν τίκτουσι πέμπτον μέρος ἐνιαυτοῦ, τοῦτ᾽ ἔστιν
ἡμέρας ἑβδομήκοντα καὶ δύο[?]}.
\lnr{18}Sed in iisdem libris idem Aristoteles: \textgreek{κύει δὲ
ἡ μὲν Λακωνικὴ ἕκτον μέρος τοῦ ἐνιαυτοῦ[?]}.
\lnr{19}\textgreek{τοῦτο δέ ἐστιν ἡμέραι ἑξήκοντα[?]}.
\lnr{19}En quinquies
textsc{lxxii} dies est annus solidus Graecorum, hoc est totidem dierum,
quot iam posuimus ex Herodoto, nempe \rnum{ccclx}.
\lnr{21}Idem etiam Cleobuli
aenigma canit, quod ex ipso Gaza confessionem expresserit.
\lnr{22}Id eiusmodi
est:
\begin{verse}
\textgreek{Εἷς ὁ πατέρ. παῖδες δὲ δυώδεκα. τῶν δὲ ἑκάστῳ[?]}\\
\textgreek{Παῖδες τριήκοντα διάνδιχα εἶδος ἔχουσαι.[?]}\\
\textgreek{Αἱ μὲν λευκαὶ ἔασιν ἰδεῖν. αἱ δ᾽ αὖτε μέλαιναι.[?]}\\
\textgreek{Α᾽θάνατοι δέ τε ὀῦσαι ἀποφθινύθουσιν ἅπασαι.[?]}
\end{verse}
% Greek Anthology 14.98
% "The Greek Anthology" tr. W.R. Paton, vol V. p76;
% Book XIV "Arithmetical problems, riddles, oracles", nr 101
% ΚΛΕΟΒΟΥΛΟΥ ΑΙΝΙΓΜΑ
% An Enigma of Cleobulus:
% There is one father and twelve children. Each
% of these has twice thirty children of different aspect:
% some of them we see to be white and the others
% black, and though immortal, they all perish.
%  Answer: The year, months, days and nights.

\lnr{28}Aenigma quidem: sed eiusmodi, ut ex eo vel pueri divinent,
 annum
Graecorum habuisse menses \textgreek{τριακονθημέρους[?]} omnes.
\lnr{29}Sed clarius Plinius,
ac sine ullo aenigmate: \textit{Nulli,} inquit,
 \textit{arbitror plures statuas dicates,
quam Demetrio Phalereo Athenis.}
\lnr{31}\textit{Siquidem} \rnum{ccclx} \textit{statuere,
quas mox laceraverunt, nondum anno hunc numerum dierum excedente.}
\lnr{33}Cuius loci Pliniani Varronem interpretem dare possumus,
qui apud Nonium scribit Demetrium Phalereum tot statuas adeptum
fuisse, quot luces habet annus absolutus.
\lnr{35}Quare modus anni Graeci
fuit dierum \rnum{ccclx}.
\lnr{36}Non igitur fuit Lunaris.
\lnr{36}Laërtius de Solone
scribit: \textgreek{ἠξίωσέ τετοὺς Αθηναίοθς τὰς ἡμέρας καὶ σελένην ἄγειν[?]}.
\lnr{37}Ergo temporibus
Solonis nondum Graecorum annus erat Lunaris.
\lnr{38}Diodorus Siculus
libro \rnum{xiii}. 331.
 \textgreek{ἔφησεν εἰς οἰκίαν μετοίκου τινὸς ἑωρακέναι τῇ νουμηνίᾳ
περὶ μέσας νύκτας εἰσιόντας[?]}.
\lnr{40}Et infra. \textgreek{ἔφησε πρὸς τὸ τὴς σελήνης φῶς ἑωρακέναι[?]}.
\lnr{41}Quomodo poterat esse \textgreek{νουμηνία[?]},
 et media nocte luna lucere?
\lnr{41}Ergo menses
non erant Lunares.

% 24
% {PDF page nr}{source page nr}{line nr}
\plnr{107}{24}{1}Alioqui si annus Lunaris fuisset, quomodo constaret
id, quod scribit Plutarchus, scilicet defectionem Lunarem, quae
praecessit cladem Persarum ad Gaugamela, incidisse in noctem mysteriorum
Atticorum, hoc est \textgreek{εἰς εἰκάδα βονδρομιῶνος[?]}?
\lnr{4}Nam si vicesima Boedromionis
confectum est plenilunium, sane sexta, hoc est \textgreek{ἕκτῃ τοῦ ἱσταμένου[?]},
fuit novilunium.
\lnr{6}Non igitur Lunaris fuit ille Boedromion.
\lnr{6}Idem Plutarchus
in Camillo scribit, victoriam Atheniensium ad Naxum, duce
Chabria, contigisse \textgreek{βονδρομιῶνος πέμπτῃ φθίνοντος, ἐν πανσελήνῳ[?]}.
\lnr{8}Ergo
duodecima Boedromionis fuit novilunium.
\lnr{9}Thucydides ait eclipsim
Solis contigisse \textgreek{νουμηνίᾳ κατὰ σελήνην[?]}.
\lnr{10}Ergo fuit quaedam \textgreek{νουμηνία μὴ κατὰ σελήνην[?]}.
\lnr{11}Diodorus Siculus libro \rnum{xii} scribit Metonem astronomum
primum novilunium enneadecaeteridis suae statuisse \textgreek{Σκιῤῥοφοριῶνος
τρισκαιδεκάτῃ[?]}, ut manifesto neomenia illius Scirrhophorionis non fuerit
Lunaris.
\lnr{14}Annus igitur Graecorum Lunaribus mensibus descriptus non
fuit, quod putavit Gaza, neque \rnum{cccliiii} diebus,
 ut annus verus Lunaris,
sed \rnum{ccclx} finiebatur, ut iam probavimus.
\lnr{16}Minor autem fuit
anni illius modus Solari diebus quinque solidis cum quadrante, maior
Lunari diebus itidem quinque cum triente fere.
\lnr{18}Certis tamen temporum
epochis alligatos fuisse, neque temere in anteriora evagari solitos,
argumento est, quod iidem menses semper propemodum aestivi
fuerunt, ut Hecatombaeon, Metagitnion, Boedromion: iidam etiam
semper verni, Munychion, Thargelion, Scirrhphorion.
\lnr{22}Idem de aliis censendum.
\lnr{23}Ex Aristotele quoque et Theophrasto discimus
certis mensibus suas conversiones anni attributas fuisse,
 ut \textgreek{τροπὰς θερινὰς[?]}
Hecatombaeoni: \textgreek{χειμερινὰς[?]} Posideoni.
\lnr{25}Quod non fieret, nisi unico
intercalationis preasidio, et praeterea certis annorum periodis.
\lnr{26}Periodum
enim eam in hoc negotio vocamus, per quam rationes Solis
et Lunae pariant, et quae, ut ita dicam, utramque paginam facit: ut in
periodo Calippica rationes Solis cum Lunaribus aequantur.
\lnr{29}Aut sane
periodus ea est, quae nihil reliquum de ratiuncula facit: ut in ratione
bisexti sit.
\lnr{31}Nam tum nihil de quadrantibus diei relinquitur.
\lnr{31}Propterea
quadriennium bisexti Iuliani est periodus rationis Solaris.
\lnr{32}Quae vero, et cuiusmodi fuerit Grecorum periodus, et quot annorum,
operae pretium fuerit scire, si quidem veram huius praestantissimae rei
doctrinam consequi volumus.
\lnr{35}Et sane si qua fuit periodus, quae anni
Graeci fines tuertur, ut quidem aliquam fuisse necessario statuendum
est, aut Tetraetiris fuit, qualis Olympica, aut nulla.
\lnr{37}Quae enim ratio
fuerit Olympici quadriennii, si in ea nulla temporum
 \textgreek{ἀποκατάστασις[?]} fieret,
non video;
\lnr{39}praesertim, cum annus ille Graecorum Solaris non esset,
neque in quadriennii legem cogeretur, nisi ita ratione periodici temporis
postulante.
\lnr{41}Sed initium et caput Olympiadis consideremus.

% 25
% {PDF page nr}{source page nr}{line nr}
\plnr{108}{25}{1}Facile finem assequemur.
% 'finem' or 'sinem'? "finem", because 'f' and 'i' are attached to each other.
\lnr{1}Eius mensis primus habebat novilunium verum:
et in quintadecima eiusdem, confecto plenilunio, Olympicum ludicrum
peragebatur.
\lnr{3}Hoc nos docet Pindarus in Olympionicis:
\begin{verse}
---\textgreek{ἤδη γὰρ ἀυτῷ[?]}\\
\textgreek{Πατρὶ μὲν βωμῶν ἁγισθέντων[?]}\\
\textgreek{Διχόμηνις ὄλον χρυσάρματος[?]}\\
\textgreek{Εσπέρας ὀφθαλμὸν ἀντέφλεξε μήνα[?]}.
\end{verse}
% Pindar, "Olympian" 3, lines 19-20
\lnr{8}Ibi scholion vetus:
 \textgreek{ἐπεὶ ἐν τῇ πανσελένῳ ὁ ὁλυμπικὸς ἀγὼν ἄγεται, καὶ τῇ ἑκκαιδεκάτῃ
τὴς σελήνης ἄγεται κρίσις[?]}.
\lnr{9}Et in eodem opere: \textgreek{ἔν δ᾽ ἕσπερον ἔφλεξεν
ἐυώπιδος σελάνας ἐρατὸν φῶς[?]}.
\lnr{10}Manifesto declarat caput primi mensis
Olympiadici in verum novilunium inurrere, et ex omnibus mensibus,
quos \textgreek{τριακονθημέρους[?]} fuisse iam docuimus, unicum primum merum
Lunarem fuisse.
\lnr{13}Reliquos Lunares esse non potuisse, palam est, cum illi
quidem pleni omnes fuerint: Lunares vero sint alternis pleni et cavi.
\lnr{14}Investiganda
igitur ratio et methodus, qua \rnum{xlviii} mensibus vertentibus,
qui omnes pleni essent, caput quadragesimi noni in novilunium
incideret.
\lnr{17}Ut hoc clarius intelligatur, ita proponemus exemplum familiare
Computi nostri, ut vulgus loquitur.
\lnr{18}Hoc saeculo, in anno Iuliano
novilunium conficitur in \rnum{xix} Iulii, anno cycli Lunaris sexto.
\lnr{19}Anno vero
eiusdem cycli \rnum{x}, novilunium confit in quinta Iulii, post quatuor
nempe annos exactos ab priore illo novilunio.
% à -> ab
\lnr{21}Et quia annus Graecus
constabat 360 diebus tantummodo, anni quatuor erunt tantum 1440
dierum.
\lnr{23}Incipiat igitur primus annus ab novilunio, quod incidit in \rnum{xix}
% à -> ab
Iulii, feria prima, cyclo Solis, exempli gratia,
 \rnum{xxv}, cum litera dominicalis
est D.
\lnr{25}Sane quintus annus incipiet ab \rnum{xxviii} Iunii,
 feria sexta, cyclo Solis
% à -> ab
primo, cum litera dominicalis est F.
\lnr{26}Nam 1440 dies per \rnum{vii} divisi,
relinquunt feriam quintam, quae cum feria prima primi anni composita
faciet feriam sextam, characterem anni quinti, ut diximus, in
 \rnum{xxviii} Iunii.
\lnr{29}A qua die ad \rnum{v} Iulii, ubi est novilunium posterius,
 intervallum est
dies \rnum{vii}: qui sane adiiciendi sunt ad 1440 dies: et ita
 Tetraeteris Graeca,
sive Olympiaca, erit dierum 1447.
\lnr{31}Quid igitur fiet illis septem diebus
otiosis?
\lnr{32}An in exitu Tetraeteridis intercalabuntur?
\lnr{32}Minime.
\lnr{32}Sed conditor
Tetraeteridis Graecae duos dies in fine annorum singulorum appendices
reliquit, qui in quatuor annis octo facti extra numerum mensium
erant, et tetraeteridem cum Lunae rationibus conciliabant.
\lnr{35}Sed tamen
illi dies adiectitii in tetraeteride attica otiosi non erant.
\lnr{36}Decem enim
tribus Attica habebat, quae \textgreek{φυλαὶ[?]} dicuntur: ex quibus
 \textgreek{ὁι πεντακέσιοι[?]}
quotannis creabantur, quinquaginta nempe ex una quaque tribu.
\lnr{38}Singuli
vero quinquaginta ex illis Quingentis per cicuitum diem unum
imperabant, ita ut summa res ad eos deferretur.
\lnr{40}Sic fiebat, ut singuli
quinquaginta anno vertente \rnum{xxxvi} dies imperarent.
\lnr{41}Hi dies dicebantur
\textgreek{πρυτανεία[?]} illius, et illius Tribus, sive \textgreek{φυλῆς[?]}.

% 26
% {PDF page nr}{source page nr}{line nr}
\plnr{109}{26}{1}Harpocratio: \textgreek{ἔστι δὲ ἀριθμὸς
ἡμερῶν ἡ προυτανεία ἠτοι τριάκοντα ἓξ, ἣ τριάκοντα πέντε[?]}.
\lnr{2}\textgreek{καὶ ἑκάστε φυλὴ πρυτανέυει[?]}.
\lnr{3}\textgreek{διείλεκται δὲ περὶ τούτων Αριστοτέλης ἐν τῇ Αθηναίων πολιτείᾳ[?]}.
\lnr{3}Quod vero
ait etiam \rnum{xxxv} dies dici \textgreek{πρυτανείαν[?]}, id infra explicatur.
\lnr{4}Quot igitur
erant \textgreek{φυλαὶ[?]}, toties \rnum{xxxvi} imperabant.
\lnr{5}Ammonius vetustissimus et eruditissimus
% ...tiss. -> ...tissimus (2x)
Grammaticus.
\lnr{6}\textgreek{πρυτανεία δὲ θηλυκῶς, ὁ χρόνος.}
\lnr{6}\textgreek{διῄρητο γὰρ παῤ Αθηναίοις
ὁ ἐνιαυτὸς εἰς δέκα πρυτανείας, ὅσαι καὶ φυλαὶ ἦςαν.[?]}
\textgreek{καὶ ἐπρυτάνευεν ἑκάστη φυλὴ
κατ᾽ ἐνιαυτὸν ἅπαξ[?]}.
\textgreek{ὅθέν καὶ τοὺς μισθοὺς, καὶ τὰ ἐνοίκια, καὶ τὰς πρυτανείας κατὰ
μῆνα ἐτέλουν[?]}.
\lnr{9}Erant autem decem.
\lnr{9}Ergo decies triginta sex dies imperabant,
qui est modus anni terrae Atticae, atque adeo totius Graeciae.
\lnr{10}Illae
vero dies duae in calcem anni reiectae dicebantur \textgreek{ὑπερβαίνουσαι[?]},
 vel \textgreek{ὑπερβάλλουσαι
ἡμέραι[?]}: in quas etiam comitia creandorum magnistratuum
 differebantur.
\lnr{13}Itaque eo nomine illud biduum vocabatur \textgreek{ἀρχαιρεσίαι[?]}, vel
\textgreek{ἄναρχοι ἡμέραι[?]}: quia scilicet per illud biduum
 Attica esset sine legitimis
magistratibus.
\lnr{15}Atqui videmur nostri obliti.
\lnr{15}Binos enim dies contulimus
in exitum anni, qui in fine Tetraeteridis fiunt octo: cum tamen
dixerimus excessum Lunae supra annum Atticum fuisse dierum septem
dumtaxat.
\lnr{18}Unus igitur dies abundat non quidem Lunae supra annum,
sed anni supra Lunam.
\lnr{19}Huic rei remedium excogitatum esse respondemus
\textgreek{τὴν ἐξαίρεσιν[?]}, quae quidem non fiebat in illis bidui appendicibus,
quos dies Comitiales vocatos esse diximus, sed in alio mense.
\lnr{21}Propterea
dies, quae eximebatur, vocata \textgreek{ἐξαιρέσιμος ἡμέρα[?]}.
\lnr{22}Cicero in Verrem: \textit{Est
consuetudo Siculorum, caterorumque Graecorum, quod suos dies mensesque
congruere volunt cum Solis Lunaeque rationibus, ut nonnunquam, siquid
discrepet, eximant unum aliquem diem, aut summum, biduum ex
mense, quos illi exaeresimos dies nominant.}
\lnr{26}\textit{Item nonnunquam uno die longiorem
mensem faciunt, aut biduo}.
\lnr{27}Haec Marcus Tullius.
\lnr{27}Quae tum demum
locum habent, si \textgreek{ἐξαίρεσις[?]} fit in mense eodem,
 qui habet biduum illud
adiectitium.
\lnr{29}Nam si mensis idem habet \textgreek{ἀνάρχους ἡμέρας[?]}, et in eodem
fiat \textgreek{ἐξαίρεσις[?]}, in omni Tetraetiride mensis ultimus quatri anni
 habebit
unum et triginta dies tantum: cum tamen in aliis annis idem mensis habuerit
triginta et duos.
\lnr{32}Sed cum sit \textgreek{ἐξαίρεσις[?]}, habet tantum \rnum{xxxi}.
\lnr{32}Atqui
in anno Attico alii mensi competebat \textgreek{ἐξαίρεσις[?]}, alii
 \textgreek{ἄναρχοι ἡμέραι[?]}, ut postea
suo loco delarabitur.
\lnr{34}Videmus tamen Ciceronem velle in Tetraeride
Syracusana ultimo mensi competere
 \textgreek{και τὴν ἐξαίρεσιν, και τὰς ἀνάρχους
ἡμέρας[?]}.
\lnr{36}Ultimo enim mensi deputabantur illae
 \textgreek{ὑπερβάλλουσαι ἡμέραι[?]},
ut auctor est apud Macrobium Glaucippus, qui de sacris Atheniensium
scripserat.
\lnr{38}Ex hac exemtione contingebat annum quartum Tetraeteridis
esse tantum dierum \rnum{ccclix}, praeter \textgreek{ἀνάρχους ἡμέρας[?]}.
\lnr{39}Itaque in anno
\textgreek{ὲξαίρεσιμαίῳ[?]} una tribus \rnum{xxxv} tantum dies imperabat.
\lnr{40}Hoc plane est,
quod dixit Harpocratio, \textgreek{πρυτανείαν[?]} esse tantum
 dierum triginta sex, aut
triginta quinque.

% 27
% {PDF page nr}{source page nr}{line nr}
\plnr{110}{27}{1}Hoc modo fiebat, ut quarto quoque anno exacto novilunium
in caput anni quinti praecise incideret: quod necessario observabatur
in Tetraeteride Elidensium, quam pleraque omnis Graecia,
et Latium Olympiadem vocarunt.
\lnr{4}Quid est Tetraeteris Graeca?
\lnr{4}Est
intervallum quatuor annorum Graecorum inter duas coniunctiones sive
syzygias Lunares interiectum.
\lnr{6}Nam primus tantum mensis Lunaris
erat, et secundi \textgreek{πρώτη ἱσταμένου[?]} habebat merum novilunium.
\lnr{7}In reliquis
claudicabat ratio Lunae, quod priorem epocham Luna anteverteret
diei semisse, et aliquot scrupulis praeterea.
\lnr{9}Considerandae igitur sunt
variantiae noviluniorum in mensibus solidis.
\lnr{10}Ex his primum colligitur
syzigiam Graecorum, esse dierum \rnum{xxix}.
 hor. \rnum{xii}. scrup. \rnum{xxx}. annum
vero Lunarem dierum 354. horarum 6.
\lnr{12}Et quatuor tales annos dierum
1417.
\lnr{13}Et cum embolimo Lunari, dierum 1447.
\lnr{13}Deductis igitur
diebus 29, horis 12, 30.' de mense solido, hoc est de diebus \rnum{xxx}.
supersunt dies 0. hor. 11. scrup. 30.'
\lnr{15}Porro prior annus excedit sequentem
biduo propter \textgreek{ἀνάρχους[?]},
 sive \textgreek{ὑπερβαλλούσας ἡμέρας[?]}.
\lnr{16}Verbi gratia:
duodecimus mensis anni primi habet terminum novilunii in \rnum{xxvi}
die.
\lnr{18}Qui cum habeat appendices \textgreek{τὰς ὑπερβαλλούσας δύο [?]},
 aufert ab primo
% à -> ab
mense secundi anni biduum praeter semissem diei, quo novilunium
mensis sequentis antevertit novilunium prioris.
\lnr{20}Itaque mensis primus
secundi anni propter \textgreek{ὑπερβαλλούσας[?]} proxime
 \textgreek{καὶ ἀμέσως[?]} praecedentes, et unum
fere diem, quo novilunia priora ab frequentibus antevertuntur, habebit
% à -> ab
novilunium triduo citius, quam duodecimus mensis prioris anni.
\lnr{24}Et sic duedecimus secundi detrahit biduum primo tertii, et duodecimus
tertii primo quarti.
\lnr{25}Quo nomine confecimus tibi Tabellam, ut
quota die mensis novilunium fieret, possis citra laborem assequi.
%% Insert table "Novilunia in mensibus Tetraeteridis Graecae"
\begin{table}[htbp]
  % Version of novilunia table with horizontal layout
%% Liber Primus, p.27, PDF 110
%%
%%% Count out columns for fixed-width source font
% 000000011111111112222222222333333333344444444445555555555666666666677777777778
% 345678901234567890123456789012345678901234567890123456789012345678901234567890
%
{
\tabnums % Select monospaced numbers
%% Select a general font size (uncomment one from the list)
%\tiny
%\scriptsize
%\footnotesize
%\small
\normalsize
%% Center the whole table left-right
\centering
%% Modify separation between columns
%\setlength{\tabcolsep}{0.5em}
%% Modify distance between rows
%\renewcommand{\arraystretch}{0.85}
%%
\begin{tabular}{@{} l *{12}{r} @{}}
\toprule
\multicolumn{13}{ c }{\Large\textsc{Novilunia in mensibus}} \\
\multicolumn{13}{ c }{\Large\textsc{Tetraeteridis Graecae}} \\
\toprule
~ &
\multicolumn{12}{ c }{Mensium}
\\
\cmidrule(l){2-13}
Annus &
1 & 2 & 3 & 4 & 5 & 6 & 7 & 8 & 9 & 10 & 11 & 12
\\
\midrule
Primus &
1 & 1\altsep{}30
        & 30 & 29 & 29 & 28 & 28 & 27 & 27 & 26 & 26 & 25 
\\
Secundus &
23 & 22 & 22 &21  &21  & 20 & 20 & 19 & 19 & 18 & 18 & 17
\\
Tertius &
15 & 14 & 14 & 13 & 13 & 12 & 12 & 11 & 11 & 10 & 10 & 9
\\
Quartus &
7 & 6 & 6 & 5 & 5 & 4 & 4 & 3 & 3 & 2 & 2 & 2
\\
\bottomrule
\end{tabular}
%
\caption{Novilunia in mensibus Tetraetirides Graecae}
\label{tab:p027}
}

\end{table}

\lnr{26}Quia
igitur in diebus 1417. sunt 48 syzygiae, necessario
ultima erit triginta dierum.
\lnr{28}Rursus quia annus
Graecus maior est Lunari octo diebus, necessario
in primo anno erunt tredecim neomeniae.
\lnr{30}Nam
praeter 354 dies, qui continent duodecim menses
alternis plenos et cavos, octo dies supersunt, ita ut
in primo die oporteat fieri novilunium.
\lnr{33}Quia veros
% Orig: "verò". "veros" (accusative) or "verorum" (genitive)
illi dies distributi sunt per totos menses, omnino
secundus habebit duas neomenias, in ipsa silicet
\textgreek{νεομηνία[?]}, et in \textgreek{τριακάδι[?]}: quae vere si qua alia,
\textgreek{ἔνη καὶ νέα [?]} vocari potest: ut vides in Tabella, ex qua
patet novilunium committi \textgreek{ἐν πρώτη ἱσταμένου[?]} primi
anni, \textgreek{ἐν ὀγδόῃ φθίνοντος[?]} secundi,
 \textgreek{ἐν πέμπτῃ μεσοῦντος[?]}
tertii, \textgreek{ἐν ἑβδόμῃ ἱσταμένου[?]} quarti, et rursus \textgreek{ἐν πρώτιῃ
ἱσταμένου[?]}
quinti, ut ab initio.
\lnr{41}Hae sunt rationes Tetraeteridis Graecae,
quae Elidensibus Olympias, Phocensibus Pythias, vocabatur.

% 28
% {PDF page nr}{source page nr}{line nr}
\plnr{111}{28}{1}Quod ne veteres quidem explicarunt.
\lnr{1}Nam qui Olympiadem meram
Lunarem constituunt, fortasse venia digni sunt, cum Censorinus ob
diem ex quadrantibus quarto quoque anno excrescentem, quod Romani
bisextum vocant eam institutam scribat:
\lnr{4}quasi vero Olympias,
quam ex duabus Trieteridibus Lunaribus constare facit, etiam Solaris
fuerit.
\lnr{6}Quod perspicue falsum est.
\lnr{6}Nam una Olympias sola nunquam
cum Solis rationibus congruit, sed potius duae: quae quidem iunctae
octaeterida constituunt.
\lnr{8}De qua nunc dicendum.

\section{De Octaeteride}
\lnr{9}Qui primus Tetraeterida instituit, hoc unum duntaxat spectavit,
ut primus eius mensis incideret in verum mensem Lunarem,
propter Olympicum agonem, qui semper fiebat in \rnum{xv}
mensis, plenilunio.
\lnr{12}Sed hoc illi ad perfectionem deerat, nempe ut etiam
cum Sole congrueret.
\lnr{13}Atqui id intra Tetraeterida contingere non potest,
ut quae dierum sit duntaxat 1447: accedente nimirum mense embolimo.
% ':' at the end?
\lnr{15}Solaris vero Tetraetiris sit dierum 1461.
\lnr{15}Differentia dies \rnum{xiiii}.
\lnr{16}Esto novilunium in \rnum{xx} Martii, quando epactae sunt \rnum{xi}.
\lnr{16}Post quartum
annum epactae erunt \rnum{xxv}.
\lnr{17}Novilunium \rnum{vi} Martii.
\lnr{17}Differentia
utriusque novilunii, dies \rnum{xiiii}.
\lnr{18}Ut autem duae Tetraeterides ad pristinam
epocham revocentur, opus est intercalatione.
\lnr{19}Nam si, ut iam
statuimus, primi anni epactae fuerint \rnum{xi},
 noni anni epactae erunt \rnum{ix},
quae convenient vicesimae secundae Martii.
\lnr{21}Ita interventu embolismi
annus summovebitur in pristinum situm.
\lnr{22}Nam absque embolismo
foret, neomenia noni anni deprehenderetur in \rnum{xix} Februarii;
\lnr{23}Ita fere
semper mensem civilem plenum alternis Tetraeteridibus intercalabant.
\lnr{25}Quod intervallum nos quidem Octaeterida vocavimus.
\lnr{25}Nam
certum est, nullam eiusmodi aequabilium annorum Octaeterida institutam
ante Cleostratum, qui primus omnium aequabilem Octaeterida
cum Lunari comparavit:
\lnr{28}
Et postea Calippus decem et novem
Tetraeteridas, quae quidem sunt anni 76, cum totidem annis Lunaribus,
quorum viginti octo erant embolimaei, comparavit:
\lnr{30}Idque
tempus iustam anni et Lunae periodum esse iudicavit, ut suo loco dicetur.

\section{De Mensibus Atticis}
\lnr{33}Antequam annorum Graecorum, praesertim Attici, methodum
aperiamus, quam ne in somnis quidem odoratus est Gaza,
illud merito in dubium vocari possit, an recte habeat ordo mensium
Atticorum ita, ut ab ipso doctissimo Gaza proditus est.

% 29
% {PDF page nr}{source page nr}{line nr}
\plnr{112}{29}{2}Primum enim
illud omnino falsum est, quod ipse multum aestuans nobis presuadere
conatur, Anthesterionem secundum mensem Autumni fuisse, ut Maemacterionem
primum eiusdem quadrantis autumnalis.
\lnr{5}Utrumque
enim falsum convincit priscorum scriptorum auctoritas, et quod Anthesterion
secundus autumnalis quadrantis fuerit, et quo Maemacterion
primus.
\lnr{8}Nam de Anthesterione multa adversantur.
\lnr{8}Sed antequam de Anthesterione loquamur, locus postulat, ut quot
 \textgreek{Διονύσια[?]}
Athenis fuerint, dicamus: quia ad hanc rem nonnihil faciunt, et lectori
imperito imponere possunt.
\lnr{11}Trina igitur fuere Liberalia Athenis.
\lnr{12}Prima, mense Posideone, quae dicta \textgreek{Διανύσια κατ᾽ ἀγροὺς[?]},
 aliter \textgreek{Λήναια[?]}.
\lnr{13}Secunda \textgreek{διονύσια τὰ κατ᾽ ἄστυ[?]}, mense Elaphebolione.
\lnr{13}Hesychius: \textgreek{Διονύσια
ἑορτὴ Αθήνῃσιν, ἣ Διονύσῳ ἤγετο τὰ μὲν κατ᾽ ἀγροὺς μηνὸς Ποσειδεῶνοσ (τὸ
δὲ πάλαι Ληναιῶνος) τὰ δὲ ἐν ἄστει, μηνὸς Ελαφεβολιῶνος[?]}.
\lnr{15}Scribit enim
Thucydides \textgreek{αὗται ἁι ςπονδαὶ ἐγένοντο τελευτῶντος τοῦ χειμῶνος,
 ἄμαἦρι, ἐκ
Διονυσίων ἐυθὺς τῶν ἀστικῶν, ἀιτοδεκαετῶν διελθόντων,
 καὶ ἡμερῶν παρενεγκουσῶν
Πλειστόλα Σπάρτης ἐφορέυοντος, Αρτεμισίου τετράδι φθίνοντος, Αλκαίου
δ᾽ Αθήνῃσιν ἄρχοντος, ἐλαφηβολιῶνος ἕκτῃ φθίνοντος[?]}.
\lnr{20}De his manifesto sentit Seneca his Anapaestis:
\begin{verse}
  \textit{Nos Cadmeis orgia ferre\\
  Tecum solitae condita cistis,\\
  Cum iam pulso sidere brumae\\
  Tertia soles evocat aestas:\\
  Et spiciferae concessa Deae\\
  Attica Mystas claudit Eleusin.}
\end{verse}
% L. Annaeus Seneca iunior, Hercules Oetaeus, line 594-599
% From "Seneca's Tragedies. With an English translation by Frank Justus Miller"
% Volume II (Agamemnon, Thyestes, Hercules Oetaeus, Phoenissae, Octavia)
% page 235:
% "To bear the mysteries 
% (sacred objects used in the orgiastic worship of Bacchus)
% in Theban (called Cadmaean here, after the founder of Thebes) baskets hidden,
% when now the wintry star had fled
% and each third summer
% (the festival of Bachhus was celebrated every third year)
% called forth the sun,
% and when grain-giving godess'  (Ceres) sacred seat
% Attic Eleuis, shut in her mystic worshippers."
% (The reference is to the Eleusinian mysteries. All these festivals these
% women hd been wont to attend together in childhood.)
\lnr{27}Quibus notatur tempus fuisse circa finem hiemis, et praeterea
 \textgreek{τριετηρικὰ[?]}
fuisse: quae et dicebantur minora mysteria.
\lnr{28}Nam maiora mysteria
erant Pentaeterica: de quibus idem Seneca Hercule Furente.
\begin{verse}
  \textit{Quantus Eleum coit ad Tonantem,\\
  Quinta cum sacrum revocavit aestas;\\
  Quanta, cum longae redit hora noctis,\\
  Crescere et somnos cupiens quietos\\
  Libra Phoebeos tenet aequa currus,\\
  Turba secretam Cererem frequentat.\\
  Et citi tectis properant relictis\\
  Attici noctem celebrare Mystae.}
\end{verse}
% Seneca: Hercules Furens, line 840-847
% From "Seneca's Tragedies. With an English translation by Frank Justus Miller"
% Volume I (Hercules Furens, Troades, Medea, Hippolytus, Oedipus)
% page 75:
% Preceded by:
% (838) "Quantus incedit populus per urbes"
% (839) "ad novi ludos avidus theatri"
% Continues with the above quote (with slight differences):
% (840) "Quantus Eleum ruit ad Tonatem" (probable print error in ruit -> coit)
% (843) "Quanta, cum longae redit hora nocti" (nocti -> noctis)
% Translation:
% (838) "Great as the host that moves through city streets,
% eager to see the spectacle in some new theater;"
% (840) "great as that which pours to the Elean¹ Thunderer,
% when the fifth summer has brought back the sacred games;
% great as the throng which (when the time comes
% again for night to lengthen and the balanced Scales²,
% yearning for quiet slumber, check Phoebus' car)
% surges to Ceres' secret rites, and the initiates of
% Attica, quitting their homes, swiftly hasten to celebrate
% their night."
% ¹) i.e. Olympian, The reference is to the Olympic games, celebrated in honour
%    of Zeus
% ²) See Index. Vol II, p 538: "SCALES (Libra), zodiacal constellation
%    marking the autumnal equinox, H. Fur. 842"
% Index Vol II, p 535: PHOEBUS, one of Appollo's names; most frequently
%  conceived of as the sun-god, driving his fiery chariot across the sky,
%  seeing all things, darkening his face or withdrawing from the sky at
%  sight of monstrous sin, lord of the changing seasons, etc.

\lnr{38}In Hyppolyto.
\begin{verse}
  \textit{Iam quarta Eleusin dona Triptolemi secat,\\
  Paremque toties libra composuit diem.}
\end{verse}
% Seneca: Phaedra (also known as Hippolytus), line 838-839
% From "Seneca's Tragedies. With an English translation by Frank Justus Miller"
% Volume I (Hercules Furens, Troades, Medea, Hippolytus, Oedipus)
% page 375:
% "Now for the forth time is Eleusis harvesting
% the bounty of Triptolemus,¹ as many times has
% Libra made day equal unto night, ..."
% ¹) Wheat: see Index s.v. "Triptilemus."
% Index, Vol II, p 541: TRIPTOLEMUS, son of the king of Eleusis, through
%  whom Ceres gave the arts of agriculture to mankind. Hip. 838

% 30
% {PDF page nr}{source page nr}{line nr}
\plnr{113}{30}{1}Nimirum fiebant \textgreek{τῆ εἰκάδι βοηδρομιῶνος[?]},
% Greek: the twentieth of Boedromion (3rd month in the Attic calendar)
% roughly around september/october
 post confectum sidus aequinoctii
autumnalis, quinto anno redeunte.
\lnr{2}Galenus loquens \textgreek{περὶ τοῦ τὴς ἐλάτης
σπέρηατος: ὂστις καιρὸς ἐν Πῶμῃ μὲν ὁ καλούμενος μὴν Σεπέμβριός ἐστιν, ἐν
Περγάμῳ δὲ παῤ ἡμῖν Υ῾περβερεταῖσο Α᾽θήνῃσι δὲ Μυστήρια.[?]}
\lnr{4}Philostr. v. 56. de
Baetica provincia: \textgreek{γεωργίας τε πάσης μεστὴν ἐῖναι, καὶ ὡρῶν, οἷαι τὴς Αττικῆς αἱ
μετόπωροί τε καὶ μυστηρυότιδες[?]}.
\lnr{6}\textgreek{μυστηεριώτιδας ὥρασ[?]} vocat autumnale tempus.
\lnr{7}Idem libro \rnum{iiii}. 46.
 \textgreek{ἐς δὲ τὸν Πειραιᾶ ἐσπλεύσας περὶ Μυστηρίων
ὥραν ἅτε Αθηναῖοι πολυδοθρωπότατα Ελλήνων πράττουσι,[?]}, et cetera.
\lnr{8}Postea \textgreek{ὧν
οἱ μέν Γυμνοὶ ἐθέροντο (καὶ γὰρ τὸ μετόπωρον ἐυήλιον τοῖς Αθηναίοις,)[?]}
 et cetera.
\lnr{9}Postea:
\textgreek{μυήσει δέμε ὁ δεῖνα (προγνῶσει χρώμενος,
 ἐςτὸν μετ᾽ ἐκεῖνον ἱεροφάντην, ὄς μετὰ
τέτταρα ἔτη τοῦ ἱεροῦ πρὄυστη.)[?]}
\lnr{11}Infra eodem anno, sequente vere: \textgreek{ἐπιπλῆξαι δὲ λέγεται
περὶ Διονυσίων ἀθηνάιοις, ἅ ποιεῖται σφίσιν ἐν ὤρᾳ τοῦ Ανθεστηριῶνος[?]}.
\lnr{12}Tertia
\textgreek{Διονύσια[?]} dicebantur \textgreek{ἀιθεστήρια[?]}.
\lnr{13}De quibus proverbium: \textgreek{θύραζε Κᾶρεσ[?]}.
\textgreek{οὐκ ἔτ᾽Ανθεστέρια[?]}.
\lnr{14}Hesychius, \textgreek{Ανθεστήρια, Διονύσια[?]}.
\lnr{14}Haec sunt, quae ad rem
pertinent; nimirum ita dicta, quod Anthesterione instaurarentur: et
dicebantur \textgreek{Διονύσια τὰ ἀρχαῖα[?]},
 item \textgreek{Διονύσια τὰ ἐν Λίμναις[?]}.
\lnr{16}Thucydides
libro \rnum{ii}. \textgreek{τὰ γὰρ ἱερὰ ἐν αὐτῇ τῇ ἀκροπόλει καὶ ἄλλων θεῶν ἐστὶ,
 καὶ τὰ ἔξω πρὸς
τοῦτο τὸ μέροσ τὴς πόλεως μᾶλλον ἵδρυται, τό, τε τοῦ Διὸς Ολυμπίου, καὶ τὸ Πύθιον,
καὶ τὸ τὴς Γῆς, καὶ τὸ ἐν Λίμναις Διονύσου, ᾧ τὰ ἀρχαιότερα Διονύσια τῇ δωδεκάτῃ
ποιεῖται ἐν μηνὶ Ανθεστηριῶνι[?]}.
\lnr{20}Fuisse autem annua docet Demosthenes
\textgreek{κατὰ Νεαίρας;
 Καὶ διὰ ταῦτα ἐν τῷ ἀρχαιοτάτῳ ἱερῷ τοῦ Διονύσου ἐν Λίμναις ἔστησαν,
ἵνα μὴ πολλοὶ εἰδῶσι τὰ γεγραμμένα[?]}.
\lnr{22}\textgreek{ἃπαξ γὰρ τοῦ ἐνιαυτοῦ ἑκάστου ἀνοίγεται,
τῇ δωδεκάτῃ τοῦ Ανθεστηριῶνος μηνός[?]}.
\lnr{23}Hunc igitur Anthesterionem, qui ab
Anthesteriis Dionysiis dictus, aio fuisse alienum ab Autumno, ubi
eum collacat Gaza.
\lnr{25}Neque est, cur adeo Philostrato iniquus sit, quod
Anthesterionem videatur in tempus veris reiicere.
\lnr{26}Neque enim solus
Philostratus hoc scripsit.
\lnr{27}Scribit idem et Appianus, cum dicit Caesarem
dictatorem \textgreek{μηνὸς Ανθεστηριῶνος[?]} ab coniuratorum fatione in senatu
% à -> ab
oppressum et caesum fuisse.
\lnr{29}Sed et Plutarchus gravissimus scriptor illi
obiiciendus, qui Athenas ab Sulla captas prodit mense Martio,eumque
% à -> ab
mensem \textgreek{ἀνθεστηριῶια[?]} Athenis dici.
\lnr{31}Idem etiam scriptor in Symposiacis
ita clare loquitur, ut per illum Gazae muto esse liceat. \textgreek{καὶ μὴν οἶνόν
γε νέον ὁι πρωϊαίτατα πίνοντες Ανθεστηριῶνι πίνουσι μηνὶ μετὰ χειμῶνα[?]}.
\lnr{33}\textgreek{καὶ τὴν ἡμέραν
ἐκείνην ἡμεῖς μὲν ἀγαθοῦ δαίμονος, Αθηναῖοι δὲ Πιθοιγίαν καλοῦσι[?]}.
\lnr{34}Mensem
\textgreek{Ανθεστηριῶνα μετὰ χειμῶνα[?]} collocat, non ante hiemem, ut Gaza.
\lnr{35}Cum
igitur alterum mensem ab Posideone Gamelionem Gaza ita, uti debuit,
collocaverit, sane tertius ab Posideone erit Anthesterion.
% à -> ab (2x)
\lnr{37}Proinde octavus
ab Hecatombaeone.
\lnr{38}Harpocratio: \textgreek{Ανθεστηριὼν ὄγδοος μὴν οὗτος παῤ
Αθηναίοις, ἱερὸς Διονύσου[?]}.
\lnr{39}\textgreek{Ιστρὸς δὲ ἐν τοῖς τὴς συναγωγῆς κεκλῆσθαί φησιν αὐτὸν
διὰ τὸ πλεῖστα τῶν ἐκ γῆς ἀνθεῖν τότε[?]}.
\lnr{40}Ne Gaza quidem, si reviviscat, dubitare
possit, Anthesterionem mensem vernalem fuisse.
\lnr{41}Demosthenes
\textgreek{περὶ στεφάνου.}[?].

% 31
% {PDF page nr}{source page nr}{line nr}
\plnr{114}{31}{1}\textgreek{ἐπὶ ἱερέως Κλιναγόρου ἐαρινῆς
 Πυλαίας ἔδοξε τοῖς Πυλαγόραις[?]}
\lnr{2}Subiicit Demothenes:
 \textgreek{Λέγε δὴ οὖὴ[?] χρόνους, ἐν οἷς ταῦτα ἐγένετο}[?].
\lnr{2}\textgreek{Χρόνοι}[?].
\lnr{2}\textgreek{Ἄρχων
Μνησιθείδης, μηνὸς Ανθεστηριῶνος ἕκτῃ ἐπὶδεκάτῃ[?]}.
\lnr{3}Vides \textgreek{ἐαρινὴν πυλαίαν[?]}
mense Anthesterione.
\lnr{4}Quod non fieret, nisi Anthesterion fuisset \textgreek{ἐαρινὸς
μήν[?]}.
\lnr{5}Unum locum producam, qui omnium instar esse poterat.
\lnr{5}Anno
Nabonassari 465, Athyr die 29, \textgreek{ἀνθεστηριῶνος ὀγδόῃ[?]},
 qui est Calippicus
Anthesterion, Timocharis observavit mediam partem Lunae in
medium Pleiadum inductam.
\lnr{8}Tempus congruit Ianuarii vicesimae
nonae.
\lnr{9}Itague neomenia Anthesterionis Calippici illius anni incidit in
\rnum{xxiii} Ianuarii, et sequenti anno post embolismum eadem
 neomenia incurrit
in \rnum{x} Februarii.
\lnr{11}Et quid tergiversamur?
\lnr{11}Annus ille erat quadragesimus
septimus primae periodi Calippicae, cuius Hecatombaeon caepit
tricesima Iunii, cyclo lunae tertio.
\lnr{13}A quo capite ad 23 Ianuarii sunt dies
206 praecise, quae sunt syzygiae septem.
\lnr{14}Ergo Anthesterion est octava
syzygia ab Hecatombaeone.
\lnr{15}Et proinde secundus mensis hibernus.
\lnr{15}Athenaeus
libro \rnum{viii} \textgreek{Ανθεστηριῶνα, καὶ Ελαφηβολιῶνα[?]} coniungit.
\lnr{16}\textgreek{Κατὰ δὲ τὸν Ανθεστηριῶνα, καὶ Ελαφηβολιῶνα λέγουσιν
 ὁι ἐπιχώριοι, ὅτι ἀποπέμπει Βολύη τὴν
Απόπυριν Ολύνθῳ[?]}.
\lnr{18}Expungatur igitur Anthesterion ex Autumno, quo
illum traduxit Gaza, vir alioqui maximus.
\lnr{19}De Maemacterione pene
% pene: clearer in other copies
persuaserat mihi, post Boedromionem locandum esse.
\lnr{20}Gazae enim
quartus ab Hecatombaeone est Maemacterion, quintus Pyanepsio, sextus
Anthesterion.
\lnr{22}Sed ut hoc penitus negem, facit primum Plutarchus,
qui in Caesare scribit Posideonem esse Ianuarium, quod confirmatur
verbis Anacreontis apud Eustathium:
\begin{verse}
  \lnr{25}\textgreek{Μεὶς μὲν δὴ Ποσειδηΐων ἕστηκε[?]}.\\
  \textgreek{Νεφέλαι δ᾽ ὕδαρι βαρύνονται[?]}.\\
  \textgreek{Α῎γριοι δὲ χειμῶνες παταγοῦσι[?]}.
\end{verse}
\lnr{28}Eiusmodi tempestates in Graecia non occurrunt, nisi confecto brumae
sidere.
\lnr{29}Idem Plutarchus \textgreek{περὶ ἴσιδος[?]}
 afferit Pyanepsionem esse Athyr
Augustalem, hoc est Novembrem.
\lnr{30}Et in Demosthene hos menses
continuat \textgreek{Μεταγειτνιῶνα, Βονδρομιῶνα, Πυανεψιῶνα, ὀυ μὴν[?]},
 inquit, \textgreek{ἐπὶ πολὺν
χρόνον ἀπέλαυσε τὴς πατρίδος κατελθών[?]}.
\lnr{32}\textgreek{ἀλλὰ ταχὺ τῶν ἑλληνικῶν πραγμάτων
συντριβέντων, μεταγειτνιῶνος μὲν ἡ περὶ Κρανῶνα μάχη συνέπεσε, Βονδρομιῶνοσ
δὲ παρῆλθεν εἰς Μουνυχίαν ἡ φρουρά[?]}.
\lnr{34}\textgreek{Πυανεψιῶνος δὲ Δεμοσθένης ἀπέθανε[?]}.
\lnr{35}Quod si Posideon est Ianuarius, (ut sane omnino verum est, cum Posideon
semper post brumam inciperet,) Pyanepsion autem est November,
sane Decembri quis mensis congruere debeat, praeter Maemacterionem,
non video.
\lnr{38}Nam inter \textgreek{Πυανεφιῶνα, καὶ τροπὰς χειμερινὰς[?]}, id est, inter
\textgreek{Πυανεφιῶνα, καὶ Πορειδεῶνα[?]},
 spatium aliquod interesse innuit Theopharstus
\textgreek{καὶ κοκκυμηλέας Αιγυπτίας[?]} differens:
 ut scias continuatos non fuisse.
\lnr{41}\textgreek{Αρχεται δ᾽ ἀνθεῖν μηνὸς Πυανεφιῶνος, τὸν δὲ καρπὸν πεπαίνει περὶ τροπὰς χειμεδινάς[?]}.

% 32
% {PDF page nr}{source page nr}{line nr}
\plnr{115}{32}{1}Sane inter \textgreek{ἄιθησιν καὶ πέπανσιν[?]},
 mensem interesse omnes concedent,
id est, inter Pyanepsionem, et Posideonem.
\lnr{2}Et quis mensis intercedet,
praeter Maemacterionem?
\lnr{3}Praeterea idem Plutarchus vocat
Pyanepsionem \textgreek{σπόριμον μῆνα[?]}, et circa Vergiliarum occasum collat,
cui adstipulatur interpres Aristophanis, qui ait rusticos in Attica
 \textgreek{τὴν
προηρόσιον θυσίαν[?]}
eo mense immolare solitos.
\lnr{6}Quod si ante arationem
et fationem incipit Pyanepsion, manifestum est, eum mensem ab Decembri
% à -> ab
abfuisse, et in Novembrem convenisse.
\lnr{8}Diodorus Siculus libro
\rnum{iii} continuat hos duos menses, Maemacterionem et Posideonem.
\lnr{10}\textgreek{ἀπὸ γὰρ μηνὸς, ὃν καλοῦσιν Αθηναῖοι Μαιμακτηριῶνα,
 τῶν ἑπτὰ τῶν κατὰ τὴν
ἄρκτον ἀστέρων ὀυδένα φασὶν ὁρᾶσθαι μέχρι τὴς πρώτης φυλακῆς[?]}.
\lnr{11}\textgreek{τῷ δὲ Ποσειδεῶνι
μέχρι δευτέρας[?]}.
\lnr{12}Et dubitamus adhuc?
\lnr{12}Harpocratio ita scribit: \textgreek{μαιμακτηριὼν
ὁ πέμπτος μὴν παρ᾽ Αθηναίοις[?]}.
\lnr{13}Si Boedromion est tertius ab Hecatombaeone,
Posideon autem sextus, quartus inter Boedromionem
et Maemacterionem erit Pyanepsion.
\lnr{15}Ac ne quis errorem librarii
putet apud Harpocrationem, subiicit:
 \textgreek{ὠνόμασται δὲ ἀπὸ διὸς μαιμάκτου[?]}.
\lnr{17}\textgreek{μαιμάκτης δὲ ἐστὶν ὁ ἐνθουσιώδησ καὶ ταρακτικὸς,
 ὥς φησι Λυσιμαχίδης
ἐν τῷ περὶ τῶν Αθήνῃσι μηνῶν[?]}.
\lnr{18}\textgreek{ἀρχὴν δὲ λαμβάνοντος τοῦ χειμῶνος ἐν τούτῳ
τῷ μηνὶ, ὁ ἀὴρ ταράττεται καὶ μεταβολὴν ἔχει[?]}.
\lnr{19}Autumno praecipitato sub
ipsum initium hiemis ponit Maemacterionem, quem in locum Pyanepsionem
confert Gaza.
\lnr{21}Item Demosthenes Olynth. \rnum{iii}: \textgreek{τότε τοίνην
μὴν ἦν Μαιμακτηριὼν[?]}.
\lnr{22}Ulpianus in eum locum: \textgreek{χειμέριοσ οὗτος ὁ
μήν[?]}, et cetera.
\lnr{23}Sed propter nomen et auctoritatem viri vix ab hominibus
etiam doctis expressero, ut Gazam hic errasse fateantur, nisi
illis testem locupletissimum obiecero.
\lnr{25}Timocharis igitur apud Ptolemaeum
anno Nabonassari 466, qui erat 48 Calippi, Thoth \rnum{vii},
\textgreek{πυανεψιῶνος \gnum{ϛ} τελευτῶντος[?]},
 observavit Lunam coniunctam Spicae Virginis.
\lnr{28}Quod tempus convenit diei octavae Novembris.
\lnr{28}Proinde Neomenia
Pyanepsionis Calippici \rnum{xvi} Octobris.
\lnr{29}Hecatombaeon autem
illius anni coepit \rnum{xviiii} Iulii.
\lnr{30}A \rnum{xviiii} Iulii, ad \rnum{xvi} Octobris, sunt
dies nonaginta: qui constituunt menses Lunares Calippi tres praeteritos,
et neomeniam quarti ineuntis.
\lnr{32}Nam 89 dies sunt menses tres
Lunares, quibus si adieceris neomeniam quarti mensis, fiunt dies 90.
\lnr{34}Ergo neomenia Pyanepsionis erat quarta ab neomenia Hecatombaeonis
% à -> ab
Antecedit igitur Maemacterionem Pyanepsio, et Posideonem
Maemacterio: Gamelionem Posideo, et Anthesterionem Gamelio.
\lnr{37}Reliquorum mensium ordo recte servatur ab Gaza, quod non operosum
% à -> ab
erat: cum uno aut altero priscorum Graecorum testimonio
id colligi possit.
\lnr{39}Quare per quadrantes anni ita, ut tempora putabant
Attici, digerendi sunt menses hoc modo.
\lnr{40}Male igitur menses
autumnales et hiberni ab Gaza digesti erant.
% à -> ab

% 33
% {PDF page nr}{source page nr}{line nr}
\plnr{116}{33}{1}Quod in hoc mensium
Laterculo perspicere potes.

%% Table: Laterculum mensium Atticorum secundum anni quadrantes.
\begin{table}[htbp]
%%% Laterculum mensium Atticorum secundam anni quadrantes
%%% Liber I p33
%%
%% Table more horizontally spread out to make it look better without
%% text wrapping.
%
%% Names copied from Wikipedia: Attic calendar
%% then modified to match the original
%% - declension of quarter names θέρος -> θΕΡΙΝΟΙ
%% - no accents on the capitals
%% - Autumn: Φθινόπωρον -> ΟΠΩΡΙΝΟΙ
%% - Μουνιχιών -> Μουνυχιών (ι -> υ)
%% - Σκιροφοριών -> Σκιῤῥροφοριών (double ρ)
%% Added numbers to ensure the reader knows the order of the months
%%
%%% Count out columns for fixed-width source font
% 000000011111111112222222222333333333344444444445555555555666666666677777777778
% 345678901234567890123456789012345678901234567890123456789012345678901234567890
%
%% Select a general font size (uncomment one from the list)
%\tiny
%\scriptsize
%\footnotesize
%\small
\normalsize
%% Center the whole table left-right
\centering
%% Modify separation between columns
%\setlength{\tabcolsep}{0.5em}
%% Modify distance between rows
%\renewcommand{\arraystretch}{0.85}
%%
%% Nested tables: two tables side-by-side, each containing two seasons
\begin{tabular}{cc}
\begin{tabular}{ c l }
\multicolumn{2}{c}{\textgreek{ΘΕΡΙΝΟΙ ΜΗΝΕΣ}} \\
1. &\textgreek{Εκατομβαιών} \\
2. &\textgreek{Μεταγειτνιών} \\
3. &\textgreek{Βοηδρομιών} \\
%% empty line between the seasons
~ & ~ \\
%%
\multicolumn{2}{c}{\textgreek{ΟΠΩΡΙΝΟΙ ΜΗΝΕΣ}} \\
4. &\textgreek{Πυανεψιών} \\
5. &\textgreek{Μαιμακτηριών} \\
6. &\textgreek{Ποσειδεών} \\
\end{tabular}
%% next column
&
%%
\begin{tabular}{ c l }
\multicolumn{2}{c}{\textgreek{ΧΕΙΜΕΡΙΝΟΙ ΜΗΝΕΣ}} \\
7. &\textgreek{Γαμηλιών} \\
8. &\textgreek{Ανθεστηριών} \\
9. &\textgreek{Ελαφηβολιών} \\
%% empty line between the seasons
~ & ~ \\
%%
\multicolumn{2}{c}{\textgreek{ΕΑΡΙΝΟΙ ΜΗΝΕΣ}} \\
10. &\textgreek{Μουνυχιών} \\
11. &\textgreek{Θαργηλιών} \\
12. &\textgreek{Σκιῤῥοφοριών} \\
\end{tabular}
\end{tabular}
%
\caption{Laterculum mensium Atticorum secundam anni quadrantes}
\label{tab:p033}
%

\end{table}

\lnr{3}Partibus anni explicatis, superest, ut ad
totum ipsum, hoc est, ad annum veniamus.
\lnr{5}Duo autem summa in eo disputanda
sunt.
\lnr{6}Aequatio et caput periodi.
\lnr{6}Aequationem
hoc in negotio Graeci \textgreek{προσθαΦαίρεσιν[?]}
vocant, verbo \textgreek{έκ τὴς προσθέσεως καὶ ἀφαιρέσεως[?]}
composito; quod elegantissimum verbum
dissimulatum est apud Aristotelem Oecon.
\rnum{i}. \textgreek{διόπερ δεῖ ποιεῖσθαι σκέψιν, καὶ διανέμειν τε, καὶ ἀνιέναι
κατ᾽ ἀξίαν ἕκαστα καὶ τροφὴν καὶ ἐσθῆτα, καὶ
ἀργίαν καὶ κολάσεις, λόγῳ καὶ ἔργῳ μιμουμένους
τὴν ἰατρῶν δύναμιν ἐν φαρμάκου λόγῳ,
προσθεωροῦντας[?]}.
\lnr{15}Omnino enim legendum
\textgreek{προσθαφαιροῦντας[?]}, non \textgreek{προσθεωροῦντας[?]}.
\lnr{16}Igitur
\textgreek{πρόσθεσις[?]} sit in anno Attico aut dierum, aut
mensis.
\lnr{18}Dierum, ut earum, quas \textgreek{ὑπερβαλλούσας
καὶ ἀνάρχους[?]} dictas fuisse iam monuimus.
\lnr{20}Mensis, quem \textgreek{ἐμβόλιμον[?]} vocabant.
\lnr{21}\textgreek{Αφαίρεσις[?]} vero fit aut unius diei, in anno
quarto Tetraeteridis, aut ad summum bidui,
ut Cicero docet.
\lnr{23}Diodorus Siculus \textgreek{ἐμβολισμοῦ
καὶ ἀφαιρέσεως[?]} Graecorum, quae
sunt duae partes aequationis anni Graecanici,
ita meminit de Thebanis Aegypti loquens:
\textgreek{ἐμβολίμους δὲ μῆνας σὠκ ἄγουσιν, ὀδ᾽
ἡμέρας ὑφαιροῦσι καθάπερ οἱ πλεῖστὸι τῶν ἑλλένων[?]}.
\lnr{28}Ita \textgreek{ἐμβολισμὸς μηνὸς[?]}, et
\textgreek{ὑφαίρεσις ἡμέρας[?]} sunt partes aequationis
 temporis civilis Graecorum.
\lnr{30}Aequationis igitur partes sunt tres, \textgreek{ὑπερβάλλουσαι[?]},
 sine \textgreek{ἄναρχοι ἡμέραι[?]},
\textgreek{ἐμβόλιμος μὴν[?]} et \textgreek{ἐξαίρεσις[?]}.
\lnr{31}Quod \textgreek{αί ὑπερβάλλουσαι[?]} in exitum anni reiicerentur,
et ratio ipsa postulat, et Glaucippus priscus scriptor apud
Macrobium docet, quod supra trictim tetigimus.
\lnr{33}Sed Macrobius testem
suum, quem producit, non intellexit, nec quae essent illae
 \textgreek{ὑπερβάλλουσαι,
καὶ ὑπερβαίνουσαι ἡμέραι[?]}.
\lnr{35}Deprehenso autem periodi Atticae
initio, facile quis esset exitus anni, et in quod tempus conferendus
sit, suo loco tractabitur.
\lnr{37}Supersunt duo, \textgreek{περὶ ἐμβολισμοῦ καὶ ἐξαιρέσεως[?]}.
\lnr{38}An scilicet eidem mensi embolismus et exaeresis attributa.
\lnr{38}An
vero utriusque rei diversi fuerint menses.
\lnr{39}Tria verba sunt, quae vulgus
cunfundit, \textgreek{ἐμβόλιμος, ἐμβολιμαῖος, ἐμβολισμός[?]}.
\lnr{40}Mensis, aut
dies, qui intercalatur, \textgreek{ἐμβόλιμος[?]},
 ut apud nos Bisextum est \textgreek{ἐμβόλιμος
ἡμέρα[?]}.

% 34
% {PDF page nr}{source page nr}{line nr}
\plnr{117}{34}{1}Annus, cui competit intercalatio, \textgreek{ἐμβολιμαῖος[?]}.
\lnr{1}Res ipsa, et
actus, \textgreek{ἐμβολισμός[?]}.
\lnr{2}\textgreek{Εμβόλιμος μὴν[?]} Atheniensium erat Posideon.
\lnr{2}Id nos
docet Ptolemaeus, qui anno Nabonassari \rnum{ccclxvii},
 \textgreek{θώθ᾽ \gnum{ιϛ}, ἄρχοντος
Αθήνῃσιν Ευάνδρου, μηνὸς Ποσειδεῶνος προτέρου[?]} Lunam defecisse scribit.
\lnr{5}Ergo \textgreek{ἐμβολισμὸς[?]} incidit in illum annum, qui ab eo dicitur
 \textgreek{ἔτος ἐμβολιμαῖον[?]},
et mensis \textgreek{ἐμβόλιμος[?]} Posideon Metonicus prior.
\lnr{6}Nam in omni
iusta ac legitima intercalatione, ex duobus
cognominibus mensibus is est \textgreek{ἐμβόλιμος[?]},
qui prior est.
\lnr{9}Ut apud Habraeos, anno
embolimaeo ex duobus Adarin, is Adar
est ordinarius, qui secundus est, in quo et
ieiunium Ester observatur.
\lnr{12}Quod si Posideon
Metonicus intercalatur, ergo et Tetraetericus.
% Table: Laterculum neomeniarum lunarium in mensibus Atticis
\begin{table}[htbp]
  %%% Liber I p34
%%
%% For testing, uncomment the folowing lines and the lines at the end of the file
%% Test ==>
%\documentclass{book}
%\usepackage{fontspec}
%\setmainfont{Hoefler Text}[]
%%\setmainfont{Times New Roman}[]
%\newfontfamily\greekfont{Times New Roman}
%\usepackage[quiet]{polyglossia}
%\setmainlanguage{latin}
%\setotherlanguage{greek}
%\begin{document}
%% <== Test
%%
\begin{tabular}{r|llrr|l}
 1 & 1.   & 23 & 15 & 7 &\textgreek{ἐκατομβαιών} \\
 2 & 1\textbar{}30 & 22 & 14 & 6 &\textgreek{μεταγειτνιών} \\
 3 & 30   & 22 & 14 & 6 &\textgreek{βοηδρομιών} \\
 4 & 29   & 21 & 13 & 5 &\textgreek{πυανεψιών} \\
 5 & 29   & 21 & 13 & 5 &\textgreek{μαιμακτηριών} \\
 6 & 28   & 20 & 12 & 4 &\textgreek{ποσειδεών} \\
 7 & 26   & 18 & 10 & 3 &\textgreek{υαμηλιών} \\
 8 & 25   & 17 &  9 & 3 &\textgreek{ανθεστηριών} \\
 9 & 25   & 17 &  9 & 2 &\textgreek{ἐλαφηβολιών} \\
10 & 24   & 16 &  8 & 2 &\textgreek{μουνυχιών} \\
11 & 24   & 16 &  8 & 1 &\textgreek{θαργηλιών} \\
12 & 23   & 15 &  7 & 1 &\textgreek{σκιῤῥοφοριών} \\
\end{tabular}
%% Test ==>
%\end{document}

\end{table}
\lnr{14}Subiecimus vero mensium Atticorum
laterculum cum neomeniis Lunaribus
comparatorum.
\lnr{16}Quod \textgreek{περὶ ἐμβολισμοῦ,
ἐμβολίμου, ἐμβολιμαίου[?]} diximus,
idem censendum \textgreek{περὶ ἐξαιρέσεως, ἐξαιρεσίμου,
ἐξαιρεσιμαίου[?]}.
\lnr{19}\textgreek{Εξαιρέσιμος ἡμέρα[?]} est dies, qui eximitur.
\lnr{19}\textgreek{Εξαιρεσιμαῖος
μὴν[?]}, mensis, in quem incidit \textgreek{ἐξαίρεσις[?]} ipsa.
\lnr{20}Mensis igitur Atheniensium
\textgreek{ἐξαιρεσιμαῖος[?]} est Boedromion.
\lnr{21}Plutarchus \textgreek{συμποσιακῶν θ᾽[?]}, segmento
% Plutarch: Symposiacs book 9, section 6
% "What is signified by the fabe about the defeat of Neptune?
% And also, why do the Athenians omit the second day of the month Boedromion?"
\rnum{vi}: \textgreek{καὶ ὁ Υλας ὥσπερ ἡδίων γενόμενος[?]}.
% And Hylas, as if he had become better tempered:
\lnr{22}\textgreek{ἐκεῖνο δέ σε, εἶπεν, ὦ Μενέφυλλε,
λέληθεν, ὅτι καὶ τὴν δευτέραν Βονδρομιῶνος ἡμέραν ἐξῃρήκαμεν, οὐ πρὸς τὴν
σελήνην, ἀλλ᾽ ὅτι ταύτῃ δοκοῦσιν ἐρῖσαι περὶ τὴς χώρας οἱ θεώ[?]}.
% One thing has escaped you, Menephylus, that we have given up
% the second day of September, not on account of the moon, but
% because on that day the gods seemed to have contended for the country.
\lnr{24}Sed caussa
\textgreek{ἐξαιρέσεως[?]}, quam ille reddit, tam ridicula, quam falsum,
 quod subiicit,
eam diem \textgreek{ἐξαιρεῖσθαι[?]} non Lunae caussa, hoc est eo nomine, ut
annus cum Lunae rationibus congruat, set ob mythologiam.
\lnr{27}Quod
sane verum fuerit, si et fabulae verae.
\lnr{28}Praeterea et falsum, quod idem
de eadem re loquens \textgreek{ἐν τῷ περὶ φιλαδελφίας[?]} ait:
% eadē -> eadem
 \textgreek{τὲν δευτέραν τοῦ βονδρομιῶνος
ἀεὶ ἐξαιρεῖσθαι[?]}.
\lnr{30}Nam non \textgreek{ἀεὶ[?]}, sed semel in Tetraeteride.
\lnr{30}Verba eius
haec sunt: \textgreek{ἀθηναῖοι δὲ τὸν περὶ τὴς ἔριδος τῶν θεῶν μῦθον,
 ἀτόπως πλάσαντες,
ἐπανόρθωμα τὴς ἀτοπίας οὐ φαῦλον ἐνέμιξαν ἀυτῷ[?]}.
\lnr{32}\textgreek{τὴν γὰρ δευτέραν ἐξαιροῦσιν
ἀεὶ τοῦ βονδρομιῶνος, ὡς ἐν ἐκείνῃ τῷ Ποσειδῶνι πρὸς τὴν Αθηνᾶν γενομένης τὲς
διαφορᾶς[?]}.
\lnr{34}In sequentibus ob id censet \textgreek{ἐξαιρέσιμον[?]} fuisse,
 quod esset \textgreek{μία
τῶν ἀποφράδων[?]}.
\lnr{35}Quod et ipsum vero adversatur.
\lnr{35}Beavit tamen nos, cuius
indicio mensem \textgreek{ἐξαιρεσιμαῖον[?]} cognoscimus.
\lnr{36}Quare anno quarto
Tetraeteridis pro \textgreek{δευτέρα βονδρομιῶνος[?]} dicebant,
 \textgreek{τρίτη ἱσταμένου[?]}.
\lnr{37}Quomodo
et computatores nostri in ratione Lunae unum diem dissimulare
praecipiunt: quem ipsi vocant Saltum Lunae,
 Graecus computus \textgreek{ὑποτομὴν
σελήνης[?]}.
\lnr{40}At quare ultimam mensis potius non eximebant?
\lnr{40}Quia
menses Attici, cum omnes fuerint \textgreek{τριακονθήμεροι[?]},
 ii in tres \textgreek{δεκάδασ[?]} dividebantur.

% 35
% {PDF page nr}{source page nr}{line nr}
\plnr{118}{35}{1}Et tertiae decadis dies retro \textgreek{κατ᾽ ἐξαίρεσιν[?]}
 numerabantur, \textgreek{δεκάτη,
ἐνάτη, ἐνάτη, ὀγκόη, ἑβδόμη, ἕκτη τελδυτῶντος[?]}.
\lnr{2}Quod si ultima exemta fuisset,
quomodo \rnum{xxi} dies vocari potuisset \textgreek{δεκάτη φθίνοντος[?]}?
\lnr{3}Adde, quod
\textgreek{ἔνη καὶ νέα[?]} erat sacra.
\lnr{4}Propterea etiam neque \textgreek{πρώτην ἱσταμένου[?]} eximebant,
quod omnes Kalendae \textgreek{καὶ ἱερομηνίαι[?]} sacrae essent.
\lnr{5}Itaque cum pro secunda
die tertiam dicebant, nihil videbatur de mense decedere.
\lnr{6}Nam si
mensis est triginta dierum, is demum videtur esse integer,
 cui \textgreek{ἡ τριακὰς
καὶ ἔνη καὶ νέα[?]} non deest.
\lnr{8}Cur Boedromione mense potius, quam alio,
\textgreek{ἐξαίρεσιος[?]} fieret, equidem cum Plutarcho puto in caussa
 fuisse mythologiam
illam contentionis Minervae cum Neptuno in Atticae vindicatione.
\lnr{11}Sed \textgreek{ἐξαίρεσιν[?]} illam propter Lunam non fieri,
 id vero constanter
nego, atque adeo pernego.
\lnr{12}At Plutarchus harum rerum ubique sese imperitissimum
% harum: clearer in other edition
prodit.
\lnr{13}Nam, ut diximus, in una Tetraeteride, hoc est in
diebus 1448, vel, si embolismus incideret, in diebus 1478, una dies detrahenda
erat, ut annus sequens rediret in gratiam cum Luna, quod
primus dies Tetraeteridis semper deberet incurrere in interlunium.
\lnr{16}Sed
in 76 annis necesse erat, ut ex una Tetraeteride non dies unus, ut solet,
sed biduum eximeretur, idque quater in illis 76 annis fieri solebat.
\lnr{18}Quod
iam ostendimus, et locus Ciceronis ex Verrinis id clare testatur.
\lnr{19}Sed
puto Plutarchum sentire, quando fit \textgreek{ἐξαίρεσις[?]},
 eam semper \textgreek{τῇ δευτέρᾳ
Βονδρομιῶνος[?]} fieri, non alio mense.
% FOD in this copy. Better view of text in other copies.
\lnr{21}Fiebat enim anno quatro Tetraeteridos.
% Clearly new sentence in 1598 Lugduni edition
\lnr{22}Itaque Plutarchus pugnam Plataeensem, quam certum est contigisse
in anno quarto Tetraeteridos Atticae, ut alibi ostendetur, confert
in quartam Boedromionis in Aristide, in Camillo autem, in tertiam:
utrumque vere: quia \textgreek{πολιτικῶς[?]} erat \textgreek{τετρὰς[?]},
 re vera \textgreek{τρίτη[?]}, propter \textgreek{τὴν
ἐκαίρεσιν[?]}.
\lnr{26}Alia vero erat ratio \textgreek{ἐκαιρεσίμων ἡμερῶν[?]} in anno Lunari.
\lnr{26}Nam in
quatuor annis Tetraeteridos fiebat una,
 aut ad summum duae \textgreek{ἐξαιρέσεις[?]}.
\lnr{28}In anno autem Lunari semper sex.
\lnr{28}Nam cum menses Lunares alternis
pleni essent, et cavi, omnes tamen \textgreek{τριακάδα[?]} habebant.
\lnr{29}Et propterea
alternis unus dies eximebatur, ut secunda secundi mensis non
 \textgreek{δευτέρα[?]},
sed \textgreek{τρίτη[?]} vocaretur.
\lnr{31}Et ita in reliquis.
\lnr{31}Aristoteles Oeconomicorum secundo,
de Mnemone Rhodio tyranno Lampsaci:
% Aristotle "Economics" book 2, section 1351
% "Memnon, the Rhodian, after making himself master of Lapsacus,
% was in need of money.
 \textgreek{τῶν τε στρατευομένων
παρῃρεῖτο τὰς σιταρκίας, καὶ τοὺς[?] μισθοὺς ἑξ ἡμερῶν τοῦ ἐνιαυτοῦ,
 φάσκων ταύταις
ταῖς ἡμέραις ὄυτε φυλακὴν ἀυτοὺς[?] ὀυδεμίαν, ὄυτε πορεῖαν, ὄυτε δαπάνην ποιεῖσθαι,
τὰς ΕΞΑΙΡΕΣΙΜΟΥΣ λέων[?]}.
\lnr{35}Lampsacenorum igitur annus
mere Lunaris erat: cuius pares menses, quanuis cavi essent, tamen
\textgreek{τριακονθήμεροι[?]} putabantur: hoc est \textgreek{τριακάδα[?]}
 habebant.
\lnr{37}Milites autem
quot mensibus stipendium accipiebant ratione triginta dierum etiam
mense cavo.
\lnr{39}At vafer tyrannus mensibus cavis rationem habebat \textgreek{τὲς
ἐξαιρέσεως[?]}.
\lnr{40}Quod et sequentia declarant.
\lnr{40}\textgreek{τόντε προτοῦ χρόνον διδοὺς τοῖς στρατιώταις
τῇ δευτέρᾳ τὲς νουμηνίας τὰς σιταρκίας τῷ μὲν πρώτῳ μηνὶ παρέβη τρεῖς ἡμέρας,
τῷ δ᾽ ἐχομένῳ πέντε[?]}.

% 36
% {PDF page nr}{source page nr}{line nr}
\plnr{119}{36}{1}\textgreek{Τοῦτον δὲ τρόπον προῆγεν, ἕως  εἰς τὴν τριακάδα
ἦλθε[?]}.
\lnr{2}Prius, inquit, solebat postridie Kalendarum dare.
\lnr{2}Cum vero esset
\textgreek{ἐξαίρεσις[?]}, in qua pro secunda mensis tertiam putabant,
 29 dies tantum
ad secundam sequentis putabat.
\lnr{4}Deinde, ut illiberalius faceret, aliud
institutum tenuit.
\lnr{5}Nam primo mense 27 dierum metiebatur annonam:
secundo 25, et ita deinceps hoc modo. 27.25.22.20.17.15.12.10.7.
5.2.0.
\lnr{7}Sic integrum mensum lucrabatur.
\lnr{7}Luculentus est hic locus
Aristotelis.

\section{De Periodo Olympica}
\lnr{9}Vetustissima res est apud Graecos quadrantis diei supra
\rnum{ccclxv} dies observatio.
\lnr{10}Graecis consultantibus quodmodo
rite sacra obirent, responsum ab oraculo,
 ut \textgreek{κατὰ τὰ πάτρια[?]} sacrificarent.
\lnr{12}Cum quaererent, quid esset \textgreek{κατὰ τὰ πάτρια θύειν[?]},
 iterum responsum,
\textgreek{κατὰ τρία[?]}.
\lnr{13}A peritis interpretatum, \textgreek{θύειν κατὰ τὰ πάτρια[?]},
 et \textgreek{κατὰ τρία[?]},
esse \textgreek{θύειν κατ᾽ ἐνιαυτοὺς, κατὰ μῆνας, καθ᾽ ἡμέρας[?]}.
\lnr{14}Si igitur annus Solaris tantum
observaretur, non per omnia satisfieret oraculo.
\lnr{15}Nam \textgreek{κατ᾽ ἐνιαυτοὺς[?]}
quidem sacrificaretur, sed non \textgreek{κατὰ μῆνας, καὶ καθ᾽ ἡμέρας[?]}.
\lnr{16}Annus enim
Solis tantum est, menses Lunae, dies utriusque sideris.
\lnr{17}Visum igitur
illis peritis, ut optimae maximae Panegyrides, quales erant Olympiades,
et Pythiades, mense mero Lunari celebrarentur, et eae certis circuitibus
in orbem redirent, et metas Solis consequerentur: quod unico
remedio embolismorum sit.
\lnr{21}Itaque statuerunt annum duodecim plenorum
mensium, cui appendices dies duos annectebant, quorum annorum
quatuor iustam periodum conficere visi sunt ad neomeniam
Lunae adipiscendam, detracto uno die de illis appendicibus.
\lnr{24}Hoc tempus
Tetraeterida vocarunt.
\lnr{25}Ita Olympias, quae debebat celebrari plenilunio,
praecise incidit in quintam decimam primi mensis.
\lnr{26}Et sic \textgreek{κατὰ μῆνα[?]}
sacrificatur, quia \textgreek{κατὰ σελένην[?]}.
\lnr{27}Rursus quia interventu embolimi mensis
secunda Tetraeteris consequitur curriculum Solis,
 sic \textgreek{κατ᾽ ἑνιαυτὸν[?]} videbantur
sacrificare, et proinde \textgreek{καθ᾽ ἡμέρας[?]},
 quia \textgreek{καὶ κατὰ σελήνην καὶ καθ᾽
ἥλιον[?]}.
\lnr{30}Sed aliquanto post deprenensum aliquot Octaeteridas, quae ex
duabus Tetraeteridibus componuntur, abundare diebus singulis supra
rationes Solis.
\lnr{32}Itaque cum omnibus Tetraeteridibus singuli dies detraherentur,
oportebat aliquando binos detrahi.
\lnr{33}Isque annus, in que hoc
accidebat, vocatur ab nobis \textgreek{δισεξαιρεσιμαῖος[?]}.
% à -> ab
\lnr{34}Cui enim mensi competebat
\textgreek{ἐξαίρεσις[?]} unius diei, eidem et bidui
 \textgreek{ἐξαίρεσις[?]} congruebat, ut diserte
nos docuit Cicero.
\lnr{36}Cum igitur pertissime ab astrologis et Hierophantis
illa bidui exemptio fieret, primus omnium Calippus animadvertit
in novemdecim Tetraeteridibus, qui sunt anni 76, quater biduum eximendum.

% 37
% {PDF page nr}{source page nr}{line nr}
\plnr{120}{37}{1}Vidit enim iniuste quadrantem diei ab Metone relictum, et
% à -> ab
27760 dies, qui Metoni essent, septuaginta sex anni, excedere annos
totidem Solares uno die: ut veri 76 anni Solares sint dierum 27759.
\lnr{4}Sed \rnum{xix} Tetraeterides Graecae, cum novem mensibus
 plenis intercalaribus
fiunt dies 27763: de quibus si detrahantur 27759 dies, relinquentur
dies quatuor eximendi ultra \rnum{xix} \textgreek{ἐξαιρεσίμους[?]}
 ordinarios.
\lnr{6}Iam
76 anni Graeci sunt dies 27360, quibus si adieceris 270 dies novem
mensium intercalarium, fient omnes dies 27630: qui detracti de
27759 diebus, relinquunt dies \textgreek{ἀνάρχους[?]},
 quos appendices vocavimus,
129.
\lnr{10}Illis 129. in 76 annos distributis, omnes \rnum{xix}
 Tetraeterides quidem
erunt \textgreek{ἐξαιρεσιμαῖοι[?]}, sed quatuor inter illas
 \textgreek{δισεξαιρεσιμαῖοι[?]} erunt istae,
prima, quinta, decima, quinta decima, hoc est annus quartus, vicesimus,
quadragesimus, sexagesimus erunt \textgreek{δισεξαιρεσιμαῖοι[?]},
 ita ut novilunia primi
mensis Tetraeteridum mire quadrent: et sane elegantissima sit haec
forma in tanta ignoratione motus Lunaris.
\lnr{15}Quod et quiuis deprehendere
possit, qui volet periculum facere.
\lnr{16}Calippus quidem nihil in
ea re innovavit, sed, ut diximus, ostendit tantum in novemdecim Tetraeteridibus
quatuor \textgreek{ἐξαιρέσεις[?]} fieri supra \rnum{xix} ordinarias,
 quomodo 23
\textgreek{ἐξαιρέσεις[?]} totidem Tetraeteridibus competant.
\lnr{19}Ideo iustam periodum
76 annorum conficit, cui annorum
 \textgreek{ἐξαιρεσιμαίων καὶ δισεξαιρεσιμαίων[?]} notas
apposuit.
\lnr{21}Eamque periodum non solum Athenienses, sed et aliae
Graecae nationes amplexae sunt.
\lnr{22}Antiquissima vero omnium Tetraeteridum
est Olympica.
\lnr{23}Ideo periodus Olympica ante omnes, et primo
loco ponenda: praesertim cum reliquae omnes ab hac derivatae sint, ut
suis locis ostendetur.
\lnr{24}Merito igitur familiam ducet.
\lnr{24}Eius initium ab diebus
% à -> ab
aestivis, teste Censorino, et Statio, qui Solstitium vocat annum
Pisaeum
\begin{verse}
--- \textit{domus improba frangit}\\
\textit{Frigora; Pisaeumque domus non aestuat annum.}
\end{verse}
% Publius Papinius Statius, Silvae 1.3.1, Villa Tiburtina Manilli Vopisci
% Line 7-8:
% "talis hiems tectis, frangunt sic improba solem
% frigora, Pisaeumque domus non aestuat annum."
\lnr{29}Id est: Ea domus non percipitur solstitiali sidere.
\lnr{29}Mensis enim primus
Olympicus, in cuius quintamdecimam incidebat ludicrum Olympicum,
incipiebat proxime post \rnum{viii} Iulii, ut citima neomenia esset
 in \rnum{ix}
Iulii, remotissima in \rnum{vi} Augusti.
\lnr{32}Primum enim certamen Olympicum
commissum anno periodi Iulianae 3938, cyclo Lunae quinto, Solis \rnum{xviii},
Neomenia Ab Iudaici 2985 feria tertia, Iulii \rnum{ix}.
\lnr{34}Character enim fuit
3.3.905.
\lnr{35}Ideo nunquam citra illos fines inibat mensis Olympiadis, sed
serius.
\lnr{36}Et tunc fere cardines mundi, hoc est
 \textgreek{τροπαὶ, καὶ ἰσημερίαι[?]} conficiebantur
in octavis partibus signorum.
\lnr{37}Nam \rnum{viii} Iulii \textgreek{πλατυκῶς[?]} tunc
erat in octava parta Cancri.
\lnr{38}Primus igitur agon Olympicus celebratus
\rnum{xxiii} Iulii.
\lnr{39}Et omnes Graeci \textgreek{τροπὰς θερινὰς[?]} collocant in
 \rnum{viii} aut \rnum{ix}
Iulii.
\lnr{40}Imo Ephorus apud Dionysium Halicarnassensem, \rnum{ix} Iulii vocat
\textgreek{τροπὰς θερινὰς[?]}: male, ut videbimus.
\lnr{41}Nam primi mensis Elidensis, sive
Olympici, neomenia est in \rnum{ix} Iulii post Solstitium, quod statuebatur
in \rnum{viii}, ita ut \rnum{ix} sit prima et citima neomenia,
 non autem Solstitium.


% 38
% {PDF page nr}{source page nr}{line nr}
\plnr{121}{38}{3}Construatur igitur Tabula primi mensis Elidensium
 in annis totius periodi
expansis (p. \pageref{tab:p038}).
% Insert table:
% Tabula Neomeniarum primi mensis Εlidensis in annis periodi Olympicae
\begin{table}[htbp]
 %%% Liber I p38
%%
%%% Count out columns for fixed-width source font
% 000000011111111112222222222333333333344444444445555555555666666666677777777778
% 345678901234567890123456789012345678901234567890123456789012345678901234567890
%
{
\tabnums % Select monospaced numbers
%% Select a general font size (uncomment one from the list)
%\tiny
%\scriptsize
\footnotesize
%\small
%\normalsize
%% Center the whole table left-right
\centering
%% Modify separation between columns
%\setlength{\tabcolsep}{0.5em}
%% Modify distance between rows
\renewcommand{\arraystretch}{0.85}
%
%% Different daggers
\newcommand{\dsize}{\scriptsize}
\newcommand{\dc}{{\dsize †}}
\newcommand{\da}{{\dsize ‡}}
\newcommand{\db}{{◊}}
%% The angle with which to slant
\newcommand{\ang}{60}
%% Text size of the headers
\newcommand{\hdsize}{\scriptsize}
%% Define the column headers so both sub-tables are the same
\newcommand{\hdr}{%
\begin{tabular}[t]{r rrr r@{~}l r l}
~ &
\multicolumn{1}{c}{\begin{rotate}{\ang}\hdsize Anni periodi\end{rotate}} &
\multicolumn{1}{c}{\begin{rotate}{\ang}\hdsize Cyclus Lunnae\end{rotate}} &
\multicolumn{1}{c}{\begin{rotate}{\ang}\hdsize Dies collecti\end{rotate}} &
\multicolumn{2}{c}{\begin{rotate}{\ang}\hdsize
  \parbox[t]{3.5cm}{Neomenia\\\hspace*{5pt}1. mensis}
\end{rotate}} &
\multicolumn{2}{l}{\begin{turn}{\ang}\hdsize \textgreek{περιτταὶ ἡμέραι}\end{turn}}
}
%%
% Implemented as two subtables side-by-side
\begin{tabular}{@{}lc@{}}
\toprule
\multicolumn{2}{ c }{\Large\textsc{Tabula neomeniarum primi mensis}} \\
\multicolumn{2}{ c }{\large\textsc{Elidensis in annis periodi Olympicae}} \\
\toprule
% Left subtable
\hdr % tabular command and column headers
\\
\cmidrule{2-7}
  ~ &  1 &  5 &  392 &  9&Iulii & 0 & \dc \\
  ~ &  2 &  6 &  754 &  5&Aug. & 27 & \\
  ~ &  3 &  7 & 1116 &  2&Aug. & 24 & \\
\db &  4 &  8 & 1476 & 29&Iul. & 20 \\
\cmidrule{2-7}
  ~ &  5 &  9 & 1838 & 24&Iul. & 15 \\
  ~ &  6 & 10 & 2200 & 21&Iul. & 12 \\
  ~ &  7 & 11 & 2562 & 18&Iul. &  9 \\
\da &  8 & 12 & 2923 & 14&Iul. &  5 \\
\cmidrule{2-7}
  ~ &  9 & 13 & 3315 & 10&Iul. &  1 & \dc \\
  ~ & 10 & 14 & 3677 &  6&Aug. & 28 \\
  ~ & 11 & 15 & 4039 &  3&Aug. & 25 \\
\da & 12 & 16 & 4400 & 30&Iul. & 21 \\
\cmidrule{2-7}
  ~ & 13 & 17 & 4702 & 26&Iul. & 17 \\
  ~ & 14 & 18 & 5124 & 23&Iul. & 14 \\
  ~ & 15 & 19 & 5480 & 20&Iul. & 11 \\
\da & 16 &  1 & 5846 & 16&Iul. &  7 \\
\cmidrule{2-7}
  ~ & 17 &  2 & 6208 & 12&Iul. &  3 \\
  ~ & 18 &  3 & 6600 &  9&Iul. &  0 & \dc \\
  ~ & 19 &  4 & 6962 &  5&Aug. & 27 \\
\db & 20 &  5 & 7322 &  1&Aug. & 23 \\
\cmidrule{2-7}
  ~ & 21 &  6 & 7686 & 27&Iul. & 18 \\
  ~ & 22 &  7 & 8047 & 24&Iul. & 15 \\
  ~ & 23 &  8 & 8409 & 21&Iul. & 12 \\
\da & 24 &  9 & 8770 & 17&Iul. &  8 \\
\cmidrule{2-7}
  ~ & 25 & 10 &  9133 & 13&Iul. &  4 \\
  ~ & 26 & 11 &  9524 & 10&Iul. &  1 & \dc  \\
  ~ & 27 & 12 &  9886 &  6&Aug. & 28 \\
\da & 28 & 13 & 10247 &  2&Aug. & 24 \\
\cmidrule{2-7}
  ~ & 29 & 14 & 10609 & 29&Iul. & 20 \\
  ~ & 30 & 15 & 10971 & 26&Iul. & 17 \\
  ~ & 31 & 16 & 11333 & 23&Iul. & 14 \\
\da & 32 & 17 & 11694 & 19&Iul. & 10 \\
\cmidrule{2-7}
  ~ & 33 & 18 & 12056 & 29&Iul. &  6 \\
  ~ & 34 & 19 & 12418 & 26&Iul. &  3 \\
  ~ & 35 &  1 & 12810 & 23&Iul. &  0 & \dc  \\
\da & 36 &  2 & 13171 & 19&Aug. & 26 \\
\cmidrule{2-7}
  ~ & 37 &  3 & 13533 & 31&Iul. & 22 \\
  ~ & 38 &  4 & 13895 & 28&Iul. & 19 \\
  ~ & 39 &  5 & 14257 & 25&Iul. & 16 \\
\db & 40 &  6 & 14617 & 21&Iul. & 12 \\
\cmidrule{2-7}
\end{tabular}
%% next column
&
%%
% Right subtable
\hdr % tabular command and column headers
\\
\cmidrule{2-7}
  ~ & 41 &  7 & 14979 & 16&Iul. &  7 \\
  ~ & 42 &  8 & 15341 & 13&Iul. &  4 \\
  ~ & 43 &  9 & 15733 & 10&Iul. &  1 & \dc  \\
\da & 44 & 10 & 16094 &  5&Aug. & 27 \\
\cmidrule{2-7}
  ~ & 45 & 11 & 16456 &  1&Aug. & 23 \\
  ~ & 46 & 12 & 16818 & 29&Iul. & 20 \\
  ~ & 47 & 13 & 17180 & 26&Iul. & 27 \\
\da & 48 & 14 & 17541 & 22&Iul. & 13 \\
\cmidrule{2-7}
  ~ & 49 & 15 & 17903 & 18&Iul. &  9 \\
  ~ & 50 & 16 & 18265 & 15&Iul. &  6 \\
  ~ & 51 & 17 & 18627 & 12&Iul. &  3 & \dc  \\
\da & 52 & 18 & 19017 &  7&Aug. & 29 \\
\cmidrule{2-7}
  ~ & 53 & 19 & 19379 &  3&Aug. & 25 \\
  ~ & 54 &  1 & 19741 & 31&Iul. & 22 \\
  ~ & 55 &  2 & 20103 & 28&Iul. & 19 \\
\da & 56 &  3 & 20464 & 24&Iul. & 14 \\
\cmidrule{2-7}
  ~ & 57 &  4 & 20826 & 20&Iul. & 11 \\
  ~ & 58 &  5 & 21188 & 17&Iul. &  8 \\
  ~ & 59 &  6 & 21550 & 14&Iul. &  5 \\
\db & 60 &  7 & 21940 & 10&Iul. &  1 & \dc  \\
\cmidrule{2-7}
  ~ & 61 &  8 & 22302 &  4&Aug. & 26 \\
  ~ & 62 &  9 & 22664 &  1&Aug. & 23 \\
  ~ & 63 & 10 & 23026 & 29&Iul. & 20 \\
\da & 64 & 11 & 23388 & 25&Iul. & 16 \\
\cmidrule{2-7}
  ~ & 65 & 12 & 23750 & 21&Iul. & 12 \\
  ~ & 66 & 13 & 24112 & 18&Iul. &  9 \\
  ~ & 67 & 14 & 24474 & 15&Iul. &  6 \\
\da & 68 & 15 & 24865 & 11&Iul. &  2 & \dc  \\
\cmidrule{2-7}
  ~ & 69 & 16 & 25227 &  6&Aug. & 28 \\
  ~ & 70 & 17 & 25589 &  3&Aug. & 25 \\
  ~ & 71 & 18 & 25951 & 31&Iul. & 22 \\
\da & 72 & 19 & 26312 & 27&Iul. & 18 \\
\cmidrule{2-7}
  ~ & 73 &  1 & 26674 & 23&Iul. & 14 \\
  ~ & 74 &  2 & 27036 & 20&Iul. & 11 \\
  ~ & 75 &  3 & 27398 & 17&Iul. &  8 \\
\da & 76 &  4 & 27759 & 13&Iul. &  4 \\
\cmidrule{2-7}
\\
~ & \multicolumn{5}{l}{\super\dc{} \textgreek{ἐμβολ.}}\\
~ & \multicolumn{5}{l}{\super\da{} \textgreek{ἐξαιρ.}}\\
~ & \multicolumn{5}{l}{\super\db{} \textgreek{δισεξαιρεσιμαῖος [?]}}\\
\end{tabular}
\end{tabular}
%
\caption{Neomeniarum primi mensis Elidensis in annis periodi Olympicae}
\label{tab:p038}
}

\end{table}
%
\lnr{4}Primus ordo habeat numerum annorum periodi expansorum:
secundus cyclum Lunae: tertius dies annorum Graecorum collectus:
quartus locum neomeniae in mensibus Iulianis.


% 39
% {PDF page nr}{source page nr}{line nr}
\plnr{122}{39}{2}Quintus et ultimus
\textgreek{περιττὰς ἡμέρας[?]} in mense Iuliano.
% Note: the only Greek word I could find in the dictionary that comes close
% is περιττές (feminine plural), with an epsilon instead of an alpha, meaning
% 1) unnecessary, needless, superfluous;
% 2) (mathematics) odd
% On the other hand, this spelling does appear in various texts found online.
\lnr{3}\textgreek{περιτταὶ ἡμέραι[?]} in anno Iuliano
est intervallum inter \rnum{viii} Iulii,
 et neomeniam sequentis Hecatombaeonis
Attici, aut primi mensis Elidensis, sive Olympici, cuius nomen
ignoramus.
\lnr{6}Itaque deprehenso capite Hecatombaeonis, quot
\textgreek{περιτταὶ ἡμέραι[?]} supersint de Scirrhophorione in anno Iuliano,
 nobis quidem
deprehendere facile est, qui rationem tantum habemus anni nostri
Iuliani.
\lnr{9}Sed Elidenses Hierophantae, quos \textgreek{Βασίλας[?]} vocabant,
intercalabant
diem inter \rnum{viii} et \rnum{ix} Iulii, fine quarti anni Solaris, diebus
400 ante bisextum Iulianum.
\lnr{11}Ideo quartus annus Olympicus Solaris
incipiebat 122 die post bisextum Iulianum, et quintus quadringentis
diebus.
\lnr{13}Quare in anno quarto ipsi ponebant 21 \textgreek{περιττὰς[?]}, cum in
anno Iuliano sint tantum 20, propter bisextum Iulianum.
\lnr{14}Annus igitur
primus Olympiadicus est embolimaeus, dierum 392. de quibus
detracto anno Solari, nempe 365 diebus, remanent
 \textgreek{περιτταὶ ἡμέραι[?]} 27.
\lnr{17}Itaque Scirrhophorion primus antevertit Hecatombaeonem primum
diebus tribus.
\lnr{18}Annus sequens cum 392 facit dies 754.
\lnr{18}Qui de duobus
annis Solaribus detracti relinquunt \textgreek{περιττὰς ἡμέρας[?]} 24.
\lnr{19}Scirrhophorion
ergo secundus antevertit primi Hacatombaeonis neomeniam sex diebus.
\lnr{21}Rursus tertius annus cum 754 diebus compositus facit dies 1116: qui
de tribus annis Solaribus detracti relinquunt 21 \textgreek{περιττὰς[?]}.
\lnr{23}Cum tamen in
Tabula sint tantum 20, propter bisextum Iulianum.
\lnr{23}Quatuor vero anni
Iuliani, hoc est 1461 dies, de 1477 diebus detracti relinquerent 16
\textgreek{περιττάς[?]}.
\lnr{25}Sed quia ille annus est \textgreek{δισεξαιρεσιμαῖος[?]}, hoc est,
 duo dies de eo
perimuntur, relinquet 15 \textgreek{περιττάς[?]}: ut est in Tabula.
\lnr{26}Quod autem prima
Tetraeteris sit \textgreek{δισεξαιρεσιμαῖος[?]} ita demonstratur.
\lnr{27}Supra diximus 129
\textgreek{ἀνάρχους ἡμέρασ[?]} tantum esse debere in 76 annis Graecis.
\lnr{28}Quae si in ipsos
annos distribuantur, habebis 1~\myfrac{53}{76} diei in singulos annos.
\lnr{29}Ita primus annus
habebit unum diem cum \myfrac{53}{76} diei:
 secundus 2 dies cum \myfrac{30}{76}. tertius item
duos cum \myfrac{7}{76}.
\lnr{31}Quartus unum cum \myfrac{60}{76}.
\lnr{31}Itaque primus, et quartus
habent singulas \textgreek{ἀνάρχους ἡμέρασ[?]}:
 et proinde tota Tetraeteris est \textgreek{δισεξαιρεσιμαῖος[?]}.
\lnr{33}Sic progressus diei 1~\myfrac{53}{76}
 ostendet tibi quintam Tetraeterida,
decimam, quintamdecimam esse \textgreek{δισεξαιρεσιμαίους[?]}.
\lnr{34}Quod autem
primus annus primae Tetraetiridis non mutilaretur statim una die, sed
biduum in quarto anno perimeretur, docet nos, ut diximus, Cicero,
qui ex eodem mense extra ordinem unum diem eximi solere scribit, ex
quo ordinarius dies eximeretur.
\lnr{38}Ita habes elegantissimam periodum,
cuius neomeniae primae omnium Tetraeteridum sunt exactissime Lunares.
\lnr{40}Nam primo novilunio deprehenso, reliqua sequentur, tanquam
catena quaedam.
\lnr{41}Propter incertas autem epochas annorum in nationibus
Graecis, et diversa principia tam mensium, quam annorum, Olympici
ludi ab praeconibus denunciabantur.
% à -> ab

% 40
% {PDF page nr}{source page nr}{line nr}
\plnr{123}{40}{2}Pindarus Isthmico secundo:
% Pindar (Πινδαρος): Greek lyric poet from Thebes
% Second collection of victory odes for the Isthmian games
% ΞΕΝΟΚΡΑΤΕΙ ΑΚΡΑΓΑΝΤΙΝΩι ΑΡΜΑΤΙ
% Verse ~34-~38
\textgreek{ὅντε καὶ κάρυκες ὡρᾶν ἀνέγνων σπονδοφόροι Κρονίδα Ζηνὸς Α᾽λείου[?]}
%\textgreek{ὅντε καὶ κάρυκες ὡρᾶν ἀνέγνον, σπονδοφόροι Κρονίδα Ζηνὸς Ἀλεῖοι [?]}
Scholion: \textgreek{οἱ κήρυκες οἱ τὰς ὥρας καὶ τὸν καιρὸν τοῦ ὀλυμπιακοῦ ἀγῶνος
ἐκήρυσσον, καθ᾽ ἅς ἐτελεῖτο[?]}.
% Greek transcription needs better diacritics.
\lnr{5}Ita ut verissima haec sit periodus, ad
quam caeterae omnes Graecae periodi directae sunt, tanquam ad certam
regulam, idque reverentia cultus Olympici, mutatis tantum Tetraeteridibus,
non autem noviluniis.
\lnr{8}Nam omnes omnium periodorum anni
conveniunt in cyclo Lunae, non autem in tempore.
\lnr{9}Exemplum.
\lnr{9}Periodus
Attica fingitur antiquior Olympica annis 38, qui sunt duo cycli.
\lnr{11}Itaque conveniet Attica cum Olympica cyclo, non tempore: quod
ad methodum Attica fingitur antiquior, Olympica posterior, et annus
primus Tetraeteridis Atticae sit tertius Olympicae.
\lnr{13}Scholiastes Pindari
Olympico \rnum{ix}.91. \textgreek{κατὰ μίαν ἡμέραν οἱ δύο ἐνίκησαν[?]}.
\lnr{14}\textgreek{ὁ μὲν Ε᾽φάρμοστος, Ο᾽λύμπια[?]}.
\lnr{15}\textgreek{ὁ δὲ Λαμπρόμαχος, Ισθμια[?]}.
\lnr{15}Semper celebrabantur eadem die
eiusdem mensis.
\lnr{16}Celebrabantur autem Kal. Hyperberetaei aestivi.
\lnr{16}Ergo
prioris tetraeteridis Festum cadebat in novilunium: posterioris in plenilunium,
quod incidebat in neomeniam mensis Isthmaici, et \rnum{xv} mensis
Elidensis.
\lnr{19}Idem tamen est situs annorum, eaedem neomeniae, ut alibi
videbitur.
\lnr{20}Quare omnes periodi inter se differunt 19, aut 57, aut 38
annis, uno, tribus, aut duobus cyclis.
\lnr{21}Si periodus quaedam differt ab
Olympica uno cyclo, vicesimus alterutrius annus erit primus alterius,
et ita quartus annus alterutrius Tetraeteridis erit primus alterius.
\lnr{23}Si differant
duobus cyclis, tertius annus alterutrius Tetraeteridis erit primus
alterius tetraeteridis.
\lnr{25}Si denique differant tribus cyclis, annus secundus
alterutrius Tetraeteridis erit primus alterius.
\lnr{26}Saepenumero Tetraeteris
est dierum 1476. ut prima periodi.
\lnr{27}Qui sunt omnio Lunares
menses quinquaginta, quum tamen Tetraeteris Lunaris sit tantum
mensium undequinquaginta: Itaque primo anno cycli Iudaici Ab mensis
Olympicus habebit neomeniam in \rnum{xi} Iulii, cyclo Paschali quarto.
\lnr{30}Quinto
anno neomenia Ab erit in 28 Iunii.
\lnr{31}At mensis Olympicus erit non iam
Ab, ut in priore Tetraeteride, sed in 28 Iulii, cyclo Paschali octavo.
\lnr{33}Quare inter utramque neomeniam sunt menses absoluti
 Lunares quinquaginta.
\lnr{34}Hoc est, quod voluit interpres Pindari in tertium Olympicum
\textgreek{γίνεται δὲ ὁ ἀγὼν ποτὲ μὲν διὰ τεσσαράκοντα ἐννέα μηνῶν[?]}.
\lnr{35}\textgreek{Ποτὲ δὲ διὰ πεντέκοντα[?]}.
\lnr{36}\textgreek{ὅθεν καὶ ποτὲ μὲν τῷ Απολλωνίῳ μηνὶ,
 ποτὲ δὲ, τῷ Παρθενίῳ ἐπιτελεῖται[?]}.
\lnr{36}Hoc
vult, aliquando celebratur
 \textgreek{τῷ ἑκατομβαιῶνι πρυτανείας, ποτὲ δὲ τῷ μεταγειτνιῶνι
πρυτανείας[?]}.
\lnr{38}At mensis, quo celebrabatur, apud vulgus semper
idem erat, puta Hecatombaeon popularis, sed non semper Hecatombaeon
\textgreek{πρυτανείας[?]}.
\lnr{40}Ab undecima autem mensis ad \rnum{xvi} per quinque
dies certamen celebrabantur.
\lnr{41}Unde \textgreek{πεμπταμέρους ἁμίλλας [?]} dixit Pindarus
Olympico~\rnum{v}.

% 41
% {PDF page nr}{source page nr}{line nr}
\plnr{124}{41}{1}Hoc etiam exemplo probatur nunqueam anticipari
\textgreek{κέντρον[?]}.
\lnr{2}Mensium vero Elidensium nonum \textgreek{Ελάφιον[?]} tantum reperi apud
Pausaniam \textgreek{ἠλιακῶν \gnum{α} κατ᾽ ἔτος
 δὲ ἕκαστον φυλάξαντεσ οἱ μάντεις τὴν ἐνάτην
ἐπὶ δέκα τοῦ Ελαφίου μηνὸς κομίζουσιν ἐκ τοῦ πρυτανείου τὴν τέφραν[?]}.
\lnr{4}In Aequinoctium
vernum eum incidere ait idem \textgreek{ἠλιακῶν \gnum{β}[?]}.
\lnr{5}\textgreek{Επὶ δὲ τοῦ ὄρους τῇ
κορυφῇ θύουσιν οἱ Βασίλαι καλούμενοι τῷ Κρόνῳ κατὰ ἰσημερίαν τὴν ἐν τῷ ἦρι
Ελαφίῳ μηνὶ παρὰ Ηλείοις[?]}.
\lnr{7}Ergo idem erat cum Elaphebolione, et
proinde nonus ab primo solstitiali.
% à -> ab
\lnr{8}Meminerunt etiam Scholia Pindari
\textgreek{του Απολλωνίου καὶ του Παρθενίου μηνὸς[?]}.
\lnr{9}Ode \rnum{iii} Olymp.
\lnr{9}\textgreek{Γίνεται δ᾽ ὁ ἀγὼν ποτὲ
μὲν διὰ τεσσαράκοντα ἐννέα μηνῶν, ποτὲ διὰ πεντέκοντα[?]}.
\lnr{10}\textgreek{ὅθεν καὶ ποτὲ μὲν τῷ Απολλωνίῳ
μηνὶ, ποτὲ δὲ τῷ Παρθενίῳ ἐπιτελεῖται[?]}.

\section{De Periodo Attica}

\lnr{12}Proximum ab periodo Olympica locum occupet Attica.
% à -> ab
\lnr{12}Sine
cuius cognitione omne studium inutile est in ratione temporis
Graeci.
\lnr{14}Cum igitur de Anno Attico disputaremus, \textgreek{τοῦ ὅλου[?]} duas
partes fecimus, \textgreek{τὴν προσθαφαίρεσιν[?]}, et caput periodi:
 quod duplicem
interpretationem postulat.
\lnr{16}Nam videndum et a quo tempore, et a quo
cyclo Lunae deducendum.
\lnr{17}Cum autem naturalis ratio postulet, ut intercalatio
in finem anni reiiciatur, recte illum anni mensem ultimum
dicemus, qui intercalatione attributus est.
\lnr{19}Is vero est Posideon, ut antea
demonstravimus.
\lnr{20}Ergo Posideon est ultimus mensis, et Gamelion,
qui proxime sequitur, princeps mensium.
\lnr{21}Quare initium
anni Attici incurrit in tempus brumae.
\lnr{22}Cui rei adstipulatur Terentius,
qui in palliata Apollodori ita loquitur: \textit{Aruspex vetuit ante brumam
aliquid novi Incipere}.
\lnr{24}--- Quia scilicet priscus annus Atticus
ab bruma incipiebat.
% à -> ab
\lnr{25}Eamque Comoediam sine dubio Apollodorus
docuit \textgreek{Διονυσίοις κατ᾽ ἀγροὺς[?]}, mense Posideone,
 qui proxime antecedit
mensem brumalem.
\lnr{27}Atqui, inquies, Hecatombaeon est mensis
Olympiadicus, ab quo Graeci propter Olympiadem putant sua
% à -> ab
tempora.
\lnr{29}Fateor.
\lnr{29}Tamen nihilominus quod dixi, verum est.
\lnr{29}Nam
contextus veteris anni Attici deductitur ab bruma.
% à -> ab
\lnr{30}Vulgus vero caepit
putare tempora ab Hecatombaeone, propter Olympiadicum agonem,
qui proximo post solstitium plenilunio celebrabantur.
\lnr{32}Itaque cum re
vera nullum sit anni principium \textgreek{φύσει[?]},
 sed potius \textgreek{θέσει[?]}, ut in circulo:
tamen liceat nobis principium naturale vocare id, unde putantur anni
ratiocinia.
\lnr{35}Nam quod de circulo proposuimus, id omnino dissimile
est in Graeca periodo, quae principium naturale habet, ut in Laterulo
noviluniorum Tetraeteridis (p.\pageref{tab:p027})
 supra posito videre potuisti.
% Table reference
\lnr{37}Nam primum
prima Tetraeteridis consideranda in periodo: deinde primus annus
in Tetraeteride: quae cum quatuor annis constet, omnes diversa
habent novilunia, et principium eiusdem Tetraeteridis constituitur
in eo novilunio, unde caetera novilunia, quasi quaedam Fati Chrysippei
catena, deducuntur.

% 42
% {PDF page nr}{source page nr}{line nr}
\plnr{125}{42}{2}Illud igitur tale principium nos naturale vocamus:
alterum autem ab Hecatombaeone, Populare dicatur.
\lnr{3}De quo
principio Plato de Legibus sexto:
 \textgreek{ἐπειδὰν μέλλοι νέος ἐνιαυτὸς μετὰ θερινὰς
τροπὰς τῷ ἐπιόντι μηνὶ γενέσθαι[?]}.
\lnr{5}Quin etiam Dionysius ex Ephoro
populariter Hecatombaeonem constituit mensium principem in ratione
temporum.
\lnr{7}Scribit enim Ilion captum \textgreek{τελευτῶντος ἤδη τοῦ ἔαρος,
ἑπτακαίδεκα ἡμέραις πρότερον τὴς θερινῆς τροπῆς, ὀγδόῃ φθίνοντος μηνὸς
Θαργηλιῶνος, ὡς Αθηναῖοι τοὺς χρόνους ἄγουσι[?]}.
\lnr{9}Sic duplex initium anni
Iudaici: naturale ab autumno.
\lnr{10}Quia ab eo deducitur anni ratiocinium:
alterum civile et Ecclasiasticum ab Nisan.
% à -> ab
\lnr{11}De naturali igitur
principio loquimur, quod quidem, ut diximus, in bruma statuendum
esse non solum rationes supra adductae argumento sunt, sed et is, quem
modo nominavimus, Dionysius.
\lnr{12}Is ait Ilion captum \rnum{xvii} diebus
ante Solstitium, vicesima tertia Thargelionis.
\lnr{15}Ergo Solstitium fuit
decima Scirrhophorionis.
\lnr{16}Post decimam Scirrhophorionis scribit superfuisse
vingiti dies anno complendo.
\lnr{17}\textgreek{περιτταὶ δὲ ἦσαν αἱ τὸν ἐνιαυτὸν
ἐκεῖνον ἐκπληροῦσαι μετὰ τὴν τροπὴν εἴκοσι ἡμέραι[?]}.
\lnr{18}Si igitur post finem
Scirrhophorionis sectum est proxime caput anni alterius: sane Scirrhophorion
non habuit appendices \textgreek{ἀνάρχους ἡμέρας[?]}.
\lnr{20}Ergo Tetraeteris
antiqua non instituebatur ab Hecatombaeone.
\lnr{21}Nam, ut docebat
Glaucippus, \textgreek{ὑπερβάλλουσαι ἡμέραι[?]} conferebantur in calcem anni.
\lnr{22}Scirrhophorion
eas dies non habet apud Dionysium: ergo Scirrhophorion
non est ultimus mensis anni.
\lnr{24}Proinde neque Hecatombaeon erit primus.
\lnr{25}Sequitur igitur eum mensem \textgreek{τὰς ὑπερβαλλούσας[?]}
 habuisse, qui
erat ultimus in ratione periodi.
\lnr{26}Is autem est Posideon.
\lnr{26}Quod cognovimus
iam ex loco intercalationis.
\lnr{27}Ergo Posideon habuit \textgreek{τὰς ἀνάρχους[?]}, et
consequenter Gamelion proximus ab eo fuit mensis primus in ratione
Tetraeteridis priscae.
\lnr{29}Cum igitur Posideon habuerit \textgreek{τὰς ἀνάρχους ἡμέρας[?]},
quibus olim magistratus creabatur: non dubium est, quin, quandiu
obtinuit, ut ante \textgreek{νουμηνίαν γαμηλιῶνος[?]} magistratus crearetur,
Gamelion ipse esset caput anni.
\lnr{32}Sed postea reverentia Olympici ludicri,
Hecatombaeon caepit esse caput anni.
\lnr{33}His animadversis, videndum,
cui anno Olympico competat caput periodi Atticae.
\lnr{34}Plutarchus scribit
pugnam Chabriae circa Naxum contigisse \textgreek{βονδρομιῶνος πέμπτῃ
φθίνοντος, περὶ τὴν πανσέληνον[?]}.
\lnr{36}Hoc tempus confertur in annum primum
Olympiadis centesimae primae apud Eusebium, hoc est, in annum
Iphiti 401.
\lnr{38}Quod si plenilunium contigit vicesima sexta Boedromionis:
\lnr{39}Ergo novilunium fuit quartadecima aut tertiadecima mensis.
\lnr{39}Et quidem Boedromion est tertius mensis.
\lnr{40}Ut autem novilunium incidat in
\rnum{xiiii} aut \rnum{xiii} mensis, id vero non potest contingere,
 nisi anno Tetraeteridis
tertio, et proinde annus primus Olympiadis est tertius Tetraeteridis
Atticae.

% 43
% {PDF page nr}{source page nr}{line nr}
\plnr{126}{43}{2}Ergo, secundum ea, quae in fine diatribae de periodo
Olympica demonstravimus, periodus Attica differt ab Olympica
duobus cyclis.
\lnr{4}Et ad methodum Atticae periodi, 38 anni de periodo
Olympica detrahendi erunt.
\lnr{5}Construximus igitur Tabulam eiusdem
periodi instar Olympicae, quam auximus omnibus neomeniis,
et earum epochis in mensibus iulianis: quare nulla magis necessaria
in ratione temporum Atticorum.

% (paragraph break before this needed or \centered and such bleed over in text)
% Insert Table: "Tabula neomeniarum Atticarum in mensibus Iulianis"
%%% Liber I p43
%%
%%% Count out columns for fixed-width source font
% 000000011111111112222222222333333333344444444445555555555666666666677777777778
% 345678901234567890123456789012345678901234567890123456789012345678901234567890
%
\begingroup
%% Select a general font size (uncomment one from the list)
%\tiny
\scriptsize
%\footnotesize
%\small
%\normalsize
%% Center the whole table left-right
\centering
%% Modify separation between columns
\setlength{\tabcolsep}{1.6pt}
%% Modify distance between rows
\renewcommand{\arraystretch}{1.2}
%
%% Define reference symbols
\newcommand{\da}{{\tiny †}}
\newcommand{\db}{{\tiny ‡}}
%% The angle with which to slant
\newcommand{\ang}{60}
%% Generate the column headers
\newcommand{\hdrs}{%
% A \multicolumn{} here would clash with \addcontentsline{}
\begin{rotate}{\ang}Anni periodi\end{rotate} &
&
\multicolumn{1}{c}{\begin{rotate}{\ang}\textgreek{Εκατομβαιών}\end{rotate}} & &
\multicolumn{1}{c}{\begin{rotate}{\ang}\textgreek{Μεταγειτνιών}\end{rotate}} & &
\multicolumn{1}{c}{\begin{rotate}{\ang}\textgreek{Βοηδρομιών}\end{rotate}} & &

\multicolumn{1}{c}{\begin{rotate}{\ang}\textgreek{Πυανεψιών}\end{rotate}} & &
\multicolumn{1}{c}{\begin{rotate}{\ang}\textgreek{Μαιμακτηριών}\end{rotate}} & &
\multicolumn{1}{c}{\begin{rotate}{\ang}\textgreek{Ποσειδεών α}\end{rotate}} & &

\multicolumn{1}{c}{\begin{rotate}{\ang}\textgreek{Ποσειδεών β}\end{rotate}} & &

\multicolumn{1}{c}{\begin{rotate}{\ang}\textgreek{Γαμηλιών}\end{rotate}} & &
\multicolumn{1}{c}{\begin{rotate}{\ang}\textgreek{Ανθεστηριών}\end{rotate}} & &
\multicolumn{1}{c}{\begin{rotate}{\ang}\textgreek{Ελαφηβολιών}\end{rotate}} & &

\multicolumn{1}{c}{\begin{rotate}{\ang}\textgreek{Μουνυχιών}\end{rotate}} & &
\multicolumn{1}{c}{\begin{rotate}{\ang}\textgreek{Θαργηλιών}\end{rotate}} & &
\multicolumn{1}{c}{\begin{rotate}{\ang}\textgreek{Σκιῤῥοφοριών}\end{rotate}} & &

\multicolumn{2}{l}{\begin{turn}{\ang}\textgreek{περιτταὶ ἡμέραι}\end{turn}} \\
}
%
%% Let longtable process the whole table in one go
\setcounter{LTchunksize}{100}
\begin{longtable}[c]{@{} r  r  *{13}{r@{~}l} r c @{}}
\toprule
\multicolumn{30}{c}{\Large\textsc{Tabula neomeniarum Atticarum}} \\
\multicolumn{30}{c}{\Large\textsc{in mensibus Iulianis}} \\
\toprule
% Put a reference to the first page of the table in the List of Tables
\addcontentsline{lot}{section}{%
\protect\numberline{\thetable}Neomeniarum Atticarum in mensibus Iulianis}
\label{tab:p043}
\hdrs % Column headers from the above definition
\midrule
\endfirsthead
%%
\toprule
\multicolumn{30}{c}{\Large\textsc{Residuum tabulae neomeniarum Atticarum}} \\
\multicolumn{30}{c}{\Large\textsc{in mensibus Iulianis}} \\
\toprule
\hdrs % Column headers from the above definition
\midrule
\endhead
%%
% The \nopagebreak commands result in a cline{} at the bottom of each page
% Putting in a bottomrule looks weird.
%\bottomrule
\addlinespace[5pt]
  & & & \multicolumn{11}{l}{\super\da \textgreek{ἐξαιρεσίμαίων [?]}}
& & & & \multicolumn{11}{l}{\super\db \textgreek{δισέξαιρεσιμαίων [?]}}
\\
\endfoot
%%
%\bottomrule
\addlinespace[5pt]
  & & & \multicolumn{11}{l}{\super\da \textgreek{ἐξαιρεσίμαίων [?]}}
& & & & \multicolumn{11}{l}{\super\db \textgreek{δισέξαιρεσιμαίων [?]}}
\\
%\addlinespace
% Put the table nr and title below the table, without entry in the LoT
\caption[]{Neomeniarum Atticarum in mensibus Iulianis}
\endlastfoot
%%
  &  1 &  9&Iul &  8&Aug &  7&Sep &  7&Oct &  6&Nov &  6&Dec & 
 5&Ian &  6&Feb &  8&Mar &  7&Apr &  7&Mai &  6&Iun &  6&Iul &  0 \\
\nopagebreak
~ &  2 &  5&Aug &  4&Sep &  5&Oct &  3&Nov &  3&Dec &  2&Ian &
  &    &  3&Feb &  5&Mar &  4&Apr &  4&Mai &  3&Iun &  3&Iul & 27 \\
\nopagebreak
~ &  3 &  2&Aug &  1&Sep &  1&Oct & 31&Oct & 30&Nov & 30&Dec &
  &    & 31&Ian &  1&Mar & 31&Mar & 30&Apr & 30&Mai & 29&Iun & 24 \\
\nopagebreak
\db
  &  4 & 29&Iul & 28&Aug & 27&Sep & 25&Oct & 24&Nov & 24&Dec &
  &    & 25&Ian & 24&Feb & 26&Mar & 25&Apr & 25&Mai & 24&Iun & 20 \\
\nopagebreak
\cline{2-29}
~ &  5 & 24&Iul & 23&Aug & 22&Sep & 22&Oct & 21&Nov & 21&Dec & 
  &    & 22&Ian & 21&Feb & 23&Mar & 22&Apr & 22&Mai & 21&Iun & 15 \\
\nopagebreak
~ &  6 & 21&Iul & 20&Aug & 19&Sep & 19&Oct & 18&Nov & 18&Dec &
  &    & 17&Ian & 16&Feb & 18&Mar & 17&Apr & 17&Mai & 16&Iun & 13 \\
\nopagebreak
~ &  7 & 18&Iul & 17&Aug & 16&Sep & 16&Oct & 15&Nov & 15&Dec &
  &    & 16&Ian & 15&Feb & 16&Mar & 15&Apr & 15&Mai & 14&Iun &  9 \\
\nopagebreak
\da
  &  8 & 14&Iul & 13&Aug & 12&Sep & 11&Oct & 10&Nov & 10&Dec &
  &    & 11&Ian & 10&Feb & 12&Mar & 11&Apr & 11&Mai & 10&Iun &  5 \\
\nopagebreak
\cline{2-29}
~ &  9 & 10&Iul &  9&Aug &  8&Sep &  8&Oct &  7&Nov &  7&Dec &
 6&Ian &  5&Feb &  7&Mar &  6&Apr &  6&Mai &  5&Iun &  5&Iul &  1 \\
\nopagebreak
~ & 10 &  4&Aug &  3&Sep &  3&Oct &  2&Nov &  2&Dec &  5&Ian &
  &    &  2&Feb &  3&Mar &  2&Apr &  2&Mai &  1&Iun &  1&Iul & 28 \\
\nopagebreak
~ & 11 & 31&Iul & 30&Aug & 29&Sep & 29&Oct & 28&Nov & 28&Dec &
  &    &  1&Feb &  2&Mar &  1&Apr &  1&Mai & 31&Mai & 30&Iun & 25 \\
\nopagebreak
\da
  & 12 & 30&Iul & 29&Aug & 28&Sep & 27&Oct & 26&Nov & 26&Dec &
  &    & 27&Ian & 26&Feb & 28&Mar & 27&Apr & 27&Mai & 26&Iun & 21 \\
\nopagebreak
\cline{2-29}
~ & 13 & 26&Iul & 25&Aug & 24&Sep & 24&Oct & 23&Nov & 23&Dec &
  &    & 24&Ian & 23&Feb & 25&Mar & 24&Apr & 24&Mai & 23&Iun & 17 \\
\nopagebreak
~ & 14 & 23&Iul & 22&Aug & 21&Sep & 21&Oct & 20&Nov & 20&Dec &
  &    & 21&Ian & 20&Feb & 22&Mar & 21&Apr & 21&Mai & 20&Iun & 14 \\
\nopagebreak
~ & 15 & 20&Iul & 19&Aug & 18&Sep & 18&Oct & 17&Nov & 17&Dec &
  &    & 18&Ian & 17&Feb & 18&Mar & 17&Apr & 17&Mai & 16&Iun & 11 \\
\nopagebreak
\da
  & 16 & 16&Iul & 15&Aug & 14&Sep & 13&Oct & 12&Nov & 12&Dec &
  &    & 13&Ian & 12&Feb & 14&Mar & 13&Apr & 13&Mai & 12&Iun &  7 \\
\nopagebreak
\cline{2-29}
~ & 17 & 12&Iul & 11&Aug & 10&Sep & 10&Oct &  9&Nov &  9&Dec &
  &    & 10&Ian &  9&Feb & 11&Mar & 10&Apr & 10&Mai &  9&Iun &  3 \\
\nopagebreak
~ & 18 &  9&Iul &  8&Aug &  7&Sep &  7&Oct &  6&Nov &  6&Dec &
 5&Ian &  6&Feb &  8&Mar &  7&Apr &  7&Mai &  6&Iun &  6&Iul &  0 \\
\nopagebreak
~ & 19 &  5&Aug &  4&Sep &  4&Oct &  3&Nov &  3&Dec &  2&Ian &
  &    &  3&Feb &  4&Mar &  3&Apr &  3&Mai &  2&Iun &  2&Iul & 27 \\
\nopagebreak
\db
  & 20 &  1&Aug & 31&Aug & 30&Sep & 28&Oct & 27&Nov & 27&Dec &
  &    & 28&Ian & 27&Feb & 29&Mar & 28&Apr & 28&Mai & 27&Iun & 23 \\
\nopagebreak
\cline{2-29}
~ & 21 & 27&Iul & 26&Aug & 25&Sep & 25&Oct & 24&Nov & 24&Dec &
  &    & 25&Ian & 24&Feb & 26&Mar & 25&Apr & 25&Mai & 24&Iun & 18 \\
\nopagebreak
~ & 22 & 24&Iul & 23&Aug & 22&Sep & 22&Oct & 21&Nov & 21&Dec &
  &    & 22&Ian & 21&Feb & 23&Mar & 22&Apr & 22&Mai & 21&Iun & 13 \\
\nopagebreak
~ & 23 & 21&Iul & 20&Aug & 19&Sep & 19&Oct & 18&Nov & 18&Dec &
  &    & 19&Ian & 18&Feb & 19&Mar & 18&Apr & 18&Mai & 17&Iun & 12 \\
\nopagebreak
\da
  & 24 & 17&Iul & 16&Aug & 15&Sep & 14&Oct & 13&Nov & 13&Dec &
  &    & 14&Ian & 13&Feb & 15&Mar & 14&Apr & 14&Mai & 13&Iun &  8 \\
\nopagebreak
\cline{2-29}
~ & 25 & 13&Iul & 12&Aug & 11&Sep & 11&Oct & 10&Nov & 10&Dec &
  &    & 11&Ian & 10&Feb & 12&Mar & 11&Apr & 11&Mai & 10&Iun &  4 \\
\nopagebreak
~ & 26 & 10&Iul &  9&Aug &  8&Sep &  8&Oct &  7&Nov &  7&Dec &
 6&Ian &  7&Feb &  9&Mar &  8&Apr &  8&Mai &  7&Iun &  7&Iul &  1 \\
\nopagebreak
~ & 27 &  6&Aug &  5&Sep &  5&Oct &  4&Nov &  4&Dec &  3&Ian &
  &    &  4&Feb &  5&Mar &  4&Apr &  4&Mai &  3&Iun &  3&Iul & 28 \\
\nopagebreak
\da
  & 28 &  2&Aug &  1&Sep &  1&Oct & 30&Oct & 29&Nov & 29&Dec &
  &    & 30&Ian &  1&Mar & 31&Mar & 30&Apr & 30&Mai & 29&Iun & 24 \\
\nopagebreak
\cline{2-29}
~ & 29 & 29&Iul & 28&Aug & 27&Sep & 27&Oct & 26&Nov & 26&Dec &
  &    & 27&Ian & 26&Feb & 28&Mar & 27&Apr & 27&Mai & 26&Iun & 20 \\
\nopagebreak
~ & 30 & 26&Iul & 25&Aug & 24&Sep & 24&Oct & 23&Nov & 23&Dec &
  &    & 24&Ian & 23&Feb & 25&Mar & 24&Apr & 24&Mai & 23&Iun & 17 \\
\nopagebreak
~ & 31 & 23&Iul & 22&Aug & 21&Sep & 21&Oct & 20&Nov & 20&Dec &
  &    & 21&Ian & 20&Feb & 21&Mar & 20&Apr & 20&Mai & 19&Iun & 14 \\
\nopagebreak
\da
  & 32 & 19&Iul & 18&Aug & 17&Sep & 16&Oct & 15&Nov & 15&Dec &
  &    & 16&Ian & 15&Feb & 17&Mar & 16&Apr & 16&Mai & 15&Iun & 10 \\
\nopagebreak
\cline{2-29}
~ & 33 & 15&Iul & 14&Aug & 13&Sep & 13&Oct & 12&Nov & 12&Dec &
  &    & 13&Ian & 12&Feb & 14&Mar & 13&Apr & 13&Mai & 12&Iun &  6 \\
\nopagebreak
~ & 34 & 12&Iul & 11&Aug & 10&Sep & 10&Oct &  9&Nov &  9&Dec &
  &    & 10&Ian &  9&Feb & 11&Mar & 10&Apr & 10&Mai &  9&Iun &  3 \\
\nopagebreak
~ & 35 &  9&Iul &  8&Aug &  7&Sep &  7&Oct &  6&Nov &  6&Dec &
 5&Ian &  6&Feb &  7&Mar &  6&Apr &  6&Mai &  5&Iun &  5&Iul &  0 \\
\nopagebreak
\da
  & 36 &  4&Aug &  3&Sep &  3&Oct &  1&Nov &  1&Dec & 31&Dec &
  &    &  1&Feb &  3&Mar &  2&Apr &  2&Mai &  1&Iun &  1&Iul & 26 \\
\nopagebreak
\cline{2-29}
~ & 37 & 31&Iul & 30&Aug & 29&Sep & 29&Oct & 28&Nov & 28&Dec &
  &    & 29&Ian & 28&Feb & 30&Mar & 29&Apr & 29&Mai & 28&Iun & 22 \\
\nopagebreak
~ & 38 & 28&Iul & 27&Aug & 26&Sep & 26&Oct & 25&Nov & 25&Dec &
  &    & 26&Ian & 25&Feb & 27&Mar & 26&Apr & 26&Mai & 25&Iun & 19 \\
\nopagebreak
~ & 39 & 25&Iul & 24&Aug & 23&Sep & 23&Oct & 22&Nov & 22&Dec &
  &    & 23&Ian & 22&Feb & 23&Mar & 22&Apr & 22&Mai & 21&Iun & 16 \\
\nopagebreak
\db
  & 40 & 21&Iul & 20&Aug & 19&Sep & 17&Oct & 16&Nov & 16&Dec &
  &    & 17&Ian & 16&Feb & 18&Mar & 17&Apr & 17&Mai & 16&Iun & 12 \\
\nopagebreak
\cline{2-29}
~ & 41 & 16&Iul & 15&Aug & 14&Sep & 14&Oct & 13&Nov & 13&Dec &
  &    & 14&Ian & 13&Feb & 15&Mar & 14&Apr & 14&Mai & 13&Iun &  7 \\
\nopagebreak
~ & 42 & 13&Iul & 12&Aug & 11&Sep & 11&Oct & 10&Nov & 10&Dec &
  &    & 11&Ian & 10&Feb & 12&Mar & 11&Apr & 11&Mai & 10&Iun &  4 \\
\nopagebreak
~ & 43 & 10&Iul &  9&Aug &  8&Sep &  8&Oct &  7&Nov &  7&Dec &
 6&Ian &  7&Feb &  8&Mar &  7&Apr &  7&Mai &  6&Iun &  6&Iul &  1 \\
\nopagebreak
\da
  & 44 &  5&Aug &  4&Sep &  4&Oct &  2&Nov &  2&Dec &  1&Ian &
  &    &  2&Feb &  4&Mar &  3&Apr &  3&Mai &  2&Iun &  2&Iul & 25 \\
\nopagebreak
\cline{2-29}
~ & 45 &  1&Aug & 31&Aug & 30&Sep & 30&Oct & 29&Nov & 29&Dec &
  &    & 30&Ian &  1&Mar & 31&Mar & 30&Apr & 30&Mai & 29&Iun & 23 \\
\nopagebreak
~ & 46 & 29&Iul & 28&Aug & 27&Sep & 27&Oct & 26&Nov & 26&Dec &
  &    & 27&Ian & 26&Feb & 28&Mar & 27&Apr & 27&Mai & 26&Iun & 20 \\
\nopagebreak
~ & 47 & 26&Iul & 25&Aug & 24&Sep & 24&Oct & 23&Nov & 23&Dec &
  &    & 24&Ian & 23&Feb & 24&Mar & 23&Apr & 23&Mai & 22&Iun & 17 \\
\nopagebreak
\da & 48 & 22&Iul & 21&Aug & 20&Sep & 19&Oct & 18&Nov & 18&Dec &
  &    & 19&Ian & 18&Feb & 20&Mar & 19&Apr & 19&Mai & 18&Iun & 13 \\
\nopagebreak
\cline{2-29}
~ & 49 & 18&Iul & 17&Aug & 16&Sep & 16&Oct & 15&Nov & 15&Dec &
  &    & 16&Ian & 15&Feb & 17&Mar & 16&Apr & 16&Mai & 15&Iun &  9 \\
\nopagebreak
~ & 50 & 15&Iul & 14&Aug & 13&Sep & 13&Oct & 12&Nov & 12&Dec &
  &    & 13&Ian & 12&Feb & 14&Mar & 13&Apr & 13&Mai & 12&Iun &  6 \\
\nopagebreak
~ & 51 & 12&Iul & 11&Aug & 10&Sep & 10&Oct &  9&Nov &  9&Dec &
 8&Ian &  9&Feb & 10&Mar &  9&Apr &  9&Mai &  8&Iun &  8&Iul &  3 \\
\nopagebreak
\da
  & 52 &  7&Aug &  6&Sep &  6&Oct &  4&Nov &  4&Dec &  3&Jan &
  &    &  4&Feb &  6&Mar &  5&Apr &  5&Mai &  4&Iun &  4&Iul & 29 \\
\nopagebreak
\cline{2-29}
~ & 53 &  3&Aug &  2&Sep &  2&Oct &  1&Nov &  1&Dec &
 31&Dec\footnote{Erratum in originalis: Ian.} &
  &    &  1&Feb &  3&Mar &  2&Apr &  2&Mai &  1&Iun &  1&Iul & 25 \\
\nopagebreak
~ & 54 & 31&Iul & 30&Aug & 29&Sep & 29&Oct & 28&Nov & 28&Dec &
  &    & 29&Ian & 28&Feb & 30&Mar & 29&Apr & 29&Mai & 28&Iun & 22 \\
\nopagebreak
~ & 55 & 28&Iul & 27&Aug & 26&Sep & 26&Oct & 25&Nov & 25&Dec &
  &    & 26&Ian & 25&Feb & 26&Mar & 25&Apr & 25&Mai & 24&Iun & 19 \\
\nopagebreak
\da
  & 56 & 24&Iul & 23&Aug & 22&Sep & 21&Oct & 20&Nov & 20&Dec &
  &    & 21&Ian & 20&Feb & 22&Mar & 21&Apr & 21&Mai & 20&Iun & 14 \\
\nopagebreak
\cline{2-29}
~ & 57 & 20&Iul & 19&Aug & 18&Sep & 18&Oct & 17&Nov & 17&Dec &
  &    & 18&Ian & 17&Feb & 19&Mar & 18&Apr & 18&Mai & 17&Iun & 11 \\
\nopagebreak
~ & 58 & 17&Iul & 16&Aug & 15&Sep & 15&Oct & 14&Nov & 14&Dec &
  &    & 15&Ian & 14&Feb & 16&Mar & 15&Apr & 15&Mai & 14&Iun &  8 \\
\nopagebreak
~ & 59 & 14&Iul & 13&Aug & 12&Sep & 12&Oct & 11&Nov & 11&Dec &
  &    & 12&Ian & 11&Feb & 12&Mar & 11&Apr & 11&Mai & 10&Iun &  5 \\
\nopagebreak
\db
  & 60 & 10&Iul &  9&Aug &  8&Sep &  6&Oct &  5&Nov &  5&Dec &
 4&Ian &  5&Feb &  7&Mar &  6&Apr &  6&Mai &  5&Iun &  5&Iul &  1 \\
\nopagebreak
\cline{2-29}
~ & 61 &  4&Aug &  3&Sep &  3&Oct &  2&Nov &  2&Dec &  1&Ian &
  &    &  2&Feb &  4&Mar &  3&Apr &  3&Mai &  2&Iun &  2&Iul & 26 \\
\nopagebreak
~ & 62 &  1&Aug & 31&Aug & 30&Sep & 30&Oct & 29&Nov & 29&Dec &
  &    & 30&Ian &  1&Mar & 31&Mar & 30&Apr & 30&Mai & 29&Iun & 23 \\
\nopagebreak
~ & 63 & 29&Iul & 28&Aug & 27&Sep & 27&Oct & 26&Nov & 26&Dec &
  &    & 27&Ian & 26&Feb & 27&Mar & 26&Apr & 26&Mai & 25&Iun & 20 \\
\nopagebreak
\da
  & 64 & 25&Iul & 24&Aug & 23&Sep & 22&Oct & 21&Nov & 21&Dec &
  &    & 22&Ian & 21&Feb & 23&Mar & 22&Apr & 22&Mai & 21&Iun & 16 \\
\nopagebreak
\cline{2-29}
~ & 65 & 21&Iul & 20&Aug & 19&Sep & 19&Oct & 18&Nov & 18&Dec &
  &    & 19&Ian & 18&Feb & 20&Mar & 19&Apr & 19&Mai & 18&Iun & 12 \\
\nopagebreak
~ & 66 & 18&Iul & 17&Aug & 16&Sep & 16&Oct & 15&Nov & 15&Dec &
  &    & 17&Ian & 16&Feb & 17&Mar & 16&Apr & 16&Mai & 15&Iun &  9 \\
\nopagebreak
~ & 67 & 15&Iul & 14&Aug & 13&Sep & 13&Oct & 12&Nov & 12&Dec &
  &    & 13&Ian & 12&Feb & 13&Mar & 12&Apr & 12&Mai & 11&Iun &  6 \\
\nopagebreak
\da
  & 68 & 11&Iul & 10&Aug &  9&Sep &  8&Oct &  7&Nov &  7&Dec &
 6&Ian &  7&Feb &  9&Mar &  8&Apr &  8&Mai &  7&Iun &  7&Iul &  2 \\
\nopagebreak
\cline{2-29}
~ & 69 &  6&Aug &  5&Sep &  5&Oct &  4&Nov &  4&Dec &  3&Ian &
  &    &  4&Feb &  6&Mar &  5&Apr &  5&Mai &  4&Iun &  4&Iul & 28 \\
\nopagebreak
~ & 70 &  3&Aug &  2&Sep &  2&Oct &  1&Nov &  1&Dec & 31&Dec &
  &    & 30&Ian &  3&Mar &  2&Apr &  2&Mai &  1&Iun &  1&Iul & 25 \\
\nopagebreak
~ & 71 & 31&Iul & 30&Aug & 29&Sep & 29&Oct & 28&Nov & 28&Dec &
  &    & 29&Ian & 28&Feb & 29&Mar & 28&Apr & 28&Mai & 27&Iun & 22 \\
\nopagebreak
\da
  & 72 & 27&Iul & 26&Aug & 25&Sep & 24&Oct & 23&Nov & 23&Dec &
  &    & 24&Ian & 23&Feb & 25&Mar & 24&Apr & 24&Mai & 23&Iun & 18 \\
\nopagebreak
\cline{2-29}
~ & 73 & 23&Iul & 22&Aug & 21&Sep & 21&Oct & 20&Nov & 20&Dec &
  &    & 21&Ian & 20&Feb & 22&Mar & 21&Apr & 21&Mai & 20&Iun & 14 \\
\nopagebreak
~ & 74 & 20&Iul & 19&Aug & 18&Sep & 18&Oct & 17&Nov & 17&Dec &
  &    & 18&Ian & 17&Feb & 19&Mar & 18&Apr & 18&Mai & 17&Iun & 11 \\
\nopagebreak
~ & 75 & 17&Iul & 16&Aug & 15&Sep & 15&Oct & 14&Nov & 14&Dec &
  &    & 15&Ian & 14&Feb & 15&Mar & 14&Apr & 14&Mai & 13&Iun &  8 \\
\nopagebreak
\da
  & 76 & 13&Iul & 12&Aug & 11&Sep & 10&Oct &  9&Nov &  9&Dec &
  &    & 10&Ian &  9&Feb & 11&Mar & 10&Apr & 10&Mai &  9&Iun &  4 \\
\nopagebreak
\cline{2-29}
\end{longtable}
\endgroup



% 44
% {PDF page nr}{source page nr}{line nr}
\plnr{127}{44}{1}Adiecimus etiam \textgreek{περιττὰσ ἡμέρας[?]}:
quae semper differunt, ut diximus, unitate ab Olympicis in anno quarto,
propter bisextum Iulianum, et bisextum Olympicum.
\lnr{3}Nunc eius
fidem periclitemur.
\lnr{4}Repetatur exemplum proximum ab nobis adductum.
% à -> ab
\lnr{5}Annus 401 Iphiti, abiectis 76, quantum fieri potest, est 21 periodi
Olympicae sextae.
\lnr{6}De quibus si accommodata integra periodo,
(ut fieri oportet; quoties maior numerus est, qui detrahitur, quam unde
detrahitur) deducantu anni 38, relinquetur annus 59 periodi
Atticae in Tabula periodi.

% 45
% {PDF page nr}{source page nr}{line nr}
\plnr{128}{45}{2}E regione annorum 59, in area Boedromionis,
occurrit neomenia Boedromionis in 12 Septembris.
\lnr{3}Ergo vicesima
sexta fuit in \rnum{vii} Octobris.
\lnr{4}Quam non solum necessario fuisse
plenilunium docet laterculus neomeniarum, sed etiam Tisri Iudaicus
anni 3386.
\lnr{6}Cuius character 5.14.55. Septembris \rnum{xxiii}, feria quinta,
Cyclo Solis 26.
\lnr{7}Ideo plenilunium \rnum{vii} Octobris, feria quinta, ut
erat propositum.
\lnr{8}Rursus annus ab Ilio capto erat 408 ante initium
primae Olympiadis.
\lnr{9}Deductis omnibus 76, remanent 28 anni.
\lnr{9}Qui de 76 detracti relinquunt 48 annos absolutos periodi Olympicae,
quam fingimus tunc fuisse.
\lnr{11}Ergo 49 anno periodi Olympicae captum
est Ilion.
\lnr{12}Detractis ex methodo perpetua 38 annis, relinquitur annus
undecimus periodi Atticae, in quo anno captum fuerit Ilion.
\lnr{13}Eum
annum dicit Dionysius habuisse \textgreek{περιττὰς ἡμέρας εἴκοσι[?]}.
\lnr{14}Sed sine dubio
aut ille fallitur, aut codex est mendosus, ubi est
 \textgreek{εἴκοσιν ἡμέραι[?]}, pro
\textgreek{εἴκοσι μία ἡμέραι[?]}.
\lnr{16}Sane is est annus quem intelligebat Dionysius ex
Ephoro.
\lnr{17}Ilion igitur captum \textgreek{Θαργηλιῶνος ὀγδόῃ φθίνοντος[?]}.
\lnr{17}Neomenia Thargelionis 31 Mai.
\lnr{18}Ideo \rnum{xxiii} Thargelionis congruebat vicesimae
secundae Iunii, ab qua ad \rnum{ix} Iulii, non \rnum{xvii},
% à -> ab
 ut vult Dionysius,
sed \rnum{xvi} tantum dies erant.
\lnr{20}Nisi forte ipse \rnum{ix} non autem \rnum{viii} iulii
\textgreek{τροπὰσ θερινὰς[?]} vocet.
\lnr{21}Quod fieri potest.
\lnr{21}Tunc conveniet: et 20 fuerint
\textgreek{περιτταὶ ἡμέραι[?]}.
\lnr{22}Quomodocumque res habet, vides et illum esse
annum, quem habebat in animo Ephorus, et ad methodum perpetuam
de periodo Olympica semper 38 deducenda, ut habeas annum
periodi Atticae.
\lnr{25}Anno secundo Olympiadis 75, id est, anno 298 Iphiti,
Athenienses sextamdicam Munychionis Dianae consecrarunt,
quod plena lampade, hoc est plenilunio, pugnatum est eo die illius
anni navali certamine ad Salamina.
\lnr{28}Deductis 76, quantum potest,
de annis Iphiti 298, remanet annus 70 tertiae periodi Olympicae.
\lnr{30}Abiectis igitur ex perpetua methodo 38, remanet annus 32 periodi
Atticae, cuius Munychion 16 Aprilis.
\lnr{31}Qui, propterea quod est ultimus
Tetraeteridis, fere Lunaris est, ut ex Laterculo neomeniarum
Lunarium perspici potest.
\lnr{33}Congruit ea neomenia Ijar Iudaico 3283,
cuius character 5.8.124. decima septima Aprilis.
\lnr{34}Ergo Kalendis
Mai plenilunium, in quas Kalendas incidit \rnum{xvi} Munychionis.
\lnr{36}Vide quam pulchre haec respondent methodo.
\lnr{36}Plutarchi verba de
ea re: \textgreek{τὴν δὲ ἕκτην ἐπὶ δέκα Μουνυχιῶνος Αρτέμιδι καθιέρωσαν, ἐν ᾗ τοῖς
Ελλησι περὶ Σαλαμῖνα νικῶσιν ἐπέλαμψεν ἡ θεός[?]}.
\lnr{38}Illud autem tempus Eusebius
recte confert in secundum annum Olympiadis 75.
\lnr{39}Quod
si non fecisset, aut si omnes Chronologi tacuissent, tamen ex methodo
id divinari poterat.
\lnr{41}Quod autem Diodorus scribit Metonem
observasse novilunium primi sui Hecatombaeonis enneadecaeterici
\textgreek{σκιῤῥοφοριῶνος τρίτῃ ἐπὶ δέκα[?]}, anno quarto desinente
 Olympiadis 86,
scito, ut suo loco declarabitur, libro secundo, Metonem novilunium
anticipasse.

% 46
% {PDF page nr}{source page nr}{line nr}
\plnr{129}{46}{4}Nullus enim \textgreek{σκιῤῥοφοριῶν[?]}
 potest habere novilunium in
\rnum{xiii}, ut cognoscis ex latercullo neomeniarum Lunarium Atticarum.
\lnr{6}Sed in \rnum{xv} aut \rnum{xiiii} Scirrophorionis potest
 novilunium committi.
\lnr{7}Annus quartus Olympiadis 86, erat 344 ab Iphito: et proinde
40 periodi quintae Olympicae.
\lnr{8}Deductis 38, remanet annus
secundus periodi Atticae.
\lnr{9}Schirrhophorion \rnum{iii} Iulii.
\lnr{9}Ab Iudaicus
erat in \rnum{xvi} Iulii.
\lnr{10}At neomenia primi Hecatombaeonis Metonici
die \rnum{xv}.
\lnr{11}Itaque uno die anticipatum.
\lnr{11}In omnibus igitur mire constat
methodus: quam studiosi Lectores in multis aliis exemplis periclitari
possunt.
\lnr{13}Sed caveant, ne illis imponat Plutarchus, qui, ut alibi
diximus, non raro in his rebus hallucinatur.
\lnr{14}Uno exemplo prudentior
fieri potest Lector.
\lnr{15}Plenilunium eclipticum, quod contigit anno
Iphiti 446, Dario ad Gaugamela profligato, Plutarchus confert
in \textgreek{μυστηρίων[?]} tempus, hoc est, ut ipse alibi interpretatur,
 \textgreek{εἰς τὴν εἰκάδα
τοῦ βονδρομιῶνος[?]}.
\lnr{18}Abiectis 76 omnibus ex 446, remanet annus 66
sextae periodi Olympicae.
\lnr{19}Deductis 38, remanet annus 28 periodi Atticae.
\lnr{20}Boedromion Kal. Octobris.
\lnr{20}Metagitnion Kal. Septembris.
\lnr{20}In cuius
vicesimam, sequente vicesima prima, incidit illud plenilunium.
\lnr{22}Ergo non \textgreek{τῇ εἰκάδι βοηδρομιῶνος[?]},
 aut \textgreek{μυστηρίοις[?]},
 ut vult Plutarchus, sed \textgreek{τῇ εἰκάδι Μεταγειτνιῶνος[?]}
contigit defectus Lunaris.
\lnr{23}Plutarchus autem habebat
in animo Boedromionem Iulianum, cum haec scriberet.
\lnr{24}Non
enim raro ille confundit menses Iulianos cum antiquis vagis: ut in anno
Actiaco ostendemus.
\lnr{26}Haec satis fuerint de periodo Attica.
\lnr{26}Superest,
ut reddamus rationem, quare quartum Boedromionem Tetraeteridis
Atticae \textgreek{ἐξαιρεσιμαῖον[?]} fecimus, et non secundum, aut tertium.
\lnr{29}Sane si nihil aliud, ratio ipsa docet in ultimo anno aequationem
omnem fieri debere.
\lnr{30}Sed et id exemplis confirmatur.
\lnr{30}Plutarchus Plataeensem pugnam contigisse ait
 \textgreek{Βοηδρομιῶνος ἱσταμένου
τετράδι[?]}:
eandem vero pugnam in Camillo scribit \textgreek{τρίτῃ[?] Βοηδρομιῶνος
ἱσταμένου[?]}
% Inkspot under η or realy ῃ?
contigisse.
\lnr{33}Erat annus secundus Olympiadis 75, teste
Eusebio: proinde septuagesimus quartae periodi Olympicae, 32
autem Atticae.
\lnr{35}Ergo Boedromion ille incidit in quartum annum
Tetraeteridis Atticae.
\lnr{36}Et quia Plutarchus tertiam vocat, quam
antea quartam Boedromionis, non est dubium, quin Boedromion
quartus fuerit \textgreek{ἐξαιρεσιμαῖος[?]}.

% 47
% {PDF page nr}{source page nr}{line nr}
% Delayed page start
\setpnrs{130}{47}
\section{De Periodo Macedonica aestiva}

\plnr{130}{47}{1}Omnes civitates Graeciae periodum Olympiadicam ab initio
amplexae sunt religionis caussa, mutato alias contextu, alias
initio, cum aliae civitates ab aliis mensibus anni, quam astivis,
aliae ab alio anno, quam a primo Tetraeteridis, initium sumerent.
\lnr{4}Ideo
fiebat, ut apud diversas civitates idem mensis easdem
 \textgreek{ἀνάρχους ἡμέρας[?]}
sive \textgreek{ὑπερβαλλούσας[?]} retineret.
\lnr{6}De hac varia initii capiendi licentia ac
prope lascivia ita scribit Plutarchus in Aristide:
 \textgreek{τὴν δὲ τῶν ἡμερῶν ἀνωμαλίαν
οὐ θαυμαστέον, ὅπου καὶ νῦν διηκριβωμένων τῶν ἐν ἀστρολογίᾳ μᾶλλαν,
ἄλλοι ἄλλην τὴν μηνὸς ἀρχὴν καὶ τελευτὴν ἄγουσιν[?]}.
\lnr{9}Athenienses a diebus
quidem aestivis incipiebant.
\lnr{10}Sed eorum menses cum Elidensium
mensibus raro congruebant: imo nunquam neomeniae mensium periodicorum
congruebant, quatenus Atheniensium Tetraeteris incipiebat
a tertio anno Olympiadicae Tetraeteridis.
\lnr{13}Sic nunquam convenient.
\lnr{14}Quia primus mensis uniuscuiusque Tetraeteridis tantum est
Lunaris.
\lnr{15}At prisci Macedones nihil immutarunt in situ et contextu
periodi Olympicae.
\lnr{16}Atque, ut uno verbo dicam, eadem fuit Macedonica
cum Olympica.
\lnr{17}Id docemur ex Decreto Atheniensium, quod in
gratiam Hyrcani Iudaeorum Pontificis Maximi emissum est.
\lnr{18}Ex quo
haec, quae ad rem facere sunt visa, deprompsimus.
\textgreek{Επὶ πρυτάνεως, καὶ ἱερέως
Διονύςου τοῦ Ασκληπιάδου, μηνὸς Πανέμου πέμπτῃ ἀπιόντος, ἐπεδόθη τοῖς στρατηγοῖς
ψήφισμα Αθηναίων, ἐπὶ Αγαθοκλέους ἄρχοντος[?]}.
\lnr{21}\textgreek{Ευκλῆς Μενάνδρου
Αλιμούσιοσ ἐγραμμάτευε, Μουνυχιῶνος ἑνδεκάτῃ τὴς πρυτανείας[?]}.
\lnr{22}Munychion
iste, quia est mensis \textgreek{πρυτανείας[?]},
 ut verbis decreti expressum est,
erat Calippicus, et Lunaris: Panemus autem Macedonicus.
\lnr{24}A Macedonibus
enim subacti Athenienses, annum eorum, ut et iugum, accipere
coacti sunt.
\lnr{26}Quia igitur vicesima sexta Panemi periodici erat undecima
mensis Lunaris: Ergo novilunium inciderat in sextam decimam
Panemi.
\lnr{28}In Laterculo mensium Lunarium Tetraeteridos Atticae,
Munychion periodicus, qui est Panemus Macedonicus, habet novilunium
in sextadecima, in anno secundo Tetraeteridos Atticae.
% Table reference: to which table?
% "Laterculo mensium Lunarium Tetraeteridos Atticae"
\lnr{30}Igitur anno
secundo Tetraeteridis emissum est illud Psephisma.
\lnr{31}Demonstretur
ex ratione temporum, in quae conferendum est decretum.
\lnr{32}Erat enim,
ut scribit Iosephus, annus nonus facerdotii Hyrcani, quod ipse, eodem
auctore, iniverat anno primo Olympiadis 177.
\lnr{34}Fuit igitur anno 714
Iphiti ad veris tempus decurrente; et proinde decimae periodi Olympicae
tricesimo.
\lnr{36}Sed, ut diximus, Munychion incidit in annum
secundum Tetraeteridis.
\lnr{37}Fuit igitur Tetraeteris illa Olympica,
non Attica.
\lnr{38}Et propterea tota periodus Macedonica mere Olympica,
neque situ neque contextu mutato.

% 48
% {PDF page nr}{source page nr}{line nr}
\plnr{131}{48}{1}Cum enim Tetraeteris Attica
incipiat ab tertio anno Tetraeteridis Olympicae, ut iam dictum est, semper
% à -> ab
annus secundus Atticus erit quartus Olympicus, et contra.
\lnr{3}Hinc
sequitur omnino omnes Graecos, aut saltem maiorem eorum partem,
in eodem mense \textgreek{τὰς ὑπερβαλλούσασ ἡμέρας[?]}
 habuisse, hoc est in eo
mense, qui sextus esset a Tropico aestivo, ut ipsi loquebantur.
\lnr{6}Cum
vero Philippus Rex Macedoniae Amyntae filius, Alexandri Magni
pater, menses Metonicos in periodum Macedonicam recepisset,
periti Macedonum in eius gratiam periodum novam orsi sunt a diebus
verni sideris, cuius initium caderet in annum septuagesimum
secundum periodi Olympicae, initio sumpto a Daesio Macedonico,
sive Munychione Attico, tam Metonis, quam Tetraeterico.
% Insert table: Menses Macedomum et Atheniesium
\begin{table}[htbp]
  %%% Liber I p48
%%
%% Menses Macedonum names copied from Wikipedia: Ancient Macedonian calendar
%% then modified to match the original
%% Note that ὑπερβερεταῖος is normally considered the last month in the Macedonian
%% calendar, but appears first in this table.
%% - πάνεμος is written with an ε in the book, not an η or an α as given
%%   on Wikipedia
%% - Wikipedia gives Λώιος while the book appears to have λῶος (with a
%%   circumflex rather than an acute accent over the omega and without the iota)
%%
%% Menses Atheniensium names copied from Wikipedia: Attic calendar
%% then modified to match the original
%% - Μουνιχιών -> μουνυχιών (ι -> υ)
%% - Σκιροφοριών -> σκιῤῥοφοριών (double ρ)
%% Added numbers to ensure the reader knows the order of the months
%%
%%% Count out columns for fixed-width source font
% 000000011111111112222222222333333333344444444445555555555666666666677777777778
% 345678901234567890123456789012345678901234567890123456789012345678901234567890
%
%% Select a general font size (uncomment one from the list)
%\tiny
%\scriptsize
%\footnotesize
%\small
\normalsize
%% Center the whole table left-right
\centering
%% Modify separation between columns
%\setlength{\tabcolsep}{0.5em}
%% Modify distance between rows
%\renewcommand{\arraystretch}{0.85}
%%
\begin{tabular}{ l  l }
\toprule
\parbox[b]{6em}{Menses \\ Macedonum} &
\parbox[b]{6em}{Menses \\ Atheniensium} \\
\midrule
\textgreek{ὑπερβερεταῖος}  &\textgreek{ἑκατομβαιών} \\
\textgreek{δίος}           &\textgreek{μεταγειτνιών} \\
\textgreek{ἀπελλαῖος}      &\textgreek{βοηδρομιών} \\
%
\textgreek{αὐδυναῖος}      &\textgreek{πυανεψιών} \\
\textgreek{περίτιος}        &\textgreek{μαιμακτηριών} \\
\textgreek{δύστρος}        &\textgreek{ποσειδεών} \\
%
\textgreek{ξανθικός}       &\textgreek{γαμηλιών} \\
\textgreek{ἀρτεμίσιος}     &\textgreek{ανθεστηριών} \\
\textgreek{δαίσιος}        &\textgreek{ἐλαφηβολιών} \\
%
\textgreek{πάνεμος}        &\textgreek{μουνυχιών} \\
\textgreek{λῶος}          &\textgreek{θαργηλιών} \\
\textgreek{γορπιαῖος}      &\textgreek{σκιῤῥοφοριών} \\
\bottomrule
\end{tabular}
%
\caption{Menses Macedonum et Atheniensium}
\label{tab:p048}
%

\end{table}
\lnr{12}Convenerunt enim in unum neomenia periodica et
Metonica: et fuit verus Nisan anni Iudaici
3437.
\lnr{15}Cuius character 7.21.68. feria prima,
Martii vicesima sexta, cyclo Lunae primo,
Solis 22, anno periodi Iulianae 4390.
\lnr{17}Et
Hyperberetaeus sequentis Olympiadis 114
caepit in 24 Iulii.
\lnr{19}Quare, ut diximus, antecedens
Panemus convenit in 26 Martii.
\lnr{20}Haec
est periodus, quam Philippeam vocarunt ab
ipso Philippo, qui tamen eius initium non
vidit, ut neque filius eius Alexander autumnalem,
quam illi instituerant Syromacedones.
\lnr{25}Nam vidit quidem finem, sed initium, quod eius cognomine fuit, non
vidit.
\lnr{26}Menses periodi veteris Macedonicae cum Atticis comparatos
infra subiecimus.

\section{De Periodo Macedonum Alexandrea autumnali}
% Originally: Periodus Mecedonum Alexandrea Autumnalis
% Changed to match the ToC entry, and the other sections in this chapter.

\lnr{28}Hactenus Macedones integra periodo Olympica usi sunt: donec
Syromacedones aliam in gratiam Alexandri constituerunt,
quam ipse, ut iam ab nobis indicatum est, non vidit, cum eius initium
% à -> ab
incidat in annum Iphiteum 465.
\lnr{31}Calippus Cyzicenus, ut iam
diximus, cum \textgreek{ἐμβολισμοὺς καὶ δισεξαιρέσεις[?]}
 in novem octaeteridibus, et
semisse, hoc est in \rnum{xix} Tetraeteridibus fieri animadvertisset,
 anno 445
Iphiteo deprehendit Lunae coitum cum Sole convenire in epocham
veterem Olympiadis, hoc est, in \rnum{ix} Iulii, anno sexagesimo
 quinto periodi
Olympicae, vicesimo septimo periodi Atticae.
\lnr{36}Fuit enim Ab Iudaici
anni 3430, cuius character 6.13.217. feria sexta, cyclo Solis \rnum{xv},
Lunae \rnum{xiii}. Iulii nona.

% 49
% {PDF page nr}{source page nr}{line nr}
\plnr{132}{49}{1}Igitur statuit Hecatombaeonem tam periodicum
sive Tetraetericum, quam Lunarem in \rnum{ix} Iulii, et retexuit veterem
situm Olympiadicae, et Atticae periodi.
\lnr{3}Nam pro sexagesimo sexto periodi
sextae Olympiadicae, et vigesimo octavo Atticae primum annum
instituit periodi suae.
\lnr{5}Restituit igitur hoc modo in priscam epocham
Olympicam neomeniam Lunarem: sed non propterea hinc exorsus
est periodum, sed ab Pyanepsione proximo, qui caepit \rnum{vii} Octobris
% à => ab
propter pugnam ad Arbela, quae in id tempus incidit: eoque nomine
periodum suam Alexandream vocavit.
\lnr{9}Et quidem versimile est ipsum
non tacuisse, quasi divinitus Lunae \textgreek{σύνοδον[?]}
 in veterem epocham Olympicam
incurisse eo anno, quo Darium Alexander vicerit.
\lnr{11}Quod
enim ab Autumno eam instituerit, docet nos Ptolemaeus.
\lnr{12}Nam apud
eum eclipsis Lunae observatur, Sole in Virgine postito, anno quinquagesimo
quarto secundae periodi Calippicae.
\lnr{14}Post menses sex altera est
observata, anno quinquagesimo quinto, Sole Pisces obtinente.
\lnr{15}En
mutatio anni a Virgine ad Pisces.
\lnr{16}Rursus eodem anno quinquagesimoquinto,
tertia Eclipsis in Virgine facta.
\lnr{17}A Piscibus ad Virginem
non est facta anni mutatio.
\lnr{18}Non igitur caepit a Solstitio.
\lnr{18}A Virgine
ad Pisces fit mutatio.
\lnr{19}Ergo ab Autumno.
\lnr{19}Quod autem a Cyclo Lunae
\rnum{xiii} caeperit, ita demonstrabitur.
\lnr{20}Eclipsis trium prima incidit in
\rnum{xxii} Septembris, anno \rnum{liiii} definente,
 qui erat decimus cycli Lunaris.
\lnr{22}Ergo quinquagesimus quintus caepit ab eodem cyclo decimo.
\lnr{23}Et propterea primus caepit a cyclo \rnum{xiii}: et consequenter
 \rnum{xii} ad epocham
Calippi autumnalem addenda, ut habeas cyclum Lunarem.
\lnr{25}Incipit igitur ab eodem Autumno, quo memorabilis clades Persarum
ad Arbela contigit.
\lnr{26}Atque ita quidem Calippus.
\lnr{26}Sed non tamen
ideo obtinuit, ut statim nomine Alexandri procederet.
\lnr{27}Cyclo
enim Lunari vertente caepta est putari, annis duodecim post mortem
Alexandri, ideo, ut primus annus Tetraeteridis Alexandreae cum primo
Tetraeteridis Olympicae concurreret: et pro anno nono periodi
Iphiteae diceretur primus Alexandreae.
\lnr{31}Neque vero tantummodo in
hoc mutatio facta, sed etiam in mensibus.
\lnr{32}Tunc enim Hyperberetaeus
a diebus aestivis ad autumnales traductus est, et ut olim Olympiadici,
ita postea Alexandrei anni caput factus.
\lnr{34}Antiochus Magnus in
rescripto quodam de Iudaeis, ut extat apud Iosephum, annuas pensiones
remittit illi genti in triennium, usque ad Hyperberetaeum mensem:
ut omnino mensis, qui annum claudit, sit is, qui Hyperberetaeum
antecedit: \textgreek{ἵνα δὲ[?]}, inquit,
 \textgreek{θᾶττον ἡ πόλις κατοικισθῇ, δίδωμι τοῖς τε
νῦν, καὶ κατελευσομένοις, ἕως Υπερβερεταίου μηνὸς,
 ἀτελεῖς εἶναι μέχρι τριῶν ἐτῦν[?]}.
\lnr{40}Quod autem Antiochus Epiphanes Samaritanis scribens ita tempus
notarit: \textgreek{ἑκατομβαιῶνος μενὸς \gnum{ιϛ}[?]},
 equidem ingenue fateor me non intelligere:
Nisi, quod credo, fuerit is \textgreek{ἑκατομβαιὼν[?]} Calippicus Lunaris.

% 50
% {PDF page nr}{source page nr}{line nr}
\plnr{133}{50}{2}Et sane non dubito, quin verum sit.
\lnr{2}Porro hanc epocham Siri
% The elevated 'i' in 'Siri' in the original is a setting error.
% Earlier editions simply have 'Siri', without the elevated 'i'.
tenuerunt ad hunc usque diem a mense Octobri Iuliano, propter veterem
Hyperberetaeum: eamque epocham Aliscandria vocant, id est
Alexandream.
\lnr{5}Sed Arabes, et quaedam aliae nationes diu usurparunt
ab Elul, id est Septembri, propter Indictionem
Constantinianam.
\lnr{7}Eaque epoche vocata
est illis \textarabic{[Arabic]} \texthebrew{[Hebrew]}
Terich dilkarnain, hoc est \textgreek{ἐποχὴ τοῦ δικέρωτος[?]}.
% Insert table "Menses periodi Alexandreae Sycomacedonum"
\begin{table}[htbp]
  %%% Liber I p50
%%
%% The column on the left is identical to the Macedonum column in table
%% 047_menses_macedonum.
%% The column on the right is the same as the Atheniensium column in that
%% table, but shifted up by three months.
%%
%% For testing, uncomment the folowing lines and the lines at the end of the file
%% Test ==>
%\documentclass{book}
%\usepackage{fontspec}
%\setmainfont{Hoefler Text}[]
%\newfontfamily\greekfont{Arial Unicode MS}
%\usepackage[quiet]{polyglossia}
%\setmainlanguage{latin}
%\setotherlanguage{greek}
%\begin{document}
%% <== Test
%%
\begin{tabular}{ l  l }
\multicolumn{2}{ c }{Menses periodi}\\
\multicolumn{2}{ c }{Alexandreae}\\
\multicolumn{2}{ c }{Syromacedonum}\\
\hline
\textgreek{ὑπερβερεταῖος} &\textgreek{πυανεψιών} \\
\textgreek{δίος}           &\textgreek{μαιμακτηριών} \\
\textgreek{ἀπελλαῖος}      &\textgreek{ποσειδεών} \\
%
\textgreek{αὐδυναῖος}      &\textgreek{γαμηλιών} \\
\textgreek{περίτιος}        &\textgreek{ἀνθεστηριών} \\
\textgreek{δύστρος}        &\textgreek{ἐλαφηβολιών} \\
%
\textgreek{ξανθικός}       &\textgreek{μυονυχιών} \\
\textgreek{ἀρτεμίσιος}     &\textgreek{θαργηλιών} \\
\textgreek{δαίσιος}        &\textgreek{σκιῤῥοφοριών} \\
%
\textgreek{πάνεμος}        &\textgreek{ἑκατομβαιών} \\
\textgreek{λῶος}          &\textgreek{μεταγειτνιών} \\
\textgreek{γορπιαῖος}      &\textgreek{βοηδρομιών} \\
\end{tabular}
%% Test ==>
%\end{document}

\end{table}
\lnr{10}Alexandrum enim vocant \textgreek{δικέρωτα[?]}, ut infra
dicetur.
\lnr{11}Quod autem Plutarchus in Alexandro
scribit \textgreek{λῶον[?]} Macedonicum mensem esse
eumdem cum Hecatombaeone, hoc verum
fuerit, si \textgreek{λῶος[?]} Tetraetericus, Hecatombaeon
vero Lunaris, et Calippicus intelligatur.
\lnr{15}Nam
secundo anno periodi Alexandreae Hecatombaeon
Lunaris vel Calippicus occupabat Iulium, \textgreek{λῶος[?]} vero invadebat
Augustum.
\lnr{18}Nam menses Tetraeterici sunt tardiores Lunaribus in
annis secundo, tertio, quarto Tetraeteridos.

\section{De Periodo Bithynorum}

\lnr{20}Anno a Nabonassaro 840, Tybi \rnum{ii} absoluto, Agrippa astrologus
in Bythynia observavit Lunam iunctam cum australi succedente
Vergiliarum, qui erat annus \rnum{xii} Domitiani, Metroi
secundum Bithynos die \rnum{vii}.
% Insert table: "Menses Bithyniorum et menses Atheniensium"
\begin{table}[htbp]
  %%% Liber I p50
%%
%% The Bithyniorum column was entered from the original as well as possible.
%% The Athenienium column is a copy of that column from 047_menses_macedonum
%%
%% For testing, uncomment the folowing lines and the lines at the end of the file
%% Test ==>
%\documentclass{book}
%\usepackage{fontspec}
%\setmainfont{Hoefler Text}[]
%\newfontfamily\greekfont{Arial Unicode MS}
%\usepackage[quiet]{polyglossia}
%\setmainlanguage{latin}
%\setotherlanguage{greek}
%\begin{document}
%% <== Test
%%
\begin{tabular}{ l  l }
Menses                    & Menses\\
Bithyniorum               & Atheniensium \\
\hline
\textgreek{ἀφροδίσιος}    &\textgreek{ἑκατομβαιών} \\
\textgreek{δημήτριος [?]} &\textgreek{μεταγειτνιών} \\
\textgreek{ἡραῖος}        &\textgreek{βοηδρομιών} \\
%
\textgreek{ἑρμεῖος}        &\textgreek{πυανεψιών} \\
\textgreek{μητρῶος}       &\textgreek{μαιμακτηριών} \\
\textgreek{διονύοιος}      &\textgreek{ποσειδεών} \\
%
\textgreek{ἡρίκλειος}       &\textgreek{γαμηλιών} \\
\textgreek{δῖος}           &\textgreek{ανθεστηριών} \\
\textgreek{βενδιαῖος}        &\textgreek{ἐλαφηβολιών} \\
%
\textgreek{στρατεῖος}      &\textgreek{μουνυχιών} \\
\textgreek{ἄρειος}          &\textgreek{θαργηλιών} \\
\textgreek{περιέπειος}      &\textgreek{σκιῤῥοφοριών} \\
\end{tabular}
%% Test ==>
%\end{document}

\end{table}
\lnr{23}Tempus, \rnum{xxix} Novembris, anno
Christi vulgari 92:
\lnr{24}Ergo neomenia \textgreek{Μητρώου[?]} congruebat vicesimae tertiae
Novembris:
\lnr{25}Et proinde mensis aestivus 26 Iulii.
\lnr{25}Erat annus Iphiteus
868, et periodi duodecimae Olympicae
32.
\lnr{27}In qua mensis aestivus ter competit
vicesimae sextae Iulii, in annis 13, 30, 47.
\lnr{29}Sed cum annus propositus esset quartus Olympiadis,
annum quoque Bithynicum
aut quartum fuisse oportet Tetraeteridis Bithynicae,
aut secundum: quod aliter anni
Olympici non usurpantur in aliis periodis.
Sed nullus quartus Olympiadicae Tetraeteridis
habet initium in 26 Iulii.
\lnr{35}Erit igitur
annus 32 Olympiadicae periodi, 30 autem
Bithynicae.

% 51
% {PDF page nr}{source page nr}{line nr}
\plnr{134}{51}{1}Et propterea duo anni de Olympica periodo detrahendi ad
methodum periodi Bithynicae.
\lnr{2}Non dubium est, quin periodus haec
in gratiam Regum Bithyniae constructa sit cum mensibus Calippicis.
\lnr{4}Nos autem subiecimus Laterculum mensium Bithynicorum, ut in
manuscripto Graeco reperi, eosque cum Atticis comparavimus.

\section{De Periodo Delphorum Pythica}

\lnr{8}Per octaeteridas tempora sua periodica transegisse Graecos, puto
iam satis doculia ingenia capere posse.
\lnr{9}Sed \textgreek{εὐμεθοδ ευσίας[?]} gratia
nos circuitum embolismorum, quae est vera
 \textgreek{ἀποκατάστασις[?]} temporum,
orbe annorum 76 peragimus.
\lnr{11}Quo intervallo comprehenduntur
octaeterides novem cum dimidia, embolismi etiam novem, quot scilicet
sunt Octaeterides.
\lnr{13}Sane, ut Censorinus scribit, hunc circuitum
vere annum magnum esse pleraque Graecia existimavit, quod ex annis
octo vertentibus solidis eum constare existimarent.
\lnr{15}Adiicit idem
scriptor, ob hoc multas in Graecia religiones hoc intervallo temporis
summa caeremonia coli.
\lnr{17}Delphis quoque ludos, qui vocantur Pythia,
post annum octavum olim instaurari solita.
\lnr{18}Verum quidem est de intervallo
antiquorum Pythiorum.
\lnr{19}Sed fallitur, cum existimat eam octaeterida
meram Lunarem fuisse, quae habuit \textgreek{τριακονθημέρους[?]} menses omnes,
ut iam declaratum a nobis.
\lnr{21}Nam Pythiorum vetus institutio,
octaeteridis autem Lunaris recentissima, si quidem cum illorum ludorum
origine comparetur.
\lnr{23}Fallitur praeterea interpres Pindari, qui
inter duo Pythia antiqua ponit intervallum annos solidos Solares
novem.
\lnr{25}Nam octaeteris dicta est etiam \textgreek{ἐννεαετηρὶς[?]},
 quod nono quoque
anno recurrat.
\lnr{26}Itaque caput dicitur \textgreek{ἐννεαετηρὶς[?]}: intervallum,
\textgreek{ὀκταετηρὶς[?]}: quemadmodum intervallum duarum Olympiadum,
 \textgreek{τετραετηρὶς[?]}:
caput vero unius, \textgreek{πενταετηρὶς[?]}.
\lnr{28}Quare et Olympias ipsa
aliquando etiam Tetrapentaeteris dicta, ut est in Censorino Fr. Pithoei.[?]
% Abbriv.
\lnr{30}Est enim initium quinti anni.
\lnr{30}Quomodo etiam dicitur ab oraculo \textgreek{πενταέτης ἐνιαυτός[?]}.
\begin{verse}
\textgreek{Τὴν αὑτῶν ῥύεσθε πάτραν, πολέμου δ᾽ ἀπέχεσθε[?]},\\
\textgreek{Κοινοδίκου φιλίας ἡγούμενοι Ελλήνεσσιν[?]},\\
\textgreek{Εστ᾽ ἄν πενταέτης ἔλθῃ φιλόφρων ἐνιαυτός[?]}.
\end{verse}
\lnr{35}Delphi igitur tres \textgreek{ἐννεαετηρίδας[?]} continuas celebrabant.
\lnr{35}Prima vocabatur
\textgreek{σεπτήριον[?]}, altera \textgreek{ἡρωῒς[?]},
 tertia \textgreek{χάριλα[?]}, ut scribit Plutarchus, quae
nunquam mutatae fuerunt.
\lnr{37}Quarta \textgreek{ἐννεαετηρὶς[?]} erat \textgreek{τῶν πυθίων[?]}, quae
postea in \textgreek{τετραετηρίδα[?]} redacta,
 ut Olympicus agon, anno, ut scribit
Pausanias, Olympiadis 48 tertio.

% 52
% {PDF page nr}{source page nr}{line nr}
\plnr{135}{52}{1}Quo Amphictyones findicinae artis, ut
antiquitus, praemia constituerunt, adiectis etiam artis tibicinae.
\lnr{2}Nam non satis consulte Sophocles in Electra Pythicum agona dicit fuisse
tempore Orestis, in eoque Oresten curru deiectum periisse fingit.
\lnr{4}Sequenti
Tetraeteride, qui erat annus Olympiadis 49 tertius, \textgreek{ἀγὼν[?]}
 ille \textgreek{στεφανίτης[?]}
factus, ut Olympicus, unde prima Pythias putari solet.
\lnr{6}Quare recte
apud Eusebium anno \rnum{iii} Olympiadis 49 prima Pythias attribuitur.
\lnr{8}Cui adstipulatur Scholiastes priscus Pindari in Pythionicarum
\rnum{iii}. \textgreek{καθίστατο δὲ[?]}, inquit,
 \textgreek{ὁ Ιέρων βασιλεὺς κατὰ τὴν ἑβδομηκοστὲν ἕκτην Ολυμπιάδα
τὴς εἰκοστῆς ὀγδόης Πυθιάδος τῇ προκειμένῃ Ολυμπιάδι συγχρόνου
ὄυσης[?]}.
\lnr{11}Si ab anno Iphiti 195 putatur Pythias, et annus primus 28
Pythiadis est annus 109 ab prima Pythiade instituta, idem annus Pythicus
% à -> ab
erit 303 Iphiti.
\lnr{13}Qui est annus tertius propositae Olympiadis 76.
\lnr{14}Rursus idem Scholiastes \rnum{xii} Olympico:
 \textgreek{Εργοτέλης Κρὴς μὲν ἦν τῷ γένει,
πόλεως Κνωσσοῦ, ὅς ἠγωνίσατο ἑβδομηκοστὴν ἑβδόμην ὀλυμπιάδα, καὶ τὴν
ἑξῆς Πυθιάδα εἰκοστὴν ἐννάτην[?]}.
\lnr{16}Ergo Pythias celebrata 49. Olymp.[?]
% Abbreviation "Olymp."; what is the full word?
\lnr{16}Idem
Scholiastes \textgreek{εἴδει[?]} Pythionicarum \rnum{xiiii}.
\lnr{17}\textgreek{γέγραπται μὲν ἡ ᾠδὴ Αρκεσιλάῳ Πολυμνήστου
παιδὶ Κυρηναίῳ τὸ γένος τὴς Λίβυης[?]}
(vide num melius \textgreek{τὸ γήνος τε
λίβυι[?]})
\textgreek{νικήσαντι τὴν τριακοστὴν πρώτην Πυθιάδα[?]}.
\lnr{19}\textgreek{ἔνιοι καὶ τὴν ὀγδοηκοστὴν
Ολυμπιάδα[?]}.
\lnr{20}\textgreek{Αλλ᾽ οὐκ ἔγραψεν εἰς τὴν Ολυμπιακὴν άυτοῦ νίκην,
 καί τοι μετὰ
τὴν Πυθικὴν γενομένην[?]}
\lnr{21}Ait tricesimam primam \textgreek{Πυθιάδα[?]} esse priorem
octagesima Olympiade.
\lnr{22}Nam tricesima prima Pythias celebrata
est anno Iphiti 315, qui erat tertius Olympiadis septuagesimae nonae.
\lnr{24}Ab annis igitur Iphiti aufer 194.
\lnr{24}Habebis annos Pythicos.
\lnr{24}Coepit
enim, ut diximus, primus \textgreek{ἀγὼν[?]} Pythiorum
 \textgreek{στεφανίτης[?]} anno Iphiti Olympiadico
195, periodi Iulianae 4132, cyclo Lunae decimo, Solis septimo
decimo, sexta mensis, qui apud illos \textgreek{Βύσιος[?]}, apud Athenienses
Thargelion dicitur.
\lnr{28}Nam \textgreek{τῇ ἕκτῃ τοῦ ἱσταμένου[?]} Apollinem natum volunt.
\lnr{29}Quamuis \textgreek{Θαργήλια[?]} Athenis
 \textgreek{τῇ ἑβδόμῃ τοῦ Θαργηλιῶνος[?]} celebrari solita
scribit Plutarchus libro Symposiacon octavo, capite primo: unde
Apollo dictus sit \textgreek{ἑβδομαγενής[?]}.
\lnr{31}Pythiadum periodus ab initio Olympiadum
contexta potest ab eo anno incipere, a quo Attica, hoc est a
tertio Iphiti, cyclo Lunae \rnum{vii},
 Solis \rnum{xx}, tempore veris praecipitati.
\lnr{34}Pindarus Olympionicarum \rnum{xiii}.
\lnr{34}\textgreek{Πυθοῖτ᾽ ἔχει σταδίου τιμὰν, διαύλουθ᾽ ἁλίῳ
ἀμφ᾽ ἑνὶ[?]}.
\lnr{35}\textgreek{Μηνός τε οἱ τωὐτοῦ κρανααῖς ἐν Αθάναισι[?]}.
\lnr{35}Manifesto innuit eodem
mense Panathenaea et Pythia instaurari solita.
\lnr{36}Quod si verum
est, ergo Panathenaea Thargelione, non autem Hecatombaeone celebrabantur:
nisi forte dicamus aliud solenne Atticum hic intelligi.
% Below: only in 1629 edition
\lnr{39}Et cur non hic aliud ludicrum a Panathenaeis intelligamus, cum constet
ex oratione \textgreek{κατὰ Τιμοκράτοις[?]} Panathenaea celebrari solita
 \textgreek{τῇ δωδεκάτῃ
Εκατομβαιῶνος[?]},
quemadmodum Olympiadem \textgreek{δωδεκάτῃ Υπερβερεταίου[?]} Macedonici?

% 53
% {PDF page nr}{source page nr}{line nr}
\plnr{136}{53}{1}Utrumque autem ludicrum tam Panathenaicum,
quam Olympicum quatuor diebus transigebatur.
\lnr{2}Ultima dies ludicri
\rnum{xv} mensis, plenilunio.
\lnr{3}Atque haec quidem dies unica erat prisco solenni
dedicata.
\lnr{4}Sed postquam proposita fuerunt praemia saltus, disci,
% Stain at the end of "disci,"
pugilatus, Pancratii, et luctae, triduum priscae solennitati praepositum
est.
\lnr{6}Itaque iam a duodecima die mensis incipiebant, desinebant
in \rnum{xv}, quae, ut diximus, erat  plenilunium, Postridie autem, hoc
est, \textgreek{τῇ ἕκτῃ ἐπὶ δέκα[?]}, erat \textgreek{κρίσις[?]}.
\lnr{8}Proinde \textgreek{ἡ δωδεκάτη[?]} erat \textgreek{ἱερομηνία πρῶτη
τῶν Παναθηναίων[?]}, quae etiam dicebatur \textgreek{Κρόνια[?]},
 ut extat in eadem
oratione.
\lnr{10}Quintadecima erat magna \textgreek{ἱερομηνία[?]}.
\lnr{10}In Psephismate Timocratis
Panathenaea coniiciuntur in \textgreek{δωδεκάτην πρυτανείας[?]}: hoc est, in
duodecimam diem mensis Lunaris: nempe quia mensis primus Tetraeteridis
erat Lunaris merus, et conveniebat cum mense Prytanias.
% above: only in 1629 edition
\lnr{14}Tamen utut haec intelligamus, videtur ex his quoque colligi posse
Atticam periodum a tertio anno Tetraeteridis Olympiacae caepisse:
ut supra a nobis demonstratum est.
\lnr{16}An vero idem caput ea periodus
cum Attica habuerit, nimirum a Bruma, an a Solstitio, id vero nobis
hactenus incompertum est.
\lnr{18}Sed cum veteres scribant Apollinem
\textgreek{τῇ ἕκτῃ τὴς σελήνης[?]}, ut Herculem \textgreek{τῇ τετράδι[?]} natum,
 verisimile est periodum
coepisse a vere adulto: hoc est a mense secondo verno, eumque
mensem, ut omnes primos Tetraeteridis, mere Lunarem fuisse.
% Below: only in 1629 edition
\lnr{22}Sed propius vero est, Apollinem natum septima Thargelionis in anno
quarto Tetraeteridis Pythicae, quo tempore novilunium est secunda
Thargelionis.
\lnr{24}Quae proculdubio vera est sententia.
\lnr{24}Nam Apollo dicitur
natus \textgreek{τῇ ἕκτῃ μηνὸς κατὰ σελήνην[?]},
 ut fere omnes veteres prodiderunt.
\lnr{26}Et tamen dicitur \textgreek{ἑβδομαγενής[?]}.
\lnr{26}Si natus est \textgreek{τῇ ἕκτῃ[?]},
 quomodo igitur \textgreek{ἑβδομαγενὴς[?]}
dicitur?
\lnr{27}Iam caussam aperuimus.
\lnr{27}Natus est \textgreek{τῇ ἕκτῃ κατὰ σελήνην[?]},
sed \textgreek{τῇ ἑβδομαδι Θαργηλιῶνος ἱσταμένου[?]}:
 nempe \textgreek{τῇ ἕκτῃ τὴς πρυτανείας[?]},
\textgreek{τῇ ἑβδόμῃ Θαργηλιῶνος[?]} Tetraeterici.
\lnr{29}Ergo Thargelia celebrabantur
\textgreek{τῇ ἕκτῃ Θαργηλιῶνος ἱσταμένου[?]}.
\lnr{30}Anni secundi Olympiadis: Pythia autem
diebus aestivis tertii sequentis.
% Above: only in 1629 edition
\lnr{31}Pythiorum aetate sua celebratorum meminit
Plutarchus \textgreek{ἐν τῷ περὶ ἐκλελοιπότων χρηστηρίων[?]}.
\lnr{32}Nemea autem et
Isthmia erant trieterica, non pentaeterica, ut auctor est Scholiastes
Pindari.
\lnr{34}Quare fallitur Ausonius, qui Agonas quatuor omnes pentaetericos
constituit.
\begin{verse}
\textit{Haec quoque temporibus quinquennia sacra notandis}\\
\textit{Isthmia Neptuno data sunt, et Pythia Phoebo.}
\end{verse}
\lnr{38}Aliud vero est quinquennium, aliud tempus quinquennale.
\lnr{38}Quinquennium est intervallum quinque annorum solidorum: quinquennale
est, quod quinto anno incipit.
\lnr{40}In eundem lapidem offendit
Ouidius, qui Olympiada ex solidis quinque annis constare dicit, ut
alibi ostendimus.

% 54
% {PDF page nr}{source page nr}{line nr}
\plnr{137}{54}{1}Sed quare Latinis succenseamus, cum Pausanias scribat
Olympicum agonem ob id ab Hercule \textgreek{διὰ πέμπτου ἔτοις[?]} celebrari
constitutum fuisse, quod ipse cum fratribus, essent numero quinque?
\lnr{4}Hoc verum poterat esse, si intervallum unius Olympiadis essent anni
quinque.
\lnr{5}Et Graeci dicunt \textgreek{τὴν ὀλυμπιάδα διὰ πεντε ἐτῶν ἄγεσθαι[?]}, non
autem \textgreek{μετὰ πέντε ἔτη[?]}.
\lnr{6}Quare non solum erravit Ausonius, quod Isthmia
et Nemea post totidem annos cum Olympiade instaurari putat, sed
etiam quod Olympiada vocat quinquennium.

\section{De Periodo Thebana}

\lnr{9}Vetustissimorum Boeotiorum annus caepit a Vergiliarum
ortu matutino, in Thargelione Attico.
\lnr{10}Hesiodus:
\begin{verse}
\textgreek{Πληϊάδων ἀτλαγενέων ἀνατελλομενάων[?]}\\ % [ἐπιτελλομενάων]
\textgreek{ἄρχεσθ’ ἀμητοῦ, ἀρότοιο δὲ δυσσομενάων.[?]}\\
\textgreek{αἳ δή τοι νυκτάς τε καὶ ἤματα τεσσαράκοντα[?]}\\
\textgreek{κεκρύφαται. αὖθις δὲ περιπλομένου ἐνιαυτοῦ[?]}\\
\textgreek{φαίνονται[?]}---.\\
%\textgreek{[τὰ πρῶτα χαρασσομένοιο σιδήρου]}.
\end{verse}
% Hesiod, "Works and Days", line 383-387
% "When the Pleiades, daughters of Atlas, are rising,
% begin your harvest, and your ploughing when they are going to set.
% Forty nights and days they are hidden and appear again as the year
% moves round, [when first you sharpen your sickle.]"
% Translation from Evelyn-White, H.G. (1914): "Hesiod, the Homeric hymns
% and Homerica, page 31
\lnr{16}Dies viginti ante, quam Sol Taurum ingrediatur, et totidem post,
assignat occultationi Pleiadum, et postea incipere dicit annum Thebanum:
quamuis neque Proclus, neque eius Mastix Tzetzes mecum
faciant.
\lnr{19}Sed clarius idem poëta alibi:
\begin{verse}
\textgreek{ταῦτα φυλασσόμενος τετελεσμένον εἰς ἐνιαυτὸν[?]}\\
\textgreek{ἰσοῦσθαι νύκτάς τε καὶ ἤματα, εἰσόκεν αὖθις[?]}\\
\textgreek{γῆ πάντων μήτηρ καρπὸν σύμμικτον ἐνείκῃ,[?]}\\
\textgreek{εὖτ᾽ ἂν ἑξήκοντα μετὰ τροπὰς ἠελίοιο[?]}\\
\textgreek{χειμέρι᾽ ἐκτελέσῃ Ζεὺς ἤματα.[?]}---
\end{verse}
% Hesiod, "Works and Days", line 561-564
% "Observe all this until the
% year is ended and you have nights and days of equal
% length, and Earth, the mother of all, bears again her
% various fruit.
%  When Zeus has finished sixty wintry days after
% the solstice, ---"
% Translation from Evelyn-White, H.G. (1914): "Hesiod, the Homeric hymns
% and Homerica, page 45
Manifesto vocat \textgreek{τετελεσμένον ἐνιαυτὸν[?]} ver adultum,
 nimirum \textgreek{ἀπὸ τὴς τῶν Πλειάδων ἐπιτολῆς[?]}:
ut omnino videatur idem annus Hesiodeus cum
Pythio Delphorum fuisse, nempe a Thargelione.
\lnr{27}Quare Virgilius
praeceptorem Hesiodum, ut in agricultura, ita in anni principio
instituendo, sequitur:
\begin{verse}
  \textit{Candidus auratis aperit cum cornibus annum}\\
  \textit{Taurus.}---
\end{verse}
\lnr{32}Neque enim haec interpretes Virgiliani magis assequuntur, quam illa
Hesiodea Proclus, et Tzetzes.
\lnr{33}Sed haec ante tempora Olympica, siquidem
Hesiodus vetustior prima Olympiade.
\lnr{34}Nunc loquendum de
anno periodico Thebanorum, florentibus rebus Atheniensium, Laconum,
et ipsorum Thebanorum.
\lnr{36}Is igitur capiebat initium \textgreek{μετὰ τροπὰς
χειμερινάς[?]}.
\lnr{37}Arbitror igitur duodecimum, qui erat idem cum Posideone,
\textgreek{τὰς ἀνάρχους ἡμέρας[?]} habuisse: quod fine anni magistratus
crearetur.

% 55
% {PDF page nr}{source page nr}{line nr}
\plnr{138}{55}{1}Plutarchus Pelopida:
 \textgreek{καί τοι χειμῶνος [μὲν] ἦσαν αἱ περὶ τροπὰς
ἀκμαί, μηνὸς δὲ τοῦ τελευταίου φθίνοντος ὀλίγαι περιῆσαν ἡμέραι, καὶ
τὴν ἀρχὴν ἔδει παραλαμβάνειν ἑτέρους εὐθὺς ἱσταμένου τοῦ πρώτου μηνός, ἢ
θνήσκειν τοὺς μὴ παραδιδόντας.[?]}.
% Plutarch: Parallel Lives; Pelopidas (and Marcellus) 24, section 1, phrase 2:
% "Still, the winter solstice was at hand, and only a few days of
% the latter part of the last month of the year remained, and as soon as
% the first month of the new year began other officials must succeed them,
% or those who would not surrender their office must die."
% From: Plutarch's Lives, with an English Translation by Bernadotte Perrin.
% Cambridge, MA. Harvard University Press.
% London. William Heinemann Ltd. 1917. 5. 
\lnr{4}Quare, ut dixi, ibi erant \textgreek{αἱ ὑπερβάλλουσαι
ἡμέραι[?]}, ut Athenis fine Posideonis.
\lnr{5}Cur enim alibi, si mensis sequens
est initium anni?
\lnr{6}Primus mensis dicebatur \textgreek{Βουκάτιος[?]}, respondens
Gamelioni Attico.
\lnr{7}Plutarchus ibidem: \textgreek{τοῦ νόμου κελεύοντος ἐν τῷ πρώτῳ
μηνὶ παραδοῦναι τὴν Bοιωταρχίαν ἑτέροις, ὃν Βουκάτιον ὀνομάζουσι[?]}.
% Plutarch: Parallel Lives; Pelopidas 25, section 1, phrase 2:
% "For both were tried for their lives when they came back, because
% they had not handed over to others their office of boeotarch, as the law
% commanded, in the first month of the new year (which they call Boukatios)"
\lnr{8}Hunc
autem eundem, quem Hesiodus \textgreek{ληναιῶνα[?]} vocat,
 esse putat Plutarchus.
Hesychius, \textgreek{Ληναιών[?]}.
\lnr{10}\textgreek{Οὐδένα μὲν τῶν μηνῶν Βοιωτοὶ οὕτω καλοῦσιν,
εἰκάζει δὲ Πλούταρχος Βουκάτιον[?]}.
\lnr{11}\textgreek{καὶ γὰρ ψυχρός ἐστιν[?]}, et cetera.
% Hesychius of Athens, Lexicon, lemma "Ληναιών"
\lnr{11}Secundus mensis
erat \textgreek{Ερμαῖος[?]}.
\lnr{12}Sequitur apud Hesychium: \textgreek{ἔνιοι δὲ τὸν Ερμαῖον, ὅς
μετὰ τὸν Βουκάτιόν ἐστι[?]}.
\lnr{13}\textgreek{Καὶ γὰρ Αθηναῖοι τὴν τῶν Ληναίων ἑορτὴν ἐν ἀυτῷ ἄγουσι[?]}.
\lnr{14}Septimus mensis \textgreek{Ιπποδρόμιος[?]},
 si Plutarcho homini Boeotio credimus:
qui ait pugnam Leuctricam contigisse, Boeotiis \textgreek{Ιπποδρομίου
μηνὸσ[?]}, Atheniensibus \textgreek{Εκατομβαιῶνος ἱσταμένου τῇ πέμπτῃ[?]}.
\lnr{16}Nonus \textgreek{Πάνεμος[?]}
respondens Boedromioni.
\lnr{17}Decimus \textgreek{Δαμάτριος[?]} idem cum
Pyanepsione.
\lnr{18}Undecimus \textgreek{Αλαλκομένιος[?]} idem cum Maemacterione.
\lnr{19}Reliquos menses nondum reperi.
\lnr{19}Caput autem et situm huius periodi
difficile est investigare, propter penuriam exemplorum.
\lnr{20}Plutarchus
in Aristide de pugna Plataeensi: \textgreek{ταύτην τὴν μάχην ἐμαχέσαντο
τῆ τετράδι τοῦ Βοηδρομιῶνος ἱσταμένου κατ᾽ Αθηναίους, κατὰ δὲ Βοιωτούς τετράδι
τοῦ Πανέμου φθίνοντος[?]}.
\lnr{23}Erat annus secundus Olympiadis \rnum{lxxv}, et
proinde 298 Iphiteus, periodi Atticae 32: cuius Boedromion \rnum{xvii}
Septembris, utique \textgreek{ἐξαιρεσιμαῖος[?]}.
\lnr{25}Ac propterea quarta Boedromionis
politica erat vera tertia: quod et ipse Plutarchus, tanquam interpres
nos docet in Camillo: \textgreek{Πέρσαι μηνὸς Βοηδρομιῶνος ἕκτῃ μὲν Μαραθῶνι,
τρίτῃ δ᾽ ἐν Πλαταιαῖς ἅμα καὶ περὶ Μυκάλην ἡττήθησαν[?]}.
\lnr{28}Hic \textgreek{τρίτην[?]}
vocat, quam in Aristide \textgreek{τετράδα[?]},
 nimirum propter \textgreek{ἐξαίρεσιν[?]}.
\lnr{30}Boedromion illius anni \rnum{xvii} Septembris,
 et propterea \textgreek{ἡ τετρὰς[?]}, habita
ratione \textgreek{τὴς ὲξαιρἐσεως[?]}, hoc est, \textgreek{ἡ τρίτη[?]},
 \rnum{xix} Septembris.
\lnr{21}Ergo, cum
\rnum{xix} Septembris fuerit \rnum{xxvii} Panemi Thebani, Neomenia Panemi
congruit vicesimae quartae Augusti.
\lnr{33}Et proinde \rnum{xxv} Iulii fuerit
neomenia mensis primi aestivi, quem Plutarchus dicit Hippodromion
a Thebanis nominari.
\lnr{35}Sed bis tantum mensis aestivi caput in
periodo Olympica incidit in \rnum{xxv} Iulii, nempe anno trigesimo nono,
et sexagesimo quarto.
\lnr{37}Annis igitur periodi Atticae addenda sunt 32,
ut habeas annos periodi Thebanae: aut de annis Olympiacis deducendi
sunt anni 6.
\lnr{39}Et sic falsum fuerit, Panemum verum esse Boedromionem,
sed medium inter Metagitnionem et Boedromionem.
\lnr{41}Rursus idem Plutarchus de pugna Leuctrica, quae coniicitur in tempus
secundi anni Olympiadis 102 ita scribit: \textgreek{τοῦτο μὲν συνέβη Βοιωτοῖς
Ιπποδρομίου μηνὸς, ῶς δ᾽ Αθηναῖοι καλοῦσιν, Εκατομβαιῶνος ἱσταμένου
πέμπτῃ[?]}.

% 56
% {PDF page nr}{source page nr}{line nr}
\plnr{139}{56}{3}Manifestum enim, ut \textgreek{Εκατομβαιῶνος[?]},
 ita etiam, \textgreek{Ιπποδρομίου
ἱσταμένου[?]} intelligi.
\lnr{4}Sed quia non apposuit \textgreek{τὴν ποστιαίαν τοῦ Ιπποδρομίου[?]}, nihil
possumus statuere.

\section{De Periodo Syracusana}

\lnr{6}Anno quarto Olympiadis 91, qui erat decimus nonus ab initio
belli Peloponesiaci, Demosthene et Nicia Atheniensium ducibus
bellum in Sicilia administrantibus, defecit Luna, nocte,
quae secuta est vicesimam septimam Augusti, anno periodi Iulianae
4301.
\lnr{10}Et statim nocte altera Nicias inhoneste sese recipiens
ex superstitione defectus, hostibus se cum toto fere exercitu interficiendum
obiecit.
\lnr{12}Polybius Excerptis libri \rnum{ix}.
\lnr{12}\textgreek{Καὶ μὲν Νικίας ὁ
τῶν Αθηναίων στρατηγὸς δυνάμενος σώζειν τὸ περὶ τὰς Συρακούσας στράτευμα,
καὶ λαβὼν τῆς νυκτὸς ἁρμόζοντα καιρὸν εἰς τὸ λαθεῖν τοὺς πολεμίους,
ἀποχωρήσας εἰς ἀσφαλὲς, κᾄπειτα τῆς σελένες ἐκλιπούσης, ὥς τι δεινὸν
προσημαινούσης, ἐπέσχε τὴν ἀναζυγὴν[?]}, et cetera.
\lnr{16}Haec autem eclipsis confertur
a Plutarcho in \textgreek{Καρνείου μηνὸς τετάρτην φθίνοντος[?]}.
\lnr{17}Et postea
Carneum eundem cum Metagitnione facit.
\lnr{18}Erat annus periodi Olympicae
60, ac propterea Atticae vicesimus secundus: cuius Hecatombaeon
25 Iulii: Metagitnion 24 Augusti.
\lnr{20}Contigit igitur
eclipsis nocte, quae secuta est quartam diem Metagitnionis.
\lnr{21}Hoc
est, \textgreek{τετάρτῃ μεταγειτνιῶνος ἱσταμένου[?]}.
\lnr{22}Iccirco neomenia \textgreek{Καρνείου[?]} incidit
in Kalendas Augusti.
\lnr{23}Fuit igitur \textgreek{Καρνεῖος[?]} illius anni aliquantum
particeps Metagitnionis.
\lnr{24}Sed ipse re vera est Hecatombaeon
Atticus.
\lnr{25}Ut autem neomenia Hecatombaeonis incidat in 10 Iulii,
hoc non potest contingere, nisi in anno periodi Olympicae 60.
\lnr{26}Atqui
is annus erat sexagesimus periodi Olympiacae.
\lnr{27}Ergo indubitate
periodus Syracusana eadem fuit cum Olympica.
\lnr{28}Quam postea
Archimedes mensibus Calippicis illustravit, edito de ea re libro
cum fabrica Sphaerae, exemplo Eudoxi, et Calippi, quorum alter
in sua Octaeteride, alter in sua periodo 76 annorum aliquot
circulos excogitavit, ut illis
 \textgreek{τὰς ἀποκαταστάσεις τῶν φαινομένων[?]} ob oculos
proponeret.
\lnr{33}In hoc nihil de suo attulit, praeter fortasse maiorem,
aut minorem numerum circulorum, ut facile est ab aliis inventa expolire.
\lnr{35}Huius Sphaerae, praeter vetustissimos, meminit et Claudianus
luculento[?] epigrammate.
% luculento or Iuculento ? Luculento is a word
% (light, bright, excelent, splendid)

% 57
% {PDF page nr}{source page nr}{line nr}
\plnr{140}{57}{1}Et sane falluntur, qui scribunt, sphaeram
sive, \textgreek{στερεὰν[?]}, sive \textgreek{κρικωτὴν[?]},
 ab Archimede inventam: quo nihil falsius
excogitari potuit.
\lnr{3}Propter utilitatem periodi et sphaerae \textgreek{πρᾶξιν[?]}, in memoriam
tanti benificii, Syracusani tumulo Archimedis a Romanis
militibus interfecti Sphaeram insculpi curarunt: quo indicio deprehensus
eius tumulus a Cicerone quaestore, cum iam non solum neglectus,
sed et ignoratus civibus fuis esset.
\lnr{7}Meminit periodi Archimedeae
et Virgilius:
\begin{verse}
  \textit{In medio duo signa, Conon, et quius fuit alter,}\\
  \textit{Descripsit totum radio qui gentibus orbem,}\\
  \textit{Tempora quae messor, quae curuus arator haberet?}
  % beta-like character in 'messor'. Interpreted as double-s
\end{verse}
\lnr{12}Nam Archimedes descripsit Fastos, quos adiunxit fabricae sphaerae,
exemplo familiaris sui Cononis Samij, quem cum ipso Archimede
coniunxit Virgilius.
\lnr{14}Orbis enim hic significat periodum, ut Manilio
de Metone loquenti Orbis est \textgreek{περίοδος ἐννεδεκαετηρική[?]}:
\begin{verse}
--- \textit{Caelumque novum versabit in orbem.}
\end{verse}
% Marcus Manilius: Astronomica, Liber IV, linea 268
\lnr{17}Interpretes non bene affecuti sunt mentem Virgilianam.
\lnr{17}Pappus ulimo
libro scribit, ab Archimede in mechanicis nihil scriptum praeter
opus de Sphaera.
\lnr{19}At quanuis eum librum nobis eripuerit vetustas, tamen
scimus, in eo opere non solum de fabricanda Sphaera periodica
tractatum fuisse, sed etiam fastos annorum septuaginta sex illo libro
complexum fuisse, nimirum Tempora, quae messor, quae curuus arator
haberet: quod parapegma vocari alibi docemus.
\lnr{23}Porro Agragantinis
erat idem mensis \textgreek{Καρνεῖος[?]} cum Syracusanis communis, et, ut
puto, geminus, ut Posideon Athenis.
\lnr{25}Nam, ni fallor, \textgreek{δίμηνος Καρνεῖος[?]}
eo nomine intelligitur in veteri inscriptione Graeca, \textgreek{ΕΚΤΑΖ[?]}.
\textgreek{ΔΙΜΗΝΟΥ[?]}. \textgreek{ΚΑΡΝΕΙΟΥ[?]}.
 \textgreek{ΕΞΗΚΟΝΤΟΖ[?]}. hoc est, sexta
bimenstris Carnei definentis.
% 1589 edition: ΕΚΤΑΣ, , ,ΕΞΗΚΟΝΤΟΣ
\lnr{28}\textgreek{τὴς ἕκτης διμήνου Καρνείου φθίνοντος[?]}.

\section{De Periodo Laconum}

\lnr{29}Anno tertio Olympiadis octagesimae nonae sub finem hiemis,
vere ineunte, foedus annorum quinquaginta inter Anthenienses
et Lacedaemonios ictum est, quod tempus ita designat Thucydides;
scribens \textgreek{τὰς σπονδὰς πεντηκοντούτεις[?]} initas fuisse,
 \textgreek{Πλειστόλα Σπάρτης
ἐφορεύοντος, Αρτεμισίου τετράδι φθίνοντος, Αλκαίου δὲ Αθήνῃσιν ἄρχοντος,
 Ελαφηβολιῶνος
ἕκτῃ φθίνοντος[?]}.
\lnr{34}Erat annus \rnum{li} periodi Olympicae, \rnum{xiii} Atticae.
\lnr{35}A periodo Olympica aufer 12.
\lnr{35}Erat enim annus 39 periodi Laconicae.
\lnr{36}Elaphebolion \rnum{xxv} Martii.
\lnr{36}Ergo Artemisius Laconicus \rnum{xxiii} Martii.
\lnr{37}In periodo Olympica hoc non accidit, nisi annis, 38, 63.
\lnr{37}Sed non congruit
cum 38, quia is annus est secundus Tetraeteridis, hic vero est
tertius.

% 58
% {PDF page nr}{source page nr}{line nr}
\plnr{141}{58}{2}Ergo annus Tetraeteridis Laconicae idem fuit cum
 anno Olympicae:
ac proinde 12 addenda annis periodi Olympicae, ut habeas
annum Laconicum.
\lnr{4}Sed non propterea congruerit cyclo.
\lnr{4}Quare
haec ignota sunt nobis.
\lnr{5}Studiosi possunt menses Laconum ex veteribus
scriptoribus colligere.
\lnr{6}Nam mihi quidem nullus succurrit, praeter
Phliasion.
\lnr{7}Stephanus in \textgreek{φλιοῦς[?]}.
\lnr{7}\textgreek{Ωνόμασται δὲ παρὰ τὸ φλεῖν, ὅ ἐστιν δὐκαρπεῖν[?]}.
\lnr{8}\textgreek{Λακεδαιμόνιοι δὲ τῶν μηνῶν ἕνα Φλιάσιον καλοῦσιν,
 ἐν ᾧ τοὺς τὴς γῆς
καρποὺς ἀκμάζειν συμβέβηκεν[?]}.
\lnr{9}Sed et \textgreek{Κάρνειος[?]} videtur esse mensis primus,
in cuius \textgreek{πανσελήνῳ[?]} celebrabantur \textgreek{τὰ Κάρνεια[?]}.
\lnr{10}Euripides Alcestide:
\begin{verse}
  \textgreek{Πολλάσε μουσοπόλοι[?]}\\
  \textgreek{Μέλφουσι, καθ᾽ ἑπτατονόν τ᾽ ὀρείαν[?]}\\
  \textgreek{Χέλυν, ἐν τ᾿ ἀλύροις κλέιοντες ὕμνοις,[?]}\\
  \textgreek{Σπάρτᾳ κυκλὰς ἁνίκα Καρνείου[?]}\\
  \textgreek{Περινίσσεται ὥρα[?]}\\
  \textgreek{Μηνὸς, ἀειρομένας[?]}\\
  \textgreek{Παννύχου σελάνας.[?]}
\end{verse}
% Euripides, Alcestis, lines 445-451
% http://www.loebclassics.com/view/euripides-alcestis/1994/pb_LCL012.199.xml
\lnr{18}Vulgo male editum \textgreek{κύκλος[?]} pro \textgreek{κυκλάς[?]}.
\lnr{18}Hic \textgreek{τὴν ὥραν κυκλάδα[?]} intelligit
\textgreek{τὴν τετραετηρίδα[?]}, qua vertente recurrebat
 \textgreek{ἡ πανσέληνος[?]} in eandem
diem eiusdem mensis.
\lnr{20}Poteris et aliarum nationum periodos
conferre cum Olympica, quae est certissima et exactissima omnium
reliquarum regula.

\section{De Periodo Samiorum et aliarum Graeciae civitatium}
% Originally: Periodus Samiorum, et aliarum Graeciae civitatium
% Changed to match the ToC entry and the other sections

\lnr{23}Samiis fuis periodum instituit primus omnium Aristarchus, anno
quinquagesimo periodi primae Calippicae, octo cyclis Metonicis
post observationem solstitii a Metone et Euctemone factam,
anno Iphiteo currente 496.
\lnr{26}Et propterea cum fuerit ultimus Olympiadis
124, et post mensem incipere debuerit Olympias 125, non est
% "124" clearer in other editions
dubium, quin periodus Samiorum Aristarchea inde caeperit, hoc est,
ut annus 41 periodi Olympicae fit primus Samiae.
\lnr{29}Accomodavit igitur
illi menses Calippicos \textgreek{κατὰ σελήνην[?]},
 et ita civibus suis proposuit, donec
quinquaginta septem post annis Conon popularis Aristarchi
Fastos edidit, et parapegmata, hoc est,
 \textgreek{φαινομένων ἀποκαταστάσεις[?]}, illi periodo
apposuit.
\lnr{33}Quod non obscure innuit Virgilius, \textit{In medio duo signa,
Conon}, et cetera.
\lnr{34}Sed clarius de Fastis et phaenomenis Catullus ex Callimacho:
\textit{Qui stellarum ortus comperit atque obitus}.
\lnr{35}Porro et ad hoc
exemplum reliquae omnes civitates suas periodos habuerunt, et omnes
\textgreek{τὰς ἀνάρχους ἡμέρας[?]} mensi ultimo autumnali annectebant.
\lnr{37}Neque
vero impedit, quod legimus, magistratus alios aliis temporibus
creari solitos.

% 59
% {PDF page nr}{source page nr}{line nr}
\plnr{142}{59}{2}Nam ii menses erant Lunares \textgreek{τὴς πρυτανείας[?]}.
\lnr{2}Sic Polybius
de Aetolis lib. \rnum{iiii}.
 \textgreek{τὰς γὰρ ἀρχαιρεσίας Αίτωλοὶ μὲν ἐποίουν μετὰ
τὴν φθινοπωρινὴν ἰσημερίαν εὐθέως[?]}.
\lnr{4}\textgreek{Αχαιοὶ δὲ τότε περὶ τὴν πλειάδος ἐπιτολήν[?]}.
% Polybius: The Histories, Book IV, Section 37, line 2
% "for the Aetolians hold their elections after the autumn equinox,
% but the Achaeans in early summer at about the time of the rising
% of the Pleiades."
\lnr{5}Nota illud \textgreek{τότε[?]}.
\lnr{5}Hoc enim designat non semper eo tempore
creari solitos, sed sine dubio ante brumam,
 \textgreek{ἐν ταῖς ὑπερβαλλούσαις ἡμέραις
δύο[?]}.
\lnr{7}Sed et idem in \rnum{v}. manifeste id docet:
 \textgreek{τὸ μὲν οὖν κατὰ τὴν
Αρἀτου τοῦ νεωτέρου στρατηγίαν ἔτος ἐτύγχανε διεληλυθὸς
 περὶ τὴν τὴς πλειάδος ἐπιτολήν}.
\lnr{8}\textgreek{οὕτω γὰρ ἦγε τοὺς χρόνους τὸ τῶν Αχαιῶν ἔθνος[?]}.
% https://el.wikisource.org/wiki/Ιστορίαι/ε'
% (first line on the page)
% Τhe Greek wikisource page has slight differences.
% Polybius: The Histories, Book V, line 1.
% "The year of office of the younger Aratus came to an end at
% the rising of the Pleiades, such being then the Achaean reckoning of time."

\section{De anno commentitio Herodoti}
% Possible misspelling: commentitio -> commenticio, though older texts appear
% to contain this spelling too. ToC has the same spelling.

\lnr{10}Non mirum, si docti aetatis nostrae viri, propter et librorum bonorum
penuriam, et immania temporum intervalla, aliquando
pro veris et compertis rebus ridicula commenta, et coniecturarum
somnia afferunt, cum etiam veteres de re praesenti aliquando
verba facientes, non solum mendacium loqui, sed etiam consulto
mentiri videantur.
\lnr{15}Eiusmodi est, quod historiae Graecae pater Herodotus
de anno Graeco scribit, in illo scilicet post biennium confectum intercalari
solitum, ita ut secundus annus Graecorum semper haberet dies
390, et omne biennium, 750.
\lnr{18}Primum enim cum scripsisset annum
Graecorum esse dierum 360, et 70 annos constare diebus 25200,
si intercalatio non interveniat, subiicit, si quis velit eos annos secundum
intercalationis Graecae consuetudinem accipere; ex 70 annis 35
fore embolimaeos, et 35 menses \textgreek{τριακονθημέρυς[?]},
% Greek: "of thirty days [each]"
 qui fiunt dies 1050.
\lnr{23}Quos ad illos priores adiectos constituere summam annorum 70, dierum
26250.
\lnr{24}Addit praeterea hoc solere fieri ab Graecis,
% à -> ab
 \textgreek{ἵνα [δὴ] αἱ ὧραι συμβαίνωσι
παραγινόμεναι ἐς τὸ δέον[?]}.
% Herodotus: Liber I (Clio), section 32, line 3
% Some part of:
% "these seventy years have twenty-five thousand, two hundred days,
% leaving out the intercalary month. But if you make every other
% year longer by one month, so that the seasons agree opportunely,
% then there are thirty-five intercalary months during the seventy
% years, and from these months there are one thousand fifty days."
\lnr{25}Et libro secundo: \textgreek{ἕλληνες[?]},
 inquit, \textgreek{διὰ τρίτου
ἔτεος ἐμβόλιμον ἐπεμβάλλουσι τῶν ῶρέων εἵνεκεν[?]}.
% Herodotus: Liber II (Euterpe), section 4, line 1
% "for the Greeks add an intercalary month every other year,
% so that the seasons agree;"
\lnr{26}Locum plane intercalationis
utrobique designat, nempe alternis bienniis vertentibus.
\lnr{27}Quod
eiusmodi est, ut non solum propter absurditatem explodendum sit, sed
etiam tanquam temere pronuntiatum, praetereundum.
\lnr{29}Primum
\textgreek{ὡρέων εἵνεκεν[?]}
 non opus erat tertio quoque anno ineunte intercalare.
\lnr{31}Nam ad Solis curriculum assequendum, biennio Graeco desunt dies
decum cum semisse tantum, hic vero undeviginti cum semisse supersunt,
interventu embolismi.
\lnr{33}Quare in duobus bienniis excurrent supra
Solis ratiocinia dies solidi 39.
\lnr{34}Et intra octo annos caput anni solstitialis
traductum erit in autumnum, et in 31 annis iterum accedet ad
solstitium, non tamen eadem die, qua antea, sed longo adhuc intervallo.
\lnr{37}Deinde ponamus hoc Graecos factitasse, (ut non negamus potuisse
fieri) non igitur \textgreek{ὡρέων εἵνεκεν[?]}.

% 60
% {PDF page nr}{source page nr}{line nr}
\plnr{143}{60}{1}Si non \textgreek{ὡρέων εἵνεκεν[?]}: ergo non fiebat
omnino.
\lnr{2}Nam omnis intercalatio fit \textgreek{ὡρέων εἵνεκεν[?]}.
\lnr{2}Quod si quis
obiiciat Trieteridas Graecorum: ego respondeo, eas nihil ad formam
anni, sed ad religiones tantum pertinuisse.
\lnr{4}Quare Nemeacus agon,
Isthmiacus, Orgia Thebanorum, Dionysia \textgreek{ἐν Λίμναις[?]}
 anno tertio Tetraeteridis
celebrabantur, non autem ipsa peculiarem periodi Trietericae
formam constituebant: sed bis in iusta et legitima Tetraeteride
observabantur.
\lnr{8}Quod igitur Herodotum moverit, ut tam absurdam
anni formam Graecis attribuerit, puto; quia quanuis annus esset dierum
354, tamen 360 dierum putabatur, cum omnes menses haberent
\textgreek{τριακάδα[?]}, quanuis non omnes essent
 \textgreek{τριακονθήμεροι[?]}: quia \textgreek{ἐξαίρεσις[?]}
alternis interveniebat.
\lnr{12}Quod cum Herodotus non consideraret, putavit
revera annum non 354, sed 360 dierum fuisse.
\lnr{13}Sed tamen non
nego antiquitus eam anni formam in usu fuisse, praesertim cum Geminus
accuratissimus scriptor idem testetur,
 \textgreek{οἱ μὲν οὖν ἀρχαῖοι[?]}, inquiens,
\textgreek{τοὺν μῆνας τριακονθημέρους ἦγον,
 τοὺν δ᾽ ἐμβολίμους παρ᾽ ἐνιαυτόν[?]}.
\lnr{17}Postea subiicit ab hac anni forma destitisse victos veritate rei,
 et absurditate
instituti.
\lnr{18}Nam apud illum \textgreek{παρ᾽ ἐνιαυτὸν[?]} est, quod apud Herodotum
\textgreek{διὰ τριτου ἔτεος[?]}.
\lnr{19}Supersunt fortasse et plura, et maiora de
anno Graeco, quam ea, quae in praesentem Diatribam coniecimus.
\lnr{21}Sed mihi plura hactenus aut meliora non succerrebant.
\lnr{21}Equidem ab
aequis Lectoribus tantum, ut haec boni consulant, non etiam ut laudent,
expecto.
% ==== End of text of Liber Primus ===

% !TEX TS-program = xelatex
% !TEX encoding = UTF-8 Unicode
% this template is specifically designed to be typeset with XeLaTeX;
% it will not work with other engines, such as pdfLaTeX

%%% Count out columns for fixed-width source font
% 000000011111111112222222222333333333344444444445555555555666666666677777777778
% 345678901234567890123456789012345678901234567890123456789012345678901234567890

\setheaders{\shorttitle{} Liber II}{\shortauthor{}}
\chapter{Liber Secundus - De anno lunari}
\normalsize

% 61
% {PDF page nr}{source page nr}{line nr}
\plnr{144}{61}{1}Annum Graecum antiquitus Lunarem fuisse,
ut alia temporum et mensium descriptio
in Graecia non fuerit, quam quae Lunae
rationibus congrueret, non solum recentiores
homines scripserunt, sed non paucos
veterum idem in literas retulisse tam
compertum esse puto, quam falso eos sensisse
convincit ratio tetraeteridum a nobis
libro proximo disputata.
\lnr{9}Praeterea ex
eadem disputatione nostra satis constat naturale anni principium antiquitus
non ab Hecatombaeone, sed a Gamelione, et ex diebus brumlibus
duci solitum.
\lnr{12}Quandiu igitur Athenienses Gamelionem et temporibus
auspicandis et rerum actibus principem mensem habuerunt,
tunc semper Comitia magistratibus creandis in calcem Posideonis
reiiciebant, ubi erant \textgreek{ἄναρχοι ἡμέραι δύο},
 extra ordinem mensium tricenariorum
positae, ita ut annus esset dierum non solum 360, propter
menses \textgreek{τριακονθημέρους}, sed et 362,
 propter illas appendices \textgreek{ὑπερβαλλούσας[?]},
quae, quia per illud biduum omnes magistratus annui abdicabantur,
propterea dicebantur \textgreek{ἄναρχοι ἡμέραι}.
\lnr{19}Praeterea quod in illis
Comitia novorum magistratuum creandorum habebantur, ideo
\textgreek{ἀρχαιρεσίαι[?]}, etiam dicebantur.
\lnr{21}Atque hoc fuit quidem magistratibus
creandis dicatum biduum, donec anni Lunaris formam Astronomi illorum
temporum publicarunt.
\lnr{23}Tunc pro bruma, solstitium: pro Gamelione,
Hecatombaeonem vulgus principium anni coepit statuere.
\lnr{24}Et
menses, in quibus singulis Comitia terna agebantur, quas
 \textgreek{κυρίας ἐκκλησίας[?]}
vocabant, pro Tetraetericis, Lunares: pro solidis, alternis cavi
usurpari coepti.
\lnr{27}Quod ut planius intelligatur, sciendum Athenis
duos summos senatus fuisse, alterum, \textgreek{τῶν ἀρειοπαγιτῶν[?]},
 qui erant iudices
% Greek: also Ἀρεοπαγῑτῶν (pl. gen.)
% A member of the ancient-Athenian conciliary court of the Areopagus.
ut plurimum rerum capitalium et quidem magni momenti: alterum autem
ordinariarum, civilium et bellicarum, et summae denique reipublicae.

% 62
% {PDF page nr}{source page nr}{line nr}
\plnr{145}{62}{2}Sed Areopagitarum consessus perpetuus erat.
\lnr{2}Hic Senatus
quotannis sorte creabatur, olim utique \textgreek{ἐν ὑπερβαλλούσιας ἡμέραις[?]},
postea anno Lunari admisso, in ultimis quatuor diebus anni
Lunaris, hoc est in illis quatuor, qui sunt supra 350.
\lnr{5}Decem
enim Tribus Attica habuit, quales Roma \rnum{xxxv}.
\lnr{6}Ex singuilis Tribubus
quinquaginta magistratus forte creati rebus gerendis admittebantur.
% Bar on quinquaginta?
\lnr{8}Ita ex decem Tribubus quinquageni Senatum Quingentorum
constituebant, qui ab eo \textgreek{οι πεντακόσιοι[?]} decebantur,
 item \textgreek{ἡβουλὴ
τῶν πεντακοσίων[?]}.
\lnr{10}Porro unaquaeque tribus forte unum diem summam
rem gerebat et imperabat.
\lnr{11}Ita cum per 354 dies, quot nimirum
habet annus Lunaris, singuli quinquaginta diem suam per
orbem imperassent, fiebat, ut 35 dies ex toto anno unaquaeque Tribus
rerum potiretur: et, quia decem erant Tribus, sequitur, ut trecentos
quinquaginta dies simul omnes imperarent.
\lnr{15}Reliquae sunt ex anno
Lunari \textgreek{ἄναρχοι ἡμέραι} quatuor.
\lnr{16}Hae igitur quatuor dies vicem illarum
\textgreek{ὑπερβαλλουσῶν[?]} magistratibus creandis reservatae.
\lnr{17}Hoc ita esse, testis Ulpianus
Rhetor, vetus Demosthenis interpres
 \textgreek{εν τω κατα Ανδροτιωνος ἔχειγοῦν[Greek]},
% Demosthenis Orationes ad optimos libros accurate emendatae
inquit,
 \textgreek{υ ενιαιτος κατα τον σεληνιακον δρομον, τριακοσιας πεντηκοι τα τεσσαρας
ημερας[Greek]}.
\lnr{20}\textgreek{και τας μει δ ημερασ εκαλουν οι Αθηναιοι αρχαυρεσιας[Greek]}.
\lnr{20}\textgreek{εν
αις οιυαρχος η Αττικη ην[Greek]}
\lnr{21}\textgreek{εν ταυταις προεβαλλοντο της αρχοντας[Greek]}.
\lnr{21}\textgreek{ηρχον ουν
οι πεντακυσιοι τας τριακοσιας πεντηκοντα ημερας[Greek]}.
\lnr{22}Trecentos igitur et
quinquaginta dies simul imperabant, qui in decem Tribus divisi dant
unicuique dies triginta quinque.
\lnr{24}Nam, exempli gratia, heri \textgreek{ἡ ἀιαντὶς φυλὺ[?]}
imperabat, hodie \textgreek{ἡ κεκροπὶς[?]}, cras \textgreek{ἡ ἀκαμοιυτὶς[?]}.
\lnr{25}Et sic deinceps una quaeque
suam \textgreek{ἐφημερίαν[?]} imperabat, prout sorte[forte?] ducta erat.
\lnr{26}Neque sane quinquaginta
simul imperabant, sed ex singulis Tribubus singuli forte
ducti viri, qui dicebantur \textgreek{πρόεδροι[?]}.
\lnr{28}Illi enim Comitiis habendis praesidebant.
\lnr{29}Nam quingenti simul dicebatur \textgreek{ἡ τῶν πεντακοσίων[?]}, Tribubus
illis decem in unum corpus confusis.
\lnr{30}Quinquaginta autem
per Tribus distincti dicebantur \textgreek{πρυτάνεις[?]}.
\lnr{31}Decem vero vocabantur
\textgreek{πρόεδροι[?]}, qui erant principes quadraginta novem
 reliquorum, singuli
scilicet in sua Tribu.
\lnr{33}Nam unus \textgreek{πρόεδρος[?]} erat quinquagesimus sui
corporis \textgreek{τῶν πρυτάνεων[?]}: ut in castris Romanis Decurio erat
 decimus
illius decuriae, cuius ipse caput erat.
\lnr{35}Menses igitur Lunares proprii
erant horum et omnium denique magistratuum, et dicebantur
 \textgreek{πρυτανεῖαι[?]}.
\lnr{37}Ut in lege Atheniensium apud Demosthenem, \textgreek{ἐν τῷ κατὰ
 Τιμοκράτοις,
ἐπὶ τὴς πρώτης πρυτανείας τῇ ἑνδεκάτῃ[?]}.
\lnr{38}Hoc est undecima mensis
primi Lunaris, id est, Hecatombaeonis Metonici.
\lnr{39}Dicibatur etiam,
\textgreek{[Greek]}.
\lnr{41}Neque enim est alius Hecatombaeon, quam Lunaris, et eo
die \textgreek{[Greek]} pertinebat ad Tribum \textgreek{[Greek]}.

% 63
% {PDF page nr}{source page nr}{line nr}
\plnr{146}{63}{1}Hoc est, is dies erat \textgreek{[Greek]}.
\lnr{2}Et \textgreek{[Greek]} in omni prytania, sive
mense Lunari, agebantur terstata die mensis, \textgreek{τῇ ἑνδεκάτῃ[?]},
 \textgreek{τῇ εἰκάδι[?]}, \textgreek{τῇ ἔνην καὶ νέαν[?]}.
\lnr{4}Quare haec anni forma tantum ad magistratus pertinebat.
\lnr{4}Neque
ea unquam populus usus est, a quo menses \textgreek{τριακονθήμεροι[?]},
 et Tetraeterides
extorqueri nunquam potuerunt, ne tunc quidem, cum annus Lunaris
ab Hipparcho emendatior editus esset.
\lnr{7}Primus itaque mensis Hecatombaeon
fuit a solstitio.
\lnr{8}Et quia per undecim dies Hecatombaeon
huius anni antevertebat caput praeteriti, inde fiebat, ut saepe Hecatombaeon
in Scirrhophorionem Tetraeteridis incideret, in quo saepe \textgreek{[Greek]}
et \textgreek{[Greek]} magistratum inibant.
\lnr{11}Demosthenes \textgreek{πρὸς Τιμόθεον[?]}.
\lnr{11}\textgreek{και [Greek]}.
\lnr{12}\textgreek{[Greek]}, et cetera.
\lnr{13}Si Thargelion Tetraeteridis erat ultimus mensis anni
\textgreek{πρυτανείας[?]}, ergo \textgreek{ὔστερον[?]} illud
 \textgreek{ἔτος[?]} incipiebat a Scirrhophorione, in
quem incurrebat mensis Metonicus.
\lnr{15}Et revera Scirrhoporion Metonicus
illius anni, qui erat 62 periodi Atticae, incidebat in 13 Thargelionis
Metonici: Hecatombaeon autem in 12 Scirrhophorionis.
\lnr{17}Idem Orator
\textgreek{[Greek]} ait fuisse in negotio
\textgreek{[Greek]}, ultimam anni, scilicet \textgreek{πρυτανείας[?]}.
\lnr{19}\textgreek{Φυλάξας[?]}, inquit, \textgreek{[Greek]}, et cetera.
\lnr{21}Quem locum non affequitur interpres.
\lnr{21}Nam orator diserte
indicat, Metonicum mensem \textgreek{πρυτανείας[?]} nunc in Scirrhophorionem
popularem, nunc in Thargelionem incidere, ac consequenter
Hecatombaeonem Metonicum nunc in Schirrhophorionem Tetraeteridis,
nunc in ipsum hecatombaeonem.
\lnr{25}Thucydides: \textgreek{[Greek]}, et cetera.
\lnr{27}Ab initio statim
Veris, hoc est, a Munychione, ait duos adhuc tantum superfuisse
menses magistratui Pythodori.
\lnr{29}Definebat igitur ille annus \textgreek{πρυτανείας[?]}
in Scirrhophorione, et fere conveniebat anno Tetraeterico.
\lnr{30}Porro neomeniae
illae dicebantur \textgreek{[Greek]}, ad differentiam
 \textgreek{τριακονθημέρων[?]}.
\lnr{32}\textgreek{Του [Greek]}, inquit Thucydides, \textgreek{[Greek]}.
\lnr{33}Alibi:
\textgreek{[Greek]}.
\lnr{35}\textgreek{[Greek]} hic intelliguntur Metonicae, et
 \textgreek{πρυτανεῖαι[?]}.
\lnr{36}Cum autem menses Lunares alternis pleni et cavi sint, semper pro
\textgreek{[Greek]} dicebant \textgreek{[Greek]},
 et quae in illis mensibus cavis
vocabatur \textgreek{[Greek]}, ea re vera erat \textgreek{[Greek]}.
\lnr{38}Huius rei rationem
reddidimus in Boedromione \textgreek{[Greek]}, item in exemplo
Oeconomicorum Aristotelis supra a nobis producto de Mnemone
Tyranno Lampsaceno.
\lnr{41}In quo diserte ostenditur sex dies de 360 alternis
eximi solitos, item \textgreek{[Greek]} pro
 \textgreek{[Greek]} alternis mensibus
dici solitum fuisse.

% 64
% {PDF page nr}{source page nr}{line nr}
\plnr{147}{64}{2}Quare hic repetenda non sunt.
\lnr{2}Sic Moschopulus
\textgreek{εἰς ἡμέρας[?]} Hesiodi 177.
\lnr{3}\textgreek{Αθμιαῖοι[?] τὴν τριακοστὴν φασιν ἐννάτην καὶ ἐικοστήν[?]}.
\lnr{4}Iudaei primum diem mensis cavi dicunt secundum.
\lnr{4}Esto exemplum
de mense Elul.
\lnr{5}Ita scribunt \texthebrew{[Hebrew]}.
\lnr{5}Ita de
omnibus cavis mensibus pronunciant.
\lnr{6}Dies, inquiunt, primus mensis
cavi compensat defectum eius.
\lnr{7}Quia quum pro primo ponimus secundum,
ultimus erit tricesimus, non \rnum{xxix}.
\lnr{8}Et ita omnes menses Iudaeorum
habent \textgreek{τριακάδα[?]}, quemadmodum et Graecorum.
\lnr{9}Diu autem
mensibus Lunaribus usi sunt Graeci, adeo ut Theon Arati interpres
dicat suis temporibus etiamnum aliquot nationes Graecorum eam anni
formam retinuisse.
\lnr{12}\textgreek{[Greek]}.

\subsection{De Octaeteride Cleostrati}

\lnr{14}Primus Solon auctor fuit Atheniensibus
 \textgreek{τὰς ἡμέρασ κατὰ σελήνην ἄγειν},
ut scribit Laertius.
% Diogenes Laertius (3rd century CE) wrote about Solon (c. 638 - c. 558 BCE),
% in "Lives and Opinions of Eminent Philosophers" (Book A, second
% chapter)
% Section 59, near the end: 
% "[ἠξίωσέ τε Ἀθηναίους] τὰς ἡμέρας κατὰ σελήνην ἄγειν." 
% "[He required the Athenians] to adopt a lunar month."
\lnr{15}Quod tamen frustra proposuit, cum
Atheniensibus menses omnes pleni, \textgreek{καὶ τριακονθήμερα} agerentur.
% Greek: and the thirtyday
\lnr{17}Tandem Cleostratus Tenedius Octaeteride Lunari edita, ab Atheniensibus
scripto expressit, quod Solon eloquentia impetrare non potuerat.
\lnr{19}Cum enim iam omnibus persuasum esset, annum Lunarem esse dierum
trecentorum quinquaginta quatuor, Solarem vero Lunari maiorem
esse diebus undecim praecise cum quadrante, Cleostratus animadvertit
octo Lunares annos cum totidem excessibus Solaribus conficere[?] syzygias
% syzygia = conjunction
nonaginta novem, id est dies 2922: quot scilicet diebus octo anni
Solares constant.
% 29 1/2 days per lunar month, 12 lunar months per lunar year = 354 days/l.y.
% 11 1/4 extra days to make a solar year = 365 1/4 days/s.y.
% "8 lunar years plus all the extra days make 99 conjuctions" (lunar months)
% 8 x (354 + 11 1/4) = 2922 days
% 2922 days / 29 1/2 days per lunar month = 99,05085 new moons
\lnr{24}Ergo in octo annis Solaribus totidem syzygias praecise
transigi existimavit: quarum syzygiarum quadraginta octo sint
cavae, reliquae plenae \textgreek{κὰι τριακοντήμερα}.
\lnr{26}Atque hoc intervallo dierum
et syzygiarum putavit \textgreek{τῶν φαινομένων ἀποκατάστασιν[?]} fieri,
 et Lunam
cum Sole, itemque omnium \textgreek{φαινομένων[?]}
 ortus et occasus ad idem punctum
redire, a quo caeperint primum.
\lnr{29}Iste autem Cleostratus (alibi
Leostratum invenio, sed male ut puto) primus Graecorum, et signorum
partes in Zodiaco notavit, et initia Arietis ac Sagittarii, ut ex Plinio
et Hygeno colligimus: idque[?] circa Olympiadem \rnum{lxi}.
% accent on idque: what do we do?
\lnr{32}Instituit
autem caput Octaeteridis a brumalibus diebus necessario, tum quia
eius castigator Harpalus postea idem tempus servavit, tum quia
 \textgreek{ποσειδεῲν δέυτερος[?]}
ostendit Gamelionem hactenus fuisse mensem primum
anni Attici, cum omnis intercalatio debeatur fini anni, Posideon autem
\textgreek{δέυτερος[?]} sit intercalaris.
\lnr{37}Sed Brumam intelligendum est non \textgreek{ἀστρονομικῶς[?]},
sed \textgreek{πολιτικῶς[?]}, quomodo dies civiles agebant Attici.
\lnr{38}Saeculo
enim Iphiti, qui primam Olympiada instauravit, aequinoctium vernum
conficiebatur in \rnum{xxiix} Martii: Solstitium Kalend. Iulii.

% 65
% {PDF page nr}{source page nr}{line nr}
\plnr{148}{65}{1}Sed neomenia
primi mensis Olympici incidit in \rnum{ix} Iulii, hoc est,
 in \rnum{viii}, aut
\rnum{ix} gradum Cancri.
\lnr{3}Neque unquam illam epocham antevertebat.
% Nor was this epoch ever prefered
\lnr{4}Quare primus omnium Cleostratus ostendit Graecis popularibus suis
cardines mundi esse in \rnum{viii} partibus Signorum: idque omnis posteritas
credidit adeo, ut Sosigenes idem Caesari, Caesar posteritati persuaserit.
\lnr{7}Quod si octavus gradus Cancri in \rnum{viii} Iulii, ergo octavus gradus
Capricorni in sexta Ianuarii, cum intervallum sit dierum 182, cum
dimidio.
\lnr{9}Quod si primus annus Olympiadicus non fuisset intercalaris,
neomenia mensis brumalis, qui est Gamelion Atticus, convenisset
in \rnum{vii} Ianuarii, statim post octavum gradum Capricorni.
\lnr{11}Nam a capite
Hecatumbaeonis ad caput Gamelionis, sunt praecise sex menses, et
duae praeterea \textgreek{ἄναρχοι ἡμέραι[?]}:
 qui sunt dies 182, quot scilicet ab \rnum{viii} gradu
Cancri ad \rnum{viii} Capricorni.
% In fact, from the start of Hecatumbaion to the start of Gamelion, there are
% precisely six months, and two extra ἄναρχοι ἡμέραι [anarchic days]:
% this makes 182 days, the number of which is or course from the 8th degree
% of Cancer to the 8th degree of Capricorn.
\lnr{14}Ergo citima neomenia mensis brumalis
est in \rnum{vii} Ianuarii, quos fines nunquam superabit; set embolismi
interventu in Februarium summovebitur.
% Therefore, the nearest new moon in the winter month is on the seventh of
% January, whos ends it will always overflow;
% But embolismic interventions will be withheld until February.
\lnr{16}Omnes igitur menses
alternis sunt pleni et cavi: \textgreek{Ποσειδεὼν[?]}
 etiam \textgreek{δέντερος[?]} plenus.
% All months are therefore alternatively full and short:
% Poseideon [6th month in the Attic calendar] likewise ?? full
%
% Insert table:
% Octaeteris Cleostrati
% -- Placement of the table uncertain. There is no clear indication where it
%    connects to the body text.
% -- There is no column header above "8 Ianua." in the original.
%    In concordance with table
%    038_neomenia_elidensis it might be "Neomenia 1. mensis" or similar.
\begin{table}[htbp]
 \centering
 \renewcommand{\arraystretch}{1.3}
 %%% Liber II p65
%% Wider variation, where the headers are written horizontally
%%
%% For testing, uncomment the folowing lines and the lines at the end
%% of the file
%% Test ==>
%\documentclass[draft,a4paper]{book}
%\usepackage{rotating}
%\usepackage{fontspec}
%\setmainfont{Hoefler Text}[]
%\setmainfont{Times New Roman}[]
%\newfontfamily\greekfont{Times New Roman}
%\usepackage[quiet]{polyglossia}
%\setmainlanguage{latin}
%\setotherlanguage{greek}
%\begin{document}
%Itaque deprehenso capite Hecatombaeonis, quot
%\textgreek{[Greek]} supersint de Scirrhophorione in anno Iuliano, nobis quidem
%deprehendere facile est, qui rationem tantum habemus anni nostri
%iuliani.
%
%\begin{table}
%%\tiny
%%\scriptsize
%\centering
%%\setlength{\tabcolsep}{3pt}
%\renewcommand{\arraystretch}{1.3}
% <== Test
%%
\begin{tabular}[t]{r c c c c c}
~ & \multicolumn{5}{c}{\Large\textsc{Octaeteris Cleostrati}}\\
\cline{2-6}
~ &
\multicolumn{1}{c}{Anni} &
\multicolumn{1}{c}{Cyclus} &
\multicolumn{1}{c}{Liter} &
~ &
\multicolumn{1}{c}{Dies}
\\
~ &
\multicolumn{1}{c}{octaeteridis} &
\multicolumn{1}{c}{Lunnae} &
\multicolumn{1}{c}{Dominicalis} &
~ &
\multicolumn{1}{c}{collecti}
\\
\cline{2-6}
\scriptsize{†}
  &  1 & 18 &  E &  8 Ianua. &  384 \\
~ &  2 & 19 & DC & 27 Ian.   &  738 \\
\scriptsize{†}
  &  3 &  1 &  B & 15 Ian.   & 1122 \\
~ &  4 &  2 &  A &  3 Febr.  & 1476 \\
~ &  5 &  3 &  G & 23 Ian.   & 1830 \\
\scriptsize{†}
  &  6 &  4 & FE & 12 Ian.   & 2214 \\
~ &  7 &  5 &  D & 30 Ian.   & 2568 \\
~ &  8 &  6 &  C & 19 Ian.   & 2922 \\
\cline{2-6}
\\
~ & \multicolumn{5}{l}{\footnotesize \super{†} \textgreek{Εμβολ.}}\\
\end{tabular}
%% Test ==>
%\end{table}
%
%Itaque deprehenso capite Hecatombaeonis, quot
%\textgreek{[Greek]} supersint de Scirrhophorione in anno Iuliano, nobis quidem
%deprehendere facile est, qui rationem tantum habemus anni nostri
%iuliani.
%\end{document}

 \caption{Octaeteris Cleostrati}
 \label{tab:octaeteris_cleostrati}
\end{table}
%
\lnr{17}Scripta est
autem Octaeteris, ut diximus, anno secundo Olympiadis sexagesimae
primae, annis decem post observatam ab Aneximandro obliquitatem
Zodiaci.
% The Octaeteris was however described, as we have said, in the year two
% of the sixty-first Olympiad, ten years after the observation by Aneximander
% of the obliquity of the zodiac.
\lnr{20}Erat cyclus Lunae \rnum{xix},
 Solis \rnum{viii}. anno Iudaico 3227, cuius
Schebat 4.1.76. Ianuarii \rnum{viii}, anno periodi Atticae quinquagesimo
secundo, \textgreek{ποσειδεῶνοσ ἔνῃ καὶ νέᾳ[?]}.
% It was Lunar cycle 19, Solar cycle 8, the Jewish year 3227, being Schebat
% [Jewish month between Tevet and Adar, roughly in Jan-Feb] 4.1.76.
% January 8, the fifty-second year of the Attic period, ποσειδεῶνοσ ἔνῃ καὶ νέᾳ.
% [Poseideon (month 6 in the Attic calendar) Old-and-New (i.e. the 30th day)]
% Leartus about Solon: Πρῶτος δὲ Σόλων τὴν τριακάδα ἔνην καὶ νέαν ἐκάλεσεν.
% "Solon was the first to call the 30th day of the month the Old-and-New day."
% "Lives and opinions..." I.58
\lnr{22}Haec Octaeteris primum fuit initium annorum
et mensium Lunarium \textgreek{πρυτανείας}: quam ipse cum Canone stellarum
Orientium et Occidentium et earum significationibus publicavit.
% And so the first Octaeteris started on the lunar month and year πρυτανείας[??].
% This is when he [Cleostratus] published the Eastern and Western stellar
% Canons and their significance.
\lnr{25}Eum Canonem veteres Graeci
\textgreek{παράπηγμα} vocant.
% This Canon the ancien Greeks called παράπηγμα.
% [Greek: Stall, booth, shed, stand]
\lnr{26}Geminus: \textgreek{Αἱ δὲ
γινόμεναι προῤῥήσεις τῶν ἐπισημασιῶν ἐν τοῖς
παραπήγμασιν οὐκ ἀπό\footnote{No space in De Emendatione} τινων παραγγελμάτων
[ὡρισμένων]\footnote{Word from Geminius missing in De Emendatione} γίνονται}.
% Geminus of Rhodes: Introduction to Phaenomena, ca 1st century BCE
% (Γεμῖνος ὁ Ῥόδιος: Εἰσαγωγὴ εἰς τὰ Φαινόμενα)
% Also simply known as the Isagoge.
% Chapter 16 "Περὶ ἐπισημασιῶν τῶν ἄστρων"
% [About predictions from the stars]
% Second paragraph
% Αἱ δὲ
% γινόμεναι προρρήσεις τῶν ἐπισημασιῶν ἐν τοῖς
% παραπήγμασιν οὐκ ἀπό τινων παραγγελμάτων
% ὡρισμένων γίνονται,
% [οὐδὲ τέχνῃ τινὶ μεθοδεύονται κατηναγκασμένον ἔχουσαι τὸ ἀποτέλεσμα,
% ἀλλ' ἐκ τοῦ ὡς ἐπίπαν γινομένου διὰ τῆς καθ' ἡμέραν παρατηρήσεως τὸ
% σύμφωνον λαμβάνοντες εἰς τὰ παραπήγματα κατεχώρισαν.]
% German translation:
% Die üblichen vorläufigen Angaben der Witterungsanzeichen in den Kalendern
% werden nicht nach bestimmten Regeln gemacht, noch beruhen sie auf einer
% wissenschaftlichen Methode, nach der sie Anspruch auf notwendige Erfülung
% hätten, sondern (die Kalendermacher) haben aus den regelmäsig eintretenden
% Erscheinungen mit Hilfe der täglichen Beobachtung das herausgenommen, was
% ihnen passt, und es in ihre Kalender gesetzt.
% == Carolus Manitius (1898), Gemini Elementa Astronomiae, ad codicum fidem
% recensuit germanica interpretatione et commentariis instruxit.
% From Google Books
\lnr{29}Hic \textgreek{παραπήγματα} vocat \textgreek{τὰς
προῤῥήσεις τῶν ἐπισημασιῶν}, \textit{Tempora quae
messor, quae curvus arator haberet.}
% P. VERGILI MARONIS ECLOGA TERTIA
% Publius Vergillius Maro (Virgil): Ecloga Tertia
% "That they who reap, or stoop behind the plough,
% Might know their several seasons?"
% == Project Gutenberg
\lnr{31}Vitruvius libro nono: \textit{Quorum inventa
secuti, siderum ortus, et occasus, tempestatumque
significatus Eodoxus,
Euctemon, Calippus, Meto, Philippus,
% Accent on Meto
Hipparchus, Aratus, invenerunt,
caeterique ex astrologia, parapegmatorum
disciplinas invenerunt, et
eas posteris explicatas reliquerunt.}
\lnr{39}\textit{Quorum
scientiae sunt hominibus suspiciendae,
quod tanta cura fuerunt, ut etiam videantur divina mente tempestatum
significatus post futuros ante pronunciare.}
% Vitruvius: De Architectura Libri Decem
% Vitruvius: Ten Books on Architecture
% Book 9, chapter VI (Astrology and weather prognostics),
% paragraph 3, 2nd and 3rd sentence
% 3. [When we come to natural philosophy, however, Thales of Miletus,
% Anaxagoras of Clazomenae, Pythagoras of Samos, Xenophanes of Colophon, and
% Democritus of Abdera have in various ways investigated and left us the laws
% and the working of the laws by which nature governs it.]
% In the track of their discoveries, Eudoxus, Euctemon, Callippus, Meto,
% Philippus, Hipparchus, Aratus, and others discovered the risings and settings % of the constellations, as well as weather prognostications from astronomy
% through the study of the calendars, and this study they set forth and left
% to posterity. Their learning deserves the admiration of mankind; for they
% were so solicitous as even to be able to predict, long beforehand, with
% divining mind, the signs of the weather which was to follow in the future.
% [On this subject, therefore, reference must be made to their labours and
% investigations.]

% 66
% {PDF page nr}{source page nr}{line nr}
\plnr{149}{66}{1}Satis clare innuit,
quid, sit \textgreek{παράπηγμα[?]}.
% This clearly indicates that this should be παράπηγμα
\lnr{2}In vulgatis editionibus Vitruvii legitur Eudaemon
Callistus, Melo; pro quo correximus Euctemon, Calippus,
Meto.
% The common edition of Vitruius reads "Eudaemon, Callistus, Melo". We
% corrected this to "Euctemon, Calippus, Meto".

\subsection{De Octaeteride Harpali}

\lnr{5}Octaeteridis Cleostrateae vitium cito deprehensum est,
quod duae Tetraeterides Olympicae cum mense embolimo sint
dierum solidorum 2924, Octaeteris autem Cleostrati dierum
totidem, duobus minus.
\lnr{8}Atqui neomeniae primae Tetraeteridos, et tertiae
ineuntium incidunt in novilunia, ut uberrime a nobis libro priore
disputatum est.
\lnr{10}Quare neomenia Octaeteridis secundae Cleostrateae
incidens in diem penultimum Tetraeteridis secundae, anticipabit novilunium
biduo solido.
\lnr{12}Vitiosa igitur est Octaeteris Cleostrati.
\lnr{12}Cum
igitur intervallum duarum Tetraeteridum inter duo novilunia interiectum
sit, non dubitavit Harpalus, quin illud sit ex iustis syzygiis compositum.
\lnr{15}Omne enim intervallum in idem punctum Lunae desinens,
a quo caeperat, est mere Lunare, hoc est meris mensibus Lunaribus
constans.
\lnr{17}Nam nisi veteres Graeci mensem Lunarem censuissent esse
dierum 29, horarum 12~\myfrac{1}{2}.
\lnr{18}Nunquam spatio dierum 2964[?] iustas syzygias
Lunares fieri posse existimassent.
\lnr{19}Hoc modo annus Lunaris est
dierum 354 horarum 12.
\lnr{20}Qui dies et horae octies multiplicata dant dies
absolutos 2834.
\lnr{21}Qui de 2924 detracti relinquunt 90 dies, qui sunt
menses tres pleni embolimi.
\lnr{22}Quod si dies 2924 per 59 dies dividantur,
habedimus in duabus Tetraeteridibus quadraginta novem paria
mensium alternis plenorum et cavorum, cum diebus praeterea triginta
tribus, hoc est, quinquaginta menses plenos, undequinquaginta
cavos, et tres dies insuper.
\lnr{26}Igitur in Octaeteride, quae constat duabus
Tetraeteridibus Olympicis, sunt syzygiae Lunares nonaginta
novem: quae si essent omnes plenae, fierent omnes dies 2970: de quibus
detractis 2924 diebus duarum Tetraeteridum, remanent 46, differentia
cavorum et plenorum mensium.
\lnr{30}Quadraginta igitur et sex
menses cavi sunt in Octaeteride: et proinde quinquaginta tres erunt
pleni.
\lnr{32}Quae oeconomia mensium immane quantum discrepat a Cleostratea.
\lnr{33}Nam in illa sunt menses alternis pleni, et cavi: in hac tertius
fere mensis est cavus, et aliquando quartus.
\lnr{34}Divisis enim 99 per
46, remanent 2~\myfrac{7}{46} id est duae syzygiae plenae,
 cum \myfrac{7}{46} unius syzygiae.
\lnr{36}Igitur tertius mensis duntaxat est cavus, idque donec ex
 \myfrac{7}{46} consurgat
integra syzygia.
\lnr{37}Tunc enim non iam tertius, sed quartus mensis est cavus,
ut et docet progressus arithmeticus, et potes ex subiecta tabella
animadvertere.

% Table: ΜΗΝΕΣ ΚΟΙΛΟΙ
\begin{table}[htbp]
 \centering
 \footnotesize
 \renewcommand{\arraystretch}{1.3}
 %%% Liber II p67
%%
%%% Count out columns for fixed-width source font
% 000000011111111112222222222333333333344444444445555555555666666666677777777778
% 345678901234567890123456789012345678901234567890123456789012345678901234567890
%
% ΜΗΝΕΣ ΚΟΙΛΟΙ
% "The hollow months" "cavae menses"
%
% This table apparently lists the hollow months that occur in each year
% of the 8 year cycle (octaeteride).
% Each row shows the information for one year in the octaeteride.
% Column 1 indicated if a year is Embolic.
% We have represented the abbreviated word ἐμβολ. by a footnote symbol,
% the same way we did for other tables where it occurs.
% Column 2 uses Byzantine Greek numerals (without a bar above the letters)
% to number the years in the cycle.
% The number 6 is represented by a cursive digamma, rendered here as
% Unicode U03DA.
% The header for the second column is allmost illegible
% ?? ?? ὀκταετερίδος ??
% Columns 3-8 list the 5 or 6 months in the Attic calendar which are hollow.
% Column 9 shows a β. or α. for the ebolic years. Though not explained in the
% table itself, they are probably Byzantine Greek numerals again. The original
% has a bar over the α, but not over the β, indicating they might be numerals.
% Such symbols can be made using the Math features of Tex, e.g.
% $\overbar{\kappa\alpha}$ for '21'.
% Because there is no code for the digamma (representing 6), we need to resort
% to using \varsigma (the end-of-word sigma) for this symbol.
%
%% For testing, uncomment the folowing lines and the lines at the end
%% of the file
%% Test ==>
%\documentclass{book}
%\usepackage{fontspec}
%\setmainfont{Hoefler Text}[]
%%\setmainfont{Times New Roman}[]
%\newfontfamily\greekfont{Times New Roman}
%\usepackage[quiet]{polyglossia}
%\setmainlanguage{latin}
%\setotherlanguage{greek}
%\begin{document}
%% <== Test
%%
\begin{tabular}{ll|lllllll}
\multicolumn{9}{c}{\large{\textgreek{ΜΗΝΕΣ ΚΟΙΛΟΙ}}}
\\
~ & \multicolumn{5}{l}{\textgreek{τ.. τ.. ὀκταετερίδος .τ.}}
\\
\hline
\scriptsize{†} &
\textgreek{α} &
\textgreek{ἐλαφηβολ.} &
\textgreek{θαργηλ.} &
\textgreek{ἑκατομβ.} &
\textgreek{βοηδρομ.} &
\textgreek{μαιμακτ.} &
\textgreek{ποσειδ.} &
\textgreek{β.}
\\
 &
\textgreek{β} &
\textgreek{ἐλαφηβολ.} &
\textgreek{θαργηλ.} &
\textgreek{ἑκατομβ.} &
\textgreek{βονδρομ.} &
\textgreek{μαιμακτ.} &
 &

\\
\scriptsize{†} &
\textgreek{γ} &
\textgreek{γαμηλ.} &
\textgreek{ἐλαφηβολ.} &
\textgreek{σκιῤῥοφ.} &
\textgreek{μεταγείτν.} &
\textgreek{πυανεψ.} &
\textgreek{ποσειδ.} &
\textgreek{α.}
\\
\hline
 &
\textgreek{δ} &
\textgreek{γαμηλ.} &
\textgreek{ἐλαφηβολ.} &
\textgreek{σκιῤῥοφ.} &
\textgreek{βονδρομ.} &
\textgreek{μαιμακτ.} &
 &

\\
 &
\textgreek{ε} &
\textgreek{ανθεστηρ.} &
\textgreek{μουνιχ.} &
\textgreek{σκιῤῥοφ.} &
\textgreek{βονδρομ.} &
\textgreek{μαιμακτ.} &
 &

\\
\scriptsize{†} &
\textgreek{Ϛ} &
\textgreek{γαμηλ.} &
\textgreek{ἐλαφηβολ.} &
\textgreek{θαρυηλ.} &
\textgreek{ἑκατομβ.} &
\textgreek{βονδρομ.} &
\textgreek{ποσειδ.} &
\textgreek{α.}
\\
\hline
 &
\textgreek{ζ} &
\textgreek{γαμηλ.} &
\textgreek{ἐλαφηβολ.} &
\textgreek{θαρυηλ.} &
\textgreek{ἑκατομβ.} &
\textgreek{βονδρομ.} &
\textgreek{ποσειδ.} &

\\
 &
\textgreek{η} &
\textgreek{ανθεστηρ.} &
\textgreek{μουνιχ.} &
\textgreek{σκιῤῥοφ.} &
\textgreek{μεταγείτν.} &
\textgreek{πυανεψ.} &
\textgreek{ποσειδ.} &

\\
\hline
\multicolumn{5}{l}{\footnotesize \super{†} \textgreek{ἐμβολ.}}\\
\end{tabular}
%% Test ==>
%\end{document}

 \caption{\textgreek{Μενες κοιλοι}}
 \label{tab:menes_koiloi}
\end{table}

% Table: ΝΕΟΜΗΝΙΑΙ ΤΗΣ ΟΚΤΑΕΤΗΡΙΔΟΣ
\begin{table}[htbp]
 \centering
 \scriptsize
 %% Modify distance between rows
 \renewcommand{\arraystretch}{1.8}
 %% Modify separation between columns
 \setlength{\tabcolsep}{2.0pt}
 %%% Liber II p67, PDF 150
%%
%% Dates of the new moon for each year in the octaeterida
%%
%% For testing, uncomment the folowing lines and the lines
%% at the end of the file
%% Test ==>
%\documentclass{book}
%\usepackage{fontspec}
%\setmainfont{Hoefler Text}[]
%\newfontfamily\greekfont{Arial Unicode MS}
%\usepackage[quiet]{polyglossia}
%\setmainlanguage{latin}
%\setotherlanguage{greek}
%\begin{document}
%Just some text on top of the table.
%
%\begin{table}[htbp]
% \centering
%%\footnotesize
%%\scriptsize
%\tiny
%% Modify distance between rows
%\renewcommand{\arraystretch}{1.8}
%% Modify separation between columns
%\setlength{\tabcolsep}{2.0pt}
%% <== Test
%%
\begin{tabular}{l llllllll}
\multicolumn{9}{ c }{\large\textgreek{ΝΕΟΜΗΝΙΑΙ ΤΗΣ ΟΚΤΑΕΤΗΡΙΔΟΣ}}\\
\multicolumn{9}{ c }{\normalsize\textgreek{ΚΑΘ' ΕΚΑΣΤΟΝ ΕΤΟΣ}}\\
%\hline
\textgreek{Μηνες} &
\textgreek{ἔτος} &
\textgreek{ἔτος} $\overline{\beta}$ &
\textgreek{ἔτος} $\overline{\gamma}$ &
\textgreek{ἔτος} $\overline{\delta}$ &
\textgreek{ἔτος} $\overline{\epsilon}$ &
\textgreek{ἔτος} $\overline{\varsigma}$ &
\textgreek{ἔτος} $\overline{\zeta}$ &
\textgreek{ἔτος} $\overline{\eta}$
\\
\textgreek{κατὰ σελήνην.} &
~\textgreek{πρῶτον.}
\\
\hline
\textgreek{γαμηλιών.} &
$\overline{\alpha}$          \textgreek{γαμηλ.} &
$\overline{\kappa\epsilon}$  \textgreek{ποσειδ.} &
$\overline{\iota\eta}$       \textgreek{ποσειδ.} &
$\overline{\eta}$            \textgreek{γαμηλ.} &
$\overline{\alpha}$          \textgreek{γαμηλ.} &
$\overline{\kappa\varsigma}$ \textgreek{ποσειδ.} &
$\overline{\iota\varsigma}$  \textgreek{γαμηλ.} &
$\overline{\eta}$            \textgreek{γαμηλ.}
\\
\textgreek{ανθεστηριών.} &
$\overline{\alpha}$          \textgreek{ἀνθεστ.} &
$\overline{\kappa\gamma}$    \textgreek{γαμηλ.} &
$\overline{\iota\epsilon}$   \textgreek{γαμηλ.} &
$\overline{\eta}$            \textgreek{ἀνθεστ.} &
$\overline{\alpha}$          \textgreek{ἀνθεστ.} &
$\overline{\kappa\gamma}$    \textgreek{γαμηλ.} &
$\overline{\iota\epsilon}$   \textgreek{ἀνθεστ.} &
$\overline{\eta}$            \textgreek{ἀνθεστ.}
\\
\textgreek{ἐλαφηβολιών.} &
$\overline{\alpha}$          \textgreek{ἐλαφη.} &
$\overline{\kappa\gamma}$    \textgreek{ἀνθεστ.} &
$\overline{\iota\epsilon}$   \textgreek{ἀνθεστ.} &
$\overline{\zeta}$           \textgreek{ἐλαφη.} &
$\overline{\lambda}$         \textgreek{ἀνθεστ.} &
$\overline{\kappa\gamma}$    \textgreek{ἀνθεστ.} &
$\overline{\iota\epsilon}$   \textgreek{ἐλαφη.} &
$\overline{\zeta}$           \textgreek{ἐλαφη.}
\\
\hline
\textgreek{μυονυχιών.} &
$\overline{\lambda}$         \textgreek{ἐλαφη.} &
$\overline{\kappa\beta}$     \textgreek{ἐλαφη.} &
$\overline{\iota\delta}$     \textgreek{ἐλαφη.} &
$\overline{\zeta}$           \textgreek{μυονυχ.} &
$\overline{\lambda}$         \textgreek{ἐλαφη.} &
$\overline{\kappa\beta}$     \textgreek{ἐλαφη.} &
$\overline{\iota\delta}$     \textgreek{μυονυχ.} &
$\overline{\zeta}$           \textgreek{μυονυχ.}
\\
\textgreek{θαργηλιών.} &
$\overline{\lambda}$         \textgreek{μυονυχ.} &
$\overline{\kappa\beta}$     \textgreek{μυονυχ.} &
$\overline{\iota\delta}$     \textgreek{μυονυχ.} &
$\overline{\varsigma}$       \textgreek{θαργη.} &
$\overline{\kappa\theta}$    \textgreek{μυονυχ.} &
$\overline{\kappa\beta}$     \textgreek{μυονυχ.} &
$\overline{\iota\delta}$     \textgreek{θαργη.} &
$\overline{\varsigma}$       \textgreek{θαργη.}
\\
\textgreek{σκιῤῥοφοριών.} &
$\overline{\kappa\theta}$    \textgreek{θαργη.} &
$\overline{\kappa\alpha}$    \textgreek{θαργη.} &
$\overline{\iota\delta}$     \textgreek{θαργη.} &
$\overline{\varsigma}$       \textgreek{σκιῤῥ.} &
$\overline{\kappa\delta}$    \textgreek{θαργη.} &
$\overline{\kappa\alpha}$    \textgreek{θαργη.} &
$\overline{\iota\gamma}$     \textgreek{σκιῤῥ.} &
$\overline{\varsigma}$       \textgreek{σκιῤῥ.}
\\
\hline
\textgreek{ἑκατομβαιών.} &
$\overline{\kappa\theta}$    \textgreek{σκιῤῥ.} &
$\overline{\kappa\alpha}$    \textgreek{σκιῤῥ.} &
$\overline{\iota\gamma}$     \textgreek{σκιῤῥ.} &
$\overline{\varsigma}$       \textgreek{ἑκατο.} &
$\overline{\kappa\theta}$    \textgreek{σκιῤῥ.} &
$\overline{\kappa\alpha}$    \textgreek{σκιῤῥ.} &
$\overline{\iota\gamma}$     \textgreek{ἑκατο.} &
$\overline{\epsilon}$        \textgreek{ἑκατο.}
\\
\textgreek{μεταγειτνιών.} &
$\overline{\kappa\eta}$      \textgreek{ἑκατο.} &
$\overline{\kappa}$          \textgreek{ἑκατο.} &
$\overline{\iota\gamma}$     \textgreek{ἑκατο.} &
$\overline{\epsilon}$        \textgreek{μεταγ.} &
$\overline{\kappa\eta}$      \textgreek{ἑκατο.} &
$\overline{\kappa}$          \textgreek{ἑκατο.} &
$\overline{\iota\beta}$      \textgreek{μεταγ.} &
$\overline{\epsilon}$        \textgreek{μεταγ.}
\\
\textgreek{βοηδρομιών.} &
$\overline{\kappa\eta}$      \textgreek{μεταγ.} &
$\overline{\kappa}$          \textgreek{μεταγ.} &
$\overline{\iota\beta}$      \textgreek{μεταγ.} &
$\overline{\epsilon}$        \textgreek{βοηδρ.} &
$\overline{\kappa\eta}$      \textgreek{μεταγ.} &
$\overline{\kappa}$          \textgreek{μεταγ.} &
$\overline{\iota\beta}$      \textgreek{βοηδρ.} &
$\overline{\delta}$          \textgreek{βοηδρ.}
\\
\hline
\textgreek{πυανεψιών.} &
$\overline{\kappa\zeta}$     \textgreek{βοηδρ.} &
$\overline{\iota\theta}$     \textgreek{βοηδρ.} &
$\overline{\iota\beta}$      \textgreek{βοηδρ.} &
$\overline{\epsilon}$        \textgreek{πυαν.} &
$\overline{\kappa\zeta}$     \textgreek{βοηδρ.} &
$\overline{\kappa}$          \textgreek{βοηδρ.} &
$\overline{\iota\alpha}$     \textgreek{πυαν.} &
$\overline{\delta}$          \textgreek{πυαν.}
\\
\textgreek{μαιμακτηριών.} &
$\overline{\kappa\zeta}$     \textgreek{πυαν.} &
$\overline{\iota\theta}$     \textgreek{πυαν.} &
$\overline{\iota\alpha}$     \textgreek{πυαν.} &
$\overline{\delta}$          \textgreek{μαιμα.} &
$\overline{\kappa\zeta}$     \textgreek{πυαν.} &
$\overline{\iota\theta}$     \textgreek{πυαν.} &
$\overline{\iota\alpha}$     \textgreek{μαιμα.} &
$\overline{\gamma}$          \textgreek{μαιμα.}
\\
\textgreek{ποσειδεών.} $\overline{\alpha}$&
$\overline{\kappa\varsigma}$ \textgreek{μαιμα.} &
$\overline{\iota\eta}$       \textgreek{μαιμα.} &
$\overline{\iota\alpha}$     \textgreek{μαιμα.} &
$\overline{\gamma}$          \textgreek{ποσειδ.} &
$\overline{\kappa\varsigma}$ \textgreek{μαιμα.} &
$\overline{\iota\theta}$     \textgreek{μαιμα.} &
$\overline{\iota\alpha}$     \textgreek{ποσειδ.} &
$\overline{\gamma}$          \textgreek{ποσειδ.}
\\
\textgreek{ποσειδεών.} $\overline{\beta}$&
$\overline{\kappa\varsigma}$ \textgreek{ποσειδ.} $\overline{\alpha}$ &
    \multicolumn{1}{c}{$\circ$} &
$\overline{\iota}$           \textgreek{ποσειδ.} &
    \multicolumn{1}{c}{$\circ$} &
    \multicolumn{1}{c}{$\circ$} &
$\overline{\iota\eta}$       \textgreek{ποσειδ.} &
    \multicolumn{1}{c}{$\circ$} &
~
\\
\end{tabular}
%% Test ==>
%\end{table}
%
%Just some text below the table.
%\end{document}

 \caption{\textgreek{Νεομηνιαι της Οκταετηριδος}}
 \label{tab:neomeniai_tes_oktaeteridos}
\end{table}

% 67
% {PDF page nr}{source page nr}{line nr}
\plnr{150}{67}{1}Subiecims praeter ea Tabulam neomeniarum Lunarium secundum
menses anni politici aequabilis, id est secundum menses Tetraetericos.
\lnr{3}Haec enim est vera Harpaleae Octaeteridis \textgreek{ψηφοφορία[?]}:
% Greek: vote (politics)
 quippe
quae nigil aliud est, quam mensium Lunarium cum aequabilibus
comparatio.
\lnr{5}Ideo neomeniam Lunaris Gamelionis cum neomenia
Gamelionis aequabilis in primo anno composuimus: non quod
ita fecerit Harpalus: (Nullus enim Gamelion aequabilis eo seculo
Lunaris fuit:) sed quia deprehenso anno primo Octaeteridis Harpali,
non operosum erit divinare quotae diei Gemelionis aequabilis
competat neomenia Gamelionis Lunaris.
\lnr{10}Nam si, verbi gratia,
neomenia primi Gamelionis Lunaris Harpelei incidit in tertiam
diem Gamelionis aequabilis, reliquae omnes neomeniae eundem progressum
servabunt: puta per triduum omnes neomeniae promovendae
erunt.

% 68
% {PDF page nr}{source page nr}{line nr}
\plnr{151}{68}{2}Sed non magis scimus epocham capitis illius Octaeteridis,
quam patriam ipsius Harpali.
\lnr{3}Constat tamen initium fecisse a bruma,
ut est testis Festus Avienus in Arateis:
\begin{quote}
\emph{Non ego nunc longo redeuntia sidera motu\\
In priscas memorem sedes. Habet ista priorum\\
Pagina, et incerta rerum ratione ferentur.\\
Nam quae solem hiberna novem putat aethere volui,\\
Ut Lunae spatium redeat, vetus Harpalus, ipsa\\
Ocius in sedes, momentaque prisca reducit.}
\end{quote}
% Avieni - Aratea, lines 1363-1368
% "Postumius Rufius Festus who is also Avienius"
% He made a somewhat inexact translation of Aratus' didactic poem "Phaenomena",
% the first part of which was in turn a verse setting of Phaenomena by Eudoxus
% of Cnidus.
\lnr{11}Ex quibus cognoscimus et a bruma incepisse,
 et \textgreek{ἐννεαετηρίδα[?]} quoque
vocasse, non, quod annis novem solidis constaret, ut hallucinatur
Festus, sed qua nono quoque anno in orbem rediret: quemadmodum
pentaeteris et trieteris dictae sunt, non a numero annorum, quibus
constabant, sed eorum, quibus ineuntibus \textgreek{ἀποκατάστασιος[?]} fiebat:
quemadmodum quartanam febrem dicimus, non quod intervalla habeat
quaternum dierum, sed quod quarto quoque die in orbem redeat.
\lnr{18}Igitur Harpali Octaeteris admissa fuit cum parapegmate
 et significationibus
stellarum, et haec est periodus secunda \textgreek{πρυτανείας[?]},
 qua Athenienses
usi sunt.
\lnr{20}Plinius quoque significationes siderum et tempestatum
\textgreek{προῤῥήσεις[?]} ex Harpalo citat in indice libri \rnum{xviii}:
 quae non aliunde
petitae, quam ex eius parapegmate.
\lnr{22}Cum autem haec Octaeteris sit
dierum 2924: ipsi dies per octo divisi dant quantitatem anni secundum
Harpalum, dierum 365~\myfrac{1}{2} sive horarum aequinoctialium 12.
\lnr{25}Quare manifestum mendum est in Censorino, ubi differitur de anni
Solaris quantitate secundum Harpalum.
\lnr{26}Nam ibi legimus Harpalum
definisse annum Solis dierum tercentorum sexaginta quinque,
et tredicem horarum praeterea aequinoctialium.
\lnr{28}Omnino enim legendum duodecim horarum.
\lnr{29}Et ne quis dubitet, idem error est infra
apud eundem, ubi dicit Arminon regem Aegypti annum ad tredecim
menses et quinque dies perduxisse.
\lnr{31}Annus Aegyptius non est tredecim,
sed duodecim mensium.
\lnr{32}Alioquin Octaeteridi Harpaleae superessent
horae octo supra 2924.
\lnr{33}Atqui ita nulla fieret aequatio.
\lnr{33}Siquidem
omnis aequationis lex iubet nihil reliquum fieri de ratione horaria,
aut scrupularia.
\lnr{35}Imo tantum abest, ut illarum octo horarum accessione
rationes Solis cum Lunaribus parient, ut etiam fine illis iusto
longior sit Octaeteris, ut alibi demonstratur.
\lnr{37}Hanc Octaeterida exclusit
Enneadecaeteris Euctemonis, et Metonis: De qua proxime dicendum
erat, nisi Eudoxi Octaeteris nos revocaret, quae Metonis quidem
periodo posterior est; propter cognationem autem materiae in
continenti reliquis Octaeteridibus subiiciendam esse arbitrati sumus.

% 69
% {PDF page nr}{source page nr}{line nr}
\plnr{152}{69}{2}De ea igitur prius dicendum.

\subsection{De Octaeteride Eudoxi}
\lnr{3}Eudoxus Cnidius, vir suo saeculo eruditissimus, et Mathematicorum
princeps, in Aegyptum profectus, ibi annum et menses
praeterea integros quatuor, facerdotibus et astrologis operam dedit,
et Octaeterida suam conscripsit, ut docet nos Laertius: \textgreek{[Greek]}.
\lnr{8}Is igitur cum in Aegypto
esset anno tertio Olympiadis \rnum{ciii}, et eclipsium intervalla, quas in
monimentis suis notatas habebant Aegyptii, inter se compararet, deprehendit
syzygiam secundum Harpalum propius abesse a vero, quam
syzygiam secundum Octaeterida Cleostrateam.
\lnr{12}Nam syzygia secundum
Harpalum est dierum 29, horarum aequinoctialium 12~\myfrac{28}{33}.
\lnr{13}At
secundum Cleostratum todidem quidem dierum et horarum, sed et \myfrac{18}{33}
duntaxat unius horae.
\lnr{15}Ratio igitur differentiae Harpaleae ad rationem
differentiae Cleostrateae est dupla quadripertiens tricesimas tertias.
\lnr{16}Quare cum maiuscula sit Harpalea, quam vera sit syzygia;
 putavit Eudoxus
\myfrac{4}{33} unius horae esse excessum Harpaleae sizygiae
 supra veram syzygiam
Lunarem, quam definivit 29 dierum, 12 horarum, \myfrac{24}{33}, vel, quod
idem est, \myfrac{8}{11} unius horae.
\lnr{20}Ideoque ratio differentiae verae syzygiae, ad rationem
Cleostrateae differentiae, est dupla.
\lnr{21}Cum igitur in Octaeteride
sint syzygiae nonaginta novem, si in 29 dies 12~\myfrac{8}{11}
 horae ducantur, habebis
modum unius verae Octaeteridis dies 2923, horas 12, qui est dimidius
dies.
\lnr{24}Ideoque in duabus Octaeteridibus tres dies supererunt supra
rationes Solis: et consequenter in viginti Octaeteridibus, quae sunt
\textgreek{ἑκκαιδεκαετηρίδες[?]} decem, superabunt dies triginta,
 qui est mensis integer.
\lnr{27}Quare Eudoxi periodus magna constat ex Octaeteridibus viginti,
vel Heckaedecaeteridibus decem, quarum ultima deminuenda sit
diebus triginta.
\lnr{29}Et proinde tota periodus Eudoxi est dierum 58440.
\lnr{30}Qui sunt praecise anni Iuliani 160.
\lnr{30}In annis igitur 160 melius putavit
mensem omitti posse, quam diem in \rnum{xix} annis Metonicis.
\lnr{31}Quod tamen
ineptum est.
\lnr{32}Et tam inconsulte ausus est introducere octaeterida
post enneadecaeterida Metonicam, quam iuste enneadecaeteris mendosa
Metonis omni Octaeteridi praelata est.
\lnr{34}Cum igitur scripserit
anno \rnum{iii} Olympiadis \rnum{ciii}, cyclo Lunae decimo sexto,
 Solis octavo, instituerit
vero initium Octaeteridis a solenni Aegyptiaco, quod vocabant
\textgreek{Ισια[?]}, vel \textgreek{Ισίεια[?]}, quae tunc incidebant
 in tempus sideris brumae
confectae, ut odoramur ex Gemino, non difficile erit divinare, quis dies
Iulianus illi tempori competat, si consideremus, cui diei mensis
 \textgreek{ἀιγῶνος[?]}
Eudoxus in parapegmate suo assignavit diem brumae.
% Ισια: Straight, right

% 70
% {PDF page nr}{source page nr}{line nr}
\plnr{153}{70}{2}Nam cum
Meto et Euctemon incipiant annum suum caelestem a \rnum{xxvii} Iunii,
Brumae autem diem centesimum octagesimum secundum numerent
a Solstitio, Eudoxus quartum diem a bruma Euctemonis dixit esse verum
diem brumae, ut notatum est in Parapegmate Attico, cui congruit
dies 28 Decembris.
\lnr{7}Si igitur \textgreek{Ισια[?]} tunc cadebant in brumam, ea necesse
est incidisse in 28 Decembris, quamuis verus dies brumae incidisset
in 27.
\lnr{9}Erat annus Nabonassari 383.
\lnr{9}Neomenia Toth 23 Novembris,
feria prima.
\lnr{10}Neomenia Paophi 23 Decembris.
\lnr{10}Ergo \textgreek{Ισια[?]} in \rnum{vi}
Paophi.
\lnr{11}Sed duo adversantur.
\lnr{11}Quorum alterum est, quod non in mense
Paophi videntur fuisse \textgreek{Ισια[?]}, sed in mense Athyr: qui, posteaquam
fixus fuit, incidit in Novembrem Iulianum.
\lnr{13}Atqui in Novembrem
Iulianum conferuntur Ifia a veteri poeta in descriptione mensium, quam
reperies in Catelectis nostris.
\lnr{15}Ibi enim de Novembri ita scriptum est:
\begin{quote}
\emph{Carbaseo post hunc artus inductus amictu}\\
\emph{\hspace*{1em}Memphios antiqua sacra, Deamque colit.}\\
\emph{A quo vix avidus sistro compescitur anser,}\\
\emph{\hspace*{1em}Devotusque sacris incola Memphidiis.}
\end{quote}

\lnr{20}Manifesto Ifia sunt in Athyr, non in Paophi.
\lnr{20}Alterum est, quod
Geminus reprehendens sui saeculi errorem, qui semper assignabat brumam
Isiis in anno vago Nabonassari, ait ante 120 annos brumam incidisse
in Isia.
\lnr{23}Ergo a tempore Eudoxi ad tempus illud, quo  suum librum
scribebat Geminus, intersunt anni tantum 120.
\lnr{24}Proinde ille
annus a Nabonassaro fuerit 503.
\lnr{25}Ita Geminus fuerit longe antiquior
Hipparcho.
\lnr{26}Quod non puto.
\lnr{26}Longe enim posterior videtur.
\lnr{26}Ad priorem
quidem dubitationem respondere possumus, solenne Memphiticum
Novembris non esse \textgreek{Ισια[?]}, sed \textgreek{εὕρεσιν Οσίριδος[?]}.
\lnr{28}Nam \textgreek{ἀφδυισμός Οσίριδος[?]}
celebrabatur \rnum{xvii} Athyr, teste Plutarcho, hoc est
\rnum{xiii} Novembris: item \textgreek{εὕρεσις Οσίριδος[?]}
 eodem mense celebrabatur,
ut constat ex Kalendario rustico, quod extat Romae, in quo
in mense Novembri cui Athyr respondet, festum \textsc{euresis} notatum
est.
\lnr{33}\textgreek{Τῆς εὑρέσεως[?]} Rutilius Numatianus
 meminit in suo Iternerario:
\begin{quote}
\emph{Et'tum forte hilares per compita rustica pagi}\\
\emph{\hspace*{1em}Mulcebant sacris pectora fessa iocis}\\
\emph{Illo quippe die tandem renovatus Osiris}\\
\emph{\hspace*{1em}Excitat in fruges semina laeta novas.}
\end{quote}
% Claudius Ritillius Namatianus
% "De Reditu Suo, sive De Reditu in Patriam, sive Iter" ca 415 C.E.
% Liber Primus, Line 374-376
% "Et tum forte hilares per compita rustica pagi
%   mulcebant sacris pectora fessa iocis.
% Illo quippe die tandem revocatus Osiris
%   excitat in fruges germina laeta novas."
% Note: renovatus <-> revocatus, semina <-> germina
\lnr{38}Paulo ante indicaverat se soluisse post ortum Pleiados.
\lnr{38}Aperte notat
illam \textgreek{εὕρέσιν[?]} celebrari mense Novembri.
\lnr{39}Et praeterea erant alia
solennia Isidis: \textsc{ut isidis navigium} mense Martio apud idem Kalendarium
et Lactant.
\lnr{41}Et Apul. lib. \rnum{xi} item \textsc{sacrum phariae,
et sarapia} mense Aprili.

% 71
% {PDF page nr}{source page nr}{line nr}
\plnr{154}{71}{1}Ad alteram autem dubitationem nihil
quod respondeamus, habemus.
\lnr{2}Sed utcunque haec fuerint, Eudoxus
sumsit initium Octaeteridis suae a proximo novilunio post 28 Decembris,
hoc est a Scebat Iudaici anni 3396, cuius character 3.23.726, feria
quarta, Decembris \rnum{xxxi}, cum iam tamen neomenia Posideonis embolimi
Metonici moraretur priscam epocham die uno, et tunc esset in
Kal. Ianuarii.
\lnr{7}Igitur inde Octaeterida suam incepit,
 \textgreek{Ποσειδεῶνος δευτέρου ἔνῃ καὶ νέα[?]}:
% Old-and-new of the second Poseidonis
cuius caput post Iulianos 161 praecipitatur usque in
30 Ianuarii.
\lnr{9}Sed \textgreek{ἐξαιρέσει[?]} triginta dierum iterum in ultimam Decembris
reditur: et ita rationes Solis cum Lunaribus aequantur.

\subsection{Elenchus Octaereridis}

\lnr{11}Ne hoc quidem modo constant rationes Lunares.
\lnr{11}Proxime
quidem abest a vero \textgreek{ἑκκαιδεκαετηρὶς[?]} Eudoxi.
\lnr{12}Nam vera \textgreek{ἑκκαιδεκαετηρὶς[?]}
Iudaica est dierum 5847, ut et Eudoxea, et praeterea
horae 1.414, quo excessu superat Eudoxeam.
\lnr{14}Sed syzygia non est praecise
accepta dierum 29, horarum 12, \myfrac{8}{11}: vel, quod idem est,
 dierum 29~\myfrac{1}{2}
et \myfrac{2}{33}.
% 29.5 + 2/33 days = 29 days, 12 hours + 2/33*24 hours
% 2/33*24 = (2/33)*(3*8) = (2*8)/11 = 16/11, which is *not* equal to 8/11
\lnr{16}(Nam tam dies 29 hor. 12~\myfrac{8}{11}, quam dies 29~\myfrac{1}{2}
 \myfrac{2}{3}
 Octaeteridis syzygiam
consummant. Error autem est in Gemino
\myfrac{\textgreek{α}}{\textgreek{λγ}} pro
 \myfrac{\textgreek{β}}{\textgreek{λγ}},
% Alternatively use math mode:
%  $\alpha\over\lambda\gamma$ pro $\beta\over\lambda\gamma$
  ut hoc obiter
moneam.)
% 1/33 pro 2/33
\lnr{18}Syzygia enim praecise est dierum 29, hor. 12~\myfrac{793}{1080}.
% 29d 12 793/1080h = 29d 12.734259h = 29.530d,
% which matches current known value
\lnr{18}Quare
vera \textgreek{ὑπεροχὴ[?]} Solaris anni supra Lunarem erit
 dierum 10, horarum 21.204.
\lnr{20}Quae omnia si octies multiplicentur, fient dies 87, horae 2.472.
\lnr{21}Qui quidem dies non consummant menses tres Lunares.
\lnr{21}Ideo in octo
annis non possunt intercalari menses tres: quod ex enneadecaeteride
probatur.
\lnr{23}Nam in octo enneadecaeteridibus fiunt intercalationes
quinquaginta sex: in \rnum{xix} autem Octaeteridibus fiunt quinquaginta
septem.
\lnr{25}Unus igitur mensis abundat post annos 152; non autem, ut
voluit Eudoxus, post annos 160.
\lnr{26}Et cur illi displicuit Enneadecaeteris
Metonis, non possum conminisci: nisi quia semper sincerum volumus
vas incrustare, et potius reprehendere, quam discere.
\lnr{28}Octaeterida autem
Eudoxi Dositheus Archimedis familiaris correxit, et iterum cum
parapegmate castigatiore edidit.
\lnr{30}Unde quoties veteres \textgreek{ἐν ἐπισημασίαις τῶν φαινομένων[?]}
Dositheum testem producunt, scito esse ex Eudoxi
quidem Octaeteride et parapegmate, sed a Disitheo correctis.
\lnr{32}Parapegma
cum Octaeteride Lucanus vocat fastos, vel, ut ipse loquitur, fastus.
\lnr{34}\emph{Nec meus Eudoxi vinctur fastibus annus.}
\lnr{34}Geminus postquam ex
Dosithei parapegmate quaedam produxit, statim subiicit etiam testimonium
ex Eudoxo.
\lnr{36}Sed utrumque est Eudoxi; prius quidem ex posteriore
editione Dosithei, posterius autem ex priore Eudoxi.
\lnr{37}Quare
multi veterum octaeterida Eudoxi Dositheo attribuunt, teste Censorino.

% 72
% {PDF page nr}{source page nr}{line nr}
\plnr{155}{72}{1}Illustravit etiam eandem Octaeterida commentario et expositione
Eratosthenes Cyrenaeus.
\lnr{2}Porro de Octaeteirde ita legitur apud
Censorinum: \emph{Hunc quoque circuitum vere annum magnum esse pleraque
Graecia existimavit, quod ex annis vertentibus solidis constaret,
ut proprie in anno magno fieri par est.}
\lnr{5}\emph{Nam dies sunt solidi uno minus
centum, annique vertentes solidi octo.}
\lnr{6}Haec a librariis mutilata
ita sunt supplenda: \emph{Nam dies sunt solidi ducenti noningenti viginti
duo, menses uno minus centum, et cetera.}
\lnr{8}Plinius vero de Octaeteride
intelligit, cum scribit libro
 \rnum{xviii}, \rnum{xxv} cap. \emph{Indicandum est et
illud, tempestates ipsas ardores suos habere quadrinis annis, et easdem
non magna differentia reverti ratione solis: octonis vero augeri easdem,
centesima revoluente se Luna, et cetera.}
\lnr{12}Hoc enim ex parapegmatis Eudoxi
et Dosithei hausit, quae quidem erant edita una cum octaeteride
Eudoxea.

\subsection{De Anno Magno Metonis, sive Enneadecaeteride}

\lnr{15}Quis status anni Lunaris Atheniensium esse potuerit, cum ad
Canonas Octaeteridis describeretur, ex comparatione utriusque
octaeteridis Tetraetericae et Lunaris Harpaleae, scire potuisti.
\lnr{18}Quamuis enim initia et tempus institutae Octaeteridis 
 \textgreek{πρυτανείας[?]}
Harpaleae ignoramus, tamen in quadraginta annis magnam turbationem
noviluniorum necessario confecutam esse, potes ex eadem
comparatione colligere.
\lnr{21}Itaque Aristophanes Olympiade 88, Amynia
praetore, Nephelas docens inducit Lunam cum Ahtenionsibus[?] expostulantem,
quod menses ad Lunam non describerent, sed \textgreek{ανω τε και
κάτω κυδοιδοπᾷν[?]}
ait.
\lnr{24}Nam putantes se solennes cultus et ferias ex Lunae
et Octaeteridis observatione obire, eas alieno tempore anni imprudenter
celebrabant.
\lnr{26}Quod plane anomaliae octaeteridis congruit:
donec Metonis anno succedente ea penitus desita est.
\lnr{27}Theophrastus
\textgreek{περὶ σημεῖων ὑδάτῶν καὶ πνευμάτων.}
% Title: περι σημειων υδατων και πνευματων [και χειμωνων και ευδιων] 
\lnr{28}\textgreek{διὸ καὶ ἀγαθοὶ γεγένηται  κατὰ τόπους τινὰς
ἀστρονόμοι ἔνιοι, οἷον Ματρικέτας ἐν Μηθύμνῃ ἀπὸ τοῦ Λεπετύμνου, καὶ Κλεόστρατος
ἐν Τενέδῳ ἀπὸ τῆς Ιδης, καὶ Φαεινὸς Αθήνῃσιν ἀπὸ τοῦ Λυκάμβη τοῦ τὰ
 περὶ τὰς τροπὰς
συνεῖδε.}
\lnr{31}\textgreek{παῤ οὗ Μέτων ἀκούσας τόν τοῦ ἑνὸς δέοντα ἔικοσιν ἐνιαυτὸν
 συνὲταξει.
ἦν δὲ ὁ μὲν Φαεινὸς μέτοικος Αθήνῃσιν ὁ δὲ Μέτων Αθηναῖος. [?]}.
% [καὶ ἄλλοι δὲ τὸν τρόπον τοῦτον ἠστρολόγησαν.]
% Theophrastus
% "De signis"
% Paragraph 4
% "Thus in some parts have been found good astronomers: for instance, Matriketas
% at Methymna observed the solstices from Mount Lepetymnos, Cleostratus in
% Tenedos from Mount Ida, Phaeinos at Athens from Mount Lycabettus: Meton, who
% made the cycle of nineteen years, was the pupil of the last-named. Phaeinos
% was a resident alien at Athens, while Meton was an Athenian.
% [Others also have made astronomical observations in like manner.]"
\lnr{32}Meton igitur Pausaniae filius, ut tunc captus erat Graecorum,
 insignis Mathematicus
floruit ineunte bello Peloponnesiaco, vir non solum peritia motuum
coelestium, sed et aquiliciis, et librationibus nobilis.
\lnr{35}Itaque et
fontes induxit Athenis, ut auctor est
 Phrynichus Comicus \textgreek{μονοτρόπῳ[?]}.
\begin{quote}
\textgreek{Τίς ἐστιν ὁ μετὰ ταῦτα ταύτης σροντιῶν;[?]}\\
\textgreek{Μέτων ὁ Λευκονοιεὺς, ὁ τὰς κρήνας ἄγων.[?]}
\end{quote}
% Phrynichus Comicus, comic poet, ca 430 BCE
% Fragments of his works survive.
% μονοτρόπῳ (Monotropos; The Solitary) is a play by him exhibited in 414 BCE.
% All fragments are collected in Theodor Kock: Comicorum atticorum fragmenta
% (Teubner, 1880). This fragment on page 376
% https://archive.org/stream/comicorumatticor01kockuoft#page/376/mode/1up
% "A. τίς δ᾽ ἔδτιν ὁ μετὰ ταῦτα φροντίζων;
%  B. Μέτων, ὁ Λευκονοιεύς.
%  A. οἶδ᾽, ὁ τὰς κρήνας ἄγων."
% "Leuconoea erat pagus Leontidis tribus, unde oriundus Meton (Aristoph.
% Av. 992 cum interpr. Aelian. V. H. 10, 7, ubu Λευκονοιεύς ex
% Salmasii coniectura pro Λάκων)."

% 73
% {PDF page nr}{source page nr}{line nr}
\plnr{156}{73}{2}Quare huc alludens Manilius noster in Apotelesmatis,
 sub Aquario
Aquilices et Astronomos ait nasci:
\begin{quote}
  \emph{Ille quoque, inflexa fontem qui proiicit urna,\\
  Cognatas tribuit iuvenilis Aquarius artes\\
  Cernere sub terris; undas inducere tectis. et cetera.\\
  Quippe etiam mundi faciem, sedesque movebit\\
  Sidereas, coelumque novum versabit in orbem.}
\end{quote}
% Astronomica (attibuted to Marcus Manilius), written sometime during
% the reign of either Caesar Augustus or Tiberius.
% Only a few copies-of-copies manuscripts survived, and the exact contents
% is, and was, contested.
% Scaliger published two critically edited editions of this work
% (in 1579 and 1599).
% This quote by Scaliger diverges from current publications. 
% Liber IV
% lines 259-261 [et cetera]
% proiicit -> proicit; iuvenilis -> iuvenalis
% sub terris; undas inducere tectis -> sub terris undat, inducere terris.
% and lines 267-268
% coelumque -> caelumque
\lnr{9}Non est locus nobilior in toto Manilio, quamuis olim eum non satis
capiebamus.
\lnr{10}Veteres, quae fuit eorum hac in re imperitia, putabant
omnium motuum coelestium et mundi ipsius integram conversionem
fieri, quando Sol et Luna in idem tempus recurrebant,
in quo antea deprehendebantur, et quamdiu Octaeteridis
fides suspecta non fuit, eam esse mensuram
 \textgreek{ἀποκαταστάσεως τοῦ παντὸς[?]}
non solum vulgis, sed et docti credebant.
\lnr{15}Itaque Festus Avienus ex
Graecorum libris dixit:
\begin{quote}
  \emph{Non ego nunc longo redeuntia sidera motu\\
  In priscas memorem sedes. ---}
\end{quote}
% Avienus - Aratea
% See also page 68, where the same lines are quoted (with slight differences)
% Lines 1363-1369
\lnr{19}Innuit, ut vides, de ortu et occasu
 \textgreek{τῶν μορφώσεων[?]} annuo.
\lnr{19}Nam si qua
est varietas, eius \textgreek{ἀποκαταστασιν[?]},
 quantocunque tempore illa reditura
sit; omittit dicere.
\lnr{21}Subiicit:
\begin{quote}
  \emph{--- Habet ist a priorum\\
  Pagina, et incerta rerum ratione feruntur.}
\end{quote}
% Quote above continues: Lines 1369-1370
Diversa de istis scriptorum et Astronomorum iudicia, et libros exstare
dicit:
\begin{quote}
  \emph{Nam quae Solem hiberna novem putat athere volui,\\
  Ut spatium Lunae redeat, vetus Harpalus, ipsam\\
  Ocius in sedes, momentaque prisca reducit.}
\end{quote}
% Quote above continues: Lines 1371-1373
Putat \textgreek{ἀποκατάστασιν[?]} inerrantium fieri, ita ut sidera,
 ut ipse loquitur, in
priscas sedes longo motu redeant.
\lnr{30}Putat, inquam, hoc accidere, quandocunque
spatium Lunae redit.
\lnr{31}Ipse interpretatur mentem suam.
\lnr{32}Quando, inquit, Luna in sedes et momenta prisca reducitur.
\lnr{32}Quod
intervallum, inquit, Harpalus annorum novem esse decrevit, sed
male, cum ocior sit iusto ista periodus.
\lnr{34}Aperte igitur censet, conversionem
coeli universalem fieri, quando neomenia redit in eandem
diem, et horam, in qua antea fuit.
\lnr{36}Subiicit Festus Avienus, hanc periodum
ob brevitatem fallere, ideoque ei decennium additum a
Metone.

% 74
% {PDF page nr}{source page nr}{line nr}
\plnr{157}{74}{2}Et veram \textgreek{ἀποκατάστασιν τοῦ φαινομένων[?]}
 intra \textgreek{ἐννεαδεκαετηρίδα[Greek]}
fieri.
\lnr{3}De hallucinatione Festi super nomine Enneaeteridos, supra
dictum est.
\lnr{4}Aratus quoque volens ostendere omnium inerrantium
\textgreek{ἐποχὴν[?]} Solem ipsum esse, absurde quidem, sed tamen eius
rei causam confert ad cyclum Metonis.
\lnr{6}Quo intervallo, \textgreek{δύσεων καὶ ἀνατολῶν τοῦ φαινομένων[?]}
fiat restitutio, orbis, et conversio quaedam
mundi universalis.
\lnr{8}Ita enim canit elegantius, quam verius:
\begin{quote}
  \textgreek{Γινώσκεις τάδε καὶ σύ. Τὰ γὰρ συναείδεται ἤδη[?]}\\
  \textgreek{Εννεακαίδεκα κύκλα φαεινοῦ ἠελίοιο[?]}\\
\end{quote}
% Aratus: Phaenomena
% Lines 752-753
% "You too know all these (for by now the nineteen cycles of the shining
% sun are all celebrated by all)"
\lnr{11}Quem locum summum virum Theonem ex toto affecutum non
esse mirum non est, cum Hipparchus, et post eum Astronomiae
Apollo Ptolemaeus, quantitatem anni Tropici ex neomeniarum
restitutione collegerint, quam restitutionem Hipparchus recte
censet post annos 304 statim fieri.
\lnr{15}Sed de his libro quarto amplius.
\lnr{16}Apertius vero Diodorus Siculus docet illos veteres,
 \textgreek{τὰ φαινόμενα ἀποκαθίστασθαι[?]}
illis novemdecim annis vertentibus, putasse.
\lnr{17}Libro
enim duodecimo de illis novemdecim annis loquens ita scribit;
\lnr{19}\textgreek{ἐν δὲ τοῖς ἐιρημένοις ἔτεσι τὰ ἄστρα τὴν ἀποκατάστασιν ποιεῖται,
 καὶ καθάπερ
ἐνιαυτοῦ τινὸς μεγάλου τὸν ἀνακυκλισμὸν λαμβάνει.}
\lnr{20}\textgreek{διὸ καί τινες αὐτὸν Μέτωνος
ἐνιαυτὸν ὀνομάζουσι.}
\lnr{21}\textgreek{δοκεῖ δὲ ὁ ἀνὴρ οὗτος ἐν τῇ προῤῥήσει καὶ προγραφῇ
ταύτῃ θαυμαστῶς ἐπιτετευχέναι.}
\lnr{22}\textgreek{τὰ γὰρ ἄστρα τήν τε κίνησιν, καὶ τὰς ἐπισημασίας
ποιεῖται συμφώνως τῇ γραφῇ.}
\lnr{23}\textgreek{διὸ μέχρι τῶν καθ´ ἡμᾶς χρόνων οἱ πλεῖστοι
τῶν ἑλλένων χρώμενοι τῇ ἐννεα[και]δεκαετηρίδι οὐ διαψεύδονται τῆς ἀληθείας.}
% Diodorus Siculus - Bibliotheca Historica
% Διόδωρος Σικελιώτης - Ἱστορικὴ Βιβλιοθήκη
% Book 12
% [12,36] (second half)
% ἐν δὲ τοῖς εἰρημένοις ἔτεσι τὰ ἄστρα τὴν ἀποκατάστασιν ποιεῖται
% καὶ καθάπερ
% ἐνιαυτοῦ τινος μεγάλου τὸν ἀνακυκλισμὸν λαμβάνει.
% διὸ καί τινες αὐτὸν Μέτωνος
% ἐνιαυτὸν ὀνομάζουσι.
% δοκεῖ δὲ ὁ ἀνὴρ οὗτος ἐν τῇ προῤῥήσει καὶ προγραφῇ
% ταύτῃ θαυμαστῶς ἐπιτετευχέναι.
% τὰ γὰρ ἄστρα τήν τε κίνησιν καὶ τὰς ἐπισημασίας
% ποιεῖται συμφώνως τῇ γραφῇ.
% διὸ μέχρι τῶν καθ´ ἡμᾶς χρόνων οἱ πλεῖστοι
% τῶν Ἑλλήνων χρώμενοι τῇ ἐννεακαιδεκαετηρίδι οὐ διαψεύδονται τῆς ἀληθείας.
\lnr{25}Quid melius potuit versus Arateos interpretari?
\lnr{25}Sed fallitur Diodorus.
\lnr{26}Nam \textgreek{ἐννεαδεκαετηρὶς[?]}, quae illius temporibus obtinebat,
 erat
Calippica, non autem Metonica: cum Metonica plus quam
quinque diebus, Calippica uno fere die illo faeculo iam antevertissit,
quo haec scribebat Diodorus.
\lnr{29}Idem scriptor libro secundo
eadem repetit, loquens de enneadecaeteride gentium Hyperborearum:
\textgreek{λέγεται δὲ καὶ τὸν θεὸν[?]} (Apollinem)
 \textgreek{δἰ ἐτῶν ἐννεακαίδεκα καταντᾷν
εἰσ την νῆσον, ἐν σεσ[?] καὶ αἱ τῶν ἄστρων ἀποκαταστάσεις ἐπὶ τελος ἄγονται.[?]}
\lnr{32}\textgreek{καὶ
διὰ τοῦτο τὸν ἐννεακαιδεκαετῆ χρόνον ὑπὸ τῶν ἑλλήνων μέγαν ἐνιαυτὸν
 ὀνομάζεσθαι[?]}
\lnr{34}Huius meminit et Aelianus libro decimo.
\lnr{34}Nunc quid
velit Manilius per te ipse potes intelligere.
\lnr{35}Ait Metonem coelum
versasse in novum orbem, hoc est, conversionem mundi
 \textgreek{καὶ ἀποκαταστασιν[?]}
novo orbe et nova periodo \textgreek{τὴς ἐννεαδεκαετηρίδος[?]} definivisse, cum
orbis vetus Harpali mendosus, et fallax tempore deprehensus sit.
\lnr{39}Hoc est, quod dicit versare coelum in novum orbem.
\lnr{39}Et novum
orbem periodum Metonicam intelligit, comparatione veteris, quam
Octaeterida Harpali esse ostendit Festus Avienus.

% 75
% {PDF page nr}{source page nr}{line nr}
\plnr{158}{75}{2}Enneadecaeteris
igitur Metonis celeberrima multis quidem nominibus commendatur:
sed eam parum hactenus notam fuisse, argumento sunt ii,
qui eandem penitus cum cyclo Paschali, et nostro vulgari, quem
numerum aureum vocamus, difiniunt.
\lnr{6}Tantum enim septem embolismos,
et novemdecim annos cum nostro numero aureo communes
habet, in reliquis immane quantum differt.
\lnr{8}Nam neque
idem situs embolismorum, neque eadem anni Solaris quantitas:
cum Cyclus Metonis sit absolute dierum 6940, discedens
ab iusta periodo horis 7. 26.' 56.'' 40.'''
% à ->ab
\lnr{11}Unde in annis 76 moratur
Lunae curriculum die uno, horis 5. 47.' 46.'' 40.'''
\lnr{12}In annis
denique 304 Luna antevertit primam epocham Metonicam diebus
solidis quinque.
\lnr{14}Annos autem embolimaeos septem aut eorum
situm, scire non possumus priusquam epocham cycli ipsius
et caput indagemus.
\lnr{16}Veteres Graeci ab bruma, ut cognoscimus
% à ->ab
ex octaeteridibus Cleostrati et Harpali, tempora sua ordiebantur,
et eos fecuti Romani, quod facilius a decrementis umbrae
horas observarent, brumae confecto die, quam ab incrementis,
solstitio.
\lnr{20}Quare omnia horologia Graecorum semper
ad rationes brumae referebantur, ut et Romana: donec primus
omnium Meton noster ab capite anni populari, aut potius ab eius
% à ->ab
epocha, horologium describere instituit, hoc est ab solstitio.
% à ->ab
\lnr{23}Unde
ipsum organon \textgreek{ἡλιοτρόπιον[?]} vocarunt Graeci,
 a solstitii observatione.
\lnr{25}Quod instrumentum nobilissimum, atque priscorum hominum
observationes longe subtilitate vincens, ipse in Comitio
Athenarum dicavit, ut testis est priscus scriptor Philochorus apud
interpretem Aristophanis \textgreek{ὀρνίθων[?]},
 instar \textgreek{ἡλιοτροπίου[?]} Pherecydis, quod
in Scyro insula patria sua auctor dedicavit.
\lnr{29}Itaque videtur non solum
a solstitio caput Enneadecaeteridis suae deduxisse, sed etiam
ab eo tempore, quo Heliotropium suum dicavit.
\lnr{31}Idem
Scholiastes Aristophanis affentitur quidem Philochoro de Heliotropii
positu: sed de tempore refragatur.
\lnr{33}Philochorus enim
censebat Metonem ante Pythodorum Archontem Heliotropium
posuisse.
\lnr{35}Ipse post Pythodori magistratum aut saltem in magistratu
ipso, non autem ante magistratum, positum contendit.
\lnr{36}Verba
Grammatici de Metone haec sunt: \textgreek{ὁ δὲ φιλόχορος ὀν Κολωνῷ μὲν
αὐτὸν οὐδὲν λέγει θεῖναι, φευδῶς δὲ πρὸ Πυθοδώρου ἡλιοτρόπιον ἐν τῇ νηῦ
λεγομένῃ ἐκκλησίᾳ, πρὸς τῶ τείχει τῷ ἐν τῇ πνυκὶ[?]}.
% Possibly: fragment of Phrynichus - Monotropos
% "Fr. 22 PCG, Σ Αν. 997"
\lnr{39}Lege: \textgreek{οὐδὲν λέγει
θεῖναι[?]}.
\lnr{40}\textgreek{ἐπ᾽ Αψεύδοις δὲ τοῦ πρὸ Πυθοδώρου.[?]}
% Same fragment
\lnr{40}Sane verum est adhuc
sub magistratu Apseudis illud Heliotropium dedicasse, et solstitium
observasse 27 Iunii, diebus 36 ante neomeniam sequentis Hecatombaeonis
Tetraeterici.

% 76
% {PDF page nr}{source page nr}{line nr}
\plnr{159}{76}{3}Itaque adhuc erat in magistratu Apseudes:
cui in sequenti anno successit Pythodorus.
\lnr{4}Parum igitur abest,
quin et a Solstitio cyclum suum incipisse, et circa tempora Pythodori
Heliotropium statuisse credamus, optimi scriptoris auctoritate
moti.
\lnr{7}Sed de Solstitio cur dubitem, cum auctorem locupletem
habeam Festum Avienum?
\lnr{8}Qui post eos versiculos a nobis paulo
ante adductos subiicit, loquens de Harpalo:
\begin{quote}
  \lnr{8}\emph{Illius ad numeros prolixa decennia rursum}\\
  \emph{Adiecisse Meton Cecropea dicitur arte,}\\
  \emph{Inseditque animis. Tenuit rem Graecia solers [sic]}\\
  \emph{Protinus, et longos inventam misit in annos.}\\
  \emph{Sed primaeva Meton exordia sumpsit ab anno,}\\
  \emph{Torreret rutilo cum Phoebus sidere Cancrum:}\\
  \emph{Cingula cum veheret pelagus procul Orionis,}\\
  \emph{Et cum caeruleo flagraret Sirius astro.}\\
\end{quote}
% Rufius Festus Avienus: Aratea, lines 1369-1376
% Speaking of Harpalus
% [Edition Alfred Breysig, Lipsiae 1882]
% https://archive.org/details/rufifestiavienia00avieuoft
% "Illius ad numeros prolixa decennia/decentia rursum
% adiecisse Meton Cecropea dicitur arte
% inseuitque/inseditque animis: tenuit rem Graecia sollers
% protinus et longos inuentum misit in annos.
% et/sed primaeua Meton exordia sumpsit ab anno,
% torreret rutilo cum Phoebus sidere cancrum,
% cingula cum ueheret pelagus procul Orionis
% et cum caeruleo flagraret Sirius astro."
% (slashed words are reported by Breysig to be different in various sources)
\lnr{18}A solstitio igitur duxit citimum novilunium.
\lnr{18}Remotissimum vero
statuit ad aestus maximos Caniculae.
\lnr{19}Iam igitur constat de exordio periodi
Metonicae.
\lnr{20}De tempore, hoc est anno observati a Metone solstitii,
item de tempore et epocha ipsius Solstitii, habemus plene apud
Ptolemaeum, libro \rnum{iii},
 qui ait diserte Solstitium a Metone et Euctemone
observatum anno Nabonassari 316, Phamenoth \rnum{xxi}, mane.
\lnr{24}Tempus congruit \rnum{xxvii} Iunii, cyclo Lunae \rnum{vii},
 cyclo Solis \rnum{xxvi}, feria
prima, anno quarto Olympiadis 86 definente, praefecto Athenis
Apseude, T. Verginio, Proculo Geganio Macerino \textsc{coss}.
\lnr{26}Qui erat
tertius annus periodi Atticae.
\lnr{27}Scirrhophorion \rnum{iii} Iulii.
\lnr{27}Ergo Neomenia
Hecatombaeonis Metonici \textgreek{σκιῤῥοφοριωνος τρίτῃ ἐπὶ δέκα}.
\lnr{28}Diodorus
libro duodecimo:
% Diodorus Siculus: Bibliotheca historica (Βιβλιοθήκη ἱστορική),
% Book 12, chapter 36, section 2:
% How Meton of Athens was the first to expound the nineteen-year cycle.
 \textgreek{ἐν δὲ ταῖς Αθήναις Μέτων ὁ Παυσανίου μὲν υἱός,
δεδοξασμένος δὲ ἐν ἀστρολογίᾳ ἐξέθηκε τὴν ὀνομαζομένην ἐννεακαιδεκαετηρίδα,
τὴν ἀρχὴν ποιησάμενος ἀπὸ μηνὸς ἐν Ἀθήναις σκιροφοριῶνος τρισκαιδεκάτης.}
% Translation by C. H. Oldfather (1946)
% ISBN 978-0-674-99413-3
% http://data.perseus.org/citations/urn:cts:greekLit:tlg0060.tlg001.perseus-eng1:12.36.2 
% "In Athens Meton, the son of Pausanias, who had won fame for
% his study of the stars, revealed to the public his nineteen-year cycle,
% as it is called, the beginning of which he fixed on the thirteenth day of
% the Athenian month of Scirophorion."
% [Continues:] In this number of years the stars
% accomplish their return to the same place in the heavens and conclude,
% as it were, the circuit of what may be called a Great Year;
% consequently it is called by some the Year of Meton."
\lnr{32}Proinde \textgreek{θαργηλιῶνος πέμπτῃ φθίνοντος[?]}
 observatum Solstitium
a Metone.
\lnr{33}Et proximo \textgreek{πρυτανείας[?]} Hecatombaeone Pythodorus
inivit magistratum, novem mensibus ante initia belli Peloponesiaci.
\lnr{35}Haec igitur est Epocha cycli Metonici, non autem \rnum{ix} Iulii,
ut solebant vetustiores: neque octava pars Cancri, ut Cleostratus.
\lnr{37}Quare mirari satis non possum, cur Columella dixerit, se, auctore
Metone, solstitium in octava parte Cancri, sicut alia \textgreek{κέντρα[?]}
in octavis partibus signorum suorum, statuere, cum res ipsa eum
satis refellat.

% 77
% {PDF page nr}{source page nr}{line nr}
\plnr{160}{77}{1}Cur enim potius Columellae de Metone, quam
Metoni ipsi credam?
\lnr{2}De modo Enneadecaeteridis Metonicae
scribit Censorinus:
% Censorinus: De Die Natali Liber, chapter 18
 \emph{Praeterea sunt anni magni complures: ut Metonicus,
quem Meton Atheniensis ex annis undeviginti constituit.}
\lnr{4}\emph{Eoque
Enneadecaeteris appellatur: et intercalatur septies: in eoque
anno sunt dierum sex millia, et quadringenti quadraginta.}
% "Praeterea sunt anni magni conplures, ut Metonicus,
% quem Meton Atheniensis ex annis undeviginti constituit,
% eoque enneadecaeteris appellatur et intercalatur septies, inque eo
% anno sunt dierum VI milia et DCCCXL"
% 'sex millia, et quadringenti quadraginta' = 6440
% 'VI milia et DCCCXL' = 6000 + 500 + 300 + 40 = 6840
\lnr{6}Legendum:
\emph{in eoque anno sunt dierum sex millia et noningenti quadraginta.}
% noningenti => nongenti = 900
% sex millia et noningenti quadraginta = 6940
\lnr{8}Nam ex notis vulgaribus fluxit error, dierum sex millia,
 et \rnum{ccccxl}.
\lnr{9}Deest enim \rnum{d}
% Preferably I (Unicode U+2160) plus reversed C (Unicode U+2183): "ⅠↃ"
% But most fonts don't support these.
% Currently (feb 2017) supported on Mac OS X by:
% Baskerville (regular, italic, semibold, semibold italic, bold, bold italic)
% Big Casion Medium
% Courier (regular, oblique, bold, bold oblique)
% Geneva
% Helvetica (regular, oblique, bold, bold oblique)
% Helvetica Neue (regular, italic, bold, bold italic)
% Lucida Grande (regular, bold)
% Trattello
% Notably *not* supported by Society of Biblical Literature (SBL) fonts.
 nota quingentorum.
\lnr{9}Cum 5940 dies comprehenderet
Enneadecaeteris Metonica, ea null modo potuit
congruere cum vera enneadecaeteride Lunari, ut infra demonstrabitur.
\lnr{12}Servata epocha in \rnum{xxvii} Iunii, facile embolismorum
situs et tempora deprehendemus.
\lnr{13}Nulla enim Neomenia Hecatombaeonis
Solstitium antevertebat.
\lnr{14}Quocirca secundus, quintus,
octavus, decimus, tertiusdecimus, sextusdecimus, decimus octavus
anni erant embolimaei; contra quam consent docti homines nostri
temporis.
\lnr{17}Cum autem periodus ipsa 6940 diebus praecise explicaretur,
in illis erant anni Lunares \rnum{xix},
 menses \textgreek{τριακονθήμεροι[?]} septem,
dies quatuor, scrupula nulla.
\lnr{19}Quare propter illos quatuor dies abundantes,
quatuor quoque anni erant \textgreek{ὑπερήμεροι[?]}, dierum scilicet
355, ut postea videbimus.
\lnr{21}Anni autem Lunaris modus, secundum
Metonem, est dierum 354~\myfrac{4}{19}, aut \myfrac{16}{76}.
\lnr{22}Neomenia prima Hecatombaeonis
Metonici fuit Iulii \rnum{xv}.
\lnr{23}Quare cyclus Metonis constat
non ex enneaeteride et dacade, ut voluit Festus Avienus, sed ex Octaeteride
et Hendecaeteride.
\lnr{25}Nam nullae aliae partes sunt, enneadecaeteridis,
quae propius absint a modulo anni Solaris: nec quaemelius
in se cohaerant.
\lnr{27}Quod enim Octaeteridi superest supra rationes
Solis, id deest Hendecaeteridi, et contra.
\lnr{28}Sed cum Meto videret
accurata observatione septimam diem Hecatombaeonis vicesimi
Tetraeterici semper in novilunium concurrere; (verbi gratia, incipiat
primus Hecatombaeon \rnum{ix}. Iulii, ut in principio periodi Atticae;
vicesimus incipiet in Kal. Augusti, et in \rnum{vii} mensis erit novilunium;
id quod me tacente indicat laterculus mensium Tetraetericorum
antea a nobis propositus) cum igitur hoc videret Meto, animadvertit
in hoc intervallo contineri duas Octaeteridas Harpali.
\lnr{35}Id
quod et puero proclive: item dies 1092.
\lnr{36}Sed ii dies est triennium Harpaleum,
vel Cleostrateum, hoc est tres anni Lunares cum uno mense
intercalari pleno.
\lnr{38}Duae autem Octaeterides sive Harpaleae, sive Tetraetericae
sunt dies 5848.
\lnr{39}Quibus si adieceris 1092, confurget summa
dierum 6940.
\lnr{40}igitur iustam periodum confici posse ex duabus
Harpali Octaeteridibus et triennio Lunari existimavit.

% 78
% {PDF page nr}{source page nr}{line nr}
\plnr{161}{78}{1}Rursus in
duabus Octaeteridibus, sex sunt embolismi; in triennio unus.
\lnr{3}Ergo septem embolismi transigentur in ea periodo: et fient omnes
syzygiae 235: quia in duabus Octaeteridibus sunt 198, et in triennio
Lunari 37.
\lnr{5}Atque adeo decemnovem annis tota periodus explicabitur.
\lnr{6}Unde eam \textgreek{ἐννεαδεκαετηρίδα} vocavit.
\lnr{6}Si igitur omnes
menses 235 huius periodi essent \textgreek{τριακονθήμεροι[?]},
 et pleni, ii fierent
dies 7050.
\lnr{8}De quibus si detrahantur dies 6940, quantitas nempe
huius periodi, relinquentur menses cavi 110, qui debentur
huic periodo: et proinde reliqui 125 erunt pleni.
\lnr{10}Longe igitur
maior erit numerus plenorum, quam cavorum: neque erunt
alternis pleni et cavi, ut nec in Harpali Octaeteride erant alternis
pleni et cavi.
\lnr{13}Si igitur 235 in 110 distribuantur, habebimus syzygias
2. dies 4~\myfrac{1}{11}.
\lnr{14}Eae enim in 110 multiplicatae faciunt 235 syzygias
praecise: quas intelligimus omnes plenas.
\lnr{15}Itaque post duas
syzygias et dies 4~\myfrac{1}{11}, utendum erit \textgreek{ἐξαιρέσα[?]},
 ut pro 4 mensis, dicatur
quinta.
\lnr{17}Item eodem modo post quatuor syzygias, pro quintae
syzygiae octava dicetur nona.
\lnr{18}Et ita progrediendo erogabis omnes
\textgreek{ἐξαιρέσας[?]}, donec ultima syzygia sit cava, et pro eius tricesima,
dicatur prima Hecatombaeonis primi secundae periodi.
\lnr{20}Id
quod in conspectu tibi dedimus in duabus sequentibus Tabulis: in
quarum priore omnes syzygiae cum characteribus suis notatae sunt.
\lnr{23}Nam Cella, quae habet tres numeros, ea indicat mensem cavum.
\lnr{24}Puta in primo anno, tertio mense, in cella habes \myfrac{45}{4}.
\lnr{24}Duo
priores numeri significant pro 4 mensis, dicendum 5: vel, ut Graeci
loquuntur pro \textgreek{τετάρτη ἱσταμένου[?]}, \textgreek{πέμπτη ἱσταμένου[?]}.
\lnr{26}Ideo mensis
est cavus.
\lnr{27}Inferior autem numerus est feria, vel character neomeniae.
\lnr{28}Quaecunque autem neomeniae habent unum numerum, eae
sunt plenorum mensium.
\lnr{29}Reliqua facilia sunt.
\lnr{29}Ex quibus vides
\textgreek{ἐξαίρεσιν[?]} non fieri in uno die,
 sed prout coniugatio duarum syzygiarum
postulat.
\lnr{31}Post binas enim syzygias fit \textgreek{ἐξαίρεσις[?]}, donec numerus
ex \myfrac{1}{11} accrescens addat diem prioribus diebus quatuor.
\lnr{32}Quare
in omni undecima syzygia accrescit dies unus.
\lnr{33}Insigniter autem
fallitur Geminus, priscus et eruditus auctor, qui scribit Metonem
divisisse 6940 dies per 110 syzygias: et quia 110 in 6940
continentur sexagesies ter, propterea censet Metonem statim
post 63 dies \textgreek{ἐξαίρεσιν τῶν ἡμερῶν[?]} fecisse.
\lnr{37}Hoc enim ratio ipsa confutat.
\lnr{38}Nam 63 dies sunt syzygiae 2, et dies praeterea 3.
\lnr{38}Quae omnia
in 110 ducta producunt syzygias, 220, dies 330.
\lnr{39}Hoc est syzygias
undecim, quae cum 220 syzygiis compositae dant tantum 231
syzygias plenas.
% Tabula Characterismi neomeniarum enneadecaeteridis metonicae
\begin{table}[htbp]
\input{tables/079_characterismi_neomeniarum.tex}
\end{table}
% Tabella Characterismi periodorum
\begin{table}[htbp]
%%% Liber II p79
%%
%%\tiny
%%\scriptsize
\centering
%%\setlength{\tabcolsep}{3pt}
%\renewcommand{\arraystretch}{1.3}
%%
\begin{tabular}{@{}c c c@{} }
\toprule
\multicolumn{3}{c}{\Large\textsc{Tabella Characterismi Periodorum}}\\
\midrule
\multicolumn{1}{c}{Enneadeca-} &
\multicolumn{1}{c}{Character} &
\multicolumn{1}{c}{~}
\\
\multicolumn{1}{c}{eterides} &
\multicolumn{1}{c}{Enneadec.} &
\multicolumn{1}{c}{~}
\\
\midrule
 \rnum{i}    &  5 & 0   \\
 \rnum{ii}   &  1 & 19  \\
 \rnum{iii}  &  4 & 38  \\
 \rnum{iiii} &  7 & 57  \\
 \rnum{v}    &  3 & 76  \\
 \rnum{vi}   &  6 & 95  \\
 \rnum{vii}  &  2 & 114 \\
\bottomrule
\end{tabular}
\caption{Tabula Characterismi Periodorum}
\end{table}
% Tabula neomeniarum metonicarum in mensibus iulianis
\begin{table}[htbp]
%%% Liber II p80
%%
%%% Count out columns for fixed-width source font
% 000000011111111112222222222333333333344444444445555555555666666666677777777778
% 345678901234567890123456789012345678901234567890123456789012345678901234567890
%
%\tiny
\scriptsize
%\footnotesize
%\small
%\normalsize
\centering
%% Modify separation between columns
\setlength{\tabcolsep}{1.6pt}
%% Modify distance between rows
\renewcommand{\arraystretch}{1.3}
\begin{tabular}{%
@{}r@{\hspace{0.3em}}r r  c
r@{~}l r@{~}l r@{~}l r@{~}l r@{~}l r@{~}l
r@{~}l
r@{~}l r@{~}l r@{~}l r@{~}l r@{~}l r@{~}l c
}
\toprule
\multicolumn{31}{c}{\Large\textsc{Tabula Neomeniarum Metonicarum}}\\
\multicolumn{31}{c}{\Large\textsc{in Mensibus Iulianis}}\\
\midrule
\addlinespace
\addlinespace
%~ &

\begin{rotate}{75}\hspace{0.3em}Anni Ennea-\end{rotate} &
\begin{rotate}{75}decaeteridis\end{rotate} &
\begin{rotate}{75}Cyclus Lunae\end{rotate} &
\begin{rotate}{75}Litera Dominica\end{rotate} &

\begin{rotate}{75}\textgreek{Εκατομβαιών}\end{rotate} & &
\begin{rotate}{75}\textgreek{Μεταγειτνιών}\end{rotate} & &
\begin{rotate}{75}\textgreek{Βοηδρομιών}\end{rotate} & &

\begin{rotate}{75}\textgreek{Πυανεψιών}\end{rotate} & &
\begin{rotate}{75}\textgreek{Μαιμακτηριών}\end{rotate} & &
\begin{rotate}{75}\textgreek{Ποσειδεών α}\end{rotate} & &
% $\overline\alpha$ does not work here (math mode does not render).
\begin{rotate}{75}\textgreek{Ποσειδεών β}\end{rotate} & &

\begin{rotate}{75}\textgreek{Γαμηλιών}\end{rotate} & &
\begin{rotate}{75}\textgreek{Ανθεστηριών}\end{rotate} & &
\begin{rotate}{75}\textgreek{Ελαφηβολιών}\end{rotate} & &

\begin{rotate}{75}\textgreek{Μουνυχιών}\end{rotate} & &
\begin{rotate}{75}\textgreek{Θαργηλιών}\end{rotate} & &
%\begin{rotate}{75}\textgreek{Σκιῤῥοφοριών}\end{rotate} & &
\multicolumn{3}{l}{\begin{turn}{75}\textgreek{Σκιῤῥοφοριών}\end{turn}}

%\multicolumn{2}{l}{\begin{turn}{75}Syzygiae xxx[?]\end{turn}}
\\
\midrule
  &  1 &  7 & C &
 15&Iul & 14&Aug & 13&Sep & 12&Oct & 11&Nov & 10&Dec &
  \multicolumn{2}{c}{0} &
  9&Ian &  7&Feb &  9&Mar &  7&Apr &  7&Mai &  5&Iun
\\
† &  2 &  8 & B &
  5&Iul &  3&Aug &  2&Sep &  1&Oct & 31&Oct & 30&Nov &
 29&Dec &
 28&Ian & 26&Feb & 28&Mar & 26&Apr & 26&Mai & 24&Iun
\\
  &  3 &  9 & A &
 24&Iul & 22&Aug & 21&Sep & 20&Oct & 19&Nov & 18&Dec &
  \multicolumn{2}{c}{0} &
 17&Ian & 16&Feb & 16&Mar & 15&Apr & 14&Mai & 13&Iun
\\
  &  4 & 10 & G F &
 12&Iul & 11&Aug &  9&Sep &  9&Oct &  7&Nov &  7&Dec &
  \multicolumn{2}{c}{0} &
  5&Ian &  4&Feb &  5&Mar &  4&Apr &  3&Mai &  2&Iun
\\
† &  5 & 11 & E &
  2&Iul & 31&Iul & 30&Aug & 28&Sep & 28&Oct & 26&Nov &
 26&Dec &
 24&Ian & 23&Feb & 24&Mar & 23&Apr & 23&Mai & 21&Iun
\\
  &  6 & 12 & D &
 20&Iul & 19&Aug & 18&Sep & 17&Oct & 16&Nov & 15&Dec &
  \multicolumn{2}{c}{0} &
 14&Ian & 12&Feb & 14&Mar & 13&Apr & 12&Mai & 10&Iun
\\
  &  7 & 13 & C &
 10&Iul &  8&Aug &  7&Sep &  6&Oct &  5&Nov &  5&Dec &
  \multicolumn{2}{c}{0} &
  3&Ian &  2&Feb &  2&Mar &  8&Apr & 30&Apr & 30&Mai
\\
† &  8 & 14 & B A &
 28&Iul & 28&Iul & 26&Aug & 25&Sep & 25&Oct & 23&Nov &
 22&Dec &
 21&Ian & 20&Feb & 21&Mar & 20&Apr & 19&Mai & 18&Iun
\\
  &  9 & 15 & G &
 17&Iul & 16&Aug & 14&Sep & 14&Oct & 12&Nov & 12&Dec &
  \multicolumn{2}{c}{0} &
 10&Ian &  9&Feb & 10&Mar &  9&Apr &  8&Mai &  7&Iun
\\
† & 10 & 16 & F &
  7&Iul &  5&Aug &  4&Sep &  3&Oct &  2&Nov &  1&Dec &
 31&Dec &
 29&Ian & 28&Feb & 29&Mar & 28&Apr & 27&Mai & 26&Iun
\\
  & 11 & 17 & E &
 25&Iul & 24&Aug & 23&Sep & 22&Oct & 21&Nov & 20&Dec &
  \multicolumn{2}{c}{0} &
 19&Ian & 17&Feb & 18&Mar & 16&Apr & 16&Mai & 14&Iun
\\
  & 12 & 18 & D C &
 14&Iul & 12&Aug & 11&Sep & 10&Oct &  9&Nov &  8&Dec &
  \multicolumn{2}{c}{0} &
  7&Ian &  6&Feb &  7&Mar &  6&Apr &  5&Mai &  4&Iun
\\
† & 13 & 19 & B &
  3&Iul &  2&Aug & 31&Aug & 30&Sep & 29&Oct & 28&Nov &
 27&Dec &
 26&Ian & 24&Feb & 26&Mar & 25&Apr & 24&Mai & 23&Iun
\\
  & 14 &  1 & A &
 22&Iul & 21&Aug & 19&Sep & 19&Oct & 17&Nov & 17&Dec &
  \multicolumn{2}{c}{0} &
 15&Ian & 14&Feb & 15&Mar & 14&Apr & 13&Mai & 12&Iun
\\
  & 15 &  2 & G &
 12&Iul & 10&Aug &  9&Sep &  8&Sep &  7&Nov &  6&Dec &
  \multicolumn{2}{c}{0} &
  5&Ian &  3&Feb &  4&Mar &  2&Apr &  2&Mai & 31&Mai
\\
† & 16 &  3 & F E &
 30&Iul & 29&Iul & 28&Aug & 26&Sep & 26&Oct & 25&Nov &
 24&Dec &
 23&Ian & 21&Feb & 23&Mar & 21&Apr & 21&Mai & 19&Iun
\\
  & 17 &  4 & D &
 19&Iul & 17&Aug & 16&Sep & 16&Oct & 14&Nov & 13&Dec &
  \multicolumn{2}{c}{0} &
 12&Ian & 11&Feb & 12&Mar & 11&Apr & 10&Mai &  8&Ian
\\
† & 18 &  5 & C &
  8&Iul &  7&Aug &  5&Sep &  5&Oct &  3&Nov &  3&Dec &
  1&Ian &
 31&Ian &  1&Mar & 31&Mar & 30&Apr & 29&Mai & 28&Iun
\\
  & 19 &  6 & B &
 27&Iul & 26&Aug & 24&Sep & 24&Oct & 22&Nov & 22&Dec &
  \multicolumn{2}{c}{0} &
 20&Ian & 19&Feb & 20&Mar & 19&Apr & 18&Mai & 17&Iun
\\
\bottomrule
\\
& & \multicolumn{29}{l}{\footnotesize \super{†} \textgreek{ἐμβ. [?]}}\\
\end{tabular}
\caption{%
Tabula Neomeniarum Metonicarum in Mensibus Iulianis}

%%
\end{table}

% 79
% {PDF page nr}{source page nr}{line nr}

\plnr{162}{79}Itaque calculus desinet in ducentesima prima syzygia.
\lnr{2}Reliquae igitur
quatuor erunt continue plenae, et duae
primae sequentis anni erunt et ipsae plenae.
\lnr{5}Ita fient sex continue plenae syzygiae.
\lnr{5}Quod est absurdum.
\lnr{6}Porro adiunximus Tabellam
characterismi periodorum, qui characterismus
cum charactere neomeniae compositus dabit feriam
Neomeniae.

% 80
% {PDF page nr}{source page nr}{line nr}

\plnr{163}{80}{2}Exempli gratia.
\lnr{2}Volo scire feriam neomeniae Metagitnionis
Metonici in anno decimo periodi quintae.
\lnr{3}In Tabella
characterismi enneadecaeteridis, sive periodi quintae, habes
3.
\lnr{5}Ille character servit toti periodo, et cum 4 charactere Metagitnionis
anni decimi, abiectis septenariis, ubi opus erit, dat
feriam secundam neomeniae Metagitnionis.
\lnr{7}Adiecimus praeterea
Tabulam neomenarium Metonicarum in mensibus Iulianis,
ut citra[?] laborem eas invenire queas.
\lnr{9}Exemplum: Anno
Nabonassari 366, \textgreek{Θὼθ} $\overline{\kappa\varsigma}$,
 secundum Athenienses autem \textgreek{Φανοστράτου
ἄρχοντος, μηνὸς Ποσειδεῶνος} defecit Luna.
\lnr{11}Tempus
\rnum{xxii} Decembris, sequente \rnum{xxiii}, feria secunda, sequente
tertia, cyclo Solis \rnum{xix}, Lunae \rnum{xviii}, anno periodi Iulianae
4331.
\lnr{14}Erat annus Iphiti 394.
\lnr{14}Abiectis ex methodo perpetua annis
344, remanet annus quinquagesimus Metonis, id est duodecimus
tertiae periodi, cuius periodi tertiae character 4 cum 6 charactere
Posideonis anni \rnum{xii} compositus, abiectis 7, dat feriam 3.

% 81
% {PDF page nr}{source page nr}{line nr}

\plnr{164}{81}{1}In Tabula neomeniarum in annis Iulianis, Posideonis neomenia 
 \rnum{viii}
Decembris, feria secunda.
\lnr{2}Ergo quintadecima Posideonis, sequente
sextadecima, contigit Deliquium.
\lnr{3}Eodem anno tam Nabonassari,
quam Metonis, defecit idem
 sidus \textgreek{φαμηνὼθ[?]} $\overline{\kappa\delta}$,
 sequente $\overline{\kappa\epsilon}$, \textgreek{μηνὺς[?]}
\textgreek{Σσκιῤῥοφοριῶνος[?]}.
% Upper case Sigma *and* lower case sigma?
\lnr{5}Tempus \rnum{xviii} Iunii, sequente \rnum{xix}, feria quinta,
sequente sexta.
\lnr{6}Character 4 periodi tertiae cum charactere
2 Scirrhophorionis in anno 12, dat feriam \rnum{vi} characterem neomeniae
Scirrhophorionis: quae cum ex altera tabula sit in 4 Iunii, feria
quinta, cyclo Solis \rnum{xx}, Eclipsis contigit rursus \rnum{xv} mensis, 
 sequente
\rnum{xvi}.
\lnr{10}Denique anno sequente Nabonassari, et tertiodecimo
tertiae periodi Metonicae idem sidus defecit
 \textgreek{Θὼθ[?]} $\overline{\iota\varsigma}$, sequente
$\overline{\iota\zeta}$, secundum Athenienses
 \textgreek{Ευάνδρου ἄρχοντος, μηνὸς Ποσειδεῶνος προτέρου[?]}.
\lnr{13}Tempus \rnum{xii} Decembris, feria septima, sequente prima.
\lnr{13}Regularis
4 cum 4 compositus, abiecto septenario, dat feriam primam
neomeniae \textgreek{Ποσειδεῶνος προτέρου[?]}.
\lnr{15}Ergo Luna defecit \rnum{xiiii} mensis, sequente
\rnum{xv} et cetera.
\lnr{16}Rursus anno secundo primi Metonici cycli, ineuente[?]
bello Peloponnesiaco, diebus aestivis,
 \textgreek{νουμηνίᾳ κατὰ σελήνω[?]}, ut loquitur
Thucydides, defecit Sol.
\lnr{18}Haec eclipsis contigit anno Iphiti 346, periodi
Iulianae 4283, Augusti tertia die, feria quarta, cyclo Solis 27, Lunae
8, anno Nabonassari 317, Pachon \rnum{viii}, anno uno, et diebus 37, post
observatum a Metone Solstitium.
\lnr{21}In tabula neomeniarum habes
in secundo anno Metonis \textgreek{νουμηνίαν μεταγειτνιῶνος[?]}
 \rnum{iii} Augusti.
\lnr{22}Quod
convenit cum Thucidide, qui vocavit \textgreek{νουμηνίαν κατὰ σελήνην[?]}.
\lnr{23}Neque
de alia neomenia intelligit, quam Metonica.
\lnr{24}Quae etiam erat neomenia
Elul Iudaici anni 3330: cuius character 4.17.609, eadem feria, ut
vides.
\lnr{26}Rursus annus erat 42 periodi quintae Olympiacae, et ideo quartus
Atticae.
\lnr{27}Cuius Hecatombaeon caepit 29 Iulii.
\lnr{27}Ergo defecit Sol \textgreek{ἕκατομβαιῶνος
τῇ ἕκτῃ ἱσταμείου[?]}.
\lnr{28}Rursus Thucydides scribit de anno octavo
belli Peloponnesiaci: \textgreek{τοῦ δ᾽ ἐπιγινομείου θέροις ἐυθὺς,
 τοῦτε ἡλίου ἐλλιπές τι
... περὶ νουμηνίαν, καὶ ἀυτου μηνὸς ἱσταμείου ἔσειδε[?]}.
\lnr{30}Contigit ille defectus
Solaris Augusti \rnum{xvi}, feria quinta, anno periodi Iulianae 4290.
\lnr{31}Erat
annus Iphiteus 353, quadragesimus nonus periodi quintae Olympicae,
ideo undecimus Atticae.
\lnr{33}Hecatombaeon \rnum{iii} Augusti.
\lnr{33}Ergo \rnum{xiiii} contigit
novilunium.
\lnr{34}Rursus erat nonus annus Metonicus.
\lnr{34}Metagitnion \rnum{xvi} Augusti.
\lnr{35}Convenit ergo.
\lnr{35}Elul quoque Iudaicus 3338 non
adversatur.
\lnr{36}Fuit enim 4.21.580. feria quinta.
\lnr{36}Quod autem supra
diximus, quando in cella cavi mensis scriptum est 4.5, id significare
pro quarta mensis, dicendum esse, quinta mensis, noli putare ita a vulgo
usurpari solitum.
\lnr{39}Nam \textgreek{πολιτικῶς[?]} omnis mensis cavi
 \textgreek{δευτέρα[?]} dicebatur
\textgreek{τρίτη[?][?]}.
\lnr{40}Sed intelligendum est Metonem tantum dixisse quartam
pro quinta, methodi caussa, ut hoc modo non ad arbitrium, sed
ad progressum numerorum syzygias erogaret.

% 82
% {PDF page nr}{source page nr}{line nr}
\plnr{165}{82}{1}Statim autem post observationem
Solstitii Metonici hic magnus annus receptus non fuit.
\lnr{3}Nam quarto anno ab eius editione Aristophanes docuit
 \textgreek{νεφέλασ, ἄρχοντος
Αμυνίου[?]},
% Aristophanes: The Clouds (Νεφέλαι)
cum adhuc Athenienses Octaeterida suam mordicus
retinentes \textgreek{τὰς ἡμέρασ οὐκ ἦγον κατὰ τήν σελήνην,
 ἀλλὰ ἄνω τε καὶ κάτω εκυδοιδόπων[?]}.
\lnr{6}Sed non multo post receptum fuisse testatur Avienus.
\begin{quote}
-- Tenuit rem Graecia solers\\
Protinus, et longos inventam misit in annos,\\
Inseditque animis.
\end{quote}
Et celebritatem eius ignorare utique non possumus, Arato canente,
\begin{quote}
-- \textgreek{Τὰ γὰρ συναείδεται ἤδη[?]}\\
\textgreek{Εννεακαίδεκα κύκλα φαεινοῦ ἠελίοιο.[?]}
\end{quote}
Neomeniae vero Metonicae, et Calippicae aliquando neomenias Tetraeteridum
antevertunt mense integro, ut ex Demosthene supra annotavimus:
aliquando paucioribus diebus.
\lnr{15}Praeterea omnes \textgreek{πρυτανεῖαι[?]}
apud Oratores Graecos sunt Metonicae, praeterquam si quae sunt veterum
legum et sanctionum.
\lnr{17}Nam illae sunt Octaetericae, cuiusmodi
quaedam extant \textgreek{ἐν τῷ κατὰ Τιμοκράτοις[?]},
 quas non dubito esse Harpaleas.
\lnr{19}Deprehenso igitur apud Demosthenem anno Iphiti, qui incurrit in
annum propositum \textgreek{πρυτανείας[?]}, abiiece annos Iphiti 344 pro perptua
methodo, ut iam diximus.
\lnr{21}Residuum sunt anni Metonis.
\lnr{21}Deinde
vide quotus annus ille sit a Solstitii Metonici
observatione: et confer illum in lineam
\textgreek{μεταπτώσεως[?]}, in numerum annorum scilicet collectorum
praecisum, si fieri potest: sin aliter in proxime
minorem.
\lnr{26}Columella, sive versus Dierum
ostendet, quot dies accreverunt epochae Metonis.
\lnr{27}Quos dies \textgreek{μεταπτώσεως[?]} adiice neomeniae priscae
Metonis.
\lnr{29}Habebis diem neomeniae Metonicae
in anno proposito.
% Table: LINEA μεταπτὼσεως Metonicae
% It looks like there are 76 Momenta to a Scru.
% and 60 Scru. to a Dies
% 19 Anni collecti give 18 3/4 Scru.
% Check: 304 Anni = 16*19 should give 16*18 3/4 = 300 Scru = 5 Dies. OK.
\begin{table}[htb]
 \centering
 %% Select a general font size (uncomment one from the list)
 %\tiny
 %\scriptsize
 %\footnotesize
 \small
 %\normalsize
 %% Modify distance between rows
 \renewcommand{\arraystretch}{1.1}
 %% Modify separation between columns
 %\setlength{\tabcolsep}{2.0pt}
 %%% Liber II p82
%%
%%% Count out columns for fixed-width source font
% 000000011111111112222222222333333333344444444445555555555666666666677777777778
% 345678901234567890123456789012345678901234567890123456789012345678901234567890
%
%% Select a general font size (uncomment one from the list)
%\tiny
%\scriptsize
%\footnotesize
%\small
%\normalsize
%% Center the whole table left-right
%\centering
%% Modify separation between columns
%\setlength{\tabcolsep}{1.6pt}
%% Modify distance between rows
%\renewcommand{\arraystretch}{1.3}
%%
\begin{tabular}{@{}c c c c c@{} }
\toprule
\multicolumn{5}{c}{\Large\textsc{Linea \textgreek{μεταπτώσεως} Metonicae}}\\
\midrule
\multicolumn{1}{c}{Anni collecti} &
\multicolumn{1}{c}{Dies} &
\multicolumn{1}{c}{Scru. diur.} & % [Abbriv]
\multicolumn{1}{c}{Momenta}
\\
\midrule
  19 &  0 & 18 & 57 \\
  38 &  0 & 37 & 38 \\
  57 &  0 & 56 & 19 \\
  76 &  1 & 15 &  0 \\
  95 &  1 & 33 & 57 \\
 114 &  1 & 52 & 38 \\
 133 &  2 & 11 & 19 \\
 152 &  2 & 30 &  0 \\
 171 &  2 & 40 & 57 \\
 190 &  3 &  7 & 38 \\
 209 &  3 & 26 & 19 \\
 228 &  2 & 45 &  0 \\
 247 &  4 &  3 & 57 \\
 266 &  4 & 22 & 38 \\
 285 &  4 & 41 & 19 \\
 304 &  5 &  0 &  0 \\
\midrule
 608 & 10 &  0 &  0 \\
1216 & 20 &  0 &  0 \\
1824 & 30 &  0 &  0 \\
2432 & 40 &  0 &  0 \\
2736 & 45 &  0 &  0 \\
\bottomrule
\end{tabular}
%

 \caption{Linea \textgreek{μεταπτώσεως} metonicae}
 \label{tab:linea_metaptoseos_metonicae}
\end{table}
\lnr{30}Exemplum.
\lnr{30}Anno Christi
vulgari 1582, aestivis diebus iniit annus Iphiti
Olympiadicus 2358.
\lnr{32}Abiectis 344, resident anni
Iuliani a Solstitio Metonis, 2014.
\lnr{33}Proxime minor
numerus annorum collectorum in linea \textgreek{μεταπτώσεως[?]},
1824.
\lnr{35}Et e regione dies \textgreek{μεταπτώσεως[?]} 30.
\lnr{36}Deductis 1824 de 2014, remanet 190.
\lnr{36}E regione
eorum, in proposita linea \textgreek{μεταπτώσεως[?]}, sunt dies
3.7'.38.
\lnr{38}Qui cum triginta illis dant \textgreek{μεταπτώσεως[?]}
Metonicae dies 33.7'.38.
\lnr{39}Annus propositus Metonis,
est, ut vides, ultimus cycli, in quo neomenia
Hecatombaeonis ab Metone constituta fuit
% à -> ab
Iulii \rnum{xxvii}.

% 83
% {PDF page nr}{source page nr}{line nr}
\plnr{166}{83}{1}Adiice ergo 33 \textgreek{μεταπτώσεως[?]}
 ad \rnum{xxvii} Iulii.
\lnr{1}Pervenitur ad
\rnum{xxix} Augusti, feria quarta.
\lnr{2}Tanta labes rationum Metonicarum facta
est ab initio huius cycli ad nostra tempora.
\lnr{3}Ex his vides, qua via
insistendum sit in neomeniis Metonis apud Demosthenem, et alios
priscos Rhetores investigandis.
\lnr{5}Annus enim Solis Metonicus, praeter
365 dies cum quadrante, habet \myfrac{5}{19}, ut autem Hipparchus dicit,
 \myfrac{1}{76}.
\lnr{6}Verba
Hipparchi apud Ptolemaeum, ex eo libro, quem \textgreek{περι ἐνιαυσίου χρόνου[?]}
scripserat: \textgreek{ὁ ἐνιαύσιος κατὰ οὖς[?] περὶ Μέτωνα,
 καὶ Εἰκτήμονα περιέχει ἡμέρας
τξέ δ´´, καὶ} $\overline{o\varsigma}$
 \textgreek{μιᾶς ἡμέρας[?]}, et cetera.
\lnr{9}Nam si in 76 annis Iulianis accrescit
unus dies Metoni, annus ergo Metonis habuerit \myfrac{1}{76} diei praeter
 365~\myfrac{1}{4}
diei.
\lnr{11}Huic consentanea scribit Censorinus Fr. % [Abbriv]
Pithoei, annum Metonis
Solarem fuisse dierum \rnum{ccclxv}, et praeterea dierum quinque
partis undevicesimae.
% five parts nineteenth
\lnr{13}Item Geminus de periodo Metonis: \textgreek{ἐν δὲ
τῇ περιόδῳ ταύτῃ δοκοῦσιν οἱ μὲν μῆνες καλῶς εἰλῆφθαι, καὶ οἱ ἐμβόλιμοι
συμφώνως τοῖς φαινομείοις διατετάχθαι. ὁ δὲ ἐνιαύσιος χρόνος ἐκ πλειόνων
ἐτῶν παρατετηρημένος συμπεφώνηκεν, ὅτι ἐστὶν ἡμερῶν}
 $\overline{\tau\xi\epsilon}$,
 \textgreek{ἐννεακαιδεκάτων} $\overline\epsilon$ [?].
% Geminus of Rhodes: Introduction to Phaenomena
% http://www.astrologicon.org/geminus/geminus-introduction-to-phaenomena.html
% Γεμῖνος Ῥόδιος, Εἰσαγωγή εἰς τὰ Φαινόμενα 
% Section: Περὶ μηνῶν, last paragraph.
% "Εν δὲ
% τῇ περιόδῳ ταύτῃ δοκοῦσιν οἱ μὲν μῆνες καλῶς εἰλῆφθαι καὶ οἱ ἐμβόλιμοι
% συμφώνως τοῖς φαινομένοις διατετάχθαι, ὁ δὲ ἐνιαύσιος χρόνος [<οὐ>
%   σύμφωνος εἴληπται τοῖς φαινομένοις. Ὁ γὰρ ἐνιαύσιος χρόνος] ἐκ πλειόνων
% ἐτῶν παρατετηρημένος συμπεφώνηκεν ὅτι ἐστὶν ἡμερῶν
% [τξε δ, ὁ δὲ ἐκ τῆς ἐννεακαιδεκαετηρίδος συναγόμενος ἐνιαυτός ἐστιν ἡμερῶν]
% τξε ἐννεακαιδεκάτων ε."
% In German translation:
% "In diesem Cyklus sind dem Anscheine nach die Monate
% richtig genommen und die Schaltmonate mit den
% Himmelserscheinungen übereinstimmend angeordnet. Aber
% die Zeit des Jahres ist nicht mit den Himmelerscheinungen
% in Einklang angenommen. Wenn nämlich die Zeit des
% Jahres aus einer längeren Reihe von Jahren durch Be-
% obachtung festgestellt wird, so hat sich das übereinstim-
% mende Resultat ergeben, dass sie 365 1/4 Tage beträgt,
% während der aus dem 19jährigen Cyklus (durch Rechnung)
% abgeleitete Wert 365 5/19 Tage beträgt. [Dieser letztere
% Wert ist um 1/76 Tag Grösser als die erstere.]"
\lnr{17}Hac ratione in annis \rnum{xix} Metonis Solaribus intercalatur bisextum
quinquies: quatro, octavo, duodecimo, sextodecimo, decimonono.
\lnr{19}Et nihil relinquitur de ratiocinio scrupulario.
\lnr{19}Quare ut cyclus
Solis Iulianus propter quadriennia aequabilia, quater septem annorum
duntaxat est: sic Metonicus cyclus Solis propter inaequalitatem
intervallorum bisexti, novemdecies septem annorum est.
\lnr{22}Neque feriae
restituuntur ante exitum anni 133.
\lnr{23}Si quae de hoc magno anno Metonis
a nobis ignorata vel omissa sunt, ea studiosis colligenda relinquimus.
\lnr{25}Tamen ea; quae diximus, satis esse puto et ad doctrinam anni
Metonici explicandam, et ad eorum iudicia castiganda, qui a nostro
Lunari differre non putant.

%--
\subsection{De Cyclo Metonis Philippeo}

% Table: Menses Tetraeterici et Metonici
\begin{table}[htbp]
 \input{./tables/084_menses_tetraeterici}
 \caption{Menses Tetraeterici et Metonici}
 \label{tab:menses_tetraeterici}
\end{table}

% Table: Cyclus Metonis Philippeus
\begin{table}[htb]
 %%% Liber II p84, PDF 167
%%
%% Select a general font size (uncomment one from the list)
%\tiny
%\scriptsize
\footnotesize
%\small
%\normalsize
%% Center the whole table left-right
\centering
%% Modify separation between columns
%%\setlength{\tabcolsep}{3pt}
%% Modify distance between rows
%\renewcommand{\arraystretch}{1.3}
%%
\begin{tabular}{@{}c r@{~}l l@{}}
\toprule
 \multicolumn{4}{c}{\Large\textsc{Cyclus}}\\
 \multicolumn{4}{c}{\Large\textsc{Metonis}}\\
 \multicolumn{4}{c}{\Large\textsc{Philippeus}}\\
\midrule
\multicolumn{1}{l}{\scriptsize{Linea}}
\\
\multicolumn{1}{l}{\scriptsize{annorum}}
\\
\midrule
  1 & 26 & Martii  & \scriptsize{†} \\
  2 & 13 & Aprilis \\
  3 &  3 & Aprilis & \scriptsize{†} \\
\midrule
  4 & 21 & Aprilis \\
  5 & 10 & Aprilis \\
  6 & 31 & Martii  & \scriptsize{†} \\
\midrule
  7 & 18 & Aprilis \\
  8 &  7 & Aprilis \\
  9 & 28 & Martii  & \scriptsize{†} \\
\midrule
 10 & 15 & Aprilis \\
 11 &  5 & Aprilis & \scriptsize{†} \\
 12 & 23 & Aprilis \\
\midrule
 13 & 12 & Aprilis \\
 14 &  5 & Aprilis & \scriptsize{†} \\
 15 & 20 & Aprilis \\
\midrule
 16 &  9 & Aprilis \\
 17 & 30 & Martii  & \scriptsize{†} \\
 18 & 17 & Aprilis \\
 19 &  6 & Aprilis \\
\bottomrule
\\
 \multicolumn{4}{l}{\footnotesize \super{†} \textgreek{ἐμβολ.}}\\
\end{tabular}
%

\caption{Cyclus Metonis Philippeus}
 \label{tab:cyclus_metonis_philippeus}
\end{table}

\lnr{30}Paucis ante excessum Philippi annis, qui contigit Olympiade
\rnum{cxi}, instituta est periodus in gratiam Philippi, cuius initium incidit
anno tertiodecimo ab eius morte, qui erat mortis Alexandri
eius filii primus: ita ut hoc initium neque ipse, neque eius filius viderint.
\lnr{34}Admissi iam erant eo vivente menses Metonici, qui et ab omnibus
Graecis miro consensu recepti sunt, ut docet Festus Avienus:
\begin{quote}
-- \emph{tenuit rem Graecia solers}\\
\emph{Protinus, et longos inventam misit in annos.}
\end{quote}
% Rufi Festi Avieni: "Carmina quae extant omnia ex recensione Wernsdorfii"
% Wernsdorf, Johann Christian, 1723-1793
% edited by J. A. Giles, LL.D
% Published 1848
%https://archive.org/details/carminaquextant00gilegoog
% p84-131 "Metaphrasis in Arati Phaenomena et Prognostica" ("Aratea")
% Lines 1371-1372 (p118)
% 1369 Illius ad numeros prolixa decennia rursum
% 1370 Adjecisse Meton Cecropia dicitur arte;
% 1371 Inseditque animis, tenuit rem Graecia solers,
% 1372 Protinus, et longos inventum misit in annos.
\lnr{38}Neque solum haec nova periodus in gratiam Philippi instituta, sed et
menses Lunares sua serie luxati.
% Sic: luxati
\lnr{39}Pro Daesio enim nonus mensis Peritius
sumptus, adeo ut neomenia Lunaris Pertii conveniret in Daesium
Tetraetericum.

% 84
% {PDF page nr}{source page nr}{line nr}
\plnr{167}{84}{1}Huius mutationis mentionem
facit Rex ipse epistola ad Pelopennesios:
quae \textgreek{ἐν τῷ περὶ στεφοιύου [?]} extat.
\lnr{3}\textgreek{συναντᾶτε[?]},
inquit, \textgreek{μετὰ τῶν ὅπλων εἰς τὴν Φωκίδα,
 ἔχοντες ἐπισιτισμόν ἡμερῶν τετταράκοντα,
τοῦ ἐνεστῶτος μηνὸς Λώου, ὡς ἡμεῖς ἄγομεν, ὡς δὲ Αθηναῖοι, Βοηδρομιῶνος,
ὡς δὲ Κορίνθιοι Πανέμου [?]}.
% Quote not found
\lnr{6}Eo nomine posuimus menses Tetraetericos
Macedonicos comparatos cum Philippeis Lunaribus, sive
Metonicis, et Metonicos Macedonicos cum Metonicis Atticis.
\lnr{8}Subiecimus etiam filum cycli Metonis Philippei cum epochis neomeniarum
in mensibus Iulianis.

%--
\subsection{De Periodo Calippi Attica Solstitiali}
\lnr{11}Antiquissima fuit fere apud omnes nationes opinio de modo
anni Solaris, quod scilicet tercentis sexaginta quinque diebus
cum quadrante explicaretur, nequis forte putet nostrum annum
non solum a C. % [Abbriv]
 Iulio Caesare publicatum, sed etiam excogitatum esse.
\lnr{15}Is eam anni formam, quam omnes sciebant quidem, sed quam in civiles
usus admiserat hactenus nemo, indixit.
\lnr{16}Ita ut usum eius edicto Caesaris,
scientiam autem antiquorum, qui eam conservarunt, monumentis
debeamus.
\lnr{18}Hinc nonnulli veterum prodidere, Olympiadem illius
diei gratia institutam, qui quarto quoque anno vertente intercalabatur.
\lnr{20}Nam cum annum dierum tantum 360 haberent, singulorum bienniorum
exitu alternis \rnum{x} et \rnum{xi} dies intercalabant, ut annus ad principia
sua rediret.
\lnr{22}Itaque biennium Tetraetericis sacris, quadriennium vero
Olympicis claudebant.
\lnr{23}Atque hoc tandiu obtinuit, donec Tetraeteridum
principia in novilunia conferrentur, quomodo libro proximo
demonstravimus.

% 85
% {PDF page nr}{source page nr}{line nr}
\plnr{168}{85}{1}Nam antiquiorem esse anni Solaris cognitionem
apud Graecos, quam putavit Strabo, facile convincit Romanorum
consuetudo, qui propter quadrantem anni prius intercalationem
instituerunt, quam ullus Graeculus in Aegyptum peregrinaretur.
\lnr{4}Verba
eius optimi scriptoris huc propterea adduximus, quia ad rem valde
pertinere videntur.
\lnr{6}Ita enim libro \rnum{xvii} de Aegyptiis loquitur:
 \textgreek{ουτοι[?] δὲ
τὰ ἐπιτρέχοντα τὴς ἡμέρας καὶ τὴς νυκτὸς μόρια τᾳῖς τριακοσίαις ἑξήκοντα πέντε
ἡμέραις εἰς τὴν ἐκπλήρωσιν τοῦ ἑνιαυσίου χρόνου παρέδοσοιυ[?]}.
\lnr{8}\textgreek{ἀλλ᾽ ἠγνοεῖτο τέως
ὁ ἐνιαυτὸς παρὰ τοῖς ἕλλησιν, ῶς καὶ ἄλλα πλείς, ἕος ὁι νεώτεροι ἀστρόλογοι
παρέλαβον παρὰ τῶν ἑρμηνουόντων εἰς τὸ ἑλληνικὸν τὰ τῶν ἱεξέων ὑπομνήματα[?]}.
\lnr{11}\textgreek{καὶ ἔτι νῦν παραλαμβάνουσι τα ἀπ᾽ ἐκείνων,
 ὁμοίως καὶ τὰ τῶν χαλδαίων[?]}.
% Strabo: Geographica, Book 17, Chapter 1, Section 29
% Στράβων: Γεωγραφικά, βιβλίον ιζʹ, Κεφάλαιον 1
% "οὗτοι δὲ
% τὰ ἐπιτρέχοντα τῆς ἡμέρας καὶ τῆς νυκτὸς μόρια ταῖς τριακοσίαις ἑξήκοντα πέντε
% ἡμέραις εἰς τὴν ἐκπλήρωσιν τοῦ ἐνιαυσίου χρόνου παρέδοσαν·
% ἀλλ᾽ ἠγνοεῖτο τέως
% ὁ ἐνιαυτὸς παρὰ τοῖς Ἕλλησιν ὡς καὶ ἄλλα πλείω͵ ἕως οἱ νεώτεροι ἀστρολόγοι
% παρέλαβον παρὰ τῶν μεθερμηνευσάντων εἰς τὸ Ἑλληνικὸν τὰ τῶν ἱερέων ὑπομνήματα·
% καὶ ἔτι νῦν παραλαμβάνουσι τὰ ἀπ᾽ ἐκείνων͵
%  ὁμοίως καὶ τὰ τῶν Χαλδαίων."
% Translation from Loeb Classical Library, by H.L. Jones,
% Harvard University Press, 1917 thru 1932 (public domain)
% <http://penelope.uchicago.edu/Thayer/E/Roman/Texts/Strabo/17A3*.html>
% "However, these men [Plato and Eudoxus] did teach them [the priests at
% Heliupolis in Egypt] the fractions of the day and night which, running over
% and above the three hundred and sixty five days, fill out the time of
% the true year. But at that time the true year was unknown among the Greeks,
% as also many other things, until the later astrologers learned from the men
% who had translated into Greek the records of the priests; and even to this day
% they learn their teachings, and likewise those of the Chaldaeans. "
\lnr{11}Ab Aegyptiis
et Chaldaeis accepisse quidem fateor, sed \textgreek{χθὲς τυ[?] πρώην[?]},
 ut vult
Strabo: id vero non ego, sed consuetudo ipsa Graecorum refellit.
\lnr{13}Nam
qui certo die \textgreek{τροπὰς[?]} sciebant designare,
 quomodo id poterant sine 365
dierum numero, et quadrantis praeterea accessione?
\lnr{15}Quare vetustissima
est haec \textgreek{ἄῤῥητος καὶ πατροπαράδοτος[?]} de anni Solaris quantitate doctrina.
\lnr{17}Calippus igitur, sive Callippus (utrumque reperio)
 Cyzicenus Mathematicus,
cuius Aristoteles in libris primae Philosophiae meminit,
cum videret quadrantem integrum Metoni supra rationes Solis
abundare in exitu Enneadecaeteridis, atque hinc progressu temporis
magnam turbationem in anni statu consequi, animum ad anni emendationem
appulit, rem pulcherrimam aggressus.
\lnr{22}Ut non minor illi
laus ex castigatione, quam Metoni ex inventione accesserit.
\lnr{23}Sciunt enim
omnes studiosi, quanto in pretio eius periodus fuerit, saltem quibus
Ptolemaeus notus sit.
\lnr{25}Sed non meliore iudicio de periodo Calippi,
quam de Metonis Enneadecaeteride, pronuntiare solent.
\lnr{26}Quia, inquiunt,
in quatuor annis unus dies de ratiociniis Solis resultat ex quatuor
diei quadrantibus consurgens, propterea cyclus Lunaris quadruplicandus,
ut Solis \textgreek{ψηφισμοὶ[?]} cum illis Lunae congruant: cum hoc
intervallo tam Solis, quam Lunae motus emendationem, qua vix exactior
haberi possit, fortiantur.
\lnr{31}Nihil magis a vero alienum legere
memini.
\lnr{32}Quid est aequatio Lunae cum Sole?
\lnr{32}Est eo epilogismos Lunae
deducere, ut nulli scrupuli, si fieri possit; sin aliter, ut saltem quam
paucissimi Lunari ratiocinio reliqui fiant.
\lnr{34}Quo pauciores igitur
Lunae relinquentur scrupuli, eo praecisior erit aequatio.
\lnr{35}Contra, quo
maior summa scrupularia, eo longius aequatio abscedit a vero.
\lnr{36}Maxime
praecisa aequatio Lunae cum Sole sit in novemdecim annis Solaribus
Iulianis.
\lnr{38}Et tunc minor est Lunae epilogismus Solari hora 1.26'.56''.40'''.
\lnr{39}Hic est excessus Solis supra Lunam in diebus 6939, horis 18:
qui sunt anni novemdecim Iuliani.
\lnr{40}In quatuor itaque cyclis excessus
Solis supra Lunam erit horarum 5.47'.46''.40'''.
\lnr{41}Non igitur praecisior
est ratio quatuor cyclorum, quam unius: imo \textgreek{ὑπεροχὴ[?]} longe maior.

% 86
% {PDF page nr}{source page nr}{line nr}
\plnr{169}{86}{2}Qui talia scribunt, nunc primum poterunt discere
 quid sit periodus
Calippica.
\lnr{3}Periodus igitur Calippica, est orbis annorum sex et septuaginta,
quo absoluto Calippus putavit nihil reliquum fieri de scrupulis
Lunae cum rationibus Solis: ita ut fini dierum 27759, nulla supersit
\textgreek{ὑπεροχὴ ἡλιακὴ[?]} supra Lunam.
\lnr{6}Nam annus Solaris censebatur 365 dierum
cum quadrante.
\lnr{7}Qui quadrans a Metone neglectus antevertebat
epochas anni Metonici in 76 annis die uno.
\lnr{8}Quare Calippus de anno
Metonis detraxit quadrantem, et ex quatuor periodis Metonicis,
de quibus singuli quadrantes diurni detracti sunt, composuit periodum
suam.
\lnr{11}Quatuor periodi Metonicae sunt dierum 27760.
\lnr{11}Detractis
quatuor quadrantibus, erit tota periodus dierum 27759: qui
fiunt anni Aegyptiaci 76, dies 19: hoc est anni Iuliani 76.
\lnr{13}Et quia
quatuor periodis Metonicis constat, uno die minus, in una autem periodo
Metonica sunt syzygiae 235, necessario in quatuor erunt 940.
\lnr{16}Quae si essent omnes plenae, \textgreek{καὶ τριακονθήμεροι[?]},
 essent dies 28200: de
quibus si detrahantur dies 27759: remanebit excessus syzygiarum
plenarum, nempe 441.
\lnr{18}Totidem enim syzygiae cavae erunt: quas ita
in totam periodum dispensavit.
\lnr{19}Vidit in Cyclo Metonis 110 cavas
syzygias esse.
\lnr{20}Quae quia, ut vidimus, erogatae sunt per binas syzygias,
et dies quaternos: hoc si fiat in quatuor cyclis, erunt syzygiae
880, dies 1760.
\lnr{22}Qui dies in menses plenos redacti, et cum 880 mensibus
compositi faciunt syzygias 938, dies 20.
\lnr{23}Qui quidem dies viginti
rursus in 440 cavos menses erogati dant \myfrac{1}{22}.
\lnr{24}Itaque si post 22 coniugationes
Lunares et toties 4 dies, detrahatur unus dies, erunt 938
syzygiae, ita ut duae supersint, quarum altera plena erit, altera cava: et
ita, ut in Metonis cyclo, facile colligetur, qui menses cavi, qui pleni
erunt.
\lnr{28}Construximus igitur Tabulam Neomeniarum omnium totius
periodi, cum characteribus suis.
\lnr{29}Si enim meministi quae in Metonica
periodo annotavimus, non opus habes iterum monitore.
\lnr{30}Cellae
enim hic, ut apud Metonem, quae tres numeros habent, eae indicant
menses cavos.
\lnr{32}Superiores bini numeri excedentes sese unitate indicant
pro priore numero dierum mensis, posteriorem sumendum.
\lnr{33}Si enim
invenis 24.25. intelligis nimirum, pro vicesima quarta mensis dicendum
vicesimam quintam.
\lnr{35}Inferior numerus notat characterem neomeniae,
sive regularem, qui cum regulari periodi currentis compositus,
abiectis septem, ubi opus erit, dabit characterem sive feriam neomeniae.
\lnr{38}Denique methodus hic, ut in Metonica periodo, eadem est.
\lnr{39}Orsus est autem periodum suam ab octavo Hecatombaeone Metonico
periodi sextae, 28 Iunii, qui est dies proximus post Solstitium a
Metone et Euctemone observatum.
\lnr{41}Quod enim a Solstitio caeperit,
fidem fecerit primum institutum ipsius Calippi, qui ea mente provinciam
aggressus est, ut Metonis annum iam labantem fulciret: et propterea
sumpsit proximam post solstitium neomeniam, quae tum opportune
statim post confectum solstitium contigit: quanquam ea neomenia
anticipata sit, cum vera neomenia Tamuz Iudaici 3431, cuius
character 3.22.13. sequenti die iniverit, feria quarta.

% 87
% {PDF page nr}{source page nr}{line nr}
\plnr{170}{87}{6}Sed annus octavus
cycli Metonis orditur Hecatombaeonem a 28 Iunii: quod Calippus
rectum esse putavit: et facile potuit notare quadrantem diei a Metone
neglectum, cum intercalatio diei ex quadrantibus consurgentis fieret
inter 27, et 28 Iunii.
\lnr{10}Deinde Ptolemaeus aperte scribit, observationem
solstitii ab Aristarcho factam incurrere annum Calippi quinquagesimum
iam definentem.
\lnr{12}Ita enim libro tertio de Hipparcho
scribit:
 \textgreek{συγκρίνας την ὑπὸ Αριστάρχου τετηρημένην θερινὴν τροπην τῷ[?]}
 $\overline\nu$
\textgreek{ἔτει λήγοντι τὴς πρώτης κατὰ Κάλιππον πριόδου[?]} .
\lnr{14}Primus igitur annus periodi
Calippicae est embolimaeus.
\lnr{15}Alioqui secundi anni neomenia anteverteret
epocham Solstitii diebus undecim.
\lnr{16}Quod ne fieret, religiose cavebant
Attici.
\lnr{17}Nullius enim Hecatombaeonis neomenia apud eos antevertebat
sedem solstitii.
\lnr{18}Embolismi autem Calippici locum in Posideone
non fuisse, patet ex \rnum{xxxvi} anno periodi, qui est embolimaeus,
cuius neomenia apud Ptolemaeum circiter finem Novembris cadit, non
autem circiter finem Decembris, quod quidem contingere debebat,
si Posideon alter intercalatus fuisset: et, quod magis ad coniecturam
facit, neomenia Elaphebolionis illius anni, caepit Februarii
22, non autem Martii 23: quod locum habebat, si Posideon intercalatus
fuisset.
\lnr{25}Rursus 47 anno embolimaeo periodi, Anthesterionis
neomenia competebat diei 22 Ianuarii: anno autem sequente, nempe
48, Pyanepsionis neomenia contigit \rnum{xvi} Octobris.
\lnr{27}Intervallum
dies 266: qui sunt praecise menses novem Lunares.
\lnr{28}Pyanepsion igitur
fuit decimus ab Anthesterione.
\lnr{29}Atqui est nonus.
\lnr{29}Intercalatum
igitur fuit inter Anthesterionem, et Pyanepsionem.
\lnr{30}At cui mensi
magis competit intercalatio, quam Scirrhophorioni, qui est terminus
anni solstitialis Atheniensium, et \textgreek{πρυτανειας[?]}?
\lnr{33}Intercalatus igitur fuit Scirrhoporion alter in fine anni.
\lnr{34}Adiecimus etiam laterculum characteris periodorum
Calippi.
% Table 087_characteris_periodorum_calippi
\begin{table}[htbp]
 %%% Liber II p87
%%
%%% Count out columns for fixed-width source font
% 000000011111111112222222222333333333344444444445555555555666666666677777777778
% 345678901234567890123456789012345678901234567890123456789012345678901234567890
%
%% Select a general font size (uncomment one from the list)
%\tiny
%\scriptsize
%\footnotesize
%\small
%\normalsize
%% Center the whole table left-right
\centering
%% Modify separation between columns
%\setlength{\tabcolsep}{3pt}
%% Modify distance between rows
%\renewcommand{\arraystretch}{1.3}
%%
\begin{tabular}{@{}c c@{}}
\toprule
%\multicolumn{2}{c}{\Large\textsc{Characteris Periodorum Calippi}}\\
%\midrule
~ &
\multicolumn{1}{c}{Character}
\\
\multicolumn{1}{c}{Periodi} &
\multicolumn{1}{c}{periodi}
\\
\midrule
 \rnum{i}    &  3 \\
 \rnum{ii}   &  7 \\
 \rnum{iii}  &  4 \\
 \rnum{iiii} &  1 \\
 \rnum{v}    &  5 \\
 \rnum{vi}   &  2 \\
 \rnum{vii}  &  6 \\
\bottomrule
\end{tabular}
%
%\caption{Characteris Periodorum Calippi}

 \caption{Characteris Periodorum Calippi}
 \label{tab:characteris_calippi}
\end{table}

\lnr{35}Characteres vocamus regulares, qui
cum regulari, sive charactere neomeniarum compositi
dant feriam neomeniae.
\lnr{37}Si enim vis habere feriam neomeniae
Pyanepsionis in anno quadragesimo nono periodi
tertiae, acceptum regularem 4, respondentem tertiae periodo
in Laterculo, compone cum 4 charactere Pyanepsionis.
\lnr{41}Is compositus e regione anni 49 in Tabula
neomeniarum, in areae communi angulo sub Pyanepsione, dabit feriam
primam neomeniae ipsius Pyanepsionis.

% 88
% {PDF page nr}{source page nr}{line nr}
\plnr{171}{88}{2}Sed experiamur ex Ptolemaeo.
\lnr{3}Anno Nabonassari 466, Timocharis observavit Lunam cum
spica Virginis coniunctam, \textgreek{Θὼθ[?]} $\overline\rho$,
anno primae periodi Calippicae 48,
\textgreek{Πυδυεψιῶνος[?]} $\overline\varsigma$ \textgreek{φθίνοντος[?]}.
\lnr{5}Tempus Iulianum huic congruens, Novembris
\rnum{viii}, feria \rnum{vi}.
\lnr{6}Ergo neomenia Pyanepsionis feria tertia, Octobris
quintadecima.
\lnr{7}Periclitare in Tabula.
% Tabula
\lnr{7}Compone Regularem 3
respondentem primae periodo in Laterculo cum 7 charactere Pyanepsionis,
è regione 48 anni.
% [Abbriv] 'ex'?
\lnr{9}Habebis 3 feriam, ut propositum erat.
\lnr{9}Convenit
igitur.
\lnr{10}Et character Marchesvan Iudaici 3479 idem praestat.
% 'Marchesvvan' in original. 'Marcheswan' (double v) in 1598 edition.
\lnr{11}Fuit enim 3.7.598.
\lnr{11}Rursus anno Nabonassari antecedente 465, \textgreek{Αθὺρ[?]}
$\overline{\kappa\theta}$, Calippi autem 47, \textgreek{Ανθεστηριῶνοσ[?]}
 $\overline\eta$, idem Timocharis observavit
mediam partem Lunae in medium Pleiadum inductam.
\lnr{13}Tempus Iulianum,
Ianuarii \rnum{xxix}, feria tertia, cyclo Solis \rnum{vii}.
\lnr{14}Regularis primae
periodi, nempe 3 cum 7 charactere Anthesterionis in anno 47 compositus
dat feriam tertiam, ut proposuit Timocharis: et convenit cum
charactere Scebat Iudaici 3478.
\lnr{17}Fuit enim 3.13.22. Ianuarii \rnum{xxii}.
\lnr{18}Anno Nabonassari 454, \textgreek{Τυβὶ[?]} $\overline\epsilon$
% Τυβὶ from 1598 edition
 exacto, sequente $\overline\varsigma$, Periodi Calippicae
36, \textgreek{ἐλαφεβολιῶνοσ[?]} $\overline{\iota\epsilon}$, idem
 Timocharis observavit coniunctionem
Lunae cum Spica.
\lnr{20}Tempus Iulianum, Martii \rnum{ix}, feria septima.
\lnr{21}Ergo neomenia Elaphebolionis \rnum{xxiii} Februarii, feria septima.
\lnr{21}Regularis
3 primae periodi cum 4 regulari Elaphebolionis in anno 36 compositus
dabit propositam feriam septimam: qui tamen non convenit
cum charactere 1.10.599 Adar Iudaici 3467, feria prima, 24 Februarii.
\lnr{25}Denique eodem anno antea idem Timocharis observavit
Lunam tangentem Septentrionalem earum stellarum, quae sunt in
fronte Tauri, Paophi \rnum{xvi}, Posideonis \rnum{xxiii}, feria quinta.
\lnr{27}Tempus \rnum{xx}
Decembris.
\lnr{28}Ergo \rnum{xxviii} Novembris neomenia Posideonis, feria
quarta.
\lnr{29}Quod falsum est.
\lnr{30}Nam hoc modo intervallum a Neomenia
Posideonis, ad neomeniam Elaphebolionis, hoc est a 28 Novembris,
ad 23 Februarii, fuerit dierum duntaxat 87.
\lnr{31}Qui sunt minores tribus syzygiis
saltem uno die.
\lnr{32}Error igitur est in Codice Ptolemaei, quanquam
cum charactere Iudaici Casleu 3467 convenit: qui fuit 3.20.380,
feria quarta.
\lnr{34}Sed in Calippi methodo fuerit feria secunda, Novembris
26.
\lnr{35}Nam 3 character periodi primae cum 6 regulari Posideonis
in anno 36 compositus, abiecto septenario, dederit feriam secundam.
\lnr{37}Necesse igitur omnino erratum fuisse.

% 89
% {PDF page nr}{source page nr}{line nr}
%\plnr{172}{89}{1}

% Table: Tabula Neomenarium Periodi Calippicae
% (two pages)
%%% Liber II p89-90
%%
%%% Count out columns for fixed-width source font
% 000000011111111112222222222333333333344444444445555555555666666666677777777778
% 345678901234567890123456789012345678901234567890123456789012345678901234567890
%
\begingroup
\tiny
%\scriptsize
%\footnotesize
%\small
%\normalsize
%% Modify separation between columns
\setlength{\tabcolsep}{2.5pt}
%% Modify distance between rows
\renewcommand{\arraystretch}{0.9}
%% Let longtable process the whole table in one go
\setcounter{LTchunksize}{100}
%
%% Define a smaller dagger (unfortunalely tiny is already the smallest)
\newcommand{\da}{{\tiny †}}
%% Command for lines between the rows
\newcommand{\streep}{\cmidrule{2-35}}
%% Command for entries that span 2 columns
\newcommand{\mc}[1]{\multicolumn{2}{c}{#1}}
%%
\begin{longtable}[c]{@{}%
 c c c  r@{~}l r@{~}l r@{~}l r@{~}l r@{~}l r@{~}l
r@{~}l r@{~}l r@{~}l r@{~}l r@{~}l r@{~}l r@{~}l  c c c c r@{~}l
@{}}
\toprule
\multicolumn{35}{c}{\Large\textsc{Tabula neomeniarum periodi Calippicae}}\\
\toprule
% Put entry into List of Tables pointing to the first page of the table
\addcontentsline{lot}{section}{%
\protect\numberline{\thetable}Neomeniarum periodi Calippicae}
\label{tab:p089}
% Read the header description from an external file
% Header for table p89-90
% Version with slanted headers for the names of the months
~ &
\begin{turn}{90}Anni periodi\end{turn} &
\begin{turn}{90}Cyclus Lunae\end{turn} & 

\begin{rotate}{75}\textgreek{Εκατομβαιών}\end{rotate} & &
\begin{rotate}{75}\textgreek{Μεταγειτνιών}\end{rotate} & &
\begin{rotate}{75}\textgreek{Βοηδρομιών}\end{rotate} & &

\begin{rotate}{75}\textgreek{Πυανεψιών}\end{rotate} & &
\begin{rotate}{75}\textgreek{Μαιμακτηριών}\end{rotate} & &
\begin{rotate}{75}\textgreek{Ποσειδεών}\end{rotate} & &

\begin{rotate}{75}\textgreek{Γαμηλιών}\end{rotate} & &
\begin{rotate}{75}\textgreek{Ανθεστηριών}\end{rotate} & &
\begin{rotate}{75}\textgreek{Ελαφηβολιών}\end{rotate} & &

\begin{rotate}{75}\textgreek{Μουνυχιών}\end{rotate} & &
\begin{rotate}{75}\textgreek{Θαργηλιών}\end{rotate} & &
\begin{rotate}{75}\textgreek{Σκιῤῥοφοριών α}\end{rotate} & &
\begin{rotate}{75}\textgreek{Σκιῤῥοφοριών β}\end{rotate} & &

\multicolumn{1}{c}{\begin{turn}{90}Dies collecti\end{turn}} & 
\multicolumn{1}{c}{\begin{turn}{90}Syzygiae collectae\end{turn}} & 
\multicolumn{1}{c}{\begin{turn}{90}Menses cavi[?]\end{turn}} & 
\multicolumn{1}{c}{\begin{turn}{90}Syclus Solis\end{turn}} & 
\multicolumn{1}{r}{\begin{turn}{90}Neomenia\end{turn}} & 
\multicolumn{1}{l}{\begin{turn}{90}Ecatombaeonis\end{turn}}
\\

\midrule
\endfirsthead
%%
\toprule
\multicolumn{35}{c}{%
\large\textsc{Residuum tabulae neomeniarum periodi Calippicae}}\\
\toprule
% Read the header description from an external file
% Header for table p89-90
% Version with slanted headers for the names of the months
~ &
\begin{turn}{90}Anni periodi\end{turn} &
\begin{turn}{90}Cyclus Lunae\end{turn} & 

\begin{rotate}{75}\textgreek{Εκατομβαιών}\end{rotate} & &
\begin{rotate}{75}\textgreek{Μεταγειτνιών}\end{rotate} & &
\begin{rotate}{75}\textgreek{Βοηδρομιών}\end{rotate} & &

\begin{rotate}{75}\textgreek{Πυανεψιών}\end{rotate} & &
\begin{rotate}{75}\textgreek{Μαιμακτηριών}\end{rotate} & &
\begin{rotate}{75}\textgreek{Ποσειδεών}\end{rotate} & &

\begin{rotate}{75}\textgreek{Γαμηλιών}\end{rotate} & &
\begin{rotate}{75}\textgreek{Ανθεστηριών}\end{rotate} & &
\begin{rotate}{75}\textgreek{Ελαφηβολιών}\end{rotate} & &

\begin{rotate}{75}\textgreek{Μουνυχιών}\end{rotate} & &
\begin{rotate}{75}\textgreek{Θαργηλιών}\end{rotate} & &
\begin{rotate}{75}\textgreek{Σκιῤῥοφοριών α}\end{rotate} & &
\begin{rotate}{75}\textgreek{Σκιῤῥοφοριών β}\end{rotate} & &

\multicolumn{1}{c}{\begin{turn}{90}Dies collecti\end{turn}} & 
\multicolumn{1}{c}{\begin{turn}{90}Syzygiae collectae\end{turn}} & 
\multicolumn{1}{c}{\begin{turn}{90}Menses cavi[?]\end{turn}} & 
\multicolumn{1}{c}{\begin{turn}{90}Syclus Solis\end{turn}} & 
\multicolumn{1}{r}{\begin{turn}{90}Neomenia\end{turn}} & 
\multicolumn{1}{l}{\begin{turn}{90}Ecatombaeonis\end{turn}}
\\

\midrule
\endhead
%%
%\bottomrule
\addlinespace[8pt]
& & \multicolumn{29}{l}{\footnotesize \super{†} \textgreek{ἐμβολ. [Abbriv.]}}\\
\endfoot
%%
\bottomrule
\addlinespace[8pt]
& & \multicolumn{29}{l}{\footnotesize \super{†} \textgreek{ἐμβολ. [Abbriv.]}}\\
\addlinespace
% Put the table nr and title below the table, without entry in the LoT
\caption[]{Neomeniarum periodi Calippicae}
\endlastfoot
%%
  &    &    &
     &   &    &   &  4.&5  &    &   &  8.&9  &    &   &
  12.&13 &    &   & 16.&17 &    &   & 20.&21 &    &   &
  24.&25 &
  \\
\nopagebreak
\da &  1 & 14 &
  \mc{7} & \mc{2} & \mc{4} & \mc{5} & \mc{7} & \mc{1} &
  \mc{3} & \mc{4} & \mc{6} & \mc{7} & \mc{2} & \mc{3} &
  \mc{5} &
   384  &  13 &   6 & B & 28&Iun \\
\nopagebreak
%
\streep
  &    &   &
     &   & 28.&29 &    &   &    &   &  2.&3  &    &   &
   6.&7  &    &   & 10.&11 &    &   & 14.&15 &    &   &
     &   &
  \\
\nopagebreak
  &  2 & 15 &
  \mc{6} & \mc{1} & \mc{2} & \mc{4} & \mc{6} & \mc{7} &
  \mc{2} & \mc{3} & \mc{5} & \mc{6} & \mc{1} & \mc{2} &
  \mc{0} &
   739  &  25 &  11 & A G & 16&Iul \\
\nopagebreak
%
\streep
  &    &    &
  18.&19 &    &   & 23.&23\footnote{23.23: Sic.} &    &   & 26.&27 &    &   &
% '23.23' in original
  30.&31 &    &   &    &   &  4.&5  &    &   &  8.&9  &
     &   &
  \\
\nopagebreak
\da &  3 & 16 &
  \mc{4} & \mc{5} & \mc{7} & \mc{1} & \mc{3} & \mc{4} &
  \mc{6} & \mc{7} & \mc{2} & \mc{4} & \mc{5} & \mc{7} &
  \mc{1} &
  1123  &  38 &  17 & F &  6&Iul \\
\nopagebreak
%
\streep
  &    &    &
  12.&13 &    &   & 16.&17 &    &   & 20.&21 &    &   &
  24.&25 &    &   & 27.&28 &    &   &    &   &  1.&2  &
     &   &
  \\
\nopagebreak
  &  4 & 17 &
  \mc{3} & \mc{4} & \mc{6} & \mc{7} & \mc{2} & \mc{3} &
  \mc{5} & \mc{6} & \mc{1} & \mc{2} & \mc{4} & \mc{6} &
  \mc{0} &
  1477  &  50 &  23 & E & 25&Iul \\
\streep
\nopagebreak
%
  &    &    &
     &   &  5.&6  &    &   &  9.&10 &    &   & 13.&14 &
     &   & 17.&18 &    &   & 21.&22 &    &   & 25.&26 &
     &   &
  \\
\nopagebreak
  &  5 & 18 &
  \mc{7} & \mc{2} & \mc{3} & \mc{5} & \mc{6} & \mc{1} &
  \mc{2} & \mc{4} & \mc{5} & \mc{7} & \mc{1} & \mc{3} &
  \mc{0} &
  1831  &  62 &  29 & D & 14&Iul \\
\nopagebreak
%
\streep
  &    &   &
     &   & 29.&30 &    &   &    &   &  3.&4  &    &   &
   7.&8  &    &   & 11.&12 &    &   & 15.&16 &    &   &
  19.&20 &
  \\
\nopagebreak
\da &  6 & 19 &
  \mc{4} & \mc{6} & \mc{7} & \mc{2} & \mc{4} & \mc{5} &
  \mc{7} & \mc{1} & \mc{3} & \mc{4} & \mc{6} & \mc{7} &
  \mc{2} &
  2215  &  75 &  35 & C B &  2&Iul \\
\nopagebreak
%
\streep
  &    &   &
     &   & 23.&24 &    &   & 27.&28 &    &   &    &   &
  11.&12 &    &   &  5.&6  &    &   &  9.&10 &    &   &
     &   &
  \\
\nopagebreak
  &  7 &  1 &
  \mc{3} & \mc{5} & \mc{6} & \mc{1} & \mc{2} & \mc{4} &
  \mc{6} & \mc{7} & \mc{2} & \mc{3} & \mc{5} & \mc{6} &
  \mc{0} &
  2570  &  87 &  40 & A &  21&Iul \\
\nopagebreak
%
\streep
  &    &    &
  13.&14 &    &   & 17.&18 &    &   & 21.&22 &    &   &
  24.&25 &    &   & 28.&29 &    &   &    &   &  2.&3  &
     &   &
  \\
\nopagebreak
  &  8 &  2 &
  \mc{1} & \mc{2} & \mc{4} & \mc{5} & \mc{7} & \mc{1} &
  \mc{3} & \mc{4} & \mc{6} & \mc{7} & \mc{2} & \mc{4} &
  \mc{0} &
  2924  &  99 &  46 & G & 11&Iul \\
%\nopagebreak
%
\streep
  &    &    &
     &   &  6.&7  &    &   & 10.&11 &    &   & 14.&15 &
     &   & 18.&19 &    &   & 22.&23 &    &   & 26.&27 &
     &   &
  \\
\nopagebreak
\da &  9 &  3 &
  \mc{5} & \mc{7} & \mc{1} & \mc{3} & \mc{4} & \mc{6} &
  \mc{7} & \mc{2} & \mc{3} & \mc{5} & \mc{6} & \mc{1} &
  \mc{2} &
  3308  & 112 &  52 & F & 30&Iun \\
\nopagebreak
%
\streep
  &    &    &
  30.&1  &    &   &    &   &  4.&5  &    &   &  8.&9  &
     &   & 12.&13 &    &   & 16.&17 &    &   & 20.&21 &
     &   &
  \\
\nopagebreak
  & 10 &  4 &
  \mc{4} & \mc{5} & \mc{7} & \mc{2} & \mc{3} & \mc{5} &
  \mc{6} & \mc{1} & \mc{2} & \mc{4} & \mc{5} & \mc{7} &
  \mc{0} &
  3662  &  12 &  58 & E D & 18&Iul \\
\nopagebreak
%
\streep
  &    &   &
     &   & 20.&25 &    &   & 28.&29 &    &   &  2.&3  &
     &   &  6.&7  &    &   &    &   & 10.&11 &    &   &
  14.&15 &
  \\
\nopagebreak
\da & 11 &  5 &
  \mc{1} & \mc{3} & \mc{4} & \mc{6} & \mc{7} & \mc{2} &
  \mc{3} & \mc{5} & \mc{6} & \mc{1} & \mc{3} & \mc{4} &
  \mc{6} &
  4046  & 135 &  64 & C &   7&Iul \\
\nopagebreak
%
\streep
  &    &   &
     &   & 18.&19 &    &   & 21.&22 &    &   & 25.&26 &
     &   & 29.&30 &    &   &    &   &  3.&4  &    &   &
     &   &
  \\
\nopagebreak
  & 12 &  6 &
  \mc{7} & \mc{2} & \mc{3} & \mc{5} & \mc{6} & \mc{1} &
  \mc{2} & \mc{4} & \mc{5} & \mc{7} & \mc{2} & \mc{3} &
  \mc{0} &
  4401  & 149 &  69 & B &  26&Iul \\
%\nopagebreak
%
\streep
  &    &    &
   7.&8  &    &   & 11.&12 &    &   & 15.&16 &    &   &
  19.&20 &    &   & 23.&24 &    &   & 27.&28 &    &   &
     &   &
  \\
\nopagebreak
  & 13 &  7 &
  \mc{5} & \mc{6} & \mc{1} & \mc{2} & \mc{4} & \mc{5} &
  \mc{7} & \mc{1} & \mc{3} & \mc{4} & \mc{6} & \mc{7} &
  \mc{0} &
  4755  & 161 &  75 & A & 16&Iul \\
\nopagebreak
%
\streep
  &    &    &
     &   &  1.&2  &    &   &  5.&6  &    &   &  9.&10 &
     &   & 13.&14 &    &   & 17.&18 &    &   & 21.&22 &
     &   &
  \\
\nopagebreak
\da & 14 &  8 &
  \mc{2} & \mc{4} & \mc{5} & \mc{7} & \mc{1} & \mc{3} &
  \mc{4} & \mc{6} & \mc{7} & \mc{2} & \mc{3} & \mc{5} &
  \mc{6} &
  5139  & 174 &  81 & G F &  4&Iul \\
\nopagebreak
%
\streep
  &    &    &
  25.&26 &    &   & 29.&30 &    &   &    &   &  3.&4  &
     &   &  7.&8  &    &   & 11.&12 &    &   & 15.&16 &
     &   &
  \\
\nopagebreak
  & 15 &  9 &
  \mc{1} & \mc{2} & \mc{4} & \mc{5} & \mc{7} & \mc{2} &
  \mc{3} & \mc{5} & \mc{6} & \mc{1} & \mc{2} & \mc{4} &
  \mc{0} &
  5493  & 186 &  87 & E & 23&Iul \\
\nopagebreak
%
\streep
  &    &   &
     &   & 18.&19 &    &   & 22.&23 &    &   & 26.&27 &
     &   & 30.&1  &    &   &    &   &  4.&5  &    &   &
     &   &
  \\
\nopagebreak
  & 16 & 10 &
  \mc{5} & \mc{7} & \mc{1} & \mc{3} & \mc{4} & \mc{6} &
  \mc{7} & \mc{2} & \mc{3} & \mc{5} & \mc{7} & \mc{1} &
  \mc{0} &
  5848  & 198 &  92 & D &  12&Iul \\
%\nopagebreak
%
\streep
  &    &    &
   8.&9  &    &   & 12.&13 &    &   & 16.&17 &    &   &
  20.&21 &    &   & 24.&25 &    &   & 28.&29 &    &   &
     &   &
  \\
\nopagebreak
\da & 17 & 11 &
  \mc{3} & \mc{4} & \mc{6} & \mc{7} & \mc{2} & \mc{3} &
  \mc{5} & \mc{6} & \mc{1} & \mc{2} & \mc{4} & \mc{5} &
  \mc{7} &
  6232  & 211 &  98 & C &  2&Iul \\
\nopagebreak
%
\streep
  &    &    &
   2.&3  &    &   &  6.&7  &    &   & 10.&11 &    &   &
  14.&15 &    &   & 18.&19 &    &   & 22.&23 &    &   &
     &   &
  \\
\nopagebreak
  & 18 & 12 &
  \mc{2} & \mc{3} & \mc{5} & \mc{6} & \mc{1} & \mc{2} &
  \mc{4} & \mc{5} & \mc{7} & \mc{1} & \mc{3} & \mc{4} &
  \mc{0} &
  6586  & 223 & 104 & B A &  20&Iul \\
\nopagebreak
%
\streep
  &    &    &
  26.&27 &    &   & 30.&1  &    &   &    &   &  4.&5  &
     &   &  8.&9  &    &   & 12.&13 &    &   & 15.&16 &
     &   &
  \\
\nopagebreak
  & 19 & 13 &
  \mc{6} & \mc{7} & \mc{2} & \mc{3} & \mc{5} & \mc{7} &
  \mc{1} & \mc{3} & \mc{4} & \mc{6} & \mc{7} & \mc{2} &
  \mc{0} &
  6940  & 235 & 110 & G &  9&Iul \\
\nopagebreak
%
\streep
  &    &   &
     &   & 19.&20 &    &   & 23.&24 &    &   & 27.&28 &
     &   &    &   &  1.&2  &    &   &  5.&6  &    &   &
   9.&10 &
  \\
\nopagebreak
\da & 20 & 14 &
  \mc{3} & \mc{5} & \mc{6} & \mc{1} & \mc{2} & \mc{4} &
  \mc{5} & \mc{7} & \mc{2} & \mc{3} & \mc{5} & \mc{6} &
  \mc{1} &
  7324  & 248 & 116 & F &  28&Iun \\
\nopagebreak
%
\streep
  &    &   &
     &   & 13.&14 &    &   & 17.&18 &    &   & 21.&22 &
     &   & 25.&26 &    &   & 29.&30 &    &   &    &   &
     &   &
  \\
\nopagebreak
  & 21 & 15 &
  \mc{2} & \mc{4} & \mc{5} & \mc{7} & \mc{1} & \mc{3} &
  \mc{4} & \mc{6} & \mc{7} & \mc{2} & \mc{3} & \mc{5} &
  \mc{0} &
  7679  & 260 & 121 & E &  17&Iul \\
\nopagebreak
%
\streep
  &    &    &
   3.&4  &    &   &  7.&8  &    &   & 11.&12 &    &   &
  15.&16 &    &   & 19.&20 &    &   & 23.&24 &    &   &
  27.&28 &
  \\
\nopagebreak
\da & 22 & 16 &
  \mc{7} & \mc{1} & \mc{3} & \mc{4} & \mc{6} & \mc{7} &
  \mc{2} & \mc{3} & \mc{5} & \mc{6} & \mc{1} & \mc{2} &
  \mc{4} &
  8062  & 273 & 128 & D C &   6&Iul \\
\nopagebreak
%
\streep
  &    &    &
     &   &    &   &  1.&2  &    &   &  5.&6  &    &   &
   9.&10 &    &   & 12.&13 &    &   & 16.&17 &    &   &
     &   &
  \\
\nopagebreak
  & 23 & 17 &
  \mc{5} & \mc{7} & \mc{2} & \mc{3} & \mc{5} & \mc{6} &
  \mc{1} & \mc{2} & \mc{4} & \mc{5} & \mc{7} & \mc{1} &
  \mc{0} &
  8417  & 285 & 133 & B &  24&Iul \\
\nopagebreak
%
\streep
  &    &    &
  20.&21 &    &   & 24.&25 &    &   & 28.&29 &    &   &
     &   &  2.&3  &    &   &  6.&7 &    &   & 10.&11 &
     &   &
  \\
\nopagebreak
  & 24 & 18 &
  \mc{3} & \mc{4} & \mc{6} & \mc{7} & \mc{2} & \mc{3} &
  \mc{5} & \mc{7} & \mc{1} & \mc{3} & \mc{4} & \mc{6} &
  \mc{0} &
  8771  & 297 & 139 & A & 14&Iul \\
\nopagebreak
%
\streep
  &    &   &
     &   & 14.&15 &    &   & 18.&19 &    &   & 22.&23 &
     &   & 26.&27 &    &   & 30.&1  &    &   &    &   &
   4.&5  &
  \\
\nopagebreak
\da & 25 & 19 &
  \mc{7} & \mc{2} & \mc{3} & \mc{5} & \mc{6} & \mc{1} &
  \mc{2} & \mc{4} & \mc{5} & \mc{7} & \mc{1} & \mc{3} &
  \mc{5} &
  9155  & 310 & 145 & G &   3&Iul \\
\nopagebreak
%
\streep
  &    &    &
     &   &  8.&9  &    &   & 12.&13 &    &   & 16.&17 &
     &   & 20.&21 &    &   & 24.&25 &    &   & 28.&29 &
     &   &
  \\
\nopagebreak
  & 26 &  1 &
  \mc{6} & \mc{1} & \mc{2} & \mc{4} & \mc{5} & \mc{7} &
  \mc{1} & \mc{3} & \mc{4} & \mc{6} & \mc{7} & \mc{2} &
  \mc{0} &
  9509  & 322 & 151 & F E & 21&Iul \\
\nopagebreak
%
\streep
  &    &    &
     &   &    &   &  2.&3  &    &   &  6.&7  &    &   &
   9.&10 &    &   & 13.&14 &    &   & 17.&18 &    &   &
     &   &
  \\
\nopagebreak
  & 27 &  2 &
  \mc{3} & \mc{5} & \mc{7} & \mc{1} & \mc{3} & \mc{4} &
  \mc{6} & \mc{7} & \mc{2} & \mc{3} & \mc{5} & \mc{6} &
  \mc{0} &
  9864  & 334 & 156 & D &  10&Iul \\
\nopagebreak
%
\streep
  &    &    &
  21.&22 &    &   & 25.&26 &    &   & 29.&30 &    &   &
     &   &  3.&4  &    &   &  7.&8 &    &   & 11.&12 &
     &   &
  \\
\nopagebreak
\da & 28 &  3 &
  \mc{1} & \mc{2} & \mc{4} & \mc{5} & \mc{7} & \mc{1} &
  \mc{3} & \mc{5} & \mc{6} & \mc{1} & \mc{2} & \mc{4} &
  \mc{5} &
 10248  & 347 & 162 & C & 30&Iun \\
\nopagebreak
%
\streep
  &    &    &
  15.&16 &    &   & 19.&20 &    &   & 23.&24 &    &   &
  27.&28 &    &   &    &   &  1.&2  &    &   &  5.&6  &
     &   &
  \\
\nopagebreak
  & 29 &  4 &
  \mc{7} & \mc{1} & \mc{3} & \mc{4} & \mc{6} & \mc{7} &
  \mc{2} & \mc{3} & \mc{5} & \mc{7} & \mc{1} & \mc{3} &
  \mc{0} &
 10602  & 359 & 168 & B & 19&Iul \\
\nopagebreak
%
\streep
  &    &    &
     &   &  9.&10 &    &   & 13.&14 &    &   & 17.&18 &
     &   & 21.&22 &    &   & 25.&26 &    &   & 29.&30 &
     &   &
  \\
\nopagebreak
\da & 30 &  5 &
  \mc{4} & \mc{6} & \mc{7} & \mc{2} & \mc{3} & \mc{5} &
  \mc{6} & \mc{1} & \mc{2} & \mc{4} & \mc{5} & \mc{7} &
  \mc{1} &
 10986  & 372 & 174 & A G &  7&Iul \\
\nopagebreak
%
\streep
  &    &    &
     &   &  3.&4  &    &   &  6.&7  &    &   & 10.&11 &
     &   & 14.&15 &    &   & 18.&19 &    &   & 22.&23 &
     &   &
  \\
\nopagebreak
  & 31 &  6 &
  \mc{3} & \mc{5} & \mc{6} & \mc{1} & \mc{2} & \mc{4} &
  \mc{5} & \mc{7} & \mc{1} & \mc{3} & \mc{4} & \mc{6} &
  \mc{0} &
 11340  & 384 & 180 & F & 26&Iul \\
\nopagebreak
%
\streep
  &    &   &
     &   & 26.&27 &    &   & 30.&1  &    &   &    &   &
   4.&5  &    &   &  8.&9  &    &   & 12.&13 &    &   &
     &   &
  \\
\nopagebreak
  & 32 &  7 &
  \mc{7} & \mc{2} & \mc{3} & \mc{5} & \mc{6} & \mc{1} &
  \mc{3} & \mc{4} & \mc{6} & \mc{7} & \mc{2} & \mc{3} &
  \mc{0} &
 11695  & 396 & 185 & E &  15&Iul \\
\nopagebreak
%
\streep
  &    &    &
  16.&17 &    &   & 20.&21 &    &   & 24.&25 &    &   &
  28.&29 &    &   &    &   &  2.&3  &    &   &  6.&7  &
     &   &
  \\
\nopagebreak
\da & 33 &  8 &
  \mc{5} & \mc{6} & \mc{1} & \mc{2} & \mc{4} & \mc{5} &
  \mc{7} & \mc{1} & \mc{3} & \mc{5} & \mc{6} & \mc{1} &
  \mc{2} &
 12079  & 409 & 191 & D &  5&Iul \\
\nopagebreak
%
\streep
  &    &    &
  10.&11 &    &   & 14.&15 &    &   & 18.&19 &    &   &
  22.&23 &    &   & 26.&27 &    &   & 30.&1  &    &   &
     &   &
  \\
\nopagebreak
  & 34 &  9 &
  \mc{4} & \mc{5} & \mc{7} & \mc{1} & \mc{3} & \mc{4} &
  \mc{6} & \mc{7} & \mc{2} & \mc{3} & \mc{5} & \mc{6} &
  \mc{0} &
 12433  & 421 & 197 & C B &  23&Iul \\
\nopagebreak
%
\streep
  &    &    &
     &   &  3.&4  &    &   &  7.&8  &    &   & 11.&12 &
     &   & 15.&16 &    &   & 19.&20 &    &   & 23.&24 &
     &   &
  \\
\nopagebreak
  & 35 & 10 &
  \mc{1} & \mc{3} & \mc{4} & \mc{6} & \mc{7} & \mc{2} &
  \mc{3} & \mc{5} & \mc{6} & \mc{1} & \mc{2} & \mc{4} &
  \mc{0} &
 12787  & 433 & 203 & A & 12&Iul \\
\nopagebreak
%
\streep
  &    &   &
     &   & 27.&28 &    &   &    &   &  1.&2  &    &   &
   5.&6  &    &   &  9.&10 &    &   & 13.&14 &    &   &
  17.&18 &
  \\
\nopagebreak
\da & 36 & 11 &
  \mc{5} & \mc{7} & \mc{1} & \mc{3} & \mc{5} & \mc{6} &
  \mc{1} & \mc{2} & \mc{4} & \mc{5} & \mc{7} & \mc{1} &
  \mc{3} &
 13171  & 446 & 209 & G & Ka.&Iul \\
\nopagebreak
%
\streep
  &    &    &
     &   & 21.&22 &    &   & 25.&26 &    &   & 29.&30 &
     &   &    &   &  3.&4  &    &   &  7.&8  &    &   &
     &   &
  \\
\nopagebreak
  & 37 & 12 &
  \mc{4} & \mc{6} & \mc{7} & \mc{2} & \mc{3} & \mc{5} &
  \mc{6} & \mc{1} & \mc{3} & \mc{4} & \mc{6} & \mc{7} &
  \mc{0} &
 13526  & 458 & 214 & F & 20&Iul \\
\nopagebreak
%
\streep
  &    &    &
  11.&12 &    &   & 15.&16 &    &   & 19.&20 &    &   &
  23.&24 &    &   & 27.&28 &    &   & 30.&1  &    &   &
     &   &
  \\
\nopagebreak
  & 38 & 13 &
  \mc{2} & \mc{3} & \mc{5} & \mc{6} & \mc{1} & \mc{2} &
  \mc{4} & \mc{5} & \mc{7} & \mc{1} & \mc{3} & \mc{4} &
  \mc{0} &
 13880  & 470 & 220 & E D &   9&Iul \\
\nopagebreak
% page 90
\streep
  &    &    &
     &   &  4.&5  &    &   &  8.&9  &    &   & 12.&13 &
     &   & 16.&17 &    &   & 20.&21 &    &   & 24.&25 &
     &   &
  \\
\nopagebreak
\da & 39 & 14 &
  \mc{6} & \mc{1} & \mc{2} & \mc{4} & \mc{5} & \mc{7} &
  \mc{1} & \mc{3} & \mc{4} & \mc{6} & \mc{7} & \mc{2} &
  \mc{3} &
 14264  & 483 & 226 & C & 28&Iun \\
\nopagebreak
%
\streep
  &    &    &
  28.&29 &    &   &    &   &  2.&3  &    &   &  6.&7&
% '7' not visible in the scan we use. Is visible in other scans and editions
     &   & 10.&11 &    &   & 14.&15 &    &   & 18.&19 &
     &   &
  \\
\nopagebreak
  & 40 & 15 &
  \mc{5} & \mc{6} & \mc{1} & \mc{3} & \mc{4} & \mc{6} &
  \mc{7} & \mc{2} & \mc{3} & \mc{5} & \mc{6} & \mc{1} &
  \mc{0} &
 14618  & 495 & 231 & B & 17&Iul \\
\nopagebreak
%
\streep
  &    &    &
     &   & 22.&23 &    &   & 26.&27 &    &   & 30.&1 &
     &   &    &   &  4.&5  &    &   &  8.&9  &    &   &
  12.&13 &
  \\
\nopagebreak
\da & 41 & 16 &
  \mc{2} & \mc{4} & \mc{5} & \mc{7} & \mc{1} & \mc{3} &
  \mc{4} & \mc{6} & \mc{1} & \mc{2} & \mc{4} & \mc{5} &
  \mc{7} &
 15002  & 508 & 238 & A &  6&Iul \\
\nopagebreak
%
\streep
  &    &    &
     &   & 16.&17 &    &   & 20.&21 &    &   & 24.&25 &
     &   & 27.&28 &    &   &    &   &  1.&2  &    &   &
     &   &
  \\
\nopagebreak
  & 42 & 17 &
  \mc{1} & \mc{3} & \mc{4} & \mc{6} & \mc{7} & \mc{2} &
  \mc{3} & \mc{5} & \mc{6} & \mc{1} & \mc{3} & \mc{4} &
  \mc{0} &
 15357  & 520 & 243 & G F & 24&Iul \\
\nopagebreak
%
\streep
  &    &    &
   5.&6  &    &   &  9.&10 &    &   & 13.&14 &    &   &
  17.&18 &    &   & 21.&22 &    &   & 25.&26 &    &   &
     &   &
  \\
\nopagebreak
  & 43 & 18 &
  \mc{6} & \mc{7} & \mc{2} & \mc{3} & \mc{5} & \mc{6} &
  \mc{1} & \mc{2} & \mc{4} & \mc{5} & \mc{7} & \mc{1} &
  \mc{0} &
 15711  & 532 & 249 & E &  14&Iul \\
\nopagebreak
%
\streep
  &    &    &
  29.&30 &    &   &    &   &  3.&4  &    &   &  7.&8  &
     &   & 11.&12 &    &   & 15.&16 &    &   & 19.&20 &
     &   &
  \\
\nopagebreak
\da & 44 & 19 &
  \mc{3} & \mc{4} & \mc{6} & \mc{1} & \mc{2} & \mc{4} &
  \mc{5} & \mc{7} & \mc{1} & \mc{3} & \mc{4} & \mc{6} &
  \mc{7} &
 16095  & 545 & 255 & D &  3&Iul \\
\nopagebreak
%
\streep
  &    &    &
  23.&24 &    &   & 27.&28 &    &   &    &   &  1.&2  &
     &   &  5.&6  &    &   &  9.&10 &    &   & 13.&14 &
     &   &
  \\
\nopagebreak
  & 45 &  1 &
  \mc{2} & \mc{3} & \mc{5} & \mc{6} & \mc{1} & \mc{3} &
  \mc{4} & \mc{6} & \mc{7} & \mc{2} & \mc{3} & \mc{5} &
  \mc{0} &
 16449  & 557 & 261 & C & 22&Iul \\
\nopagebreak
%
\streep
  &    &    &
     &   & 17.&18 &    &   & 21.&22 &    &   & 24.&25 &
% ".&18" more clear in 1598 edition
     &   & 28.&29 &    &   &    &   &  2.&3  &    &   &
     &   &
  \\
\nopagebreak
  & 46 &  2 &
  \mc{6} & \mc{1} & \mc{2} & \mc{4} & \mc{5} & \mc{7} &
  \mc{1} & \mc{3} & \mc{4} & \mc{6} & \mc{1} & \mc{2} &
  \mc{0} &
 16804  & 569 & 266 & B A & 10&Iul \\
\nopagebreak
%
\streep
  &    &    &
   6.&7  &    &   & 10.&11 &    &   & 14.&15 &    &   &
  18.&19 &    &   & 22.&23 &    &   & 26.&27 &    &   &
     &   &
  \\
\nopagebreak
\da & 47 &  3 &
  \mc{4} & \mc{5} & \mc{7} & \mc{1} & \mc{3} & \mc{4} &
  \mc{6} & \mc{7} & \mc{2} & \mc{3} & \mc{5} & \mc{6} &
  \mc{1} &
 17188  & 582 & 272 & G &  30&Iun \\
\nopagebreak
%
\streep
  &    &    &
   3.&4  &    &   &  4.&5  &    &   &  8.&9  &    &   &
  12.&13 &    &   & 16.&17 &    &   & 20.&21 &    &   &
     &   &
  \\
\nopagebreak
  & 48 &  4 &
  \mc{3} & \mc{4} & \mc{6} & \mc{7} & \mc{1} & \mc{3} &
  \mc{5} & \mc{6} & \mc{1} & \mc{2} & \mc{4} & \mc{5} &
  \mc{0} &
 17542  & 594 & 278 & F &  19&Iul \\
\nopagebreak
%
\streep
  &    &    &
  24.&25 &    &   & 28.&29 &    &   &    &   &  2.&3  &
     &   &  6.&7  &    &   & 10.&11 &    &   & 14.&15 &
     &   &
  \\
\nopagebreak
\da & 49 &  5 &
  \mc{7} & \mc{1} & \mc{3} & \mc{4} & \mc{6} & \mc{1} &
  \mc{2} & \mc{4} & \mc{5} & \mc{7} & \mc{1} & \mc{3} &
  \mc{4} &
 17926  & 607 & 284 & E &  8&Iul \\
\nopagebreak
%
\streep
  &    &    &
  18.&19 &    &   & 21.&22 &    &   & 25.&26 &    &   &
  29.&30 &    &   &    &   &  3.&4  &    &   &  7.&8  &
     &   &
  \\
\nopagebreak
  & 50 &  6 &
  \mc{6} & \mc{7} & \mc{2} & \mc{3} & \mc{5} & \mc{6} &
  \mc{1} & \mc{2} & \mc{4} & \mc{6} & \mc{7} & \mc{2} &
  \mc{0} &
 18280  & 619 & 290 & D C &  26&Iul \\
\nopagebreak
%
\streep
  &    &    &
     &   & 11.&12 &    &   & 15.&16 &    &   & 19.&20 &
     &   & 23.&24 &    &   & 27.&28 &    &   &    &   &
     &   &
  \\
\nopagebreak
  & 51 &  7 &
  \mc{3} & \mc{5} & \mc{6} & \mc{1} & \mc{2} & \mc{4} &
  \mc{5} & \mc{7} & \mc{1} & \mc{3} & \mc{4} & \mc{6} &
  \mc{0} &
 18635  & 631 & 295 & B & 15&Iul \\
\nopagebreak
%
\streep
  &    &    &
   1.&2  &    &   &  5.&6  &    &   &  9.&10 &    &   &
  13.&14 &    &   & 17.&18 &    &   & 21.&22 &    &   &
  25.&26 &
  \\
\nopagebreak
\da & 52 &  8 &
  \mc{1} & \mc{2} & \mc{4} & \mc{5} & \mc{7} & \mc{1} &
  \mc{3} & \mc{4} & \mc{6} & \mc{7} & \mc{2} & \mc{3} &
  \mc{5} &
 19018  & 644 & 302 & A &   5&Iul \\
% '644' clearer in 1598 edition
\nopagebreak
%
\streep
  &    &   &
     &   & 29.&30 &    &   &    &   &  3.&4  &    &   &
   7.&8  &    &   & 11.&12 &    &   & 15.&16 &    &   &
     &   &
  \\
\nopagebreak
  & 53 &  9 &
  \mc{6} & \mc{1} & \mc{2} & \mc{4} & \mc{6} & \mc{7} &
  \mc{2} & \mc{3} & \mc{4} & \mc{6} & \mc{1} & \mc{2} &
  \mc{0} &
 19373  & 656 & 307 & G &  23&Iul \\
\nopagebreak
%
\streep
  &    &    &
  18.&19 &    &   & 22.&23 &    &   & 26.&27 &    &   &
  30.&1  &    &   &    &   &  4.&5  &    &   &  8.&9  &
     &   &
  \\
\nopagebreak
  & 54 & 10 &
  \mc{4} & \mc{5} & \mc{7} & \mc{1} & \mc{3} & \mc{4} &
  \mc{6} & \mc{7} & \mc{2} & \mc{4} & \mc{5} & \mc{7} &
  \mc{0} &
 19727  & 668 & 313 & F E &  12&Iul \\
\nopagebreak
%
\streep
  &    &    &
     &   & 12.&13 &    &   & 12.&13 &    &   & 20.&21 &
     &   & 24.&25 &    &   & 28.&29 &    &   &    &   &
   2.&3  &
  \\
\nopagebreak
\da & 55 & 11 &
  \mc{1} & \mc{3} & \mc{4} & \mc{6} & \mc{7} & \mc{2} &
  \mc{3} & \mc{5} & \mc{6} & \mc{1} & \mc{2} & \mc{4} &
  \mc{6} &
 20111  & 681 & 319 & D & Ka.&Iul \\
\nopagebreak
%
\streep
  &    &    &
     &   &  6.&7  &    &   & 10.&11 &    &   & 14.&15 &
     &   & 18.&19 &    &   & 22.&23 &    &   & 26.&27 &
     &   &
  \\
\nopagebreak
  & 56 & 12 &
  \mc{7} & \mc{2} & \mc{3} & \mc{5} & \mc{6} & \mc{1} &
  \mc{2} & \mc{4} & \mc{5} & \mc{7} & \mc{1} & \mc{3} &
  \mc{0} &
 20465  & 693 & 325 & C &  20&Iul \\
\nopagebreak
%
\streep
  &    &   &
     &   & 30.&1  &    &   &    &   &  4.&5  &    &   &
   8.&9  &    &   & 12.&13 &    &   & 15.&16 &    &   &
     &   &
  \\
\nopagebreak
  & 57 & 13 &
  \mc{4} & \mc{6} & \mc{7} & \mc{2} & \mc{4} & \mc{5} &
  \mc{7} & \mc{1} & \mc{3} & \mc{4} & \mc{6} & \mc{7} &
  \mc{0} &
 20820  & 705 & 330 & B &   9&Iul \\
\nopagebreak
%
\streep
  &    &    &
  19.&20 &    &   & 23.&24 &    &   & 27.&28 &    &   &
     &   &  1.&2  &    &   &  5.&6  &    &   &  9.&10 &
     &   &
  \\
\nopagebreak
\da & 58 & 14 &
  \mc{2} & \mc{3} & \mc{5} & \mc{6} & \mc{1} & \mc{2} &
  \mc{4} & \mc{6} & \mc{7} & \mc{2} & \mc{3} & \mc{5} &
  \mc{6} &
 21204  & 718 & 336 & A G &  28&Iun \\
\nopagebreak
%
\streep
  &    &    &
  13.&14 &    &   & 17.&18 &    &   & 21.&22 &    &   &
  25.&26 &    &   & 29.&30 &    &   &    &   &  3.&4  &
     &   &
  \\
\nopagebreak
  & 59 & 15 &
  \mc{1} & \mc{2} & \mc{4} & \mc{5} & \mc{7} & \mc{1} &
  \mc{3} & \mc{4} & \mc{6} & \mc{7} & \mc{2} & \mc{4} &
  \mc{0} &
 21558  & 730 & 342 & F &  17&Iul \\
\nopagebreak
%
\streep
  &    &    &
     &   &  7.&8  &    &   & 11.&12 &    &   & 15.&16 &
     &   & 19.&20 &    &   & 23.&24 &    &   & 27.&28 &
     &   &
  \\
\nopagebreak
\da & 60 & 16 &
  \mc{5} & \mc{7} & \mc{1} & \mc{3} & \mc{4} & \mc{6} &
  \mc{7} & \mc{2} & \mc{3} & \mc{5} & \mc{6} & \mc{1} &
  \mc{2} &
 21942  & 743 & 348 & E &   6&Iul \\
\nopagebreak
%
\streep
  &    &    &
     &   &  1.&2  &    &   &  5.&6  &    &   &  9.&10 &
     &   & 13.&14 &    &   & 17.&18 &    &   & 21.&22 &
     &   &
  \\
\nopagebreak
  & 61 & 17 &
  \mc{4} & \mc{6} & \mc{7} & \mc{2} & \mc{3} & \mc{5} &
  \mc{6} & \mc{1} & \mc{2} & \mc{4} & \mc{5} & \mc{7} &
  \mc{0} &
 22296  & 755 & 354 & D &  25&Iul \\
\nopagebreak
%
\streep
  &    &    &
     &   & 24.&25 &    &   & 28.&29 &    &   &    &   &
   2.&3  &    &   &  6.&7  &    &   & 10.&11 &    &   &
     &   &
  \\
\nopagebreak
  & 62 & 18 &
  \mc{1} & \mc{3} & \mc{4} & \mc{6} & \mc{7} & \mc{2} &
  \mc{4} & \mc{5} & \mc{7} & \mc{1} & \mc{3} & \mc{4} &
  \mc{0} &
 22631  & 767 & 359 & C B &  13&Iul \\
\nopagebreak
%
\streep
  &    &    &
  14.&15 &    &   & 18.&19 &    &   & 22.&23 &    &   &
  26.&27 &    &   & 30.&1  &    &   &    &   &  4.&5  &
     &   &
  \\
\nopagebreak
\da & 63 & 19 &
  \mc{6} & \mc{7} & \mc{2} & \mc{3} & \mc{5} & \mc{6} &
  \mc{1} & \mc{2} & \mc{4} & \mc{5} & \mc{7} & \mc{2} &
  \mc{3} &
 23035  & 780 & 365 & A &   3&Iul \\
% '365' unclear; better in other scans
\nopagebreak
%
\streep
  &    &    &
   8.&9  &    &   & 12.&13 &    &   & 16.&17 &    &   &
  20.&21 &    &   & 24.&25 &    &   & 28.&29 &    &   &
     &   &
  \\
\nopagebreak
  & 64 &  1 &
  \mc{5} & \mc{6} & \mc{1} & \mc{2} & \mc{4} & \mc{5} &
  \mc{7} & \mc{1} & \mc{3} & \mc{4} & \mc{6} & \mc{7} &
  \mc{0} &
 23389  & 792 & 371 & G &  22&Iul \\
\nopagebreak
%
\streep
  &    &    &
     &   &  2.&3  &    &   &  6.&7  &    &   &  9.&10 &
     &   & 13.&14 &    &   & 17.&18 &    &   & 21.&22 &
     &   &
  \\
\nopagebreak
  & 65 &  2 &
  \mc{2} & \mc{4} & \mc{5} & \mc{7} & \mc{1} & \mc{3} &
  \mc{4} & \mc{6} & \mc{7} & \mc{2} & \mc{3} & \mc{5} &
  \mc{0} &
 23734  & 804 & 377 & F &  11&Iul \\
\nopagebreak
%
\streep
  &    &    &
     &   & 25.&26 &    &   & 29.&30 &    &   &  3.&4  &
     &   &  7.&6  &    &   & 11.&12 &    &   & 15.&16 &
     &   &
  \\
\nopagebreak
\da & 66 &  3 &
  \mc{6} & \mc{1} & \mc{2} & \mc{4} & \mc{5} & \mc{7} &
  \mc{1} & \mc{3} & \mc{4} & \mc{6} & \mc{7} & \mc{2} &
  \mc{3} &
 24127  & 817 & 383 & E D &  29&Iun \\
\nopagebreak
%
\streep
  &    &    &
     &   & 19.&20 &    &   & 23.&24 &    &   & 27.&28 &
     &   &    &   &  1.&2  &    &   &  5.&6  &    &   &
     &   &
  \\
\nopagebreak
  & 67 &  4 &
  \mc{5} & \mc{7} & \mc{1} & \mc{3} & \mc{4} & \mc{6} &
  \mc{7} & \mc{2} & \mc{4} & \mc{5} & \mc{7} & \mc{1} &
  \mc{0} &
 24482  & 829 & 388 & C &  18&Iul \\
\nopagebreak
%
\streep
  &    &    &
   9.&10 &    &   & 13.&14 &    &   & 17.&18 &    &   &
  21.&22 &    &   & 25.&26 &    &   & 29.&30 &    &   &
     &   &
  \\
\nopagebreak
\da & 68 &  5 &
  \mc{3} & \mc{4} & \mc{6} & \mc{7} & \mc{2} & \mc{3} &
  \mc{5} & \mc{6} & \mc{1} & \mc{2} & \mc{4} & \mc{5} &
  \mc{7} &
 24866  & 842 & 394 & B &   8&Iul \\
\nopagebreak
%
\streep
  &    &    &
   3.&4  &    &   &  6.&7  &    &   & 10.&11 &    &   &
  14.&15 &    &   & 18.&19 &    &   & 22.&23 &    &   &
     &   &
  \\
\nopagebreak
  & 69 &  6 &
  \mc{2} & \mc{3} & \mc{5} & \mc{6} & \mc{1} & \mc{2} &
  \mc{4} & \mc{5} & \mc{7} & \mc{1} & \mc{3} & \mc{4} &
  \mc{0} &
 25220  & 854 & 400 & A &  27&Iul \\
% '25220' clearer in other editions and scans
\nopagebreak
%
\streep
  &    &    &
  26.&27 &    &   & 30.&1  &    &   &    &   &  4.&3\footnote{4.3: Sic.}  &
% '4.3' in the original is odd, as in all other entries
% the second number is 1 larger than the first number.
% This is nevertheless the same in other editions, so we assume this is what
% the author wrote, though we suspect it is supposed to be '4.5'.
     &   &  8.&9  &    &   & 12.&13 &    &   & 16.&17 &
     &   &
  \\
\nopagebreak
  & 70 &  7 &
  \mc{6} & \mc{7} & \mc{2} & \mc{3} & \mc{5} & \mc{7} &
  \mc{1} & \mc{3} & \mc{4} & \mc{6} & \mc{7} & \mc{2} &
  \mc{0} &
 25574  & 866 & 406 & G F & 15&Iul \\
\nopagebreak
%
\streep
  &    &    &
     &   & 20.&21 &    &   & 24.&25 &    &   & 28.&29 &
     &   &    &   &  2.&3  &    &   &  6.&7  &    &   &
  10.&11 &
  \\
\nopagebreak
\da & 71 &  8 &
  \mc{3} & \mc{5} & \mc{6} & \mc{1} & \mc{2} & \mc{4} &
  \mc{5} & \mc{7} & \mc{2} & \mc{3} & \mc{5} & \mc{6} &
  \mc{1} &
 25958  & 879 & 412 & E &   4&Iul \\
\nopagebreak
%
\streep
  &    &    &
     &   & 14.&15 &    &   & 18.&19 &    &   & 22.&23 &
     &   & 26.&27 &    &   & 30.&1  &    &   &    &   &
     &   &
  \\
\nopagebreak
  & 72 &  9 &
  \mc{2} & \mc{4} & \mc{5} & \mc{7} & \mc{1} & \mc{3} &
  \mc{4} & \mc{6} & \mc{7} & \mc{2} & \mc{3} & \mc{5} &
  \mc{0} &
 26313  & 891 & 417 & D &  23&Iul \\
\nopagebreak
%
\streep
  &    &    &
   3.&4  &    &   &  7.&8  &    &   & 11.&12 &    &   &
  15.&16 &    &   & 19.&20 &    &   & 23.&24 &    &   &
     &   &
  \\
\nopagebreak
  & 73 & 10 &
  \mc{7} & \mc{1} & \mc{3} & \mc{4} & \mc{6} & \mc{7} &
  \mc{2} & \mc{3} & \mc{5} & \mc{6} & \mc{1} & \mc{2} &
  \mc{0} &
 26667  & 903 & 423 & C &  13&Iul \\
\nopagebreak
%
\streep
  &    &    &
  27.&28 &    &   &    &   &  1.&2  &    &   &  5.&6  &
     &   &  9.&10 &    &   & 13.&14 &    &   & 17.&18 &
     &   &
  \\
\nopagebreak
\da & 74 & 11 &
  \mc{4} & \mc{5} & \mc{7} & \mc{2} & \mc{3} & \mc{5} &
  \mc{6} & \mc{1} & \mc{2} & \mc{4} & \mc{5} & \mc{7} &
  \mc{1} &
 27051  & 916 & 429 & B A &  Ka.&Iul \\
\nopagebreak
%
\streep
  &    &    &
  21.&22 &    &   & 25.&26 &    &   & 29.&30 &    &   &
     &   &  3.&4  &    &   &  7.&8  &    &   & 11.&12 &
     &   &
  \\
\nopagebreak
  & 75 & 12 &
  \mc{3} & \mc{4} & \mc{6} & \mc{7} & \mc{2} & \mc{3} &
  \mc{5} & \mc{7} & \mc{1} & \mc{3} & \mc{4} & \mc{6} &
  \mc{0} &
 27405  & 928 & 435 & G &  20&Iul \\
\nopagebreak
%
\streep
  &    &    &
     &   & 15.&16 &    &   & 19.&20 &    &   & 23.&24 &
     &   & 27.&28 &    &   & 30.&1  &    &   & 30.&1  &
     &   &
  \\
\nopagebreak
  & 76 & 13 &
  \mc{7} & \mc{2} & \mc{3} & \mc{5} & \mc{6} & \mc{1} &
  \mc{2} & \mc{4} & \mc{5} & \mc{7} & \mc{1} & \mc{3} &
  \mc{0} &
 27759  & 940 & 441 & F &   9&Iul \\
%%
\end{longtable}
\endgroup


% 90
% {PDF page nr}{source page nr}{line nr}
%\plnr{173}{90}{1}

% 91
\subsection{De Periodo Lunari Calippica ab Autumno}
% {PDF page nr}{source page nr}{line nr}
\plnr{174}{91}{1}Calippum periodum suam in gratiam Alexandri instituisse,
satis libro priore demonstratum.
\lnr{2}Exorsus enim fuit a neomenia
\textgreek{πυδυεψιῶνος[?]}, quae indicit in \rnum{vi} Octobris.
\lnr{3}Nam eo anno
Hecatombaeon Calippicus erat in \rnum{ix}, feria \rnum{vi}.
\lnr{4}Est enim ultimus
annus Periodi Calippicae aestivae, siquidem fingamus illam periodum
multis saeculis ante obtinuisse.
\lnr{6}In Tabula Neomeniarum Hecatombaeonis,
% Table
\begin{table}[htbp]
%\tiny
%\scriptsize
%\footnotesize
%\small
%\normalsize
\centering
%\setlength{\tabcolsep}{3pt}
%\renewcommand{\arraystretch}{1.3}
\input{tables/091_menses_macedonici_attici.tex}
\end{table}
%%
annus ultimus periodi incipit in 9 Iulii, cyclo Lunae 13.
\lnr{8}Atqui ea est vera epocha Olympica, ut alibi disputatum est.
\lnr{8}Hoc
movit Calippum, ut totam periodum Olympicam retexeret.
\lnr{9}Et
praeterea Boedromion Calippicus caepit sexta Septembris, feria secunda:
in cuius medio defecit Luna, eadem feria: et postea copiae
Persarum ab Alexandro victae.
\lnr{12}Incipit igitur illa periodus Lunaris
a Pyanepsione, \rnum{vi} Octobris, quae erat
 \textgreek{ἔνη καὶ νέα[?]} Boedromionis Tetraeterici
novi, sive Calippici, quinta autem veteris Boedromionis
Attici Tetraeterici, et vicesima Apellaei periodi Mecedonieae antiquae.
\lnr{16}In qua nihil puto innovatum praeter menses Macedonum, cum
pro Audynaeo autumnali Hyperberetaeus antea solstitialis in eius
locum fuccessit: quod in honorem Alexandri factum.
\lnr{18}Praeterea neque
locus mensis intercalaris mutatus, cum is mensis Macedomicus,
qui ab Hyperberetaeo putatus respondet Scirrhophorioni Attico,
fuerit sedes embolismi, ut in periodo Attica.
\lnr{21}Anno 54 secundae periodi
fuit Eclipsis Lunae, Nabonassari 547, Mesori 16, feria \rnum{vi}, Sole
in Virgine sito.
\lnr{23}Tempus 22 Septermbris, cyclo Lunae 10, cyclo
Solis 3.
\lnr{24}Anno 55 periodi secundae, Nabonassari 548, Mechir 9,
feria 2, iterum Luna defecit, Sole in Piscibus collocato.
\lnr{25}Tempus, 19
Martii, cyclo Lunae \rnum{xi}, Solis \rnum{vi}.
\lnr{27}Denique eodem anno 55, eodem
etiam Nabonassari, Mesori 5, feria
3[?], Luna tertium defecit, Sole \rnum{xv}
Virginis obtinente.
\lnr{30}Tempus \rnum{xi}
Septembris.
\lnr{31}Videmus hic a Virgine
ad Pisces mutationem anni factam,
a Piscibus ad Virginem non
factam.
\lnr{34}Ergo annus caepit ab Autumno:
et prionde periodi annus
primus a cyclo Lunae \rnum{xiii}, non a
\rnum{xiiii}, ut annus primus periodi Atticae.
\lnr{38}Sed quare Prolemaeus menses
subticuit, qui alibi tam curiose illos citare solet?
\lnr{39}Sane vera et indubitata
caussa est, quod metuit, ne si menses Macedonici, ita ut a Calippo
transpositi fuerant, citerentur, hoc confusionem lectori pareret.

% 92
% {PDF page nr}{source page nr}{line nr}
\plnr{175}{92}{2}Quod
sane evitari non poterat.
\lnr{3}Sed nos reddamus his Eclipsibus menses suos.
\lnr{4}Anno 54 periodi Calippi Atticae, Hecatombaeon fuit 12 Iulii, in litera
D.
\lnr{5}Fuit ergo feria \rnum{iv}, cyclo Solis \rnum{v}.
\lnr{5}Quae cum characterismo Boedromionis
composita dat feriam \rnum{vii} neomeniae Boedromionis,
 die \rnum{ix} Septembris.
\lnr{7}Incidit ergo Eclipsis in \rnum{xiiii} Boedromionis Atheniensibus,
Macedonibus autem novi Gorpiaei.
\lnr{8}Rursus quarta feria cum quinta characterismi
% cū -> cum
Elaphebolionis composita, abiecto septenario, dat feriam
secundam Elaphebolionis, vel novi Dystri Macedonici in \rnum{v} Martii.
\lnr{10}Igitur
Neomenia Elaphebolionis, aut Dystri fuit in \rnum{v} Martii, et proinde
Eclipsis \rnum{xv} Elaphebolionis, aut Dystri in anno 54 Attico, 55
Macedonico.
\lnr{13}Denique Hecatombaeon 55 anni Attici et Macedonici Kalendae
% Deniq; => Denique
% Hecatōbaeon => Hecatombaeon
Iulii, feria prima: quae cum 3 charactere Boedromionis componit
feriam 4 neomeniae Boedromionis Attici, Gorpiaei autem novi in
29 Augusti.
\lnr{16}Ergo Eclipsis 14 Boedromionis, aut Gorpiaei.
\lnr{16}Rursus anno
37 tertiae periodi defecit Luna 607 Nabonassari, Tybi die 2, sequente
3, feria 3, anno periodi Iuliannae 4573.
\lnr{18}Tempus 27 Ianuarii.
\lnr{18}Initium
illius anni cyclo Solis 8 Neomenia Hecatombaeonis 20 Iulii, feria
% Hecatombęonis => Hecatombaeonis
secunda.
\lnr{20}Quae composita cum 2 charactere Gamelionis, dat feriam quartam
% cū => cum
neomeniae Gamelionis in 14 Ianuarii, cum litera Dominicalis esset
D~C.\ 
\lnr{22}Ergo Eclipsis incidit in 14 Gamelionis vel Audynaei novi.
\lnr{24}Quod autem notavimus de mensibus Tetraetericis apud Macedonas
observatis, id est, quod Syromecedones duplici anno uterentur, mensium
\textgreek{τριακονθημέρων[?]}
 et Lunarium, ut solebant Attici, praeter alia, quae firmandae
opinionis adducere poteramus, fidem fecerit locuples testis Iosephus,
qui ita scribit
 \textgreek{ὑπερβερεταίου τῇ δεκάτῃ κατὰ σελήνην νηστείουσι[?]}.
\lnr{27}Dixit Hyperberetaeum
\textgreek{κατὰ σελήνην[?]}, quia erat
 \textgreek{ὑπερβερεταῖος τετραετηρικὸς τριακονθήμερος
μη κατὰ σελήνην[?]}.
\lnr{29}Ita etiam Thucydides dixit \textgreek{νουμηνίαν κατὰ σελήνην[?]},
quia scilicet alia erat \textgreek{νουμηνία μη κατὰ σελήνην[?]},
 utpote tetraeterica mensis
\textgreek{τριακονθημέρου[?]}.
\lnr{31}Ex quibus etiam verbis Iosephi manifestum est ipsus Iudaeis
mensium Macedonicorum appellationem in usu fuisse in contractibus
scilicet, in quibus utebantur aera Macedonica.

\subsection{De Periodo Syromacedonum Alexandrea}
\lnr{34}Bene quidem Calippus, qui excessum unius diei in 76 annis Metonicis
primus deprehendit, indicavit, et emendavit.
% emēndavit => emendavit
\lnr{35}Sed ad oeconomiam
cavorum mensium instituendam, tantisper sustinere, donec 64
% cauorū => cavorum
% instituends̄ => instituendam
dies tranfigantur, hoc vero est vim ipsi motui sideris facere.
\lnr{37}Nam hoc
pacto duo menses pleni statim in capite anni continuantur: cum tamen
ab initio cuiuscunque periodi Lunaris duo menses primi minores sint,
quam sexaginta dies, scrupulis diurnis 56 fere, qui proxime absunt a
magnitudine iusti diei.

% 93
% {PDF page nr}{source page nr}{line nr}
\plnr{176}{93}{2}Atqui alternatio plenorum et cavorum mensium
magis et motui sideris, et usui politico, et captui plebis accomodata
est: quamquam non semper continuari potest illa alternatio, cum
longe plures sint pleni menses in periodo ista, quam cavi.
\lnr{5}Nam 499 sunt
pleni, 441 autem cavi.
\lnr{6}Ad quos cavos menses dispensandos non erunt
expectandi dies 64, ut in periodo Calippica, et Metonica; quandoquidem
hoc alienum est a motu sideris, et usu politico: sed alia potius
via tendanda.
\lnr{9}In Periodo Calippica sunt dies 27759, syzygiae 940:
quas si in dies periodi distribueris, erit intervallum unius syzygiae, dies
29, \myfrac{492}{940}[?] unius diei.
\lnr{11}Si igitur primum mensem anni plenum, uti ratio
postulat, instituamus, secundum autem cavum, alternatio aequabiliter
procedet, donec ex excessu scrupulorum 940 supra diem unum,
componatur dies solidus, qui faciat annum communem
 \textgreek{ὑπερήμερον[?]},
hoc est dierum 355, embolimaeum autem dierum 385, ita tamen, ut
tametsi ille excessus unius diei incidat in medium annum aut mensem,
reiiciatur in calcem tam mensis, quam anni.
\lnr{17} His ita animadversis
repetenda sunt, quae libro priore diximus de anno populari Graecorum.
\lnr{19}Annus popularis omnium Graecorum constabat mensibus omnibus
plenis, cum appendice bidui, quas
 \textgreek{ἀνάρχους ἡμέρας[?]} vocabant.
\lnr{20}Et cum
primi Hecatombaeonis neomenia incideret semper in novilunium,
neomenia item quinti conveniebat rursus in novilunium.
\lnr{22}Hoc spatium Tetraeterida vocabant.
\lnr{23}Itaque quoties de illo populari anno
Graecorum loquemur, vocabimus annum aequabilem, vel menses
Tetraetericos.
\lnr{25}Erat autem et alius annus mere Lunaris, quem
 \textgreek{πρυτανείας[?]}
vocabant, ab auctoribus primum Octaeteridum institutus, deinde
ab Enneadecaeteridibus Metonicis, Calippicis, et Hipparchicis reformatus.
\lnr{28}Primus omnium Calippus vidit in 76 annis aequabilibus, 940
syzygias comprehendi praecise, ut putavit ideoque periodum totidem annorum
% periodū => periodum
instituit, Atticam quidem a Solstitio, Alexandream vero ab Autumno.
\lnr{31}Iam alibi diximus Hyperberetaeum aequabilem traductum a
Macedonibus a \rnum{ix} Iulii, ad \rnum{vii} Octobris:
 hoc est ex Audynaeo factum
Hyperberetaeum.
\lnr{33}Graeci enim in Macedonia hanc periodum renovarunt,
quam annis \rnum{xix} solidis antea instituerat Calippus.
\lnr{34}Itaque
cum ipsorum annus inciperet a mense Olympiaco, cuius citima epocha
erat in \rnum{ix} Iulii, ibi statuerunt neomeniam Hyperberetaei periodici
in coitu luminarium neomeniae.
% Period not in original
Ab Iudaici 3449, cuius character 2.5.812,
feria secunda, \rnum{ix} Iulii, cyclo Solis \rnum{vi}, Lunae \rnum{xiii}.
\lnr{38}Convenerunt
igitur ambae neomeniae tam aequabiles, quam Lunares mensis Hyperberetaei,
idque \rnum{xix} annis vertentibus post cladem ad Arbela.
\lnr{40}Itaque
visum Syromacedonibus Audynaeum in Hyperberetaeum mutare, atque
hinc periodum aequabilem auspicari, quam Graeci Macedones in Macedonia
% Gręci => Graeci
tribus mensibus antea incipiebant.

% 94
% {PDF page nr}{source page nr}{line nr}
\plnr{177}{94}{2}Fuit enim mere Olympica,
ut toties diximus.
\lnr{3}Quare Hyperberetaeus aequabilis primus periodi incidit
in \rnum{viii} Octobris.































% ==== End of text of Liber Secundus ===

% !TEX root = ../de-emendatione-temporum-1629.tex
% !TEX TS-program = xelatex
% !TEX encoding = UTF-8 Unicode
% this template is specifically designed to be typeset with XeLaTeX;
% it will not work with other engines, such as pdfLaTeX

%%% Count out columns for fixed-width source font
% 000000011111111112222222222333333333344444444445555555555666666666677777777778
% 345678901234567890123456789012345678901234567890123456789012345678901234567890

\setheaders{\shorttitle{} Liber III}{\shortauthor{}}
\chapter{De Anno Solari, Tributus in Partes Quatuor}
%
% 188
% {PDF page nr}{source page nr}{line nr}
\pars{Pars Prima} % Add entry to Table of Contents
\plnr{271}{188}{1}In Astronomicis \textgreek{[Greek]} vocantur
eae, quibus ex Anomaliarum Canonibus
competentes \textgreek{[Greek]} adhibitae non
sunt.
\lnr{4}Ideo Illis motibus aut deest semper,
aut superest aliquid.
\lnr{5}Sic in ratione anni
eum annum aequabilem vocabus, cui propter
commodiorem usum aut abest, aut superest aliquid.
\lnr{8}Exempli gratia: Persico anno
quadrans de Solis ratiociniis deest.
\lnr{9}





































% ==== End of text of Liber Secundus ===

\include{./tex/liber-quartus}
\include{./tex/liber-quintus}
\include{./tex/liber-sextus}
\include{./tex/liber-septimus}

\appendix
\include{./tex/nomenclator}
\include{./tex/index-rerum-et-verborum}
\include{./tex/index-graecarum-vocum}
\include{./tex/index-orientalium-vocum}
\include{./tex/errata}

\backmatter
\include{./tex/operis-finis}
\include{./tex/fragmenta-selecta}

\end{document}
