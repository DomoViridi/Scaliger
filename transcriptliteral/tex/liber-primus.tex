% !TEX TS-program = xelatex
% !TEX encoding = UTF-8 Unicode
% this template is specifically designed to be typeset with XeLaTeX;
% it will not work with other engines, such as pdfLaTeX

%%% Count out columns for fixed-width source font
% 000000011111111112222222222333333333344444444445555555555666666666677777777778
% 345678901234567890123456789012345678901234567890123456789012345678901234567890

\chapter{}
\addcontentsline{toc}{chapter}{Primus Liber  - De anno aequabili minore}
\begin{center}
\begin{textsc}
\Large IOSEPHI\\
\Huge SCALIGERI\\
\Large IVLII CÆSARIS F.\\
\large DE\\
\Huge EMENDATIONE\\
\Large TEMPORUM\\
\large LIBER PRIMVS.\\
\end{textsc}
\em{Ad candidum Lectorem.}
\end{center}
\normalsize

\mletter{A} 
\setcounter{parcount}{0}
\begin{parnumbers}
\dropcapil{9}{S}{i vervm}
est, quod sciscit Stoicorum schola, Tempus esse normam rerum, \& custodiam, quia veritatis index atque examen est, \& rerum gestarum memoriam, ac diuturnitatem posteritati tuetur: ij non vulgari laude digni sunt, qui temporum rationes conscribere, atque fugitiuam antiquitatem retrahere conantur.
\\ \p
Qua in re cum tam priscis scriptoribus, quam æqualibus temporum nostrum opera egregie nauata sit, dolendum tamen, aut
\mletter{B}
ferius, quam oportebat, antiquos sese ad id studium contulisse, aut pauciora ea de re monumenta, quam ab ipsis
auctoribus relicta sunt, ad nos peruenisse.

Nam vt omnia extent veterum Græcorum scripta, ea tamen paucorum temporum interuallum complectebantur.

Græcis enim ante initia Olymiadum suarum nihil plane exploratum est: \&, quod dolendum est, de illorum scriptis, quæ ad Chronologiam spectabant, nihil nobis præter desiderium relictum est.

Nam quæ Eusebij exstant, quamuis è Græcorum monumentis hausta sunt, \& multa egregia ac cognitu digna nobis conseruarunt: tamen dissimulandum non est, multa in illis reperiri, quæ castigatioribus iudiciis non satisfaciant.

\mletter{C}
Quod si Thalli, Castoris, Phlegontis, Eratosthenis canones exstarent, perparua, aut nulla potius ratio haberetur librorum quorundam, qui hodi in penuria meliorum nobis in pretio sunt.

Apud Romanos vero, ea scriptio infeliciter cessit, quod eam cognitionem ferius amplexi sint.

Nam ante Consulatum Bruti nihil certi apud illos: omnia fabulosa: \&, si rem propius spectemus, ne ipsius quidem Bruti Consulatum, ac tempus Regifugij satis exploratum habent.

\end{parnumbers}
\clearpage
p. 2 [pdf 85]

\begin{parnumbers}

quamius, vt prodidit Censorinus, Varro collatis diuersarum ciuitatum temporibus, \& interualla retexens, verum in lucem protulerit, \& viam reperit, qua certus
\mletter{A}
annorum Vrbis conditæ numerus iniri posset.

Sed, vt suo loco disputabitur, non magis constabat Varroni de initiis Vrbis, quam Græcis de anno excidij Troiæ.

Nam ea demum est vera demonstratio, quæ cogit, non quæ persuadet.

Soli sacri libri supersunt, ex quorum fontibus certa temporum ratio hauriri possit.

Sed omnis temporum cognitio inutilis est, nisi certa epocha in illis deprehendatur, ad quam omnium temporum contextus, tam antecedentium, quam consequentium referri possit.

Nam, vt præclare dixit vetus inter Christianos scriptor Tatianus, apud quos temporum notatio non cohæret, apud illos neque veritatis \& fidei historicæ ratio vlla constare potest.

Quod si aliquis sacræ historiæ peritissimus, hoc est, qui interualla rerum gestarum
\mletter{B}
nobilissima certissimis ratiociniis ex Mose, \& reliquis sacris Bibliis explorata habeat, nihil tamen ex illis a certam epocham historiæ Græcæ, aut Romanæ referre possit: quodnam adiumentum is ex eiusmodi diligentia adferre potest aut sibi, aut studiosis rerum antiquarum?

Nam omnis cognitionis finis ad vsum aliquem spectat, quem si ex medio literarum sustuleris, ingratus est omnis labor \& opera, quæcunque in omne studium impenditur.

Eiusmodi est Iudæorum scientia, qui in ratiociniis quidem sacrorum temporum colligendis tantum studio \& diligentia consecuti sunt, vt proxime à veritate abesse dici possint: sed dum nullam aut saltem deprauatam rerum extrarum cognitionem tenent, multum errant, quod sine externa historia sacram tractare
\mletter{C}
aggrediuntur.

Venio ad nostros, recentiores dico, qui hodie summo cum fructu, sacræ, Græcæ, \& Romanæ historiæ tempora digesserunt.

Ij heroica virtute chronologiam negligentia \& contemtu maiorum intermortuam ac sepultam, è tenebris \& obliuionis silentio quotidie eruere conantur.

Certe meum semper iudicium fuit, eam rem maiore cum laude ab illis restitutam, quam ab antiquis proditam fuisse.

Nam non solum pleraque in ratione temporum pristinæ integritati reddiderunt, sed \& longe meliora effecerunt.

In multis tamen iudicium, in quibusdam etiam diligentiam requiro.

neq; enim dum verum adepti sunt.

Argumento suerint omnium, quotquot de his rebus tractarunt, dissensiones: vt inter tot millia Chronologorum vix inter duos de eadem re
\mletter{D}
conueniat.

Quanta adhuc contentione de Septimanis Danielis, de initio, medio, \& fine earum velitantur?

Tamen nihil plane eorum, quæ volunt, assecuti sunt.

Ab eorum lectione incertior atque indoctior sum, quam dudum.

Quis vnquam eorum veram epocham Exodi Habræorum; quis, quod pudendum est, verum annum natalis Dominici odoratus est?

Ecce trita, obuia, vulgaria, vt nobis videtur, ignoramus, \& remotiorum ac reconditiorum indicium promittimus!

\end{parnumbers}
\clearpage
p. 3 [pdf 86]

\begin{parnumbers}

Quis eorum Danielis \mletter{A} Hebdomadas interpretandas suscepit, qui inscitiæ suæ latebram non quæsiuerit, \& reges Persidis, qui nunquam in rerum natura fuerunt, non commentus sit?

Quod si Danielem accuratissime legissent, eis ad negotium explicandum non aliis regibus Persidis opus fuisset, quam iis, quos Herodotus, Diodorus, \& omnis Græcorum antiquitas nouit.

Sed quo non progressa est \textgreek{[Greek]}?

Berosos, Metasthenes, \& nescio quos Catones, ac Philones consulunt, qui ante hos centum annos ex officina nescio cuius indocti \& impudentis prodierunt.

Et sese Criticos in temporum notatione profitentur, quibus tam facili genere, tam pueriliter vnus homo otiosus in tanta luce literarū quotidie imoponit.

\mletter{B} Cuius hominis inscitiā si nihil aliud, certe illud arguere possit, quod Metasthenem pro Megasthene posuit. Si Iosephum Græce, aut Strabonem, aut Athenæum legisset, is Megasthenem vocari deprehendisset, quem Metasthenem vocat.

Si Græce scisset, numquam \textgreek{[Greek]} in illa lingua reperiri, neque hanc compositionem in eadem probari intellexisset.

Vt igitur ij resipiscant, qui \& nouos reges in Perside crearunt, \& Assueros Priscos, Assueros Longimanos, Assueros Pios, duos Cyros, \& nescio quæ alia somnia Annij Viterbiensis in medium producunt, primum vno verbo indicabo fontem erroris eorum: deinde qui medicina huic morbo fieri possit, docebo.

Quod igitur in veri inuestigatione \mletter{C} eos ratio fugerit, duas summas causas reperio: vnam, quod veterum tempora ciuilia, annorum, mensium formas, status, ac genera ignorarunt: alteram, quod characterem, \& notationem ei anno, quem sibi proposuerant, non adhibuerunt.

Ex vtraque quidem causa temporum confusio manauit, sed diuerso genere.

Ex priore causa ignoratus est annus, mensis \& dies multarum nobilium epocharum.

Huius enim rei cognitio pertinet ad tempus ciuile nationum.

Ex altera causa Palilia vrbis Romæ nunc tertio anno Olympiadis, nunc quarto attribuuntur.

Item Consulatus Bruti nunc in hunc, nunc in illum annum Olympiadis confertur.

Vt igitur nouam rationem emendationis temporum ineamus, duo illa præcipue nobis discutienda sunt: sed 
prius \mletter{D} de omnium nationum temporibus ciuilibus: quam assequi perdifficile est, nisi prius tempore in sua principia, hoc est ab annis, periodis, mensibus in vltimum terminum, dies, horas ac scrupula resoluto.

Nam qui ante nos hanc prouinciam aggressi sunt, si modo hanc nostram, non aliam aggressi sunt, ij satis de tempore, \& eius natura disputarunt.

Sed hanc disputationem melius interpres \textgreek{[Greek]} sibi vindicasset.

Neque vero nos id agimus, vt difiniamus tempus esse hoc secundum Peripateticos, aut illud seundum Stoicos, aut Academicos.

\end{parnumbers}
\clearpage
p. 4 [pdf 87]

\begin{parnumbers}

Qui istis definitionibus diu immorati sunt, \& hac sola scientia Chronologiæ scribendæ modum terminarunt, illi fatis \mletter{A} verborum quiedem, sed rerum nihil definiuerunt.

Nequid tamen \textgreek{[Greek]} transigatur, decreui singularum, vel minimarum temporis partium prius conspectum aliquem dare, quam ad descriptionem \textgreek{[Greek]} temporum ciuilium, \& eorum methodum aggrediar.

Incipiam igitur ab vltimo termino, a die scilicet, \& eius partibus, hoc est hora, \& scrupulis.

Ab hora igitur, si libet, principium esto.
\end{parnumbers}

\subsection[De Horis \& partibus diei reliquis.]{De Horis et partibvs diei reliqvis.}
\setcounter{parcount}{0}
\begin{parnumbers}

\dropcap{3}{V}{eteribus} statim ab initio has diei partes, quas H O R A S vocamus, in vsu non fuisse, argumento fuerint priscæ locutiones, \mletter{B} quibus dies non in partes secatur, sed actionibus quotidianis distiguitur: vt cum \textgreek{[Greek]} vesperam vocabant, nimirum, vt poëta inquit, Demeret emeritis cum iuga Phœbus equis.

Item quod tempus antemeridianum disignantes dicebant \textgreek{[Greek]} vel \textgreek{[Greek]}, conuenientibus scilicet eo tempore in Comitium viris: vt Hesiodus dicit, \textgreek{[Greek]}.

Quod tamen longe aliter interpretes Græci illius poëtæ exponunt.

Aiunt enim Hesiodum intellexisse de tricesima mensis Lunaris: \& sensum loci Hesiodei esse perinde ac si dixisset, Quando homines veram \textgreek{[Greek]} Lunarem agunt, \& non secundum vsum politicum, sed secundum motum Lunæ.

Quod \mletter{C} tamen nobis valde coactum videtur: \& mentem Hesiodi hanc fuisse dicimus: \textgreek{[Greek]} esse valde idoneam rebus gerendis ea hora, qua homines ad ius in forum conueniunt.

Homerus Odyss. \textgreek{[Greek]}

\textgreek{[Greek]}

\textgreek{[Greek]}

Quæ sane interpretatio melior vulgari.

Sic etiam paulo post dicit, \textgreek{[Greek]}, loquens de vndecima: cuius partem designat, cum dicit \textgreek{[Greek]}.

Quod nos interpretamur iam adulto die.

Sic Homerus meridiem designat, \textgreek{[Greek]}.

Porro neq; hoc verbum \textgreek{[Greek]} id, quod nunc, valebat.

Sed tempus actuum quotidianorum illo notabatur: vt cum dicebant \textgreek{[Greek]}.

\mletter{D} Latinis vero Tempestas dicebatur.

In Legibus Decemuirum Atticis fuit: SOL OCCASVS SVPREMA TEMPESTAS ESTO.

Neque recte quidam hinc expungunt TEMPESTAS. quod SVPREMA absolute diceretur, vt apud Plautum.

Nam plane in legibus Solonis, vnde illud caput traductum, scriptum fuit, \textgreek{[Greek]}.

Stoicus scriptor apud Stobæum loquens de Socratis iudicio capitali: [Greek Greek Greek].

Idem censeas de veteribus Hebræis, \mletter{A} qui diei nullas alias partes, quam mane, meridiem, \& vesperam norant. \& ita dies diuiditur Psalmo L V, commate X V I I I.

\end{parnumbers}
\clearpage
p. 5 [pdf 88]

\begin{parnumbers}

Sic Homero, \textgreek{[Greek]}.

Sed hic dies intelligitur Lux, exclusa nocte.

Nam totum \textgreek{[Greek]} Hebræi in quatuor partes diuidebant, quas vigilias vocabant.

Prima vigilia erat à vespere: secunda à media nocte: tertia à mane: quarta à meridie.

Alioqui nomen hoc \texthebrew{[Hebrew]} quo hodie horam designant, ne notum quidem illis erat: atque apud Danielem aliud significat.

Posterorum inuentum est Horologium, \& \textgreek{ηλιοτρόπια[Greek: heliotropia; probably: sundails]}, quibus dies per lineas, \& interualla vmbrarum distinguebatur. vnde prodiit locutio \textgreek{[Greek]}, pro hora cœnæ. vel \textgreek{[Greek]}: \mletter{B} quia notis literarum singularium horæ distinguebantur.

Testatur \& Epigrammatium de Horologio:

\textgreek{[Greek]}

\textgreek{[Greek]}

Nam ante \textgreek{Ζ, Η, Θ, Ι,} erat \textgreek{Α, Β, Γ, Δ, Ε, ς.}

\begin{wraptable}{r}{0.6\textwidth}
\footnotesize
\setlength{\tabcolsep}{3pt}
\begin{tabular}{ |r @{}| r  r  r | c |r | r | r r | }
\multicolumn{4}{@{}p{4cm}@{}}{\parbox[t]{4cm}
 {\scshape\small TABVLA CON-\\
 \footnotesize vertendi osten-\\
 \upshape ta in sexagesimas.}}
& &
\multicolumn{4}{@{}p{4cm}@{}}{\parbox[t]{4cm}
 {\scshape\small TABVLA CON-\\
 \footnotesize vertendi sexage-\\
 \upshape simas in ostenta.}}
\\
\cline{1-4} \cline{6-9}
\itshape\scriptsize Ostenta. &
\itshape\scriptsize Sexag. &
\itshape\scriptsize Sexag. &
\itshape\scriptsize Sexag. &
\hspace{5mm} &
\itshape\scriptsize Sexag. &
\itshape\scriptsize Sexag. &
\itshape\scriptsize Ostenta. &
\itshape\scriptsize Ostenta.
\\
\cline{1-4} \cline{6-9}
   1 &  0' &  3'' & 20''' & &  0' &  1'' &    0' & 324'' \\
   2 &  0' &  6'' & 40''' & &  0' &  2'' &    0' & 648'' \\
   3 &  0' & 10'' &  0''' & &  0' &  3'' &    0' & 972'' \\
   4 &  0' & 13'' & 20''' & &  0' &  4'' &    1' & 210'' \\
   5 &  0' & 16'' & 40''' & &  0' &  5'' &    1' & 540'' \\
   6 &  0' & 20'' &  0''' & &  0' &  6'' &    1' & 864'' \\
   7 &  0' & 23'' & 20''' & &  0' &  7'' &    2' & 108'' \\
   8 &  0' & 26'' & 40''' & &  0' &  8'' &    2' & 432'' \\
   9 &  0' & 30'' &  0''' & &  0' &  9'' &    2' & 756'' \\
  10 &  0' & 33'' & 20''' & &  0' & 10'' &    3' &   0'' \\
  20 &  1' &  6'' & 40''' & &  0' & 20'' &    6' &   0'' \\
  30 &  1' & 40'' &  0''' & &  0' & 30'' &    9' &   0'' \\
  40 &  2' & 13'' & 20''' & &  0' & 40'' &   12' &   0'' \\
  50 &  2' & 46'' & 40''' & &  0' & 50'' &   15' &   0'' \\
  60 &  3' & 20'' &  0''' & &  1' & 60'' &   18' &   0'' \\
  70 &  3' & 53'' & 20''' & &  2' &  0'' &   36' &   0'' \\
  80 &  4' & 26'' & 40''' & &  3' &  0'' &   54' &   0'' \\
  90 &  5' &  0'' &  0''' & &  4' &  0'' &   72' &   0'' \\
 100 &  5' & 33'' & 20''' & &  5' &  0'' &   90' &   0'' \\
 200 & 11' &  6'' & 40''' & &  6' &  0'' &  108' &   0'' \\
 300 & 16' & 40'' &  0''' & &  7' &  0'' &  126' &   0'' \\
 400 & 22' & 13'' & 20''' & &  8' &  0'' &  144' &   0'' \\
 500 & 27' & 46'' & 40''' & &  9' &  0'' &  162' &   0'' \\
 600 & 33' & 20'' &  0''' & & 10' &  0'' &  180' &   0'' \\
 700 & 38' & 53'' & 20''' & & 20' &  0'' &  360' &   0'' \\
 800 & 44' & 26'' & 40''' & & 30' &  0'' &  540' &   0'' \\
 900 & 50' &  0'' &  0''' & & 40' &  0'' &  720' &   0'' \\
1000 & 55' & 33'' & 20''' & & 50' &  0'' &  900' &   0'' \\
     &     &      &       & & 60' &  0'' & 1080' &   0'' \\
\cline{1-4} \cline{6-9}
\end{tabular}
%\end{table}
\end{wraptable}


Arabibus, Persis, \& reliquis Orientis gentibus non horologiis, sed naturalibus matutini, meridiani, \& vespertini temporis interuallis diem notare, etiam hodie consuetudo manet.

Astronomis propria \mletter{C} est diuisio diei in sexagesimas primas, secundas, tertias, \& sic deinceps.

Artificibus computi annalis in horas, puncta, ostena, minuta, partes.

Hora est punctorum 4. mintorum 40. partium 480. momentorum 1760. ostenta autē sunt arbitraria, quibuslibet aliarum diuisionum in illa resolutis.

\mletter{D} Orientalibus vero Computatoribus compendiosa horarum resolutio est.

Non enim in sexagesimas assem diuidunt, sed in 1080 partes ita vt 18 particulæ vni minuto horario respondeant.

Hac diuisione hodie Iudæi, Samaritani, Arabes, Persæ, \& aliæ Orientis nationes vtuntur.

\end{parnumbers}
\clearpage
p. 6 [pdf 89]

\begin{parnumbers}

\mletter{A} Quorum in sexagesimas, \& contra, sexagesimarum in hæc conuertendarum, Tabellas duas posuimus.

\end{parnumbers}

\subsection{De Diebus.}
\setcounter{parcount}{0}
\begin{parnumbers}

\textgreek{Το νυχθήμερον[Greek: the day and night, i.e. a full 24 hour cycle]}, quod est spatium viginti quatuor horarum, Daniel eleganter vocat \texthebrew{[Hebrew]} quasi dicas \textgreek{[Greek]}, initio diei ciuilis sumto Iudiace ab eo tempore, quod proxime Solem occasum sequitur.

Nam illud interuallum, quatenus vigintiquatuor horarum est, naturale est: quatenus aliud atque aliud initium habet, dicitur ciuile, Atticis \& Iudæis ab occasu Solis: Ægyptiis \& Romanis à media nocte: Chaldæis Genethliacis ab ortu Solis: Vmbris à meridie initium \mletter{B} sumentibus.

Dierum notationes duplices: aut secundum numerum, \& ordinem: vt prima, secunda, tertia mensis. aut secudum \textgreek{[Greek]}, qua dies alicui rei cognomines. vt dies mensis Persici sunt cognomines regum priscorum: \& dies mensis Mexicanorum, animalium, aut aliarum rerum: \& \textgreek{[Greek]} Ægyptiorum nominibus singulorum Deorum vocatæ. \& dies festi, vt quinquatrus, \textgreek{κρόνια[Greek: of Kronos, i.e. Saturn]}, \textgreek{ϑαργήλια[Greek]}, Quirinalia. \& ab euentu, dies Alliensis, Regifugium. à stellis, dies Septimanæ.

Ecclesia Romana vocat ferias. quia veteris anni Ecclesiastici initium à Pascha.

Et Pascha dicebatur annus nouus, vt etiam hodie ab Ecclesia Antiochena: à Constantinopolitana autem \textgreek{[Greek]}, ab eadem mente.

Illius autem Hebdomadis dies omnes septem erant \mletter{C} feriati, vt testis est Hieronymus, \& alij veteres.

Hinc obtinuit, vt reliquarum hebdomadum dies etiam Feriæ vocarentur, præcipuo quodam principis septimanæ Paschalis auspicio \& omine.

Solon autem primus omnium \textgreek{[Greek]} vocauit, cum antea \textgreek{[Greek]} esset prima mensis.

Hesiodus: \textgreek{[Greek]}.

Diei diuisio summa ab actibus quotidianis, in fastos, nefastos, atros, religiosos, intercisos, iustos: vt Græcis \textgreek{[Greek]}, vel, vt alij, \textgreek{[Greek]}, \textgreek{[Greek]}. aut ab æquatione annui temporis, Solaris, \& Lunaris, in \textgreek{[Greek]}, \textgreek{[Greek]}, \textgreek{[Greek]}, \textgreek{[Greek]}, \textgreek{[Greek]}, \textgreek{[Greek]}, \textgreek{[Greek]}.

\textgreek{[Greek]} Computatoribus Græcis dicuntur, quæ Latinis Regulares, quæ cum \mletter{D} Concurentibus. id est Epactis Solaribus compositæ dant characterem Kalendarum, aut alius diei mensis.

\textgreek{[Greek]} sunt duplicis generis, Solares, \& Lunares.

Solares fiunt abiectis septenariis ex cyclo Solari, addito præterea die bisextili.

Lunares producuntur, excessu Solis, qui est \textsc{x~i} dierum, in numerum aureum ducto, abiectis tricenariis.

\end{parnumbers}
\clearpage
p. 7 [pdf 90]

\begin{parnumbers}

Præterea vtrarumque Epactarum sua methodus: Solarium ad characterem dierum: Lunarium ad ætatem Lunæ, vt Computatores Latini loquuntur, vt \mletter{A} Græci autem, \textgreek{[Greek]}.

\textgreek{[Greek]} sunt, quæ eximuntur de mense, duplici ex causa: aut vt rationes Solis cum Lunaribus congruant, vt in anno veteri Græcorum: \& in enneadecaeteride Paschali Saltus Lunæ Latinis dictus, Græcis \textgreek{[Greek]}. aut vt solennia festa cum feria Septimanæ, vt in anno Iudaico.

\textgreek{[Greek]}, vel \textgreek{[Greek]} sunt, quæ ex caussa religionis, transferuntur, \& dissimulantur per speciem comperendinationis, vt in anno Iudaico, \& olim in prisco Romano.

In Iudaico enim \textgreek{[Greek]} \& comperendinationes institutæ, ne feria secunda, quarta, sexta in caput anni incurrat. in Romano prisco comperendinabantur Nundinæ, vt à religiosis diebus summoueientur, auctore Macrobio.

\textgreek{Εμβόλίμοι [Greek]} sunt, vt notio verbi declarat, insititij \mletter{B} dies: \& erant naturales, aut ciuiles.

Naturales, qui ex scrupulis, \& horis appendicibus colliguntur, vt quatro quoque anno exeunte vnus dies ex quadrantibus anni Iuliani, quod \textsc{b~i~s~e~x~t~u~m} vocatur: item in periodo Arabica vndecies vnus dies intercalatur in fine Dulhagiathi, qui est vltimus mensis anni Hagareni Mohamedici.

Ciuiles sunt, qui præter naturalem anni rationem \& modum inseruntur, vt vnus dies in fine Marcheschvvan Iudaici, anno qui dicitur superfluus, aut abundans.

\textgreek{[Greek]}, quæ explendis spatiis anni adiiciuntur potius, quam inseruntur, vt quinque, quæ anno æquabili extra ordinem mensium adiectæ Ægyptiis dicuntur \textsc{n~i~s~i}, Persis, \& Armeniis \textsc{m~v~s~t~e~r~a~k~a} : item duæ, quæ extra modum anni Attici in calce Posideonis \mletter{C} appensæ, \textgreek{[Greek]} dicebantur, aut \textgreek{[Greek]}, aut \textgreek{[Greek]}.

At \textgreek{[Greek]} locum habent in anno mobili.

Est autem interuallum inter epocham \& caput anni, vtroque termino excluso.

Hoc constat semper in annis, quorum caput nunquam epocham anteuertebat.

Vt in anno Attico caput Hecatombæonis nunquam ante Solstitij veterem epocham statuebatur.

Itaque quod inter Solstitium, \& propositum Hecatombæonem interiacet spatij, vtroque termino excluso, dicebantur \textgreek{[Greek]}.

Idem obseruabatur in annis magnis Metonis \& Calippi.

Rursus Romanorum sacri dies Kalendæ, Nonæ, Eidus: Græcorum autem \textgreek{[Greek]}.

Quod ex versu Hesiodi à nobis adductor constat.

Sunt præterea nomina imposita diebus mensium \mletter{D} singulis, vt suo loco referetur.

Sunt \& secundum hebdomadas vt infra subiecimus.
\end{parnumbers}
\clearpage
p. 8 [pdf 91]

\mletter{A}
%% Can't put \mletter{} next to a table. It needs to be in a paragraph level
%% So we skip B and C

\begin{table}[h]
\large
\begin{tabular*}%
{\textwidth}{%
@{\extracolsep{\fill} } r r r @{\hspace{4pt}} || r @{\hspace{4pt}} | @{} l 
}
\multicolumn{3}{c}{\textsc{DIES HEBDOMADIS}} &
\multicolumn{2}{c}{\textsc{ALITER PERSICE.}}
\\
\multicolumn{3}{c}{\textsc{persicæ.}} & \multicolumn{2}{c}{}
\\
\hline
\texthebrew{שנב} % some nonsense filler text
& \textarabic{شزذيثب} % some nonsense filler text
& \textarabic{ل}
& 1
& \textit{Ruz iache}
\\
\texthebrew{[Hebrew]}
& \textarabic{[Persian]}
& \textarabic{ب}
& 2
& \textit{Ruz duiemi}
\\
\texthebrew{[Hebrew]}
& \textarabic{[Persian]}
& \textarabic{ج}
& 3
& \textit{Ruz siumi}
\\
\texthebrew{[Hebrew]}
& \textarabic{[Persian]}
& \textarabic{ﺩ}
& 4
& \textit{Ruz tzeharmi}
\\
\texthebrew{[Hebrew]}
& \textarabic{[Persian]}
& \textarabic{م}
& 5
& \textit{Ruz pengemin}
\\
\texthebrew{[Hebrew]}
& \textarabic{[Persian]}
& \textarabic{و}
& 6
& \textit{Ruz schesmin}
\\
\texthebrew{[Hebrew]}
& \textarabic{[Persian]}
& \textarabic{ز}
& 7
& \textit{Ruz haphthemi}
\end{tabular*}

\vspace{\baselineskip}

\begin{tabular*}
{\textwidth}{%
    @{\extracolsep{\fill} } r r @{\hspace{4pt}} || r @{\hspace{4pt}} r c
}
\multicolumn{2}{c}{\textsc{TVRCIÆ HEBDOMADIS}} & \multicolumn{3}{c}{\textsc{SECVNDVM PLANETAS.}}
\\
\multicolumn{2}{c}{\textsc{dies.}} & \multicolumn{3}{c}{}
\\
\texthebrew{[Hebrew]}
& \textarabic{[Arabic]}
& \texthebrew{[Hebrew]}
& \textarabic{[Arabic]}
& \astro{♄}
\\
\texthebrew{[Hebrew]}
& \textarabic{[Arabic]}
& \texthebrew{[Hebrew]}
& \textarabic{[Arabic]}
& \astro{♃}
\\
\texthebrew{[Hebrew]}
& \textarabic{[Arabic]}
& \texthebrew{[Hebrew]}
& \textarabic{[Arabic]}
& \astro{♂}
\\
\texthebrew{[Hebrew]}
& \textarabic{[Arabic]}
& \texthebrew{[Hebrew]}
& \textarabic{[Arabic]}
& \astro{☉}
\\
\texthebrew{[Hebrew]}
& \textarabic{[Arabic]}
& \texthebrew{[Hebrew]}
& \textarabic{[Arabic]}
& \astro{♀}
\\
\texthebrew{[Hebrew]}
& \textarabic{[Arabic]}
& \texthebrew{[Hebrew]}
& \textarabic{[Arabic]}
& \astro{☿}
\\
\texthebrew{[Hebrew]}
& \textarabic{[Arabic]}
& \texthebrew{[Hebrew]}
& \textarabic{[Arabic]}
& \astro{☾}
\end{tabular*}
\end{table}

\begin{parnumbers}
Cur autem dies cognomines Planetarum non sequuntur ordinem \& situm siderum, quorum cognomines sunt, vt scilicet post diem Saturni non sequatur dies Iouis, sed dies Solis, hæc caussa est.

% Diagram: circle with heptagram, with planets at the points:
% Moon ☾, mercury ☿, venus ♀, sun ☉,
% mars ♂, jupiter ♃, saturn ♄
%\begin{wrapfigure}[9]{r}{10\baselineskip}
\begin{wrapfigure}[9]{R}{9\baselineskip}
  \centering
  \def\svgwidth{9\baselineskip}
  {\astrofont\input{./img/planets.pdf_tex}}
\end{wrapfigure}

Septem Planetæ per circulum secumdum ordinem suum dispositæ, æquabili interuallo constituunt septem Triangula isoscele ad peripheriā, \mletter{D} quorum bases sunt latera Heptagoni circulo inscripti, vt habes in circulo proposito, ad cuius peripheriam septem errantes sunt secundum feriē suam sitæ, constituentes triangula isoscele \astro{♄♀♃}, \astro{♃☿♂}, \astro{♂☽☉}, \astro{☉♄♀}, \astro{♀♃☿}, \astro{☿♂☽}, \astro{☽☉♄}.

In quibus Triangulis dexter angulus ad basim est prima stella Trianguli, secunda in angulo ad verticem, tertia angulus sinister ad basim: ita vt omnis stella anguli dextri habeat oppositam \mletter{A} stellam anguli in vertice, stella autem anguli à vertice stellæ anguli sinistri ad basim sit opposita.

\end{parnumbers}
\clearpage
p. 9 [pdf 92]

\begin{parnumbers}

Sequentur igitur sese omnes septem Planetæ non per seriem suam, sed per interualla laterum, quæ veræ sunt oppositiones.

Sit igitur Triangulum \astro{☉☽♂} primum ordine.

\astro{☉} in angulo basis dextro præibit. sequetur Luna ei opposita in vertice, eam oppositus Mars in angulo sinistro basis. qui quidem Mars cum in Tiangulo \astro{☉☽♂}, sinistrum angulum basis occupet, in triangulo \astro{♂☿♃} occupabit dextrum basis angulum, habens oppositum Mercurium, Mercurius autem oppositum Iouem in angulo sinistro. qui Iuppiter faciet angulum dextrum in Triangulo \astro{♃♀♄}, habens oppositam in vertice \mletter{B} Venerem, vt ea opposita est Saturno in angulo sinistro.

Sed angulus ille rursus erit dexter in Triangulo \astro{♄☉☽}.

Et sic erogati sunt septem planetæ in totidem dies, quas Ecclesia Romana vocat ferias.

Hæc est vera harum appelationum ratio.

\end{parnumbers}

\subsection{De Mensibus.}
\setcounter{parcount}{0}
\begin{parnumbers}

\dropcap{3}{E}{x} diebus fiunt \textgreek{[Greek]}, quæ notationes \& epochat temporum constituunt.
\\ \p
Primum \textgreek{[Greek]} ex diebus dicitur Septimana, res omnibus quidem Orientis populis ab vltima vsque \mletter{C} antiquitate vsitata, nobis autem Europæis vix tandem post Christianismum recepta.

De ea iam dictum est.

Tum Romanorum \textgreek{[Greek]}: cui successit hebdomas nostra.

Nam nono quoque die Nundinæ erant. \& spatium illud in Kalendario vetere Romano notatum est literis ab A ad H, vt in nostro Kalendario Hebdomas notata est ab A ad G, inclusiue, vt loquuntur.

Mexicanorum \textgreek{[Greek]} sequitur.

Quod enim spatium nobis septenis diebus, illis finitur ternis denis.

Ita Iudæorum est \textgreek{[Greek]} veterum Romanorum \textgreek{[Greek]}, Mexicanorum \textgreek{[Greek]}.

Proximum ab hoc \textgreek{[Greek]} dierum est Mensis: qui \& naturaliter, \& ciuiliter sumitur.

Naturalis mensis \& ipse duplex.

\mletter{D} Aut enim Lunaris, aut Solaris.

Rursus Lunaris triplicis generis: aut quatenus Luna ab eodem puncto Zodiaci profecta, ad idem reuertitur: qui dicitur \textgreek{[Greek]}, item \textgreek{[Greek]}.

quod interuallum minus est, quam viginti octo dierum: maius quam viginti septem.

Secundum genus est eiusdem sideris à Sole profecti ad eundem reditus.

Hæc dicitur \textgreek{[Greek]}.

Tertij generis mensis est secundus dies \textgreek{[Greek]}, quae dicitur \textgreek{[Greek]}, \& \textgreek{[Greek]}.

Secundum \& tertium genus in temporibus ciuilibus locum habent.

Nam Athenienses \textgreek{[Greek]} neomenias suas putabant: hodie vero Hagareni \textgreek{[Greek]}.

\end{parnumbers}
\clearpage
p. 10 [pdf 93]

\begin{parnumbers}

Græcorum enim neomenias ab ipso iugo Lunæ putari solitas testis Vitruuius ex Aristarcho Samio, his verbis, loquens de Luna: Quot mensibus sub rotam Solis radiosque primo die \mletter{A} antequam præterit, latens obscuratur.

\&, cum est sub Sole, noua vocatur.

Postero autem die, quo numeratur secunda, præteriens à Sole, visitationem facit tenuem extremæ rotundationis. [Vitruvius, De architectura libri decem, Liber IX, Capitulum II, Sect. 3: "Ita quot mensibus sub rotam solis radiosque uno die, antequam praeterit, latens obscuratur. Cum est cum sole, nova vocatur. Postero autem die, quo numeratur secunda, praeteriens ab sole visitationem facit tenuem extremae rotundationis." Transl.: "whence, on the first day of its [the moon's] monthly course, hiding itself under the sun, it is invisible; and when thus in conjunction with the sun, it is called the new moon. The following day, which is called the second, removing a little from the sun, it receives a small portion of light on its disc."]

Vbi etiam dixit visitationem extremæ rotundationis, quam ille Samius sine vllo dubio \textgreek{φαίσιν μιωοειδῆ[Greek: phase …]} vocabat.

Sed \& Onomacritus, qui sub nomine Orphei \textgreek{τελετὰς[Greek: ceremony]} scripsit, in opere, quod \textgreek{ἡμέρας[Greek: day, hours of daylight]} vocauit, mensem Lunarem à iugo Lunæ incipit.

Cuius versus apposui:

\begin{greek}
Παίτ᾽ ἐδάης Μουσαῖε ϑεοφραδἐς. εἰδέ σ᾽ αἰώγει[Greek ?]

ϑυμὸς ἐπωνυμίας μήνης κατὰ μοῖραν ἀκοῦσαι,[Greek ?]

ῤεῖά τοι ἐξερέω, σὺ δ᾽ ἐνὶ φρεσὶ βάλλεο σῆσιν,[Greek] \mletter{B}

οἵην τάξιν ἔχοντα κυρεῖ. μάλαν γαρ χρέος ἐστὶν[Greek]

ἴδμεναι, ῶς αὕτη παρέχει κλέος ἄντυγι[?] μηνός.[Greek]

\textgreek{[Greek]}

\textgreek{[Greek]}

\textgreek{[Greek]}

\textgreek{[Greek]}

\textgreek{[Greek]}

\textgreek{[Greek]}
\end{greek}

Sed Neomenia Arabica, excedit modum \textgreek{φάσεως[Greek: phase]} vt plurimum. ita vt ciuiles neomeniæ mensium Lunarium sint non vnius generis: Atticæ \mletter{C} \textgreek{[Greek]}: Iudaicæ sæpe \textgreek{[Greek]}.

Arabicæ semper \textgreek{[Greek]}, à tertia, inquam, die.

Mensis Solis naturalis est, qui naturalibus circuli cœlestis segmentis definitur, qualis est transitus Solis à signo ad signum.

Hi, \& Lunares, sunt vere cœlestes menses.

Mensis ciuilis Solis est, qui non naturali modo, sed æqualiter tributus est. vt in anno Ægyptiaco \& Græco omnes æqualiter sunt \textgreek{[Greek]}: \& in Lunari alternis pleni, \& caui. in anno Mexicano \textgreek{[Greek]} cum ex X V I I I. mensibus eorum annus constituatur.

Apud Albanos Martius erat sex \& triginta dierum, Maius viginti duum, Sextilis duodeuiginti, September sedecim.

Tusculanorum Quintilis habuit tirginta sex, October triginta duos, Aricinorum October trigintanouem.

At rationes Lunæ non patiuntur, vt menses sint alternis perpetuo pleni, \& caui. sed hoc ad methodum ciuilis temporis institutum.

Sunt \& alij menses ex superfluis diebus collecti, qui Embolimi dicuntur: iique aut naturales, aut ciuiles: ambo autem ad æquationem Solis directi.

Naturales embolimi sunt, qui ex Solis excessu collecti ad spatia Lunæ complenda adhibentur. cuiusmodi est Iudaicus Adar prior, \& Samaritanus Adar alter. isque mensis est semper tricenum dierum.

Ciuilis embolimus, qui ex diebus Solis superfluis consurgens fulciendo anno cauo adiictur.

\end{parnumbers}
\clearpage
p. 11 [pdf 94]

\begin{parnumbers}

Eiusmodi erat Merkendonius \mletter{A} prisci anni Romani alternis binum \& vicenum, item trinum \& vicenum dierum.
Eiusmodi \& Posideon Atticus.

Neque enim Posideon naturalis esse potest, quamuis triginta dierum, cum neque Lunaris esset, quod eius neomenia longe à lunari discederet: neque Solaris, quod pars esset illius anni, qui ad Solis cursum descriptus non esset.

Idem de Merkedonio dicas, qui neque ad Solarem annum, neque ad Lunarem pertineret, neque modum eum haberet, qui iusto mensi competit, cum esset tantum XXII, aut ad summum XXIII dierum.

Mensis diuisio Atticis in \textgreek{[Greek]}. prima \textgreek{[Greek]} dicebatur \textgreek{[Greek]}, secunda \textgreek{[Greek]}, tertia \textgreek{[Greek]}.

Idque factum, quia illorum menses omnes erant \textgreek{[Greek]}.

Persæ vero in \textgreek{[Greek]}, \mletter{B} non solum, quia eorum menses omnes \textgreek{[Greek]}, sed etiam, quia totus annus constat ex quinariis tribus \& septuaginta.

In mense \textgreek{[Greek]} Athenienses pro \textgreek{[Greek]} dicebant \textgreek{[Greek]}. Quamuis enim mensem vno die mutilabant, tamen cum tertia mensis pro secunda dicebant, non videbantur mensem mutilare, cuius \textgreek{[Greek]} numerabant.

Meton vero \& Calippus eam diem eximunt, quæ post duas syzygias \& dies quatuor succedebat.

Mensium nomina in antiqua Hebraici anni forma nulla fuerunt, neque in hodierna Sinarum, Iaponensium, \& Indorum.

Menses enim illis ab ordine primi, secundi, tertij dicuntur.

In anno Romano mistæ sunt appellationes, ex cognominibus, \& ordine numerario.

Quidam etiam cognomines imperatorum Romanorum, vt Cypriis \textgreek{[Greek]}.

Romanis ipsus Iulius, Augustus: \& temporibus Domitiani Germanicus pro Septembri, Domitianus pro Octobri.

Martialis: Dum Ianus hiemes, Domitianus autumnos, \&c.

[Insert: page with latin text]

Sed Statius omnes Kalendas vindicat Domitiano, præter Iulium, \& Augustum, – Nondum omnis honorem Annus habet, cupiuntq; decem tua nomina menses.

Insania quoque Commodiidem cōsecuta esset, si \mletter{D} longior vita mōstro illi data fuisset.

\end{parnumbers}
\clearpage
p. 12 [pdf 95]

\begin{parnumbers}

Augustum enim Cōmodum, Septembrem Herculeum, Octobrem Inuictum, Nouembrem Exuperatorium, Decembrem Amazonium vocari edicit. Extat quoq; lapis Lauinij, in quo mentio Iduum Commodarum. vbi \& nomen Commodi Senatusconsulto prius derasum, postea alia manu \mletter{A} incusum.
Quædam nationes etiam geminos menses cognomines habent.

Annus Syrochaldaicus habet geminum Tisrin, item geminum Conum.

Annus Hagarenus geminum Regiab, \& geminum Giumadi.

Annus Sxonicus geminum Giuli, \& geminum Lida.

Sed in anno embolimæo Lida est tergeminus.

Et tunc annus ille dicebatur Trilida.

Item, diuersarum nationum iidem menses communes.

Nam Panemus in anno Macedonico fuit, item Corinthiaco, \& Thebano.

Artemisius communis fuit Laconum, \& Macedonum: Carneus Syracusanis, \& Cyrenensibus vsitatus.

Sed differbant situ anni \& tempore: vt suo loco disputabitur.

Sic Martius primus erat \mletter{B} Romanorum: tertius Albanorum, Aricinorum, Formianorum: quartus Forensium, Pelignorum, Sabinorum: quintus Faliscorum, Laurentum: sextus Hernicorum: decimus Æquicolorum. Hæc in genere de mensibus.

\end{parnumbers}

\subsection{De Anno.}
\setcounter{parcount}{0}
\begin{parnumbers}

Maximum \textgreek{[Greek]} dierum annus, sed qui multipliciter dictus sit.

Tot enim constitui possunt, quot sunt siderum errantium periodi.

Est enim annus circuitus eius periodi, cuius cognominis ipse est.

Vt annus Solaris est cognominis circuitus eius \mletter{C} sideris, qui quidem circuitus dupliciter sumitur, aut à Solstitio ad Solstitium, à bruma ad brumam: \& est minor anno Iuliano.

aut à puncto Zodiaci, ad idem punctum Zodiaci.

qui est maior anno Iuliano.

hoc est maior 365 1/4 diei.

quo ad id puncum Zodiaci redit, vnde profectum erat.

Eadem fere quantitas quæ \& Soli, attribuitur Veneri \& Mercurio.

Saturni periodus est dierum 10747.18'.59''.13'''.

Hoc est annorum Ægyptiorum 29. dierum 162.

Iouis annus dierum 4330. horarum 17.14'.

Id est annorum Ægyptiorum 11.315.

Martis annus dierum 686. horarum 22.24'.

annorum Ægyptiorum 1.321 dierum.

Lunæ, dierum 29.31'50''.8'''.

Obtinuit tamen vulgo, vt duorum siderum, Solis \& Lunæ, labentem cœlo qui ducunt annum, ratio in \mletter{D} temporibus ciuilibus haberetur.

Et Lunæ quidem primum vnus circuitus pro anno habebatur, vt apud Ægyptios. deinde tres, vt apud eosdem Ægyptios \& Arcades.

Tandem duodecim periodi Lunares annum ciuilem constituerunt dierum 354 cum triente, \& paulo plus quam duum trientum horariorum.

Duodecim quoque segmenta Zodiaci componunt annum Solarem tantum, quantum diximus.

\end{parnumbers}
\clearpage
p. 13 [pdf 96]

\begin{parnumbers}

Sed ignoratio motuum vtriusque sideris alias atque alias anni formas veteribus \mletter{A} ptperit: quarum vetustissima est ea, quæ annum quidem ad cursum Lunæ describebat: sed incertis neomeniis, quæ non produent ex obseruatione motus Lunæ, quales vulgus rusticorum obseruare solet, \& quæ proprie ciuilem mensem constituere non possunt.
Cum igitur hoc modo incertæ essent neomeniæ, conuenit primum, vt menses omnes tricenis diebus explicarent, annumque dierum sexaginta \& trecentum constituerent.

quod genus longe desciscebat à modo anni Lunaris.

Hæc diu seruata fuit apud Græcos anni forma.

In Oriente septuagesima secunda pars illius anni, hoc est quinq; dies, accesserunt anno Græco: vt anni modus suerit dierum trecentorum sexaginta quinque: \mletter{B} qua ratione ab anno solari se minimum discedere arbitrati sunt.

Vnde duo præcipua genera anni apud veteres suerunt neque Lunaria, neque Solaria, sed ambigui inter vtrumque generis.

Prior forma in Græcia resedit: altera in Oriente.

Græci vero non vna via ad emendationem suæ aggressi sunt.

Difficile erat menses plenos omnes ad Lunæ rationes exigere: \& tamen in quibusdam actibus ciuilibus opus habebant motu Lunæ.

Nam semper Olympias plenilunio, \& X V die mensis celebrabatur.

Vt igitur annus Græcus æquabilis Olympiadem deprehenderet in X V mensis, hoc difficile non erat.

Vt autem X V mensis in X V Lunæ incidat in mensibus æquabilibus, hoc fieri non potest, nisi post fingula quadriennia, adiectis vnicuique anno singulis \mletter{C} biduis, quas \textgreek{[Greek]} vocabant.

Hæc Tetraeteris Elidensibus vocata est Olympias, Delphis Pythias.

eiusque mensis primus duantxat erat Lunaris: reliquorum ratio claudicābat.

Primus Cleostratus eum annum in Lunarem modum reformare conatus est, excogitata octaeteride dierum 2922, cuius menses alternis pleni \& caui: anni vero singuli communes 354 dierum: embolimæi 384. communes quidem quinque, embolimæitres.

Syzygiæ autem nouem \& nonginta.

Octaeteridum vitio deprehenso, Meton enneadecaeterida excogitauit dierum solidorum 6940.

Cui castigandæ periodus Calippica successit dierum 27759, sine vllis scrupulis appendicibus, anno ab editione Metonica centesimo tertio.

Hanc excepit vltimus, tanquam secutor quidam, \mletter{D} Hipparchus, annis circiter centum octoginta octo ab epocha Calippica, periodo publicata dierum 111035: quæ minor est Calippicis rationibus die vno, Metonicis autem quinq;.

Quare duæ castigationes adhibitæ anno æquabili Græco.

Altera est coniugatio alterna vel interrupta mensium plenorum \& cauorum, vt cum ipsa Luna congruerent, quod annus Græcus maior esset Lunari. altera est embolismus mensium, vt cum sole æquaretur, quod annus Lunaris minor est Solari.

Sed alternatio plenorum \& cauorum mensium aliquando variat: idque sit aut naturaliter, aut ciuiliter.

\end{parnumbers}
\clearpage
p. 14 [pdf 97]

\begin{parnumbers}

Naturalis varietas committitur propter embolismum \mletter{A} aut mensis, aut diei.
Vtroque enim modo duo menses pleni continuantur.

Vt in anno Iudaico cum intercalatur mensis Adar, tunc Schebat, \& Adar embolimus ambo sunt pleni.

In anno vero Arabico cum accedit dies mensi vltimo, qui Dulhagiathi dicitur, tunc \& ipse Dulhagiathi, \& antecedens Dulkaadathi ambo fiunt tricenum dierum.

Sed in Samaritano sæpe continuantur tricenarij menses, \& in antiquo Iudaico, vt ex Talmud \& Iad Mosis cognoscimus: \& menses Harpali, Metonis, \& Calippi non semper alternis continuati sunt.

sed sæpe bini pleni continuati, nunquam autem bini caui.

Quin etiam cum dies accedit vltimo mensi Arabico, tres continui menses sunt pleni, Dulkaadathi, Dulhagiathi, \& Muharam sequētis anni.

Isque annus ab Arabibus dicitur \textarabic{[Arabic]} hoc est embolimæus.

Sic etiam anno Iudaico pleno tres menses continui sunt pleni, Tisri, Marchesvvan, Casleu.

Ciuilis varietas accidit anno Iudaico tantū, accrescente mensi Marcheschvvan die vno: \& Marchesvvan ex cauo sit plenus.

Rursus \& iin embolismo mensium differentia situ, \& tempore.

Situ, si aut in medio, aut in calce intercalatio fiat.

vt in anno Attico vltimus mensis intercalabatur, qui dicebatur \textgreek{[Greek]}.

in Iudaico sextus mensis intercalatur, \& dicitur Adar prior. In anno Hagereno mensis embolimus erat desultor, qui omnes menses anni percurrebat in annis 228, quæ sunt enneadecaeterides duodecim.

qua intercalatione memoria proauorum nostrorum vtebantur Turcæ Cilices, donec annum Hegiræ simplicern \mletter{C} Muhamedicum vsurpare cœperunt.

At in anno prisco Romanorum situs embolismi longe diuersus ab aliis.

non enim is inter duos menses interiiciebatur, vt alias solet: sed in mensem ipsum, tanquarn surculus in truncum infindebatur.

Inter X X I I I enim, aut X X I I I I, aut inter X X I I, \& X X I I I Februarij inserebatur. neque vero sine caussa.

Hoc enim semper obseruabant, vt mēsis proximus Martio semper esset dierum X X V I I I. eratque Februarius ordinarius. at interuallum inter exitum Ianuarij, \& Kalendas Februarij ordinarij imputabatur Merkedonio. \& Kelendæ Februarij ordinarij in anno embolimæo nunc in Regisugium, nunc in Terminalia, incurrebant.

Neque enim semper inter Terminalia, \& Regisugium intercalabantur, vt vult Censorinus. \mletter{D}

quia hoc pacto Februarius ordinarius nunc viginti octo; nunc vndetricenum dierum fuisset.

Quod tamen salsum ex Varrone conuic[ē]tur.

Tempore differt intercalatio, quatenus Iadæi nunquam intercalant, priusquam \textgreek{[Greek]}, qui sunt dies decem cum horis paulo magis quam vna \& viginti, eo rationes Solis deduxerint, vt commode mensis Lunaris conflari possit.

Quod spatium numquam maius est triennio, nunquam minus biennio: \& in X I X. annis semper septies fit.

\end{parnumbers}
\clearpage
p. 15 [pdf 98]

\begin{parnumbers}

At in Calippico \& Metonico anno aliquando citius, aliquando ferius \mletter{A} intercalabatur, quam ratiocinia \textgreek{[Greek]} postulare videntur.

quandoquidem hoc vnum cauent præcipue Athenienses, ne Hecatombæonis neome[?]a Solstitij priscam epocham anteuertat: cum in anno Iudaico vt plurimum neomenia Tisri æquinoctium autumnale, neomenia vero Nisan æquinoctium veris antiquum, si ratio Iuliani anni habeatur, anteuertat.

Anni Lunaris non vnum genus est: sed summa diuisio in duo fastigia discedit: in annos periodicos, \& simplices.

Anni periodici dicuntur, qui certo annorum orbe, interuentu embolismorum, recurrunt.

Huius interualli modum veteres certo definire non potuerunt. quippe Cleostratus dierum 2922, Harpalus \mletter{B} 2924, Eudoxus plusquam 2922, minus quam 2924: Meton aliter: \& ab omnibus diuerse Calippus, \& deniq; ab eo discedens Hipparchus.

Cuius sententia, sed cælestibus rationibus leuiter castigata, enneadecaeterida Lunarem minorem Iuliana statuit, hora vna cumscrup. paulo plus quam viginti septem.

Simplices anni \& ipsi quidem sine remedio intercalationis in pristinam epocham recurrunt, sed longo interuallo, annorum scilicet Iulianorum 228, qui sunt anni simplices Arabici 235, scrupuli diurni quinquaginta.

Sunt \& in annis Lunaribus caui, superflui, æquabiles.

Annus cauus is est, cui competit \textgreek{[Greek]}.

Ideo à nobis \textgreek{[Greek]} vocabitur. ex eo enim eximitur dies vel propter ciuile institutum, cuiusmodi est annus Iudaicus, quem defectiuum \mletter{C} Computatores Iudæorum vocant. (in eo quippe Casleu, qui natura est plenus, instituto fit cauus.) vel naturali de caussa: vt anno decimonono Cycli Paschalis Dionysius diem vnum eximit, quem vocauit Saltum Lunæ: Græci vero Computatores \textgreek{[Greek]}.

quamquam inepte annum vltimum enneadecaeteridis constituit dierum duntaxat 353, cum eiusmodi annus natura nullus fit.

Superfluus annus vocetur à nobis \textgreek{[Greek]}.

Accedit enim illi \textgreek{[Greek]} tam ex caussa ciuilli, vt in anno Iudaico marcheschvvan naturaliter cauus, ciuiliter fit plenus: quam e caussa naturali: vt vndecim anni in Triacontaeteride Arabica augentur singulis diebus ex ratiociniis Lunæ collectis.

Annus æquabilis vocetur \textgreek{[Greek]}.

Iudæis computatoribus \mletter{D} dicitur annus ordinarius.

Is est, cui nihil accedit, nihil decedit.

Huc vsque ad annum Lunarem deduxit nos æquabilis minoris disputatio.

Nunc de altero æquabili maiore disputandum, quo Ægyptij, Persæ, \& Armenij, Mexicani, \& Perusiani vsi.

Hic antiquitus Orientis nationibus vnus idemque fuit: præter quam si quando \textgreek{[Greek]} quinque in alium locum traductæ, diuersum anni caput constituebant. qua \textgreek{[Greek]} tralatione vtebantur ij, qui post annos 120 æquabiles mensem solidum intercalabant, vt Persæ: qui quidem \textgreek{[Greek]} suas in æquinoctium vernum semper reiiciebant.

Terminum autem vocabant N E V R V Z.
\end{parnumbers}
\clearpage
p. 16 [pdf 99]

\begin{parnumbers}

\& habebant mensem desultorem \mletter{A} \textgreek{[Greek]}, omnes menses anni peruagantem, donec in primum mensem recurreret.

qui orbis non redibat, nisi anno æquabili 1461 vertente, qui sunt anni Iuliani perfecti 1460.

Hic est magnus annus, cuius menses sunt annorum æquabilium tricenum, quot dierum simplex mensis.

\textgreek{[Greek]} autem sunt quinquies quatuor annorum, vt illæ simplices quinque dierum.

Quod autem illa anni forma retenta fit, in caussa fuit non tam ignoratio annis solaris, quam facilis, \& tractabilis, ac vere popularis eius vsus.

Alioqui nulla fere natio fuit, quæ quadrantem anni Solaris ignorarit: sed modum illius dispensandi nesciebant.

præterea à mensibus superfluis, qui sunt maiores tricenis diebus, refugiebant, quos necesse est retincri, quadrante illo retento. \mletter{B}

Ægyptij singulis quadrienniis exactis diem intercalabant in ortu Caniculæ, \& quadriennium illud exactum \textgreek{[Greek]}, \textgreek{[Greek]}, \textgreek{[Greek]}, vocabant.

Attici diem quarto quoque anno exacto intercalabant inter septimum \& octauum diem Ianuarij.

Elidenses inter octauum, \& nonum Iulij.

Syromacedones, Chaldæi, \& Iudæi inter septimum \& octauum Octobris.

Eamque diei intercalationem à Seleucidarum temporibus vsque ad imperium Constantini \& infra retinuerunt Iudæi: quam vtique simul cum anni Calippici forma à victoribus Syromacedonibus acceperant.

Romani Atticos secuti brumæ sidere confecto intercalabant; quæ ipsis Olympiadum mysteria vocabantur. Nam \& Attici \& reliqui omnes Græci annum Solarem in \mletter{C} quatuor quadrantes diuidebant, quæ \textgreek{[Greek]} vocabant, singulis dies 91. hor. 7 ½ attribuentes.

quod à temporibus Seleucidarum, ad hanc vsq; diem, Iudæi constanter obseruant. Itaque V I I I Iulij erant \textgreek{[Greek]}, V I I Octobris \textgreek{[Greek]}: V I I Ianuarij \textgreek{[Greek]}, V I I I Aprilis \textgreek{[Greek]}.

Quare cum legis \textgreek{[Greek]}, \& \textgreek{[Greek]}, nullas alias intellige, præter has. quod \& \textgreek{[Greek]} quoque intelligendum.

Hæc \textgreek{[Greek]} Iudæi Tekuphoth vocant.

Germani, Celtæ, Saxones inter X X V \& X X V I Decembris intercalabant: quam noctem vocabant M V D R A N E C H T.

Tartari hodi inter vltimam Ianuarij, \& Kalendas Februarij. quas Kalendas patrio sermone Festum Alborum vocant. quia albis vestibus eam diem colunt.

Denique quanuis \mletter{D} Lunari anno, aut alio longe diuerso à Solari vterentur, tamen tacita quadam obseruatione post dies 1460 vnum diem intercalandum esse sentiebant.

Neque enim aliter Habræi quatuor Tekuphas suas tueri potuissent, nisi quadrante post quartū quemq; annum rationibus accedente.

Et sane vnaquæq; Tekupha est dierū 91, horarum 7 ½ Vnde quatuor tantæ Tekuphæ fiunt dies 365 ¼.

\end{parnumbers}
\clearpage
p. 18 [pdf 100]

\begin{parnumbers}

Displicuit tamen hæc quadrantis obseruatio Græcis Astronomis, propter causam admodum futilem \mletter{A} \& puerilem, qua Solis quantitatem ad Lunæ ratiocinia exigebant, \& cum vtriusque sideris exactum modum adhuc non tenerent, ex Lunæ comparatione Solares rationes eliciebant.
Itaque tantam censuerunt Solis quantitatem, quantam summam dies periodi in annos periodi distributæ relinquebant.

Metonis periodus est dierum 6940.

Diuisa per 19 annos relinquit quantitatem anni Solaris Metonici dierum 365. scrup. diurnorum 15 5/19

Calippi periodus dierum 27759 per 76 annos diuisa relinquit modum anni Calippici Solaris dierum 365 ¼ qualis est annus noster Iulianus.

Periodus Hipparchi est dierum 111035, annorum 304.

Sed neglectis illis 2, trecentesima pars diei detrahitur de quantitate anni Calippici Solaris, \mletter{B} vt fiat annus Solaris Himmarchus dierum 365. hor. 5. 55.' 15.'' 15/19

Detractis ex quadrante hor. 0. 4.' 44.'' 4/19 quæ etiam fuit sententia Ptolemæi.

Itaque ex sententia Hipparchi \& Ptolemæi annus Tropicus, est annus Iulianus, vel Calippicus nonadecima parte differentiæ enneadecaeteridis Lunaris \& Iulianæ diminutus: qui est verus annus Rabbi Ada: de quo alibi.

Philolai Pythagorei magnus annus dierum 21505 ½ per 59 annos diuisus constituit modum Solarem dierum 365. Oenopidæ annus magnus dierum 21557 itidem per 59 annos diuisus dat modum anni Solaris dierum 365 cum parte dierum duum \& viginti vndesexagesima.

Harpali octaeteride per 8 annos diuisa remanet modus anni Solaris dierum 365 ½.

Annus magnus \mletter{C} Democriti dierum 29950 ½ per 82 annos diuisus relinquit annum Solarem dierum 365, cum quadrante \& centesima sexagesimaquatra parte vnius diei.

Denique nullus veterum non patauit rationes Solis ad Lunam exigendas esse.

Et quotiescunque ex certa collectione dierum vtriusq; sideris rationes congruerent, dies illi per to[t] annos diuisi, quot ex illa summa dierum constitui poterant, visi sunt illis certam anni Solaris quantitatem feninire posse.

Sapientiores vero, quanuis incomprehensibilem illam existimarēt, tamen pro vero quod proximum putabant amplexi sunt, dies trecentos sexaginta quinque cum quadrante, qui est modus anni Iuliani.

cui singulis quadrienniis exactis vnus dies accrescit.

sed hic annus comparatione Ægyptiaci \mletter{D} est Solaris: comparatione autem Tropici est æquabilis.

Maior enim est vera anni ratione scrup. horariis 11.' 6.'' 40.'' secundum Gelalæam formam, aut 10.' 48.'' fere, vt Alfonsini docent.

Neque Prutenicæ tabulæ multum abludunt, quæ constituunt motum æqualem Solis ab æquinoctio dierum 365. Hor. 5. 49.' 15.'' 46.'''

Itaque hinc nasci possunt aliquot genera anni Solaris.

Æquabilis, vt Iulianus.

Tropicus, vt Persarum Gelalæus.

Rursus Tropicus aut æquabilis, aut cælestis.

\end{parnumbers}
\clearpage
p. 18 [pdf 101]

\begin{parnumbers}

Æquabilis Tropicus, cuius quantitas Tropica est, partes autem, hoc est menses, æquales \& ciuiles: vt is, \mletter{A} quem modo dixi, Galelæus.
Descriptus est enim mensibus æqualibus, omnibus tricenum dierum, cum epagomenis appendicibus, quæ in communi anno sunt quinque, in embolimæo sex.

Cælestis Tropicus, cuius partes in naturalia Zodiaci segmenta tributæ sunt.

Rursus \& annus Solis æquabilis in ciuilem \& cælestem diuidi potest.

Ciuilis, vt Iulianus Romanorum, Syrogræcorum, Græcorum Elkupti.

Cælestis, vt Dionysianus Prolemæi Philadelphi.

Nam \& is quoque quadrantem Canicularem quadriennio exactor accipiebat.

Finis vero omnis periodi est, vt caput recurrat \& reuoluatur in idem principium, quam \textgreek{[Greek]} Græci vocant: quæ quidem pessum iuerit tandem, non seruata veri anni Tropici mensura.

\& qua annus Iulianus \mletter{B} suam tueri non potuit, manifestum est Kalendas Ianuarias ab V I I I parte Capricorni, in qua statuerat eas Cæsar, in vicesimam primam fere traductas esse hodie.

Sed nihilo commodius epocha in enneadecaeteride seruari potest.

Nam enneadecaeteris Tropica est velocior Lunari horis plusquam duabus.

Contra enneadecaeteris Iuliana maior Lunari hora vna, \& scrup. plusquam 26.

Cum vero peccatur vtraque ratione, Tropica \& Iuliana, Luna, cuius rationes mediæ sunt inter illas duas, fines epochæ suæ tueri non potest: vt in cyclo Dionysij Paschali accidit, cuius neque rationes ad enneadecaeterida Luanrem collectæ sunt, neque epocha ad Solis motum castigata: sed eius forma potius tota mere Calippica est.

ita vt eius statum post trecentos \mletter{C} 4. annos variare necesse sit.

Quare vt epochat suas seruarent illi veteres, immanes periodos excogitauerunt, quales illæ Calippi, Philolai, Democriti, Ocnopidæ. Sunt etiam periodi, quæ omnem modum excedebant.

Et cum in omnibus illis orbibus annorum præcipuam vtriusque sideris ationem haberent, tamen nescio quæ confidens eos incessebat opinio, non solum vtriusque sideris, sed etiam omnium \textgreek{[Greek]} illo circuitu fieri.

Sic Harpalus \& Eudoxus putarunt in sua Octaeteride omnes \textgreek{[Greek]} \& \textgreek{[Greek]} in orbem redire.

Idem etiam censet fieri Aratus in Metonica enneadecaeteride, Eudoxum suum sectus, qui in fabrica Sphæræ suæ eam planetarum \& inerrantium harmoniam in eorum orbibus ostendit esse, vt sequente \mletter{D} restitutione vtriusque sideris, necessario \& omnium inerrantium reditum contingere concluderet.

Propterea tot Sphæras \textgreek{[Greek]} commentus est, quot narat Aristoteles libro X I \textgreek{[Greek]} quem consulas licet.

Quin etiam Calippus alios orbes præter Eudoxum addidit, ea ratione, vt \textgreek{[Greek]} adstrueret, \textgreek{[Greek]}, vt Aristoteles de ea re scribens pronunciauit.

\end{parnumbers}
\clearpage
p. 19 [pdf 102]

\begin{parnumbers}

Itaque \textgreek{[Greek]} nomine intelligendum ortus, \& occasus \textgreek{[Greek]}, \mletter{A} non autem \textgreek{[Greek]}, hoc est significationes eorum: quas in orbem redire cum Luna \& Sole in enneadecaeteride Meto quidem, Calippus, \& Hipparchus putarunt, \& aliis persuaserunt, donec deprehenso vero anni Tropici modulo vidium harum periodorum castigatum est.
Cicero quoque apud Macrobium, sexto de republica, annum illum immanem, quem ex tot millibus annorum simplicium componit, non aliter in orbem rediturum cum omnibus errantibus \& inerrantibus censet, quam si eadem defectio Solis in eodem loco, eodem tempore fiat: quanuis defectiones cyclo enneadecaeterico recurrant non raro.

Et tamen ea eclipsi putat non tantum Solis \& Lunæ, sed etiam quinque errantium ad eandem \mletter{B} inter se comparationem, confectis omnium spatiis, reditum fieri, quo eadem cæli positio, siderumque, quæ ab initio maxime fuit, rursus existit.

Quare eclipses ad eam rem notabant veteres, vt etiam \textgreek{[Greek]} excogitarint \textgreek{[Greek]} vocabant.

Eorum vetustissimus fuit dierum 6585 ⅓, qui sunt anni Arabici 18, syzygiæ 7. in genere vero sunt syzygiæ 223.

Quamobrem in secundo libro Plinij perparem legitur siue culpa ipsius Plinij, siue librarij, defectus luminum ducentis viginti duobus mensibus redire.

Hipparchus alium \textgreek{[Greek]} longe maiorem excogitauit dierum 126007, syzygiarum 4267, annorum Arabicorum 355 cum syzygiis 7: annorum Iulianorum 344 cum diebus 361.

Quæ sunt tolerabiles periodi.

Nam à caussis naturalibus, \mletter{C} nempe à defectionibus luminum proficiscuntur.

quemadmodum etiam enneadecaeteris Lunaris, \& Cyclus Solis: quorum illa Lunam Soli restituit, hic Solem Septimanæ, \& præterea periodus Mexicanorum constans annis L I I, quæ restituit \textgreek{[Greek]}, quæ ist ipsis vicem nostræ Hebdomadis.

Neque alia fuit periodus magna Persatum veterū, quam Salchodai vocabant.

Sunt \& aliæ, sed ciuiles, \& Indictio; Aliæ inanibus coniecturis insistunt, vt Dodecaeteris Chaldaica Genethliacorum, item Heracliti, Lini, Orphei, Dionis, \& Maorum: quorum periodus ad modum octauæ sphæræ composita est annorum 360000 à conditu Mundi, vt ipsi putant. quorum annorum hic est centies octagies quater millesimus, sexcentesimus nonagesimus quartus. \mletter{D} Sed longe illa Sinarum prodigiosior, iuxta quam hic annus Christi 1594 est à conditu rerum octigenties octagies quater millesimus, septingentesimus septuagesimus tertius.

Bonziorum vero Iaponensium periodus annorum 470 desiuit cum anno Christi 1561. \& 1562 cœpit sequens. eiusque hic est vicesimus currens.

Ea vertente scelera extirpatum iri: reliquum tempus omnia pacata fore credunt.

Taceo diuersas Christianorum, Iudæorum, Samaritanorum de conditu rerum opiniones: item Romanorum lustrum qunque annorum, sæculum centum \& decem.

\end{parnumbers}
\clearpage
p. 20 [pdf 103]

\begin{parnumbers}

Sunt \& periodi Computatorum: vt Iudæa \mletter{A} annorum 6916, quæ constat cyclis Lunaribus 364, Solaribus 247, periodis magnis Dionysianis 13.

Habetque tot cyclorum septimanas, quot dierum septimanæ sunt in anno Solari: tot periodos Dionysianas, quot menses annus embolimæus: tot cyclos Solares, quot cyclos Lunares magnus cyclus Iudaicus.

Itaque elegantissima est, \& artificiosissima. eiusq; hic agitur annus 5354, anno Christi vulgari 1594.

Et inibit 1595 annus eiusdem proximo autumno, vnde omnes epilogismi neomeniarum Iudaicarum.

Periodus Dionysiana \& ipsa ad annalem computum pertinet, annis constans 532, ducto in sese vtroque cyclo.

Veræ quidem periodi magnæ caput incurrit in annum primum vtriusque cycli, pertinetque ad methodum Lunæ \& Solis. \mletter{B}

\& locum habet dumtaxat in anno Iuliano, hoc est in eo, cui præter 365 dies quadrans attibuitur.

Itaque eius initium est à Kal. Ianuariis in anno Romano: in anno Constantinopolitano à Kal. Septembris. in Antiocheno à Kal. Octobris. in Alexandrino \& Samaritano ab a. d. 1111 [I I I I ?]. Kal. Septemb.

Periodus vero Dionysij pertinet ad methodum neomeniæ Paschalis, initio sumto ab anno primo natalis Christi, vt ipse quidem putabat: item ab anno decimo cycli Solis Iuliani, \& ab ea neomenia, cuius quartadecima dies proxime post X X I, aut in X X I I Martij conficeretur.

Hactenus à minimis initiis ad summa temporum incrementa, quam \textgreek{[Greek]} Græci vocant, Chronologum perduximus, \& eum in conspectu totius antiquitatis collocauimus. \mletter{C}

Superest nunc, vt quæ carptim \& obiter perstrinximus, ea vberius suis locis expicentur.

Resumamus igitur eos annos, ex quibus tanquam elementis, ad tot tamque diuersa genera annorum progressus factus est.

Ex anno Græco, qui est æquabilis minor, omnes anni, Lunaris formas propagatas esse vidimus: vt ex Ægyptiaco, qui est æquabilis maior, omnes Solares.

Non igitur confuse, \& per saturam hæc tractanda, sed suo quæque \& loco \& ordine.

Quatuor igitur libris quatuor genera anni summa explicare decreuimus.

Primus erit de anno æquabili minore. Eo enim omnis Græcia vsa tam diuersis generibus, quam multæ fuerunt eius terræ nationes, \& \textgreek{[Greek]}.

Itaque ea erit reliqua pars huius libri.

Secundum locum sibi vindicat annus \mletter{D} Lunaris, quia ex illo priore deriuatus.

Tertius liber complectetur anni æquabilis maioris formas, \textgreek{[Greek]}, \& differentias.

Quartus illius anni traduces \& propagines persequetur, diuersa nempe anni Solaris genera, \& mutationes.

Hæc est pars prior, quam initio huius diatribæ.

Chronologo promisimus, de annorum \& temporum Ciuilium generibus.

\end{parnumbers}
\clearpage
p. 21 [pdf 104]

\begin{parnumbers}

Altera pars est de charactere, qui necessarius est notandis temporum interuallis, quæ sequentibus libris tractabimus, item diuersis \mletter{A} computis nationum annalibus, de quibus librum singularem ad calcem operis adiiciemus, non tanquam appendicem, sed partem vnam operis nostri.
Quis igitur sit vsus characteris temporum, docet nos Dionysius ex Ephoro, qui cum annum excidij Troiæ ex Olympiadum epocha notare non posset, cum is casus aliquot seculis antiquior sit prima Olympiade, dixit id accidisse eo anno Attico, quo viginti \textgreek{[Greek]} annum explebant.

Statim peritis anni Attici subolebat, quo anno id accidere potuerit.

Sciebant enim quoties in quanto interuallo annorum id fieri posset. Exemplo Ephori aut Dionysij erit nobis character excogitandus, quo animus anceps in triuio constitutus quæsitum ad fontem manu deducatur.

Erit igitur primum \mletter{B} totius instituti nostri fundamentum annus Iulianus, quem fingimus ante multa millia annorum fuisse.

Characteres vero illi duos dabimus, cyclum Lunæ Dionysianum, cuius hic est annus X V I I I.

\& cyclum Solis Iulianum, cuius hodie annus V I I currit.

Tertium etiam, vbi ratio temporum patietur, Indictiones non aspernabimur.

Nam qui his characteribus semel vti institerint, illi, quæ sit constantia, \& fides illius methodi pulcherrimæ in ratione temporum, experentur.

Si quis hoc anno Christi 1594 incertus, quot annos natus sit, tamen \& maiorem se quadraginta nouem annorum, \& minorem quinquaginta sex sciat, is imitatur imperitiam Chronologorum Græcorum, qui circiter illius, \& illius regis tempora illud, \& illud accidisse dicunt, annum \mletter{C} vero certum non difiniunt.

Sed cum idem adiicit natum se Nonis Augusti, feria quinta, is addit characterem certum \& indubitatum, quales sunt viginti \textgreek{[Greek]} Ephori.

Nam feria quinta non potuit incurrere in Nonas Augusti, nisi cum litera Dominicalis est C. Ante 49 autem annos id accidit anno Domini 1540, cyclo Solis nono.

Itaque hoc characterismo constantissime affirmanus eo anno hominem natum, \& proximis Nonis Augusti Iulianis illi quinquagesimum quintum natelem initurum.

Idem vsus cycli Lunaris, adhibita castigatione, vt à prima Olympiade, ad annum Domini 1400, tot dies neomeniis adhibeas, quoties 304 annos reperies.

Exemplum. hic est annus à prima Olympiade 2370.

In quibus annis septies reperitur \mletter{D} numerus 304. septem igitur dies neomeniis hodiernis adiiciendi.

Verbi gratia. anno primo cycli epactæ sunt X I. nouilunium martij X V I I I. additis V I I. diebus, nouilunium, vel potius coniunctio lumanarium erat in X X V.

Martij anno quarto anto primam Olympiadem, aut quintodecimo post eandem primam Olympiadem, \& deinceps ad 304 annos.

Sed ab hoc sæculo nostro post 150 annos minuendæ erunt neominiæ totidem diebus, quoties 304 anni reperientur post annum Christi 1700. \& fortasse citius.

\end{parnumbers}
\clearpage
p. 22 [pdf 105]

\begin{parnumbers}

Sed quia nullam epocham veterem certiorem Olympiadum capite habemus: illud autem \mletter{A} cum vetustate comparatum nouitium esse videtur: inutiles erunt characteres cyclorum \& Indictionis, nisi à quadam remotissima epocha initium temporum instituamus.

Excogitemus igitur periodum, quæ vtrunque cyclum, \& Indictionem contineat: quod fiet, si periodum Dionysij Exigui quindecies multiplicemus: qui fient anni 7980.

Ita periodus illa incipiet ab anno primo tum vtriusque cycli, tum Indictionis: \& proinde eiusdem vltimus annus desinit in vltimis vtriusque cycli, \& Indictionis.

Sed annus Christi, vt vulgo putamus, 3267 desinet in vltimum vtriusque cycli, \& Indictionis.

Ergo deductis 3267 de 7980 annis, relinquetur epocha anni ante vulgarem \mletter{B} Christi, nempe 4713.

Ita vt 4714 sit primus annus Christi vulgaris cyclo Solis X, Lunæ 2, Indictionis 4, à Kal. Ianuarij: quamuis \& Indictio autumno proxime antecedenti, Cyclus autem Lunæ Martio sequenti cæperit.

Quare annus iste, qui ex errore vulgi putatur 1594, est 6307. periodi huius, quam Iulianam vocamus, quod ad Iulianam anni formam accommodata sit.

Ideo 6307 diuisis per 28, per 19, per 15 habebimus huius anni 6307 periodi Iulianæ, vel vulgaris Christi 1594, cyclum Solis septimum à Kal. Ianuarij: Lunæ decimumoctauum à Martio sequente: Indictionis septimum Cæsarianæ quidem ab ante d. V I I I Kal. Octobris antecedentis anni 6306: Pontificiæ vero à \mletter{C} Kalendis Ianuarij anni propositi 6307.

Non prædicabo laudes huiusce periodi: Chronologi \& astrologi, qui omnia \textgreek{[Greek]} disputare volunt, non poterunt eam satis laudare.

Qui igitur eclipses ex Tabul[i]s Prutenicis putare volent, ex anno periodi Iulianæ auferant 2408.

\& cum residuo toto excerperant tempora epochæ diluuij.

Exemplum: Eclipsis Lunaris accidit in Septembri anno Olympiadico 446, qui est annus periodi Iulianæ 4383.

Deductis 2408, remanent 1975.

Excerpo primum 1900 ex epocha Diluuij: deinde 75, ex filo annorum expansorum.

Postremo menses vsque ad Septembrem.

Et reliqua vt ex methodo Prutenica.

Qui omne dubium ex temporum ratione tollere volet, vti debet hac periodo, sine qua nihil vnquam certi in natione \mletter{D} temporum adferre poterit.

\end{parnumbers}
