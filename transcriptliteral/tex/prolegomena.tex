% !TEX TS-program = xelatex
% !TEX encoding = UTF-8 Unicode
% this template is specifically designed to be typeset with XeLaTeX;
% it will not work with other engines, such as pdfLaTeX

%%% Count out columns for fixed-width source font
% 000000011111111112222222222333333333344444444445555555555666666666677777777778
% 345678901234567890123456789012345678901234567890123456789012345678901234567890

\chapter[Prolegomena]{}
\begin{center} \vspace{-18mm}
{\scshape
\head{3.0}{35}{PROLEGOMENA}\\ \vspace{6mm}
\head{1.5}{60}{IN}\\ \vspace{5mm}
\head{2.0}{50}{LIBROS}\\ \vspace{9mm}
\head{1.5}{60}{DE}\\ \vspace{6mm}
\head{3.3}{25}{EMENDATIONE}\\ \vspace{7mm}
\head{2.0}{40}{TEMPORVM}\\ \vspace{7mm}
} % scshape
\em{Ad candidum Lectorem.}
\end{center}
\normalsize

\setcounter{parcount}{0}
\begin{parnumbers}
\dropcapil{8}{Q}{uintusdecimus} hic annus agitur, candide Lector, postquam opus nostrum de Emendatione Temporum emisimus.
\\ \p
Persuaseram mihi, homines studiosos aliquam nobis gratiam habituros tot rerum, quas \& scitu dignas, \& a nobis primum indicatas negare non poterant.
\\ \p
Sed longe aliter animatos experti sumus: atque adeo rem potius inuidiosam atque obtrectationi opportunam, quam illis gratam me suscepisse intellexi.

Denique nihil aliud quam significarunt, quiduis potius se ignorare malle, quam a nobis aliquid discere.

In quibusdam candorem, in aliis studium, in omnibus sensum bonarum rerum desideraui.

Nos vero, qui nihil unquam prius habuimus, quam ut horum orationes sinamus praeterfluere, modo verum eruere, \& inimicos nostros etiam inuitos iuuare possimus, opus nostrum iterum in manus sumptum auximus, illustrauimus, emendauimus, vt, quanuis idem sit, aliud tamen a nova cultura videri possit.

Quæ huic editioni accesserunt, haud promptum est dicere.

Sed in quibus a priore demutat, postea intelliges, siquidem instituti nostri rationem aperuero.

Subiectum operis nostri est ratio Temporum civilium, \& eorum, quæ in vetustatis cognitione versantur: finis, Emendatio: quod quidem me tacente, \& Titulus ipse promittit.

Ciuilium temporum cognitio, eorumque historia, vertitur in multiplici diversorum annorum forma \& eorum methodis vulgaribus, quos Computos posterior ætas vocauit.

\textgreek{Τα ιςορομγυα[?]} civilium temporum habes in primoribus tribus libris, \& maiore parte quarti: methodum autem in septimo.

A emendationis duæ partes sunt.
%\end{parnumbers}
\clearpage
p. II [pdf 29]
%\begin{parnumbers}
Prior versatur circa epocharum investigationem, posterior circa verum annum tropicum, 
\& periodos Lunares: quam materiam posterior pars libri quarti, item toti quintus \& sextus sibi vindicant.

Iam quemadmodum Enochæ sunt notationes, \& tituli temporum; ita ipsarum epocharum quædam debent esse propria \textgreek{γνωρίσματα} \& characteres: quorum characterum alij sunt naturales, alij ciuiles. 

Naturales quidem a rationibus utrisque sideris, unde nati cycli Solaris, \& Lunaris: civiles ab instituto, cuiusmodi indictiones \& anni Sabbatici: sine quibus in harum rerum tractatione omnis conatus irritus. 

Rursus \& eorum quoque fallax usus est, nisi quædam annorum ex illis periodus instituatur.

Sed eæ sunt totidem, quot aut formæ annorum, aut civilia initia.

Nam in anno Ægyptiaco Nabonassari alia opus est, ac in anno Solari, quia diversa forma: item in anno Actiaco siue Diocletianeo alia, ac in Iuliano, propter diversa initia.

In anno Ægyptiaco vago naturales characteres sunt \textgreek{εἰκοσιπεν σαετηρις[?]} Lunaris, \& \textgreek{έπταετηρις[?]} Solaris: civilis autem character est quadriennium, quem canicularem annum minorem vocabant Ægyptij.

Hi tres characteres in se ducti producunt periodum magnam annorum 700 Ægyptiacorum: qua uti debet disputator temporum, siquidem rationes suas ad annos Nabonassari, Armeniorum, aut Persarum exigit.

At qui anno Iuliano, quæ omnium formarum temporibus est convenientissima, uti volet, is cyclo utriusque fideris quindecies ducto componet elegantissimam periodum annorum 7980, cuius initium in cyclo Solari, \& Indictione Romana, a Kal. Ianuarij, in cyclo Lunari a Martio, in anno Sabbatico ab autumno.

Itaque non minus utilis, quam necessaria est.

Sine ea nihil agit Chronologus: cum ea tempori, \& sæculis imperat.

Quam enim lubricum sit retro ab aliqua epocha notare tempora, quod maior pars doctorum virorum facit, satis nos usus docuit.

His ita positis, ad singula huius operis membra venio.

Libro primo præter divisionem temporum, \& iucundissimam mensium, \& annorum historiam, de antiquissima anni forma disputatur, quæ in menses æquabiles annum describit, qua pleraque omnes Græcia usa est, \& ab ea omnis ratio Olympiadum pendet: nisi potius eam e ratione Olympiadum propagatam dicas: quod sine cognitione Olympiadum numquam tam eximium vetustatis \& 
temporum monimentum in lucem eruissemus. 

Ex tanta autem Græcorum scriptorum copia unicus Pindarus nobis facem alluxit, qui solus nos docuit tempus ludicri Olympici.

Aliter, quæ paucitas est bonorum scriptorum, nulla erat via ad hæc interiora perveniendi.

Huius anni Græci formæ doctrina tanto acceptior esse debet, quanto obscurior eius rei apud maiores nostros scientia fuit: cum ante hos mille quadringentos plus minus annos eius rei neque volam, neque vestigium vetustas retiuerit.
%\end{parnumbers}
\clearpage
p. III [pdf 30]
%\begin{parnumbers}
Nam falso veteres multi, ac post eos infamæ 
antiquitatis scriptores, Macrobius ac Solinus, atque proauorum memoria summus vir Theodorus Gaza, annum Græcorum statim ab initio merum Lunarem fuisse prodiderunt.

Quamuis enim in Panegyribus suis, ac nobilioribus sacris, quæ certo annorum circuitu redibant, vnius Lunæ rationem habebant, tamen, vt vno verbo dicam, eorum anni forma Lunaris non erat.

Olympicum enim ludicrum ipsa Luunæ plena lampade celebrabatur, vt solus veterum nos docet Pindarus. 

Prætera Laconibus ante plenilunium, aut nouilunium aliquid incipere religio erat.

Vnde \textgreek{Λαχώνιχας ςελωοαζ}[?] vicinorum prouerbio iactatas, \& contra Arcadibus prouerbiali conuicio neglectum religionis obiectum legimus. 

Quod enim ante nouilunium, aut plenilunium vt plurimim bella aut alia seriora aggrederentur, ob eam rem a finitimis nationibus [Græca] vocabantur: quæ conuicij caussa ab ipsis Arcadibus interpretatione elusa est, probrum in laudem cõuersum ad vetustatem originis suæ referentibus, \& antiquiorem sidere gentem suam gloriantibus. 

Quod igitur nouilunij ac plenilunij tempora Panegyricis ludicris deligebant, propterea sacra trieterica instituta : cuiusmodi erant orgia Bacchi, Nemea, Isthmia, alia.

Ea enim est anni Græcanici forma, vt si, verbi gratia, nouilunium in neomeniam Gamelionis incurrat, plenilunium in eandem neomeniam incidat anno tertio redeunte.

Itaque cum in Tetraeteride orgia Bacchi trieterica celebrabantur, tertio anno redibant in eum sistum Lunæ, qui priorum orgiorum situi oppositus erat.

Quare elegantissime Statius trieterida vocat alternam : quia alternis in nouilunium, \& plenilunium incurreret.

At sacra, quæ necessario eodem Lunæ tempore obibantur, ea semer erant tetraeterica : vt in Attica Panathenaica maiuscula, in Elide Olympias, vt iam tetigimus, plenilunio.

quod sane fieri non poterat, nisi absoluta Tetraeteride, \& Pentaeteride ineunte.

Atque ita Tetraeterides in idem \textgreek{χῆμα} Lunæ, non vtique in idem tempus Solis redibant.

Vt enim in orbem Solis \& Lunæ redirent, non aliter putabant fieri, quam octaeteride confecta, eneateride ineunte.

Ex quo quædam eneaeterica sacra eo nomine instituta: cuiusmodi ab initio Pythia suerunt: \& quidem merito.

Apollini enim, quem eundem cum Sole faciebant, erant attributa.

Hinc colligimus, non solum Olymiadis interuallum annis quator solidis explicatum fuisse; sed etiam puerliter peccare eos, qui annorum quinque solidorum fuisse putant.

Neque vero quibusdam recentioribus succensendum, qui ita censent, ita scribunt, sed \& Ausonium nostratem culpa liberat Ouidius, scriptor longe antiquior, \& nobilior, qui ætatem suam quinquaginta annorum decem Olympiadibus definit:
%\end{parnumbers}
\clearpage
p. IV [pdf 31]
%\begin{parnumbers}
quo magis mirum Pausaniam hominem Græcum in ea hæresi fuisse, ve suo loco a nobis relatum est.
Nam minus mirandum de Solino, qui cap. \textsc{xiii} Isthmia vocat quinquennalia, quæ erant tantum triennalia, quod certamen a Cypselo tyranno intermissum anno primo Olympiadis 49 instauratum fuisse dicit.

Horum igitur omnium caussæ ad typum anni Græci referendæ sunt.

In quo argumento nihil eorum prætermisimus, quæ ei rei illustrandæ faciebant, quanquam pene omnibus præsidiis destituti.

Et quidem primum in genere, quod semper solemus, deinde priuatim multarum Græciæ nationum periodos proposuimus, quæ quidem non anni forma, sed situ \& capite inter se differunt: in qua tractatione diu nobis res fuit cum præstantissimo viro Theodoro Gaza, vel potius cum eius sequacibus, a quibus extorqueri non potest doctrina \& situs mensium ab illo primum 
proditus. 

quæ quidem velitatio nobilioribus ingeniis, \& ab omni inuidia remotis, vt spero, iucunda erit.

Quid enim est toto libro primo, cuius vel minima pars non dicam istis querulis, qui nihil sciunt, sed etiam doctioribus, hoc sæculo, \& ante multa retro sæcula oboluerit?

Quid dicam \textgreek{[Greek]}?

quis illarum caussas, \& vsum sciebat?

quis locum nobilem de illis in Verrina Ciceronis intelligebat?

quis locum \textgreek{Εξαιρέσεως[?]} in secunda Boedromionis?

quis Posideonem intercalarem mensem fuisse?

Huic materiæ accessit \textgreek{ἐποχη χέντςα[?] θερινᾶ} in \textsc{viii} Iulij, quæ in priori editione omissa erat.

Id erat \textgreek{χάντσον[?]} populare, quod nomine \textgreek{τςοπων[?] θερινῶν[?]} Aristoteles, Thophrastus, Plutarchus, \& omnes veteres intelligunt, non autem ipsum verum Solstitium : quæ rei pulcherrimæ notatio nobis viam ad illustriora præiuit.

Quod Solstitiorum, \& æquinoctiorum puncta \textgreek{χέντρα} vocentur, satis sciunt, qui veterum Græcorum libros legerunt.

Columella cardines vocauit.

In præstantissimo Parapegmate, quod falso Ptolemæo attribuitor (est enim antiquius Ptolemæo) ad \textsc{viii} Kalen. Iulij (quod est Solstitium Sosigenis) annotatum est: \textit{Æstiuus cardo, \& momentanea aeris perturbatio}.

In Græco (vtinam haberemus!) sine dubio fuit: \textgreek{Θερινὸν χέντρον, χαὶ ςιγμιαία αἔσος Ιασοιχή[?]}.

Igitur \textgreek{χέντρον θερινὸν} nihil aliud, quam \textgreek{τροπαὶ θεριναί}.

Cur \textsc{viii} dies Iulij erat epocha æestiua in vsu ciuilis anni, non semel caussam reddidimus. 

Adiecta etiam pernecessaria neomeniarum Atticarum Tabula : quæ non solum priori editioni, sed etiam doctrinæ anni Attici deerat.

Liber secundus anno Lunari dicatus est ideo, quia is annus ex illo Græco æquabili manasse videtur.

Ibi aperitur omnis antiquitas \textgreek{ἔτοις[?] πρυτανείας[?]}, Octaeteridon Cleostrati, Harpali, \& Eudoxi.

quæ omnia hodie nomine tenus nota erant.

Eudoxea Octaeteris nunquam in vsus ciuiles admissa est.

Anni vero \textgreek{πρυτανείας[?]} in vetustissimis Psephismasin Atheniensium primo quidem ex Cleostratea, deinde, illa abrogata, ex Harpalea petiti sunt.
%\end{parnumbers}
\clearpage
p. V [pdf 32]
%\begin{parnumbers}
Sequitur magnus annus Metonicus
ambarum, \& Calippicus Metonici castigator.
Et quidem hi ambo nomine noti tantum: caussarum autem, \& omnium, quæ ad illa pertinent, mira ignoratio hactenus fuit.

Accesserunt huic editioni Tabulæ operosissimæ dispensationum neomeniarum Metonicarum, \& Calippicarum: cuiusmodi etiam in Harpalea Octaeteride exhibuimus. 

Quod de Eudoxea Octaeteride diximus, idem de periodo Chaldæorum dicendum, eam nunquam ad ciuilia tempora, sed ad Genethliacorum themata vsurpatam fuisse.

Id quod tum multa argumenta, tum vnicum certissimum illud est, quod eorum menses appellationibus Macedonicis, non vero Chaldaicis fuerunt.

Propterea recte cum illius anni diatriba doctrinam dodecaeteridis Chaldaicæ Genethliacorum coniunximus, cuius nomen quidem solum notum erat ex Censorino: cognitio autem nobis ex Arabum, \& Orentalium vsu repetenda fuit.

An aliquis Græcorum \textgreek{δωδεχαετρίδος Χαλδαιχῆς} meminerit, haud promptum est dicere.

Vnum tantum Orpheum siue Onomactitum eius meminisse scimus. 

\textgreek{ὀρφδὶςὀν[?] ταῖς[?] δωδεχαετνρίσιν[?]:}

\begin{greek}
ἔςαι δ᾽ αὖθις ἀνὴρ, ἢ χοίραν[G-circ], ἠὲ τύρανν[G-circ],

ἢ βαοιλδὶς, ὂς τῆμ[G-circ] ἐς οὐρανὸν ἴξε[right curl] αἰπιιύ [all doubtful].
\end{greek}

Est apotelesina cuiusdam Genethliaci consulti super alicuius genesi, de quo ipse respondit, eum fore magnum regem aut Dynastam, \&c. Citat Tzetzes. 

Hæc multum illustrant doctrinam Dodecaeteridos genethliacæ parum antehac notæ.

Itaque quemadmodum \textgreek{τελετὰς} ita etiam \textgreek{δωδεκαετνρίδας[?]} scripserat Onomacritus sub nomine Orphei.

Qualis fuerit Iudæorum annus sub Seleucidis, quibus parebant, multis exemplis testatum reliquimus: in quibus etiam translationis feriarum in capite anni antiquitatem asseruimus aduersus homines nostrorum temporum, qui nugantur commentum nuperum iudæorum esse.

In illis Doctor Theologus ingenti commentario suo in Euangelium secuntum Iohannem exultabundus ait illam translationem confutari ex loco Iosephi, in quo scribit, quo anno Hyrcanus fœdus icit cum Antiocho Sidete, Pentecosten fuisse feria prima.

Hunc locum Iosephi nos olim in priore editione produximus, vnde is, aut qui illi indicauie[?], accepit.

En, inquit, duo Sabbata continua.

Si propter continuationem duorum Sabbatorum, feria transfertur, ergo vbi sunt duo continua Sabbata, non transfertur. 

In quibus aperte ostendit se ignorare caussam feriæ transferendæ, quæ fiebat propter solum Tisri, non autem propter alios menses; propterea quod ille mesis multa solennia habet, adeo vt si non habeatur ratio translationis, aliquando non solum duo, sed entiam tria continua sabbata concurrere necesse sit.
%\end{parnumbers}
\clearpage
p. VI [pdf 33]
%\begin{parnumbers}
Si enim feria sexta inciperet neomenia Tisri, omnino tria sabbata continuarentur, neomenia,
siue cangor tubæ, sabbatum ordinarium, ieiunium Godoliæ.

Continuantur autem Sæpernumero in aliquo reliquorum mensium duo Sabbata: idque fit, quando solenne est aut feria prima, aut feria sexta.

quorum alterutrum quotannis incidere, nisi quando Tisri incipit feria tertia, Doctor ignorauit.

In pimam feriam incidunt hæc solennia, \textsc{xxv} Casleu, \& \textsc{x} Tebeth in anno defectiuo tam communi, quam embolimæao, quotiescunque Tisri incipit feria secunda: \textsc{vi} Siwan; quando Nisan incipit feria septima: \textsc{xv} Nisan, \textsc{xvii} Tamuz, \textsc{ix} Ab, quando Nisan incipit feria prima.

In feriam autem sextam conuenit solenne \textsc{xxv} Casleu \& \textsc{x} Tebeth, quando Tisri est feria septima in anno tam communi, quam embolimæo.

\textsc{xiiii} Adar, quando Nisan sequens est feria prima: \textsc{vi} Sivvan, quando Nisan feria quinta.

Vides, quot Sabbata quotannis, nisi quando Tisri incipit feria tertia, Iudæi continuent in aliquo mensium, præterquam in solo Tisri, cuius vnius gratia illa cautio instituta.

Itaque doctor tam frustra, quam ridicule Iosephi testiminium adduxit de sexta Sivvan, id est Pentecoste feria prima; cum illo anno neomenia Nisan fuerit Sabbatum.

Atqui nihil superesse putauit, quam vt Vaticani montis imago redderet \textgreek{ἰὴ παιαή[?]}.

Sed ipse valde ignarus est harum rerum, vtreliqui omnes, qui contendunt nouitium esse Iudæorum commentum.

Nos validissime demonstrauimus, \& sæculo Christi, \& retro sub Seleucidis translationes in vsu fuisse.

\& sane res peruetusta est.

quæ tamen non minus ignorata, quam periodus Calippica, qua Seleucidæ, \& Seleucidarum edicto Iudæi vsi.

quod non solum ex Nisan anni excidij Hierosolymorum a nobis demonstratum est, sed etiam patet ex definitione Rabbi Adda.

Is annum definit dierum \textsc{ccclxv}, horarum 5, 997/1080. 48/76.

Quid hac definitione aliud vult, quam periodum Iudaicam fuisse annorum 76?

Cum Meto definit annum dierum 365. hor. 5. 1/19. ex eo coniiciendum relinquit, se vti periodo annorum 19.

Vtebantur igitur periodo 76 annorum, id est, Calippica: \& tamen in omnibus neomeniis Lunæ \textgreek{φάσιν} obseruabant, non quod eam ex præscripto periodi non indicerent, sed ideo, vt eam sanctificarent.

Nam \& hodie quoque obseruant \textgreek{φάσιν}, non vt ex ea neomeniam indicant, sed vt eam sanctificent.

Itaque Luna statim visa dicunt: \texthebrew{[Hebrew]}.

\textgreek{ἀγαθὸν τέρας ἔςω ἡμῖν ης[??] παντὶ Ισραήλ.}

Idem faciunt \& Muhammedani, quamuis neomenias ex scripto indicere soleant.

Neque aliud intellexit fabulosus quidem, sed tatem vetus auctor \textgreek{[verbi Græci]} apud Clementem: \textgreek{[multibus verbi Græci]}.
%\end{parnumbers}
\clearpage
p. VII [pdf 34]
%\begin{parnumbers}
Præclara quidem ista: sed nescit, quid dicit.

Nam in Iudæorum potestate nunquam fuit, vt exspectarent \textgreek{φάσιν}: quia raro Luna se ostendit, nisi secundo post coitum die.

Quod si expectandum ipsis esset, res ridicula accideret, vt Elul, qui semper est cauus mensis, non solum plenus, sed etiam aliqando vnius \& triginta dierum esset.

Sine dubio translationem feriæ intelligit, cuius caussam ignorat.

\textgreek{[Greek]} vocat \texthebrew{[Hebrew]} caput anni.

Nam Sabbatum vocat, quia Festus dies, \textgreek{[Greek]}.

Ita etiam vocatur Leuitici \textsc{xxxii}, 24.

\textgreek{[Greek]} intellige \textgreek{[Greek]}: quod ita Hebraice vocetur, nempe \texthebrew{[Hebrew]}.

Vide in Computo Iudaico.

At \textgreek{[Greek]} vocat \textgreek{[Greek], [Greek]} quoque, id est \texthebrew{[Hebrew]}.

Nam aliæ erant \textgreek{[Greek]}, proinde vt \& \texthebrew{[Hebrew]}.

Sic Tertullianus magnos dies dixit, quos Hebræi \texthebrew{[Hebrew]} vel \texthebrew{[Hebrew]}.

Eius verba sunt ex v in Marcionem: Dies obseruatis, \& menses, \& tempora, \& annos, \& Sabbata, vt opinor, \& cenas puras, \& ieiunia, \& dies magnos. 

Sed quid Tertullianum aduoco?

Ecce Biblia Græca ita vertunt ex primo cap.

Isaiæ: \textgreek{[Greek], [Greek], [Greek]}.

Quod Hebraice est \texthebrew{[Hebrew]}, vertunt \textgreek{[Greek]}, quod idem ist quod \texthebrew{[Hebrew]}: \& quidem manifesto Sabbata distinguuntur a magnis diebus. 

Quare perperam quidam \textgreek{[Greek]} interpretantur Sabbatum apud Iohannem, \textgreek{[Greek]}. de quo infra.

Quin \& Tertullianus ipse \textgreek{[Greek]}, quas cenas puras vocat, a diebus magnis, \& a ieiuniis, \& a Sabbatis distinguit. De Cena pura, præter id quid diximus ad Festum, ita reperi in veteri \& peroptimo Glossario Latinoarabico : Parasceue, cena pura, id est, praparatio, que fit prosabbato.

Conditor Annalium Ecclesiasticorum turbat de cena pura, \& negat esse parasceuen, quia cena pura apud Festum habeat offam suillam.

Sed ipse, (pace docti viri dixerim) non aduertit Puram dici, non quia careat carnibus, sed quia religionis \& dicis caussa fit.

Nam \& parasceuæ Iudaicæ habent carnes, \& nihilominus dicuntur cenæ puræ, quod dicis caussa coquebantur, coquunturque hodie prosabbato, quia in Sabbato coqui non liceat.

Non negabis, candide Lector, hæc vulto non intelligi.

Itaque locus ille est nobilissimus. 

Tamen quotus quisque est ex tot Lectoribus, qui non hæc aut præteribit, aut calumniabitur?

Sequuntur periodi magnæ Hagerenorum, ex quibus ratio anni soluti Indorum, \& Muhammedanorum tota pendet.

Omnia nunc primum ex Arabum scriptis prodeunt: atque adeo omnis tractatio nostris hominibus noua est.
%\end{parnumbers}
\clearpage
p. VIII [pdf 35]
%\begin{parnumbers}
Excipit hanc doctrina anni Iudaici hodierni, res, quod sæpe diximus, artificiosissima, ideoque eximia, quia melior
anni Lunaris forma constitui non potest.

Docemus præterea, vnde natus sit ille annorum computus, quo vtuntur hodie, a \textsc{vii} Octobris: quem inepte putant a conditu rerum.

Post multarum Periodorum, Cyclorum, Octaeteridum, Paschalium historias, in locum vltimum \textgreek{[Greek]} veteris anni Romanorum coniecimus, ideo quod ea forma proxime abesset a Lunari: vbi de sæculo Romano, \& capite veteris anni Romani, temporibus vltimis C. Iulij Cæsaris, multa accuratissime disputata.

Itaque ex singulis rebus singula capita confecimus, cum potius singuli libri \& quidem ingentes confieri possent, si, quæ hominum hodiernorum est ambitio, eadem nobis incessisset.

Tertio libro opportune annus æquabilis datus est, cum annus Solaris Ægyptiacus, adscitis diebus quinque, ex Græco propagatus sit: (quemadmodum annus Lunaris ex eodem Græco manauit, abiectis ab eo totidem diebus cum horis \textsc{xv}, paulo amplius) quod, metacente, Plutarchus docuit in libro \textgreek{[Greek]}.

Adeo inter se libri nostri mutuo conspirant, neq; ab eis ratio, methodus, \& ordo abest.

In eo libro de Neuruz antiqui Persarum periodo annorum \textsc{cxx}, deq; cognominibus dierum Persicorum, de translatione \textgreek{[Greek]} in enthronismis nouorum Regnum, item de caussis anni Iezdegird, de annis Armeniorum, \& eorum mensibus, omnia noua protilimus. 

Sed hæc non expergesacient animos hominum, nisi forte ad obtrectandum.

Quartus liber est emendatio tertij, vt secundus primo erat subsidiarius: qua methodo imperfectus Lunaris Græcus libro primo disputatur, vt perfectus secundo.

Sic etiam perfectus Solaris, \& siqui alij naturam perfecti immitantur, supplent id, quod æquabili Ægyptiaco, Persico, \& Armeniaco vetuitas detraxerat.

Itaq; in quatuor partes tribuendus fuit.

In prima continentur anni, quibus quarto anno exeunte dies ex quatuor quadrantibus conflatus accrescit.

Ex illis nobiliores selegimus, Iulianum, Actiacum, Antiochenum, Samaritanum, \& alios. 

Nam \& alios quoq; eius formæ habebamus, vt Tyriorum, quorum menses appelationib. Macedonicis, diuersa initia a Iulianis habent.

Sic etiam Gazensium annus mere Actiacus fuit, appellationibus mensium Macedonicis, mensibus tricenariis. 

Marcus Ecclesiæ Gazensis Diaconus, in actis Porphyrij Gezensis Episcopi vocat \textgreek{Διον} Nouembrem \textgreek{Απελλαιον} Decembrem quæ nomina habent a Macedonibus. 

Sed idem scribit Gazenses celebrasse Theophaniorum diem vndecima Audynæi, quæ est sexta Ianuarij Iuliani, se autem redisse Constantinopoli, Xanthici vicesima tertia, quam ait fuisse decimam octauam Aprilis secundum Romanos: quiebus ostendit formam illius anni mere Actiacam fuisse, mensibus tricenariis, appellationibus Macedonicis. 
%\end{parnumbers}
\clearpage
p. IX [pdf 36]
%\begin{parnumbers}
Seconda pars annos emendatos, eorumque emendandorum
rationem complectitur: tertia periodos multiplices, quarum finis conciliatio anni ciuilis cum Solari, cui dies quinto quoque anno ineunte accrescit.

Quarta pars agit de vera emendatione anni, \& de anno cælesti instituendo, qui pertinet ad methodum epochæ mundi.

Quemadmodum autem nulla Lunaris anni ciuilis ratio recta iniri potest, præter eam, qua Iudæi vtuntur: ita nullus annus cælestis Tropicus recte institui potest, nisi ex forma, quam edidimus, quam nemo vituperabit, nisi qui ignorauerit; omnis laudabit, qui intellexerit.

Alioquin scio \& malignos \& obtrectatores non defuturos. 

Annus tam noster, quam Iudaicus ciuilis quidem, sed naturalis, vterque ad motum quisque sui sideris descriptus. 

Ideo eius saltem in scriptis usus esse debet, qualis olim Philadelphi Dionysianus, Chaldæorum Calippicus, hodie Persarum Gelaleus. 

Tres igitur libri primi, \& prima pars quarti pertinent ad \textgreek{[Greek]} temporum ciuilium cum septimo.

At reliquæ tres partes quarti cum duobus libris sequentibus pertinent ad ipsam emendationem temporum.

Atque vt a munti primordiis omnes res deducuntur, ita mundi epocham primam ordine posuimus: qua in re quam pueriliter hallucinati sint omnes, non sine admiratione tam imperitiæ quam pertinaciæ eorum dicere possum.

Non loquor de iis, qui sæculo vno, aut pluribus altius originem rei repetunt.

Nam quemadmodum ij nullam rationem sibi propusuerunt, quam sequerentur, ita nullos lectores nancisci possunt, nisi imperitos. 

Qui intra sæculum maiorem mundi epocham faciunt, eorum duo genera reperio.

Prius genus est eorum, qui solutionem captiuitatis in primum annum Olympiadis \textsc{lv} conferunt: alterum eorum, qui tempus illud \textsc{xviii} aut \textsc{xix} annis ante \textsc{lxiiii} Olympiadem definiunt.

In priori hæresi fuerunt \& quidam veterum Ecclesiasticorum, vt alicubi indicauimus. 

Aiunt Cyrum cæpisse imperare primo anno Olymiadis \textsc{lv}, hoc est 217 anno Iphiti, quod verum est: de quibus deductis septuaginta, relinquitur annus excidij Hierosolymorum, \& casus Sedekiæ 147 a primo ludicro Olympico.

Sed puerilis sententia multis absurditatibus eluditur.

Primo computatione non recta annorum \textsc{lxx} a capto Sedekia.

Deinde quod Cyrum statim initio regni sui Regem Mediæ, Persidos, Susidos, Assyriæ, Babyloniæ, totius Asiæ minoris, Indiæ, totius Syriæ constituunt, qui vnius Presidos Rex fuerit aliquot annis ante casum Astyagis, \& post illud tempus pauculis annis ante obitum Babylone potitus sit.

Hæc sola absurditas facit, vt non solum eorum nulla ratio habeatur, sed vt ludibrium quoque debeant.

Tertio 147 annus Iphiti est 118 Nabonassari: qui erat annus quintus ante initium Nabopollassari patris Nabuchodonosori.
%\end{parnumbers}
\clearpage
p. X [pdf 37]
%\begin{parnumbers}
Ergo Nabuchodonosor anno decimono regni sui templum \& Hiersolyma euertit annis quinq;
ante quam pater ipsius, cui ipse successit, regnaret.

Digna profecto talibus doctoribus sententia.

Tamen tantum abest, vt hac tam insigni absurditate a sententia desistant, vt animos ab eiusmodi portentis opinionum sumant.

Postremo ignorant diuersa esse initia Regnum, vt ipsius Nabuchodonosori, cum patre, \& solius: Alexandri, ab excessu Philippi patris, \& ab initio Seleuci: Diocletiani, ab æra martyrum, \& a primo anno imperij.

Sic etiam Cyri, apud Græcos, ab initio regni Persidis: apud Babylonios, vel a subacto toto Babyloniæ imperio, vel ab aliquo insigni facto, quodcunque illud fuerit, siue ex edicto ipsius Cyri, siue translatione \textgreek{πδν ἐπα γο υ[?]ων}, vt solebat fieri.

Qui tantam inscitiam sequi noluerunt, non tamen rectam viam institerunt, quia quindecim aut amplius annis ante \textsc{xlvi} Olymiadem casum Sedekiæ coniiciunt.

Nos ante annum quartum illius Olympiadis id non potuisse accidere ita demonstramus. 

Ezekias Rex Iuda, postquam singulari Dei beneficio ab ancipiti morbo conualuisset, anno \textsc{xiiii} regni sui, accepit Legatos \& \textgreek{ςωτήρια[?]}, a Merodach Rege Chaldæorum.

Ponamus \textsc{xiiii} annum Ezekiæ in primo anno Merodach, hoc est, in \textsc{xxvii} Nabonassari.

Nam is est annus primus Merodach apud Ptolemæum ex Chaldaicis obseruationibus. 

Hoc modo annus primus Ezekiæ conuenerit in annum \textsc{xiiii} Nabonassari. Ab initio Ezekiæ, ad excidium templi, anni sunt absoluti 138.

Hoc est, annus ipsius excidij est 139 labens ab initio Ezekiæ.

quod ita demonstramus. 

Annus primus Sedekiæ est quartus Hebdomadis, teste Ierimia, initio cap. \textsc{xxviii}: \& proinde vndecimus, qui \& vltimus, est Sabbaticus. 

de quo extat testimonium apud Ieremiam, \& nemo dubitat.

Rursus annus tertius decimus Ezekiæ erat Sabbaticus. 

auctor Isaias \textsc{xxxvii}, 30.

ex quibus manifesto colligitur, \textsc{xiiii} Ezekiæ esse primū Hebdomadis, \& primum Ezekiæ esse sextum Hebdomadis. 

Ergo annis ab initio Exekiæ vnitas addenda, ad methodum anni Sabbatici.

Addita vnitate annis 139, numerus erit septenarius. 

Quare annus labens 139 est verus annus ab initio Ezekiæ.

Quibus additis 13 annis Nabonassari præteris (quia posuimus 14 Nabonassari primum Ezekiæ) componitur annus Nabonassari 152, in quo casus Sedekiæ ex hac hypothesi locādus est, hoc est, in anno periodi Iulianę 4118: de quibus deductis 907 absolutis ab Exodo, remanet annus Exodi 3211 in periodo Iuliana.

Porro Nisan Exodi cæpit feria quinta, vt toties diximus, \& ex Mose rectissime ante nos Iudęi docuerūt.

At in anno periodi Iulianę 3211 Nisan nō cæpit feria quinta, sed feria tertia, Martij \textsc{xi}, cyclo tā Solis, quā Lunæ \textsc{xix}.
%\end{parnumbers}
\clearpage
p. XI [pdf 38]
%\begin{parnumbers}
Ergo annus proximus, quo Nisan cæpit feria quinta, is debuit saltē esse annus Exodi: atq; adeo is fuerit annus periodi Iulianæ 3214: in quo sane nisan cæpit feria quinta, Aprilis \textsc{vi},
cyclo Solis \textsc{xxii}, Lunæ \textsc{iii}.

Additis 907 annis absolutis ab Exodo,
annus 4121 periodi Iulianæ suerit is, in quo excidium templi contigit:
qui est quartus Olympiadis 46, vt erat propositum.

Sed \& post
Olympiadem 46 ponendum esse casum Sedekiæ ita probabimus.

Amasis rex Ægypti, postquam regnasset annos 55, obiit circiter annum
7 Cambysis, anno ante excessum ipsius Cambysis, hoc est, anno
225  Nabonassari.

Nechao intersectus est a Nabuchodonosoro anno
quarto Ioiakim regis Iuda.

Ieremias \textsc{xlvi}, 2.

Post eum regnauit
Psammitichus annos \textsc{vi}.

Cui Aprias, cuius meminit Ieremias
\textsc{xliiii}, 30, succedit.

Is post \textsc{xxv} annos relinquit regnum Amasi.

Summa annorum a cæde Nechao ad obitum Amasis anni 86, qui
deducti de 225, relinquunt annum Nabonassari 139, quartum Ioiakim
Regis Iudæ, primum Nabuchodonosori.

Ergo Sedekias captus
anno 158 Nabonassari, qui erat tertius 47 Olympiadis.

Idque verum
esse postea validissime demonstrabimus.

Diodorus Siculus,
auctor omnium Græcorum certissimus, attribuit, \textsc{lv} annos Amasidi. reliquos Apriæ \& Psammatichi habemus ex Herodoto.

Temere igitur, \& imperite faciunt, qui casum Sedekiæ antiquiorem illo tempore constituunt: neque his cassibus sese explicare poterunt,
quantumuis sua commoueant sacra, vt Plautus loquitur.

His valide
demonstratis, \& licentia chronologorum intra aliquos fines summota
quos amplius migrare non possunt, ad originas ipsas penetremus.

Sed prius vt in Mathematicis cōcessa quædam, aut quæ negari
non possunt, assumuntur, ita \& nobis quoque faciendum.

Tempora \& initia Regum Babyloniæ a Chaldæis notata in obseruationibus
eclipticis, quæ reiicere \& damnare extremæ impudentiæ \&
inscitiæ est: item, eorundem regum initia \& tempora a Beroso Chaldæo,
qui minus quam tribus sæculis post illos vixit, \& qui quæ Actis
ac fastis Babyloniorum publicis continebantur, ignorare nō potuit,
hæc inquam, non tantum tāquam vera haberi postulamus, sed etiā
qui aliter putant, tanquam indignos censeri, qui aut audiri a nobis
mereantur, aut vllas literas attingant, aut aliquem locum inter
doctos habeant.

Tricesimum annum, cuius initio Prophatiæ suæ
meminit Ezekiel, quique capti Iechoniæ quintus erat, Iudæi inepti
deducunt a libro legis reperto, anno \textsc{xviii} Iosiæ Regis.

Quis vnquam a libro reperto vllam æram, aut edicto Iosiæ institutam, aut a
Prophetis vsurpatam legit?

Si tanti erat illa temporis nota, quare
eam nō vsurpat Ieremias, qui tā accurate annos Regum Iuda Iosiæ,
Ioiakim, Iechoniæ, Sedekiæ notare solet?

Capite \textsc{xxv}, quare dicit
anno quarto Ioiakim, cum dicendū esset vicesimo secundo a libro
inuento?

Esto, cur Ezekiel dixit tricesimo, non tricesimo a libro inuento?

qui tamen dixit anno quinto deportationis Regis Ioachin.

[Prolegomena continues up to page LII]

\end{parnumbers}
