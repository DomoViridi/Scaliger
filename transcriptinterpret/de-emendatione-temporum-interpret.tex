% !TEX TS-program = xelatex
% !TEX encoding = UTF-8 Unicode
% this template is specifically designed to be typeset with XeLaTeX;
% it will not work with other engines, such as pdfLaTeX

%%% Count out columns for fixed-width source font
% 000000011111111112222222222333333333344444444445555555555666666666677777777778
% 345678901234567890123456789012345678901234567890123456789012345678901234567890

%% use [draft] for a draft version with boxes instead of images
%% Paper size: a4paper (modern size) or legalpaper (close to original)
%% Original size = 217.6 x 338.3 mm
%% Legal size    = 215.9 x 355.6 mm
\documentclass[12pt,twoside,a4paper]{book}

%%% Geometry package to make the margins better match the original
%% use [showframe] to show the frames
%\usepackage[top=18mm, bottom=55mm, inner=24mm, outer=38mm]{geometry}

%%% Allow generation of pseudo-latin body text for testing
%% Optional parameter [x-y] where x and y are numbers: generate paragraph
%% number x through y (150 paragraphs available).
%\usepackage{lipsum}

%%% Array package so we can use m{'width'} and adjust the row spacing in tables
\usepackage{array}

%%% Control the layout of chapter headers
\usepackage{titlesec}

%%% Disable chapter/section numbering.
%\setcounter{secnumdepth}{0}
%%  Put Chapters/Sections/Subsections in the Table of Contents
\setcounter{tocdepth}{3}

%%% Allow the text to flow around tables and figures.
%% (Must load before bidi package, i.e. before polyglossia package)
% We might need this for chapters with actual figures (in contrast to just
% dropcaps)
\usepackage{wrapfig}

\usepackage{fontspec} % Openfont specifications for XeTeX;
% fontspec automatically loads xltxtra and xunicode, both of which are needed.
% Though using \vfrac{}{} (part of xltxtra) without explicit loading of
% xltxtra does not work.
\usepackage{xltxtra}

%% Required to keep captions to tables normal sized
%% Used in table in Liber Primus, page 43
\usepackage[normalsize]{caption}

%% Package to split the table over more than one page
%% Used in table in Liber Primus, page 43
\usepackage{longtable}

%%% Enable vertical text
%% Used in table in Liber Primus, page 27
%% (but not for the version with the horizontal layout)
%% Also in table in Liber Primus, page 38
%% Also in table in Liber Primus, page 43
\usepackage{rotating}

%%% Control the way page headers look
%\userpackage{fancyheadings}
%% Package needs to be downloaded specially

%%% Main body text is in latin. We want:
%% Modern typesetting
%% The long s shown as 's'
%% No ligatures
%% No swishes, not on 'Q' nor on anything else
%% Modern numbers
\setmainfont{Hoefler Text}[
]
%\setmainfont{Times New Roman}[
%]

%%% Add a font that supports Greek
%\newfontfamily\greekfont{Arial Unicode MS}
\newfontfamily\greekfont{Times New Roman}

%%% Add a font that supports Hebrew
\newfontfamily\hebrewfont{Arial Hebrew}

%%% Add a font that supports Arabic
\newfontfamily\arabicfont{Arial Unicode MS}

%%% Add a font that has Astrological symbols
\newfontfamily\astrofont{Menlo}
%% Command to allow "\astro{some text}"
\newcommand\astro[1]{{\astrofont #1}}

%%% Add a font for the references to line numbers in the original
\newfontfamily\nrfont{Helvetica}

%%% Dummy Command to make header of any size
%% use: head{<scalesize>}{<spacing>}{<text>}
%% <scalesize> is a sizefactor relative to normal size
%% <spacing> is the LetterSpace parameter
\newcommand\head[3]{}

%%% Commands to number paragraphs
%% http://tex.stackexchange.com/questions/10513/automatically-assign-a-number-to-every-paragraph
%% get marginpars to always show up on the correct side (need to compile twice)
\usepackage{mparhack}

%% No indent on paragraphs, because we make a paragraph out of each sentence.
%\setlength\parindent{0cm}

%% Command \parnum to format the paragraph counter number
%% (in bold, arabic numbers)
\newcommand\parnum{%
    \bfseries\footnotesize\arabic{parcount}%
}

%% Special paragraph counter which resets with every 'subject'
%% (represented as a subsection)
\newcounter{parcount}[subsection]

%% Command for an isolated paragraph counter mark
%% Useful e.g. for big, multi-paragraph spanning Initials
\newcommand\p{%
%    \stepcounter{parcount}%
%    \leavevmode\marginpar[\hfill\parnum]{\parnum}%
}

%% Define an environment to use in the source.
\newenvironment{parnumbers}{%
%    \par%
%    \everypar{%
%        \stepcounter{parcount}%
%        \leavevmode\marginpar[\hfill\parnum]{\parnum}%
%    }%
}{}

%%% Commands to refer to line numbers in the original
%% Each beginning of a sentence is marked with the line number in the original
%% where the sentence begins. Together with the page number in the original
%% (and chapter name for those parts that are outside the normal page numbers)
%% this uniquely identifies a line in the document.
%% In the interpreted transcription, this number can be printed before
%% the sentence, between parentheses. Example:
%%  "(7) Sed longe aliter animatos experti sumus ..."
%% In the litteral transcription we can leave these numbers out. Example:
%%  "Sed longe aliter animatos experti sumus ..."
%% For clarity we add the page number to the first sentence on a page.
%% Example:
%%  "(II:2) Iam quemadmodum Epochae sunt notationes, ..."
%% (page numbers in Roman for the Prolegomena, arabic for the rest)

%% Counter to contain the source page number, as written on the page of the book
\newcounter{sourcepagenr}

%% Counter to contain the PDF page number
\newcounter{pdfpagenr}

%% Command \ln{number} to show the line number, followed by a
%% non-breaking space.
\newcommand{\lnr}[1]{\nrfont({#1})~\normalfont}

%% Command \plnr{PDFpage]{page}{linenumber}
%% Register and the PDF page number and the source page number
%% Show the source page number and the line number,
%% followed by a non-breaking space.
\newcommand{\plnr}[3]{%
	\setcounter{pdfpagenr}{{#1}}%
	\setcounter{sourcepagenr}{{#2}}%
	\nrfont(\arabic{sourcepagenr}:{{#3}})~\normalfont%
}

%% Command \Rplnr{PDFpage]{page}{linenumber}
%% Same as \plnr{}{}{}, but printing the page as a Roman number
\newcommand{\Rplnr}[3]{%
	\setcounter{pdfpagenr}{{#1}}%
	\setcounter{sourcepagenr}{{#2}}%
	\nrfont(\Roman{sourcepagenr}:{{#3}})~\normalfont%
}

%% Our own cookup to put letters in the margin to indicate quarters of a
%% page (A-D) as used in the original.
%% Uses mparhack.
%% Because of limitations of \marginpar{} the margin letters must go
%% on the same side as the paragraph numbers. We put them a bit further
%% away from the body text to make them more distinct.
\newcommand\mletter[1]{}

%%% No drop caps (illuminated initial letters)
%% Dummy commands that just print the text
\newcommand\dropcap[3]{{#2}{#3}}
\newcommand\dropcapQ[3]{{#2}{#3}}
\newcommand\dropcapil[3]{{#2}{#3}}

%%% Use polyglossia to handle multiple languages
%% We have (at least): latin, polytonic greek, hebrew, arabic, persian
%% (Contains bidi package, and must be loaded *after*:
%%  wrapfig, lettrine)
\usepackage[quiet]{polyglossia}
\setmainlanguage{latin}
\setotherlanguage{greek}
\setotherlanguage{hebrew}
\setotherlanguage{arabic}
\setotherlanguage{english}

%%% Reduce the white space above and below a chapter title
%%% The word "Chapter" (or "Caput") and the number are removed by making
%%% the third parameter empty.
%% Note: these commands must be given *after* plyglossia is loaded and
%% its commands (\setotherlanguage{}) are given
%\titleformat{\chapter}[hang]
%  {\normalfont\tiny\bfseries}{}{0em}{\Huge}
%\titlespacing*{\chapter}{0pt}{-10pt}{0.5cm}

%%% Reduce the white space above and below figures
%\setlength\intextsep{0pt}

%% Prevent text from going outside the margins, by making
%% the word spacing sloppy
\sloppy

%%% Tell XeTeX where to look for graphics files
%% Note: all file paths are relative to the root document, i.e.
%% the tex file that is passed as an argument to the compiler.
%% In other words: this file.
\graphicspath{{./img/}}

%%% Give Table of Contents the same name as in the original
\renewcommand{\contentsname}{Conspectus omnium capitum}
% === Does not appear to work

%%% Pre-define a string to use in the left-page header
\newcommand{\shorttitle}{De Emendatione Temporum}
%%% Same for right-page header
\newcommand{\shortauthor}{Iosephi Scaligeri}

%%% Command to set the headers at the start of a new chapter
%% Put them in small caps
%% Center them using \hfill{}
\newcommand{\setheaders}[2]{%
\markboth{\sc{\hfill{}#1\hfill}}%
         {\sc{\hfill{}#2\hfill}}%
}

%%%========================================================%%%

%% list the parts we want to compile
\includeonly{./tex/prolegomena,./tex/liber-primus}

\begin{document}

%% Top matter
\title{De Emendatione Temporum}
\author{Julius J. Scaliger}
\pagestyle{myheadings}
\setheaders{}{}

%% Source file for testing. Comment out when not testing
%% Does not need to be mentioned in the above \includeonly{}
%% !TEX TS-program = xelatex
% !TEX encoding = UTF-8 Unicode
% this template is specifically designed to be typeset with XeLaTeX;
% it will not work with other engines, such as pdfLaTeX

%%% Count out columns for fixed-width source font
% 000000011111111112222222222333333333344444444445555555555666666666677777777778
% 345678901234567890123456789012345678901234567890123456789012345678901234567890

\chapter[Dropcap]{Dropper Capper}

\setcounter{parcount}{0}
\begin{parnumbers}
\dropcap{12}{Q}{Vintusdecimus} hic annus agitur, candide Lector, postquam opus nostrum de Emendatione Temporum emisimus.
\\ \p
Persuaseram mihi, homines studiosos aliquam nobis gratiam habituros tot rerum, quas \& scitu dignas, \& a nobis primum indicatas negare non poterant.
\\ \p
Sed longe aliter animatos experti sumus: atque adeo rem potius inuidiosam atque obtrectationi opportunam, quam illis gratam me suscepisse intellexi.

Denique nihil aliud quam significarunt, quiduis potius se ignorare malle, quam a nobis aliquid discere.

In quibusdam candorem, in aliis studium, in omnibus sensum bonarum rerum \framebox[1.0\width]{desideraui}.

Normal font:\\
ABCDEFGHIJKLMNOPQRSTUVWXYZ\\
abcdefghijklmnopqrstuvwxyz\\
Quistactum solissima

Dropcap font:\\
{\dropcapfont
ABCDEFGHIJKLMNOPQRSTUVWXYZ\\
abcdefghijklmnopqrstuvwxyz\\
Quistactum solissima
}

\end{parnumbers}
Columnsep: \the\columnsep
\setlength{\columnsep}{30pt} \\
Columnsep: \the\columnsep

\dropcapw{8}{Q}{Vintusdecimus} hic annus agitur, candide Lector, postquam opus nostrum de Emendatione Temporum emisimus.

Persuaseram mihi, homines studiosos aliquam nobis gratiam habituros tot rerum, quas \& scitu dignas, \& a nobis primum indicatas negare non poterant.

Sed longe aliter animatos experti sumus: atque adeo rem potius inuidiosam atque obtrectationi opportunam, quam illis gratam me suscepisse intellexi.

Denique nihil aliud quam significarunt, quiduis potius se ignorare malle, quam a nobis aliquid discere.

In quibusdam candorem, in aliis studium, in omnibus sensum bonarum rerum desideraui.
Columnsep: \the\columnsep

\setcounter{parcount}{0}
\begin{parnumbers}

\newlength{\dcunit}\setlength{\dcunit}{1.4\baselineskip}
\newlength{\dcscale}\setlength{\dcscale}{12\dcunit}
\addtolength{\dcscale}{-0.5\dcunit}
Dcheight: \the\dcscale

\setlength{\columnsep}{1pt}\begin{wrapfigure}[12]{l}{0mm}{\fontsize{\dcscale}{1em}\selectfont Q}\end{wrapfigure}
\textsc{Vintusdecimus} hic annus agitur, candide Lector, postquam opus nostrum de Emendatione Temporum emisimus.

Persuaseram mihi, homines studiosos aliquam nobis gratiam habituros tot rerum, quas \& scitu dignas, \& a nobis primum indicatas negare non poterant.

Sed longe aliter animatos experti sumus: atque adeo rem potius inuidiosam atque obtrectationi opportunam, quam illis gratam me suscepisse intellexi.

Denique nihil aliud quam significarunt, quiduis potius se ignorare malle, quam a nobis aliquid discere.

In quibusdam candorem, in aliis studium, in omnibus sensum bonarum rerum desideraui.
\end{parnumbers}

\chapter{Q dropcap (with descender)}
\setcounter{parcount}{0}
\begin{parnumbers}

\setlength{\dcunit}{1.1\baselineskip}
\setlength{\dcscale}{3\dcunit}
\addtolength{\dcscale}{-0.5\dcunit}
DCscale: \the\dcscale

\setlength{\columnsep}{1pt}\begin{wrapfigure}[3]{l}{0mm}{\dropcapfont\fontsize{\dcscale}{1em}\selectfont Q}\end{wrapfigure}
\textsc{Vintusdecimus} hic annus agitur, candide Lector, postquam opus nostrum de Emendatione Temporum emisimus.

Persuaseram mihi, homines studiosos aliquam nobis gratiam habituros tot rerum, quas \& scitu dignas, \& a nobis primum indicatas negare non poterant.

Sed longe aliter animatos experti sumus: atque adeo rem potius inuidiosam atque obtrectationi opportunam, quam illis gratam me suscepisse intellexi.

Denique nihil aliud quam significarunt, quiduis potius se ignorare malle, quam a nobis aliquid discere.

In quibusdam candorem, in aliis studium, in omnibus sensum bonarum rerum desideraui.

Columnsep: \the\columnsep\\
\setlength{\columnsep}{1pt}
Columnsep: \the\columnsep\\

\dropcapQ{9}{Q}{Vamquam} V a m q u a m lorem ipsum dolor ſit amet, consectetuer adipiscing elit. 

Ut purus elit, veſtibulum ut, placerat ac, adipiscing vitae, felis. 

Curabitur dictum gravida mauris. Nam arcu libero, nonummy eget, consectetuer id, vulputate a, magna. Donec vehicula augue eu neque. 

Pellentesque habitant morbi triſtique se- nectus et netus et malesuada fames ac turpis egeſtas. Mauris ut leo. 

Cras viverra metus rhoncus sem. Nulla et lectus veſtibulum urna fringilla ultrices. Phasellus eu tellus ſit amet tortor gravida placerat. 

Integer sapien eſt, iaculis in, pretium quis, viverra ac,
nunc. Praesent eget sem vel leo ultrices bibendum. Aenean faucibus. 

Morbi dolor nulla, malesuada eu, pulvinar at, mollis ac, nulla. Curabitur auctor semper nulla. Donec varius orci eget risus. Duis nibh mi, congue eu, accumsan eleifend, sagittis quis, diam. Duis eget orci ſit amet orci digniſſim rutrum.
\end{parnumbers}

\chapter{Illuminated dropcap}
\begin{parnumbers}
\dropcapil{16}{Q}{Vintusdecimus} hic annus agitur, candide Lector, postquam opus nostrum de Emendatione Temporum emisimus.
\\ \p
Persuaseram mihi, homines studiosos aliquam nobis gratiam habituros tot rerum, quas \& scitu dignas, \& a nobis primum indicatas negare non poterant.
\\ \p
Sed longe aliter animatos experti sumus: atque adeo rem potius inuidiosam atque obtrectationi opportunam, quam illis gratam me suscepisse intellexi.
\\ \p
Denique nihil aliud quam significarunt, quiduis potius se ignorare malle, quam a nobis aliquid discere.
\\ \p
In quibusdam candorem, in aliis studium, in omnibus sensum bonarum rerum desideraui.

%\begin{wrapfigure}[<nr of lines>]{<placement: r l ...>}[<overhang>]{<width>} 〈figure〉 \end{wrapfigure}
% nr of lines: force height of figure in nr of lines
% width: set the width reserved for the figure. Zero (0pt) means use width of <figure>
\setlength{\intextsep}{-0.2ex}
\setlength{\dcunit}{1.25\baselineskip}
\setlength{\dcscale}{9\dcunit}
\addtolength{\dcscale}{-0.5\baselineskip}
\setlength{\columnsep}{1pt}\begin{wrapfigure}[9]{l}{0mm}{\initfamily\fontsize{\dcscale}{1em}\selectfont Q}\end{wrapfigure}
\textsc{Vintusdecimus} hic annus agitur, candide Lector, postquam opus nostrum de Emendatione Temporum emisimus.

Persuaseram mihi, homines studiosos aliquam nobis gratiam habituros tot framebox{rerum, quas \& scitu dignas,} \& a nobis primum indicatas negare non poterant. Sed longe aliter

Sed longe aliter animatos experti sumus: atque adeo rem potius inuidiosam atque obtrectationi opportunam, quam illis gratam me suscepisse intellexi.

Denique nihil aliud quam significarunt, quiduis potius se ignorare malle, quam a nobis aliquid discere.

In quibusdam candorem, in aliis studium, in omnibus sensum bonarum rerum desideraui.

{\initfamily XABCDEFX}XABCDEFX

\dropcapwi{9}{Q}{Vintsdecimus} hic annus agitur, candide Lector, postquam opus nostrum de Emendatione Temporum emisimus.

Persuaseram mihi, homines studiosos aliquam nobis gratiam habituros tot framebox{rerum, quas \& scitu dignas,} \& a nobis primum indicatas negare non poterant. Sed longe aliter

Sed longe aliter animatos experti sumus: atque adeo rem potius inuidiosam atque obtrectationi opportunam, quam illis gratam me suscepisse intellexi.

Denique nihil aliud quam significarunt, quiduis potius se ignorare malle, quam a nobis aliquid discere.

In quibusdam candorem, in aliis studium, in omnibus sensum bonarum rerum desideraui.

\end{parnumbers}


\frontmatter
\include{./tex/title}
\include{./tex/dedication}
\include{./tex/greek-quotes}
% !TEX TS-program = xelatex
% !TEX encoding = UTF-8 Unicode
% this template is specifically designed to be typeset with XeLaTeX;
% it will not work with other engines, such as pdfLaTeX

%%% Count out columns for fixed-width source font
% 000000011111111112222222222333333333344444444445555555555666666666677777777778
% 345678901234567890123456789012345678901234567890123456789012345678901234567890

\chapter{Prolegomena}
\sc{Prolegomena in Libros de Emendatione Temporum}

\em{Ad candidum Lectorem.}

\normalfont{}

\setcounter{parcount}{0}
\begin{parnumbers}
Quintusdecimus hic annus agitur, candide
Lector, postquam opus nostrum de
Emendatione Temporum emisimus.
\\ \p
Persuaseram
mihi, homines studiosos aliquam nobis
gratiam habituros tot rerum, quas et scitu
dignas, et a nobis primum indicatas negare
non poterant.
\\ \p
Sed longe aliter animatos experti
sumus: atque adeo rem potius invidiosam
atque obtrectationi opportunam, quam illis gratam me suscepisse
intellexi.

Denique nihil aliud quam significarunt, quiduis potius
se ignorare malle, quam a nobis aliquid discere.

In quibusdam
candorem, in aliis studium, in omnibus sensum bonarum rerum desideraui.

Nos vero, qui nihil unquam prius habuimus, quam ut horum
orationes sinamus praeterfluere, modo verum eruere, et inimicos
nostros etiam inuitos iuvare possimus, opus nostrum iterum in
manus sumptum auximus, illustravimus, emendauimus, ut, quanuis
idem sit, aliud tamen a nova cultura videri possit.

Quae huic editioni
accesserunt, haud promptum est dicere.

Sed in quibus a priore demutat,
postea intelliges, siquidem instituti nostri rationem aperuero.

Subiectum operis nostri est ratio Temporum civilium, et eorum,
quae in vetustatis cognitione versantur: finis, Emendatio: quod quidem
me tacente, et Titulus ipse promittit.

Civilium temporum cognitio,
eorumque historia, vertitur in multiplici diversorum annorum
forma et eorum methodis vulgaribus, quos Computos posterior
aetas vocauit.

\textgreek{Τα ιςορομγυα[?]} civilium temporum habes in primoribus
tribus libris, et maiore parte quarti: methodum autem in septimo.

A emendationis duae partes sunt.

%\end{parnumbers}
\clearpage
p. II [pdf 29]
%\begin{parnumbers}
Prior versatur circa epocharum
investigationem, posterior circa verum annum tropicum, 
et periodos Lunares: quam materiam posterior pars libri quarti,
item toti quintus et sextus sibi vindicant.

Iam quemadmodum Epochae
sunt notationes, et tituli temporum, ita ipsarum epocharum
quaedam debent esse propria \textgreek{γνωρίσματα} et characteres: quorum
characterum alii sunt naturales, alii civiles. 

Naturales quidem a rationibus
utriusque sideris, unde nati cycli Solaris, et Lunaris: civiles
ab instituto, cuiusmodi indictiones et anni Sabbatici: sine quibus in
harum rerum tractatione omnis conatus irritus. 

Rursus et eorum
quoque fallax usus est, nisi quaedam annorum ex illis periodus instituatur.

Sed eae sunt totidem, quot aut formae annorum, aut civilia
initia.

Nam in anno Aegyptiaco Nabonassari alia opus est, ac in anno
Solari, quia diversa forma: item in anno Actiaco sive Diocletianeo
alia, ac in Iuliano, propter diversa initia.

In anno Aegyptiaco vago
naturales characteres sunt \textgreek{εἰκοσιπεν σαετηρις[?]} Lunaris, et
\textgreek{έπταετηρις[?]} Solaris:
civilis autem character est quadriennium, quem canicularem
annum minorem vocabant Aegyptii.

Hi tres characteres in se ducti
producunt periodum magnam annorum 700 Aegyptiacorum: qua
uti debet disputator temporum, siquidem rationes suas ad annos
Nabonassari, Armeniorum, aut Persarum exigit.

At qui anno Iuliano,
quae omnium formarum temporibus est convenientissima, uti
volet, is cyclo utriusque sideris quindecies ducto componet elegantissimam
periodum annorum 7980, cuius initium in cyclo Solari,
et Indictione Romana, a Kal. Ianuarii, in cyclo Lunari a Martio, in
anno Sabbatico ab autumno.

Itaque non minus utilis, quam necessaria
est.

Sine ea nihil agit Chronologus: cum ea tempori, et saeculis
imperat.

Quam enim lubricum sit retro ab aliqua epocha notare tempora,
quod maior pars doctorum virorum facit, satis nos usus docuit.

His ita positis, ad singula huius operis membra venio.

%% == Libro primo
Libro primo
praeter divisionem temporum, et iucundissimam mensium, et
annorum historiam, de antiquissima anni forma disputatur, quae in
menses aequabiles annum describit, qua pleraque omnes Graecia usa
est, et ab ea omnis ratio Olympiadum pendet: nisi potius eam e ratione
Olympiadum propagatam dicas: quod sine cognitione Olympiadum
numquam tam eximium vetustatis et temporum monimentum
in lucem eruissemus. 

Ex tanta autem Graecorum scriptorum
copia unicus Pindarus nobis facem alluxit, qui solus nos docuit tempus
ludicri Olympici.

Aliter, quae paucitas est bonorum scriptorum,
nulla erat via ad haec interiora perveniendi.

Huius anni Graeci
formae doctrina tanto acceptior esse debet, quanto obscurior eius
rei apud maiores nostros scientia fuit: cum ante hos mille quadringentos
plus minus annos eius rei neque volam, neque vestigium
vetustas retinuerit.

%\end{parnumbers}
\clearpage
p. III [pdf 30]
%\begin{parnumbers}
Nam falso veteres multi, ac post eos infamae antiquitatis
scriptores, Macrobius ac Solinus, atque proavorum memoria
summus vir Theodorus Gaza, annum Graecorum statim
ab initio merum Lunarem fuisse prodiderunt.

Quamuis enim in
Panegyribus suis, ac nobilioribus sacris, quae certo annorum circuitu
redibant, unius Lunae rationem habebant, tamen, ut uno verbo
dicam, eorum anni forma Lunaris non erat.

Olympicum enim ludicrum
ipsa Lunae plena lampade celebrabatur, ut solus veterum nos
docet Pindarus. 

Praetera Laconibus ante plenilunium, aut novilunium
aliquid incipere religio erat.

Unde \textgreek{Λαχώνιχας ςελωοαζ}[?] vicinorum
proverbio iactatas, et contra Arcadibus proverbiali convicio
neglectum religionis obiectum legimus. 

Quod enim ante novilunium,
aut plenilunium ut plurimim bella aut alia seriora aggrederentur,
ob eam rem a finitimis nationibus \textgreek{[Greek]} vocabantur:
quae convicii caussa ab ipsis Arcadibus interpretatione elusa est,
probrum in laudem conversum ad vetustatem originis suae referentibus,
et antiquiorem sidere gentem suam gloriantibus. 

Quod igitur
novilunii ac plenilunii tempora Panegyricis ludicris deligebant,
propterea sacra trieterica instituta: cuiusmodi erant orgia Bacchi,
Nemea, Isthmia, alia.

Ea enim est anni Graecanici forma, ut si, verbi
gratia, novilunium in neomeniam Gamelionis incurrat, plenilunium
in eandem neomeniam incidat anno tertio redeunte.

Itaque
cum in Tetraeteride orgia Bacchi trieterica celebrabantur, tertio
anno redibant in eum sistum Lunae, qui priorum orgiorum situi oppositus
erat.

Quare elegantissime Statius trieterida vocat alternam:
quia alternis in novilunium, et plenilunium incurreret.

At sacra,
quae necessario eodem Lunae tempore obibantur, ea semper erant
tetraeterica: ut in Attica Panathenaica maiuscula, in Elide Olympias,
ut iam tetigimus, plenilunio.

Quod sane fieri non poterat, nisi absoluta
Tetraeteride, et Pentaeteride ineunte.

Atque ita Tetraeterides
in idem \textgreek{χῆμα} Lunae, non utique in idem tempus Solis redibant.

Ut
enim in orbem Solis et Lunae redirent, non aliter putabant fieri,
quam octaeteride confecta, eneaeteride ineunte.

Ex quo quaedam
eneaeterica sacra eo nomine instituta: cuiusmodi ab initio Pythia
fuerunt: et quidem merito.

Apollini enim, quem eundem cum Sole
faciebant, erant attributa.

Hinc colligimus, non solum Olymiadis
intervallum annis quatuor solidis explicatum fuisse; sed etiam puerliter
peccare eos, qui annorum quinque solidorum fuisse putant.

Neque vero quibusdam recentioribus succensendum, qui ita censent,
ita scribunt, sed et Ausonium nostratem culpa liberat Ovidius, scriptor
longe antiquior, et nobilior, qui aetatem suam quinquaginta annorum
decem Olympiadibus definit: quo magis mirum Pausaniam
hominem Graecum in ea haeresi fuisse, ut suo loco a nobis relatum est.

%\end{parnumbers}
\clearpage
p. IV [pdf 31]
%\begin{parnumbers}
Nam minus mirandum de Solino, qui cap. \textsc{xiii} Isthmia vocat
quinquennalia, quae erant tantum triennalia, quod certamen a Cypselo
tyranno intermissum, anno primo Olympiadis 49 instauratum
fuisse dicit.

Horum igitur omnium caussae ad typum anni Graeci referendae
sunt.

In quo argumento nihil eorum praetermisimus, quae
ei rei illustrandae faciebant, quanquam pene omnibus praesidiis
destituti.

Et quidem primum in genere, quod semper solemus, deinde
privatim multarum Graeciae nationum periodos proposuimus,
quae quidem non anni forma, sed situ et capite inter se differunt: in
qua tractatione diu nobis res fuit cum praestantissimo viro Theodoro
Gaza, vel potius cum eius sequacibus, a quibus extorqueri non
potest doctrina et situs mensium, ab illo primum proditus. 

Quae quidem
velitatio nobilioribus ingeniis, et ab omni invidia remotis, ut
spero, iucunda erit.

Quid enim est toto libro primo, cuius vel minima
pars, non dicam istis querulis, qui nihil sciunt, sed etiam doctioribus,
hoc saeculo, et ante multa retro saecula oboluerit?

Quid dicam \textgreek{[Greek]}?

Quis illarum caussas, et usum sciebat?

Quis
locum nobilem de illis in Verrina Ciceronis intelligebat?

Quis locum
\textgreek{Εξαιρέσεως[?]} in secunda Boedromionis?

Quis Posideonem intercalarem
mensem fuisse?

Huic materiae accessit \textgreek{ἐποχη χέντςα[?] θερινᾶ}
in \textsc{viii} Iulii, quae in priori editione omissa erat.

Id erat \textgreek{χάντσον[?]}
populare, quod nomine \textgreek{τςοπων[?] θερινῶν[?]} Aristoteles, Theophrastus,
Plutarchus, et omnes veteres intelligunt, non autem ipsum verum
Solstitium: quae rei pulcherrimae notatio nobis viam ad illustriora
praeiuit.

Quod Solstitiorum, et Aequinoctiorum puncta \textgreek{χέντρα} vocentur,
satis sciunt, qui veterum Graecorum libros legerunt.

Columella
cardines vocavit.

In praestantissimo Parapegmate, quod falso
Ptolemaeo attribuitor (est enim antiquius Ptolemaeo) ad \textsc{viii} Kal.
Iulii (quod est Solstitium Sosigenis) annotatum est: \textit{Aestivus cardo,
et momentanea aeris perturbatio}.

In Graeco (utinam haberemus!)
sine dubio fuit: \textgreek{Θερινὸν χέντρον, χαὶ ςιγμιαία αἔσος Ιασοιχή[?]}.

Igitur \textgreek{χέντρον
θερινὸν} nihil aliud, quam \textgreek{τροπαὶ θεριναί}.

Cur \textsc{viii} dies Iulii erat
epocha aestiva in usu civilis anni, non semel caussam reddidimus. 

Adiecta etiam pernecessaria neomeniarum Atticarum Tabula: quae
non solum priori editioni, sed etiam doctrinae anni Attici deerat.

Liber secundus anno Lunari dicatus est ideo, quia is annus ex illo
Graeco aequabili manasse videtur.

Ibi aperitur omnis antiquitas \textgreek{ἔτοις[?]
πρυτανείας[?]}, Octaeteridum Cleostrati, Harpali, et Eudoxi: quae omnia
hodie nomine tenus nota erant.

Eudoxea Octaeteris numquam
in usus civiles admissa est.

Anni vero \textgreek{πρυτανείας[?]} in vetustissimis Psephismasin
Atheniensium primo quidem ex Cleostratea, deinde, illa
abrogata, ex Harpalea petiti sunt.

%\end{parnumbers}
\clearpage
p. V [pdf 32]
%\begin{parnumbers}
Sequitur magnus annus Metonicus
ambarum, et Calippicus Metonici castigator.

Et quidem hi
ambo nomine noti tantum: caussarum autem, et omnium, quae ad
illa pertinent, mira ignoratio hactenus fuit.

Accesserunt huic editioni
Tabulae operosissimae dispensationum neomeniarum Metonicarum,
et Calippicarum: cuiusmodi etiam in Harpalea Octaeteride
exhibuimus. 

Quod de Eudoxea Octaeteride diximus, idem de
periodo Chaldaeorum dicendum, eam nunquam ad civilia tempora,
sed ad Genethliacorum themata usurpatam fuisse.

Id quod tum
multa argumenta, tum unicum certissimum illud est, quod eorum
menses appellationibus Macedonicis, non vero Chaldaicis fuerunt.

Propterea recte cum illius anni diatriba doctrinam dodecaeteridis
Chaldaicae Genethliacorum coniunximus, cuius nomen quidem
solum notum erat ex Censorino: cognitio autem nobis ex Arabum,
et Orentalium usu repetenda fuit.

An aliquis Graecorum \textgreek{δωδεχαετρίδος Χαλδαιχῆς}
meminerit, haud promptum est dicere.

Unum tantum Orpheum sive Onomacritum eius meminisse scimus. 

\textgreek{ὀρφδὶςὀν[?] ταῖς[?] δωδεχαετνρίσιν[?]:}

\begin{greek}
ἔςαι δ᾽ αὖθις ἀνὴρ, ἢ χοίραν[G-circ], ἠὲ τύρανν[G-circ],

ἢ βαοιλδὶς, ὂς τῆμ[G-circ] ἐς οὐρανὸν ἴξε[right curl] αἰπιιύ [all doubtful].
\end{greek}

Est apotelesma cuiusdam Genethliaci consulti super alicuius genesi,
de quo ipse respondit, eum fore magnum regem aut Dynastam, etcetera.

Citat Tzetzes. 

Haec multum illustrant doctrinam Dodecaeteridos
genethliacae parum antehac notae.

Itaque quemadmodum \textgreek{τελετὰς}
ita etiam \textgreek{δωδεκαετνρίδας[?]} scripserat Onomacritus sub nomine
Orphei.

Qualis fuerit Iudaeorum annus sub Seleucidis, quibus parebant,
multis exemplis testatum reliquimus: in quibus etiam translationis
feriarum in capite anni antiquitatem asseruimus adversus homines
nostrorum temporum, qui nugantur commentum nuperum
Iudaeorum esse.

In illis Doctor Theologus ingenti commentario
suo in Evangelium secundum Iohannem exultabundus ait illam
translationem confutari ex loco Iosephi, in quo scribit, quo anno
Hyrcanus foedus icit cum Antiocho Sidete, Pentecosten fuisse feria
prima.

Hunc locum Iosephi nos olim in priore editione produximus,
unde is, aut qui illi indicavit, accepit.

En, inquit, duo Sabbata continua.

Si propter continuationem duorum Sabbatorum, feria transfertur,
ergo ubi sunt duo continua Sabbata, non transfertur. 

In quibus aperte ostendit se ignorare caussam feriae transferendae, quae fiebat
propter solum Tisri, non autem propter alios menses; propterea
quod ille mesis multa solennia habet, adeo ut si non habeatur
ratio translationis, aliquando non solum duo, sed entiam tria continua
sabbata concurrere necesse sit.

%\end{parnumbers}
\clearpage
p. VI [pdf 33]
%\begin{parnumbers}
Si enim feria sexta inciperet
neomenia Tisri, omnino tria sabbata continuarentur, neomenia,
sive clangor tubae, sabbatum ordinarium, et ieiunium Godoliae.

Continuantur autem saepernumero in aliquo reliquorum mensium
duo Sabbata: idque fit, quando solenne est aut feria prima, aut feria
sexta.

Quorum alterutrum quotannis incidere, nisi quando Tisri
incipit feria tertia, Doctor ignoravit.

In primam feriam incidunt
haec solennia, \textsc{xxv} Casleu, et \textsc{x} Tebeth in anno defectivo tam
communi, quam embolimaeao, quotiescunque Tisri incipit feria secunda:
\textsc{vi} Sivvan; quando Nisan incipit feria septima:
\textsc{xv} Nisan, \textsc{xvii} Tamuz,
\textsc{ix} Ab, quando Nisan incipit feria prima.

In feriam autem sextam
convenit solenne \textsc{xxv} Casleu et
 \textsc{x} Tebeth, quando Tisri est feria
septima in anno tam communi, quam embolimaeo.

\textsc{xiiii} Adar, quando
Nisan sequens est feria prima: \textsc{vi} Sivvan, quando Nisan feria quinta.

Vides, quot Sabbata quotannis, nisi quando Tisri incipit feria tertia, Iudaei
continuent in aliquo mensium, praeterquam in solo Tisri, cuius
unius gratia illa cautio instituta
% No period at end of sentence

Itaque doctor tam frustra, quam ridicule
Iosephi testimonium adduxit de sexta Sivvan, id est, Pentecoste
feria prima; cum illo anno neomenia Nisan fuerit Sabbatum.

Atqui
nihil superesse putavit, quam ut Vaticani montis imago redderet
\textgreek{ἰὴ παιαή[?]}.

Sed ipse valde ignarus est harum rerum, ut reliqui omnes,
qui contendunt novitium esse Iudaeorum commentum.

Nos
validissime demonstravimus, et saeculo Christi, et retro sub Seleucidis
translationes in usu fuisse.

Et sane res peruetusta est.

Quae tamen
non minus ignorata, quam periodus Calippica, qua Seleucidae, et
Seleucidarum edicto Iudaei usi.

Quod non solum ex Nisan anni excidii
Hierosolymorum a nobis demonstratum est, sed etiam patet
ex definitione Rabbi Adda.

Is annum definit dierum \textsc{ccclxv},
horarum 5, 997/1080. 48/76.

Quid hac definitione aliud vult, quam periodum
Iudaicam fuisse annorum 76?

Cum Meto definit annum dierum
365. hor. 5. 1/19. ex eo coniiciendum relinquit, se uti periodo annorum
19.

Utebantur igitur periodo 76 annorum, id est, Calippica:
et tamen in omnibus neomeniis Lunae \textgreek{φάσιν} observabant, non
quod eam ex praescripto periodi non indicerent, sed ideo, ut eam
sanctificarent.

Nam et hodie quoque observant \textgreek{φάσιν}, non ut ex ea
neomeniam indicant, sed ut eam sanctificent.

Itaque Luna statim
visa dicunt: \texthebrew{[Hebrew]}.

\textgreek{ἀγαθὸν τέρας ἔςω ἡμῖν ης[??] παντὶ Ισραήλ.}

Idem faciunt et Muhammedani, quamuis neomenias ex
scripto indicere soleant.

Neque aliud intellexit fabulosus quidem,
sed tatem vetus auctor \textgreek{[Greek]} apud Clementem:
\textgreek{[Greek][Lots of Greek]}.

%\end{parnumbers}
\clearpage
p. VII [pdf 34]
%\begin{parnumbers}
Praeclara quidem ista: sed nescit, quid dicit.

Nam in Iudaeorum potestate
nunquam fuit, ut exspectarent \textgreek{φάσιν}:
% Greek: phase
 quia raro Luna se ostendit,
nisi secundo post coitum die.

Quod si expectandum ipsis esset,
res ridicula accideret, ut Elul, qui semper est cavus mensis, non solum
plenus, sed etiam aliqando unius et triginta dierum esset.

Sine dubio translationem feriae intelligit, cuius caussam ignorat.

\textgreek{πρῶτον σάββατον} vocat \texthebrew{רֹאשׁ הַשָּׁנָה‎} (Rosh Hashanah) caput anni.

Nam Sabbatum vocat, quia Festus
dies, \textgreek{κὶα᾽εργός}[?].

Ita etiam vocatur Levitici \textsc{xxiii}, 24.
% Leviticus 23:24: “Tell the people of Isra’el, ‘In the seventh month, the first of the month is to be for you a day of complete rest for remembering, a holy convocation announced with blasts on the shofar.'"
% λάλησον [Speak] τοις [to the] υιοίς [sons] Ισραήλ [of Israel,] λέγων [saying!]
% του [The] μηνός [(²month] του [] εβδόμου [¹seventh),] μία [day one]
% του [of the] μηνός [month] έσται [will be] υμίν [to you] ανάπαυσις [a rest,]
% μνημόσυνον [a memorial] σαλπίγγων [of trumpets,] κλητή [(²convocation]
% αγία [¹a holy)] τω [to the] κύριος [LORD.]

\textgreek{ἑορτὴν}[?]
% feast
 intellige
\textgreek{κατ᾽ ἐξοχήν τὴν πεντηκοστήν}:
% eminently the Pentekost [?]
 quod ita Hebraice vocetur, nempe \texthebrew{עֲצֶרֶת}[?] ([sh'miní] 'atséret).
% "The eighth [day] of assembly".
 
Vide in Computo Iudaico.

At \textgreek{μεγάλην ἡμέραν}
% great day
 vocat \textgreek{τὴν σκηνοπηγιαν, κατ᾽ ἐξοχήν}
 quoque, id est \texthebrew{הַנ}[?].

Nam aliae erant \textgreek{μεγάλαι ἡμέραι},
% Great Days
proinde ut et \texthebrew{חַנִּים}[?].

Sic Tertullianus magnos dies dixit, quos
Hebraei \texthebrew{[Hebrew]} vel \texthebrew{[Hebrew]}.

Eius verba sunt ex v in Marcionem:
\textit{Dies observatis, et menses, et tempora, et annos, et Sabbata, ut opinor,
et cenas puras, et ieiunia, et dies magnos.}
% Tertullianus: De Adversus Marcionem, Book 5, chapter 4, section 6.

Sed quid Tertullianum
advoco?

Ecce Biblia Graeca ita vertunt ex primo caput Isaiae:
\textgreek{τὰς νουμἠνίας ὑμῶν, καὶ τὰ σάββατα,
 καὶ ἡμέραν μεγάλην οὐκ ἀνέχομαι}[?].
% Isaiah 1:13 ?: I cannot bear your new moons, and your sabbaths,
 and the great day;

Quod Hebraice est \texthebrew{[Hebrew]}, vertunt \textgreek{[Greek]}, quod idem
est quod \texthebrew{[Hebrew]}: et quidem manifesto Sabbata distinguuntur a
magnis diebus. 

Quare perperam quidam \textgreek{[Greek]} interpretantur
Sabbatum apud Iohannem, \textgreek{[Greek]}. de quo infra.

Quin et Tertullianus ipse \textgreek{[Greek]},
quas cenas puras vocat, a diebus magnis, et a ieiuniis, et a
Sabbatis distinguit.

De Cena pura, praeter id quod diximus ad
Festum, ita reperi in veteri et peroptimo Glossario Latinoarabico:
\textit{Parasceue, cena pura, id est, praparatio, que fit prosabbato.}

Conditor Annalium Ecclesiasticorum turbat de cena
pura, et negat esse parasceuen, quia cena pura apud Festum
habeat offam suillam.

Sed ipse, (pace docti viri dixerim) non
aduertit Puram dici, non quia careat carnibus, sed quia religionis
et dicis caussa fit.

Nam et parasceuae Iudaicae habent carnes,
et nihilominus dicuntur cenae purae, quod dicis caussa coquebantur,
coquunturque hodie prosabbato, quia in Sabbato
coqui non liceat.

Non negabis, candide Lector, haec vulgo non intelligi.

Itaque locus ille est nobilissimus. 

Tamen quotus quisque est ex tot Lectoribus, qui non haec aut praeteribit,
aut calumniabitur?

Sequuntur periodi magnae Hagerenorum,
ex quibus ratio anni soluti Indorum, et Muhammedanorum
tota pendet.

Omnia nunc primum ex Arabum scriptis
prodeunt: atque adeo omnis tractatio nostris hominibus
nova est.

%\end{parnumbers}
\clearpage
p. VIII [pdf 35]
%\begin{parnumbers}
Excipit hanc doctrina anni Iudaici hodierni, res, quod
saepe diximus, artificiosissima, ideoque eximia, quia melior
anni Lunaris forma constitui non potest.

Docemus praeterea, unde
natus sit ille annorum computus, quo utuntur hodie, a \textsc{vii} Octobris:
quem inepte putant a conditu rerum.

Post multarum Periodorum,
Cyclorum, Octaeteridum, Paschalium historias, in locum vltimum
\textgreek{[Greek]} veteris anni Romanorum coniecimus, ideo
quod ea forma proxime abesset a Lunari: ubi de saeculo Romano,
et capite veteris anni Romani, temporibus vltimis C. Iulii Caesaris,
multa accuratissime disputata.

Itaque ex singulis rebus singula capita
confecimus, cum potius singuli libri et quidem ingentes confieri
possent, si, quae hominum hodiernorum est ambitio, eadem nobis
incessisset.

Tertio libro opportune annus aequabilis datus est,
cum annus Solaris Aegyptiacus, adscitis diebus quinque, ex Graeco
propagatus sit: (quemadmodum annus Lunaris ex eodem Graeco
manauit, abiectis ab eo totidem diebus cum horis \textsc{xv}, paulo amplius)
quod, metacente, Plutarchus docuit in libro \textgreek{[Greek]}.

Adeo inter se libri nostri mutuo conspirant, neq; ab eis ratio,
methodus, et ordo abest.

In eo libro de Neuruz antiqui Persarum
periodo annorum \textsc{cxx}, deq; cognominibus dierum Persicorum,
de translatione \textgreek{[Greek]} in enthronismis nouorum Regnum,
item de caussis anni Iezdegird, de annis Armeniorum, et eorum
mensibus, omnia nova protulimus. 

Sed haec non expergefacient animos
hominum, nisi forte ad obtrectandum.

Quartus liber est emendatio
tertii, ut secundus primo erat subsidiarius: qua methodo imperfectus
Lunaris Graecus libro primo disputatur, ut perfectus secundo.

Sic etiam perfectus Solaris, et siqui alii naturam perfecti imitantur,
supplent id, quod aequabili Aegyptiaco, Persico, et Armeniaco
vetuitas detraxerat.

Itaq; in quatuor partes tribuendus fuit.

In
prima continentur anni, quibus quarto anno exeunte dies ex quatuor
quadrantibus conflatus accrescit.

Ex illis nobiliores selegimus,
Iulianum, Actiacum, Antiochenum, Samaritanum, et alios. 

Nam et alios quoq; eius formae habebamus, ut Tyriorum, quorum menses
appelationibus Macedonicis, diuersa initia a Iulianis habent.
% "appelationb." interpreted as abbriviation for "appelationibus"

Sic
etiam Gazensium annus mere Actiacus fuit, appellationibus mensium 
Macedonicis, mensibus tricenariis. 

Marcus Ecclesiae Gazensis Diaconus,
in actis Porphyrii Gazensis Episcopi vocat \textgreek{Διον} Nouembrem
\textgreek{Απελλαιον} Decembrem quae nomina habent a Macedonibus. 

Sed
idem scribit Gazenses celebrasse Theophaniorum diem undecima
Audynaei, quae est sexta Ianuarii Iuliani, se autē redisse Constantinopoli,
Xanthici vicesima tertia, quam ait fuisse decimam octauam
Aprilis secundum Romanos: quibus ostēdit formam illius anni mere
Actiacam fuisse, mensibus tricenariis, appellationibus Macedonicis. 

%\end{parnumbers}
\clearpage
p. IX [pdf 36]
%\begin{parnumbers}
Secunda pars annos emendatos, eorumque emendandorum
rationem complectitur: tertia periodos multiplices, quarum finis
conciliatio anni civilis cum Solari, cui dies quinto quoque anno
ineunte accrescit.

Quarta pars agit de vera emendatione anni, et
de anno caelesti instituendo, qui pertinet ad methodum epochae
mundi.

Quemadmodum autem nulla Lunaris anni civilis ratio
recta iniri potest, praeter eam, qua Iudaei utuntur: ita nullus annus
caelestis Tropicus recte institui potest, nisi ex forma, quam edidimus,
quam nemo vituperabit, nisi qui ignorauerit; omnis laudabit,
qui intellexerit.

Alioquin scio et malignos et obtrectatores non defuturos. 

Annus tam noster, quam Iudaicus civilis quidem, sed naturalis,
vterq; ad motum quisq; sui sideris descriptus. 

Ideo eius saltem
in scriptis usus esse debet, qualis olim Philadelphi Dionysianus,
Chaldaeorum Calippicus, hodie Persarum Gelaleus. 

Tres igitur libri
primi, et prima pars quarti pertinent ad \textgreek{[Greek]} temporum civilium
cum septimo.

At reliquae tres partes quarti cum duobus libris
sequentibus pertinent ad ipsam emendationem temporum.

Atque
ut a mundi primordiis omnes res deducuntur, ita mundi epocham
primam ordine posuimus: qua in re quam pueriliter hallucinati sint
omnes, non sine admiratione tam imperitiae quam pertinaciae eorum
dicere possum.

Non loquor de iis, qui saeculo uno, aut pluribus altius
originem rei repetunt.

Nam quemadmodum ii nullam rationem
sibi proposuerunt, quam sequerentur, ita nullos lectores nancisci
possunt, nisi imperitos. 

Qui intra saeculum maiorem mundi
epocham faciunt, eorum duo genera reperio.

Prius genus est eorum,
qui solutionem captiuitatis in primum annum Olympiadis \textsc{lv} conferunt:
alterum eorum, qui tempus illud \textsc{xviii} aut \textsc{xix} annis ante
\textsc{lxiiii} Olympiadem definiunt.

In priori haeresi fuerunt et
quidam veterum Ecclesiasticorum, ut alicubi indicauimus. 

Aiunt
Cyrum caepisse imperare primo anno Olymiadis \textsc{lv}, hoc est 217
anno Iphiti, quod verum est: de quibus deductis septuaginta, relinquitur
annus excidii Hierosolymorum, et casus Sedekiae 147 a primo ludicro
Olympico.

Sed puerilis sentētia multis absurditatibus eluditur.

Primo computatione non recta annorum \textsc{lxx} a capto Sedekia.

Deinde quod Cyrum statim initio regni sui Regem Mediae, Persidos,
Susidos, Assyriae, Babyloniae, totius Asiae minoris, Indiae, totius
Syriae constituunt, qui unius Persidos Rex fuerit aliquot annis ante
casum Astyagis[?], et post illud tempus pauculis annis ante obitum
Babylone potitus sit.
% Astyagis or Aftyagis?

Haec sola absurditas facit, ut non solum eorum
nulla ratio habeatur, sed ut ludibrium quoq; debeant.

Tertio 147 annus
Iphiti est 118 Nabonassari: qui erat annus quintus ante initium Nabopollassari
patris Nabuchodonosori.

%\end{parnumbers}
\clearpage
p. X [pdf 37]
%\begin{parnumbers}
Ergo Nabuchodonosor anno
decimono regni sui templum et Hiersolyma euertit annis quinq;
ante quam pater ipsius, cui ipse successit, regnaret.

Digna profecto
talibus doctoribus sententia.

Tamen tantum abest, ut hac tam insigni
absurditate a sententia desistant, ut animos ab eiusmodi portentis
opinionum sumant.

Postremo ignorant diuersa esse initia Regnum,
ut ipsius Nabuchodonosori, cum patre, et solius: Alexandri,
ab excessu Philippi patris, et ab initio Seleuci: Diocletiani, ab aera
martyrum, et a primo anno imperii.

Sic etiam Cyri, apud Graecos, ab
initio regni Persidis: apud Babylonios, vel a subacto toto Babyloniae
imperio, vel ab aliquo insigni facto, quodcunque illud fuerit, sive
ex edicto ipsius Cyri, sive translatione \textgreek{πδν ἐπα γο υ[?]ων},
ut solebat fieri.

Qui tantam inscitiam sequi noluerunt, non tamen rectam viam
institerunt, quia quindecim aut amplius annis ante \textsc{xlvi} Olymiadem
casum Sedekiae coniiciunt.

Nos ante annum quartum illius
Olympiadis id non potuisse accidere ita demonstramus. 

Ezekias
Rex Iuda, postquam singulari Dei beneficio ab ancipiti morbo conualuisset,
anno \textsc{xiiii} regni sui, accepit Legatos et \textgreek{ςωτήρια[?]}, a
Merodach Rege Chaldaeorum.

Ponamus \textsc{xiiii} annum Ezekiae in
primo anno Merodach, hoc est, in \textsc{xxvii} Nabonassari.

Nam is est
annus primus Merodach apud Ptolemaeum ex Chaldaicis obseruationibus. 

Hoc modo annus primus Ezekiae conuenerit in annum
\textsc{xiiii} Nabonassari. Ab initio Ezekiae, ad excidium templi, anni
sunt absoluti 138.

Hoc est, annus ipsius excidii est 139 labens ab initio
Ezekiae.

quod ita demonstramus. 

Annus primus Sedekiae est
quartus Hebdomadis, teste Ierimia, initio cap. \textsc{xxviii}: et proinde
undecimus, qui et vltimus, est Sabbaticus. 

de quo extat testimonium
apud Ieremiam, et nemo dubitat.

Rursus annus tertius decimus
Ezekiae erat Sabbaticus. 

auctor Isaias \textsc{xxxvii}, 30.

ex quibus
manifesto colligitur, \textsc{xiiii} Ezekiae esse primum Hebdomadis, et primum
Ezekiae esse sextum Hebdomadis. 

Ergo annis ab initio Ezekiae unitas addenda, ad methodum anni Sabbatici.

Addita unitate annis
139, numerus erit septenarius. 

Quare annus labens 139 est verus
annus ab initio Ezekiae.

Quibus additis 13 annis Nabonassari praeteris
(quia posuimus 14 Nabonassari primum Ezekiae) componitur
annus Nabonassari 152, in quo casus Sedekiae ex hac hypothesi
locandus est, hoc est, in anno periodi Iulianę 4118: de quibus deductis
907 absolutis ab Exodo, remanet annus Exodi 3211 in periodo Iuliana.

Porro Nisan Exodi caepit feria quinta, ut toties diximus, et ex
Mose rectissime ante nos Iudęi docuerunt.

At in anno periodi Iulianę
3211 Nisan non caepit feria quinta, sed feria tertia, Martii \textsc{xi}, cyclo tam
Solis, quam Lunae \textsc{xix}.

%\end{parnumbers}
\clearpage
p. XI [pdf 38]
%\begin{parnumbers}
Ergo annus proximus, quo Nisan caepit feria
quinta, is debuit saltē esse annus Exodi: atq; adeo is fuerit annus periodi
Iulianae 3214: in quo sane nisan caepit feria quinta, Aprilis \textsc{vi},
cyclo Solis \textsc{xxii}, Lunae \textsc{iii}.

Additis 907 annis absolutis ab Exodo,
annus 4121 periodi Iulianae suerit is, in quo excidium templi contigit:
qui est quartus Olympiadis 46, ut erat propositum.

Sed et post
Olympiadem 46 ponendum esse casum Sedekiae ita probabimus.

Amasis rex Aegypti, postquam regnasset annos 55, obiit circiter annum
7 Cambysis, anno ante excessum ipsius Cambysis, hoc est, anno
225  Nabonassari.

Nechao intersectus est a Nabuchodonosoro anno
quarto Ioiakim regis Iuda.

Ieremias \textsc{xlvi}, 2.

Post eum regnauit
Psammitichus annos \textsc{vi}.

Cui Aprias, cuius meminit Ieremias
\textsc{xliiii}, 30, succedit.

Is post \textsc{xxv} annos relinquit regnum Amasi.

Summa annorum a caede Nechao ad obitum Amasis anni 86, qui
deducti de 225, relinquunt annum Nabonassari 139, quartum Ioiakim
Regis Iudae, primum Nabuchodonosori.

Ergo Sedekias captus
anno 158 Nabonassari, qui erat tertius 47 Olympiadis.

Idque verum
esse postea validissime demonstrabimus.

Diodorus Siculus,
auctor omnium Graecorum certissimus, attribuit, \textsc{lv} annos Amasidi.
reliquos Apriae et Psammatichi habemus ex Herodoto.

Temere
igitur, et imperite faciunt, qui casum Sedekiae antiquiorem illo
tempore constituunt: neque his cassibus sese explicare poterunt,
quantumuis sua commoueant sacra, ut Plautus loquitur.

His valide
demonstratis, et licentia chronologorum intra aliquos fines summota
quos amplius migrare non possunt, ad originas ipsas penetremus.

Sed prius ut in Mathematicis concessa quaedam, aut quae negari
non possunt, assumuntur, ita et nobis quoque faciendum.

Tempora et initia Regum Babyloniae a Chaldaeis notata in obseruationibus
eclipticis, quae reiicere et damnare extremae impudentiae et
inscitiae est: item, eorundem regum initia et tempora a Beroso Chaldaeo,
qui minus quam tribus saeculis post illos vixit, et qui quae Actis
ac fastis Babyloniorum publicis continebantur, ignorare non potuit,
haec inquam, non tantum tāquam vera haberi postulamus, sed etiam
qui aliter putant, tanquam indignos censeri, qui aut audiri a nobis
mereantur, aut vllas literas attingant, aut aliquem locum inter
doctos habeant.

Tricesimum annum, cuius initio Prophatiae suae
meminit Ezekiel, quique capti Iechoniae quintus erat, Iudaei inepti
deducunt a libro legis reperto, anno \textsc{xviii} Iosiae Regis.

Quis unquam a libro reperto vllam aeram, aut edicto Iosiae institutam, aut a
Prophetis usurpatam legit?

Si tanti erat illa temporis nota, quare
eam non usurpat Ieremias, qui tam accurate annos Regum Iuda Iosiae,
Ioiakim, Iechoniae, Sedekiae notare solet?

Capite \textsc{xxv}, quare dicit
anno quarto Ioiakim, cum dicendum esset vicesimo secundo a libro
inuento?

Esto, cur Ezekiel dixit tricesimo, non tricesimo a libro inuento?

qui tamen dixit anno quinto deportationis Regis Ioachin.

%\end{parnumbers}
\clearpage
p. XII [pdf 39]
%\begin{parnumbers}
Certe mos est uti epocha, quae omnibus et nota et in usu sut.

Quare
igitur epocham producit, neque plebi notam, neque in usu positam?

Sed quid ea epocha opus in Babylonia, inter deportatos?

Nugae Iudaeorum,
nugae sunt istae, et halluciationes doctorum, qui eos sequuntur.

Quare eruditiores Iudaeorum, huius absurditatis et nugatoriae
caussae conscii, his ineptiis explosis, dicunt, illum annum non a
libro inuento, sed Iubilei fuisse tricesimum.

At hoc est litem lite decidere.

Nam, quomodo Iudaei annos a Iubileo putarent, qui Iubilea
numquam usurparunt?

Annos quidem Hebdomadis notant, utinitio
\textsc{xxviii} Ieremiae mentio anni quarti septimanae: \textit{Initio regni
Sedekia, anno quarto.}

Rursus mentio anni primi, et secundi in annis
\textsc{xiiii}, et \textsc{xv} Ezekiae, apud Isaiam \textsc{xxxvii}, 30.

Sed notationem
per Iubilea, imo ne Iubilei quidem mentionem, nusquam, nisi
in lege, reperies.

Praecepta fuit tantum, non recepta Iubilei obseruatio.

Sed quae haec plumbea Iudaeorum sententia a \textsc{xviii} Iosiae
Iubileum putare?

Iubilea putantur a primo anno hebdomadis, non
a septimo.

At \textsc{xviii} Iosiae suit septimus septimanae, non primus.

Quare, si a Iubileis annos putare mos esset, suerit hic annus non utique
tricesimus, sed undetricesimus Iubilei, a \textsc{xviiii}, non a
\textsc{xviii} Iosiae.

Denique is erat annus 862 ab excessu Mosis, 855 a
diuisione terrae sive \textgreek{[Greek]}.

Ergo suit vicesimus secundus, non
undetricesimus Iubilei.

En quot errores locus praepostere sumptus
nobis peperit.

Cum igitur neque a libro legis inuento, quod est absurdissimum,
neque a Iubileo, quod est falsum dupliciter, ille tricesimus
annus putandus sit; sequitur, quod negari non potest, a
quodam rege tunc imperante putandum esse.

Nam deportati et captiui
inter victores, qua epocha uti possunt, nisi victoris?

In Palaestina,
cum aliqua esset Iudaeorum Respublica, et Ecclesia bene constituta,
Iudae cogebantur uti anno Alexandreo dominorum Seleucidarum:
quanto magis Chaldaeorum, in media Chaldaea, nullis legibus,
nulla Republica, nulla Ecclesia.

Nehemias initio libri sui ita
scribit: \textit{Accidit mense Casleu, anno vicesimo, cùm eßem in castro Susan.}

Si alibi non expressisset se de vicesimo anno Artaxerxis loqui, haud
dubie aliquod Iubileum hic commenti essent inepti Iudaei, et inepti
quidam hominum nostrorum sequuti essent.

Eodem quoq; modo
loquitur Ezekiel \textit{anno tricesimo}, non adiecto regis nomine.

Quid enim opus erat in Chaldaea?

Duo ergo Reges simul imperabant,
Nabuchodonosor, et ille, qui iam tricesimum annum currentem
imperabat.

Quisnam Rex, obsecro, potuit trecesimum annum in regno
agere, cum iam Nabuchodonosor tertium decimum regnaret?

Non alius igitur suerit, praeter Nabopollassarum patrem Nabuchodonosori,
quod verum est.

%\end{parnumbers}
\clearpage
p. XIII [pdf 40]
%\begin{parnumbers}
Nam \textsc{xxix} solidos annos imperauit,
teste Beroso.

Quod si filius eius anno \textsc{xxx} partis iam duodecimum
absoluerat, profecto imperare caeperit anno partis decimooctauo,
qui erat Nabonassari 140.

Nam primus Nabopollassari est 123 Nabonassari,
testibus Chaldaeis apud Ptolemaeum, ex defectibus Lunaribus
obseruatis.

Et proinde Sedekias captus fuit anno 158 Nabonassari,
tertio autem Olympiadis 47.

Vide locum Berosi apud Iosephum.

Nabopollassarus audita rebellione Aegypti misit filium eo
cum regio imperio, et regio exercitu: a quo tempore consurgit initium
Nabuchodonosori cum patre regnantis.

Mos erat Regum Babyloniae
et Persidis, ut aut prosecturi in expeditionem, filios reges declarent,
aut in expeditionem mitterent cum regio nomine, tanquam
designatos, si contigisset ipsum patrem mori, absente filio, ne
vllus de rege futuro tumultus oriretur.

Exemplum habemus apud
Herodotum de Cyro Cambysen in solium suum collocante in expeditione
in Scythas.

Hinc Ctesias Cambysi attribuit annos 18,
cum tamen solus regnarit octo annos, testibus omnibus veteribus
Graecis, et Chaldaeis ipsis apud Ptolemaeum.

Dario vero Notho annos
idem attribuit 35, cum tantum 19 solus imperarit.

Rursus Berosus
\textsc{xxxxiii} annos ait Nabuchodonosorum imperasse, comprehensis
nimirum 13 annis, quos cum patre communicauit, cum
illis quos solus in imperio transegit.

Quare Nabuchodonosori regnum
dixit non Satrapian, tanquam a patre non ut Satrapes, sed Rex
et socius imperii in rebelles missus.

Verba eius sunt haec: \textgreek{[Greek]}.

\textit{Victo rebelli, eius regionem regno suo subiecit.}

Mox subiicit, Nabopollassari patris morto[?]
audita, qui \textsc{xxix} annos solidos regnauerat, ipsum Babylonem se
contulisse: quod accidit proculdubio aliquot diebus post illud tēpus
ab Ezekiele designatum.

Obiit enim Nabopollassarus anno regni
sui \textsc{xxx}.

\textgreek{[Greek]}.

Pulcherrima haec est obseruatio, quam Beroso vernaculo
Babylonicarum rerum scriptori debemus.

Eadem verba repetit Eusebius
De pręparatione euangelica, ubi plane \textgreek{[Greek]}, quemadmodum
est apud Ptolemęum, nominat, non \textgreek{[Greek]}, ut perperam
est editum in Iosepho: ex quo ineptus quidam duos esse coniecit
Nabulassarum et Nabopollassarum; cum tamen eadē verba sint, ne
una quidem syllaba minus, praeter illud nomen.

Rursus apud Iosephum
lib.\textsc{x} ca.\textsc{ii}.eadem verba Berosi repetuntur.

Sed ubi hic est \textgreek{[Greek]},
ibi est bis \textgreek{[Greek]}, utrobique male pro \textgreek{[Greek]}.

Quam bene haec diuinis scripturis conueniunt?

%\end{parnumbers}
\clearpage
p. XIV [pdf 41]
%\begin{parnumbers}
Vnde etiam sequitur, mortuo Nabopollassaro, non tricesimum annum Nabuchodonosori
dici caeptum in Chaldaea, sed primum quae res obseruatione
digna.

Iudaei primum annum putarunt ab eo tempore, quo
cum imperio missus est. Sed in Chaldaea primus eius annus consurgit
ab obitu patris.

Itaque Danielis 11, annus secundus Nabuchodonosori
est sine dubio secundus ab obitu Nabopollassari, tricesimus
primus ab initio eiusdem, 152 ab initio Nabonassari, sextus
Sedekiae.

Vnde indubitata eruintur temporis nota illius Capitis secundi
apud Danielem, qui erat quartusdecimus nnus capti Danielis,
et sociorum cum rege Ioiakim, sextus autem regni Sedekiae.

Proinde annus ille erat \textsc{xiiii} Nabuchodonosori in Syria, secundus
autem in Babylonia: non autem \textsc{xxv}, ut coniicit Hieronymus
ex quadam victoria Nabuchodonosori de Syria, et Arabia, cuius
meminerit Borosus.

At Berosus loquitur tantum vsque ad obitum
Nabopollassari, qui erat \textsc{xiii} Nabuchodonosori eius filii.

His tam illustribus demonstrationibus sua somnia praeserant, quibus antiquius
est somniare, quam vera dicere, aut nosse.

Nos ad reliqua
pergamus.

Annus capti Sedekiae est 158 Nabonassari, 4124 in periodo Iuliana.

Deductis annis 907 solidis, relinquitur annus 3217
Exodi, qui est 2264 Iudaici Computi in quo sane Neomenia Nisan
habuit characterem feriam quintam, secunda Aprilis, Cyclo
Solis \textsc{xxv}, Lunaea \textsc{vi}.

Sed quadragesimus annus, et quadragesimus
septimus, hoc est 2303, et 2310 Iudaicus fuit sabbaticus.

Iosuae
\textsc{xiiii}, 7, 10.

Iudaei dicunt septenarios annorum Computi sive aerae
suae esse Sabbaticos.

Atqui 2303, et 2310 sunt septenarii.

Ergo recte
Sabbaticos annos putant Iudaei; ut apud illos post legem nihil
hac obseruatione vetustius sit; res prosecto, quae firmissimum
minimentum futura sit harum rerum investigatoribus.

Neomenia Nisan Exodi conueniebat cum neomenia Krionos.

Ita vere naturalis suit illa neomenia.

Praeterea quadragesimus septimus
annus conuenit sabbatico Iudaico: 902 autem annus est tricesimus
Nabopollassari conueniens cum testimonio Ezekielis.

Deniq; anni 86 a septimo Cambysae retro putati desinunt in anno
caedis Nechao Aegyptii, eodemque 139 Nabonassari: quod conuenit
eidem computationi.

Negari igitur non potest, hanc esse veram
Exodi epocham, quam et verbum diuinum, et usus anni Sabbatici,
et historiae fides penes eximium scriptorem Chaldaeum
Berosum, et naturales neomeniae utriusq; sideris in unum conuenientes
confirmant.

Quid postulamus praeterea?

An ut tam certis,
tam egregiis, tam firmis argumentis somnia Corybantum anteponamus?

Quis unquam ita haec demonstrauit?

Quid demonstrauit?

%\end{parnumbers}
\clearpage
p. XV [pdf 42]
%\begin{parnumbers}
Quis aliter potest demonstrare?

Iam a conditu rerum, ad exodum,
anni sunt absoluti 2452 cum mensibus sex ab autumno, anni vero
absoluti 2453 a vere.

Sed ante Exodum initium anni putabatur ab
autumno, et eodem initio in tempus veris translato, tekupha tamen,
hoc est, finis anni Solaris mansit in autumno, circa quam tekupham
Deus \textgreek{[Greek]} celebrari praecepit.

Igitur ubi initium anni
ab vltima antiquitate suit, inde et rerum quoq; initium repetendum.

quod quidem a nobis factum, damnata priori sententia, quae
initium rerum statuebat in vere.

Reliqua pete ex capite de conditu
rerum.

Praeterea, quibus annus Lunaris in usu est, illis commodius
initium, et rationibus Tropicis conuenientius ab autumno, quam
a vere, ut Iudaeis propter \textgreek{[Greek]}, et Pascha.

Nam si annum
nostrum caelestem admitterent, et hoc unum cauerent, ut \textgreek{[Greek]}
citima sit in secunda Zygonos, semper citimum Pascha esset in neomenia
Krionos.

quia interuallum a neomenia Zygonos, ad neomeniam
Krionos, est semper 178 dierum, uno die plus, quam a scenopegia
ad Pascha.

Anni Sabbatici caussas iam reddidimus, et verum
annum sabbaticum a Iudaeis hactenus obseruari demonstrauimus,
initio hebdomadum sumpto, non utique a defectu Mannae,
quod fanatici quidam, et veritatis hostes faciunt, sed a 48 anno Exodi,
ex capite \textsc{xiiii} Iosue, et rationibus doctorum Habraeorum, qui
dicunt septem annos \texthebrew{[Hebrew]}, id est, subiugationis terrae,
septem \texthebrew{[Hebrew]}
fuisse, id est, diuisionis.

quod rectissimum est: ideoq; hebdomadem
primam diuisionis, non subiugationis procedere in numerum.

An
potuit annus sabbaticus esse ante agrorum culturam?

Furor est aliter putare.

Tamen non desunt, non deerunt, qui solo contradicendi
studio, ut sapere videantur, aliter statuent: quibus per me non solum
hoc facere, sed etiam nos irridere licet; quandoquidem veritas apud
illos nullo in precio est.

Unde nata sit diversitas epochae excidii Ilii,
cum alii 407 annis, alii 405, eum casum antiquiorem prima Olympiade
statuant, aperuimus ex doctrina anni Attici, cui acceptum
referimus quicquid eximium ex alta obliuione eruimus.

Veram sententiam
Eratosthenis esse deprehendimus, quae illam cladem coniicit
in annum 407 ante caput primae Olymiadis: eiusque veram
diem in anno Iuliano ostendimus.

Primam autem Olympiadem
ex doctrina itidem anni Graeci \textsc{xxiii} die Iulii celebratam fuisse ante
nos aperuerat nemo.

Et tamen quidam Simioli tanquam rem
vulgatam in suis vanidicis Chronologiis retulerunt: cuius rei cognitionem
unus Pindarus, quem illi neq; viderunt, neq; norunt, nos
docuit.


%\end{parnumbers}
\clearpage
p. XVI [pdf 43]
%\begin{parnumbers}
Quemadmodum autem Olympia, ita etiam Karnia plenilunio
celebrata fuisse, libro primo, capite de periodo Laconum
ostendimus. neque solum plenilunio, sed etiam eodem anno, quo
Olympia.

Itaq; Herodotus libro \textsc{viii} Olympia et Karnia anno primo
Olympiadis 75 celebrata suisse scribit, pag. 307 editionis Henrici
Stephani nostri.

Cum multi eruditissimi viri, et quidem in iis
Onufrius Panuinius Pater historiae, multa accurate de Palilibus Vrbis
disseruerint, ut ei doctrinae nihil ad perfectionem deesse videatur,
tamen et plura deesse ex nostris disputationibus colligi potest.

Monere vero debent Annalium et Fastorum scriptores, qui tempora
sua ad annos Vrbis dirigunt, utra Palilia sequantur, Varroniana,
an Catoniana.

Nam certe Onufrius noster, tametsi Catonem sequitur,
tamen quibusdam imprudens ad Varronem transfugit.
% "transfugit" should not be rendered with a long s

Nisi
haec distinctio adhibeatur, ridicula multa consequi necesse est.

Exemplum habemus in annis Christi per annos Vrbis eruendis,
quod hactenus ab omnibus factitatum.

Christus in annis Varronianus
uno anno maior est apud aliquem, quam in Catonianis apud alium.

Quare, ut dixi, ridicula sunt.

In sequentibus epochis quanuis
non ea occurrit obscuritas, quae in prioribus: tamen semper aliquid
noue demonstratur, praeter superiorum scriptorum consuetudinem:
in quibus sunt quaedam de vero die et anno natalis Alexandri, eiusque
obitus: de Encaeniis Machabaei, de initio Simonis Iudaeorum
Ethnarchae, quem Iudaei Iohannem vocant, de aera Hispanica.

De quibus omnibus pluria nova disseruntur, quam trita et vulgaria.

Iam
excessum Herodis ad suum verum annum ex Iosepho retulimus,
qui ad epocham Actiacam illud tempus diligenter exigit, et praeterea
notationem, cui contradici non possit, adducit, defectum Lunarem,
qui contigit \textsc{ix} Ianuarii, anno 45 Iuliano ineunte, in cuius
anni sequenti Decembri Dionysius Exiguus imperite statuit natalem
Christi, nouem solidis mensibus scilicet post excessum Herodis.

Itaq; diligentissimus \textgreek{[Greek]} omnium scriptorum Iosephus
recte ait decessisse \textsc{xxxv} anno labente regni eius a captis a Sofio[?]
Hirosolymis. in quo tamen interpretatio adhibenda.
% Sofio or Sosio

Nam reuera Herodes
obiit anno tricesimo sexto ex diebus aestiuis noni anni Iuliani.

Ergo tricesimus sextus annus Herodis iniuit ex diebus aestiuis anni
Iuliani \textsc{xliiii}.

Obiit autem initio Nisan.

Igitur sine dubio decessit
anno Iuliano \textsc{xlv}, qui erat tricesimus sextus iniens ex diebus aestiuis,
ut diximus.

Sed ex computatione civili Iudaeorum, nondum
\textsc{xxxvi} annus iniuerat.

Iosephus enim, et Iudaei eo saeculo putabant
omnia tempora a \textsc{xxiii} Ijar, ut albi ostendimus: cuius consuetudinis
ignoratio multos decepit.

Ab Ijar igitur Hyrcani, sive, ut Iudaei
vocant, Iohannis Hasmunai, tricesimus sextus annus Herodis inibat,
qui tamen iam nouem mēsibus ante ex consuetudine Romana iniuisset.

%\end{parnumbers}
\clearpage
p. XVII [pdf 44]
%\begin{parnumbers}

Itaq; eius decessus confirmatur primum accurata putatione
diligentissimi scriptoris, deinde notatione eclipsis, quae omnem contradictionem
excludit.

At ex epilogismis Eusebii Herodes obierit
anno Iuliano \textsc{lii}, septem annis solidis post illum defectum.

qui stupor non meret castigationem, cum tanquam sorex indicio suo perierit.

Nam statim ab eius decessu tetrarchiam suam Archelaus eius filius
iniuit: quod quidem, si huic oraculo Eusebiano credimus, contigerit
anno Christi Dionysiano septimo labente.

Ergo Christus fuerit
annorum septem, cum ex Aegypto monitu Angeli reuoctus est.

Quod est ridiculum.

Rursus anno decimo regni, aut tetrarchiae suae
Archelaus ab Augusto relegatus est Viennam Allobrogum.

Secundum tempus ab Eusebio determinatum, hoc contigerit anno Iuliano
\textsc{lxi}, qui erat annus Tiberii tertius currens, biennio absoluto
post excessum Augusti.

Hoc modo anno tertio excessus sui Augustus
Archelaum relegauerit.

Vides \textgreek{[Greek]}.

Atqui innumeros videas,
quibus hoc somnium placet.

Nam sane omnes fere Chronologiae
et Annales hoc stigmate inusta sunt.

Atque utinam in illis hominibus
non esset vir eximia doctrina praeditus Dominus Caesar Baronius,
Annalium Ecclesiasticorum scriptor, cuius operis copia nobis
facta est ab amicis, cum haec \textgreek{[Greek]} scriberemus.

Is eruditissimus
vir ex hoc loco Eusebii Iosephum exagitat, tanquam imperitum
temporum: cum Eusebius potius ex Iosepho castigandus fuisset.

Nam absque Iosepho esset, quid certi de Herode haberemus?

Quis haec tractauit, praeter illum?

Qui fieri potuit, ut scriptor, cuius diligentia
et fides in notatione temporum spectatissima, in iis peccauerit,
quae sine illo Eusebius et alii ignorassent?

Sed ipse doctus Annalium
conditor potest iam videre, utri fides de hac re habenda, Iosepho,
cuius ratiocinia cum motibus caelestibus congruunt, an Eusebio,
cuius sententia et historiae, et rationi aduersatur?

Sed de Iosepho
nos hoc audacter dicimus, non solum in rebus Iudaicis, sed etiam
in externis tutius illi credi, quam omnibus Graecis, et Latinis.

Itaque
definat mirari doctus vir, cur tot eruditi, et nos quoq; qui non in illis
eruditis, sed in huius scriptoris lectione peregrini non sumus, tantum
illi deseramus, cuius fides et eruditio in omnibus elucet.

Caeterum de Eusebii anilibus hallucinationibus, praeter hanc, quam
modo protulimus, satis libro sexto differuimus.

Sed ad Epochas
nostras venio: quarum omnium rationem reddere longum esset.

De Epocha Martyrum Diocletianea non possumus tacere, eam hactenus
etiam doctissimis imposuisse, quod eam ab initio Diocletiani
incipere omnes credunt.

Hinc prodigiosi errores, et magna Consulum
confusio in Annales et Fastos deriuata sunt, praesertim in annis.

%\end{parnumbers}
\clearpage
p. XVIII [pdf 45]
%\begin{parnumbers}
Nam initio Diocletiani perperam sumpto, perperam quoque
persecutionis Epocha initur.

Ea semper antiquitus a solis Aegyptiis
Christianis hactenus usurpata fuit.

Itaque Historici et Chronologi,
qui temporibus Caroli Magni dicunt caeptum putari ab annis
Christi, cum antea mos esset annis Diocletiani uti, errant.

Nam
nullis nationibus in usu fuit.

Vnica autem Ecclesia duntaxat Alexandrina,
et quae illi subditae sunt, hac Epocha vsa est semper, utirurque
hactenus, et vocatur ab Aegyptiis, qui Elkupt dicuntur,
\textarabic{[Arabic]} \textit{Aera Martyrum sanctorum.}

Nam
hallucinatus est ille, qui nuper \textarabic{[Arabic]}
\textit{Captiuitatem} vertit in literis
Alexandrinae Ecclesiae Romam missis, anno Martyrum 1310, qui
erat Christi 1593.

Epocha igitur Martyrum iniuit \textsc{xxix} Augusti,
id est, neomenia Thoth Actiaci, vel Mascaram Habesseni, anno Christi
Dionysiano 284.

Initium autem imperii Diocletiani a Palilibus
anni 287.

Differentia anni duo, menses octo.

Perturbatio, quae est in
Consulibus a temporibus Maximinorum, vsq; ad filios Constantini,
ea utique ab antiquo est.

Sed et non minor confusio in annis persecutionis:
ubi magnae sunt \textgreek{[Greek]} apud Eusebium: quanuis
recte sentit de initio Diocletiani, et primo anno persecutionis.

Tamen
omnium Chronologorum fides hac in parte nutat.

Nam edictum
Diocletiani de tradendis codicibus prius est Ecclesiarum euersione,
euersio Ecclesiarum prior caede Martyrum.

Felix Africanus Episcopus
et socii eius supplicio in Campania affecti ideo, quod codices
Deificos, id est, sacram scripturam tradere noluissent.

Itaque in
Actis illorum scriptum fuit: \textit{Et ductus est ad passionis locum, cum etiam
ipsa Luna in sanguinem conuersa est, die tertio Kalendas Septembris}.

De Eclipsi
Lunari loqui manifestum est, cuius is color fuerit, quem sanguineum
astrologi vocant: cuiusmodi proculdubio accidit anno Christi
301, cyclo lunae 17, annis quatuor solidis ante edictum de euertendis
Ecclesiis, idque \textsc{iii} Nonas Septembris, non autem \textsc{iii} Kal.
Septembris, diebus quatuor post passionem Martyrum.

Itaque perturbatus
est ordo verborum.

Legendum enim videtur: \textit{Et ductus est ad passionis
locum, die tertio Kal. Sept. cum etiam ipsa Luna in sanguinem conversa
est.}

Id est, quo tempore Luna defecit, proximo nimirum novilunio.

Nam cum constet passos \textsc{iii} Kal. Septembris, et ita habeat
Kalendarium, non videtur esse error in notatione temporis.

At Dominus Baronius haec gesta confert in annum 302, tribus annis ante
persecutionem: et tamen putat eum esse secundum annum persecutionis,
qui erat decimus nonus Aerae Martyrum, decimus autem
septimus currens ab imperio Diocletiani.

%\end{parnumbers}
\clearpage
p. XIX [pdf 46]
%\begin{parnumbers}
Sed \textgreek{[Greek]} illorum
Annalium propagati sunt partim ex erroribus aliorum Chronologorum,
quos auctor sequitur, partim ex annis Christi male ad
suam et veram epocham reductis.

Vnde factum, ut ap initio operis,
ad tempora Nicenae synodi, ne unus quidem annus Christi
verae epochae suae redditus sit.

Itaque triennio aliquando, aliquando
quadriennio, ut plurimum autem biennio erratum est.

Exempli
gratia: Excidium Hierosolymorum contigit anno Christi
Dionysiano \textsc{lxx}, quo neomenia Nisan conueniebat cum neomenia
Xanthici, teste Iosepho.

In Annalibus refertur ad annum
72: qui est error Eusebii, sed alibi ab eodem castigatus.

Certum est, Fructuosum Episcopum, Christi Martyrem, cum fociis
passum anno antequam pax et interspiratio data esset Ecclesiis
sub Marco Aurelio Antonino, et L. Aelio Vero.

quod tempus Eusebius confert in annum quartum Olympiadis \textsc{ccxxxiiii},
id est Christi Dionysianum 160.

Ergo passus est Fructuosus anno Christi
159.

Hoc aliter demonstrabimus.

In Actis agonis Fructuosi et
sociorum legitur: \textit{Producti sunt duodecimo Kalend. Februarii, feria
sexta.}

Ergo litera Dominicalis erat B.

Proinde hoc accidit anno
159, triennio citius, quam notatum in Annalibus.

In Actis Andreae
militis et sociorum scriptum extat, eos necatos fuisse decimoquarto
Kalendas Septembris, Dominico die, hora secunda.

Igitur litera Dominicalis erat G.

Hoc necessario contigit anno 305,
qui erat primus persecutionis a Pascha illius anni antecedente, post
euersas Ecclesias: quod quidem Pascha celebratum 25 Martii, ipso
die termini.

At in Annalibus hoc refertur in annum 301, quadriennio
ante rem gestam.

Rursus in Epistola Vigilii Episcopi Tridentini
de Passione Sanctorum Sisinnii, Martyrii, et Alexandri,
ita legitur: \textit{Die paßionis Sanctorum, quarto Kalendas lunias, feria
sexta, nascente luce.}

Passi ergo sunt anno 403, cyclo Solis \textsc{xx}, quando
\textsc{xxix} Maii erat feria \textsc{vi}.

At in Annalibus dicitur scripta
anno 400 Christi.

Scripta ergo fuisset triennio ante caedem
ipsorum Martyrum.

Cui absurditati ipse non adscribet, certo scio.

In iisdem
Annalibus ex codice Antonii Augustini mentio fit Homiliae
Cyrilli Episcopi dictae in natiuitate Ioannis Baptistae, Pharmuthi
vicesima octaua, indictione prima, sub Theodosio iuniore et Valentiniano.

Ergo dicta fuit Homilia anno Christi 433, April. vicesima
tertia.

At in Annalibus refertur in annum 432, April. 29. S. Benedictus
Monachorum Occidentis Pater, obiit \textsc{xi} Kal. Aprilis, Sabbato
sancto, ut refert Aimoinus monachus ex Actis S. Mauri ipsius
Benedicti discipuli.

Toto illo saeculo hoc non potuit contingere, nisi
anno 536.

%\end{parnumbers}
\clearpage
p. XX [pdf 47]
%\begin{parnumbers}
Tamē in Annalibus Ecclesiasticis obitus Benedicti confertur
in annum 542, sex annis serius.

Multa igitur peccari necesse est
in Gestis Benedicti, quae in illis Annalibus referuntur.

In Encyclica
epistola Vigilii Papae scriptum fuit: \textit{Piißimus atque clementißimus
Imperator Dominico die, id est, Kalendis Februarii, gloriosos Iudices suos
ad nos destinare dignatus est.}

Anno 554 Kalendis Februarii fuit dies
Dominica.

At in Annalibus hoc confertur in annum 552, duobus
annis citius.

Anno 546 turbatio facta in Pascha, ut ex Cendreno docuimus,
capite de periodo Dionysiana, libro \textsc{iiii}.

In Annalibus referetur
sub anno 545.

Martinus Episcopus Turonensis obiit anno
395, ut accurate a nobis disputatum est.

Auctor Annalium Sigebertum
sequutus coniicit in annum 402.

Ex eo errore multum peccatum
est in temporibus Regum Francorum.

de quibus consulatur vltima
diatriba libri sexti huius operis nostri.

Non semel monuimus magnam
perturbationem esse in initiis Imperatorum, a Maximinis
ad Valentinianum.

Vt alios taceam, Constantini initium ab aliis in
305, ab aliis in 306 annum coniicitur.

At Constantinus iniuit imperium
post obitum patris sui Chlori.

Obiit autem Chlorus in Britannia
anno primo Olympiadis 271, ut inquit Socrates.

Nos ostendimus,
apud Socratem, Hieronymi Supplementum, Ausonium, et alios,
semper Olympiadem sumi pro lustro Iuliano, non pro lustro Olympico
Elidensium, idq; lustrum Iulianum biennio posterius esse Elidensi,
cum incipiat ab anno Iuliano bisextili.

Itaq; is fuit annus bisextilis,
quo obiit Chlorus, et imperium iniuit Constantinus.

Sed duae
cautiones adhibendae.

Prior est, ut scias annum Constantinopolitanum,
sive Nicenum hic intelligi, qui incipiebat a \textsc{xxiiii} Septembris.

Altera, ut prolepsis usurpata intelligatur in anno mortis Chlori.

Nam obiit \textsc{xxv} Iulii, \textsc{lxi} diebus ante
\textsc{xxiiii} Septembris, et
tamen obitus eius ad eundem annum refertur quo iniuit imperium
eius filius, \textgreek{[Greek]}, ut dixi.

Omnino igitur iniuit imperium anno
303, aut 307.

Nam primus annus Olympiadis Iulianae incipit semper
diebus 153 ante bisextum.

Sed nemo concedet Chlorum obiisse
anno 303.

Obiit ergo 307.

Et proinde anno 307 iniuit imperium
Constantinus, ex ante diem \textsc{viii} Kal. Octobr. eiusdem anni 307.

In his prouocamur a docto Annalium scriptore, et rem absurdissimam
prodidisse nos dicit, Constantini imperium iniisse ex anno
308, cum, ut inquit ipse, iniuerit anno primo Olympiadis 271,
Christi vero 306[?].
% 300 or 306 ?

Nos vero negamus vllam culpam aut absurditatem
in nobis admissam.

Nam annus Christi 308 Constantinopolitanus
incipit a Septembri anni 307, ut iam dictum est.

Et proinde ipsum, et alios errare, qui annum Christi 306 a Kalendis Ianuarii
dicunt esse annum labentem Constantini.

%\end{parnumbers}
\clearpage
p. XXI [pdf 48]
%\begin{parnumbers}
Hoc enim volunt,
cum putant primum 271 Olympiadis Elidensis annum esse primum
Constantini.

Olympias enim illa Iphitea caepit ex diebus aestiuis
anni 305, qui fuit annus primus presecutionis.

Quare in annis
Constantini, ut in aliis, insigniter peccatum est a viro docto.

His
postis, quinquennalia Constantini data sunt anno 312: vicennalia
autem anno 327.

Interuallum inter illas duas celebritates interiectum
haud dubie vocatur Indictio, iniens a datis quinquennalibus,
desinens[?] in vicennalibus, quibus concilium Nicenum dimissum.
% desinens or definens?

Sed neq; hoc placet Domino Baronio: neque caussam appellationis
Indictionum admittit.

At nos dicimus, non minus iniuste nos
hic, quam in initio imperii Constantiniani reprehendi.

An negat
Indictiones in quinquennia indici, et in quinquennalibus Principum
panegyribus remitti?

Si non credit, legat et quae priore, et quae
hac editione ad eam rem collegimus.

Quinquennalia illa dicuntur
\textgreek{[Greek]}, hoc est ad verbum, sparsiones, largitiones, profusiones, in
quibus liberalitas Principis ad remissionem vsq; tributorum, et indictionum,
editiones munerum et spectaculorum, congiaria, et donatiua
extendebatur.

Inde \textgreek{[Greek]} non solum pro illa largitione
sumitur, sed et pro ipsa indictionis temporalis nota.

Nam quod Latini
dicunt, Indictione prima, secunda, tertia hoc factum est, Graeci
dicunt, \textgreek{[Greek]}.

Non ergo nos, sed ipse fallitur.

Quid?

si initium Constantini a nobis ignoraretur, tamē quinquennalia
eius nos manu ad illud deducerent.

Itaque ignorari n n [?]
potest.
% Probable printing error. "n n" should read "non".

Neq; minus errat, cum cladem Maxentii coniicit in annum
312.

Quot modis enim hoc refelli potest?

Sed de eo suo loco.

Nam
Maxientius anno 313, non 312 extinctus est, ut recte Panuinius notat,
sed male inde Indictionum initia et caussas repetit: quod a nobis
olim diligenter discussum fuit.

%%% === Sextus Liber
Sextus liber continet residuum Epocharum,
in quo nobiliores quaestiones de Natali die, et Passione Christi,
de Hebdomadibus Danielis, quae breuibus diatribis explicari
non possunt, presequimur.

Ne autem aut rudiores, aut refractarii auctoritate
veterum scriptorum nobis praescribere possent, pauca de
Eusebii erroribus in antecessum delibauimus, in quibus, praeter frequentes
\textgreek{[Greek]}, puerile illud deliramentum de Effenis confutauimus,
quos Christianos fuisse hoc unico argumento probat, quod
\textgreek{[Greek]} essent, et solitarie viuerent, et monasteria haberent:

quasi Bonzios
Iapanensium Christianos esse censeamus, quia et coenobitae sunt,
et Psalmos quosdam instar monachorum Europaeorum alternis modulantur,
et horas Canonicales eorum exemplo habent.

Eorum Essenorum alii \textgreek{[Greek]}, alii \textgreek{[Greek]} fuerunt.

Sed horum non videtur
secta diuturna fuisse.

%\end{parnumbers}
\clearpage
p. XXII [pdf 49]
%\begin{parnumbers}
Ast \textgreek{[Greek]}, aut eorum non dissimilium
synagogae fuerunt ad tempora Iustiniani.

Sunt enim ii, qui Caelicolae
vocantur.

Nam et nomen id indicat.

Caelicolae enim sunt
Angeli.

Ita vocari volebant, propter sanctum, et caeleste, ut ipsis videbatur,
vitae institutum.

In perueteri Glossario Latinoarabico \textit{Caelicola}
[Greek][Arabic][?]. id est, Angelus.

Praeterea quia erant \textgreek{[Greek]}, novi
baptismi auctores Donatistis fuerunt.

Princeps eorum vocatur
Maior, ut et aliorum Iudaeorum.

Hoc enim est \texthebrew{[Hebrew]}.

Philo dubitans
quare Esseni illi dicti sint \textgreek{[Greek]},
 utrum quia medicinam profiterentur,

an quia Deum colerent, ex eo coniiciendum relinquit,
eos non dictos esse quasi \texthebrew{[Hebrew]} \textgreek{[Greek]},
 ut volebat quidam Lunaticus
literarum Hebraicarum professor, sed quia \textgreek{[Greek]} vocat, eo ostendit
\texthebrew{[Hebrew]} dictos, hoc est, \textgreek{[Greek]}.

Quod Christiani non essent, sed
mere Esseni, statim initio libri ostendit Philo.

sed et Sabbati summus
cultus, et reliqua, quae a Philone de ipsis narrantur, satis leuitatis
damnant Eusebium, et reliquos veteres, qui Eusebium sequuti,
idem hariolati sunt.

Sed in Annalium tomo primo tacite perstringitur
sententia nostra ab auctore, qui tamen fatetur veros Essenos Iudaeos
fuisse.

Mirati sumus, quomodo ille putauit in unum haec bene
conuenire posse, Iudaismum et Christianismum.

Vt hoc probet, ait
veteres patres idem scribere, quod Eusebium.

Atqui ex Eusebio
hoc desumpserunt, et eius auctoritate contenti Philonem non consuluerunt.

quem si legissent, nunquam tam ridiculae sententiae assensum
accommodassent.

Haec vero puerilia sunt.

Venio nunc ad natalem
Christi, quem vetustas Christianismi ad \textsc{xxviii} annum Actiacum
retulit, recte.

Nam Christus iniens annum unum a tricesimo
aetatis suae accessit ad baptismum, ut omnes vetustissimi Patres ex
Luca retulerunt, et post eos eruditus Annalium scriptor.

Baptizatus est anno \textsc{xv} Tiberii, duobus Geminis \textsc{coss}. anno
Iuliano 74.

Ergo \textsc{xxv} Decembris anni 73 illi inibat annus primus a tricesimo.

Deductis 30 annis absolutis de 73, remanet annus Iulianus
43, in cuius \textsc{xxv} Decembris natus fuerit Dominus, cyclo Lunae
\textsc{xviii}, anno Actiaco \textsc{xxviii},
 ut illi vetustissimi partes crediderunt,
duobus annis solidis ante epocham hodiernam Dionysianam,
anno solido cum diebus aliquot ante excessum Herodis.

Hoc proculdubio
verum est.

Sed in Annalibus peccatur ab auctore in anno
\textsc{xv} Tiberii.

Quem enim putat \textsc{xv}, is est \textsc{xvi}, et magno errore illi
attribuit Consules duos Geminos, quibus Consulibus annus \textsc{xvi}
Tiberii iniit ex \textsc{xix} Augusti, cyclo Lunae undecimo, anno Iuliano
74.

Nisan igitur is, qui proxime sectus est baptismum Christi,
Consulibus duobus Geminis, antecessit annum \textsc{xvi} Tiberii ineuntem,
mensibus quinque.

%\end{parnumbers}
\clearpage
p. XXIII [pdf 50]
%\begin{parnumbers}
At scriptor Annalium putat duos Geminos
Consulatum gessisse cyclo Lunae \textsc{xvi}: in quo ne sic quidem
sibi constat.

Nam is fuerit annus 75 Iulianus iniens.

Hoc modo Decembri anni 74 Christus iniuerit annum primum a tricesimo: et
deductis 30 absolutis, remanebit annus 44 Iulianus, in quo natus
Christus fuerit, tribus circiter mensibus ante excessum Herodis, anno
solido ante epocham Dionysianam, qua hodie Ecclesia utitur.

quae sane multorum veterum, inque illis Eusebii fuit opinio.

Sed
Christus baptizatus anno 74 Iuliano: passus 78.

Differentia, anni
quatuor solidi, paschata quinque.

Quorum nullum vestigium in illis
Annalibus extat.

Quinetiam auctor, quando numerus annorum
non succedit ex voto, culpam in Iosephum reiicit, mendacem multis
modis arguens: inter alia, quod scripserit \textgreek{[Greek]} factam post
Archelai relegationem, cum, inquit, ea \textgreek{[Greek]} Christo nascente
contigerit, et aperte Eusebius id indicauerit.

Nos hallucinationem
Eusebii loco suo confutauimus, in quo descriptionem patrimonii
Archelai cum descriptione totius orbis Romani confundit more
suo, neq; meminit verbis illis, \textgreek{[Greek]}, designari non
unicam fuisse illam descriptionem, cum \textgreek{[Greek]} mentio fiat.
% Final period not visible in original.

Quare
idem Euangelistes quemadmodum prioris meminit in Euangelio,
ita alterius mentionē facit in Actis.

ut non sit audiendus doctus Annalium
scriptor, qui non solum hac in parte Eusebii auctoritatem
Iosepho opponit, sed etiam adiicit descriptionem illam[?] eandem esse,
de qua Aethicus statim initio libri sui loquitur: cum tamen neque
tempus, neque res conueniat[?] descriptioni nascente Christo factae.

Nam descriptio, de qua intelligit Aethicus, caepit ab anno caedis
Caesaris, desiuit[?] in anno \textsc{xxxiii},
 qui erat tricesimus quartus a primis
Kalendis Ianuariis Iulianis, decem annis absolutis ante verum
natalem Christi, duodecim ante epocham Christi hodiernam Dionysianam.

Res autem eadem non est, imo longe diuersa: atq; adeo
tantem differt[?] descriptio, de qua Aethicus loquitur, a descriptione,
quae facta Christo nascente, quantum decempeda, et tabulae [censuales][?].

Nam illa descriptio Aethici mandata est agrimensoribus, et
Geometris, haec Rationalibus.

Illa orbis mensura, \textgreek{[Greek]},
hac census et facultates in Tabulas relatae.

Sed neq; recte concludit,
Iosephum hallucinatum, quod paulo ante initia belli Iudaici
auditam ex adytis templi vocem scripserit, quae diceret \textsc{hinc
migremvs}: cum, inquit, Eusebius id in passionis Dominicae tempus
referat.

Quomodo Eusebius melius scire potuit ea, quae contigerunt
Christi et belli Iudaici tempore, quam Iosephu? aut unde,
quam ex Iosepho? de illis dico, quae non pertinent ad historiam euangelicam.



%\end{parnumbers}
\clearpage
p. XXIV [pdf 51]
%\begin{parnumbers}
Sed tam friuolum argumentum eluditur iis, quae aduersus
hanc Eusebii hallucinationem libro sexto decimus.

Denique iniuste
vbique Iosephum reprehendit, omnium scriptorum veracissimum
et religiosissimum, quod quidem ipsius scripta loquuntur.

quem
auctorem si non tam contempsisset, nunquam eos
 \textgreek{ανἀχρονισμους [Greek:anachronism]} commisisset,
quibus totus contextus temporum primi tomi perturbatus
est.

Sed antequam ex hac velitatione facessimus, qua et nos et cognominem
nostrum scriptorem ab animaduersione docti viri vindicamus,
nos homines Aquitani expostulamus cum eo, quod a nobis
tres summos viros abdixit, Paulinum, Phoebadium, et Sulpitium
Seuerum:

qui cum suerint natione, et domo Aquitani, tamen
Paulinum et Sulpitium Romae natos scribit, Phoebadium in Hispania.

Quis illum docuit Paulinum non esse natum Burdigalae, ubi
antiquitus Paulina gens, hodieque quaedam regio vrbis Burdigalensis
Paulino cognominis est?

Phoebadium autem Aginni Nitiobrigum
Episcopum quare in Hispania natum dicit, aut quo auctore?

Apud Hieronymum male excusum est Soebadius, qui error irrepsit
ex Sophronio, ubi legitur \textgreek{[Greek]}.

Sed liber manu scriptus
Sanctae Mariae de Granateria liquido habet Febadium.

Apud Sulpitium
Seuerum deprauatum quoq; est, ubi legitur Fegadius, pro
Febadius, ut quidem librarii scribunt.

nam orthographia est \textgreek{[Greek]},
Phoebadius: satis hodie notus erudita sua in Arrianos Epistola,
quae ante \textsc{xxv} annos primum edita.

Mei municipes Fiarium vocant,
cuius memoriam bis quotannis instaurant, ineunte ieiunio
quadragesimae, et die Marci Euangelistae, mense Aprili, si bene
memini.

Huic successit Gauidius in episcopatu.

Sulpitium Seuerum
nemo hactenus Aquitanum fuisse dubitauit: sed patria ignoratur,
cum tamen ipse Nitiobrigem sese manifesto prodat, cum Seruationem
Tungrorum, Phoebadium autem suum Episcopum fuisse scribit.

Phoebadius autem erat Nitiobrigum Episcopus.

Iste Sulpitius
Ecclasiasticorum purissimus scriptor, post transitum Martini recepit
sese Elusonem, quo tempore ad eum scribebat Paulinus.

Id oppidum est cum arce veteri in finibus Nitiobrigum, qua amni Draguto
a Petrocoriis diuiduntur.

Vulgo \textit{Lausun}.

Sed de hoc satis.

Mei
Nitiobriges pro Sulpitio Supplicium dicunt, quomodo et Bituriges
suum illum vocant, quem eundem cum hoc faciunt perperam,
cum inter transitum Martini, cuius noster Sulpitius discipulus fuit,
et ordinationem Sulpitii Episcopi Bituricensis sub Guntchramno
Rege, intercedant plus minus anni 190.

Non iniuriam facimus
docto viro, si cum bona eius venia doctissimos viros Aquitanos,
et Christianissimos originibus suis vindicamus.

%\end{parnumbers}
\clearpage
p. XXV [pdf 52]
%\begin{parnumbers}
Sed quemadmodum tribus viris Aquitaniam orbaverat, ita eandem duabus
alienis civitatibus donavit, Reiensi, et Vasensi.

Prosperum non uno
loco dicit Regiensium in Aquitania fuisse Episcopum, cum dicendum
fuerit, Prosperum Aquitanum fuisse Episcopum Reiensium,
aut Regiensium in secunda prouincia Narbonensi.

Hodie \textit{Ries} vocatur.

Nugantur qui eum Regii Lepidi Episcopum et scripserunt,
et in fronte eius sacrorum poematum apponi curarunt: quasi Reienses,
in secunda prouincia Narbonensi, iidem sint cum Regio Lepidi
in Aemilia.

Vasense autem consilium idiotismus illorum temporum
vocauit, quod potius Vasionense dicendum erat.

Vasio Vocontiorum hodie \textit{Vaison} dicitur.

Est Episcopatus Auenioni metropoli
attributus.

Imperite quidam cum foro Vocontiorum confundunt.

Itaque Vasense, vel Vasionense, in Vasatense mutandum non
erat.

quemadmodum in anno Christi 552 perperam Firminum
Vticensem mutat in Venciensem.

Vticenses, vulgo dicuntur \textit{Vsetz}.

Est Episcopatus in prima Narbonensi.

Dicuntur etiam Vcetenses,
et Vcetiae Episcopus.

Apud Gregorium Turonensem libro \textsc{vi},
mentio est Ferreoli Episcopi Vcetensis: ubi vulgo male Vcecensis.

Sed tam imperite vulgus Vticenses deprauauit in Vcetenses, quam
Arausio in Aurasio: Vasensis dixit, pro Vasionensis.

At ciuitas sive
Episcopatus Venciensis, est in secunda Narbonensi. Vulgo S. Paulus
de Venciis.

Scribendum vero per t.[?] Ventiensis, \textgreek{[Greek]} enim dicitur
Ptolemaeo.

Fuitque Nerusiorum in Alpibus Graiis Metropolis.

Sequuntur in sexto libro illa quinque Paschata a baptismo
ad resurrectionem, fuis temporibus, Consulibus, et cyclis notata.

In tertio Paschate quid fit \textgreek{[Greek]}, explicamus,
quae verissima interpretatio adhuc assensum vel mereri, vel
exprimere a doctis hominibus non potuit: quod valde miror,
cum absurdissima sit ea, quam sequuntur ipsi.

Omnes igitur uno
ore putant \textgreek{[Greek]}, pro \textgreek{[Greek]} dictum esse.

Id ad verbum
Hebraice esset \texthebrew{[Hebrew]}:
 aliter \textgreek{[Greek]}, Latine Praeposterum.

quo nihil praeposterius dici potuit.

Nam quid est praeposterum Sabbatum?

Non pudet iocularis interpretationis?

Sed ita est.

Alius fortasse assensum extorsisset.

Sed quia a nobis, ideo
minus acceptum.

Theophylactus post Epiphanium, et alios veteres,
interpretatur \textgreek{[Greek]}.

Itaque verum est, quod diximus, omnes tam veteres, quam
recentiores \textgreek{[Greek]} interpretari
 \textgreek{[Greek]}, id est \textgreek{[Greek]},
praeposterum.

Vt illud probet, idem Theophylactus
subiicit: \textgreek{[Greek]}.

%\end{parnumbers}
\clearpage
p. XXVI [pdf 53]
%\begin{parnumbers}
Quod falsum est,
propter translationes, quas imperiti negant saeculo Christi usurpatas,
cum tamen longe ante Christum in usu fuisse demonstrauerimus,
ut locus non sit pertinaciae.

Sed valeant Sabbata praepostera.

Igitur \textgreek{[Greek]}, non quod \textgreek{[Greek]}, sed
quod \textgreek{[Greek]}.

Nam \textgreek{[Greek]} inibat
computus \textgreek{[Greek]}.

Hebraei etiam hodie vocant \texthebrew{[Hebrew]}
\textgreek{[Greek]}: id est, Sabbatum, quod est primum a
\textsc{xvi} Nisan, quae est \textgreek{[Greek]}.

Vera, et recta interpretatio sine ulla praeposteritate.

De quinto Paschate verbum non addidissem,
nisi cum haec commentarer, incidissem in Commentarios quorundam,
qui Christum passum volunt ipso solenni Paschatis, \textsc{xv}
Nisan, feria sexta, nempe parasceve Sabbati.

Quanuis auctoritas
Evangelistarum, ratio ipsa, doctrina veteris anni Iudaici, omnia
denique contra illos faciant, tamen potius etiam Evangelistas ipsos
valere iubebunt, quam ut sententiam mutent.

Caussa pertinaciae
verba Evangelistarum, \textgreek{[Greek]}.

Contra obiicitur:
\textgreek{[Greek]}.

Iohan. \textsc{xix}, 14.

Hic miseram latebram quaerunt,
et strenuo mendacio ictum declinant.

Aiunt \textgreek{[Greek]} tantum
dici de Sabbato.

Acuti homines!

Quare dicit \textgreek{[Greek]},
nisi \textgreek{[Greek]}, quasi et sit \textgreek{[Greek]}
 alius rei, quam \textgreek{[Greek]}?
 
Quod quidem verum est.

Nam \textgreek{[Greek]} est genus, cuius species
\textgreek{[Greek]}, \textgreek{[Greek]}.

Utrumque uno verbo Hebraeis est
\texthebrew{[Hebrew]}.

Itaque \texthebrew{[Hebrew]} \textgreek{[Greek]},
 sive \textgreek{[Greek]} dicitur,
\textgreek{[Greek]}, quod sint aliae \textgreek{[Greek]},
 sive \textgreek{[Greek]}, quae a festis
suis appellationem sortiuntur.

\texthebrew{[Hebrew]} \textgreek{[Greek]}, \textgreek{[Greek]}.

\texthebrew{[Hebrew]} \textgreek{[Greek]}, \textgreek{[Greek]}.

Itaque ei parascevae, in qua Christus passus, accidit tum
vt ex consuetudine esset \textgreek{[Greek]}, id est, \textgreek{[Greek]}:
tum, ut casu \textgreek{[Greek]}, id est, \textgreek{[Greek]}.

Sabbatum enim
illud erat \textgreek{[Greek]}.

Quae propterea dicitur \textgreek{[Greek]}
ab Evangelista.

\textgreek{[Greek]}.

Quod
et ipsum quoque \textgreek{[Greek]} dictum, tanquam sit quaedam
 \textgreek{[Greek]},
quae non sit \textgreek{[Greek]}.

Nam quod Hebraice dicitur \texthebrew{[Hebrew]},
id est, Solenne, id \textgreek{[Greek]} Iudaei, et Apostolus vocant
 \textgreek{[Greek]}.

Unde \textgreek{[Greek]} sive \textgreek{[Greek]} solenne \textgreek{[Greek]}
dicitur \textgreek{[Greek]}, ut supra \textgreek{[Greek]} citavimus, nimirum
exemplo Iudaeorum, et Samaritarum, qui \textgreek{[Greek]}
\texthebrew{[Hebrew]} vocant \textgreek{[Greek]}.

Tria autem proprie Hebraice vocantur \texthebrew{[Hebrew]},
aliter \texthebrew{[Hebrew]}, quas \textgreek{[Greek]} quoque vocarunt
 \textgreek{[Greek]}, \textsc{xv} Nisan, id est, \textgreek{[Greek]},
 cum \textsc{xxi} Nisan.

\textsc{vi} Sivvan,
id est, \textgreek{[Greek]}.

\textsc{xv} Tisri, id est, \textgreek{[Greek]}, cum \textsc{xxii} Tisri.

%\end{parnumbers}
\clearpage
p. XXVII [pdf 54]
%\begin{parnumbers}

Acron quidem in illud \textit{– hodie tricesima sabbata} scholio isto
\textit{quae Neomenias esse dicunt: quoniam per Sabbata Iudei numeros Lunares
accipiunt.}

\textit{Et Sabbatum magnum in renovatione Luna a Iudaeis
hodie celebratur}: videtur omnes neomenias nomine Sabbati
magni indigetare: sed, quid sit Sabbatum magnum, ignorat.

At
Sabbatum ordinarium nunquam dicitur \textgreek{μεγάλη ἡμέρα},
% Great Day
 non magis,
quam \texthebrew{[Hebrew]}.

Sic Philo libro \textgreek{[Greek]} dixit \textgreek{[Greek]},
loquens de Pentecoste.

At ordinarium
Sabbatum nunquam dicitur \textgreek{[Greek]}, aut \textgreek{[Greek]}.

Temere igitur
doctor Theologus Commentario in Iohannem ait \textgreek{[Greek]}
tantum dictam de feria sexta, et omne Sabbatum dici posse \textgreek{[Greek]}.

Ecquae Grammatica haec est, ut \textgreek{[Greek]}
non sit \textgreek{[Greek]}?

Quid potest dici absurtius?

Ac propterea
longe iocularius dicit eodem modo dictum \textgreek{[Greek]},
ut Ioh. \textsc{vii}. 37. \textgreek{[Greek]}.

Nam verum est eodem modo dictum, sed
contra animi eius sententiam: quod nimirum dicta sit \textgreek{[Greek]},
quia octava Tabernaculorum, non quia Sabbatum.

Quae quidem
octava fuit eo anno feria quinta, non Sabbatum, 169 diebus ante
passionem.

Itaque dum hoc effugium parat, suo se gladio iugulat.

Quis unquam tam obstinatos adversus veritatem animos credidisset?

Illam autem obiectionem quam argute eludunt ! [?] \textgreek{[Greek]}.

Aiunt \textgreek{φαγεῖν[?] τὸ πάσχα}, hic non esse
agnum Paschalem comedere, sed alia sacrificia Paschalia.

Argute, docte, eleganter, ut nihil supra: quasi in parasceve comedere
Pascha aliud sit, quam agnum Paschalem comedere.

Imo nos negamus, \textgreek{θύαν κὶ[?] φαγεῖν τὸ πάσχα[Greek]},
 aliud esse, quam agnum Paschalem
immolare, aut manducare.

Exodi \textsc{xii}, 21.

Neque ullus paulo doctior illud sine risu audire potest.

Sed quid ex tot mendaciis consequuntur?

Quid, quam ut se deridendos propinent?

Si quintadecima Nisan, quando Christus passus, fuit feria sexta: ergo Pentecoste
illius anni fuit Sabbatum.

Nam Pentecoste est feria secunda
quintaedecimae Nisan, ut diximus ad Computum Iudaeorum.

Fallitur ergo Ecclesia, et omnis Christianitas, quae ab ultima usque
 antiquitate
credidit illam Pentecosten fuisse diem Dominicam, non
Sabbatum.

Quid igitur?

Quid?

\textit{— Dic aliquem, dic, Quintiliane, colorem.}

Itaque Doctori Theologo non bene procedit commentum.

Rursus urgetur absurditate.

Deus die magno Azymorum districte
vetat opus facere.

Exodi \textsc{xii}, 16.

Levitici \textsc{xxiii}, 7.

Hinc quoque
aliquo insigni facinore elabendum erit [?].

Adducit locum ex libro Iudaico,
cui titulus \texthebrew{[Hebrew]} id est, ligatio Isaaci, ut probetur etiam
Sabbato licuisse[?] opus facere: in quo is, qui locum vertit, imposuit homini
quaestionum, quam Hebraismi peritiori.

%\end{parnumbers}
\clearpage
p. XXVIII [pdf 55]
%\begin{parnumbers}

Nam qui interrogat,
an Sol occasus sit, item, an sit Sabbatum, eandem rem duabus interrogationibus
significat.

Si enim Sol nondum occidit, est Sabbatum,
in quo non auderet mittere falcem in messem.

Oportet igitur,
ut Sol occiderit prius, et consequenter non erit amplius Sabbatum,
id est, iam praeterierit quintadecima Nisan, quae quacunque feria
inciderit, dicitur Sabbatum, Levitici \textsc{xxiii}, 15.

Igitur interroganti,
an Sol occidit, respondetur, occidisse.

Rursus, an sit Sabbatum, id
est, an \textgreek{[Greek]} nondum praeterierit, si respondetur adhuc
esse, nihil agitur: si respondetur non esse Sabbatum, tunc confidenter
immittit falcem in messem.

Haec profector est mens illius loci,
quanquam libri copia non est.

Sed qui vertit illi haec verba, dicit responderi
esse Sabbatum.

Ergo hoc modo oportebat omnem \textsc{xvi}
Nisan esse Sabbatum, omni anno.

Quod quis non miretur a Doctore
Theologo non animaduersum?

Atque adeo illi commentum
placet.

Denique tot lapides movit, ut tandem concluderetur, Ecclesiam
falso putare diem Pentecostes, quando Spiritus sanctus super
Apostolos descendit, fuisse Dominicam, cum fuerit sabbatum, ex
hypothesibus Doctoris.

Concludimus igitur, quod nemo sani capitis
negaverit, Chistum Pascha comedisse tertia decima Nisan civilis,
quartadecima Luna.

Unde recte Evangelistae: \textgreek{[Greek]},
nempe \textgreek{[Greek]}.

Nam sane quoties fit translatio feriae,
tunc duplex est neomenia, prior quidem \textgreek{[Greek]}, posterior vero
\textgreek{[Greek]}.

Sed, inquiet, alius Evangelistes dicit, \textgreek{[Greek]}.

Ergo omnes \textgreek{[Greek]}.

Non sequitur: quare alius interpretatur, \textgreek{[Greek]}.

Christus \textgreek{[Greek]}.

Christur immolavit Pascha in qua die
oportebat, nempe quartadecima Luna.

Iudaei postridie \textgreek{[Greek]},
in qua non oportebat, nempe quintadecima Luna.

Et ita quoque
hunc nodum soluerunt Monachus Veronensis Hilario, et Paulus
Episcopus Burgensis ex Iudaeo Christianus.

Neque melior solutio
dari potest.

Neque vero illi duo viri docti tam vacui capitis fuerunt,
ut crederent eo anno \textgreek{[Greek]} fuisse feriam sextam.

Sed Doctor melius Latine intelligens, quam Graece, vulgatam tralationem [sic]
sequitur: \textit{In qua necessarium erat immolare}.

Nos negamus
\textgreek{[Greek]} bene traductum, \textit{necessarium erat}.

Atque adeo intererat Logici
scrire, quatenus \textit{Oportere, et Necessarium esse} differunt.

Absurde igitur, imperite, et adversus Evangelistarum mentem, dicitur
Christum crucifixum ipsa die solennis Paschatis.

Nos vero et
hic mutavimus sententiam, cum huic stultae interpretationi haereremus
priore editione: quemadmodum cum Chistum cyclo \textsc{xvi} crucifixum
asserebamus, sequuti Dionysium Exiguum, et alios veteres.

%\end{parnumbers}
\clearpage
p. XXIX [pdf 56]
%\begin{parnumbers}

Nam Christus passus cyclo \textsc{xv} Lunae, \textsc{xiiii} Solis,
 tertia Aprilis,
anno quarto absoluto a baptismo, quando Sol extra ordinem caligavit,
cuius casus etiam meminit Phlegon, feria sexta, quando
post verum agnum immolatum, immolatus est et typicus, qui tum
primum perperam immolari ceptus, usque ad 70 annum Christi
Dionysianum, quando inclusis in urbem die primo Azymorum
contigit Pascha ultimum immolare.

Atque hactenus quidem de
priore parte sexti libri.

Venio ad alteram, cuius subiectum quo nobilius,
eo plures tractatores habuit: ut nullus non ex plebe scriptorum
ex hoc mustaceo lauream sibi quaesiverit.

Ac quanquam sine
summa doctrina externae historiae, et peritia bonarum literarum
ad ista arcana penetrari non potest, tamen quo quisque imparatior
ab omni copia humaniorum doctrinarum, eo audacius ad hanc
tractationem se contulit.

Quin etiam tantum abest, ut praesidia,
sine quibus hic labor irritus est, isti adhibuerint, ut eos insanire
putent, qui per illa viam sibi ad haec indaganda muniverunt.

Minima quaeque persequi esset horas perdere.

Tria praecipua attingere
satis pro tempore erit, nempe de septuaginta annis captivitatis,
de Regibus Persidis et Babyloniae, de epilogismo Hebdomadum
Danielis.

Septuaginta annorum caput a capto Iechonia sumendum
esse, auctor Ieremias scribens ad eos, quos com Iechonia Babylonem
Rex Nebuchodonosor deportaverat, cap. \textsc{xxix}, post
alia: \textit{Quia Dominus ita dicit: Quando septuaginta anni Babyloni completi
fuerint, ego visitabo vos, et verbum meum bonum super vos
suscitabo, ut vos huc reducam}.

Quid clarius hoc commate?

Vos,
quos cum Iechonia captivos Babylonem taduxit Rex, ego huc
reducam, postquam septuaginta anni completi fuerint Babyloni,
quae vos captivos detinet.

At contra hos 70 annos acuti homines
ineunt a capto Sedekia.

Ergo septuaginta anni sunt octaginta.

Mirum vero Ieremiam nescisse septuaginta esse septuaginta.

Quemadmodum
igitur negando parasceven Pascha esse parasceven
Pascha, res nova et inaudita concluditor, Spiritum sanctum in 
Apostolos descendisse Sabbato, non die Dominico: ita etiam
negando septuaginta annos esse septuaginta annos, haud dubie
aliquid \textgreek{[Greek]}, \textgreek{[Greek]} parturitur.

Audiamus
caussam tam inopinatae interpretationis.

Scriptum est, inquiunt, urbem
Hierosolyma per septuaginta annos sua sabbata requieturam.

Locos, quem designant, est in fine posterioris Chronicorum:
\textit{Ad complendum verbum Dei in ore Ieremia,
 donec terra acquiescat sabbatis
suis. omnes dies desolationis sabbatizavit, usque ad complendum
septuaginta annos}.

%\end{parnumbers}
\clearpage
p. XXX [pdf 57]
%\begin{parnumbers}

Clare loquitur, omni ambage remota, terram,
quamdiu desolata fuit, sabbatizasse, id est, incultam cessasse, donec
complerentur anni septuaginta ab Ieremia determinati.

[Prolegomena continues up to page LII]

\end{parnumbers}


% Clear our self-defined headers
\setheaders{}{}

%% ToC generation gives obscure errors; solution: delete the .aux files
%% and re-compile (twice)
\tableofcontents{}

\mainmatter
% !TEX TS-program = xelatex
% !TEX encoding = UTF-8 Unicode
% this template is specifically designed to be typeset with XeLaTeX;
% it will not work with other engines, such as pdfLaTeX

%%% Count out columns for fixed-width source font
% 000000011111111112222222222333333333344444444445555555555666666666677777777778
% 345678901234567890123456789012345678901234567890123456789012345678901234567890

\chapter{}
\addcontentsline{toc}{chapter}{Primus Liber  - De anno aequabili minore}
\begin{center}
\begin{textsc}
\Large IOSEPHI\\
\Huge SCALIGERI\\
\Large IVLII CÆSARIS F.\\
\large DE\\
\Huge EMENDATIONE\\
\Large TEMPORUM\\
\large LIBER PRIMVS.\\
\end{textsc}
\em{Ad candidum Lectorem.}
\end{center}
\normalsize

\mletter{A} 
\setcounter{parcount}{0}
\begin{parnumbers}
\dropcapil{9}{S}{i vervm}
est, quod sciscit Stoicorum schola, Tempus esse normam rerum, \& custodiam, quia veritatis index atque examen est, \& rerum gestarum memoriam, ac diuturnitatem posteritati tuetur: ij non vulgari laude digni sunt, qui temporum rationes conscribere, atque fugitiuam antiquitatem retrahere conantur.
\\ \p
Qua in re cum tam priscis scriptoribus, quam æqualibus temporum nostrum opera egregie nauata sit, dolendum tamen, aut
\mletter{B}
ferius, quam oportebat, antiquos sese ad id studium contulisse, aut pauciora ea de re monumenta, quam ab ipsis
auctoribus relicta sunt, ad nos peruenisse.

Nam vt omnia extent veterum Græcorum scripta, ea tamen paucorum temporum interuallum complectebantur.

Græcis enim ante initia Olymiadum suarum nihil plane exploratum est: \&, quod dolendum est, de illorum scriptis, quæ ad Chronologiam spectabant, nihil nobis præter desiderium relictum est.

Nam quæ Eusebij exstant, quamuis è Græcorum monumentis hausta sunt, \& multa egregia ac cognitu digna nobis conseruarunt: tamen dissimulandum non est, multa in illis reperiri, quæ castigatioribus iudiciis non satisfaciant.

\mletter{C}
Quod si Thalli, Castoris, Phlegontis, Eratosthenis canones exstarent, perparua, aut nulla potius ratio haberetur librorum quorundam, qui hodi in penuria meliorum nobis in pretio sunt.

Apud Romanos vero, ea scriptio infeliciter cessit, quod eam cognitionem ferius amplexi sint.

Nam ante Consulatum Bruti nihil certi apud illos: omnia fabulosa: \&, si rem propius spectemus, ne ipsius quidem Bruti Consulatum, ac tempus Regifugij satis exploratum habent.

\end{parnumbers}
\clearpage
p. 2 [pdf 85]

\begin{parnumbers}

quamius, vt prodidit Censorinus, Varro collatis diuersarum ciuitatum temporibus, \& interualla retexens, verum in lucem protulerit, \& viam reperit, qua certus
\mletter{A}
annorum Vrbis conditæ numerus iniri posset.

Sed, vt suo loco disputabitur, non magis constabat Varroni de initiis Vrbis, quam Græcis de anno excidij Troiæ.

Nam ea demum est vera demonstratio, quæ cogit, non quæ persuadet.

Soli sacri libri supersunt, ex quorum fontibus certa temporum ratio hauriri possit.

Sed omnis temporum cognitio inutilis est, nisi certa epocha in illis deprehendatur, ad quam omnium temporum contextus, tam antecedentium, quam consequentium referri possit.

Nam, vt præclare dixit vetus inter Christianos scriptor Tatianus, apud quos temporum notatio non cohæret, apud illos neque veritatis \& fidei historicæ ratio vlla constare potest.

Quod si aliquis sacræ historiæ peritissimus, hoc est, qui interualla rerum gestarum
\mletter{B}
nobilissima certissimis ratiociniis ex Mose, \& reliquis sacris Bibliis explorata habeat, nihil tamen ex illis a certam epocham historiæ Græcæ, aut Romanæ referre possit: quodnam adiumentum is ex eiusmodi diligentia adferre potest aut sibi, aut studiosis rerum antiquarum?

Nam omnis cognitionis finis ad vsum aliquem spectat, quem si ex medio literarum sustuleris, ingratus est omnis labor \& opera, quæcunque in omne studium impenditur.

Eiusmodi est Iudæorum scientia, qui in ratiociniis quidem sacrorum temporum colligendis tantum studio \& diligentia consecuti sunt, vt proxime à veritate abesse dici possint: sed dum nullam aut saltem deprauatam rerum extrarum cognitionem tenent, multum errant, quod sine externa historia sacram tractare
\mletter{C}
aggrediuntur.

Venio ad nostros, recentiores dico, qui hodie summo cum fructu, sacræ, Græcæ, \& Romanæ historiæ tempora digesserunt.

Ij heroica virtute chronologiam negligentia \& contemtu maiorum intermortuam ac sepultam, è tenebris \& obliuionis silentio quotidie eruere conantur.

Certe meum semper iudicium fuit, eam rem maiore cum laude ab illis restitutam, quam ab antiquis proditam fuisse.

Nam non solum pleraque in ratione temporum pristinæ integritati reddiderunt, sed \& longe meliora effecerunt.

In multis tamen iudicium, in quibusdam etiam diligentiam requiro.

neq; enim dum verum adepti sunt.

Argumento suerint omnium, quotquot de his rebus tractarunt, dissensiones: vt inter tot millia Chronologorum vix inter duos de eadem re
\mletter{D}
conueniat.

Quanta adhuc contentione de Septimanis Danielis, de initio, medio, \& fine earum velitantur?

Tamen nihil plane eorum, quæ volunt, assecuti sunt.

Ab eorum lectione incertior atque indoctior sum, quam dudum.

Quis vnquam eorum veram epocham Exodi Habræorum; quis, quod pudendum est, verum annum natalis Dominici odoratus est?

Ecce trita, obuia, vulgaria, vt nobis videtur, ignoramus, \& remotiorum ac reconditiorum indicium promittimus!

\end{parnumbers}
\clearpage
p. 3 [pdf 86]

\begin{parnumbers}

Quis eorum Danielis \mletter{A} Hebdomadas interpretandas suscepit, qui inscitiæ suæ latebram non quæsiuerit, \& reges Persidis, qui nunquam in rerum natura fuerunt, non commentus sit?

Quod si Danielem accuratissime legissent, eis ad negotium explicandum non aliis regibus Persidis opus fuisset, quam iis, quos Herodotus, Diodorus, \& omnis Græcorum antiquitas nouit.

Sed quo non progressa est \textgreek{[Greek]}?

Berosos, Metasthenes, \& nescio quos Catones, ac Philones consulunt, qui ante hos centum annos ex officina nescio cuius indocti \& impudentis prodierunt.

Et sese Criticos in temporum notatione profitentur, quibus tam facili genere, tam pueriliter vnus homo otiosus in tanta luce literarū quotidie imoponit.

\mletter{B} Cuius hominis inscitiā si nihil aliud, certe illud arguere possit, quod Metasthenem pro Megasthene posuit. Si Iosephum Græce, aut Strabonem, aut Athenæum legisset, is Megasthenem vocari deprehendisset, quem Metasthenem vocat.

Si Græce scisset, numquam \textgreek{[Greek]} in illa lingua reperiri, neque hanc compositionem in eadem probari intellexisset.

Vt igitur ij resipiscant, qui \& nouos reges in Perside crearunt, \& Assueros Priscos, Assueros Longimanos, Assueros Pios, duos Cyros, \& nescio quæ alia somnia Annij Viterbiensis in medium producunt, primum vno verbo indicabo fontem erroris eorum: deinde qui medicina huic morbo fieri possit, docebo.

Quod igitur in veri inuestigatione \mletter{C} eos ratio fugerit, duas summas causas reperio: vnam, quod veterum tempora ciuilia, annorum, mensium formas, status, ac genera ignorarunt: alteram, quod characterem, \& notationem ei anno, quem sibi proposuerant, non adhibuerunt.

Ex vtraque quidem causa temporum confusio manauit, sed diuerso genere.

Ex priore causa ignoratus est annus, mensis \& dies multarum nobilium epocharum.

Huius enim rei cognitio pertinet ad tempus ciuile nationum.

Ex altera causa Palilia vrbis Romæ nunc tertio anno Olympiadis, nunc quarto attribuuntur.

Item Consulatus Bruti nunc in hunc, nunc in illum annum Olympiadis confertur.

Vt igitur nouam rationem emendationis temporum ineamus, duo illa præcipue nobis discutienda sunt: sed 
prius \mletter{D} de omnium nationum temporibus ciuilibus: quam assequi perdifficile est, nisi prius tempore in sua principia, hoc est ab annis, periodis, mensibus in vltimum terminum, dies, horas ac scrupula resoluto.

Nam qui ante nos hanc prouinciam aggressi sunt, si modo hanc nostram, non aliam aggressi sunt, ij satis de tempore, \& eius natura disputarunt.

Sed hanc disputationem melius interpres \textgreek{[Greek]} sibi vindicasset.

Neque vero nos id agimus, vt difiniamus tempus esse hoc secundum Peripateticos, aut illud seundum Stoicos, aut Academicos.

\end{parnumbers}
\clearpage
p. 4 [pdf 87]

\begin{parnumbers}

Qui istis definitionibus diu immorati sunt, \& hac sola scientia Chronologiæ scribendæ modum terminarunt, illi fatis \mletter{A} verborum quiedem, sed rerum nihil definiuerunt.

Nequid tamen \textgreek{[Greek]} transigatur, decreui singularum, vel minimarum temporis partium prius conspectum aliquem dare, quam ad descriptionem \textgreek{[Greek]} temporum ciuilium, \& eorum methodum aggrediar.

Incipiam igitur ab vltimo termino, a die scilicet, \& eius partibus, hoc est hora, \& scrupulis.

Ab hora igitur, si libet, principium esto.
\end{parnumbers}

\subsection[De Horis \& partibus diei reliquis.]{De Horis et partibvs diei reliqvis.}
\setcounter{parcount}{0}
\begin{parnumbers}

\dropcap{3}{V}{Eteribus} statim ab initio has diei partes, quas \textsc{H o r a s} vocamus, in vsu non fuisse, argumento fuerint priscæ locutiones, \mletter{B} quibus dies non in partes secatur, sed actionibus quotidianis distiguitur: vt cum \textgreek{[Greek]} vesperam vocabant, nimirum, vt poëta inquit, Demeret emeritis cum iuga Phœbus equis.

Item quod tempus antemeridianum disignantes dicebant \textgreek{[Greek]} vel \textgreek{[Greek]}, conuenientibus scilicet eo tempore in Comitium viris: vt Hesiodus dicit, \textgreek{[Greek]}.

Quod tamen longe aliter interpretes Græci illius poëtæ exponunt.

Aiunt enim Hesiodum intellexisse de tricesima mensis Lunaris: \& sensum loci Hesiodei esse perinde ac si dixisset, Quando homines veram \textgreek{[Greek]} Lunarem agunt, \& non secundum vsum politicum, sed secundum motum Lunæ.

Quod \mletter{C} tamen nobis valde coactum videtur: \& mentem Hesiodi hanc fuisse dicimus: \textgreek{[Greek]} esse valde idoneam rebus gerendis ea hora, qua homines ad ius in forum conueniunt.

Homerus Odyss. \textgreek{[Greek]}

\textgreek{[Greek]}

\textgreek{[Greek]}

Quæ sane interpretatio melior vulgari.

Sic etiam paulo post dicit, \textgreek{[Greek]}, loquens de vndecima: cuius partem designat, cum dicit \textgreek{[Greek]}.

Quod nos interpretamur iam adulto die.

Sic Homerus meridiem designat, \textgreek{[Greek]}.

Porro neq; hoc verbum \textgreek{[Greek]} id, quod nunc, valebat.

Sed tempus actuum quotidianorum illo notabatur: vt cum dicebant \textgreek{[Greek]}.

\mletter{D} Latinis vero Tempestas dicebatur.

In Legibus Decemuirum Atticis fuit: SOL OCCASVS SVPREMA TEMPESTAS ESTO.

Neque recte quidam hinc expungunt TEMPESTAS. quod SVPREMA absolute diceretur, vt apud Plautum.

Nam plane in legibus Solonis, vnde illud caput traductum, scriptum fuit, \textgreek{[Greek]}.

Stoicus scriptor apud Stobæum loquens de Socratis iudicio capitali: [Greek Greek Greek].

Idem censeas de veteribus Hebræis, \mletter{A} qui diei nullas alias partes, quam mane, meridiem, \& vesperam norant. \& ita dies diuiditur Psalmo L V, commate X V I I I.

\end{parnumbers}
\clearpage
p. 5 [pdf 88]

\begin{parnumbers}

Sic Homero, \textgreek{[Greek]}.

Sed hic dies intelligitur Lux, exclusa nocte.

Nam totum \textgreek{[Greek]} Hebræi in quatuor partes diuidebant, quas vigilias vocabant.

Prima vigilia erat à vespere: secunda à media nocte: tertia à mane: quarta à meridie.

Alioqui nomen hoc \texthebrew{[Hebrew]} quo hodie horam designant, ne notum quidem illis erat: atque apud Danielem aliud significat.

Posterorum inuentum est Horologium, \& \textgreek{ηλιοτρόπια[Greek: heliotropia; probably: sundails]}, quibus dies per lineas, \& interualla vmbrarum distinguebatur. vnde prodiit locutio \textgreek{[Greek]}, pro hora cœnæ. vel \textgreek{[Greek]}: \mletter{B} quia notis literarum singularium horæ distinguebantur.

Testatur \& Epigrammatium de Horologio:

\textgreek{[Greek]}

\textgreek{[Greek]}

Nam ante \textgreek{Ζ, Η, Θ, Ι,} erat \textgreek{Α, Β, Γ, Δ, Ε, ς.}

\begin{wraptable}{r}{0mm}
\footnotesize
\setlength{\tabcolsep}{3pt}
\renewcommand{\arraystretch}{1.1}
\begin{tabular}{ |r @{}| r  r  r | c |r | r | r r | }
\multicolumn{4}{p{3cm}}{\parbox[t]{3cm}
 {\scshape\small TABVLA CON-\\
 \footnotesize vertendi osten-\\
 \upshape ta in sexagesimas.}}
& \multicolumn{1}{c}{} &
\multicolumn{4}{p{3cm}}{\parbox[t]{3cm}
 {\scshape\small TABVLA CON-\\
 \footnotesize vertendi sexage-\\
 \upshape simas in ostenta.}}
\\
\cline{1-4} \cline{6-9}
\itshape\scriptsize Ostenta. &
\itshape\scriptsize Sexag. &
\itshape\scriptsize Sexag. &
\itshape\scriptsize Sexag. &
\hspace{5mm} &
\itshape\scriptsize Sexag. &
\itshape\scriptsize Sexag. &
\itshape\scriptsize Ostenta. &
\itshape\scriptsize Ostenta.
\\
\cline{1-4} \cline{6-9}
   1 &  0' &  3'' & 20''' & &  0' &  1'' &    0' & 324'' \\
\cline{2-4} \cline{8-9}
   2 &  0' &  6'' & 40''' & &  0' &  2'' &    0' & 648'' \\
\cline{2-4} \cline{8-9}
   3 &  0' & 10'' &  0''' & &  0' &  3'' &    0' & 972'' \\
\cline{2-4} \cline{8-9}
   4 &  0' & 13'' & 20''' & &  0' &  4'' &    1' & 210'' \\
\cline{2-4} \cline{8-9}
   5 &  0' & 16'' & 40''' & &  0' &  5'' &    1' & 540'' \\
\cline{2-4} \cline{8-9}
   6 &  0' & 20'' &  0''' & &  0' &  6'' &    1' & 864'' \\
\cline{2-4} \cline{8-9}
   7 &  0' & 23'' & 20''' & &  0' &  7'' &    2' & 108'' \\
\cline{2-4} \cline{8-9}
   8 &  0' & 26'' & 40''' & &  0' &  8'' &    2' & 432'' \\
\cline{2-4} \cline{8-9}
   9 &  0' & 30'' &  0''' & &  0' &  9'' &    2' & 756'' \\
\cline{2-4} \cline{8-9}
  10 &  0' & 33'' & 20''' & &  0' & 10'' &    3' &   0'' \\
\cline{2-4} \cline{8-9}
  20 &  1' &  6'' & 40''' & &  0' & 20'' &    6' &   0'' \\
\cline{2-4} \cline{8-9}
  30 &  1' & 40'' &  0''' & &  0' & 30'' &    9' &   0'' \\
\cline{2-4} \cline{8-9}
  40 &  2' & 13'' & 20''' & &  0' & 40'' &   12' &   0'' \\
\cline{2-4} \cline{8-9}
  50 &  2' & 46'' & 40''' & &  0' & 50'' &   15' &   0'' \\
\cline{2-4} \cline{8-9}
  60 &  3' & 20'' &  0''' & &  1' & 60'' &   18' &   0'' \\
\cline{2-4} \cline{8-9}
  70 &  3' & 53'' & 20''' & &  2' &  0'' &   36' &   0'' \\
\cline{2-4} \cline{8-9}
  80 &  4' & 26'' & 40''' & &  3' &  0'' &   54' &   0'' \\
\cline{2-4} \cline{8-9}
  90 &  5' &  0'' &  0''' & &  4' &  0'' &   72' &   0'' \\
\cline{2-4} \cline{8-9}
 100 &  5' & 33'' & 20''' & &  5' &  0'' &   90' &   0'' \\
\cline{2-4} \cline{8-9}
 200 & 11' &  6'' & 40''' & &  6' &  0'' &  108' &   0'' \\
\cline{2-4} \cline{8-9}
 300 & 16' & 40'' &  0''' & &  7' &  0'' &  126' &   0'' \\
\cline{2-4} \cline{8-9}
 400 & 22' & 13'' & 20''' & &  8' &  0'' &  144' &   0'' \\
\cline{2-4} \cline{8-9}
 500 & 27' & 46'' & 40''' & &  9' &  0'' &  162' &   0'' \\
\cline{2-4} \cline{8-9}
 600 & 33' & 20'' &  0''' & & 10' &  0'' &  180' &   0'' \\
\cline{2-4} \cline{8-9}
 700 & 38' & 53'' & 20''' & & 20' &  0'' &  360' &   0'' \\
\cline{2-4} \cline{8-9}
 800 & 44' & 26'' & 40''' & & 30' &  0'' &  540' &   0'' \\
\cline{2-4} \cline{8-9}
 900 & 50' &  0'' &  0''' & & 40' &  0'' &  720' &   0'' \\
\cline{2-4} \cline{8-9}
1000 & 55' & 33'' & 20''' & & 50' &  0'' &  900' &   0'' \\
\cline{1-4} \cline{8-9}
\multicolumn{4}{c}{}      & & 60' &  0'' & 1080' &   0'' \\
            \cline{6-9}
\end{tabular}
%\end{table}
\end{wraptable}


Arabibus, Persis, \& reliquis Orientis gentibus non horologiis, sed naturalibus matutini, meridiani, \& vespertini temporis interuallis diem notare, etiam hodie consuetudo manet.

Astronomis propria \mletter{C} est diuisio diei in sexagesimas primas, secundas, tertias, \& sic deinceps.

Artificibus computi annalis in horas, puncta, ostena, minuta, partes.

Hora est punctorum 4. mintorum 40. partium 480. momentorum 1760. ostenta autē sunt arbitraria, quibuslibet aliarum diuisionum in illa resolutis.

\mletter{D} Orientalibus vero Computatoribus compendiosa horarum resolutio est.

Non enim in sexagesimas assem diuidunt, sed in 1080 partes ita vt 18 particulæ vni minuto horario respondeant.

Hac diuisione hodie Iudæi, Samaritani, Arabes, Persæ, \& aliæ Orientis nationes vtuntur.

\end{parnumbers}
\clearpage
p. 6 [pdf 89]

\begin{parnumbers}

\mletter{A} Quorum in sexagesimas, \& contra, sexagesimarum in hæc conuertendarum, Tabellas duas posuimus.

\end{parnumbers}

\subsection{De Diebus.}
\setcounter{parcount}{0}
\begin{parnumbers}

\dropcap{3}{T}{O}
\textgreek{νυχθήμερον[Greek: the day and night, i.e. a full 24 hour cycle]}, quod est spatium viginti quatuor horarum, Daniel eleganter vocat \texthebrew{[Hebrew]} quasi dicas \textgreek{[Greek]}, initio diei ciuilis sumto Iudiace ab eo tempore, quod proxime Solem occasum sequitur.

Nam illud interuallum, quatenus vigintiquatuor horarum est, naturale est: quatenus aliud atque aliud initium habet, dicitur ciuile, Atticis \& Iudæis ab occasu Solis: Ægyptiis \& Romanis à media nocte: Chaldæis Genethliacis ab ortu Solis: Vmbris à meridie initium \mletter{B} sumentibus.

Dierum notationes duplices: aut secundum numerum, \& ordinem: vt prima, secunda, tertia mensis. aut secudum \textgreek{[Greek]}, qua dies alicui rei cognomines. vt dies mensis Persici sunt cognomines regum priscorum: \& dies mensis Mexicanorum, animalium, aut aliarum rerum: \& \textgreek{[Greek]} Ægyptiorum nominibus singulorum Deorum vocatæ. \& dies festi, vt quinquatrus, \textgreek{κρόνια[Greek: of Kronos, i.e. Saturn]}, \textgreek{ϑαργήλια[Greek]}, Quirinalia. \& ab euentu, dies Alliensis, Regifugium. à stellis, dies Septimanæ.

Ecclesia Romana vocat ferias. quia veteris anni Ecclesiastici initium à Pascha.

Et Pascha dicebatur annus nouus, vt etiam hodie ab Ecclesia Antiochena: à Constantinopolitana autem \textgreek{[Greek]}, ab eadem mente.

Illius autem Hebdomadis dies omnes septem erant \mletter{C} feriati, vt testis est Hieronymus, \& alij veteres.

Hinc obtinuit, vt reliquarum hebdomadum dies etiam Feriæ vocarentur, præcipuo quodam principis septimanæ Paschalis auspicio \& omine.

Solon autem primus omnium \textgreek{[Greek]} vocauit, cum antea \textgreek{[Greek]} esset prima mensis.

Hesiodus: \textgreek{[Greek]}.

Diei diuisio summa ab actibus quotidianis, in fastos, nefastos, atros, religiosos, intercisos, iustos: vt Græcis \textgreek{[Greek]}, vel, vt alij, \textgreek{[Greek]}, \textgreek{[Greek]}. aut ab æquatione annui temporis, Solaris, \& Lunaris, in \textgreek{[Greek]}, \textgreek{[Greek]}, \textgreek{[Greek]}, \textgreek{[Greek]}, \textgreek{[Greek]}, \textgreek{[Greek]}, \textgreek{[Greek]}.

\textgreek{[Greek]} Computatoribus Græcis dicuntur, quæ Latinis Regulares, quæ cum \mletter{D} Concurentibus. id est Epactis Solaribus compositæ dant characterem Kalendarum, aut alius diei mensis.

\textgreek{[Greek]} sunt duplicis generis, Solares, \& Lunares.

Solares fiunt abiectis septenariis ex cyclo Solari, addito præterea die bisextili.

Lunares producuntur, excessu Solis, qui est \textsc{x~i} dierum, in numerum aureum ducto, abiectis tricenariis.

\end{parnumbers}
\clearpage
p. 7 [pdf 90]

\begin{parnumbers}

Præterea vtrarumque Epactarum sua methodus: Solarium ad characterem dierum: Lunarium ad ætatem Lunæ, vt Computatores Latini loquuntur, vt \mletter{A} Græci autem, \textgreek{[Greek]}.

\textgreek{[Greek]} sunt, quæ eximuntur de mense, duplici ex causa: aut vt rationes Solis cum Lunaribus congruant, vt in anno veteri Græcorum: \& in enneadecaeteride Paschali Saltus Lunæ Latinis dictus, Græcis \textgreek{[Greek]}. aut vt solennia festa cum feria Septimanæ, vt in anno Iudaico.

\textgreek{[Greek]}, vel \textgreek{[Greek]} sunt, quæ ex caussa religionis, transferuntur, \& dissimulantur per speciem comperendinationis, vt in anno Iudaico, \& olim in prisco Romano.

In Iudaico enim \textgreek{[Greek]} \& comperendinationes institutæ, ne feria secunda, quarta, sexta in caput anni incurrat. in Romano prisco comperendinabantur Nundinæ, vt à religiosis diebus summoueientur, auctore Macrobio.

\textgreek{Εμβόλίμοι [Greek]} sunt, vt notio verbi declarat, insititij \mletter{B} dies: \& erant naturales, aut ciuiles.

Naturales, qui ex scrupulis, \& horis appendicibus colliguntur, vt quatro quoque anno exeunte vnus dies ex quadrantibus anni Iuliani, quod \textsc{b~i~s~e~x~t~u~m} vocatur: item in periodo Arabica vndecies vnus dies intercalatur in fine Dulhagiathi, qui est vltimus mensis anni Hagareni Mohamedici.

Ciuiles sunt, qui præter naturalem anni rationem \& modum inseruntur, vt vnus dies in fine Marcheschvvan Iudaici, anno qui dicitur superfluus, aut abundans.

\textgreek{[Greek]}, quæ explendis spatiis anni adiiciuntur potius, quam inseruntur, vt quinque, quæ anno æquabili extra ordinem mensium adiectæ Ægyptiis dicuntur \textsc{n~i~s~i}, Persis, \& Armeniis \textsc{m~v~s~t~e~r~a~k~a} : item duæ, quæ extra modum anni Attici in calce Posideonis \mletter{C} appensæ, \textgreek{[Greek]} dicebantur, aut \textgreek{[Greek]}, aut \textgreek{[Greek]}.

At \textgreek{[Greek]} locum habent in anno mobili.

Est autem interuallum inter epocham \& caput anni, vtroque termino excluso.

Hoc constat semper in annis, quorum caput nunquam epocham anteuertebat.

Vt in anno Attico caput Hecatombæonis nunquam ante Solstitij veterem epocham statuebatur.

Itaque quod inter Solstitium, \& propositum Hecatombæonem interiacet spatij, vtroque termino excluso, dicebantur \textgreek{[Greek]}.

Idem obseruabatur in annis magnis Metonis \& Calippi.

Rursus Romanorum sacri dies Kalendæ, Nonæ, Eidus: Græcorum autem \textgreek{[Greek]}.

Quod ex versu Hesiodi à nobis adductor constat.

Sunt præterea nomina imposita diebus mensium \mletter{D} singulis, vt suo loco referetur.

Sunt \& secundum hebdomadas vt infra subiecimus.
\end{parnumbers}
\clearpage
p. 8 [pdf 91]

\mletter{A}
%% Can't put \mletter{} next to a table. It needs to be in a paragraph level
%% So we skip B and C

\begin{table}[h]
\large
\begin{tabular*}%
{\textwidth}{%
@{\extracolsep{\fill} } r r r @{\hspace{4pt}} || r @{\hspace{4pt}} | @{} l 
}
\multicolumn{3}{c}{\textsc{DIES HEBDOMADIS}} &
\multicolumn{2}{c}{\textsc{ALITER PERSICE.}}
\\
\multicolumn{3}{c}{\textsc{persicæ.}} & \multicolumn{2}{c}{}
\\
\hline
\texthebrew{שנב} % some nonsense filler text
& \textarabic{شزذيثب} % some nonsense filler text
& \textarabic{ل}
& 1
& \textit{Ruz iache}
\\
\texthebrew{[Hebrew]}
& \textarabic{[Persian]}
& \textarabic{ب}
& 2
& \textit{Ruz duiemi}
\\
\texthebrew{[Hebrew]}
& \textarabic{[Persian]}
& \textarabic{ج}
& 3
& \textit{Ruz siumi}
\\
\texthebrew{[Hebrew]}
& \textarabic{[Persian]}
& \textarabic{ﺩ}
& 4
& \textit{Ruz tzeharmi}
\\
\texthebrew{[Hebrew]}
& \textarabic{[Persian]}
& \textarabic{م}
& 5
& \textit{Ruz pengemin}
\\
\texthebrew{[Hebrew]}
& \textarabic{[Persian]}
& \textarabic{و}
& 6
& \textit{Ruz schesmin}
\\
\texthebrew{[Hebrew]}
& \textarabic{[Persian]}
& \textarabic{ز}
& 7
& \textit{Ruz haphthemi}
\end{tabular*}

\vspace{\baselineskip}

\begin{tabular*}
{\textwidth}{%
    @{\extracolsep{\fill} } r r @{\hspace{4pt}} || r @{\hspace{4pt}} r c
}
\multicolumn{2}{c}{\textsc{TVRCIÆ HEBDOMADIS}} & \multicolumn{3}{c}{\textsc{SECVNDVM PLANETAS.}}
\\
\multicolumn{2}{c}{\textsc{dies.}} & \multicolumn{3}{c}{}
\\
\texthebrew{[Hebrew]}
& \textarabic{[Arabic]}
& \texthebrew{[Hebrew]}
& \textarabic{[Arabic]}
& \astro{♄}
\\
\texthebrew{[Hebrew]}
& \textarabic{[Arabic]}
& \texthebrew{[Hebrew]}
& \textarabic{[Arabic]}
& \astro{♃}
\\
\texthebrew{[Hebrew]}
& \textarabic{[Arabic]}
& \texthebrew{[Hebrew]}
& \textarabic{[Arabic]}
& \astro{♂}
\\
\texthebrew{[Hebrew]}
& \textarabic{[Arabic]}
& \texthebrew{[Hebrew]}
& \textarabic{[Arabic]}
& \astro{☉}
\\
\texthebrew{[Hebrew]}
& \textarabic{[Arabic]}
& \texthebrew{[Hebrew]}
& \textarabic{[Arabic]}
& \astro{♀}
\\
\texthebrew{[Hebrew]}
& \textarabic{[Arabic]}
& \texthebrew{[Hebrew]}
& \textarabic{[Arabic]}
& \astro{☿}
\\
\texthebrew{[Hebrew]}
& \textarabic{[Arabic]}
& \texthebrew{[Hebrew]}
& \textarabic{[Arabic]}
& \astro{☾}
\end{tabular*}
\end{table}

\begin{parnumbers}
Cur autem dies cognomines Planetarum non sequuntur ordinem \& situm siderum, quorum cognomines sunt, vt scilicet post diem Saturni non sequatur dies Iouis, sed dies Solis, hæc caussa est.

% Diagram: circle with heptagram, with planets at the points:
% Moon ☾, mercury ☿, venus ♀, sun ☉,
% mars ♂, jupiter ♃, saturn ♄
%\begin{wrapfigure}[9]{r}{10\baselineskip}
\begin{wrapfigure}[9]{R}{9\baselineskip}
  \centering
  \def\svgwidth{9\baselineskip}
  {\astrofont\input{./img/planets.pdf_tex}}
\end{wrapfigure}

Septem Planetæ per circulum secumdum ordinem suum dispositæ, æquabili interuallo constituunt septem Triangula isoscele ad peripheriā, \mletter{D} quorum bases sunt latera Heptagoni circulo inscripti, vt habes in circulo proposito, ad cuius peripheriam septem errantes sunt secundum feriē suam sitæ, constituentes triangula isoscele \astro{♄♀♃}, \astro{♃☿♂}, \astro{♂☽☉}, \astro{☉♄♀}, \astro{♀♃☿}, \astro{☿♂☽}, \astro{☽☉♄}.

In quibus Triangulis dexter angulus ad basim est prima stella Trianguli, secunda in angulo ad verticem, tertia angulus sinister ad basim: ita vt omnis stella anguli dextri habeat oppositam \mletter{A} stellam anguli in vertice, stella autem anguli à vertice stellæ anguli sinistri ad basim sit opposita.

\end{parnumbers}
\clearpage
p. 9 [pdf 92]

\begin{parnumbers}

Sequentur igitur sese omnes septem Planetæ non per seriem suam, sed per interualla laterum, quæ veræ sunt oppositiones.

Sit igitur Triangulum \astro{☉☽♂} primum ordine.

\astro{☉} in angulo basis dextro præibit. sequetur Luna ei opposita in vertice, eam oppositus Mars in angulo sinistro basis. qui quidem Mars cum in Tiangulo \astro{☉☽♂}, sinistrum angulum basis occupet, in triangulo \astro{♂☿♃} occupabit dextrum basis angulum, habens oppositum Mercurium, Mercurius autem oppositum Iouem in angulo sinistro. qui Iuppiter faciet angulum dextrum in Triangulo \astro{♃♀♄}, habens oppositam in vertice \mletter{B} Venerem, vt ea opposita est Saturno in angulo sinistro.

Sed angulus ille rursus erit dexter in Triangulo \astro{♄☉☽}.

Et sic erogati sunt septem planetæ in totidem dies, quas Ecclesia Romana vocat ferias.

Hæc est vera harum appelationum ratio.

\end{parnumbers}

\subsection{De Mensibus.}
\setcounter{parcount}{0}
\begin{parnumbers}

\dropcap{3}{E}{X} diebus fiunt \textgreek{[Greek]}, quæ notationes \& epochat temporum constituunt.
\\ \p
Primum \textgreek{[Greek]} ex diebus dicitur Septimana, res omnibus quidem Orientis populis ab vltima vsque \mletter{C} antiquitate vsitata, nobis autem Europæis vix tandem post Christianismum recepta.

De ea iam dictum est.

Tum Romanorum \textgreek{[Greek]}: cui successit hebdomas nostra.

Nam nono quoque die Nundinæ erant. \& spatium illud in Kalendario vetere Romano notatum est literis ab A ad H, vt in nostro Kalendario Hebdomas notata est ab A ad G, inclusiue, vt loquuntur.

Mexicanorum \textgreek{[Greek]} sequitur.

Quod enim spatium nobis septenis diebus, illis finitur ternis denis.

Ita Iudæorum est \textgreek{[Greek]} veterum Romanorum \textgreek{[Greek]}, Mexicanorum \textgreek{[Greek]}.

Proximum ab hoc \textgreek{[Greek]} dierum est Mensis: qui \& naturaliter, \& ciuiliter sumitur.

Naturalis mensis \& ipse duplex.

\mletter{D} Aut enim Lunaris, aut Solaris.

Rursus Lunaris triplicis generis: aut quatenus Luna ab eodem puncto Zodiaci profecta, ad idem reuertitur: qui dicitur \textgreek{[Greek]}, item \textgreek{[Greek]}.

quod interuallum minus est, quam viginti octo dierum: maius quam viginti septem.

Secundum genus est eiusdem sideris à Sole profecti ad eundem reditus.

Hæc dicitur \textgreek{[Greek]}.

Tertij generis mensis est secundus dies \textgreek{[Greek]}, quae dicitur \textgreek{[Greek]}, \& \textgreek{[Greek]}.

Secundum \& tertium genus in temporibus ciuilibus locum habent.

Nam Athenienses \textgreek{[Greek]} neomenias suas putabant: hodie vero Hagareni \textgreek{[Greek]}.

\end{parnumbers}
\clearpage
p. 10 [pdf 93]

\begin{parnumbers}

Græcorum enim neomenias ab ipso iugo Lunæ putari solitas testis Vitruuius ex Aristarcho Samio, his verbis, loquens de Luna: Quot mensibus sub rotam Solis radiosque primo die \mletter{A} antequam præterit, latens obscuratur.

\&, cum est sub Sole, noua vocatur.

Postero autem die, quo numeratur secunda, præteriens à Sole, visitationem facit tenuem extremæ rotundationis. [Vitruvius, De architectura libri decem, Liber IX, Capitulum II, Sect. 3: "Ita quot mensibus sub rotam solis radiosque uno die, antequam praeterit, latens obscuratur. Cum est cum sole, nova vocatur. Postero autem die, quo numeratur secunda, praeteriens ab sole visitationem facit tenuem extremae rotundationis." Transl.: "whence, on the first day of its [the moon's] monthly course, hiding itself under the sun, it is invisible; and when thus in conjunction with the sun, it is called the new moon. The following day, which is called the second, removing a little from the sun, it receives a small portion of light on its disc."]

Vbi etiam dixit visitationem extremæ rotundationis, quam ille Samius sine vllo dubio \textgreek{φαίσιν μιωοειδῆ[Greek: phase …]} vocabat.

Sed \& Onomacritus, qui sub nomine Orphei \textgreek{τελετὰς[Greek: ceremony]} scripsit, in opere, quod \textgreek{ἡμέρας[Greek: day, hours of daylight]} vocauit, mensem Lunarem à iugo Lunæ incipit.

Cuius versus apposui:

\begin{greek}
Παίτ᾽ ἐδάης Μουσαῖε ϑεοφραδἐς. εἰδέ σ᾽ αἰώγει[Greek ?]

ϑυμὸς ἐπωνυμίας μήνης κατὰ μοῖραν ἀκοῦσαι,[Greek ?]

ῤεῖά τοι ἐξερέω, σὺ δ᾽ ἐνὶ φρεσὶ βάλλεο σῆσιν,[Greek] \mletter{B}

οἵην τάξιν ἔχοντα κυρεῖ. μάλαν γαρ χρέος ἐστὶν[Greek]

ἴδμεναι, ῶς αὕτη παρέχει κλέος ἄντυγι[?] μηνός.[Greek]

\textgreek{[Greek]}

\textgreek{[Greek]}

\textgreek{[Greek]}

\textgreek{[Greek]}

\textgreek{[Greek]}

\textgreek{[Greek]}
\end{greek}

Sed Neomenia Arabica, excedit modum \textgreek{φάσεως[Greek: phase]} vt plurimum. ita vt ciuiles neomeniæ mensium Lunarium sint non vnius generis: Atticæ \mletter{C} \textgreek{[Greek]}: Iudaicæ sæpe \textgreek{[Greek]}.

Arabicæ semper \textgreek{[Greek]}, à tertia, inquam, die.

Mensis Solis naturalis est, qui naturalibus circuli cœlestis segmentis definitur, qualis est transitus Solis à signo ad signum.

Hi, \& Lunares, sunt vere cœlestes menses.

Mensis ciuilis Solis est, qui non naturali modo, sed æqualiter tributus est. vt in anno Ægyptiaco \& Græco omnes æqualiter sunt \textgreek{[Greek]}: \& in Lunari alternis pleni, \& caui. in anno Mexicano \textgreek{[Greek]} cum ex X V I I I. mensibus eorum annus constituatur.

Apud Albanos Martius erat sex \& triginta dierum, Maius viginti duum, Sextilis duodeuiginti, September sedecim.

Tusculanorum Quintilis habuit tirginta sex, October triginta duos, Aricinorum October trigintanouem.

At rationes Lunæ non patiuntur, vt menses sint alternis perpetuo pleni, \& caui. sed hoc ad methodum ciuilis temporis institutum.

Sunt \& alij menses ex superfluis diebus collecti, qui Embolimi dicuntur: iique aut naturales, aut ciuiles: ambo autem ad æquationem Solis directi.

Naturales embolimi sunt, qui ex Solis excessu collecti ad spatia Lunæ complenda adhibentur. cuiusmodi est Iudaicus Adar prior, \& Samaritanus Adar alter. isque mensis est semper tricenum dierum.

Ciuilis embolimus, qui ex diebus Solis superfluis consurgens fulciendo anno cauo adiictur.

\end{parnumbers}
\clearpage
p. 11 [pdf 94]

\begin{parnumbers}

Eiusmodi erat Merkendonius \mletter{A} prisci anni Romani alternis binum \& vicenum, item trinum \& vicenum dierum.
Eiusmodi \& Posideon Atticus.

Neque enim Posideon naturalis esse potest, quamuis triginta dierum, cum neque Lunaris esset, quod eius neomenia longe à lunari discederet: neque Solaris, quod pars esset illius anni, qui ad Solis cursum descriptus non esset.

Idem de Merkedonio dicas, qui neque ad Solarem annum, neque ad Lunarem pertineret, neque modum eum haberet, qui iusto mensi competit, cum esset tantum XXII, aut ad summum XXIII dierum.

Mensis diuisio Atticis in \textgreek{[Greek]}. prima \textgreek{[Greek]} dicebatur \textgreek{[Greek]}, secunda \textgreek{[Greek]}, tertia \textgreek{[Greek]}.

Idque factum, quia illorum menses omnes erant \textgreek{[Greek]}.

Persæ vero in \textgreek{[Greek]}, \mletter{B} non solum, quia eorum menses omnes \textgreek{[Greek]}, sed etiam, quia totus annus constat ex quinariis tribus \& septuaginta.

In mense \textgreek{[Greek]} Athenienses pro \textgreek{[Greek]} dicebant \textgreek{[Greek]}. Quamuis enim mensem vno die mutilabant, tamen cum tertia mensis pro secunda dicebant, non videbantur mensem mutilare, cuius \textgreek{[Greek]} numerabant.

Meton vero \& Calippus eam diem eximunt, quæ post duas syzygias \& dies quatuor succedebat.

Mensium nomina in antiqua Hebraici anni forma nulla fuerunt, neque in hodierna Sinarum, Iaponensium, \& Indorum.

Menses enim illis ab ordine primi, secundi, tertij dicuntur.

In anno Romano mistæ sunt appellationes, ex cognominibus, \& ordine numerario.

Quidam etiam cognomines imperatorum Romanorum, vt Cypriis \textgreek{[Greek]}.

Romanis ipsus Iulius, Augustus: \& temporibus Domitiani Germanicus pro Septembri, Domitianus pro Octobri.

% Insert: page of latin text "M. AVR. AVG. LIB."
\begin{wrapfigure}[16]{R}{0.5\textwidth}
  \centering
  {M. AVR. AVG. LIB.}
\end{wrapfigure}

Martialis: Dum Ianus hiemes, Domitianus autumnos, \&c.

Sed Statius omnes Kalendas vindicat Domitiano, præter Iulium, \& Augustum, – Nondum omnis honorem Annus habet, cupiuntq; decem tua nomina menses.

Insania quoque Commodiidem cōsecuta esset, si \mletter{D} longior vita mōstro illi data fuisset.

\end{parnumbers}
\clearpage
p. 12 [pdf 95]

\begin{parnumbers}

Augustum enim Cōmodum, Septembrem Herculeum, Octobrem Inuictum, Nouembrem Exuperatorium, Decembrem Amazonium vocari edicit. Extat quoq; lapis Lauinij, in quo mentio Iduum Commodarum. vbi \& nomen Commodi Senatusconsulto prius derasum, postea alia manu \mletter{A} incusum.
Quædam nationes etiam geminos menses cognomines habent.

Annus Syrochaldaicus habet geminum Tisrin, item geminum Conum.

Annus Hagarenus geminum Regiab, \& geminum Giumadi.

Annus Sxonicus geminum Giuli, \& geminum Lida.

Sed in anno embolimæo Lida est tergeminus.

Et tunc annus ille dicebatur Trilida.

Item, diuersarum nationum iidem menses communes.

Nam Panemus in anno Macedonico fuit, item Corinthiaco, \& Thebano.

Artemisius communis fuit Laconum, \& Macedonum: Carneus Syracusanis, \& Cyrenensibus vsitatus.

Sed differbant situ anni \& tempore: vt suo loco disputabitur.

Sic Martius primus erat \mletter{B} Romanorum: tertius Albanorum, Aricinorum, Formianorum: quartus Forensium, Pelignorum, Sabinorum: quintus Faliscorum, Laurentum: sextus Hernicorum: decimus Æquicolorum. Hæc in genere de mensibus.

\end{parnumbers}

\subsection{De Anno.}
\setcounter{parcount}{0}
\begin{parnumbers}

\dropcap{3}{M}{Aximum} \textgreek{[Greek]} dierum annus, sed qui multipliciter dictus sit. \\ \p
Tot enim constitui possunt, quot sunt siderum errantium periodi.\\ \p
Est enim annus circuitus eius periodi, cuius cognominis ipse est.

Vt annus Solaris est cognominis circuitus eius \mletter{C} sideris, qui quidem circuitus dupliciter sumitur, aut à Solstitio ad Solstitium, à bruma ad brumam: \& est minor anno Iuliano.

aut à puncto Zodiaci, ad idem punctum Zodiaci.

qui est maior anno Iuliano.

hoc est maior 365 1/4 diei.

quo ad id puncum Zodiaci redit, vnde profectum erat.

Eadem fere quantitas quæ \& Soli, attribuitur Veneri \& Mercurio.

Saturni periodus est dierum 10747.18'.59''.13'''.

Hoc est annorum Ægyptiorum 29. dierum 162.

Iouis annus dierum 4330. horarum 17.14'.

Id est annorum Ægyptiorum 11.315.

Martis annus dierum 686. horarum 22.24'.

annorum Ægyptiorum 1.321 dierum.

Lunæ, dierum 29.31'50''.8'''.

Obtinuit tamen vulgo, vt duorum siderum, Solis \& Lunæ, labentem cœlo qui ducunt annum, ratio in \mletter{D} temporibus ciuilibus haberetur.

Et Lunæ quidem primum vnus circuitus pro anno habebatur, vt apud Ægyptios. deinde tres, vt apud eosdem Ægyptios \& Arcades.

Tandem duodecim periodi Lunares annum ciuilem constituerunt dierum 354 cum triente, \& paulo plus quam duum trientum horariorum.

Duodecim quoque segmenta Zodiaci componunt annum Solarem tantum, quantum diximus.

\end{parnumbers}
\clearpage
p. 13 [pdf 96]

\begin{parnumbers}

Sed ignoratio motuum vtriusque sideris alias atque alias anni formas veteribus \mletter{A} ptperit: quarum vetustissima est ea, quæ annum quidem ad cursum Lunæ describebat: sed incertis neomeniis, quæ non produent ex obseruatione motus Lunæ, quales vulgus rusticorum obseruare solet, \& quæ proprie ciuilem mensem constituere non possunt.
Cum igitur hoc modo incertæ essent neomeniæ, conuenit primum, vt menses omnes tricenis diebus explicarent, annumque dierum sexaginta \& trecentum constituerent.

quod genus longe desciscebat à modo anni Lunaris.

Hæc diu seruata fuit apud Græcos anni forma.

In Oriente septuagesima secunda pars illius anni, hoc est quinq; dies, accesserunt anno Græco: vt anni modus suerit dierum trecentorum sexaginta quinque: \mletter{B} qua ratione ab anno solari se minimum discedere arbitrati sunt.

Vnde duo præcipua genera anni apud veteres suerunt neque Lunaria, neque Solaria, sed ambigui inter vtrumque generis.

Prior forma in Græcia resedit: altera in Oriente.

Græci vero non vna via ad emendationem suæ aggressi sunt.

Difficile erat menses plenos omnes ad Lunæ rationes exigere: \& tamen in quibusdam actibus ciuilibus opus habebant motu Lunæ.

Nam semper Olympias plenilunio, \& X V die mensis celebrabatur.

Vt igitur annus Græcus æquabilis Olympiadem deprehenderet in X V mensis, hoc difficile non erat.

Vt autem X V mensis in X V Lunæ incidat in mensibus æquabilibus, hoc fieri non potest, nisi post fingula quadriennia, adiectis vnicuique anno singulis \mletter{C} biduis, quas \textgreek{[Greek]} vocabant.

Hæc Tetraeteris Elidensibus vocata est Olympias, Delphis Pythias.

eiusque mensis primus duantxat erat Lunaris: reliquorum ratio claudicābat.

Primus Cleostratus eum annum in Lunarem modum reformare conatus est, excogitata octaeteride dierum 2922, cuius menses alternis pleni \& caui: anni vero singuli communes 354 dierum: embolimæi 384. communes quidem quinque, embolimæitres.

Syzygiæ autem nouem \& nonginta.

Octaeteridum vitio deprehenso, Meton enneadecaeterida excogitauit dierum solidorum 6940.

Cui castigandæ periodus Calippica successit dierum 27759, sine vllis scrupulis appendicibus, anno ab editione Metonica centesimo tertio.

Hanc excepit vltimus, tanquam secutor quidam, \mletter{D} Hipparchus, annis circiter centum octoginta octo ab epocha Calippica, periodo publicata dierum 111035: quæ minor est Calippicis rationibus die vno, Metonicis autem quinq;.

Quare duæ castigationes adhibitæ anno æquabili Græco.

Altera est coniugatio alterna vel interrupta mensium plenorum \& cauorum, vt cum ipsa Luna congruerent, quod annus Græcus maior esset Lunari. altera est embolismus mensium, vt cum sole æquaretur, quod annus Lunaris minor est Solari.

Sed alternatio plenorum \& cauorum mensium aliquando variat: idque sit aut naturaliter, aut ciuiliter.

\end{parnumbers}
\clearpage
p. 14 [pdf 97]

\begin{parnumbers}

Naturalis varietas committitur propter embolismum \mletter{A} aut mensis, aut diei.
Vtroque enim modo duo menses pleni continuantur.

Vt in anno Iudaico cum intercalatur mensis Adar, tunc Schebat, \& Adar embolimus ambo sunt pleni.

In anno vero Arabico cum accedit dies mensi vltimo, qui Dulhagiathi dicitur, tunc \& ipse Dulhagiathi, \& antecedens Dulkaadathi ambo fiunt tricenum dierum.

Sed in Samaritano sæpe continuantur tricenarij menses, \& in antiquo Iudaico, vt ex Talmud \& Iad Mosis cognoscimus: \& menses Harpali, Metonis, \& Calippi non semper alternis continuati sunt.

sed sæpe bini pleni continuati, nunquam autem bini caui.

Quin etiam cum dies accedit vltimo mensi Arabico, tres continui menses sunt pleni, Dulkaadathi, Dulhagiathi, \& Muharam sequētis anni.

Isque annus ab Arabibus dicitur \textarabic{[Arabic]} hoc est embolimæus.

Sic etiam anno Iudaico pleno tres menses continui sunt pleni, Tisri, Marchesvvan, Casleu.

Ciuilis varietas accidit anno Iudaico tantū, accrescente mensi Marcheschvvan die vno: \& Marchesvvan ex cauo sit plenus.

Rursus \& iin embolismo mensium differentia situ, \& tempore.

Situ, si aut in medio, aut in calce intercalatio fiat.

vt in anno Attico vltimus mensis intercalabatur, qui dicebatur \textgreek{[Greek]}.

in Iudaico sextus mensis intercalatur, \& dicitur Adar prior. In anno Hagereno mensis embolimus erat desultor, qui omnes menses anni percurrebat in annis 228, quæ sunt enneadecaeterides duodecim.

qua intercalatione memoria proauorum nostrorum vtebantur Turcæ Cilices, donec annum Hegiræ simplicern \mletter{C} Muhamedicum vsurpare cœperunt.

At in anno prisco Romanorum situs embolismi longe diuersus ab aliis.

non enim is inter duos menses interiiciebatur, vt alias solet: sed in mensem ipsum, tanquarn surculus in truncum infindebatur.

Inter X X I I I enim, aut X X I I I I, aut inter X X I I, \& X X I I I Februarij inserebatur. neque vero sine caussa.

Hoc enim semper obseruabant, vt mēsis proximus Martio semper esset dierum X X V I I I. eratque Februarius ordinarius. at interuallum inter exitum Ianuarij, \& Kalendas Februarij ordinarij imputabatur Merkedonio. \& Kelendæ Februarij ordinarij in anno embolimæo nunc in Regisugium, nunc in Terminalia, incurrebant.

Neque enim semper inter Terminalia, \& Regisugium intercalabantur, vt vult Censorinus. \mletter{D}

quia hoc pacto Februarius ordinarius nunc viginti octo; nunc vndetricenum dierum fuisset.

Quod tamen salsum ex Varrone conuic[ē]tur.

Tempore differt intercalatio, quatenus Iadæi nunquam intercalant, priusquam \textgreek{[Greek]}, qui sunt dies decem cum horis paulo magis quam vna \& viginti, eo rationes Solis deduxerint, vt commode mensis Lunaris conflari possit.

Quod spatium numquam maius est triennio, nunquam minus biennio: \& in X I X. annis semper septies fit.

\end{parnumbers}
\clearpage
p. 15 [pdf 98]

\begin{parnumbers}

At in Calippico \& Metonico anno aliquando citius, aliquando ferius \mletter{A} intercalabatur, quam ratiocinia \textgreek{[Greek]} postulare videntur.

quandoquidem hoc vnum cauent præcipue Athenienses, ne Hecatombæonis neome[?]a Solstitij priscam epocham anteuertat: cum in anno Iudaico vt plurimum neomenia Tisri æquinoctium autumnale, neomenia vero Nisan æquinoctium veris antiquum, si ratio Iuliani anni habeatur, anteuertat.

Anni Lunaris non vnum genus est: sed summa diuisio in duo fastigia discedit: in annos periodicos, \& simplices.

Anni periodici dicuntur, qui certo annorum orbe, interuentu embolismorum, recurrunt.

Huius interualli modum veteres certo definire non potuerunt. quippe Cleostratus dierum 2922, Harpalus \mletter{B} 2924, Eudoxus plusquam 2922, minus quam 2924: Meton aliter: \& ab omnibus diuerse Calippus, \& deniq; ab eo discedens Hipparchus.

Cuius sententia, sed cælestibus rationibus leuiter castigata, enneadecaeterida Lunarem minorem Iuliana statuit, hora vna cumscrup. paulo plus quam viginti septem.

Simplices anni \& ipsi quidem sine remedio intercalationis in pristinam epocham recurrunt, sed longo interuallo, annorum scilicet Iulianorum 228, qui sunt anni simplices Arabici 235, scrupuli diurni quinquaginta.

Sunt \& in annis Lunaribus caui, superflui, æquabiles.

Annus cauus is est, cui competit \textgreek{[Greek]}.

Ideo à nobis \textgreek{[Greek]} vocabitur. ex eo enim eximitur dies vel propter ciuile institutum, cuiusmodi est annus Iudaicus, quem defectiuum \mletter{C} Computatores Iudæorum vocant. (in eo quippe Casleu, qui natura est plenus, instituto fit cauus.) vel naturali de caussa: vt anno decimonono Cycli Paschalis Dionysius diem vnum eximit, quem vocauit Saltum Lunæ: Græci vero Computatores \textgreek{[Greek]}.

quamquam inepte annum vltimum enneadecaeteridis constituit dierum duntaxat 353, cum eiusmodi annus natura nullus fit.

Superfluus annus vocetur à nobis \textgreek{[Greek]}.

Accedit enim illi \textgreek{[Greek]} tam ex caussa ciuilli, vt in anno Iudaico marcheschvvan naturaliter cauus, ciuiliter fit plenus: quam e caussa naturali: vt vndecim anni in Triacontaeteride Arabica augentur singulis diebus ex ratiociniis Lunæ collectis.

Annus æquabilis vocetur \textgreek{[Greek]}.

Iudæis computatoribus \mletter{D} dicitur annus ordinarius.

Is est, cui nihil accedit, nihil decedit.

Huc vsque ad annum Lunarem deduxit nos æquabilis minoris disputatio.

Nunc de altero æquabili maiore disputandum, quo Ægyptij, Persæ, \& Armenij, Mexicani, \& Perusiani vsi.

Hic antiquitus Orientis nationibus vnus idemque fuit: præter quam si quando \textgreek{[Greek]} quinque in alium locum traductæ, diuersum anni caput constituebant. qua \textgreek{[Greek]} tralatione vtebantur ij, qui post annos 120 æquabiles mensem solidum intercalabant, vt Persæ: qui quidem \textgreek{[Greek]} suas in æquinoctium vernum semper reiiciebant.

Terminum autem vocabant N E V R V Z.
\end{parnumbers}
\clearpage
p. 16 [pdf 99]

\begin{parnumbers}

\& habebant mensem desultorem \mletter{A} \textgreek{[Greek]}, omnes menses anni peruagantem, donec in primum mensem recurreret.

qui orbis non redibat, nisi anno æquabili 1461 vertente, qui sunt anni Iuliani perfecti 1460.

Hic est magnus annus, cuius menses sunt annorum æquabilium tricenum, quot dierum simplex mensis.

\textgreek{[Greek]} autem sunt quinquies quatuor annorum, vt illæ simplices quinque dierum.

Quod autem illa anni forma retenta fit, in caussa fuit non tam ignoratio annis solaris, quam facilis, \& tractabilis, ac vere popularis eius vsus.

Alioqui nulla fere natio fuit, quæ quadrantem anni Solaris ignorarit: sed modum illius dispensandi nesciebant.

præterea à mensibus superfluis, qui sunt maiores tricenis diebus, refugiebant, quos necesse est retincri, quadrante illo retento. \mletter{B}

Ægyptij singulis quadrienniis exactis diem intercalabant in ortu Caniculæ, \& quadriennium illud exactum \textgreek{[Greek]}, \textgreek{[Greek]}, \textgreek{[Greek]}, vocabant.

Attici diem quarto quoque anno exacto intercalabant inter septimum \& octauum diem Ianuarij.

Elidenses inter octauum, \& nonum Iulij.

Syromacedones, Chaldæi, \& Iudæi inter septimum \& octauum Octobris.

Eamque diei intercalationem à Seleucidarum temporibus vsque ad imperium Constantini \& infra retinuerunt Iudæi: quam vtique simul cum anni Calippici forma à victoribus Syromacedonibus acceperant.

Romani Atticos secuti brumæ sidere confecto intercalabant; quæ ipsis Olympiadum mysteria vocabantur. Nam \& Attici \& reliqui omnes Græci annum Solarem in \mletter{C} quatuor quadrantes diuidebant, quæ \textgreek{[Greek]} vocabant, singulis dies 91. hor. 7 ½ attribuentes.

quod à temporibus Seleucidarum, ad hanc vsq; diem, Iudæi constanter obseruant. Itaque V I I I Iulij erant \textgreek{[Greek]}, V I I Octobris \textgreek{[Greek]}: V I I Ianuarij \textgreek{[Greek]}, V I I I Aprilis \textgreek{[Greek]}.

Quare cum legis \textgreek{[Greek]}, \& \textgreek{[Greek]}, nullas alias intellige, præter has. quod \& \textgreek{[Greek]} quoque intelligendum.

Hæc \textgreek{[Greek]} Iudæi Tekuphoth vocant.

Germani, Celtæ, Saxones inter X X V \& X X V I Decembris intercalabant: quam noctem vocabant M V D R A N E C H T.

Tartari hodi inter vltimam Ianuarij, \& Kalendas Februarij. quas Kalendas patrio sermone Festum Alborum vocant. quia albis vestibus eam diem colunt.

Denique quanuis \mletter{D} Lunari anno, aut alio longe diuerso à Solari vterentur, tamen tacita quadam obseruatione post dies 1460 vnum diem intercalandum esse sentiebant.

Neque enim aliter Habræi quatuor Tekuphas suas tueri potuissent, nisi quadrante post quartū quemq; annum rationibus accedente.

Et sane vnaquæq; Tekupha est dierū 91, horarum 7 ½ Vnde quatuor tantæ Tekuphæ fiunt dies 365 ¼.

\end{parnumbers}
\clearpage
p. 18 [pdf 100]

\begin{parnumbers}

Displicuit tamen hæc quadrantis obseruatio Græcis Astronomis, propter causam admodum futilem \mletter{A} \& puerilem, qua Solis quantitatem ad Lunæ ratiocinia exigebant, \& cum vtriusque sideris exactum modum adhuc non tenerent, ex Lunæ comparatione Solares rationes eliciebant.
Itaque tantam censuerunt Solis quantitatem, quantam summam dies periodi in annos periodi distributæ relinquebant.

Metonis periodus est dierum 6940.

Diuisa per 19 annos relinquit quantitatem anni Solaris Metonici dierum 365. scrup. diurnorum 15 5/19

Calippi periodus dierum 27759 per 76 annos diuisa relinquit modum anni Calippici Solaris dierum 365 ¼ qualis est annus noster Iulianus.

Periodus Hipparchi est dierum 111035, annorum 304.

Sed neglectis illis 2, trecentesima pars diei detrahitur de quantitate anni Calippici Solaris, \mletter{B} vt fiat annus Solaris Himmarchus dierum 365. hor. 5. 55.' 15.'' 15/19

Detractis ex quadrante hor. 0. 4.' 44.'' 4/19 quæ etiam fuit sententia Ptolemæi.

Itaque ex sententia Hipparchi \& Ptolemæi annus Tropicus, est annus Iulianus, vel Calippicus nonadecima parte differentiæ enneadecaeteridis Lunaris \& Iulianæ diminutus: qui est verus annus Rabbi Ada: de quo alibi.

Philolai Pythagorei magnus annus dierum 21505 ½ per 59 annos diuisus constituit modum Solarem dierum 365. Oenopidæ annus magnus dierum 21557 itidem per 59 annos diuisus dat modum anni Solaris dierum 365 cum parte dierum duum \& viginti vndesexagesima.

Harpali octaeteride per 8 annos diuisa remanet modus anni Solaris dierum 365 ½.

Annus magnus \mletter{C} Democriti dierum 29950 ½ per 82 annos diuisus relinquit annum Solarem dierum 365, cum quadrante \& centesima sexagesimaquatra parte vnius diei.

Denique nullus veterum non patauit rationes Solis ad Lunam exigendas esse.

Et quotiescunque ex certa collectione dierum vtriusq; sideris rationes congruerent, dies illi per to[t] annos diuisi, quot ex illa summa dierum constitui poterant, visi sunt illis certam anni Solaris quantitatem feninire posse.

Sapientiores vero, quanuis incomprehensibilem illam existimarēt, tamen pro vero quod proximum putabant amplexi sunt, dies trecentos sexaginta quinque cum quadrante, qui est modus anni Iuliani.

cui singulis quadrienniis exactis vnus dies accrescit.

sed hic annus comparatione Ægyptiaci \mletter{D} est Solaris: comparatione autem Tropici est æquabilis.

Maior enim est vera anni ratione scrup. horariis 11.' 6.'' 40.'' secundum Gelalæam formam, aut 10.' 48.'' fere, vt Alfonsini docent.

Neque Prutenicæ tabulæ multum abludunt, quæ constituunt motum æqualem Solis ab æquinoctio dierum 365. Hor. 5. 49.' 15.'' 46.'''

Itaque hinc nasci possunt aliquot genera anni Solaris.

Æquabilis, vt Iulianus.

Tropicus, vt Persarum Gelalæus.

Rursus Tropicus aut æquabilis, aut cælestis.

\end{parnumbers}
\clearpage
p. 18 [pdf 101]

\begin{parnumbers}

Æquabilis Tropicus, cuius quantitas Tropica est, partes autem, hoc est menses, æquales \& ciuiles: vt is, \mletter{A} quem modo dixi, Galelæus.
Descriptus est enim mensibus æqualibus, omnibus tricenum dierum, cum epagomenis appendicibus, quæ in communi anno sunt quinque, in embolimæo sex.

Cælestis Tropicus, cuius partes in naturalia Zodiaci segmenta tributæ sunt.

Rursus \& annus Solis æquabilis in ciuilem \& cælestem diuidi potest.

Ciuilis, vt Iulianus Romanorum, Syrogræcorum, Græcorum Elkupti.

Cælestis, vt Dionysianus Prolemæi Philadelphi.

Nam \& is quoque quadrantem Canicularem quadriennio exactor accipiebat.

Finis vero omnis periodi est, vt caput recurrat \& reuoluatur in idem principium, quam \textgreek{[Greek]} Græci vocant: quæ quidem pessum iuerit tandem, non seruata veri anni Tropici mensura.

\& qua annus Iulianus \mletter{B} suam tueri non potuit, manifestum est Kalendas Ianuarias ab V I I I parte Capricorni, in qua statuerat eas Cæsar, in vicesimam primam fere traductas esse hodie.

Sed nihilo commodius epocha in enneadecaeteride seruari potest.

Nam enneadecaeteris Tropica est velocior Lunari horis plusquam duabus.

Contra enneadecaeteris Iuliana maior Lunari hora vna, \& scrup. plusquam 26.

Cum vero peccatur vtraque ratione, Tropica \& Iuliana, Luna, cuius rationes mediæ sunt inter illas duas, fines epochæ suæ tueri non potest: vt in cyclo Dionysij Paschali accidit, cuius neque rationes ad enneadecaeterida Luanrem collectæ sunt, neque epocha ad Solis motum castigata: sed eius forma potius tota mere Calippica est.

ita vt eius statum post trecentos \mletter{C} 4. annos variare necesse sit.

Quare vt epochat suas seruarent illi veteres, immanes periodos excogitauerunt, quales illæ Calippi, Philolai, Democriti, Ocnopidæ. Sunt etiam periodi, quæ omnem modum excedebant.

Et cum in omnibus illis orbibus annorum præcipuam vtriusque sideris ationem haberent, tamen nescio quæ confidens eos incessebat opinio, non solum vtriusque sideris, sed etiam omnium \textgreek{[Greek]} illo circuitu fieri.

Sic Harpalus \& Eudoxus putarunt in sua Octaeteride omnes \textgreek{[Greek]} \& \textgreek{[Greek]} in orbem redire.

Idem etiam censet fieri Aratus in Metonica enneadecaeteride, Eudoxum suum sectus, qui in fabrica Sphæræ suæ eam planetarum \& inerrantium harmoniam in eorum orbibus ostendit esse, vt sequente \mletter{D} restitutione vtriusque sideris, necessario \& omnium inerrantium reditum contingere concluderet.

Propterea tot Sphæras \textgreek{[Greek]} commentus est, quot narat Aristoteles libro X I \textgreek{[Greek]} quem consulas licet.

Quin etiam Calippus alios orbes præter Eudoxum addidit, ea ratione, vt \textgreek{[Greek]} adstrueret, \textgreek{[Greek]}, vt Aristoteles de ea re scribens pronunciauit.

\end{parnumbers}
\clearpage
p. 19 [pdf 102]

\begin{parnumbers}

Itaque \textgreek{[Greek]} nomine intelligendum ortus, \& occasus \textgreek{[Greek]}, \mletter{A} non autem \textgreek{[Greek]}, hoc est significationes eorum: quas in orbem redire cum Luna \& Sole in enneadecaeteride Meto quidem, Calippus, \& Hipparchus putarunt, \& aliis persuaserunt, donec deprehenso vero anni Tropici modulo vidium harum periodorum castigatum est.
Cicero quoque apud Macrobium, sexto de republica, annum illum immanem, quem ex tot millibus annorum simplicium componit, non aliter in orbem rediturum cum omnibus errantibus \& inerrantibus censet, quam si eadem defectio Solis in eodem loco, eodem tempore fiat: quanuis defectiones cyclo enneadecaeterico recurrant non raro.

Et tamen ea eclipsi putat non tantum Solis \& Lunæ, sed etiam quinque errantium ad eandem \mletter{B} inter se comparationem, confectis omnium spatiis, reditum fieri, quo eadem cæli positio, siderumque, quæ ab initio maxime fuit, rursus existit.

Quare eclipses ad eam rem notabant veteres, vt etiam \textgreek{[Greek]} excogitarint \textgreek{[Greek]} vocabant.

Eorum vetustissimus fuit dierum 6585 ⅓, qui sunt anni Arabici 18, syzygiæ 7. in genere vero sunt syzygiæ 223.

Quamobrem in secundo libro Plinij perparem legitur siue culpa ipsius Plinij, siue librarij, defectus luminum ducentis viginti duobus mensibus redire.

Hipparchus alium \textgreek{[Greek]} longe maiorem excogitauit dierum 126007, syzygiarum 4267, annorum Arabicorum 355 cum syzygiis 7: annorum Iulianorum 344 cum diebus 361.

Quæ sunt tolerabiles periodi.

Nam à caussis naturalibus, \mletter{C} nempe à defectionibus luminum proficiscuntur.

quemadmodum etiam enneadecaeteris Lunaris, \& Cyclus Solis: quorum illa Lunam Soli restituit, hic Solem Septimanæ, \& præterea periodus Mexicanorum constans annis L I I, quæ restituit \textgreek{[Greek]}, quæ ist ipsis vicem nostræ Hebdomadis.

Neque alia fuit periodus magna Persatum veterū, quam Salchodai vocabant.

Sunt \& aliæ, sed ciuiles, \& Indictio; Aliæ inanibus coniecturis insistunt, vt Dodecaeteris Chaldaica Genethliacorum, item Heracliti, Lini, Orphei, Dionis, \& Maorum: quorum periodus ad modum octauæ sphæræ composita est annorum 360000 à conditu Mundi, vt ipsi putant. quorum annorum hic est centies octagies quater millesimus, sexcentesimus nonagesimus quartus. \mletter{D} Sed longe illa Sinarum prodigiosior, iuxta quam hic annus Christi 1594 est à conditu rerum octigenties octagies quater millesimus, septingentesimus septuagesimus tertius.

Bonziorum vero Iaponensium periodus annorum 470 desiuit cum anno Christi 1561. \& 1562 cœpit sequens. eiusque hic est vicesimus currens.

Ea vertente scelera extirpatum iri: reliquum tempus omnia pacata fore credunt.

Taceo diuersas Christianorum, Iudæorum, Samaritanorum de conditu rerum opiniones: item Romanorum lustrum qunque annorum, sæculum centum \& decem.

\end{parnumbers}
\clearpage
p. 20 [pdf 103]

\begin{parnumbers}

Sunt \& periodi Computatorum: vt Iudæa \mletter{A} annorum 6916, quæ constat cyclis Lunaribus 364, Solaribus 247, periodis magnis Dionysianis 13.

Habetque tot cyclorum septimanas, quot dierum septimanæ sunt in anno Solari: tot periodos Dionysianas, quot menses annus embolimæus: tot cyclos Solares, quot cyclos Lunares magnus cyclus Iudaicus.

Itaque elegantissima est, \& artificiosissima. eiusq; hic agitur annus 5354, anno Christi vulgari 1594.

Et inibit 1595 annus eiusdem proximo autumno, vnde omnes epilogismi neomeniarum Iudaicarum.

Periodus Dionysiana \& ipsa ad annalem computum pertinet, annis constans 532, ducto in sese vtroque cyclo.

Veræ quidem periodi magnæ caput incurrit in annum primum vtriusque cycli, pertinetque ad methodum Lunæ \& Solis. \mletter{B}

\& locum habet dumtaxat in anno Iuliano, hoc est in eo, cui præter 365 dies quadrans attibuitur.

Itaque eius initium est à Kal. Ianuariis in anno Romano: in anno Constantinopolitano à Kal. Septembris. in Antiocheno à Kal. Octobris. in Alexandrino \& Samaritano ab a. d. 1111 [I I I I ?]. Kal. Septemb.

Periodus vero Dionysij pertinet ad methodum neomeniæ Paschalis, initio sumto ab anno primo natalis Christi, vt ipse quidem putabat: item ab anno decimo cycli Solis Iuliani, \& ab ea neomenia, cuius quartadecima dies proxime post X X I, aut in X X I I Martij conficeretur.

Hactenus à minimis initiis ad summa temporum incrementa, quam \textgreek{[Greek]} Græci vocant, Chronologum perduximus, \& eum in conspectu totius antiquitatis collocauimus.\mletter{C}

Superest nunc, vt quæ carptim \& obiter perstrinximus, ea vberius suis locis expicentur.

Resumamus igitur eos annos, ex quibus tanquam elementis, ad tot tamque diuersa genera annorum progressus factus est.

Ex anno Græco, qui est æquabilis minor, omnes anni, Lunaris formas propagatas esse vidimus: vt ex Ægyptiaco, qui est æquabilis maior, omnes Solares.

Non igitur confuse, \& per saturam hæc tractanda, sed suo quæque \& loco \& ordine.

Quatuor igitur libris quatuor genera anni summa explicare decreuimus.

Primus erit de anno æquabili minore. Eo enim omnis Græcia vsa tam diuersis generibus, quam multæ fuerunt eius terræ nationes, \& \textgreek{[Greek]}.

Itaque ea erit reliqua pars huius libri.

Secundum locum sibi vindicat annus \mletter{D} Lunaris, quia ex illo priore deriuatus.

Tertius liber complectetur anni æquabilis maioris formas, \textgreek{[Greek]}, \& differentias.

Quartus illius anni traduces \& propagines persequetur, diuersa nempe anni Solaris genera, \& mutationes.

Hæc est pars prior, quam initio huius diatribæ.

Chronologo promisimus, de annorum \& temporum Ciuilium generibus.

\end{parnumbers}
\clearpage
p. 21 [pdf 104]

\begin{parnumbers}

Altera pars est de charactere, qui necessarius est notandis temporum interuallis, quæ sequentibus libris tractabimus, item diuersis \mletter{A} computis nationum annalibus, de quibus librum singularem ad calcem operis adiiciemus, non tanquam appendicem, sed partem vnam operis nostri.
Quis igitur sit vsus characteris temporum, docet nos Dionysius ex Ephoro, qui cum annum excidij Troiæ ex Olympiadum epocha notare non posset, cum is casus aliquot seculis antiquior sit prima Olympiade, dixit id accidisse eo anno Attico, quo viginti \textgreek{[Greek]} annum explebant.

Statim peritis anni Attici subolebat, quo anno id accidere potuerit.

Sciebant enim quoties in quanto interuallo annorum id fieri posset. Exemplo Ephori aut Dionysij erit nobis character excogitandus, quo animus anceps in triuio constitutus quæsitum ad fontem manu deducatur.

Erit igitur primum \mletter{B} totius instituti nostri fundamentum annus Iulianus, quem fingimus ante multa millia annorum fuisse.

Characteres vero illi duos dabimus, cyclum Lunæ Dionysianum, cuius hic est annus X V I I I.

\& cyclum Solis Iulianum, cuius hodie annus V I I currit.

Tertium etiam, vbi ratio temporum patietur, Indictiones non aspernabimur.

Nam qui his characteribus semel vti institerint, illi, quæ sit constantia, \& fides illius methodi pulcherrimæ in ratione temporum, experentur.

Si quis hoc anno Christi 1594 incertus, quot annos natus sit, tamen \& maiorem se quadraginta nouem annorum, \& minorem quinquaginta sex sciat, is imitatur imperitiam Chronologorum Græcorum, qui circiter illius, \& illius regis tempora illud, \& illud accidisse dicunt, annum \mletter{C} vero certum non difiniunt.

Sed cum idem adiicit natum se Nonis Augusti, feria quinta, is addit characterem certum \& indubitatum, quales sunt viginti \textgreek{[Greek]} Ephori.

Nam feria quinta non potuit incurrere in Nonas Augusti, nisi cum litera Dominicalis est C. Ante 49 autem annos id accidit anno Domini 1540, cyclo Solis nono.

Itaque hoc characterismo constantissime affirmanus eo anno hominem natum, \& proximis Nonis Augusti Iulianis illi quinquagesimum quintum natelem initurum.

Idem vsus cycli Lunaris, adhibita castigatione, vt à prima Olympiade, ad annum Domini 1400, tot dies neomeniis adhibeas, quoties 304 annos reperies.

Exemplum. hic est annus à prima Olympiade 2370.

In quibus annis septies reperitur \mletter{D} numerus 304. septem igitur dies neomeniis hodiernis adiiciendi.

Verbi gratia. anno primo cycli epactæ sunt X I. nouilunium martij X V I I I. additis V I I. diebus, nouilunium, vel potius coniunctio lumanarium erat in X X V.

Martij anno quarto anto primam Olympiadem, aut quintodecimo post eandem primam Olympiadem, \& deinceps ad 304 annos.

Sed ab hoc sæculo nostro post 150 annos minuendæ erunt neominiæ totidem diebus, quoties 304 anni reperientur post annum Christi 1700. \& fortasse citius.

\end{parnumbers}
\clearpage
p. 22 [pdf 105]

\begin{parnumbers}

Sed quia nullam epocham veterem certiorem Olympiadum capite habemus: illud autem \mletter{A} cum vetustate comparatum nouitium esse videtur: inutiles erunt characteres cyclorum \& Indictionis, nisi à quadam remotissima epocha initium temporum instituamus.

Excogitemus igitur periodum, quæ vtrunque cyclum, \& Indictionem contineat: quod fiet, si periodum Dionysij Exigui quindecies multiplicemus: qui fient anni 7980.

Ita periodus illa incipiet ab anno primo tum vtriusque cycli, tum Indictionis: \& proinde eiusdem vltimus annus desinit in vltimis vtriusque cycli, \& Indictionis.

Sed annus Christi, vt vulgo putamus, 3267 desinet in vltimum vtriusque cycli, \& Indictionis.

Ergo deductis 3267 de 7980 annis, relinquetur epocha anni ante vulgarem \mletter{B} Christi, nempe 4713.

Ita vt 4714 sit primus annus Christi vulgaris cyclo Solis X, Lunæ 2, Indictionis 4, à Kal. Ianuarij: quamuis \& Indictio autumno proxime antecedenti, Cyclus autem Lunæ Martio sequenti cæperit.

Quare annus iste, qui ex errore vulgi putatur 1594, est 6307. periodi huius, quam Iulianam vocamus, quod ad Iulianam anni formam accommodata sit.

Ideo 6307 diuisis per 28, per 19, per 15 habebimus huius anni 6307 periodi Iulianæ, vel vulgaris Christi 1594, cyclum Solis septimum à Kal. Ianuarij: Lunæ decimumoctauum à Martio sequente: Indictionis septimum Cæsarianæ quidem ab ante d. V I I I Kal. Octobris antecedentis anni 6306: Pontificiæ vero à \mletter{C} Kalendis Ianuarij anni propositi 6307.

Non prædicabo laudes huiusce periodi: Chronologi \& astrologi, qui omnia \textgreek{[Greek]} disputare volunt, non poterunt eam satis laudare.

Qui igitur eclipses ex Tabul[i]s Prutenicis putare volent, ex anno periodi Iulianæ auferant 2408.

\& cum residuo toto excerperant tempora epochæ diluuij.

Exemplum: Eclipsis Lunaris accidit in Septembri anno Olympiadico 446, qui est annus periodi Iulianæ 4383.

Deductis 2408, remanent 1975.

Excerpo primum 1900 ex epocha Diluuij: deinde 75, ex filo annorum expansorum.

Postremo menses vsque ad Septembrem.

Et reliqua vt ex methodo Prutenica.

Qui omne dubium ex temporum ratione tollere volet, vti debet hac periodo, sine qua nihil vnquam certi in natione \mletter{D} temporum adferre poterit.

\end{parnumbers}

% !TEX TS-program = xelatex
% !TEX encoding = UTF-8 Unicode
% this template is specifically designed to be typeset with XeLaTeX;
% it will not work with other engines, such as pdfLaTeX

%%% Count out columns for fixed-width source font
% 000000011111111112222222222333333333344444444445555555555666666666677777777778
% 345678901234567890123456789012345678901234567890123456789012345678901234567890

\setheaders{\shorttitle{} Liber II}{\shortauthor{}}
\chapter{Liber Secundus - De anno lunari}
\normalsize

% 61
% {PDF page nr}{source page nr}{line nr}
\plnr{144}{61}{1}Annum Graecum antiquitus Lunarem fuisse,
ut alia temporum et mensium descriptio
in Graecia non fuerit, quam quae Lunae
rationibus congrueret, non solum recentiores
homines scripserunt, sed non paucos
veterum idem in literas retulisse tam
compertum esse puto, quam falso eos sensisse
convincit ratio tetraeteridum a nobis
libro proximo disputata.
\lnr{9}Praeterea ex
eadem disputatione nostra satis constat naturale anni principium antiquitus
non ab Hecatombaeone, sed a Gamelione, et ex diebus brumlibus
duci solitum.
\lnr{12}Quandiu igitur Athenienses Gamelionem et temporibus
auspicandis et rerum actibus principem mensem habuerunt,
tunc semper Comitia magistratibus creandis in calcem Posideonis
reiiciebant, ubi erant \textgreek{ἄναρχοι ἡμέραι δύο},
 extra ordinem mensium tricenariorum
positae, ita ut annus esset dierum non solum 360, propter
menses \textgreek{τριακονθημέρους}, sed et 362,
 propter illas appendices \textgreek{ὑπερβαλλούσας[?]},
quae, quia per illud biduum omnes magistratus annui abdicabantur,
propterea dicebantur \textgreek{ἄναρχοι ἡμέραι}.
\lnr{19}Praeterea quod in illis
Comitia novorum magistratuum creandorum habebantur, ideo
\textgreek{ἀρχαιρεσίαι[?]}, etiam dicebantur.
\lnr{21}Atque hoc fuit quidem magistratibus
creandis dicatum biduum, donec anni Lunaris formam Astronomi illorum
temporum publicarunt.
\lnr{23}Tunc pro bruma, solstitium: pro Gamelione,
Hecatombaeonem vulgus principium anni coepit statuere.
\lnr{24}Et
menses, in quibus singulis Comitia terna agebantur, quas
 \textgreek{κυρίας ἐκκλησίας[?]}
vocabant, pro Tetraetericis, Lunares: pro solidis, alternis cavi
usurpari coepti.
\lnr{27}Quod ut planius intelligatur, sciendum Athenis
duos summos senatus fuisse, alterum, \textgreek{τῶν ἀρειοπαγιτῶν[?]},
 qui erant iudices
% Greek: also Ἀρεοπαγῑτῶν (pl. gen.)
% A member of the ancient-Athenian conciliary court of the Areopagus.
ut plurimum rerum capitalium et quidem magni momenti: alterum autem
ordinariarum, civilium et bellicarum, et summae denique reipublicae.

% 62
% {PDF page nr}{source page nr}{line nr}
\plnr{145}{62}{2}Sed Areopagitarum consessus perpetuus erat.
\lnr{2}Hic Senatus
quotannis sorte creabatur, olim utique \textgreek{ἐν ὑπερβαλλούσιας ἡμέραις[?]},
postea anno Lunari admisso, in ultimis quatuor diebus anni
Lunaris, hoc est in illis quatuor, qui sunt supra 350.
\lnr{5}Decem
enim Tribus Attica habuit, quales Roma \rnum{xxxv}.
\lnr{6}Ex singuilis Tribubus
quinquaginta magistratus forte creati rebus gerendis admittebantur.
% Bar on quinquaginta?
\lnr{8}Ita ex decem Tribubus quinquageni Senatum Quingentorum
constituebant, qui ab eo \textgreek{οι πεντακόσιοι[?]} decebantur,
 item \textgreek{ἡβουλὴ
τῶν πεντακοσίων[?]}.
\lnr{10}Porro unaquaeque tribus forte unum diem summam
rem gerebat et imperabat.
\lnr{11}Ita cum per 354 dies, quot nimirum
habet annus Lunaris, singuli quinquaginta diem suam per
orbem imperassent, fiebat, ut 35 dies ex toto anno unaquaeque Tribus
rerum potiretur: et, quia decem erant Tribus, sequitur, ut trecentos
quinquaginta dies simul omnes imperarent.
\lnr{15}Reliquae sunt ex anno
Lunari \textgreek{ἄναρχοι ἡμέραι} quatuor.
\lnr{16}Hae igitur quatuor dies vicem illarum
\textgreek{ὑπερβαλλουσῶν[?]} magistratibus creandis reservatae.
\lnr{17}Hoc ita esse, testis Ulpianus
Rhetor, vetus Demosthenis interpres
 \textgreek{εν τω κατα Ανδροτιωνος ἔχειγοῦν[Greek]},
% Demosthenis Orationes ad optimos libros accurate emendatae
inquit,
 \textgreek{υ ενιαιτος κατα τον σεληνιακον δρομον, τριακοσιας πεντηκοι τα τεσσαρας
ημερας[Greek]}.
\lnr{20}\textgreek{και τας μει δ ημερασ εκαλουν οι Αθηναιοι αρχαυρεσιας[Greek]}.
\lnr{20}\textgreek{εν
αις οιυαρχος η Αττικη ην[Greek]}
\lnr{21}\textgreek{εν ταυταις προεβαλλοντο της αρχοντας[Greek]}.
\lnr{21}\textgreek{ηρχον ουν
οι πεντακυσιοι τας τριακοσιας πεντηκοντα ημερας[Greek]}.
\lnr{22}Trecentos igitur et
quinquaginta dies simul imperabant, qui in decem Tribus divisi dant
unicuique dies triginta quinque.
\lnr{24}Nam, exempli gratia, heri \textgreek{ἡ ἀιαντὶς φυλὺ[?]}
imperabat, hodie \textgreek{ἡ κεκροπὶς[?]}, cras \textgreek{ἡ ἀκαμοιυτὶς[?]}.
\lnr{25}Et sic deinceps una quaeque
suam \textgreek{ἐφημερίαν[?]} imperabat, prout sorte[forte?] ducta erat.
\lnr{26}Neque sane quinquaginta
simul imperabant, sed ex singulis Tribubus singuli forte
ducti viri, qui dicebantur \textgreek{πρόεδροι[?]}.
\lnr{28}Illi enim Comitiis habendis praesidebant.
\lnr{29}Nam quingenti simul dicebatur \textgreek{ἡ τῶν πεντακοσίων[?]}, Tribubus
illis decem in unum corpus confusis.
\lnr{30}Quinquaginta autem
per Tribus distincti dicebantur \textgreek{πρυτάνεις[?]}.
\lnr{31}Decem vero vocabantur
\textgreek{πρόεδροι[?]}, qui erant principes quadraginta novem
 reliquorum, singuli
scilicet in sua Tribu.
\lnr{33}Nam unus \textgreek{πρόεδρος[?]} erat quinquagesimus sui
corporis \textgreek{τῶν πρυτάνεων[?]}: ut in castris Romanis Decurio erat
 decimus
illius decuriae, cuius ipse caput erat.
\lnr{35}Menses igitur Lunares proprii
erant horum et omnium denique magistratuum, et dicebantur
 \textgreek{πρυτανεῖαι[?]}.
\lnr{37}Ut in lege Atheniensium apud Demosthenem, \textgreek{ἐν τῷ κατὰ
 Τιμοκράτοις,
ἐπὶ τὴς πρώτης πρυτανείας τῇ ἑνδεκάτῃ[?]}.
\lnr{38}Hoc est undecima mensis
primi Lunaris, id est, Hecatombaeonis Metonici.
\lnr{39}Dicibatur etiam,
\textgreek{[Greek]}.
\lnr{41}Neque enim est alius Hecatombaeon, quam Lunaris, et eo
die \textgreek{[Greek]} pertinebat ad Tribum \textgreek{[Greek]}.

% 63
% {PDF page nr}{source page nr}{line nr}
\plnr{146}{63}{1}Hoc est, is dies erat \textgreek{[Greek]}.
\lnr{2}Et \textgreek{[Greek]} in omni prytania, sive
mense Lunari, agebantur terstata die mensis, \textgreek{τῇ ἑνδεκάτῃ[?]},
 \textgreek{τῇ εἰκάδι[?]}, \textgreek{τῇ ἔνην καὶ νέαν[?]}.
\lnr{4}Quare haec anni forma tantum ad magistratus pertinebat.
\lnr{4}Neque
ea unquam populus usus est, a quo menses \textgreek{τριακονθήμεροι[?]},
 et Tetraeterides
extorqueri nunquam potuerunt, ne tunc quidem, cum annus Lunaris
ab Hipparcho emendatior editus esset.
\lnr{7}Primus itaque mensis Hecatombaeon
fuit a solstitio.
\lnr{8}Et quia per undecim dies Hecatombaeon
huius anni antevertebat caput praeteriti, inde fiebat, ut saepe Hecatombaeon
in Scirrhophorionem Tetraeteridis incideret, in quo saepe \textgreek{[Greek]}
et \textgreek{[Greek]} magistratum inibant.
\lnr{11}Demosthenes \textgreek{πρὸς Τιμόθεον[?]}.
\lnr{11}\textgreek{και [Greek]}.
\lnr{12}\textgreek{[Greek]}, et cetera.
\lnr{13}Si Thargelion Tetraeteridis erat ultimus mensis anni
\textgreek{πρυτανείας[?]}, ergo \textgreek{ὔστερον[?]} illud
 \textgreek{ἔτος[?]} incipiebat a Scirrhophorione, in
quem incurrebat mensis Metonicus.
\lnr{15}Et revera Scirrhoporion Metonicus
illius anni, qui erat 62 periodi Atticae, incidebat in 13 Thargelionis
Metonici: Hecatombaeon autem in 12 Scirrhophorionis.
\lnr{17}Idem Orator
\textgreek{[Greek]} ait fuisse in negotio
\textgreek{[Greek]}, ultimam anni, scilicet \textgreek{πρυτανείας[?]}.
\lnr{19}\textgreek{Φυλάξας[?]}, inquit, \textgreek{[Greek]}, et cetera.
\lnr{21}Quem locum non affequitur interpres.
\lnr{21}Nam orator diserte
indicat, Metonicum mensem \textgreek{πρυτανείας[?]} nunc in Scirrhophorionem
popularem, nunc in Thargelionem incidere, ac consequenter
Hecatombaeonem Metonicum nunc in Schirrhophorionem Tetraeteridis,
nunc in ipsum hecatombaeonem.
\lnr{25}Thucydides: \textgreek{[Greek]}, et cetera.
\lnr{27}Ab initio statim
Veris, hoc est, a Munychione, ait duos adhuc tantum superfuisse
menses magistratui Pythodori.
\lnr{29}Definebat igitur ille annus \textgreek{πρυτανείας[?]}
in Scirrhophorione, et fere conveniebat anno Tetraeterico.
\lnr{30}Porro neomeniae
illae dicebantur \textgreek{[Greek]}, ad differentiam
 \textgreek{τριακονθημέρων[?]}.
\lnr{32}\textgreek{Του [Greek]}, inquit Thucydides, \textgreek{[Greek]}.
\lnr{33}Alibi:
\textgreek{[Greek]}.
\lnr{35}\textgreek{[Greek]} hic intelliguntur Metonicae, et
 \textgreek{πρυτανεῖαι[?]}.
\lnr{36}Cum autem menses Lunares alternis pleni et cavi sint, semper pro
\textgreek{[Greek]} dicebant \textgreek{[Greek]},
 et quae in illis mensibus cavis
vocabatur \textgreek{[Greek]}, ea re vera erat \textgreek{[Greek]}.
\lnr{38}Huius rei rationem
reddidimus in Boedromione \textgreek{[Greek]}, item in exemplo
Oeconomicorum Aristotelis supra a nobis producto de Mnemone
Tyranno Lampsaceno.
\lnr{41}In quo diserte ostenditur sex dies de 360 alternis
eximi solitos, item \textgreek{[Greek]} pro
 \textgreek{[Greek]} alternis mensibus
dici solitum fuisse.

% 64
% {PDF page nr}{source page nr}{line nr}
\plnr{147}{64}{2}Quare hic repetenda non sunt.
\lnr{2}Sic Moschopulus
\textgreek{εἰς ἡμέρας[?]} Hesiodi 177.
\lnr{3}\textgreek{Αθμιαῖοι[?] τὴν τριακοστὴν φασιν ἐννάτην καὶ ἐικοστήν[?]}.
\lnr{4}Iudaei primum diem mensis cavi dicunt secundum.
\lnr{4}Esto exemplum
de mense Elul.
\lnr{5}Ita scribunt \texthebrew{[Hebrew]}.
\lnr{5}Ita de
omnibus cavis mensibus pronunciant.
\lnr{6}Dies, inquiunt, primus mensis
cavi compensat defectum eius.
\lnr{7}Quia quum pro primo ponimus secundum,
ultimus erit tricesimus, non \rnum{xxix}.
\lnr{8}Et ita omnes menses Iudaeorum
habent \textgreek{τριακάδα[?]}, quemadmodum et Graecorum.
\lnr{9}Diu autem
mensibus Lunaribus usi sunt Graeci, adeo ut Theon Arati interpres
dicat suis temporibus etiamnum aliquot nationes Graecorum eam anni
formam retinuisse.
\lnr{12}\textgreek{[Greek]}.

\subsection{De Octaeteride Cleostrati}

\lnr{14}Primus Solon auctor fuit Atheniensibus
 \textgreek{τὰς ἡμέρασ κατὰ σελήνην ἄγειν},
ut scribit Laertius.
% Diogenes Laertius (3rd century CE) wrote about Solon (c. 638 - c. 558 BCE),
% in "Lives and Opinions of Eminent Philosophers" (Book A, second
% chapter)
% Section 59, near the end: 
% "[ἠξίωσέ τε Ἀθηναίους] τὰς ἡμέρας κατὰ σελήνην ἄγειν." 
% "[He required the Athenians] to adopt a lunar month."
\lnr{15}Quod tamen frustra proposuit, cum
Atheniensibus menses omnes pleni, \textgreek{καὶ τριακονθήμερα} agerentur.
% Greek: and the thirtyday
\lnr{17}Tandem Cleostratus Tenedius Octaeteride Lunari edita, ab Atheniensibus
scripto expressit, quod Solon eloquentia impetrare non potuerat.
\lnr{19}Cum enim iam omnibus persuasum esset, annum Lunarem esse dierum
trecentorum quinquaginta quatuor, Solarem vero Lunari maiorem
esse diebus undecim praecise cum quadrante, Cleostratus animadvertit
octo Lunares annos cum totidem excessibus Solaribus conficere[?] syzygias
% syzygia = conjunction
nonaginta novem, id est dies 2922: quot scilicet diebus octo anni
Solares constant.
% 29 1/2 days per lunar month, 12 lunar months per lunar year = 354 days/l.y.
% 11 1/4 extra days to make a solar year = 365 1/4 days/s.y.
% "8 lunar years plus all the extra days make 99 conjuctions" (lunar months)
% 8 x (354 + 11 1/4) = 2922 days
% 2922 days / 29 1/2 days per lunar month = 99,05085 new moons
\lnr{24}Ergo in octo annis Solaribus totidem syzygias praecise
transigi existimavit: quarum syzygiarum quadraginta octo sint
cavae, reliquae plenae \textgreek{κὰι τριακοντήμερα}.
\lnr{26}Atque hoc intervallo dierum
et syzygiarum putavit \textgreek{τῶν φαινομένων ἀποκατάστασιν[?]} fieri,
 et Lunam
cum Sole, itemque omnium \textgreek{φαινομένων[?]}
 ortus et occasus ad idem punctum
redire, a quo caeperint primum.
\lnr{29}Iste autem Cleostratus (alibi
Leostratum invenio, sed male ut puto) primus Graecorum, et signorum
partes in Zodiaco notavit, et initia Arietis ac Sagittarii, ut ex Plinio
et Hygeno colligimus: idque[?] circa Olympiadem \rnum{lxi}.
% accent on idque: what do we do?
\lnr{32}Instituit
autem caput Octaeteridis a brumalibus diebus necessario, tum quia
eius castigator Harpalus postea idem tempus servavit, tum quia
 \textgreek{ποσειδεῲν δέυτερος[?]}
ostendit Gamelionem hactenus fuisse mensem primum
anni Attici, cum omnis intercalatio debeatur fini anni, Posideon autem
\textgreek{δέυτερος[?]} sit intercalaris.
\lnr{37}Sed Brumam intelligendum est non \textgreek{ἀστρονομικῶς[?]},
sed \textgreek{πολιτικῶς[?]}, quomodo dies civiles agebant Attici.
\lnr{38}Saeculo
enim Iphiti, qui primam Olympiada instauravit, aequinoctium vernum
conficiebatur in \rnum{xxiix} Martii: Solstitium Kalend. Iulii.

% 65
% {PDF page nr}{source page nr}{line nr}
\plnr{148}{65}{1}Sed neomenia
primi mensis Olympici incidit in \rnum{ix} Iulii, hoc est,
 in \rnum{viii}, aut
\rnum{ix} gradum Cancri.
\lnr{3}Neque unquam illam epocham antevertebat.
% Nor was this epoch ever prefered
\lnr{4}Quare primus omnium Cleostratus ostendit Graecis popularibus suis
cardines mundi esse in \rnum{viii} partibus Signorum: idque omnis posteritas
credidit adeo, ut Sosigenes idem Caesari, Caesar posteritati persuaserit.
\lnr{7}Quod si octavus gradus Cancri in \rnum{viii} Iulii, ergo octavus gradus
Capricorni in sexta Ianuarii, cum intervallum sit dierum 182, cum
dimidio.
\lnr{9}Quod si primus annus Olympiadicus non fuisset intercalaris,
neomenia mensis brumalis, qui est Gamelion Atticus, convenisset
in \rnum{vii} Ianuarii, statim post octavum gradum Capricorni.
\lnr{11}Nam a capite
Hecatumbaeonis ad caput Gamelionis, sunt praecise sex menses, et
duae praeterea \textgreek{ἄναρχοι ἡμέραι[?]}:
 qui sunt dies 182, quot scilicet ab \rnum{viii} gradu
Cancri ad \rnum{viii} Capricorni.
% In fact, from the start of Hecatumbaion to the start of Gamelion, there are
% precisely six months, and two extra ἄναρχοι ἡμέραι [anarchic days]:
% this makes 182 days, the number of which is or course from the 8th degree
% of Cancer to the 8th degree of Capricorn.
\lnr{14}Ergo citima neomenia mensis brumalis
est in \rnum{vii} Ianuarii, quos fines nunquam superabit; set embolismi
interventu in Februarium summovebitur.
% Therefore, the nearest new moon in the winter month is on the seventh of
% January, whos ends it will always overflow;
% But embolismic interventions will be withheld until February.
\lnr{16}Omnes igitur menses
alternis sunt pleni et cavi: \textgreek{Ποσειδεὼν[?]}
 etiam \textgreek{δέντερος[?]} plenus.
% All months are therefore alternatively full and short:
% Poseideon [6th month in the Attic calendar] likewise ?? full
%
% Insert table:
% Octaeteris Cleostrati
% -- Placement of the table uncertain. There is no clear indication where it
%    connects to the body text.
% -- There is no column header above "8 Ianua." in the original.
%    In concordance with table
%    038_neomenia_elidensis it might be "Neomenia 1. mensis" or similar.
\begin{table}[htbp]
 \centering
 \renewcommand{\arraystretch}{1.3}
 %%% Liber II p65
%% Wider variation, where the headers are written horizontally
%%
%%% Count out columns for fixed-width source font
% 000000011111111112222222222333333333344444444445555555555666666666677777777778
% 345678901234567890123456789012345678901234567890123456789012345678901234567890
%
%\tiny
%\scriptsize
%\footnotesize
%\small
\normalsize
%% Center the whole table left-right
\centering
%% Modify separation between columns
%\setlength{\tabcolsep}{1.6pt}
%% Modify distance between rows
\renewcommand{\arraystretch}{1.3}
%% Angle to rotate the headers
%\newcommand{\ang}{60}
%%
\begin{tabular}[t]{r c c c c c}
~ & \multicolumn{5}{c}{\Large\textsc{Octaeteris Cleostrati}}\\
\cline{2-6}
~ &
\multicolumn{1}{c}{Anni} &
\multicolumn{1}{c}{Cyclus} &
\multicolumn{1}{c}{Liter} &
~ &
\multicolumn{1}{c}{Dies}
\\
~ &
\multicolumn{1}{c}{octaeteridis} &
\multicolumn{1}{c}{Lunnae} &
\multicolumn{1}{c}{Dominicalis} &
~ &
\multicolumn{1}{c}{collecti}
\\
\cline{2-6}
\scriptsize{†}
  &  1 & 18 &  E &  8 Ianua. &  384 \\
~ &  2 & 19 & DC & 27 Ian.   &  738 \\
\scriptsize{†}
  &  3 &  1 &  B & 15 Ian.   & 1122 \\
~ &  4 &  2 &  A &  3 Febr.  & 1476 \\
~ &  5 &  3 &  G & 23 Ian.   & 1830 \\
\scriptsize{†}
  &  6 &  4 & FE & 12 Ian.   & 2214 \\
~ &  7 &  5 &  D & 30 Ian.   & 2568 \\
~ &  8 &  6 &  C & 19 Ian.   & 2922 \\
\cline{2-6}
\\
~ & \multicolumn{5}{l}{\footnotesize \super{†} \textgreek{Εμβολ.}}\\
\end{tabular}
\caption{Octaeteris Cleostrati}
\label{tab:p065}
%
 \caption{Octaeteris Cleostrati}
 \label{tab:octaeteris_cleostrati}
\end{table}
%
\lnr{17}Scripta est
autem Octaeteris, ut diximus, anno secundo Olympiadis sexagesimae
primae, annis decem post observatam ab Aneximandro obliquitatem
Zodiaci.
% The Octaeteris was however described, as we have said, in the year two
% of the sixty-first Olympiad, ten years after the observation by Aneximander
% of the obliquity of the zodiac.
\lnr{20}Erat cyclus Lunae \rnum{xix},
 Solis \rnum{viii}. anno Iudaico 3227, cuius
Schebat 4.1.76. Ianuarii \rnum{viii}, anno periodi Atticae quinquagesimo
secundo, \textgreek{ποσειδεῶνοσ ἔνῃ καὶ νέᾳ[?]}.
% It was Lunar cycle 19, Solar cycle 8, the Jewish year 3227, being Schebat
% [Jewish month between Tevet and Adar, roughly in Jan-Feb] 4.1.76.
% January 8, the fifty-second year of the Attic period, ποσειδεῶνοσ ἔνῃ καὶ νέᾳ.
% [Poseideon (month 6 in the Attic calendar) Old-and-New (i.e. the 30th day)]
% Leartus about Solon: Πρῶτος δὲ Σόλων τὴν τριακάδα ἔνην καὶ νέαν ἐκάλεσεν.
% "Solon was the first to call the 30th day of the month the Old-and-New day."
% "Lives and opinions..." I.58
\lnr{22}Haec Octaeteris primum fuit initium annorum
et mensium Lunarium \textgreek{πρυτανείας}: quam ipse cum Canone stellarum
Orientium et Occidentium et earum significationibus publicavit.
% And so the first Octaeteris started on the lunar month and year πρυτανείας[??].
% This is when he [Cleostratus] published the Eastern and Western stellar
% Canons and their significance.
\lnr{25}Eum Canonem veteres Graeci
\textgreek{παράπηγμα} vocant.
% This Canon the ancien Greeks called παράπηγμα.
% [Greek: Stall, booth, shed, stand]
\lnr{26}Geminus: \textgreek{Αἱ δὲ
γινόμεναι προῤῥήσεις τῶν ἐπισημασιῶν ἐν τοῖς
παραπήγμασιν οὐκ ἀπό\footnote{No space in De Emendatione} τινων παραγγελμάτων
[ὡρισμένων]\footnote{Word from Geminius missing in De Emendatione} γίνονται}.
% Geminus of Rhodes: Introduction to Phaenomena, ca 1st century BCE
% (Γεμῖνος ὁ Ῥόδιος: Εἰσαγωγὴ εἰς τὰ Φαινόμενα)
% Also simply known as the Isagoge.
% Chapter 16 "Περὶ ἐπισημασιῶν τῶν ἄστρων"
% [About predictions from the stars]
% Second paragraph
% Αἱ δὲ
% γινόμεναι προρρήσεις τῶν ἐπισημασιῶν ἐν τοῖς
% παραπήγμασιν οὐκ ἀπό τινων παραγγελμάτων
% ὡρισμένων γίνονται,
% [οὐδὲ τέχνῃ τινὶ μεθοδεύονται κατηναγκασμένον ἔχουσαι τὸ ἀποτέλεσμα,
% ἀλλ' ἐκ τοῦ ὡς ἐπίπαν γινομένου διὰ τῆς καθ' ἡμέραν παρατηρήσεως τὸ
% σύμφωνον λαμβάνοντες εἰς τὰ παραπήγματα κατεχώρισαν.]
% German translation:
% Die üblichen vorläufigen Angaben der Witterungsanzeichen in den Kalendern
% werden nicht nach bestimmten Regeln gemacht, noch beruhen sie auf einer
% wissenschaftlichen Methode, nach der sie Anspruch auf notwendige Erfülung
% hätten, sondern (die Kalendermacher) haben aus den regelmäsig eintretenden
% Erscheinungen mit Hilfe der täglichen Beobachtung das herausgenommen, was
% ihnen passt, und es in ihre Kalender gesetzt.
% == Carolus Manitius (1898), Gemini Elementa Astronomiae, ad codicum fidem
% recensuit germanica interpretatione et commentariis instruxit.
% From Google Books
\lnr{29}Hic \textgreek{παραπήγματα} vocat \textgreek{τὰς
προῤῥήσεις τῶν ἐπισημασιῶν}, \textit{Tempora quae
messor, quae curvus arator haberet.}
% P. VERGILI MARONIS ECLOGA TERTIA
% Publius Vergillius Maro (Virgil): Ecloga Tertia
% "That they who reap, or stoop behind the plough,
% Might know their several seasons?"
% == Project Gutenberg
\lnr{31}Vitruvius libro nono: \textit{Quorum inventa
secuti, siderum ortus, et occasus, tempestatumque
significatus Eodoxus,
Euctemon, Calippus, Meto, Philippus,
% Accent on Meto
Hipparchus, Aratus, invenerunt,
caeterique ex astrologia, parapegmatorum
disciplinas invenerunt, et
eas posteris explicatas reliquerunt.}
\lnr{39}\textit{Quorum
scientiae sunt hominibus suspiciendae,
quod tanta cura fuerunt, ut etiam videantur divina mente tempestatum
significatus post futuros ante pronunciare.}
% Vitruvius: De Architectura Libri Decem
% Vitruvius: Ten Books on Architecture
% Book 9, chapter VI (Astrology and weather prognostics),
% paragraph 3, 2nd and 3rd sentence
% 3. [When we come to natural philosophy, however, Thales of Miletus,
% Anaxagoras of Clazomenae, Pythagoras of Samos, Xenophanes of Colophon, and
% Democritus of Abdera have in various ways investigated and left us the laws
% and the working of the laws by which nature governs it.]
% In the track of their discoveries, Eudoxus, Euctemon, Callippus, Meto,
% Philippus, Hipparchus, Aratus, and others discovered the risings and settings % of the constellations, as well as weather prognostications from astronomy
% through the study of the calendars, and this study they set forth and left
% to posterity. Their learning deserves the admiration of mankind; for they
% were so solicitous as even to be able to predict, long beforehand, with
% divining mind, the signs of the weather which was to follow in the future.
% [On this subject, therefore, reference must be made to their labours and
% investigations.]

% 66
% {PDF page nr}{source page nr}{line nr}
\plnr{149}{66}{1}Satis clare innuit,
quid, sit \textgreek{παράπηγμα[?]}.
% This clearly indicates that this should be παράπηγμα
\lnr{2}In vulgatis editionibus Vitruvii legitur Eudaemon
Callistus, Melo; pro quo correximus Euctemon, Calippus,
Meto.
% The common edition of Vitruius reads "Eudaemon, Callistus, Melo". We
% corrected this to "Euctemon, Calippus, Meto".

\subsection{De Octaeteride Harpali}

\lnr{5}Octaeteridis Cleostrateae vitium cito deprehensum est,
quod duae Tetraeterides Olympicae cum mense embolimo sint
dierum solidorum 2924, Octaeteris autem Cleostrati dierum
totidem, duobus minus.
\lnr{8}Atqui neomeniae primae Tetraeteridos, et tertiae
ineuntium incidunt in novilunia, ut uberrime a nobis libro priore
disputatum est.
\lnr{10}Quare neomenia Octaeteridis secundae Cleostrateae
incidens in diem penultimum Tetraeteridis secundae, anticipabit novilunium
biduo solido.
\lnr{12}Vitiosa igitur est Octaeteris Cleostrati.
\lnr{12}Cum
igitur intervallum duarum Tetraeteridum inter duo novilunia interiectum
sit, non dubitavit Harpalus, quin illud sit ex iustis syzygiis compositum.
\lnr{15}Omne enim intervallum in idem punctum Lunae desinens,
a quo caeperat, est mere Lunare, hoc est meris mensibus Lunaribus
constans.
\lnr{17}Nam nisi veteres Graeci mensem Lunarem censuissent esse
dierum 29, horarum 12~\myfrac{1}{2}.
\lnr{18}Nunquam spatio dierum 2964[?] iustas syzygias
Lunares fieri posse existimassent.
\lnr{19}Hoc modo annus Lunaris est
dierum 354 horarum 12.
\lnr{20}Qui dies et horae octies multiplicata dant dies
absolutos 2834.
\lnr{21}Qui de 2924 detracti relinquunt 90 dies, qui sunt
menses tres pleni embolimi.
\lnr{22}Quod si dies 2924 per 59 dies dividantur,
habedimus in duabus Tetraeteridibus quadraginta novem paria
mensium alternis plenorum et cavorum, cum diebus praeterea triginta
tribus, hoc est, quinquaginta menses plenos, undequinquaginta
cavos, et tres dies insuper.
\lnr{26}Igitur in Octaeteride, quae constat duabus
Tetraeteridibus Olympicis, sunt syzygiae Lunares nonaginta
novem: quae si essent omnes plenae, fierent omnes dies 2970: de quibus
detractis 2924 diebus duarum Tetraeteridum, remanent 46, differentia
cavorum et plenorum mensium.
\lnr{30}Quadraginta igitur et sex
menses cavi sunt in Octaeteride: et proinde quinquaginta tres erunt
pleni.
\lnr{32}Quae oeconomia mensium immane quantum discrepat a Cleostratea.
\lnr{33}Nam in illa sunt menses alternis pleni, et cavi: in hac tertius
fere mensis est cavus, et aliquando quartus.
\lnr{34}Divisis enim 99 per
46, remanent 2~\myfrac{7}{46} id est duae syzygiae plenae,
 cum \myfrac{7}{46} unius syzygiae.
\lnr{36}Igitur tertius mensis duntaxat est cavus, idque donec ex
 \myfrac{7}{46} consurgat
integra syzygia.
\lnr{37}Tunc enim non iam tertius, sed quartus mensis est cavus,
ut et docet progressus arithmeticus, et potes ex subiecta tabella
animadvertere.

% Table: ΜΗΝΕΣ ΚΟΙΛΟΙ
\begin{table}[htbp]
 \centering
 \footnotesize
 \renewcommand{\arraystretch}{1.3}
 %%% Liber II p67
%%
%%% Count out columns for fixed-width source font
% 000000011111111112222222222333333333344444444445555555555666666666677777777778
% 345678901234567890123456789012345678901234567890123456789012345678901234567890
%
% ΜΗΝΕΣ ΚΟΙΛΟΙ
% "The hollow months" "cavae menses"
%
% This table apparently lists the hollow months that occur in each year
% of the 8 year cycle (octaeteride).
% Each row shows the information for one year in the octaeteride.
% Column 1 indicated if a year is Embolic.
% We have represented the abbreviated word ἐμβολ. by a footnote symbol,
% the same way we did for other tables where it occurs.
% Column 2 uses Byzantine Greek numerals (without a bar above the letters)
% to number the years in the cycle.
% The number 6 is represented by a cursive digamma, rendered here as
% Unicode U03DA.
% The header for the second column is allmost illegible
% ?? ?? ὀκταετερίδος ??
% Columns 3-8 list the 5 or 6 months in the Attic calendar which are hollow.
% Column 9 shows a β. or α. for the ebolic years. Though not explained in the
% table itself, they are probably Byzantine Greek numerals again. The original
% has a bar over the α, but not over the β, indicating they might be numerals.
% Such symbols can be made using the Math features of Tex, e.g.
% $\overbar{\kappa\alpha}$ for '21'.
% Because there is no code for the digamma (representing 6), we need to resort
% to using \varsigma (the end-of-word sigma) for this symbol.
%
%% For testing, uncomment the folowing lines and the lines at the end
%% of the file
%% Test ==>
%\documentclass{book}
%\usepackage{fontspec}
%\setmainfont{Hoefler Text}[]
%%\setmainfont{Times New Roman}[]
%\newfontfamily\greekfont{Times New Roman}
%\usepackage[quiet]{polyglossia}
%\setmainlanguage{latin}
%\setotherlanguage{greek}
%\begin{document}
%% <== Test
%%
\begin{tabular}{ll|lllllll}
\multicolumn{9}{c}{\large{\textgreek{ΜΗΝΕΣ ΚΟΙΛΟΙ}}}
\\
~ & \multicolumn{5}{l}{\textgreek{τ.. τ.. ὀκταετερίδος .τ.}}
\\
\hline
\scriptsize{†} &
\textgreek{α} &
\textgreek{ἐλαφηβολ.} &
\textgreek{θαργηλ.} &
\textgreek{ἑκατομβ.} &
\textgreek{βοηδρομ.} &
\textgreek{μαιμακτ.} &
\textgreek{ποσειδ.} &
\textgreek{β.}
\\
 &
\textgreek{β} &
\textgreek{ἐλαφηβολ.} &
\textgreek{θαργηλ.} &
\textgreek{ἑκατομβ.} &
\textgreek{βονδρομ.} &
\textgreek{μαιμακτ.} &
 &

\\
\scriptsize{†} &
\textgreek{γ} &
\textgreek{γαμηλ.} &
\textgreek{ἐλαφηβολ.} &
\textgreek{σκιῤῥοφ.} &
\textgreek{μεταγείτν.} &
\textgreek{πυανεψ.} &
\textgreek{ποσειδ.} &
\textgreek{α.}
\\
\hline
 &
\textgreek{δ} &
\textgreek{γαμηλ.} &
\textgreek{ἐλαφηβολ.} &
\textgreek{σκιῤῥοφ.} &
\textgreek{βονδρομ.} &
\textgreek{μαιμακτ.} &
 &

\\
 &
\textgreek{ε} &
\textgreek{ανθεστηρ.} &
\textgreek{μουνιχ.} &
\textgreek{σκιῤῥοφ.} &
\textgreek{βονδρομ.} &
\textgreek{μαιμακτ.} &
 &

\\
\scriptsize{†} &
\textgreek{Ϛ} &
\textgreek{γαμηλ.} &
\textgreek{ἐλαφηβολ.} &
\textgreek{θαρυηλ.} &
\textgreek{ἑκατομβ.} &
\textgreek{βονδρομ.} &
\textgreek{ποσειδ.} &
\textgreek{α.}
\\
\hline
 &
\textgreek{ζ} &
\textgreek{γαμηλ.} &
\textgreek{ἐλαφηβολ.} &
\textgreek{θαρυηλ.} &
\textgreek{ἑκατομβ.} &
\textgreek{βονδρομ.} &
\textgreek{ποσειδ.} &

\\
 &
\textgreek{η} &
\textgreek{ανθεστηρ.} &
\textgreek{μουνιχ.} &
\textgreek{σκιῤῥοφ.} &
\textgreek{μεταγείτν.} &
\textgreek{πυανεψ.} &
\textgreek{ποσειδ.} &

\\
\hline
\multicolumn{5}{l}{\footnotesize \super{†} \textgreek{ἐμβολ.}}\\
\end{tabular}
%% Test ==>
%\end{document}

 \caption{\textgreek{Μενες κοιλοι}}
 \label{tab:menes_koiloi}
\end{table}

% Table: ΝΕΟΜΗΝΙΑΙ ΤΗΣ ΟΚΤΑΕΤΗΡΙΔΟΣ
\begin{table}[htbp]
 \centering
 \scriptsize
 %% Modify distance between rows
 \renewcommand{\arraystretch}{1.8}
 %% Modify separation between columns
 \setlength{\tabcolsep}{2.0pt}
 %%% Liber II p67, PDF 150
%%
%%% Count out columns for fixed-width source font
% 000000011111111112222222222333333333344444444445555555555666666666677777777778
% 345678901234567890123456789012345678901234567890123456789012345678901234567890
%
%% Select a general font size (uncomment one from the list)
%\tiny
%\scriptsize
%\footnotesize
%\small
%\normalsize
%% Center the whole table left-right
\centering
%% Modify distance between rows
%\renewcommand{\arraystretch}{1.8}
%% Modify separation between columns
%\setlength{\tabcolsep}{2.0pt}
%%
\begin{tabular}{l llllllll}
\multicolumn{9}{ c }{\large\textgreek{ΝΕΟΜΗΝΙΑΙ ΤΗΣ ΟΚΤΑΕΤΗΡΙΔΟΣ}}\\
\multicolumn{9}{ c }{\normalsize\textgreek{ΚΑΘ' ΕΚΑΣΤΟΝ ΕΤΟΣ}}\\
%\hline
\textgreek{Μηνες} &
\textgreek{ἔτος} &
\textgreek{ἔτος} $\overline{\beta}$ &
\textgreek{ἔτος} $\overline{\gamma}$ &
\textgreek{ἔτος} $\overline{\delta}$ &
\textgreek{ἔτος} $\overline{\epsilon}$ &
\textgreek{ἔτος} $\overline{\varsigma}$ &
\textgreek{ἔτος} $\overline{\zeta}$ &
\textgreek{ἔτος} $\overline{\eta}$
\\
\textgreek{κατὰ σελήνην.} &
~\textgreek{πρῶτον.}
\\
\hline
\textgreek{γαμηλιών.} &
$\overline{\alpha}$          \textgreek{γαμηλ.} &
$\overline{\kappa\epsilon}$  \textgreek{ποσειδ.} &
$\overline{\iota\eta}$       \textgreek{ποσειδ.} &
$\overline{\eta}$            \textgreek{γαμηλ.} &
$\overline{\alpha}$          \textgreek{γαμηλ.} &
$\overline{\kappa\varsigma}$ \textgreek{ποσειδ.} &
$\overline{\iota\varsigma}$  \textgreek{γαμηλ.} &
$\overline{\eta}$            \textgreek{γαμηλ.}
\\
\textgreek{ανθεστηριών.} &
$\overline{\alpha}$          \textgreek{ἀνθεστ.} &
$\overline{\kappa\gamma}$    \textgreek{γαμηλ.} &
$\overline{\iota\epsilon}$   \textgreek{γαμηλ.} &
$\overline{\eta}$            \textgreek{ἀνθεστ.} &
$\overline{\alpha}$          \textgreek{ἀνθεστ.} &
$\overline{\kappa\gamma}$    \textgreek{γαμηλ.} &
$\overline{\iota\epsilon}$   \textgreek{ἀνθεστ.} &
$\overline{\eta}$            \textgreek{ἀνθεστ.}
\\
\textgreek{ἐλαφηβολιών.} &
$\overline{\alpha}$          \textgreek{ἐλαφη.} &
$\overline{\kappa\gamma}$    \textgreek{ἀνθεστ.} &
$\overline{\iota\epsilon}$   \textgreek{ἀνθεστ.} &
$\overline{\zeta}$           \textgreek{ἐλαφη.} &
$\overline{\lambda}$         \textgreek{ἀνθεστ.} &
$\overline{\kappa\gamma}$    \textgreek{ἀνθεστ.} &
$\overline{\iota\epsilon}$   \textgreek{ἐλαφη.} &
$\overline{\zeta}$           \textgreek{ἐλαφη.}
\\
\hline
\textgreek{μυονυχιών.} &
$\overline{\lambda}$         \textgreek{ἐλαφη.} &
$\overline{\kappa\beta}$     \textgreek{ἐλαφη.} &
$\overline{\iota\delta}$     \textgreek{ἐλαφη.} &
$\overline{\zeta}$           \textgreek{μυονυχ.} &
$\overline{\lambda}$         \textgreek{ἐλαφη.} &
$\overline{\kappa\beta}$     \textgreek{ἐλαφη.} &
$\overline{\iota\delta}$     \textgreek{μυονυχ.} &
$\overline{\zeta}$           \textgreek{μυονυχ.}
\\
\textgreek{θαργηλιών.} &
$\overline{\lambda}$         \textgreek{μυονυχ.} &
$\overline{\kappa\beta}$     \textgreek{μυονυχ.} &
$\overline{\iota\delta}$     \textgreek{μυονυχ.} &
$\overline{\varsigma}$       \textgreek{θαργη.} &
$\overline{\kappa\theta}$    \textgreek{μυονυχ.} &
$\overline{\kappa\beta}$     \textgreek{μυονυχ.} &
$\overline{\iota\delta}$     \textgreek{θαργη.} &
$\overline{\varsigma}$       \textgreek{θαργη.}
\\
\textgreek{σκιῤῥοφοριών.} &
$\overline{\kappa\theta}$    \textgreek{θαργη.} &
$\overline{\kappa\alpha}$    \textgreek{θαργη.} &
$\overline{\iota\delta}$     \textgreek{θαργη.} &
$\overline{\varsigma}$       \textgreek{σκιῤῥ.} &
$\overline{\kappa\delta}$    \textgreek{θαργη.} &
$\overline{\kappa\alpha}$    \textgreek{θαργη.} &
$\overline{\iota\gamma}$     \textgreek{σκιῤῥ.} &
$\overline{\varsigma}$       \textgreek{σκιῤῥ.}
\\
\hline
\textgreek{ἑκατομβαιών.} &
$\overline{\kappa\theta}$    \textgreek{σκιῤῥ.} &
$\overline{\kappa\alpha}$    \textgreek{σκιῤῥ.} &
$\overline{\iota\gamma}$     \textgreek{σκιῤῥ.} &
$\overline{\varsigma}$       \textgreek{ἑκατο.} &
$\overline{\kappa\theta}$    \textgreek{σκιῤῥ.} &
$\overline{\kappa\alpha}$    \textgreek{σκιῤῥ.} &
$\overline{\iota\gamma}$     \textgreek{ἑκατο.} &
$\overline{\epsilon}$        \textgreek{ἑκατο.}
\\
\textgreek{μεταγειτνιών.} &
$\overline{\kappa\eta}$      \textgreek{ἑκατο.} &
$\overline{\kappa}$          \textgreek{ἑκατο.} &
$\overline{\iota\gamma}$     \textgreek{ἑκατο.} &
$\overline{\epsilon}$        \textgreek{μεταγ.} &
$\overline{\kappa\eta}$      \textgreek{ἑκατο.} &
$\overline{\kappa}$          \textgreek{ἑκατο.} &
$\overline{\iota\beta}$      \textgreek{μεταγ.} &
$\overline{\epsilon}$        \textgreek{μεταγ.}
\\
\textgreek{βοηδρομιών.} &
$\overline{\kappa\eta}$      \textgreek{μεταγ.} &
$\overline{\kappa}$          \textgreek{μεταγ.} &
$\overline{\iota\beta}$      \textgreek{μεταγ.} &
$\overline{\epsilon}$        \textgreek{βοηδρ.} &
$\overline{\kappa\eta}$      \textgreek{μεταγ.} &
$\overline{\kappa}$          \textgreek{μεταγ.} &
$\overline{\iota\beta}$      \textgreek{βοηδρ.} &
$\overline{\delta}$          \textgreek{βοηδρ.}
\\
\hline
\textgreek{πυανεψιών.} &
$\overline{\kappa\zeta}$     \textgreek{βοηδρ.} &
$\overline{\iota\theta}$     \textgreek{βοηδρ.} &
$\overline{\iota\beta}$      \textgreek{βοηδρ.} &
$\overline{\epsilon}$        \textgreek{πυαν.} &
$\overline{\kappa\zeta}$     \textgreek{βοηδρ.} &
$\overline{\kappa}$          \textgreek{βοηδρ.} &
$\overline{\iota\alpha}$     \textgreek{πυαν.} &
$\overline{\delta}$          \textgreek{πυαν.}
\\
\textgreek{μαιμακτηριών.} &
$\overline{\kappa\zeta}$     \textgreek{πυαν.} &
$\overline{\iota\theta}$     \textgreek{πυαν.} &
$\overline{\iota\alpha}$     \textgreek{πυαν.} &
$\overline{\delta}$          \textgreek{μαιμα.} &
$\overline{\kappa\zeta}$     \textgreek{πυαν.} &
$\overline{\iota\theta}$     \textgreek{πυαν.} &
$\overline{\iota\alpha}$     \textgreek{μαιμα.} &
$\overline{\gamma}$          \textgreek{μαιμα.}
\\
\textgreek{ποσειδεών.} $\overline{\alpha}$&
$\overline{\kappa\varsigma}$ \textgreek{μαιμα.} &
$\overline{\iota\eta}$       \textgreek{μαιμα.} &
$\overline{\iota\alpha}$     \textgreek{μαιμα.} &
$\overline{\gamma}$          \textgreek{ποσειδ.} &
$\overline{\kappa\varsigma}$ \textgreek{μαιμα.} &
$\overline{\iota\theta}$     \textgreek{μαιμα.} &
$\overline{\iota\alpha}$     \textgreek{ποσειδ.} &
$\overline{\gamma}$          \textgreek{ποσειδ.}
\\
\textgreek{ποσειδεών.} $\overline{\beta}$&
$\overline{\kappa\varsigma}$ \textgreek{ποσειδ.} $\overline{\alpha}$ &
    \multicolumn{1}{c}{$\circ$} &
$\overline{\iota}$           \textgreek{ποσειδ.} &
    \multicolumn{1}{c}{$\circ$} &
    \multicolumn{1}{c}{$\circ$} &
$\overline{\iota\eta}$       \textgreek{ποσειδ.} &
    \multicolumn{1}{c}{$\circ$} &
~
\\
\end{tabular}
%
\caption{Neomeniai tes oktaeteridos kath ekaston etos}
 \caption{\textgreek{Νεομηνιαι της Οκταετηριδος}}
 \label{tab:neomeniai_tes_oktaeteridos}
\end{table}

% 67
% {PDF page nr}{source page nr}{line nr}
\plnr{150}{67}{1}Subiecims praeter ea Tabulam neomeniarum Lunarium secundum
menses anni politici aequabilis, id est secundum menses Tetraetericos.
\lnr{3}Haec enim est vera Harpaleae Octaeteridis \textgreek{ψηφοφορία[?]}:
% Greek: vote (politics)
 quippe
quae nigil aliud est, quam mensium Lunarium cum aequabilibus
comparatio.
\lnr{5}Ideo neomeniam Lunaris Gamelionis cum neomenia
Gamelionis aequabilis in primo anno composuimus: non quod
ita fecerit Harpalus: (Nullus enim Gamelion aequabilis eo seculo
Lunaris fuit:) sed quia deprehenso anno primo Octaeteridis Harpali,
non operosum erit divinare quotae diei Gemelionis aequabilis
competat neomenia Gamelionis Lunaris.
\lnr{10}Nam si, verbi gratia,
neomenia primi Gamelionis Lunaris Harpelei incidit in tertiam
diem Gamelionis aequabilis, reliquae omnes neomeniae eundem progressum
servabunt: puta per triduum omnes neomeniae promovendae
erunt.

% 68
% {PDF page nr}{source page nr}{line nr}
\plnr{151}{68}{2}Sed non magis scimus epocham capitis illius Octaeteridis,
quam patriam ipsius Harpali.
\lnr{3}Constat tamen initium fecisse a bruma,
ut est testis Festus Avienus in Arateis:
\begin{quote}
\emph{Non ego nunc longo redeuntia sidera motu\\
In priscas memorem sedes. Habet ista priorum\\
Pagina, et incerta rerum ratione ferentur.\\
Nam quae solem hiberna novem putat aethere volui,\\
Ut Lunae spatium redeat, vetus Harpalus, ipsa\\
Ocius in sedes, momentaque prisca reducit.}
\end{quote}
% Avieni - Aratea, lines 1363-1368
% "Postumius Rufius Festus who is also Avienius"
% He made a somewhat inexact translation of Aratus' didactic poem "Phaenomena",
% the first part of which was in turn a verse setting of Phaenomena by Eudoxus
% of Cnidus.
\lnr{11}Ex quibus cognoscimus et a bruma incepisse,
 et \textgreek{ἐννεαετηρίδα[?]} quoque
vocasse, non, quod annis novem solidis constaret, ut hallucinatur
Festus, sed qua nono quoque anno in orbem rediret: quemadmodum
pentaeteris et trieteris dictae sunt, non a numero annorum, quibus
constabant, sed eorum, quibus ineuntibus \textgreek{ἀποκατάστασιος[?]} fiebat:
quemadmodum quartanam febrem dicimus, non quod intervalla habeat
quaternum dierum, sed quod quarto quoque die in orbem redeat.
\lnr{18}Igitur Harpali Octaeteris admissa fuit cum parapegmate
 et significationibus
stellarum, et haec est periodus secunda \textgreek{πρυτανείας[?]},
 qua Athenienses
usi sunt.
\lnr{20}Plinius quoque significationes siderum et tempestatum
\textgreek{προῤῥήσεις[?]} ex Harpalo citat in indice libri \rnum{xviii}:
 quae non aliunde
petitae, quam ex eius parapegmate.
\lnr{22}Cum autem haec Octaeteris sit
dierum 2924: ipsi dies per octo divisi dant quantitatem anni secundum
Harpalum, dierum 365~\myfrac{1}{2} sive horarum aequinoctialium 12.
\lnr{25}Quare manifestum mendum est in Censorino, ubi differitur de anni
Solaris quantitate secundum Harpalum.
\lnr{26}Nam ibi legimus Harpalum
definisse annum Solis dierum tercentorum sexaginta quinque,
et tredicem horarum praeterea aequinoctialium.
\lnr{28}Omnino enim legendum duodecim horarum.
\lnr{29}Et ne quis dubitet, idem error est infra
apud eundem, ubi dicit Arminon regem Aegypti annum ad tredecim
menses et quinque dies perduxisse.
\lnr{31}Annus Aegyptius non est tredecim,
sed duodecim mensium.
\lnr{32}Alioquin Octaeteridi Harpaleae superessent
horae octo supra 2924.
\lnr{33}Atqui ita nulla fieret aequatio.
\lnr{33}Siquidem
omnis aequationis lex iubet nihil reliquum fieri de ratione horaria,
aut scrupularia.
\lnr{35}Imo tantum abest, ut illarum octo horarum accessione
rationes Solis cum Lunaribus parient, ut etiam fine illis iusto
longior sit Octaeteris, ut alibi demonstratur.
\lnr{37}Hanc Octaeterida exclusit
Enneadecaeteris Euctemonis, et Metonis: De qua proxime dicendum
erat, nisi Eudoxi Octaeteris nos revocaret, quae Metonis quidem
periodo posterior est; propter cognationem autem materiae in
continenti reliquis Octaeteridibus subiiciendam esse arbitrati sumus.

% 69
% {PDF page nr}{source page nr}{line nr}
\plnr{152}{69}{2}De ea igitur prius dicendum.

\subsection{De Octaeteride Eudoxi}
\lnr{3}Eudoxus Cnidius, vir suo saeculo eruditissimus, et Mathematicorum
princeps, in Aegyptum profectus, ibi annum et menses
praeterea integros quatuor, facerdotibus et astrologis operam dedit,
et Octaeterida suam conscripsit, ut docet nos Laertius: \textgreek{[Greek]}.
\lnr{8}Is igitur cum in Aegypto
esset anno tertio Olympiadis \rnum{ciii}, et eclipsium intervalla, quas in
monimentis suis notatas habebant Aegyptii, inter se compararet, deprehendit
syzygiam secundum Harpalum propius abesse a vero, quam
syzygiam secundum Octaeterida Cleostrateam.
\lnr{12}Nam syzygia secundum
Harpalum est dierum 29, horarum aequinoctialium 12~\myfrac{28}{33}.
\lnr{13}At
secundum Cleostratum todidem quidem dierum et horarum, sed et \myfrac{18}{33}
duntaxat unius horae.
\lnr{15}Ratio igitur differentiae Harpaleae ad rationem
differentiae Cleostrateae est dupla quadripertiens tricesimas tertias.
\lnr{16}Quare cum maiuscula sit Harpalea, quam vera sit syzygia;
 putavit Eudoxus
\myfrac{4}{33} unius horae esse excessum Harpaleae sizygiae
 supra veram syzygiam
Lunarem, quam definivit 29 dierum, 12 horarum, \myfrac{24}{33}, vel, quod
idem est, \myfrac{8}{11} unius horae.
\lnr{20}Ideoque ratio differentiae verae syzygiae, ad rationem
Cleostrateae differentiae, est dupla.
\lnr{21}Cum igitur in Octaeteride
sint syzygiae nonaginta novem, si in 29 dies 12~\myfrac{8}{11}
 horae ducantur, habebis
modum unius verae Octaeteridis dies 2923, horas 12, qui est dimidius
dies.
\lnr{24}Ideoque in duabus Octaeteridibus tres dies supererunt supra
rationes Solis: et consequenter in viginti Octaeteridibus, quae sunt
\textgreek{ἑκκαιδεκαετηρίδες[?]} decem, superabunt dies triginta,
 qui est mensis integer.
\lnr{27}Quare Eudoxi periodus magna constat ex Octaeteridibus viginti,
vel Heckaedecaeteridibus decem, quarum ultima deminuenda sit
diebus triginta.
\lnr{29}Et proinde tota periodus Eudoxi est dierum 58440.
\lnr{30}Qui sunt praecise anni Iuliani 160.
\lnr{30}In annis igitur 160 melius putavit
mensem omitti posse, quam diem in \rnum{xix} annis Metonicis.
\lnr{31}Quod tamen
ineptum est.
\lnr{32}Et tam inconsulte ausus est introducere octaeterida
post enneadecaeterida Metonicam, quam iuste enneadecaeteris mendosa
Metonis omni Octaeteridi praelata est.
\lnr{34}Cum igitur scripserit
anno \rnum{iii} Olympiadis \rnum{ciii}, cyclo Lunae decimo sexto,
 Solis octavo, instituerit
vero initium Octaeteridis a solenni Aegyptiaco, quod vocabant
\textgreek{Ισια[?]}, vel \textgreek{Ισίεια[?]}, quae tunc incidebant
 in tempus sideris brumae
confectae, ut odoramur ex Gemino, non difficile erit divinare, quis dies
Iulianus illi tempori competat, si consideremus, cui diei mensis
 \textgreek{ἀιγῶνος[?]}
Eudoxus in parapegmate suo assignavit diem brumae.
% Ισια: Straight, right

% 70
% {PDF page nr}{source page nr}{line nr}
\plnr{153}{70}{2}Nam cum
Meto et Euctemon incipiant annum suum caelestem a \rnum{xxvii} Iunii,
Brumae autem diem centesimum octagesimum secundum numerent
a Solstitio, Eudoxus quartum diem a bruma Euctemonis dixit esse verum
diem brumae, ut notatum est in Parapegmate Attico, cui congruit
dies 28 Decembris.
\lnr{7}Si igitur \textgreek{Ισια[?]} tunc cadebant in brumam, ea necesse
est incidisse in 28 Decembris, quamuis verus dies brumae incidisset
in 27.
\lnr{9}Erat annus Nabonassari 383.
\lnr{9}Neomenia Toth 23 Novembris,
feria prima.
\lnr{10}Neomenia Paophi 23 Decembris.
\lnr{10}Ergo \textgreek{Ισια[?]} in \rnum{vi}
Paophi.
\lnr{11}Sed duo adversantur.
\lnr{11}Quorum alterum est, quod non in mense
Paophi videntur fuisse \textgreek{Ισια[?]}, sed in mense Athyr: qui, posteaquam
fixus fuit, incidit in Novembrem Iulianum.
\lnr{13}Atqui in Novembrem
Iulianum conferuntur Ifia a veteri poeta in descriptione mensium, quam
reperies in Catelectis nostris.
\lnr{15}Ibi enim de Novembri ita scriptum est:
\begin{quote}
\emph{Carbaseo post hunc artus inductus amictu}\\
\emph{\hspace*{1em}Memphios antiqua sacra, Deamque colit.}\\
\emph{A quo vix avidus sistro compescitur anser,}\\
\emph{\hspace*{1em}Devotusque sacris incola Memphidiis.}
\end{quote}

\lnr{20}Manifesto Ifia sunt in Athyr, non in Paophi.
\lnr{20}Alterum est, quod
Geminus reprehendens sui saeculi errorem, qui semper assignabat brumam
Isiis in anno vago Nabonassari, ait ante 120 annos brumam incidisse
in Isia.
\lnr{23}Ergo a tempore Eudoxi ad tempus illud, quo  suum librum
scribebat Geminus, intersunt anni tantum 120.
\lnr{24}Proinde ille
annus a Nabonassaro fuerit 503.
\lnr{25}Ita Geminus fuerit longe antiquior
Hipparcho.
\lnr{26}Quod non puto.
\lnr{26}Longe enim posterior videtur.
\lnr{26}Ad priorem
quidem dubitationem respondere possumus, solenne Memphiticum
Novembris non esse \textgreek{Ισια[?]}, sed \textgreek{εὕρεσιν Οσίριδος[?]}.
\lnr{28}Nam \textgreek{ἀφδυισμός Οσίριδος[?]}
celebrabatur \rnum{xvii} Athyr, teste Plutarcho, hoc est
\rnum{xiii} Novembris: item \textgreek{εὕρεσις Οσίριδος[?]}
 eodem mense celebrabatur,
ut constat ex Kalendario rustico, quod extat Romae, in quo
in mense Novembri cui Athyr respondet, festum \textsc{euresis} notatum
est.
\lnr{33}\textgreek{Τῆς εὑρέσεως[?]} Rutilius Numatianus
 meminit in suo Iternerario:
\begin{quote}
\emph{Et'tum forte hilares per compita rustica pagi}\\
\emph{\hspace*{1em}Mulcebant sacris pectora fessa iocis}\\
\emph{Illo quippe die tandem renovatus Osiris}\\
\emph{\hspace*{1em}Excitat in fruges semina laeta novas.}
\end{quote}
% Claudius Ritillius Namatianus
% "De Reditu Suo, sive De Reditu in Patriam, sive Iter" ca 415 C.E.
% Liber Primus, Line 374-376
% "Et tum forte hilares per compita rustica pagi
%   mulcebant sacris pectora fessa iocis.
% Illo quippe die tandem revocatus Osiris
%   excitat in fruges germina laeta novas."
% Note: renovatus <-> revocatus, semina <-> germina
\lnr{38}Paulo ante indicaverat se soluisse post ortum Pleiados.
\lnr{38}Aperte notat
illam \textgreek{εὕρέσιν[?]} celebrari mense Novembri.
\lnr{39}Et praeterea erant alia
solennia Isidis: \textsc{ut isidis navigium} mense Martio apud idem Kalendarium
et Lactant.
\lnr{41}Et Apul. lib. \rnum{xi} item \textsc{sacrum phariae,
et sarapia} mense Aprili.

% 71
% {PDF page nr}{source page nr}{line nr}
\plnr{154}{71}{1}Ad alteram autem dubitationem nihil
quod respondeamus, habemus.
\lnr{2}Sed utcunque haec fuerint, Eudoxus
sumsit initium Octaeteridis suae a proximo novilunio post 28 Decembris,
hoc est a Scebat Iudaici anni 3396, cuius character 3.23.726, feria
quarta, Decembris \rnum{xxxi}, cum iam tamen neomenia Posideonis embolimi
Metonici moraretur priscam epocham die uno, et tunc esset in
Kal. Ianuarii.
\lnr{7}Igitur inde Octaeterida suam incepit,
 \textgreek{Ποσειδεῶνος δευτέρου ἔνῃ καὶ νέα[?]}:
% Old-and-new of the second Poseidonis
cuius caput post Iulianos 161 praecipitatur usque in
30 Ianuarii.
\lnr{9}Sed \textgreek{ἐξαιρέσει[?]} triginta dierum iterum in ultimam Decembris
reditur: et ita rationes Solis cum Lunaribus aequantur.

\subsection{Elenchus Octaereridis}

\lnr{11}Ne hoc quidem modo constant rationes Lunares.
\lnr{11}Proxime
quidem abest a vero \textgreek{ἑκκαιδεκαετηρὶς[?]} Eudoxi.
\lnr{12}Nam vera \textgreek{ἑκκαιδεκαετηρὶς[?]}
Iudaica est dierum 5847, ut et Eudoxea, et praeterea
horae 1.414, quo excessu superat Eudoxeam.
\lnr{14}Sed syzygia non est praecise
accepta dierum 29, horarum 12, \myfrac{8}{11}: vel, quod idem est,
 dierum 29~\myfrac{1}{2}
et \myfrac{2}{33}.
% 29.5 + 2/33 days = 29 days, 12 hours + 2/33*24 hours
% 2/33*24 = (2/33)*(3*8) = (2*8)/11 = 16/11, which is *not* equal to 8/11
\lnr{16}(Nam tam dies 29 hor. 12~\myfrac{8}{11}, quam dies 29~\myfrac{1}{2}
 \myfrac{2}{3}
 Octaeteridis syzygiam
consummant. Error autem est in Gemino
\myfrac{\textgreek{α}}{\textgreek{λγ}} pro
 \myfrac{\textgreek{β}}{\textgreek{λγ}},
% Alternatively use math mode:
%  $\alpha\over\lambda\gamma$ pro $\beta\over\lambda\gamma$
  ut hoc obiter
moneam.)
% 1/33 pro 2/33
\lnr{18}Syzygia enim praecise est dierum 29, hor. 12~\myfrac{793}{1080}.
% 29d 12 793/1080h = 29d 12.734259h = 29.530d,
% which matches current known value
\lnr{18}Quare
vera \textgreek{ὑπεροχὴ[?]} Solaris anni supra Lunarem erit
 dierum 10, horarum 21.204.
\lnr{20}Quae omnia si octies multiplicentur, fient dies 87, horae 2.472.
\lnr{21}Qui quidem dies non consummant menses tres Lunares.
\lnr{21}Ideo in octo
annis non possunt intercalari menses tres: quod ex enneadecaeteride
probatur.
\lnr{23}Nam in octo enneadecaeteridibus fiunt intercalationes
quinquaginta sex: in \rnum{xix} autem Octaeteridibus fiunt quinquaginta
septem.
\lnr{25}Unus igitur mensis abundat post annos 152; non autem, ut
voluit Eudoxus, post annos 160.
\lnr{26}Et cur illi displicuit Enneadecaeteris
Metonis, non possum conminisci: nisi quia semper sincerum volumus
vas incrustare, et potius reprehendere, quam discere.
\lnr{28}Octaeterida autem
Eudoxi Dositheus Archimedis familiaris correxit, et iterum cum
parapegmate castigatiore edidit.
\lnr{30}Unde quoties veteres \textgreek{ἐν ἐπισημασίαις τῶν φαινομένων[?]}
Dositheum testem producunt, scito esse ex Eudoxi
quidem Octaeteride et parapegmate, sed a Disitheo correctis.
\lnr{32}Parapegma
cum Octaeteride Lucanus vocat fastos, vel, ut ipse loquitur, fastus.
\lnr{34}\emph{Nec meus Eudoxi vinctur fastibus annus.}
\lnr{34}Geminus postquam ex
Dosithei parapegmate quaedam produxit, statim subiicit etiam testimonium
ex Eudoxo.
\lnr{36}Sed utrumque est Eudoxi; prius quidem ex posteriore
editione Dosithei, posterius autem ex priore Eudoxi.
\lnr{37}Quare
multi veterum octaeterida Eudoxi Dositheo attribuunt, teste Censorino.

% 72
% {PDF page nr}{source page nr}{line nr}
\plnr{155}{72}{1}Illustravit etiam eandem Octaeterida commentario et expositione
Eratosthenes Cyrenaeus.
\lnr{2}Porro de Octaeteirde ita legitur apud
Censorinum: \emph{Hunc quoque circuitum vere annum magnum esse pleraque
Graecia existimavit, quod ex annis vertentibus solidis constaret,
ut proprie in anno magno fieri par est.}
\lnr{5}\emph{Nam dies sunt solidi uno minus
centum, annique vertentes solidi octo.}
\lnr{6}Haec a librariis mutilata
ita sunt supplenda: \emph{Nam dies sunt solidi ducenti noningenti viginti
duo, menses uno minus centum, et cetera.}
\lnr{8}Plinius vero de Octaeteride
intelligit, cum scribit libro
 \rnum{xviii}, \rnum{xxv} cap. \emph{Indicandum est et
illud, tempestates ipsas ardores suos habere quadrinis annis, et easdem
non magna differentia reverti ratione solis: octonis vero augeri easdem,
centesima revoluente se Luna, et cetera.}
\lnr{12}Hoc enim ex parapegmatis Eudoxi
et Dosithei hausit, quae quidem erant edita una cum octaeteride
Eudoxea.

\subsection{De Anno Magno Metonis, sive Enneadecaeteride}

\lnr{15}Quis status anni Lunaris Atheniensium esse potuerit, cum ad
Canonas Octaeteridis describeretur, ex comparatione utriusque
octaeteridis Tetraetericae et Lunaris Harpaleae, scire potuisti.
\lnr{18}Quamuis enim initia et tempus institutae Octaeteridis 
 \textgreek{πρυτανείας[?]}
Harpaleae ignoramus, tamen in quadraginta annis magnam turbationem
noviluniorum necessario confecutam esse, potes ex eadem
comparatione colligere.
\lnr{21}Itaque Aristophanes Olympiade 88, Amynia
praetore, Nephelas docens inducit Lunam cum Ahtenionsibus[?] expostulantem,
quod menses ad Lunam non describerent, sed \textgreek{ανω τε και
κάτω κυδοιδοπᾷν[?]}
ait.
\lnr{24}Nam putantes se solennes cultus et ferias ex Lunae
et Octaeteridis observatione obire, eas alieno tempore anni imprudenter
celebrabant.
\lnr{26}Quod plane anomaliae octaeteridis congruit:
donec Metonis anno succedente ea penitus desita est.
\lnr{27}Theophrastus
\textgreek{περὶ σημεῖων ὑδάτῶν καὶ πνευμάτων.}
% Title: περι σημειων υδατων και πνευματων [και χειμωνων και ευδιων] 
\lnr{28}\textgreek{διὸ καὶ ἀγαθοὶ γεγένηται  κατὰ τόπους τινὰς
ἀστρονόμοι ἔνιοι, οἷον Ματρικέτας ἐν Μηθύμνῃ ἀπὸ τοῦ Λεπετύμνου, καὶ Κλεόστρατος
ἐν Τενέδῳ ἀπὸ τῆς Ιδης, καὶ Φαεινὸς Αθήνῃσιν ἀπὸ τοῦ Λυκάμβη τοῦ τὰ
 περὶ τὰς τροπὰς
συνεῖδε.}
\lnr{31}\textgreek{παῤ οὗ Μέτων ἀκούσας τόν τοῦ ἑνὸς δέοντα ἔικοσιν ἐνιαυτὸν
 συνὲταξει.
ἦν δὲ ὁ μὲν Φαεινὸς μέτοικος Αθήνῃσιν ὁ δὲ Μέτων Αθηναῖος. [?]}.
% [καὶ ἄλλοι δὲ τὸν τρόπον τοῦτον ἠστρολόγησαν.]
% Theophrastus
% "De signis"
% Paragraph 4
% "Thus in some parts have been found good astronomers: for instance, Matriketas
% at Methymna observed the solstices from Mount Lepetymnos, Cleostratus in
% Tenedos from Mount Ida, Phaeinos at Athens from Mount Lycabettus: Meton, who
% made the cycle of nineteen years, was the pupil of the last-named. Phaeinos
% was a resident alien at Athens, while Meton was an Athenian.
% [Others also have made astronomical observations in like manner.]"
\lnr{32}Meton igitur Pausaniae filius, ut tunc captus erat Graecorum,
 insignis Mathematicus
floruit ineunte bello Peloponnesiaco, vir non solum peritia motuum
coelestium, sed et aquiliciis, et librationibus nobilis.
\lnr{35}Itaque et
fontes induxit Athenis, ut auctor est
 Phrynichus Comicus \textgreek{μονοτρόπῳ[?]}.
\begin{quote}
\textgreek{Τίς ἐστιν ὁ μετὰ ταῦτα ταύτης σροντιῶν;[?]}\\
\textgreek{Μέτων ὁ Λευκονοιεὺς, ὁ τὰς κρήνας ἄγων.[?]}
\end{quote}
% Phrynichus Comicus, comic poet, ca 430 BCE
% Fragments of his works survive.
% μονοτρόπῳ (Monotropos; The Solitary) is a play by him exhibited in 414 BCE.
% All fragments are collected in Theodor Kock: Comicorum atticorum fragmenta
% (Teubner, 1880). This fragment on page 376
% https://archive.org/stream/comicorumatticor01kockuoft#page/376/mode/1up
% "A. τίς δ᾽ ἔδτιν ὁ μετὰ ταῦτα φροντίζων;
%  B. Μέτων, ὁ Λευκονοιεύς.
%  A. οἶδ᾽, ὁ τὰς κρήνας ἄγων."
% "Leuconoea erat pagus Leontidis tribus, unde oriundus Meton (Aristoph.
% Av. 992 cum interpr. Aelian. V. H. 10, 7, ubu Λευκονοιεύς ex
% Salmasii coniectura pro Λάκων)."

% 73
% {PDF page nr}{source page nr}{line nr}
\plnr{156}{73}{2}Quare huc alludens Manilius noster in Apotelesmatis,
 sub Aquario
Aquilices et Astronomos ait nasci:
\begin{quote}
  \emph{Ille quoque, inflexa fontem qui proiicit urna,\\
  Cognatas tribuit iuvenilis Aquarius artes\\
  Cernere sub terris; undas inducere tectis. et cetera.\\
  Quippe etiam mundi faciem, sedesque movebit\\
  Sidereas, coelumque novum versabit in orbem.}
\end{quote}
% Astronomica (attibuted to Marcus Manilius), written sometime during
% the reign of either Caesar Augustus or Tiberius.
% Only a few copies-of-copies manuscripts survived, and the exact contents
% is, and was, contested.
% Scaliger published two critically edited editions of this work
% (in 1579 and 1599).
% This quote by Scaliger diverges from current publications. 
% Liber IV
% lines 259-261 [et cetera]
% proiicit -> proicit; iuvenilis -> iuvenalis
% sub terris; undas inducere tectis -> sub terris undat, inducere terris.
% and lines 267-268
% coelumque -> caelumque
\lnr{9}Non est locus nobilior in toto Manilio, quamuis olim eum non satis
capiebamus.
\lnr{10}Veteres, quae fuit eorum hac in re imperitia, putabant
omnium motuum coelestium et mundi ipsius integram conversionem
fieri, quando Sol et Luna in idem tempus recurrebant,
in quo antea deprehendebantur, et quamdiu Octaeteridis
fides suspecta non fuit, eam esse mensuram
 \textgreek{ἀποκαταστάσεως τοῦ παντὸς[?]}
non solum vulgis, sed et docti credebant.
\lnr{15}Itaque Festus Avienus ex
Graecorum libris dixit:
\begin{quote}
  \emph{Non ego nunc longo redeuntia sidera motu\\
  In priscas memorem sedes. ---}
\end{quote}
% Avienus - Aratea
% See also page 68, where the same lines are quoted (with slight differences)
% Lines 1363-1369
\lnr{19}Innuit, ut vides, de ortu et occasu
 \textgreek{τῶν μορφώσεων[?]} annuo.
\lnr{19}Nam si qua
est varietas, eius \textgreek{ἀποκαταστασιν[?]},
 quantocunque tempore illa reditura
sit; omittit dicere.
\lnr{21}Subiicit:
\begin{quote}
  \emph{--- Habet ist a priorum\\
  Pagina, et incerta rerum ratione feruntur.}
\end{quote}
% Quote above continues: Lines 1369-1370
Diversa de istis scriptorum et Astronomorum iudicia, et libros exstare
dicit:
\begin{quote}
  \emph{Nam quae Solem hiberna novem putat athere volui,\\
  Ut spatium Lunae redeat, vetus Harpalus, ipsam\\
  Ocius in sedes, momentaque prisca reducit.}
\end{quote}
% Quote above continues: Lines 1371-1373
Putat \textgreek{ἀποκατάστασιν[?]} inerrantium fieri, ita ut sidera,
 ut ipse loquitur, in
priscas sedes longo motu redeant.
\lnr{30}Putat, inquam, hoc accidere, quandocunque
spatium Lunae redit.
\lnr{31}Ipse interpretatur mentem suam.
\lnr{32}Quando, inquit, Luna in sedes et momenta prisca reducitur.
\lnr{32}Quod
intervallum, inquit, Harpalus annorum novem esse decrevit, sed
male, cum ocior sit iusto ista periodus.
\lnr{34}Aperte igitur censet, conversionem
coeli universalem fieri, quando neomenia redit in eandem
diem, et horam, in qua antea fuit.
\lnr{36}Subiicit Festus Avienus, hanc periodum
ob brevitatem fallere, ideoque ei decennium additum a
Metone.

% 74
% {PDF page nr}{source page nr}{line nr}
\plnr{157}{74}{2}Et veram \textgreek{ἀποκατάστασιν τοῦ φαινομένων[?]}
 intra \textgreek{ἐννεαδεκαετηρίδα[Greek]}
fieri.
\lnr{3}De hallucinatione Festi super nomine Enneaeteridos, supra
dictum est.
\lnr{4}Aratus quoque volens ostendere omnium inerrantium
\textgreek{ἐποχὴν[?]} Solem ipsum esse, absurde quidem, sed tamen eius
rei causam confert ad cyclum Metonis.
\lnr{6}Quo intervallo, \textgreek{δύσεων καὶ ἀνατολῶν τοῦ φαινομένων[?]}
fiat restitutio, orbis, et conversio quaedam
mundi universalis.
\lnr{8}Ita enim canit elegantius, quam verius:
\begin{quote}
  \textgreek{Γινώσκεις τάδε καὶ σύ. Τὰ γὰρ συναείδεται ἤδη[?]}\\
  \textgreek{Εννεακαίδεκα κύκλα φαεινοῦ ἠελίοιο[?]}\\
\end{quote}
% Aratus: Phaenomena
% Lines 752-753
% "You too know all these (for by now the nineteen cycles of the shining
% sun are all celebrated by all)"
\lnr{11}Quem locum summum virum Theonem ex toto affecutum non
esse mirum non est, cum Hipparchus, et post eum Astronomiae
Apollo Ptolemaeus, quantitatem anni Tropici ex neomeniarum
restitutione collegerint, quam restitutionem Hipparchus recte
censet post annos 304 statim fieri.
\lnr{15}Sed de his libro quarto amplius.
\lnr{16}Apertius vero Diodorus Siculus docet illos veteres,
 \textgreek{τὰ φαινόμενα ἀποκαθίστασθαι[?]}
illis novemdecim annis vertentibus, putasse.
\lnr{17}Libro
enim duodecimo de illis novemdecim annis loquens ita scribit;
\lnr{19}\textgreek{ἐν δὲ τοῖς ἐιρημένοις ἔτεσι τὰ ἄστρα τὴν ἀποκατάστασιν ποιεῖται,
 καὶ καθάπερ
ἐνιαυτοῦ τινὸς μεγάλου τὸν ἀνακυκλισμὸν λαμβάνει.}
\lnr{20}\textgreek{διὸ καί τινες αὐτὸν Μέτωνος
ἐνιαυτὸν ὀνομάζουσι.}
\lnr{21}\textgreek{δοκεῖ δὲ ὁ ἀνὴρ οὗτος ἐν τῇ προῤῥήσει καὶ προγραφῇ
ταύτῃ θαυμαστῶς ἐπιτετευχέναι.}
\lnr{22}\textgreek{τὰ γὰρ ἄστρα τήν τε κίνησιν, καὶ τὰς ἐπισημασίας
ποιεῖται συμφώνως τῇ γραφῇ.}
\lnr{23}\textgreek{διὸ μέχρι τῶν καθ´ ἡμᾶς χρόνων οἱ πλεῖστοι
τῶν ἑλλένων χρώμενοι τῇ ἐννεα[και]δεκαετηρίδι οὐ διαψεύδονται τῆς ἀληθείας.}
% Diodorus Siculus - Bibliotheca Historica
% Διόδωρος Σικελιώτης - Ἱστορικὴ Βιβλιοθήκη
% Book 12
% [12,36] (second half)
% ἐν δὲ τοῖς εἰρημένοις ἔτεσι τὰ ἄστρα τὴν ἀποκατάστασιν ποιεῖται
% καὶ καθάπερ
% ἐνιαυτοῦ τινος μεγάλου τὸν ἀνακυκλισμὸν λαμβάνει.
% διὸ καί τινες αὐτὸν Μέτωνος
% ἐνιαυτὸν ὀνομάζουσι.
% δοκεῖ δὲ ὁ ἀνὴρ οὗτος ἐν τῇ προῤῥήσει καὶ προγραφῇ
% ταύτῃ θαυμαστῶς ἐπιτετευχέναι.
% τὰ γὰρ ἄστρα τήν τε κίνησιν καὶ τὰς ἐπισημασίας
% ποιεῖται συμφώνως τῇ γραφῇ.
% διὸ μέχρι τῶν καθ´ ἡμᾶς χρόνων οἱ πλεῖστοι
% τῶν Ἑλλήνων χρώμενοι τῇ ἐννεακαιδεκαετηρίδι οὐ διαψεύδονται τῆς ἀληθείας.
\lnr{25}Quid melius potuit versus Arateos interpretari?
\lnr{25}Sed fallitur Diodorus.
\lnr{26}Nam \textgreek{ἐννεαδεκαετηρὶς[?]}, quae illius temporibus obtinebat,
 erat
Calippica, non autem Metonica: cum Metonica plus quam
quinque diebus, Calippica uno fere die illo faeculo iam antevertissit,
quo haec scribebat Diodorus.
\lnr{29}Idem scriptor libro secundo
eadem repetit, loquens de enneadecaeteride gentium Hyperborearum:
\textgreek{λέγεται δὲ καὶ τὸν θεὸν[?]} (Apollinem)
 \textgreek{δἰ ἐτῶν ἐννεακαίδεκα καταντᾷν
εἰσ την νῆσον, ἐν σεσ[?] καὶ αἱ τῶν ἄστρων ἀποκαταστάσεις ἐπὶ τελος ἄγονται.[?]}
\lnr{32}\textgreek{καὶ
διὰ τοῦτο τὸν ἐννεακαιδεκαετῆ χρόνον ὑπὸ τῶν ἑλλήνων μέγαν ἐνιαυτὸν
 ὀνομάζεσθαι[?]}
\lnr{34}Huius meminit et Aelianus libro decimo.
\lnr{34}Nunc quid
velit Manilius per te ipse potes intelligere.
\lnr{35}Ait Metonem coelum
versasse in novum orbem, hoc est, conversionem mundi
 \textgreek{καὶ ἀποκαταστασιν[?]}
novo orbe et nova periodo \textgreek{τὴς ἐννεαδεκαετηρίδος[?]} definivisse, cum
orbis vetus Harpali mendosus, et fallax tempore deprehensus sit.
\lnr{39}Hoc est, quod dicit versare coelum in novum orbem.
\lnr{39}Et novum
orbem periodum Metonicam intelligit, comparatione veteris, quam
Octaeterida Harpali esse ostendit Festus Avienus.

% 75
% {PDF page nr}{source page nr}{line nr}
\plnr{158}{75}{2}Enneadecaeteris
igitur Metonis celeberrima multis quidem nominibus commendatur:
sed eam parum hactenus notam fuisse, argumento sunt ii,
qui eandem penitus cum cyclo Paschali, et nostro vulgari, quem
numerum aureum vocamus, difiniunt.
\lnr{6}Tantum enim septem embolismos,
et novemdecim annos cum nostro numero aureo communes
habet, in reliquis immane quantum differt.
\lnr{8}Nam neque
idem situs embolismorum, neque eadem anni Solaris quantitas:
cum Cyclus Metonis sit absolute dierum 6940, discedens
ab iusta periodo horis 7. 26.' 56.'' 40.'''
% à ->ab
\lnr{11}Unde in annis 76 moratur
Lunae curriculum die uno, horis 5. 47.' 46.'' 40.'''
\lnr{12}In annis
denique 304 Luna antevertit primam epocham Metonicam diebus
solidis quinque.
\lnr{14}Annos autem embolimaeos septem aut eorum
situm, scire non possumus priusquam epocham cycli ipsius
et caput indagemus.
\lnr{16}Veteres Graeci ab bruma, ut cognoscimus
% à ->ab
ex octaeteridibus Cleostrati et Harpali, tempora sua ordiebantur,
et eos fecuti Romani, quod facilius a decrementis umbrae
horas observarent, brumae confecto die, quam ab incrementis,
solstitio.
\lnr{20}Quare omnia horologia Graecorum semper
ad rationes brumae referebantur, ut et Romana: donec primus
omnium Meton noster ab capite anni populari, aut potius ab eius
% à ->ab
epocha, horologium describere instituit, hoc est ab solstitio.
% à ->ab
\lnr{23}Unde
ipsum organon \textgreek{ἡλιοτρόπιον[?]} vocarunt Graeci,
 a solstitii observatione.
\lnr{25}Quod instrumentum nobilissimum, atque priscorum hominum
observationes longe subtilitate vincens, ipse in Comitio
Athenarum dicavit, ut testis est priscus scriptor Philochorus apud
interpretem Aristophanis \textgreek{ὀρνίθων[?]},
 instar \textgreek{ἡλιοτροπίου[?]} Pherecydis, quod
in Scyro insula patria sua auctor dedicavit.
\lnr{29}Itaque videtur non solum
a solstitio caput Enneadecaeteridis suae deduxisse, sed etiam
ab eo tempore, quo Heliotropium suum dicavit.
\lnr{31}Idem
Scholiastes Aristophanis affentitur quidem Philochoro de Heliotropii
positu: sed de tempore refragatur.
\lnr{33}Philochorus enim
censebat Metonem ante Pythodorum Archontem Heliotropium
posuisse.
\lnr{35}Ipse post Pythodori magistratum aut saltem in magistratu
ipso, non autem ante magistratum, positum contendit.
\lnr{36}Verba
Grammatici de Metone haec sunt: \textgreek{ὁ δὲ φιλόχορος ὀν Κολωνῷ μὲν
αὐτὸν οὐδὲν λέγει θεῖναι, φευδῶς δὲ πρὸ Πυθοδώρου ἡλιοτρόπιον ἐν τῇ νηῦ
λεγομένῃ ἐκκλησίᾳ, πρὸς τῶ τείχει τῷ ἐν τῇ πνυκὶ[?]}.
% Possibly: fragment of Phrynichus - Monotropos
% "Fr. 22 PCG, Σ Αν. 997"
\lnr{39}Lege: \textgreek{οὐδὲν λέγει
θεῖναι[?]}.
\lnr{40}\textgreek{ἐπ᾽ Αψεύδοις δὲ τοῦ πρὸ Πυθοδώρου.[?]}
% Same fragment
\lnr{40}Sane verum est adhuc
sub magistratu Apseudis illud Heliotropium dedicasse, et solstitium
observasse 27 Iunii, diebus 36 ante neomeniam sequentis Hecatombaeonis
Tetraeterici.

% 76
% {PDF page nr}{source page nr}{line nr}
\plnr{159}{76}{3}Itaque adhuc erat in magistratu Apseudes:
cui in sequenti anno successit Pythodorus.
\lnr{4}Parum igitur abest,
quin et a Solstitio cyclum suum incipisse, et circa tempora Pythodori
Heliotropium statuisse credamus, optimi scriptoris auctoritate
moti.
\lnr{7}Sed de Solstitio cur dubitem, cum auctorem locupletem
habeam Festum Avienum?
\lnr{8}Qui post eos versiculos a nobis paulo
ante adductos subiicit, loquens de Harpalo:
\begin{quote}
  \lnr{8}\emph{Illius ad numeros prolixa decennia rursum}\\
  \emph{Adiecisse Meton Cecropea dicitur arte,}\\
  \emph{Inseditque animis. Tenuit rem Graecia solers [sic]}\\
  \emph{Protinus, et longos inventam misit in annos.}\\
  \emph{Sed primaeva Meton exordia sumpsit ab anno,}\\
  \emph{Torreret rutilo cum Phoebus sidere Cancrum:}\\
  \emph{Cingula cum veheret pelagus procul Orionis,}\\
  \emph{Et cum caeruleo flagraret Sirius astro.}\\
\end{quote}
% Rufius Festus Avienus: Aratea, lines 1369-1376
% Speaking of Harpalus
% [Edition Alfred Breysig, Lipsiae 1882]
% https://archive.org/details/rufifestiavienia00avieuoft
% "Illius ad numeros prolixa decennia/decentia rursum
% adiecisse Meton Cecropea dicitur arte
% inseuitque/inseditque animis: tenuit rem Graecia sollers
% protinus et longos inuentum misit in annos.
% et/sed primaeua Meton exordia sumpsit ab anno,
% torreret rutilo cum Phoebus sidere cancrum,
% cingula cum ueheret pelagus procul Orionis
% et cum caeruleo flagraret Sirius astro."
% (slashed words are reported by Breysig to be different in various sources)
\lnr{18}A solstitio igitur duxit citimum novilunium.
\lnr{18}Remotissimum vero
statuit ad aestus maximos Caniculae.
\lnr{19}Iam igitur constat de exordio periodi
Metonicae.
\lnr{20}De tempore, hoc est anno observati a Metone solstitii,
item de tempore et epocha ipsius Solstitii, habemus plene apud
Ptolemaeum, libro \rnum{iii},
 qui ait diserte Solstitium a Metone et Euctemone
observatum anno Nabonassari 316, Phamenoth \rnum{xxi}, mane.
\lnr{24}Tempus congruit \rnum{xxvii} Iunii, cyclo Lunae \rnum{vii},
 cyclo Solis \rnum{xxvi}, feria
prima, anno quarto Olympiadis 86 definente, praefecto Athenis
Apseude, T. Verginio, Proculo Geganio Macerino \textsc{coss}.
\lnr{26}Qui erat
tertius annus periodi Atticae.
\lnr{27}Scirrhophorion \rnum{iii} Iulii.
\lnr{27}Ergo Neomenia
Hecatombaeonis Metonici \textgreek{σκιῤῥοφοριωνος τρίτῃ ἐπὶ δέκα}.
\lnr{28}Diodorus
libro duodecimo:
% Diodorus Siculus: Bibliotheca historica (Βιβλιοθήκη ἱστορική),
% Book 12, chapter 36, section 2:
% How Meton of Athens was the first to expound the nineteen-year cycle.
 \textgreek{ἐν δὲ ταῖς Αθήναις Μέτων ὁ Παυσανίου μὲν υἱός,
δεδοξασμένος δὲ ἐν ἀστρολογίᾳ ἐξέθηκε τὴν ὀνομαζομένην ἐννεακαιδεκαετηρίδα,
τὴν ἀρχὴν ποιησάμενος ἀπὸ μηνὸς ἐν Ἀθήναις σκιροφοριῶνος τρισκαιδεκάτης.}
% Translation by C. H. Oldfather (1946)
% ISBN 978-0-674-99413-3
% http://data.perseus.org/citations/urn:cts:greekLit:tlg0060.tlg001.perseus-eng1:12.36.2 
% "In Athens Meton, the son of Pausanias, who had won fame for
% his study of the stars, revealed to the public his nineteen-year cycle,
% as it is called, the beginning of which he fixed on the thirteenth day of
% the Athenian month of Scirophorion."
% [Continues:] In this number of years the stars
% accomplish their return to the same place in the heavens and conclude,
% as it were, the circuit of what may be called a Great Year;
% consequently it is called by some the Year of Meton."
\lnr{32}Proinde \textgreek{θαργηλιῶνος πέμπτῃ φθίνοντος[?]}
 observatum Solstitium
a Metone.
\lnr{33}Et proximo \textgreek{πρυτανείας[?]} Hecatombaeone Pythodorus
inivit magistratum, novem mensibus ante initia belli Peloponesiaci.
\lnr{35}Haec igitur est Epocha cycli Metonici, non autem \rnum{ix} Iulii,
ut solebant vetustiores: neque octava pars Cancri, ut Cleostratus.
\lnr{37}Quare mirari satis non possum, cur Columella dixerit, se, auctore
Metone, solstitium in octava parte Cancri, sicut alia \textgreek{κέντρα[?]}
in octavis partibus signorum suorum, statuere, cum res ipsa eum
satis refellat.

% 77
% {PDF page nr}{source page nr}{line nr}
\plnr{160}{77}{1}Cur enim potius Columellae de Metone, quam
Metoni ipsi credam?
\lnr{2}De modo Enneadecaeteridis Metonicae
scribit Censorinus:
% Censorinus: De Die Natali Liber, chapter 18
 \emph{Praeterea sunt anni magni complures: ut Metonicus,
quem Meton Atheniensis ex annis undeviginti constituit.}
\lnr{4}\emph{Eoque
Enneadecaeteris appellatur: et intercalatur septies: in eoque
anno sunt dierum sex millia, et quadringenti quadraginta.}
% "Praeterea sunt anni magni conplures, ut Metonicus,
% quem Meton Atheniensis ex annis undeviginti constituit,
% eoque enneadecaeteris appellatur et intercalatur septies, inque eo
% anno sunt dierum VI milia et DCCCXL"
% 'sex millia, et quadringenti quadraginta' = 6440
% 'VI milia et DCCCXL' = 6000 + 500 + 300 + 40 = 6840
\lnr{6}Legendum:
\emph{in eoque anno sunt dierum sex millia et noningenti quadraginta.}
% noningenti => nongenti = 900
% sex millia et noningenti quadraginta = 6940
\lnr{8}Nam ex notis vulgaribus fluxit error, dierum sex millia,
 et \rnum{ccccxl}.
\lnr{9}Deest enim \rnum{d}
% Preferably I (Unicode U+2160) plus reversed C (Unicode U+2183): "ⅠↃ"
% But most fonts don't support these.
% Currently (feb 2017) supported on Mac OS X by:
% Baskerville (regular, italic, semibold, semibold italic, bold, bold italic)
% Big Casion Medium
% Courier (regular, oblique, bold, bold oblique)
% Geneva
% Helvetica (regular, oblique, bold, bold oblique)
% Helvetica Neue (regular, italic, bold, bold italic)
% Lucida Grande (regular, bold)
% Trattello
% Notably *not* supported by Society of Biblical Literature (SBL) fonts.
 nota quingentorum.
\lnr{9}Cum 5940 dies comprehenderet
Enneadecaeteris Metonica, ea null modo potuit
congruere cum vera enneadecaeteride Lunari, ut infra demonstrabitur.
\lnr{12}Servata epocha in \rnum{xxvii} Iunii, facile embolismorum
situs et tempora deprehendemus.
\lnr{13}Nulla enim Neomenia Hecatombaeonis
Solstitium antevertebat.
\lnr{14}Quocirca secundus, quintus,
octavus, decimus, tertiusdecimus, sextusdecimus, decimus octavus
anni erant embolimaei; contra quam consent docti homines nostri
temporis.
\lnr{17}Cum autem periodus ipsa 6940 diebus praecise explicaretur,
in illis erant anni Lunares \rnum{xix},
 menses \textgreek{τριακονθήμεροι[?]} septem,
dies quatuor, scrupula nulla.
\lnr{19}Quare propter illos quatuor dies abundantes,
quatuor quoque anni erant \textgreek{ὑπερήμεροι[?]}, dierum scilicet
355, ut postea videbimus.
\lnr{21}Anni autem Lunaris modus, secundum
Metonem, est dierum 354~\myfrac{4}{19}, aut \myfrac{16}{76}.
\lnr{22}Neomenia prima Hecatombaeonis
Metonici fuit Iulii \rnum{xv}.
\lnr{23}Quare cyclus Metonis constat
non ex enneaeteride et dacade, ut voluit Festus Avienus, sed ex Octaeteride
et Hendecaeteride.
\lnr{25}Nam nullae aliae partes sunt, enneadecaeteridis,
quae propius absint a modulo anni Solaris: nec quaemelius
in se cohaerant.
\lnr{27}Quod enim Octaeteridi superest supra rationes
Solis, id deest Hendecaeteridi, et contra.
\lnr{28}Sed cum Meto videret
accurata observatione septimam diem Hecatombaeonis vicesimi
Tetraeterici semper in novilunium concurrere; (verbi gratia, incipiat
primus Hecatombaeon \rnum{ix}. Iulii, ut in principio periodi Atticae;
vicesimus incipiet in Kal. Augusti, et in \rnum{vii} mensis erit novilunium;
id quod me tacente indicat laterculus mensium Tetraetericorum
antea a nobis propositus) cum igitur hoc videret Meto, animadvertit
in hoc intervallo contineri duas Octaeteridas Harpali.
\lnr{35}Id
quod et puero proclive: item dies 1092.
\lnr{36}Sed ii dies est triennium Harpaleum,
vel Cleostrateum, hoc est tres anni Lunares cum uno mense
intercalari pleno.
\lnr{38}Duae autem Octaeterides sive Harpaleae, sive Tetraetericae
sunt dies 5848.
\lnr{39}Quibus si adieceris 1092, confurget summa
dierum 6940.
\lnr{40}igitur iustam periodum confici posse ex duabus
Harpali Octaeteridibus et triennio Lunari existimavit.

% 78
% {PDF page nr}{source page nr}{line nr}
\plnr{161}{78}{1}Rursus in
duabus Octaeteridibus, sex sunt embolismi; in triennio unus.
\lnr{3}Ergo septem embolismi transigentur in ea periodo: et fient omnes
syzygiae 235: quia in duabus Octaeteridibus sunt 198, et in triennio
Lunari 37.
\lnr{5}Atque adeo decemnovem annis tota periodus explicabitur.
\lnr{6}Unde eam \textgreek{ἐννεαδεκαετηρίδα} vocavit.
\lnr{6}Si igitur omnes
menses 235 huius periodi essent \textgreek{τριακονθήμεροι[?]},
 et pleni, ii fierent
dies 7050.
\lnr{8}De quibus si detrahantur dies 6940, quantitas nempe
huius periodi, relinquentur menses cavi 110, qui debentur
huic periodo: et proinde reliqui 125 erunt pleni.
\lnr{10}Longe igitur
maior erit numerus plenorum, quam cavorum: neque erunt
alternis pleni et cavi, ut nec in Harpali Octaeteride erant alternis
pleni et cavi.
\lnr{13}Si igitur 235 in 110 distribuantur, habebimus syzygias
2. dies 4~\myfrac{1}{11}.
\lnr{14}Eae enim in 110 multiplicatae faciunt 235 syzygias
praecise: quas intelligimus omnes plenas.
\lnr{15}Itaque post duas
syzygias et dies 4~\myfrac{1}{11}, utendum erit \textgreek{ἐξαιρέσα[?]},
 ut pro 4 mensis, dicatur
quinta.
\lnr{17}Item eodem modo post quatuor syzygias, pro quintae
syzygiae octava dicetur nona.
\lnr{18}Et ita progrediendo erogabis omnes
\textgreek{ἐξαιρέσας[?]}, donec ultima syzygia sit cava, et pro eius tricesima,
dicatur prima Hecatombaeonis primi secundae periodi.
\lnr{20}Id
quod in conspectu tibi dedimus in duabus sequentibus Tabulis: in
quarum priore omnes syzygiae cum characteribus suis notatae sunt.
\lnr{23}Nam Cella, quae habet tres numeros, ea indicat mensem cavum.
\lnr{24}Puta in primo anno, tertio mense, in cella habes \myfrac{45}{4}.
\lnr{24}Duo
priores numeri significant pro 4 mensis, dicendum 5: vel, ut Graeci
loquuntur pro \textgreek{τετάρτη ἱσταμένου[?]}, \textgreek{πέμπτη ἱσταμένου[?]}.
\lnr{26}Ideo mensis
est cavus.
\lnr{27}Inferior autem numerus est feria, vel character neomeniae.
\lnr{28}Quaecunque autem neomeniae habent unum numerum, eae
sunt plenorum mensium.
\lnr{29}Reliqua facilia sunt.
\lnr{29}Ex quibus vides
\textgreek{ἐξαίρεσιν[?]} non fieri in uno die,
 sed prout coniugatio duarum syzygiarum
postulat.
\lnr{31}Post binas enim syzygias fit \textgreek{ἐξαίρεσις[?]}, donec numerus
ex \myfrac{1}{11} accrescens addat diem prioribus diebus quatuor.
\lnr{32}Quare
in omni undecima syzygia accrescit dies unus.
\lnr{33}Insigniter autem
fallitur Geminus, priscus et eruditus auctor, qui scribit Metonem
divisisse 6940 dies per 110 syzygias: et quia 110 in 6940
continentur sexagesies ter, propterea censet Metonem statim
post 63 dies \textgreek{ἐξαίρεσιν τῶν ἡμερῶν[?]} fecisse.
\lnr{37}Hoc enim ratio ipsa confutat.
\lnr{38}Nam 63 dies sunt syzygiae 2, et dies praeterea 3.
\lnr{38}Quae omnia
in 110 ducta producunt syzygias, 220, dies 330.
\lnr{39}Hoc est syzygias
undecim, quae cum 220 syzygiis compositae dant tantum 231
syzygias plenas.

% 79
% {PDF page nr}{source page nr}{line nr}

\plnr{162}{79}{1}% Tabula Characterismi neomeniarum enneadecaeteridis metonicae
\lnr{1}Itaque calculus desinet in ducentesima prima syzygia.
\lnr{2}Reliquae igitur
quatuor erunt continue plenae, et duae
primae sequentis anni erunt et ipsae plenae.
\lnr{5}Ita fient sex continue plenae syzygiae.
\lnr{5}Quod est absurdum.
\lnr{6}Porro adiunximus Tabellam
characterismi periodorum, qui characterismus
cum charactere neomeniae compositus dabit feriam
Neomeniae.

% 80
% {PDF page nr}{source page nr}{line nr}

\plnr{163}{80}{2}Exempli gratia.
\lnr{2}Volo scire feriam neomeniae Metagitnionis
Metonici in anno decimo periodi quintae.
\lnr{3}In Tabella
characterismi enneadecaeteridis, sive periodi quintae, habes
3.
\lnr{5}Ille character servit toti periodo, et cum 4 charactere Metagitnionis
anni decimi, abiectis septenariis, ubi opus erit, dat
feriam secundam neomeniae Metagitnionis.
\lnr{7}Adiecimus praeterea
Tabulam neomenarium Metonicarum in mensibus Iulianis,
ut citra[?] laborem eas invenire queas.
\lnr{9}Exemplum: Anno
Nabonassari 366, \textgreek{Θὼθ} $\overline{\kappa\varsigma}$,
 secundum Athenienses autem \textgreek{Φανοστράτου
ἄρχοντος, μηνὸς Ποσειδεῶνος} defecit Luna.
\lnr{11}Tempus
\rnum{xxii} Decembris, sequente \rnum{xxiii}, feria secunda, sequente
tertia, cyclo Solis \rnum{xix}, Lunae \rnum{xviii}, anno periodi Iulianae
4331.
\lnr{14}Erat annus Iphiti 394.
\lnr{14}Abiectis ex methodo perpetua annis
344, remanet annus quinquagesimus Metonis, id est duodecimus
tertiae periodi, cuius periodi tertiae character 4 cum 6 charactere
Posideonis anni \rnum{xii} compositus, abiectis 7, dat feriam 3.

% 81
% {PDF page nr}{source page nr}{line nr}

\plnr{164}{81}{1}In Tabula neomeniarum in annis Iulianis, Posideonis neomenia 
 \rnum{viii}
Decembris, feria secunda.
\lnr{2}Ergo quintadecima Posideonis, sequente
sextadecima, contigit Deliquium.
\lnr{3}Eodem anno tam Nabonassari,
quam Metonis, defecit idem
 sidus \textgreek{φαμηνὼθ[?]} $\overline{\kappa\delta}$,
 sequente $\overline{\kappa\epsilon}$, \textgreek{μηνὺς[?]}
\textgreek{Σσκιῤῥοφοριῶνος[?]}.
% Upper case Sigma *and* lower case sigma?
\lnr{5}Tempus \rnum{xviii} Iunii, sequente \rnum{xix}, feria quinta,
sequente sexta.
\lnr{6}Character 4 periodi tertiae cum charactere
2 Scirrhophorionis in anno 12, dat feriam \rnum{vi} characterem neomeniae
Scirrhophorionis: quae cum ex altera tabula sit in 4 Iunii, feria
quinta, cyclo Solis \rnum{xx}, Eclipsis contigit rursus \rnum{xv} mensis, 
 sequente
\rnum{xvi}.
\lnr{10}Denique anno sequente Nabonassari, et tertiodecimo
tertiae periodi Metonicae idem sidus defecit
 \textgreek{Θὼθ[?]} $\overline{\iota\varsigma}$, sequente
$\overline{\iota\zeta}$, secundum Athenienses
 \textgreek{Ευάνδρου ἄρχοντος, μηνὸς Ποσειδεῶνος προτέρου[?]}.
\lnr{13}Tempus \rnum{xii} Decembris, feria septima, sequente prima.
\lnr{13}Regularis
4 cum 4 compositus, abiecto septenario, dat feriam primam
neomeniae \textgreek{Ποσειδεῶνος προτέρου[?]}.
\lnr{15}Ergo Luna defecit \rnum{xiiii} mensis, sequente
\rnum{xv} et cetera.
\lnr{16}Rursus anno secundo primi Metonici cycli, ineuente[?]
bello Peloponnesiaco, diebus aestivis,
 \textgreek{νουμηνίᾳ κατὰ σελήνω[?]}, ut loquitur
Thucydides, defecit Sol.
\lnr{18}Haec eclipsis contigit anno Iphiti 346, periodi
Iulianae 4283, Augusti tertia die, feria quarta, cyclo Solis 27, Lunae
8, anno Nabonassari 317, Pachon \rnum{viii}, anno uno, et diebus 37, post
observatum a Metone Solstitium.
\lnr{21}In tabula neomeniarum habes
in secundo anno Metonis \textgreek{νουμηνίαν μεταγειτνιῶνος[?]}
 \rnum{iii} Augusti.
\lnr{22}Quod
convenit cum Thucidide, qui vocavit \textgreek{νουμηνίαν κατὰ σελήνην[?]}.
\lnr{23}Neque
de alia neomenia intelligit, quam Metonica.
\lnr{24}Quae etiam erat neomenia
Elul Iudaici anni 3330: cuius character 4.17.609, eadem feria, ut
vides.
\lnr{26}Rursus annus erat 42 periodi quintae Olympiacae, et ideo quartus
Atticae.
\lnr{27}Cuius Hecatombaeon caepit 29 Iulii.
\lnr{27}Ergo defecit Sol \textgreek{ἕκατομβαιῶνος
τῇ ἕκτῃ ἱσταμείου[?]}.
\lnr{28}Rursus Thucydides scribit de anno octavo
belli Peloponnesiaci: \textgreek{τοῦ δ᾽ ἐπιγινομείου θέροις ἐυθὺς,
 τοῦτε ἡλίου ἐλλιπές τι
... περὶ νουμηνίαν, καὶ ἀυτου μηνὸς ἱσταμείου ἔσειδε[?]}.
\lnr{30}Contigit ille defectus
Solaris Augusti \rnum{xvi}, feria quinta, anno periodi Iulianae 4290.
\lnr{31}Erat
annus Iphiteus 353, quadragesimus nonus periodi quintae Olympicae,
ideo undecimus Atticae.
\lnr{33}Hecatombaeon \rnum{iii} Augusti.
\lnr{33}Ergo \rnum{xiiii} contigit
novilunium.
\lnr{34}Rursus erat nonus annus Metonicus.
\lnr{34}Metagitnion \rnum{xvi} Augusti.
\lnr{35}Convenit ergo.
\lnr{35}Elul quoque Iudaicus 3338 non
adversatur.
\lnr{36}Fuit enim 4.21.580. feria quinta.
\lnr{36}Quod autem supra
diximus, quando in cella cavi mensis scriptum est 4.5, id significare
pro quarta mensis, dicendum esse, quinta mensis, noli putare ita a vulgo
usurpari solitum.
\lnr{39}Nam \textgreek{πολιτικῶς[?]} omnis mensis cavi
 \textgreek{δευτέρα[?]} dicebatur
\textgreek{τρίτη[?][?]}.
\lnr{40}Sed intelligendum est Metonem tantum dixisse quartam
pro quinta, methodi caussa, ut hoc modo non ad arbitrium, sed
ad progressum numerorum syzygias erogaret.

% 82
% {PDF page nr}{source page nr}{line nr}
\plnr{165}{82}{1}Statim autem post observationem
Solstitii Metonici hic magnus annus receptus non fuit.
\lnr{3}Nam quarto anno ab eius editione Aristophanes docuit
 \textgreek{νεφέλασ, ἄρχοντος
Αμυνίου[?]},
% Aristophanes: The Clouds (Νεφέλαι)
cum adhuc Athenienses Octaeterida suam mordicus
retinentes \textgreek{τὰς ἡμέρασ οὐκ ἦγον κατὰ τήν σελήνην,
 ἀλλὰ ἄνω τε καὶ κάτω εκυδοιδόπων[?]}.
\lnr{6}Sed non multo post receptum fuisse testatur Avienus.
\begin{quote}
-- Tenuit rem Graecia solers\\
Protinus, et longos inventam misit in annos,\\
Inseditque animis.
\end{quote}
Et celebritatem eius ignorare utique non possumus, Arato canente,
\begin{quote}
-- \textgreek{Τὰ γὰρ συναείδεται ἤδη[?]}\\
\textgreek{Εννεακαίδεκα κύκλα φαεινοῦ ἠελίοιο.[?]}
\end{quote}
Neomeniae vero Metonicae, et Calippicae aliquando neomenias Tetraeteridum
antevertunt mense integro, ut ex demosthene supra annotavimus:
aliquando paucioribus diebus.
\lnr{15}Praeterea omnes \textgreek{πρυτανεῖαι[?]}
apud Oratores Graecos sunt Metonicae, praeterquam si quae sunt veterum
legum et sanctionum.
\lnr{17}Nam illae sunt Octaetericae, cuiusmodi
quaedam extant \textgreek{ἐν τῷ κατὰ Τιμοκράτοις[?]}, quas non dubito esse Harpaleas.
\lnr{19}Deprehenso igitur apud Demosthenem anno Iphiti, qui incurrit in
annum propositum \textgreek{πρυτανείας[?]}, abiiece annos Iphiti 344 pro perptua
methodo, ut iam diximus.
\lnr{21}Residuum sunt anni metonis.
\lnr{21}Deinde
vide quotus annus ille sit a Solstitii Metonici
observatione: et confer illum in lineam
\textgreek{μεταπτώσεως[?]}, in numerum annorum scilicet collectorum
praecisum, si fieri potest: sin aliter in proxime
minorem.
\lnr{26}Columella, sive versus Dierum
ostendet, quot dies accreverunt epochae Metonis.
\lnr{27}Quos dies \textgreek{μεταπτώσεως[?]} adiice neomeniae priscae
Metonis.
\lnr{29}Habebis diem neomeniae Metonicae
in anno proposito.
% Table: LINEA μεταπτὼσεως Metonicae
% It looks like there are 76 Momenta to a Scru.
% and 60 Scru. to a Dies
% 19 Anni collecti give 18 3/4 Scru.
% Check: 304 Anni = 16*19 should give 16*18 3/4 = 300 Scru = 5 Dies. OK.
\begin{table}[htb]
 \centering
 %% Modify distance between rows
 \renewcommand{\arraystretch}{1.1}
 %% Modify separation between columns
 %\setlength{\tabcolsep}{2.0pt}
 %%% Liber II p82
%%
%%% Count out columns for fixed-width source font
% 000000011111111112222222222333333333344444444445555555555666666666677777777778
% 345678901234567890123456789012345678901234567890123456789012345678901234567890
%
%% Select a general font size (uncomment one from the list)
%\tiny
%\scriptsize
%\footnotesize
%\small
\normalsize
%% Center the whole table left-right
\centering
%% Modify separation between columns
%\setlength{\tabcolsep}{1.6pt}
%% Modify distance between rows
%\renewcommand{\arraystretch}{1.3}
%%
\begin{tabular}{@{}c c c c c@{} }
\toprule
\multicolumn{5}{c}{\Large\textsc{Linea \textgreek{μεταπτώσεως} Metonicae}}\\
\midrule
\multicolumn{1}{c}{Anni collecti} &
\multicolumn{1}{c}{Dies} &
\multicolumn{1}{c}{Scru. diur.} & % [Abbriv]
\multicolumn{1}{c}{Momenta}
\\
\midrule
  19 &  0 & 18 & 57 \\
  38 &  0 & 37 & 38 \\
  57 &  0 & 56 & 19 \\
  76 &  1 & 15 &  0 \\
  95 &  1 & 33 & 57 \\
 114 &  1 & 52 & 38 \\
 133 &  2 & 11 & 19 \\
 152 &  2 & 30 &  0 \\
 171 &  2 & 40 & 57 \\
 190 &  3 &  7 & 38 \\
 209 &  3 & 26 & 19 \\
 228 &  2 & 45 &  0 \\
 247 &  4 &  3 & 57 \\
 266 &  4 & 22 & 38 \\
 285 &  4 & 41 & 19 \\
 304 &  5 &  0 &  0 \\
\midrule
 608 & 10 &  0 &  0 \\
1216 & 20 &  0 &  0 \\
1824 & 30 &  0 &  0 \\
2432 & 40 &  0 &  0 \\
2736 & 45 &  0 &  0 \\
\bottomrule
\end{tabular}
%
\caption{Linea metaptoseos Metonicae}

 \caption{Linea \textgreek{μεταπτώσεως} metonicae}
 \label{tab:linea_metaptoseos_metonicae}
\end{table}
\lnr{30}Exemplum.
\lnr{30}Anno Christi
vulgari 1582, aestivis diebus iniit annus Iphiti
Olympiadicus 2358.
\lnr{32}Abiectis 344, resident anni
Iuliani a Solstitio Metonis, 2014.
\lnr{33}Proxime minor
numerus annorum collectorum in linea \textgreek{μεταπτώσεως[?]},
1824.
\lnr{35}Et e regione dies \textgreek{μεταπτώσεως[?]} 30.
\lnr{36}Deductis 1824 de 2014, remanet 190.
\lnr{36}E regione
eorum, in proposita linea \textgreek{μεταπτώσεως[?]}, sunt dies
3.7'.38.
\lnr{38}Qui cum triginta illis dant \textgreek{μεταπτώσεως[?]}
Metonicae dies 33.7'.38.
\lnr{39}Annus propositus Metonis,
est, ut vides, ultimus cycli, in quo neomenia
Hecatombaeonis ab Metone constituta fuit
Iulii \rnum{xxvii}.

% 83
% {PDF page nr}{source page nr}{line nr}
\plnr{166}{83}{1}Adiice ergo 33 \textgreek{μεταπτώσεως[?]}
 ad \rnum{xxvii} Iulii.
\lnr{1}Pervenitur ad
\rnum{xxix} Augusti, feria quarta.
\lnr{2}Tanta labes rationum Metonicarum facta
est ab initio huius cycli ad nostra tempora.
\lnr{3}Ex his vides, qua via
insistendum sit in neomeniis Metonis apud Demosthenem, et alios
priscos Rhetores investigandis.
\lnr{5}Annus enim Solis Metonicus, praeter
365 dies cum quadrante, habet \myfrac{5}{19}, ut autem Hipparchus dicit,
 \myfrac{1}{76}.
\lnr{6}Verba
Hipparchi apud Ptolemaeum, ex eo libro, quem \textgreek{περι ἐνιαυσίου χρόνου[?]}
scripserat: \textgreek{ὁ ἐνιαύσιος κατὰ οὖς[?] περὶ Μέτωνα,
 καὶ Εἰκτήμονα περιέχει ἡμέρας
τξέ δ´´, καὶ} $\overline{o\varsigma}$
 \textgreek{μιᾶς ἡμέρας[?]}, et cetera.
\lnr{9}Nam si in 76 annis Iulianis accrescit
unus dies Metoni, annus ergo Metonis habuerit \myfrac{1}{76} diei praeter
 365~\myfrac{1}{4}
diei.
\lnr{11}Huic consentanea scribit Gensorinus Fr. Pithoei, annum Metonis
Solarem fuisse dierum \rnum{ccclxv}, et praeterea dierum quinque
partis undeuicesimae.
\lnr{13}Item Geminus de periodo Metonis: \textgreek{ἐν δὲ
τῇ περιόδῳ ταύτῃ δοκοῦσιν οἱ μὲν μῆνες καλῶς εἰλῆφθαι, καὶ οἱ ἐμβόλιμοι
συμφώνως τοῖς φαινομείοις διατετάχθαι. ὁ δὲ ἐνιαύσιος χρόνος ἐκ πλειόνων
ἐτῶν παρατετηρημένος συμπεφώνηκεν, ὅτι ἐστὶν ἡμερῶν}
 $\overline{\tau\xi\epsilon}$,
 \textgreek{ἐννεακαιδεκάτων} $\overline\epsilon$ [?].
% Geminus of Rhodes: Introduction to Phaenomena
% http://www.astrologicon.org/geminus/geminus-introduction-to-phaenomena.html
% Γεμῖνος Ῥόδιος, Εἰσαγωγή εἰς τὰ Φαινόμενα 
% Section: Περὶ μηνῶν, last paragraph.
% "Εν δὲ
% τῇ περιόδῳ ταύτῃ δοκοῦσιν οἱ μὲν μῆνες καλῶς εἰλῆφθαι καὶ οἱ ἐμβόλιμοι
% συμφώνως τοῖς φαινομένοις διατετάχθαι, ὁ δὲ ἐνιαύσιος χρόνος [<οὐ>
%   σύμφωνος εἴληπται τοῖς φαινομένοις. Ὁ γὰρ ἐνιαύσιος χρόνος] ἐκ πλειόνων
% ἐτῶν παρατετηρημένος συμπεφώνηκεν ὅτι ἐστὶν ἡμερῶν
% [τξε δ, ὁ δὲ ἐκ τῆς ἐννεακαιδεκαετηρίδος συναγόμενος ἐνιαυτός ἐστιν ἡμερῶν]
% τξε ἐννεακαιδεκάτων ε."
% In German translation:
% "In diesem Cyklus sind dem Anscheine nach die Monate
% richtig genommen und die Schaltmonate mit den
% Himmelserscheinungen übereinstimmend angeordnet. Aber
% die Zeit des Jahres ist nicht mit den Himmelerscheinungen
% in Einklang angenommen. Wenn nämlich die Zeit des
% Jahres aus einer längeren Reihe von Jahren durch Be-
% obachtung festgestellt wird, so hat sich das übereinstim-
% mende Resultat ergeben, dass sie 365 1/4 Tage beträgt,
% während der aus dem 19jährigen Cyklus (durch Rechnung)
% abgeleitete Wert 365 5/19 Tage beträgt. [Dieser letztere
% Wert ist um 1/76 Tag Grösser als die erstere.]"
\lnr{17}Hac ratione in annis \rnum{xix} Metonis Solaribus intercalatur bisextum
quinquies: quatro, octavo, duodecimo, sextodecimo, decimonono.
\lnr{19}Et nihil relinquitur de ratiocinio scrupulario.
\lnr{19}Quae ut cyclus
Solis Iulianus propter quadriennia aequabilia, quater septem annorum
duntaxat est: sic Metonicus cyclus Solis propter inaequalitatem
intervallorum bisexti, novemdecies septem annorum est.
\lnr{22}Neque feriae
restituuntur ante exitum anni 133.
\lnr{23}Si quae de hoc magno anno Metonis
a nobis ignorata vel omissa sunt, ea studiosis colligenda relinquimus.
\lnr{25}Tamen ea; quae diximus, satis esse puto et ad doctrinam anni
Metonici explicandam, et ad eorum iudicia castiganda, qui a nostro
Lunari differre non putant.

%--
\subsection{De Cyclo Metonis Philippeo}
\lnr{30}Paucis ante excessum Philippi annis, qui contigit Olympiade
\rnum{cxi}, instituta est periodus in gratiam Philippi, cuius initium incidit
anno tertiodecimo ab eius morte, qui erat mortis Alexandri
eius filii primus: ita ut hoc initium neque ipse, neque eius filius viderint.
\lnr{34}Graecis miro consensu recipti sunt, ut docet Festus Avienus:
\begin{quote}
-- \emph{tenuit rem Graecia solers}\\
\emph{Protinus, et longos inventam misit in annos.}
\end{quote}
\lnr{38}Neque solum haec nova periodus in gratiam Philippi instituta, sed et
menses Lunares sua serie luxati.
\lnr{39}Pro Daesio enim nosus mensis Peritius
sumptus, adeo ut neomenia Lunaris Pertii conveniret in Daesium
Tetraetericum.

% 84
% {PDF page nr}{source page nr}{line nr}
\plnr{167}{84}{1}Huius mutationis mentionem
facit Rex ipse epistola ad Pelopennesios:
quae \textgreek{ἐν τῷ περὶ στεφοιύου [?]} extat.
% Table: Menses Metonici
% Table: Cyclus Metonis Philippeus
\lnr{3}\textgreek{συναντᾶτε[?]},
inquit, \textgreek{μετὰ τῶν ὅπλων εἰς τὴν Φωκίδα,
 ἔχοντες ἐπισιτισμόν ἡμερῶν τετταράκοντα,
τοῦ ἐνεστῶτος μηνὸς Λώου, ὡς ἡμεῖς ἄγομεν, ὡς δὲ Αθηναῖοι, Βοηδρομιῶνος,
ὡς δὲ Κορίνθιοι Πανέμου [?]}.
\lnr{6}Eo nomine posuimus menses Tetraetericos
Macedonicos comparatos cum Philippeis Lunaribus, sive
Metonicis, et Metonicos Macedonicos cum Metonicis Atticis.
\lnr{8}Subiecimus etiam filum cycli Metonis Philippei cum epochis neomeniarum
in mensibus Iulianis.

%--
\subsection{De Periodo Calippi Attica Solstitiali}
\lnr{11}Antiquissima fuit fere apud omnes nationes opinio de modo
anni Solaris, quod scilices tercentis sexaginta quinque diebus
cum quadrnte explicaretur, nequis forte putet nostrum annum
non solum a C. Iulio Caesare publicatum, sed etiam excogitatum esse.
\lnr{15}Is eam anni formam, quam omnes sciebant quidem, sed quam in civiles
usus admiserat factenus nemo, indixit.
\lnr{16}Ita ut usum eius edicto Caesaris,
scientiam autem antiquorum, qui eam conservarunt, monumentis
debeamus.
\lnr{18}Hinc nonnulli veterum prodire, Olympiadem illius
diei gratia institutam, qui quarto quoque anno vertente intercalabatur.
\lnr{20}Nam cum annum dierum tantum 360 haberent, singulorum bienniorum
exitu alternis \rnum{x} et \rnum{xi} dies intercalabant, ut annus ad principia
sua rediret.
\lnr{22}Itaque biennium Tetraetericis facris, quadriennium vero
Olympicis claudebant.
\lnr{23}Atque hoc tandiu obtinuit, donec Tetraeteridum
principia in novilunia conferrentur, quomodo libro proximo
demonstravimus.

% 85
% {PDF page nr}{source page nr}{line nr}
\plnr{168}{85}{1}Nam antiquiorem esse anni Solaris cognitionem
apud Graecos, quam putavit Strabo, facile convincit Romanorum
consuetudo, qui propter quadrantem anni prius interalationem
instituerunt, quam ullus Graeculus in Aegyptum peregrinaretur.



































% ==== End of text of Liber Secundus ===

% !TEX root = ../de-emendatione-temporum-1629.tex
% !TEX TS-program = xelatex
% !TEX encoding = UTF-8 Unicode
% this template is specifically designed to be typeset with XeLaTeX;
% it will not work with other engines, such as pdfLaTeX

%%% Count out columns for fixed-width source font
% 000000011111111112222222222333333333344444444445555555555666666666677777777778
% 345678901234567890123456789012345678901234567890123456789012345678901234567890

\setheaders{\shorttitle{} Liber III}{\shortauthor{}}
\chapter{De Anno Aequabili Maiore}
%
% 188
% {PDF page nr}{source page nr}{line nr}
\plnr{271}{188}{1}In Astronomicis \textgreek{ὁμαλαὶ κινήσεις} vocantur
eae, quibus ex Anomaliarum Canonibus
competentes \textgreek{προσθαφαιρέσεις} adhibitae non
sunt.
\lnr{4}Ideo Illis motibus aut deest semper,
aut superest aliquid.
\lnr{5}Sic in ratione anni
eum annum aequabilem vocamus, cui propter
commodiorem usum aut abest, aut superest aliquid.
\lnr{8}Exempli gratia: Persico anno
quadrans de Solis ratiociniis deest.
\lnr{9}At
Graeco supra Lunae ratiocinia dies fere sex supersunt.
\lnr{10}Utrique hoc contigit
propter et commodiorem mensium in triginta dies tributionem,
et aequabilem dierum descriptionem: qua quidem aequabilitate mensium
Graeci volebant Lunae motum assequi.
\lnr{13}Sed hoc erat ultra fines
iaculum expedire.
\lnr{14}Plus enim dierum ad eam rem assumebatur, quam
modus anni postulabat.
\lnr{15}Contraria ratione Orientis nationes, cum
eodem anno antea uterentur, atque ad vitandum taedium embolismorum
eum castigare vellent, primum illum a Solis propius, quam
a Lunae rationibus abesse animaduerterunt: quod videlicet populares
temporum errores magis in Luna, propter illius sideris celeritatem,
deprehenduntur, quam in Sole: deinde convenit inter
eos, ut ex modulo anni sui, qui 360 dierum tantum erat,
partem 72 decerperent, atque eam anno appenderent.
\lnr{22}Divisa itaque
anni quantitate in 72, excerptae sunt dies \rnum{v}, et in calcem anni
post 360 dies reiectae: quas \textgreek{ἐπαγομήνας}[?] Graeci vocant, Aegyptii
prisca linqua \textsc{nesi}: Persae et Arabes \textarabic{}[?]
 \textsc{musterakath}.
\lnr{26}Aethiopes corrupto Graeco vocabulo etiamnum hodie \textsc{pagomen}
appellant.
\lnr{27}Cum ita annum 365 dierum constituissent, non dubitarunt
eum Solarem appellare: cum tamen non ignorarent, illi anno ad
perfectionem quadrantem diurni temporis deesse.
%
% 189
% {PDF page nr}{source page nr}{line nr}
\plnr{272}{189}{1}Quo neglecto in
120 annis per mensem integrum recedit caput anni ab epocha Solari.
\lnr{3}Verbi gratia: Caput anni, quod hodie fuerit in Kal. Mai Iuliani, id
post 120 annos, incidet in Kal. Apr.
% End of line? 1598 edition p180: yes.
% Lots more space between "Apr." and "Quare" in that edition.
\lnr{4}Quare duodecim magnis mensibus
vertentibus disceditur a prima epocha, dies 360.
\lnr{5}Quod tempus
est annorum duodecies 120, hoc est, 1440.
\lnr{6}Post viginti annos receditur
dies 365.
\lnr{7}Et summa annorum fit 1460 anni aequabiles.
\lnr{7}Quo
intervallo caput anni huius totam ordinationis Iulianae feriem pervagatur.
\lnr{9}Id spatium annum magnum vocabant.
\lnr{9}Constat enim ex duodecim
magnis mensibus, qui singuli sunt annorum 120, vel decies duodecim,
et quinque magnis \textgreek{ἐπαγομήναις}[?], vel quinquies quater annis.
\lnr{12}Sed quod Censorinus huiusmodi magnum annum \textgreek{κυνικὸν}[?]
 vocari ait,
id verum est, si spectes neomeniam Thoth.
\lnr{13}Quoties enim neomenia
Thoth in ortum caniculae incidit, is dicitur recte
 \textgreek{κυνικὸς ἐνιαυτός}[?].
\lnr{15}Idque iterum ut accidat, non paucis annorum centuriis opus est.
\lnr{16}Sed quod illud intervallum, quo Thoth redit ad ortum Caniculae,
unde profectus erat initio, constet annis 1460, id vero perabsurdum
est.
\lnr{18}Nam quem Thoth Canicularem dicit fuisse, Ulpio et
Brutio Praesente \textsc{coss.} is post 1460 annos Canicularis esse non
poterit, nimirum anno Christi 1599.
\lnr{20}Qui erit annus aequabilis 1460
absolutus ab illo Thoth Caniculari Censorini.
\lnr{21}Ut enim Solstitia et
aequinoctia intra illud tempus antevertunt circiter dies \rnum{xi}, ita ortus
siderum tardius oriuntur: quia annus sidereus,
% Semicolon or comma? 1598 edition: comma.
 ut vulgus astronomorum
loquitur, tardior est anno Iuliano, Iulianus anno Tropico,
\lnr{24}Quare
si circa annum Christi 138 Caniculae ortus incidit in \rnum{xx} Iulii, anno
Christi 1598 incidet in \rnum{xxix} Iulii, si quidem Thebithio anni siderei
auctori credimus.
\lnr{27}Sed, inquies, nondum veteres illi \textgreek{τὴν ἀλήθειαν
ἐξηκριβώσανυτο}[?].
\lnr{28}Itaque magnus illorum annus, qui constabat ex 1460
Iulianis, erat potius magnus orbis anni Aegyptiaci aequabilis, in epocham
Iulianam restituti, quam Caniculae ad Thoth vagum redeuntis.
\lnr{31}Sed et Dio Cassius, hunc magnum annum in animo habens,
iniecta mentione intercalationis Bisexti Iuliani, pueriliter admodum
de ea re pronuntiavit.
\lnr{33}Putat in illo intervallo 1460 annorum,
unum diem adiiciendum esse, praeter illum ordinarium, quod \textsc{Bisextum}
dicitur.
\lnr{35}Quod tantum abest, ut verum sit, ut in totidem
annis Iulianis undecim potius dies excreverint, quam ut unus desit.
\lnr{37}Verba Dionis:
 \textgreek{τὴν μέντοι μίαν τὴν ἐκ τῶν τεταρτημορίων συμπληρουμένην
διὰ τεσσάρων καὶ αὐτὸς ἐτῶν ἐισήγαγεν, ὤστε μηδὲν ἔτι τὰς ὤρας αὐτῶν,
πλὴν ἐλαχίστου, παραλλάττειν.}[?]
\lnr{39}\textgreek{ἐν γοῦν χιλίοις καὶ τετρακοσίοις, καὶ ἑξήκοντα,
καὶ ἑνὶ ἔτει μιᾶς ἄλλης ἡμέρας ἐμβολίμου δέονται.}[?]
% Cassius Dio: Historiae Romanae XLIII.26.3
% [ἑνὸς μηνὸς ἀφεῖλεν, ἐνήρμοσε.]
% τὴν μέντοι μίαν τὴν ἐκ τῶν τεταρτημορίων συμπληρουμένην
% διὰ πέμπτων[!] καὶ αὐτὸς ἐτῶν ἐσήγαγεν[!] ὥστε μηδὲν ἔτι τὰς ὥρας αὐτῶν
% πλὴν ἐλαχίστου παραλλάττειν:
% ἐν γοῦν χιλίοις καὶ τετρακοσίοις καὶ ἑξήκοντα
% καὶ ἑνὶ ἔτει μιᾶς ἄλλης ἡμέρας ἐμβολίμου δέονται.
% Translation by LacusCurtius:
% "The one day, however, which results from the fourths he introduced
% into every fourth year, so as to make the annual seasons no longer differ
% at all except in the slightest degree; at any rate in fourteen hundred and
% sixty-one years there is need of only one additional intercalary day.
\lnr{40}Vide quot
errores.
\lnr{41}Ait \textgreek{διὰ τεσσάρων ἐτῶν}[?] embolismum bisexti fieri.
\lnr{41}Quod Graeco
proprie est, tertio anno exacto.
%
% 190
% {PDF page nr}{source page nr}{line nr}
\plnr{273}{190}{1}Sic Herodotus significans Graecos singulis
bienniis exactis intercalare, dicit, \textgreek{ἕλληνες διὰ τρίτου ἔτεος ἐμβόλιμον
ἐπεμβάλλουσι}[?].
\lnr{3}Hoc est inter secundum annum exactum, et tertium
ineuntem.
\lnr{4}Deinde arbitratur praeter quadrantem diei aliquid
superesse.
\lnr{5}Postremo post annos mille quadringentos sexaginta unum,
celebrandam unius diei intercalationem.
\lnr{6}Nam si id, quod ille male
intellexit, verum erat, unus annus supra 1460, ad intercalationem
non pertinet.
\lnr{8}Sed post 1460 Iulianos exactos, anno proxime
ineunte, Thoth Aegyptiacus redit iterum in eam diem mensis Iuliani,
in qua ante 1461 annos aequabiles fuerat.
\lnr{10}Praeterea anni aequabiles
1461 sunt 1460 absoluti Iuliani.
\lnr{11}Ego sane in hoc loco Dionis
immorandum amplius esse non censuissem, nisi is locus magnum
virum Gazam in eundem errorem impulisset: qui eum, quanquam
non iisdem verbis, in suum libellum de Mensibus traduxit.
\lnr{14}Cuius
verba necessario apponenda esse iudicavi.
\lnr{15}\textgreek{δοκοῦσιν ὁν Αἰγύπτιοι πρῶτοι
μησὶ χρήσασθαι καθ᾽ ἥλιον τριακονθημέροις δώδεκα, καὶ πέντε ἡμέρας ἐπαγαγέιν
κατ᾽ ἐνιαυτὸν ἕκαστον, τό, τε ἐπιτρέχον μόριον τὴς ἡμέρας ἐις ἐκπλήρωσιν
τοῦ ὅλου ὲνιαυτοῦ ἐκ περιόδων ἐτῶν ἀπολαβόντες συνθέσται μίαν ἡμέραν}[?].
\lnr{18}Aperte
ex Dione expiscatus est, ut vides.
\lnr{19}Et \textgreek{περίοδον πλειόνων ἐτῶν}[?] intelligit
mille quadrigentos sexaginta annos.
\lnr{20}Error sane in Philosopho potius,
quam historico castigandus.
\lnr{21}Non ergo \textgreek{μίαν ἡμέραν σηνθέσθαι}[?], sed
\textgreek{ἓν ἔτος}[?].
\lnr{22}Sed quid Iulio Firmico facias, qui ita in prooemio
% procemio? 1598 ed: "oe" ligature
 operis sui
scribit?
\lnr{23}\textit{Quantis etiam conversionibus maior ille, quem ferunt,
 persiceretur
annus, qui quinque has stellas, Lunam etiam, ac solem locis
suis, originibusque restituit, qui mille quadringentorum, et sexaginta
unius annorum circuitu terminatur.}
\lnr{26}An non plane \textgreek{ἀποκατάστασιν}[?] omnium
planetarum illo circuitu anni Canicularis fieri putat?
\lnr{27}Quid
imperitius potuit dici?
\lnr{28}Haec sunt, quae de anno aequabili Aegyptiaco
homines eruditi tradiderunt.
\lnr{29}Superest nunc, ut methodum Lunae in
hoc anno reperiamus, si qua periodus proxime cursum eius sideris
cum anno Aegyptiaco exaequare possit.
\lnr{31}Quae sane deprehensa est esse
viginti quinque annorum, ut quemanmodum enneadecaeteris in anno
Iuliano, ita \textgreek{εἰκοσιπενταετηρὶς}[?]
 in anno aequabili proxime ad praecisam
aequationem accdat.
\lnr{34}Enneadecaeteris Iuliana maiuscula est Lunari
hor. 1,485.
\lnr{35}\textgreek{εἰκοσιπενταετηρὶς}[?]
 autem Aegyptica relinquit supra ratiocinia
Lunae, hor. 1,123.
\lnr{36}Tanto praecisior est \textgreek{εἰκοσιπενταετηρὶς}[?] Enneadecaeteride.
\lnr{37}Modus anni Aegyptiaci, sive aequabilis, dies 365.
\lnr{37}Annus Lunaris
354, 8, 876.
\lnr{38}Differentia, dies 10, 15, 204.
\lnr{38}Duc igitur vicesies
quinquies hanc \textgreek{ὑπεροχὴν}[?].
\lnr{39}Prodeunt dies 265, hor. 19, scr. 780.
\lnr{39}Mensis
Lunaris 29, 12, 793.
\lnr{40}Quos multiplica novies.
\lnr{40}Quia tot embolismi sunt
in hac periodo.
\lnr{41}Producuntur dies 265, 18, 657.
\lnr{41}Deducantur de illo excessu
anni Aegyptiaci.
%
% 191
% {PDF page nr}{source page nr}{line nr}
\plnr{274}{191}{1}Remanent dies 0, 1, 123.
\lnr{1}Nunc videamus,
quomodo et quando intercalandum.
\lnr{2}Tres \textgreek{ὑπεροχαὶ}[?] anni Aegyptiaci
fiunt dies 31, 21, 612.
\lnr{3}Deducto mense Lunari, remanent epactae anni
tertii, 2, 8, 899.
\lnr{4}Quae cum triplicato iterum excessu, abiecto mense
Lunari, relinquunt epactas anni sexti, 4, 17, 718.
\lnr{5}Tertio triplicatur
excessus: qui cum epactis anni sexti, deducto mense Lunari,
relinquit epactas anni noni, 7, 2, 537.
\lnr{7}Quas si cum triplicato
excessu iunxero, deducto mense Lunari, epactae, quae hinc prodibunt,
nimium quantum excedent.
\lnr{9}Contenti simus igitur duplicato
excessu, 21, 6, 408.
\lnr{10}Cum epactis proxime praecedentis embolismi
fiunt dies 28, 8, 945.
\lnr{11}En Lunae rationes maiores anno Aegyptiaco.
\lnr{12}Nam mensis Lunaris excedit illam summam, ut scis.
\lnr{12}Deducatur
igitur nunc summa anni Aegyptiaci, de summa Lunari.
\lnr{13}Relinquuntur
epactae undecimi embolismi.
\lnr{14}Item triplietur excessus, neque iungatur
superioribus epactis: quia iam non anni Aegyptiaci excessus est,
sed Lunaris.
\lnr{16}Ex triplicato excessu Solis prodeunt epactae 2, 8, 899.
\lnr{17}Et quia hae epactae anni Aegyptiaci maiores sunt proximis epactis
 Lunaribus,
aufer Lunares 1, 3, 928, ab Aegyptiacis 2, 8, 899.
\lnr{18}Exeunt
1, 4, 1051, epactae 14 embolismi.
\lnr{19}Triplica iterum excessum.
\lnr{19}Iunge
epactas proximas.
\lnr{20}Abiice mensem Lunarem.
\lnr{20}Habes 3, 13, 870,
epactas anni 17.
\lnr{21}Triplica excessum.
\lnr{21}Iunge epactas.
\lnr{21}Abiice mensem.
\lnr{22}Prodeunt epactae vicesimi embolismi, 5, 22, 689.
\lnr{22}Iunge triplicato
excessui.
\lnr{23}Abiecto mense Lunari, remanent epactae vicesimi tertii
embolismi, 8, 7, 508.
\lnr{24}Quae cum duplicato excessu, deducta syzygiae
unius quantitate, relinquunt in vicesimo quinto anno differentiam
anni Aegyptiaci, et Lunaris, dies 0, 1, 123.
\lnr{26}Novies igitur intercalatur:
Ternis quidem annis, tertio, sexto, nono, quartodecimo, decimo
septimo, vicesimo, vicesimo tertio: Binis autem, undecimo,
et ultimo.
\lnr{29}Rursus quemadmodum ex Solis excessu supra Lunam in
cyclo enneadecaeterico epactae formantur ad indicandam aetatem
Lunae: ita etiam in hac periodo ex anni Aegyptiaci excessu Epactae
produci possunt.
\lnr{32}Epactae igitur primi anni erit excessus ipse anni
Aegyptiaci, 10, 15, 204.
\lnr{33}Secundi anni, duplum illarum 21, 6, 408.
\lnr{34}Tertii triplum.
\lnr{34}Et sic deinceps.
\lnr{34}Hae epactae in anno Aegyptiaco eum
usum habent, quem illae nostrae in anno Iuliano ad aetatem Lunae, ut
Computatores nostri loquuntur, ad \textgreek{ποστιαίαν}[?],
 ut Graeci, indicandam:
nempe ut Kalendarum Regulares cum die proposita mensis, et cum
epactis simul, abiectis tricenariis, cum opus erit,
 ostendant \textgreek{ποστιαίαυ τῆς
σελήνης}[?].
\lnr{39}Sed hoc differunt Regulares Aegyptiaci a regularibus Iulianis:
quod omnium Kalendarum Regulares Iuliani in methodo epactarum
adhibeantur: ut, verbi gratia, in Februario, qui est duodecimus
mensis Paschalis, cum investigatur aetas Lunae, duodecim regulares
mensium adhibentur: tot nimirum, quot sunt Kalendae a
Martio.
%
% 192
% {PDF page nr}{source page nr}{line nr}
\plnr{275}{192}{3}At in anno aequabili alterni regulares adiiciuntur.
\lnr{3}Nam omnium
mensium paris numeri nulli sunt regulares.
\lnr{4}Sed illorum vicem
antecedentium mensium regulares funguntur.
\lnr{5}Exempli gratia: Paophi
est secundus mensis, ideo paris numeri.
\lnr{6}Proinde regularis unus
erit Thoth praecedentis, et Paophi sequentis.
\lnr{7}Volo novilunium illi
competens investigare in anno, in quo epactae Lunares erunt \rnum{v}.
\lnr{8}Non
binos Regulares, quad binae Kalendae fluxerint, sed unos tantum
assumo: et invenio novilunium in \rnum{xxiiii} Paophi.
\lnr{10}Ratio huius haec
est: Novilunium in mensibus \textgreek{τριακονθημέροις}[?]
 binis in eadem die semper
deprehenditur, quamquam semisse die prius praeveritur a sequenti.
\lnr{13}Primus itaque et secundus mensis habent novilunium in eadem
die: tertius item et quartus.
\lnr{14}Et sic deinceps bini menses in
eodem die conficiunt novilunia.
\lnr{15}Quod non \textgreek{ἀκριβῶς}[?], sed \textgreek{πλατικῶς}[?]
dictum velim, quatenus patitur popularium Epactarum ratio.
\lnr{16}Nam
hac in re in anno Iuliano peccatur.
\lnr{17}Sunto epactae Lunares 25.
% Sentence starts with roman numeral.
\lnr{17}\rnum{xxix}dies
Aprilis erit \rnum{xxvi} Lunae.
\lnr{18}Et \rnum{xxx}dies eiusdem erit \rnum{xxvii} Lunae.
\lnr{19}Quare Kal. Mai Luna erit \rnum{xxviii}, ut docent Epactae.
\lnr{19}At assumptis
regularibus trinarum Kalendarum, prima dies Mai erit \rnum{xxix}.
\lnr{20}Fallit
ergo regula epactarum in mensibus illis, quos praecedunt menses
\textgreek{τριακονθήμεροι}[?].
\lnr{22}Propter regulares igitur fit \textgreek{ὑπέρβατοσ}[?] una dies.
\lnr{22}Cui
errori occurrendum est.
\lnr{23}Nam antecedente mense tricenario non videbantur
assumendi regulares sequentis.
\lnr{24}In anno igitur Aegyptiaco
menses, qui sunt pari numero, nullos regulares habent.
\lnr{25}In mensibus
vero imparibus, quot neomeniae fluxerint imparium, tot regulares
assumendi.
% No capital on 'ita'
\lnr{27}Ita ut numerus mensium dimidiandus sit, et productum
sint Regulares.
\lnr{28}Hic est usus epactarum, quarum diagramma
infra subiecimus.
\lnr{29}Cuius duplex usus.
\lnr{29}Nam epactae eatenus nomen
hoc retinent quandiu descendunt.
\lnr{30}Ascendentes autem sunt termini
noviluniorum.
\lnr{31}Exemplum: Epactae primi anni \textgreek{εἰκοστπενταετηρίδος}[?] in annis
Nabonassari sunt, 5, 22, 689.
\lnr{32}Quas reperies in 20 anno diagrammatis.
\lnr{33}Sequentis anni epactae sunt in 21 anno diagrammatis, nempe,
16, 13, 893, et ita deinceps.
\lnr{34}Contra terminus noviluniorum primi anni
Nabonassari est in eo anno, qui ascendens proxime praecedit annum
primum Epactarum, nempe in anno 19 Diagrammatis: in
quo notatus est terminus 24, 20, 198.
\lnr{37}Dies igitur \rnum{v} Epactarum,
cum uno regulari Thoth constituit novilunium in 24 die Thoth,
qui est Terminus novilunii.
\lnr{39}Eadem methodus in aliis.
\lnr{39}Quae facilima
est, dummodo Epactas scias descendere, Terminos autem ascendere:
item proximum annum ascendentem a primis Epactis, esse primum
Terminorum.
%
% 193
% {PDF page nr}{source page nr}{line nr}
\plnr{276}{193}{1}Haec ratio expedita erat, si epactae praecise essent
dierum certorum, neque horas et scrupulos appendices haberent.
\lnr{2}Quo fit, ut dies tantum % tantū; 1598 ed.: tantum
 terminorum, exclusis horis, et scrupulis progressu
temporis novilunii fines non attingant.
\lnr{4}Nam in 23 periodis Luna
antevertit cyclum anni aequabilis die uno.
\lnr{5}Itaque castigatio adhibenda,
quoties summa annorum Aegyptiorum, aut Armeniorum,
aut Persicorum excedit 500 annos, aut paulo amplius.
\lnr{7}Melius igitur
per characterem et feriam diei methodum
Lunae tractabis.
%
\begin{table}[p]
  % define table height
  \newcommand{\tabh}{\textheight}
%  \setlength{\tabcolsep}{0.0ex}
%  \centering
  \begin{tabular}{r @{\hspace{0.02\textwidth}} r}
%  \resizebox{0.45\textwidth}{!}{%
    \begin{minipage}[][\tabh][t]{0.45\textwidth}
      %%% Liber 3 p193, PDF 276
%%
%%% Count out columns for fixed-width source font
% 000000011111111112222222222333333333344444444445555555555666666666677777777778
% 345678901234567890123456789012345678901234567890123456789012345678901234567890
%
\begin{tabnums} % Select monospaced numbers
%% Select a general font size (uncomment one from the list)
%\tiny
%\scriptsize
%\footnotesize
%\small
\normalsize
%% Center the whole table left-right
\centering
%% Modify separation between columns
\setlength{\tabcolsep}{1.0ex}
%% Modify distance between rows
%\renewcommand{\arraystretch}{1.2}
%
%% Width of a column
\newcommand{\cwd}{3.2em}
%% Define reference symbols
\newcommand{\da}{{\tiny †}}
\newcommand{\db}{{\scriptsize o}}
%% The angle with which to slant
\newcommand{\ang}{90}
%% Header text size: row above row above bottom row
\newcommand{\hsc}[1]{\small{#1}}
%% Header text size: row above bottom row
\newcommand{\hsb}[1]{\scriptsize{#1}}
%% Header text size: bottom row
\newcommand{\hsa}[1]{\tiny{#1}}
%% Generate the column headers
\newcommand{\hdrC}{%
  \multicolumn{6}{c}{\hsc{Periodus prior}} &
  &
  \multicolumn{6}{c}{\hsc{Periodus altera}}  
}
%
\newcommand{\hdrB}{%
  \multicolumn{4}{c}{\hsb{Pars Prior.}} &
  &
  \multicolumn{3}{c}{\hsb{Pars Poster.}}  
}
%
\newcommand{\hdrA}{%
  \ch{888}{\hsa{Dies collecti}} &
  \ch{81}{\hsa{Feria}}&
  \ch{88}{\hsa{Horae}} &
  \ch{1888}{\hsa{Scrup.}} &
  &
  \ch{81}{\hsa{Dies}} &
  \ch{88}{\hsa{Horae}} &
  \ch{1888}{\hsa{Scrup.}}
}
%
\newcommand{\hdrs}{%
 ~ & \hdrB \\
\cmidrule(lr){2-5} \cmidrule(lr){7-9}
 ~ & \hdrA \\
}
%
\begin{tabular}[c]{@{} r rrrr c rrr @{}}
\toprule
\multicolumn{9}{c}{\Large\textsc{Tabella Mensium}} \\
\toprule
\hdrs % Column headers from the above definition
\midrule
%%
 1 &  29 & 1 & 12 &  793 && 0 & 11 & 287 \\
 2 &  59 & 3 &  1 &  506 && 0 & 22 & 574 \\
 3 &  88 & 4 & 14 &  219 && 1 &  9 & 861 \\
 4 & 118 & 6 &  2 & 1012 && 1 & 21 &  68 \\
 5 & 147 & 7 & 15 &  725 && 2 &  8 & 355 \\
 6 & 177 & 2 &  4 &  438 && 2 & 19 & 642 \\
 7 & 206 & 3 & 17 &  151 && 3 &  6 & 929 \\
 8 & 236 & 5 &  5 &  944 && 3 & 18 & 136 \\
 9 & 265 & 6 & 18 &  657 && 4 &  5 & 423 \\
10 & 295 & 1 &  7 &  370 && 4 & 16 & 710 \\
11 & 324 & 2 & 20 &   83 && 5 &  3 & 997 \\
12 & 354 & 4 &  8 &  876 && 5 & 15 & 204 \\
13 & 383 & 5 & 21 &  589 && 6 &  2 & 491 \\
\bottomrule
\end{tabular}
\caption{Tabella Mensium}
\label{tab:p193}
\end{tabnums}

      \bigskip
      %%% Liber 3 p193, PDF 276
%%
%%% Count out columns for fixed-width source font
% 000000011111111112222222222333333333344444444445555555555666666666677777777778
% 345678901234567890123456789012345678901234567890123456789012345678901234567890
%
\begin{tabnums} % Select monospaced numbers
%% Select a general font size (uncomment one from the list)
%\tiny
%\scriptsize
\footnotesize
%\small
%\normalsize
%% Center the whole table left-right
\centering
%% Modify separation between columns
\setlength{\tabcolsep}{1.0ex}
%% Modify distance between rows
%\renewcommand{\arraystretch}{1.2}
%
%% Width of a column
\newcommand{\cwd}{3.2em}
%% Define reference symbols
\newcommand{\da}{{\tiny †}}
\newcommand{\db}{{\scriptsize o}}
%% The angle with which to slant
\newcommand{\ang}{90}
%% Header text size: row above row above bottom row
\newcommand{\hsc}[1]{\small{#1}}
%% Header text size: row above bottom row
\newcommand{\hsb}[1]{\scriptsize{#1}}
%% Header text size: bottom row
\newcommand{\hsa}[1]{\tiny{#1}}
%% Generate the column headers
%
\newcommand{\hdrA}{%
  \ch{888}{\hsa{Anni collecti}} &
  \ch{888}{\hsa{Dies.}}&
  \ch{888}{\hsa{Hor.}} &
  \ch{1888}{\hsa{Scrup.}} &
  &
  \ch{888}{\hsa{Feria.}} &
  \ch{888}{\hsa{Hor.}} &
  \ch{1888}{\hsa{Scrup.}}
}
%
\newcommand{\hdrs}{%
\hdrA \\
}
%
\begin{tabular}[c]{@{} r rrr c rrr @{}}
\toprule
\multicolumn{8}{c}{\Large\textsc{Tabella Annorum}} \\
\multicolumn{8}{c}{\large\textsc{Collectorum}} \\
\toprule
\hdrs % Column headers from the above definition
\midrule
%%
  25 & 0 &  1 &  123 && 3 & 22 &  957 \\
  50 & 0 &  2 &  246 && 7 & 21 &  834 \\
  75 & 0 &  3 &  369 && 4 & 20 &  711 \\
 100 & 0 &  4 &  492 && 1 & 19 &  588 \\
 125 & 0 &  5 &  615 && 5 & 18 &  465 \\
 150 & 0 &  6 &  738 && 2 & 17 &  342 \\
 175 & 0 &  7 &  861 && 6 & 16 &  219 \\
 200 & 0 &  8 &  984 && 3 & 15 &   96 \\
 225 & 0 & 10 &   27 && 7 & 13 & 1053 \\
 250 & 0 & 11 &  150 && 4 & 12 &  930 \\
 500 & 0 & 22 &  300 && 2 &  1 &  780 \\
 750 & 1 &  9 &  450 && 6 & 14 &  630 \\
1000 & 1 & 20 &  600 && 4 &  3 &  480 \\
1250 & 2 &  7 &  750 && 1 & 16 &  330 \\
1500 & 2 & 18 &  900 && 6 &  5 &  180 \\
1750 & 3 &  5 & 1050 && 3 & 18 &   30 \\
2000 & 3 & 17 &  120 && 1 &  6 &  960 \\
2250 & 4 &  4 &  270 && 5 & 19 &  810 \\
2500 & 4 & 15 &  420 && 3 &  8 &  660 \\
5000 & 5 &  2 &  570 && 6 & 17 &  240 \\
\bottomrule
\end{tabular}
\caption{Annorum Collectorum}
\label{tab:p193c}
\end{tabnums}

    \end{minipage}
%  }
&
%  \resizebox{0.45\textwidth}{!}{%
    \begin{minipage}[][\tabh][t]{0.53\textwidth}
      %%% Liber 3 p193, PDF 276
%%
%%% Count out columns for fixed-width source font
% 000000011111111112222222222333333333344444444445555555555666666666677777777778
% 345678901234567890123456789012345678901234567890123456789012345678901234567890
%
\begin{tabnums} % Select monospaced numbers
%% Select a general font size (uncomment one from the list)
%\tiny
%\scriptsize
%\footnotesize
%\small
\normalsize
%% Center the whole table left-right
\centering
%% Modify separation between columns
\setlength{\tabcolsep}{1.0ex}
%% Modify distance between rows
%\renewcommand{\arraystretch}{1.2}
%
%% Width of a column
\newcommand{\cwd}{3.2em}
%% Define reference symbols
\newcommand{\da}{{\tiny †}}
\newcommand{\db}{{\scriptsize o}}
%% The angle with which to slant
\newcommand{\ang}{90}
%% Header text size: row above row above bottom row
\newcommand{\hsc}[1]{\small{#1}}
%% Header text size: row above bottom row
\newcommand{\hsb}[1]{\scriptsize{#1}}
%% Header text size: bottom row
\newcommand{\hsa}[1]{\tiny{#1}}
%% Generate the column headers
%
\newcommand{\hdrB}{%
  ~ &
  \multicolumn{3}{c}{\hsb{Epactae.}} &
  &
  \multicolumn{3}{c}{\hsb{Novilunia.}}  
}
%
\newcommand{\hdrA}{%
  \ch{888}{\hsa{Anni expansi}} &
  \ch{81}{\hsa{Epact.}}&
  \ch{88}{\hsa{Hor.}} &
  \ch{1888}{\hsa{Scrup.}} &
  &
  \ch{81}{\hsa{Feria.}} &
  \ch{88}{\hsa{Hor.}} &
  \ch{1888}{\hsa{Scrup.}}
}
%
\newcommand{\hdrs}{%
\hdrB \\
\cmidrule(lr){2-4} \cmidrule(lr){6-8}
\hdrA \\
}
%
\begin{tabular}[c]{@{} r rrr c rrr l@{}}
\toprule
\multicolumn{9}{c}{\Large\textsc{Tabella Annorum}} \\
\multicolumn{9}{c}{\large\textsc{Expansorum}} \\
\toprule
\hdrs % Column headers from the above definition
\midrule
%%
 1 & 10 & 15 &  204 && 4 &  8 &  876 & ~\\
 2 & 21 &  6 &  408 && 1 & 17 &  672 & ~\\
 3 &  2 &  8 &  899 && 7 & 15 &  181 & \da \\
 4 & 13 &  0 &   23 && 4 & 23 & 1057 & ~\\
 5 & 23 & 15 &  227 && 2 &  8 &  853 & ~\\
 6 &  4 & 17 &  718 && 1 &  6 &  362 & \da \\
 7 & 15 &  8 &  922 && 5 & 15 &  158 & ~\\
 8 & 26 &  0 &   46 && 2 & 23 & 1034 & ~\\
 9 &  7 &  2 &  537 && 1 & 21 &  543 & \da \\
10 & 17 & 17 &  741 && 6 &  6 &  339 & ~\\
11 & 28 &  8 &  945 && 5 &  3 &  928 & \da \\
12 &  9 & 11 &  356 && 2 & 12 &  724 & ~\\
13 & 20 &  2 &  560 && 6 & 21 &  520 & ~\\
14 &  1 &  4 & 1051 && 5 & 19 &   29 & \da \\
15 & 11 & 20 &  175 && 3 &  3 &  905 & ~\\
16 & 22 & 11 &  379 && 7 & 12 &  701 & ~\\
17 &  3 & 13 &  870 && 6 & 10 &  210 & \da \\
18 & 14 &  4 & 1074 && 3 & 19 &    6 & ~\\
19 & 24 & 20 &  198 && 1 &  3 &  882 & ~\\
20 &  5 & 22 &  689 && 7 &  1 &  391 & \da \\
21 & 16 & 13 &  893 && 4 & 10 &  187 & ~\\
22 & 27 &  5 &   17 && 1 & 18 & 1063 & ~\\
23 &  8 &  7 &  508 && 7 & 16 &  572 & \da \\
24 & 18 & 22 &  712 && 5 &  1 &  368 & ~\\
25 &  0 &  1 &  123 && 3 & 22 &  957 & \da \\
\bottomrule
\addlinespace[5pt]
 & \multicolumn{3}{l}{\footnotesize\super{†}Emb.}
\end{tabular}
\caption{Tabella Mensium}
\label{tab:p193}
\end{tabnums}

    \end{minipage}
%  }
\\
%    \addlinespace[0.85in] % force a lot of space to make room for the captions
  \end{tabular}
\end{table}
%
\lnr{9}Cuius rei
gratia Tabulas duplices tibi confecimus
in mensibus, annis expansis
et collectis per periodos suas.\super{p.~\pageref{tab:p193a}}
\lnr{13}Cuius usus duplex, ut et ipsa duplex.
\lnr{14}Aut enim per partem anteriorem,
aut per posteriorem, operari
potes.
%
% 194
% {PDF page nr}{source page nr}{line nr}
\plnr{277}{194}{1}Et pars quidem anterior, cum collecta fuerit, auferenda
est de 30 diebus, ut suo loco dicetur.
\lnr{2}Residuum erit exactissimum
novilunium, in horis, et scrupulis: ut Abacus Astronomicus certius
daturus non sit.
\lnr{4}Contraria methodus in posteriore parte.
\lnr{4}Nam
per adiectionem, non per detractionem res tractatur.
\lnr{5}Quod suo quidque
loco explicandum relinquimus.
%
%====
\section{De Anno Aegyptiaco}
% Capitalisation follows ToC
%
\lnr{7}Antiquissima, et simplicissima anni forma, ac popularibus
temporibus accommodatissima, est ea, quae in tricenarios
numeros tribuitur.
\lnr{9}Nam sane antiquitus, praesertim apud
Aegyptios, annus constabat tantum diebus trecentis sexaginta.
\lnr{10}Qui numerus
propter aequabilitatem divisionis est aptissimus.
\lnr{11}Aegyptii vero
Hierophantae, cum scirent anno populari ad perfectionem deesse dies
integros quinque ac quadrantem, eum annum perfectum siquando
significare vellent, Serpentem in orbem ac circulum convolutum,
cauda ore admorsa pingebant in suis Hieroglyphicis monumentis,
tanquam Sol perfectum et absolutum Zodiaci curriculum non conficiat,
si quinque illae dies appendices desint.
\lnr{17}(Aliter apud Eusebium
libro primo \textgreek{προπαρασκευῆς}[?];
% Greek: "Preparation"
\textgreek{ἔτι μὴν οἱ Αἰγύπτιοι ἀπὸ τὴς αὐτῆς ἐννοίας
τὸν κόσμον γράφοντες, περιφερῆ κύκλον ἀεροειδῆ καὶ πυρωπὸν χαράσσουσιν, καὶ μέσον
τεταμένον ὄφιν ἱερακόμορφον. καὶ ἔστι τὸ πᾶν σχῆμα ὡς τὸ παρ᾽ ἡμῖν Θῆτα.
τὸν μὲν κύκλον, κόσμον μηνύοντες. τὸν δὲ μέσον ὄφιν συνεκτικὸν τούτου ἀγαθὸν
δαίμονα σημαίνοντες.}[?])
% Eusebius of Caesarea: Preparation for the Gospel (Εὐαγγελικὴ προπαρασκευή)
% (Praeparatio evangelica). Book 1; Chapter 10: The theology of the Phoenecians.
% 4th paragraph from the end.
% Ἔτι μὴν οἱ Αἰγύπτιοι ἀπὸ τῆς αὐτῆς ἐννοίας
% τὸν κόσμον γράφοντες περιφερῆ κύκλον ἀεροειδῆ καὶ πυρωπὸν χαράσσουσιν καὶ μέσα
% τεταμένον ὄφιν ἱερακόμορφον (καὶ ἔστι τὸ πᾶν σχῆμα ὡς τὸ παρ' ἡμῖν Θῆτα),
% τὸν μὲν κύκλον κόσμον μηνύοντες, τὸν δὲ μέσον ὄφιν συνεκτικὸν τούτου Ἀγαθὸν
% Δαίμονα σημαίνοντες.
% Translation by E. H. Gifford (1903)
% "Moreover the Egyptians, describing the world from the same idea, engrave the
% circumference of a circle, of the colour of the sky and of fire, and a hawk-
% shaped serpent stretched across the middle of it, and the whole shape is like 
% our Theta (θ), representing the circle as the world, and signifying by the
% serpent which connects it in the middle the good daemon."
\lnr{22}Eum Serpentem \textgreek{νεισὶ}[?] vocabant: quodmodo hodie
Coptitae, et Aegyptii vetustissimi Christiani
 \textgreek{τὰς ὲπαγομένας}[?] vocare
solent lingua prisca Aegyptiaca, qua sacros utriusque Testamenti
libros conscriptos habent, et sacra itidem in templis obeunt.
\lnr{26}Inuenio enim illas dies \textarabic{}[?] ab illis, Arabice denotari.
\lnr{26}Nam Horus
Apollo Serpentem illum \textgreek{κοσμοειδῶς ἐσχηματισμείον}[?],
 quo annum, mundum,
Regem, et alia significabant, scribit vocari \textgreek{Μεισὶ}[?].
\lnr{28}Sed mendum
esse librarii puto.
\lnr{29}Nam, ut dixi, finem anni Solaris etiamnum
hodie Aegyptii Christiani Nesi vocant.
\lnr{30}Eum autem \textsc{nesi} sive \textsc{Nisi}
his elegantissimis versibus describit Claudianus Panegyrico
 \ruleover{\rnum{ii}} in
Stiliconem.
\begin{verse}
\textit{Est ignota procul, nostraeque impervia menti\\
  Vix adeunda Deis, annorum squalida mater,\\
  Immensi spelunca \textsc{aevi}, quae tempora vasto\\
  Suppeditat, revocatque sinu. Complectitur antrum,\\
  Omnia qui placido consumit numine \textsc{serpens}:\\
  Perpetuumque viret squamis, caudamque reducto\\
  Ore vorat, tacito relegens exordia lapsu.
}
% Claudius Claudianus: De Consulatu Stilichonis
% Claudian: On Stilicho's Consulship
% Lines 424-430
% Immensi/Immensis/Inmensi
% reducto -> reductam
\end{verse}
%
% 195
% {PDF page nr}{source page nr}{line nr}
\plnr{278}{195}{1}Porro nomen \textgreek{ἐπαγομένων}[?] dictum est
 \textgreek{κατὰ στέρησιν}[?], tanquam olim annus
360 dierum simpliciter fuisset.
\lnr{2}Quod, ut dixi, Hierophantae
ipsi non solum in libris suis scriptum habebant, sed et fabulam origini
et causae \textgreek{τῶν ἐπαγομένων}[?] adiiciebant: Mercurium scilicet alea
cum Luna ludentem vicisse, et septuagesimam secundam partem
anni ab ea extorsisse, quam postea 360 diebus, qui erat modus anni
prisci, adiecerit.
\lnr{7}Plutarchus ita rem narrat: \textgreek{Τῆς Ρέας, φασὶ, κρύφα
τῷ Κρόνῳ συγγενομένης, αἰσθόμενον ὲπαράσασθαι τὸν Ηλιον αὐτῇ, μήτε μηνὶ,
μήτε ἐνιαυτῷ τεκεῖν.}[?]
\lnr{9}\textgreek{ἐρῶντα δὲ τὸν Ερμῆν τῆς θεοῦ συνελθεῖν.}[?]
\lnr{9}\textgreek{εἶτα παίξαντα
πεττία πρὸς τὴν σελήνην, καὶ ἀφελόντα τῶν φώτων ἑκάστου ἑβδομηκοστὸν δεύτερον,
ἐκ πάντων ἡμέρας πέντε συνελεῖν, καὶ ταῖς ἑξήκοντα καὶ τριακοσίαις
ἐπαγαγεῖν, ἃς νῦν ἐπαγομένας
Αἰγύπτιοι καλοῦσι, καὶ τῶν θεῶν γενεθλίους
ἄγουσι.}[?]
% Plutarch: Moralia;Book 5: Isis and Osiris [12]
% Τῆς Ῥέας φασὶ κρύφα
% τῷ Κρόνῳ συγγενομένης αἰσθόμενον ἐπαράσασθαι τὸν Ἥλιον αὐτῇ μήτε μηνὶ
% μήτ᾿ ἐνιαυτῷ τεκεῖν·
% ἐρῶντα δὲ τὸν Ἑρμῆν τῆς θεοῦ συνελθεῖν,
% εἶτα παίξαντα
% πεττία πρὸς τὴν σελήνην καὶ ἀφελόντα τῶν φώτων ἑκάστου [τὸ] ἑβδομηκοστὸν
% ἐκ πάντων ἡμέρας πέντε συνελεῖν καὶ ταῖς ἑξήκοντα καὶ τριακοσίαις
% ἐπαγαγεῖν, ἃς νῦν ἐπαγομένας
% Αἰγύπτιοι καλοῦσι καὶ τῶν θεῶν γενεθλίους
% ἄγουσι.
% Translation:
% They say that the Sun, when he became aware of Rhea’s intercourse with
% Cronus, invoked a curse upon her that she should not give birth to a child
% in any month or any year; but Hermes, being enamoured of the goddess,
% consorted with her. Later, playing at draughts with the moon, he won from her
% the seventieth part of each of her periods of illumination, and from all the
% winnings he composed five days, and intercalated them as an addition to the
% three hundred and sixty days. The Egyptians even now call these five days
% intercalated and celebrate them as the birthdays of the gods.
\lnr{14}Ergo ab initio \textgreek{ἐπαγόμεναι}[?]
nullae erant.
\lnr{15}Et solo
verbo indicatur privatio earum
antiquitus.
\lnr{17}Quod nimirum antea
non fuisse videantur, eo
quod adiectae sint.
%
% Small text table
%
\begin{table}[h]
  %%% Liber 3 p195, PDF 278
%%
%%% Count out columns for fixed-width source font
% 000000011111111112222222222333333333344444444445555555555666666666677777777778
% 345678901234567890123456789012345678901234567890123456789012345678901234567890
%
%% Center the whole table left-right
\centering
%
% Contents is not a real table, more a set of text lines
% Implemented as a minipage
\begin{minipage}{0.6\textwidth}
  \begin{center}
    \textgreek{ΕΠΑΓΟΜΕΝΑΙ ἤτοι ΝΕΙΣΙ.}[?]\\
  \end{center}
  \textgreek{Πρώτη, ΟΣΙΡΙΣ.} \textit{Frater Isidis.}\\
  \textgreek{Δευτέρα, ΑΡΟΥΗΡΙΣ.} \textit{Putamus esse anubim.}[?]\\
  \textgreek{Τρίτη ΤΥΦΩΕΥΣ}[?], \textit{vir Isidis.}\\
  \textgreek{Τετάρτη ΙΣΙΣ.}[?]\\
  \textgreek{Πέμητη ΝΕΦΘΗ, ἤτοι ΑΠΟΦΡΑΣ.}[?]
    \textit{Firmico de errore profanarum religionum dicitur}
    \textgreek{ΝΕΦΘΟΥΝΗ}[?], \textit{Soror Isidis.}
\end{minipage}
\caption[\textgreek{Επαγομεναι ἤτοι Νεισι}]{}
\label{tab:p195}

\end{table}
\lnr{19}\textgreek{Επαγομένων}[?]
autem Aegyptiacarum haec cognomina erant, ut infra subiecta sunt.
\lnr{21}Annus ergo Aegyptiacus fuit dierum trecentum sexaginta quinque,
sine quadrantis adiectione.
\lnr{22}Cuius Thoth primus necessario coepit ab
ortu Caniculae, Sole in Leonem transeunte, novilunio.
\lnr{23}Quia
observatio anni et temporum in Caniculae ortu ab Aegyptiis statuebatur:
eaque erat ipsis cursus Solaris epocha.
\lnr{25}Et quamuis eorum Thoth
laxis habenis in anteriora fugeret, tamen quatro quoque anno diem
intercalabant, ut testatur Diodorus Siculus libro primo, cum de
Thebanis Aegypti loquitur: \textgreek{τὰς γὰρ ἡμέρας}[?], inquit,
 \textgreek{οὐκ ἄγουσι κατὰ σελήνην,
ἀλλὰ κατὰ τὸν ἥλιον, τριακονθημέρους μὲν τιθέμὲνοι τοὺς μῆνας, πέντε
δὲ ἡμέρας, καὶ τέταρτον τοῖς δώδεκα μησὶν ἐπάγουσι.}[?]
\lnr{30}\textgreek{καὶ τούτῳ τῷ τρόπῳ τὸν ἐνιαύσιον κύκλον ἀναπληροῦσιν.}[?]
% Diodorus Siculus, Bibliotheca Historica, Libro primo, caput L, phrase 3
% Τὰς γὰρ ἡμέρας
% [, inquit,]
%  οὐκ ἄγουσι κατὰ σελήνην,
% ἀλλὰ κατὰ τὸν ἥλιον, τριακονθημέρους μὲν τιθέμενοι τοὺς μῆνας, πέντε
% δ´ ἡμέρας καὶ τέταρτον τοῖς δώδεκα μησὶν ἐπάγουσι,
% καὶ τούτῳ τῷ τρόπῳ τὸν ἐνιαύσιον κύκλον ἀναπληροῦσιν.
% Translation: "For they do not reckon the days by the moon, but by the sun,
% making their month of thirty days, and they add five and a quarter days
% to the twelve months and in this way fill out the cycle of the year."
\lnr{31}Id autem fiebat hoc modo: Esto neomenia
Thoth, Kalendis Augusti, Canicula oriente: post quartum annum
Thoth superatis Kal. Augusti ascendet in \rnum{xxxi} Iulii.
\lnr{33}Quare Aegyptiorum
Hierophantae quinto anno ineunte, annum suum \textgreek{ἱερογλυφικὸν}[?]
a secunda die Thoth auspicabantur, et spatium illud quadriennii
\textgreek{κυνικὸν ἐνιαυτὸν}[?] vocabant, item \textgreek{ἡλιακὸν ἔτος}[?],
 item \textgreek{ἔτος Θεοῦ}[?], id
est Solis.
\lnr{37}Sic secundo quadriennio exacto, noni anni principium a
tertia Thoth putabant.
\lnr{38}Et quot quadirennia praeterierant, tot \textgreek{ἐνιαυτοὺς
κυνικοὺς}[?] putabant; quadriennium quidem exactum
 \textgreek{ἔτος Θεοῦ}[?] vocantes,
annum autem aequabilem labentem, \textsc{quadrantem}.
\lnr{40}Verbi
gratia: Proponatur vicesima sexta dies Paophi, qui est secundus
mensis.
%
% 196
% {PDF page nr}{source page nr}{line nr}
\plnr{279}{196}{1}A neomenia Thoth, ad vicesimam sextam Paophi, fluxerunt
dies quinquaginta quinque solidi.
\lnr{2}Qui per quatuor divisi
dant tredecim \textgreek{ἐνιαυτοὺς θεού}[?] cum tribus quadrantibus, et vicesima
sexta dies est quartus quadrans anni \rnum{xiiii} Canicularis.
\lnr{4}Itaque hoc
modo in literis subsignabant, et sacris libris: \textsc{actum anni
dei quarti decimi quadrante quarto}.
\lnr{6}Et quia
annum, ut dixi, aequabilem vocabant \textsc{quadrantem}, propterea
illum \textgreek{ἱερογλυφικῶς}[?] designare volentes,
 quartam partem modi arvalis,
seu iugeri pingebant.
\lnr{9}Horus Apollo: \textgreek{ἔτοσ τὸ ἐνιστάμηνον γράφοντες,
τέταρτον ἀρούρας γράφουσιν.}[?]
\lnr{10}\textgreek{ἔστι δὲ μέτρον γῆς ἡ ἄρουρα, πηχῶν
ἑκατόν.}[?]
\lnr{11}\textgreek{βουλόμηνοί τε ἔτος εἰπεῖν, ΤΕΤΑΡΤΟΝ λέγουσιν.}[?]
\lnr{11}\textgreek{ἐπειδή,
φασι, κατὰ τὴν ἀνατολὴν τοῦ ἄστρου τῆς Σώθεως, μέχρι τῆς ἄλλης ἀνατολῆς,
τέταρτον ἡμέρας προστίθεται, ὡς εἶναι τὸ ἔτος τοῦ ΘΕΟΥ τριακοσίων
ἑξήκοντα πέντε ἡμερῶν.}[?]
\lnr{14}\textgreek{ὅθεν καὶ διὰ τετραετηρίδος περισσὴν ἡμέραν ἀριθμοῦσιν
Αἰγύπτιοι.}[?]
\lnr{15}\textgreek{τὰ τὴς τέσσαρα τέταρτα ἡμέραν ἀπαρτίζει.}[?]
\lnr{15}Idem lib.
\rnum{II}. \rnum{lxxxix}.
\lnr{16}\textgreek{τὸ δ᾽ ἔτος κατ᾽ Αἰγυπτίους τεσσάρων ἐνιαυτῶν.}[?]
\lnr{16}Dilucide et
plane non solum ex quadriennio Aegyptiaco, et quadrante \textgreek{κυνικὸν
ἐνιαυτὸν}[?] constare dicit, sed etiam simplicem annum \textsc{quadrantem}
vocari.
\lnr{19}Sed minus proprie, imo falso dixit, \textgreek{διὰ
τετραετηρίδος περισσὴν ἡμέραν}[?] numerare, imo \textgreek{διὰ πενταετηρίδος}[?], hoc
est anno quarto absoluto, quinto ineunte.
\lnr{21}Videntur vero manca scriptoris
Graeci verba, et legendum, \textgreek{ῶς εἶναι τὸ ἔτος τοῦ ΘΕΟΥ τριακοσίων
ἑξήκοιτα πέντε ἡμερῶν, καὶ ἔτι πρὸς τεταρτημορίου}[?].
\lnr{23}Quare errat Censorinus
dupliciter, cum de hoc anno loquitur: primum, quod \textgreek{κυνικὸν
ἐνιαθτὸν}[?] vocari tantum dicit, eum qui ex multis centuriis annorum
constet.
\lnr{26}Sed etiam, qui putet 1460 anno Iuliano vertente Thoth in
eundem Solis revolui locum, unde ante tot annos profectus fuerat.
\lnr{28}Nam ut paulo ante disputavimus, Thoth profectus ab ortu Caniculae
Kal. Augusti, post 1460 annos aberit ab ipso ortu Caniculae in consequentia
diebus plus minus novem.
\lnr{30}Quod autem quarto quoque
anno diem intercalerent Aegyptii, et in ipso ortu Caniculae, satis
constat ex observatione Eudoxi, qui periodos et ambitus tempestatum
putabat confici quarto quoque anno.
\lnr{33}Eoque fine ille lustrum
suum circumscripsit, anno intercalari.
\lnr{34}Plinius libro \ruleover{\rnum{ii}}: \textit{Omnium
quidem, si libeat observare minimos ambitus, redire easdem vices quadriennio
exacto Eudoxus putat, non ventorum modo, verum et reliquarum
tempestatum magna ex parte.}
\lnr{37}\textit{Est principium lustri eius semper intercalari
anno, Caniculae ortu.}
% Gaius Plinius Secundus, Naturalis Historia, liber 2
% 130.1-4
\lnr{38}Hactenus Plinius.
\lnr{38}Qui et libro \rnum{xviii} ait tempestates
ipsas quadrinis annis suos ardores habere, et easdem non magna
differentia reverti ratione Solis: octonis vero augeri easdem, centesima
revoluente Luna.
\lnr{41}Quod nihil aliud est, quam quatuor annis Iulianis
fieri \textgreek{ἀποκατάστασιν τῶν ἐπισημασιῶν}[?], eamque geminari bis totidem
annis.
%
% 197
% {PDF page nr}{source page nr}{line nr}
\plnr{280}{197}{2}Non solum autem in Aegyptum profectum fuisse Eudoxum
ex Laertio et aliis constat, sed etiam diei quarto quoque anno
exacto intercalandi methodum Graecis prodidisse auctor est Strabo.
\lnr{5}Quanquam multo antea Graecis usurpatam scio.
\lnr{5}Quomodo enim
epocham primi mensis Iphitei servare potuissent?
\lnr{6}Quare vel hinc
clarum est, lustrum Eudoxi nihil aliud, quam quadriennium Caniculare
Aegyptiorum fuisse.
\lnr{8}Idque ipsum Eudoxum a sacerditibus
Aegyptiis didicisse, cum eorum gratia in Aegyptum profectus esset,
ibique suam Octaeterida conscripsisset.
\lnr{10}Sed omne dubium tollit Canicularis
ortus adiecta mentio.
\lnr{11}Ut autem maximus annus Canicularis
revertatur, fateor equidem, opus esse, ut Thoth in ortum Caniculae
incurrat, non autem, ut anni Aegyptiaci solidi 1460 elabantur.
\lnr{14}Hoc accidisse Ulpio, et Brutio Praesente \textsc{coss}.
 testis est idem Censorinus.
\lnr{15}Quo tempore Thoth incidit in \rnum{xx} Iulii, Sole in Leonem transitum
faciente, et Caniculae ortu iam imminente, anno ab % à
 Nabonassaro
886, Christi 138.
\lnr{17}Verus igitur \textgreek{ἐνιαυτὸς ΘΕΟΥ}[?], sive \textgreek{κηνικὸς}[?],
constat quatuor annis Aegyptiacis, et die ex quatuor quadrantibus
diurni temporis conflato.
\lnr{19}Magna autem periodus, quem \textgreek{κηνικὸν
ἐνιαυτὸν}[?] frustra vocat ab ortu Caniculae Censorinus,
 constat annis Canicularibus
solidis 365, quot nempe dies sunt in anno aequabili.
\lnr{22}Et hic est vere annus magnus Canicularis,
 dictus quia ex tot Canicularibus
annis constet, quot dies habet aequabilis annus absolutus:
non autem quod eius Thoth incidat in ortum Caniculae.
\lnr{24}Nam
unde libuerit, Periodum illam magnam constituere potes.
\lnr{25}Triplex
igitur fuerit annus Canicularis.
\lnr{26}Magnus constans quatuor annis et
quadrante: ita dictus, quod quadrans eius observatus sit in ortu Caniculae.
\lnr{28}Item maior Canicularis ex 365 magnis compositus.
\lnr{28}Et maximus, mense Thoth in ortum Caniculae incidente.
\lnr{29}Quod si annus
ille, qui incidit in Consulatum Ulpii, et Brutii Praesentis, fuisset verus
annus vertens periodi magnae Canicularis, iam nihil obstaret
nobis, quominus possemus caput verae periodi Aegyptiae investigare.
\lnr{33}Nunc cum Censorinus referat caussam anni Canicularis
magni ad ortum Caniculae, in quem Thoth forte eo anno inciderat,
manifestum est, eam non esse periodum, quam ipse intelligit.
\lnr{36}Quamuis non dubium est, primum Thoth ab ortu \textgreek{Σώθιος}[?] sive
Caniculae, repetendum esse.
\lnr{37}Apud Herodotum in Euterpe, de anni
Aegyptiaci antiquitate haec exstant: Temporibus ipsius Herodoti,
Aegyptios a mundi conditu putare annos 11340.
\lnr{39}Eosque dicere intra
illud tempus Solem bis ortum, et occasum mutasse.
\lnr{40}Quod quamuis
prima fronte fabulosum videtur, habet tamen implicitam speciem
veri.
%
% 198
% {PDF page nr}{source page nr}{line nr}
\plnr{281}{198}{1}Nam in una magna periodo Sol mutat sedem semel in mensibus
Aegyptiacis, ut qui principio in Thoth Solstitium ingrederetur,
post 730 annos in brumam incideret in aliqua parte eius mensis.
\lnr{3}Sed
hoc non fuerit occasum et orientem mutari.
\lnr{5}Missa igitur illa mendacia et somnia
Aegyptiorum faciamus.
\lnr{6}Necesse est periodum
illorum definere in eundem diem Iulianum,
unde profecta erat, non autem in
ortum Caniculae.
\lnr{9}Et contra principium
periodi fuisse ab ortu Caniculae, quandoquidem
quadrantes in ortu Caniculae intercalabantur,
et ab eodem ortu putabantur.
\lnr{13}Quamuis, ut diximus, anni sideri ratio
hoc non patitur.
\lnr{14}Ex his apparet, in
illa magna periodo quatuor annos pro
una die maximi anni Canicularis accipi,
et triginta eiusmodi dies, sive annos Caniculares,
esse mensem illius maximi anni, qui constaret annis aequabilibus
120.
\lnr{19}Sed de his amplius in anno Persico disputabitur.
\lnr{19}Menses autem Aegyptiorum, cum characteribus suis in hunc laterculum
coniecimus.
% Table
%
\begin{table}[tb]
  %%% Liber III p198, PDF 281
%%
%%% Count out columns for fixed-width source font
% 000000011111111112222222222333333333344444444445555555555666666666677777777778
% 345678901234567890123456789012345678901234567890123456789012345678901234567890
%
\begin{tabnums} % Select monospaced numbers
%% Select a general font size (uncomment one from the list)
%\tiny
%\scriptsize
%\footnotesize
%\small
\normalsize
%% Center the whole table left-right
\centering
%% Modify separation between columns
%\setlength{\tabcolsep}{2.0pt}
%% Modify distance between rows
\renewcommand{\arraystretch}{1.000} % Tuned to eliminate Underfull \vbox
% Just lucky? No stretch required :-)
%% Size of header text
\newcommand{\hts}{\scriptsize}
%% Width of a column
\newcommand{\cwd}{4em}
%
\newcommand{\da}{\scriptsize{†}}
%%
\begin{tabular}{@{} l r r @{}}
\toprule
  \ch{Aegyptiorum}{Menses Aegyptiorum} &
  \ch{\hts{Dies men-}}{\hts{Dies mensium collecti}} &
  \ch{\hts{Character}}{\hts{Character mensium}}
\\
\midrule
\textgreek{θώθ}[?]
 &  30 & 0 \\
\textgreek{παοφί}[?]
 &  60 & 2 \\
\textgreek{ἀθύρ}[?]
 &  90 & 4 \\
\textgreek{χοιάκ}[?]
 & 120 & 6 \\
\textgreek{τυβί}[?]
 & 150 & 1 \\
\textgreek{μεχείρ}[?]
 & 180 & 3 \\
\textgreek{φαμενώθ}[?]
 & 210 & 5 \\
\textgreek{φαρμουθί}[?]
 & 240 & 7 \\
\textgreek{παχών}[?]
 & 270 & 2 \\
\textgreek{παὒνί}[?]
 & 300 & 4 \\
\textgreek{ἐπιφί}[?]
 & 330 & 6 \\
\textgreek{μεσορί}[?]
 & 360 & 1 \\
\textgreek{ἐπαγόμεναι}[?]
 & 365 & 3 \\
\bottomrule
\end{tabular}
%
\caption{Menses Aegyptiorum}
\label{tab:p198}
%
\end{tabnums}

\end{table}
%
%====
\section{De Annis Nabonassari Aegyptiacis}
% Capitalisation follows ToC
%
\lnr{22}Veteres Aegyptii innovarunt intervalla annorum suorum,
quoties a novis regibus et victoribus legem acciepiebant.
\lnr{23}Itaque
eos id saepe factitasse dubium non est.
\lnr{24}Antiquissimae eorum
epochae, quae quidem ad memoriam nostram pervenerunt,
sunt hae: Nabonassari, Philippi, vel mortis Alexandri, et Philadelphi.
\lnr{27}Mox anno vago fixo accepta est epocha Augusti Actiaca: et
postrema omnium Diocletiani, quam etiamnum hodie retinent
Aegyptii Christiani \textarabic{}[Arabic] Elkupt.
\lnr{29}Primus Thoth Nabonassari feria
quarta, Februarii 26, anno periodi Iulianae 3967.
\lnr{30}Ideo cum
annos et dies Iulianos in Aegyptiacos Nabonassari rediges, semper
aufer 56 de diebus, et bisextis Iulianis.
\lnr{32}Exemplum: Annus vulgaris
Christi, scilicet a natali, est 4713 periodi Iulianae.
\lnr{33}Deductis
3967 de 4713, remanent 747 anni Iuliani, a fine Decembris anni
3967, ad finem Decembris anni 4713: set ratione epochae Nabonassari,
a 26 Februarii 3967, ad 26 Februarii 4713.
\lnr{36}Detractis
igitur 56 diebus, remanent anni solidi Iuliani 746, dies 309
Bisexta Iuliana annorum 746, id est dies 186, componantur cum
diebus 309.
\lnr{2}Fiunt dies 495, id est annus unus Aegyptiacus cum diebus
130.
%
% 199
% {PDF page nr}{source page nr}{line nr}
\plnr{282}{199}{3}Ergo a primo Thoth Nabonassari, ad finem anni 4713 periodi
Iulianae, sunt anni Aegyptiaci absoluti 747: diesque 130 praeterea
de anno 748 inchoato.
\lnr{5}Deduc 130 de 365, nempe de anno
integro.
\lnr{6}Remanent dies 235 a Kal. Ianuarii definentes in 23 Augusti
inclusive.
\lnr{7}Ergo Thoth 748 Nabonassari incidit in illud tempus.
\lnr{8}Semper adiice 3 ad annos propositus Nabonassari.
\lnr{8}Habebis feriam
Thoth, abiectis scilicet omnibus septenariis.
\lnr{9}Ergo Thoth 748
caepit feria secunda, cyclo Solis \rnum{ix}, Augusti 23.
\lnr{10}Proinde a Thoth
Nabonassari, ad finem Decembris anni 4713 periodi Iulianae, anni
sunt Aegyptiaci, ut diximus, absoluti 747, dies 130: qui fiunt
simul dies 272786, ut volunt quoque Alfonsini.
\lnr{13}Rursus ut annos
Aegyptiacos in Iulianos convertas, bisexta, quae tot annis Iulianis
competunt, quot anni Aegyptiaci propositi sunt, deducantur de annis
et diebus propositis.
\lnr{16}Reliquum erunt anni et dies Iuliani.
\lnr{16}Exemplum:
Proponantur anni Aegyptiaci 747 absoluti, dies 130, in annos
Iulianos et dies redigendi.
\lnr{18}Bisexta, quae competunt annis 747,
id est dies 186, deducantur de annis Aegyptiacis 747, diebus 130:
vel de annis 746, diebus 495.
\lnr{20}Remanent anni Iuliani absoluti 746,
dies 309, ut antea.
\lnr{21}Rursus, cum solos annos Nabonassari investigas,
ut locum Thoth in anno Iuliano invenias, quadrantem annorum
Nabonassari abiice a 56, siquidem quadrans minor sit.
\lnr{23}Reliquum
sunt dies Iuliani a Kal. Ianuarii.
\lnr{24}Volo scire locum neomeniae Thoth
in anno Nabonassari 120.
\lnr{25}Quadrans 30 deductus de 56 relinquit
27 Ianuarii, diem ultimam anni 119.
\lnr{26}Ergo 27 Ianuarii erit neomenia
Thoth.
\lnr{27}Si quadrans excesserit 56, aufer 56 a quadrante.
\lnr{27}Reliquum
sunt dies retro numerandi ab ultima Decembris: vel, quod
idem est, deducendi de quantitate anni Iuliani 365.
\lnr{29}Relinquetur ultima
dies anni praeteriti.
\lnr{30}Alexander decessit anno Nabonassari 425.
\lnr{31}Nam a Thoth Nabonassari, ad Thoth Philippi, sive mortis Alexandri,
Ptolomaeus ponit intervallum, annos 424 praecisos.
\lnr{32}Quadrans
106, deductione facta, relinquit 50.
\lnr{33}Ea porro 50 a 365
diebus Iulianis detracta relinquunt ultimam diem anni 424
in \rnum{xi} Novembris.
\lnr{35}Itaque Thoth annorum Philippi debetur
duodecimae Novembris, feria prima.
\lnr{36}Rursus annis Nabonassari
adiice 18,
% Newline? No; 1598 ed (p190.B) has comma
siquidem non excesserint 228.
\lnr{37}Reliquum per 28
divisum relinquit cyclum Solis Iulianum.
\lnr{38}Si excedunt 228, adiice
17.
\lnr{39}Denique si excedunt 1688, adiice 16.
\lnr{39}Sed nequis error
incautis iuvenibus obrepat, praestat Laterculum characteris omnium
mensium in annis Nabonassari proponere cum cyclo Solis
Iuliano.
%
% 200
% {PDF page nr}{source page nr}{line nr}
\plnr{283}{200}{1}Eius Laterculi exemplar infra subiecimus.
% Table "Laterculus characteris mensium in annis Nabonassari"
%
\begin{table}[tbp]
  %%% Liber III p200
%% Layout based on Liber II p89
% !TEX root = ../../test-table.tex
%%
%%% Count out columns for fixed-width source font
% 000000011111111112222222222333333333344444444445555555555666666666677777777778
% 345678901234567890123456789012345678901234567890123456789012345678901234567890
%
\begin{tabnums} % Select monospaced numbers
%\tiny
%\scriptsize
%\footnotesize
%\small
\normalsize
%% Center the whole table left-right
\centering
%% Modify separation between columns
\setlength{\tabcolsep}{3.0pt}
%% Modify distance between rows
\renewcommand{\arraystretch}{1.0}
%
%% Define a smaller dagger (unfortunalely tiny is already the smallest)
\newcommand{\da}{{\tiny †}}
%% The angle with which to slant
\newcommand{\ang}{75}
%% Header text size: slanted text
\newcommand{\hsb}[1]{\small{#1}}
%% Header text size: bottom row
\newcommand{\hsa}[1]{\scriptsize{#1}}
%% Width of a column
\newcommand{\cwd}{1.0em}
%%
%%
\begin{tabular}[c]{@{} r  c c c c c c c c c c c c c  c c c c @{}}
%%
\toprule
%% Table title
\multicolumn{18}{c}{\Large\textsc{Laterculus Characteris Mensium in Annis}}\\
\multicolumn{18}{c}{\large\textsc{Nabonassari, per Cyclum Solis Iulianum}}\\
\toprule
%% Table header
\hsa{\ch{Cyclus sol-}{Cyclus so\-lis Na\-bo\-nas\-sa\-ri}} &

\hsb{\parbox[b]{\cwd}{\begin{rotate}{\ang}\textgreek{Θώθ}\end{rotate}}} &
\hsb{\parbox[b]{\cwd}{\begin{rotate}{\ang}\textgreek{Παωφί}\end{rotate}}} &
\hsb{\parbox[b]{\cwd}{\begin{rotate}{\ang}\textgreek{Αθύρ}\end{rotate}}} &

\hsb{\parbox[b]{\cwd}{\begin{rotate}{\ang}\textgreek{Χοιάκ}\end{rotate}}} &
\hsb{\parbox[b]{\cwd}{\begin{rotate}{\ang}\textgreek{Τυβί}\end{rotate}}} &
\hsb{\parbox[b]{\cwd}{\begin{rotate}{\ang}\textgreek{Μεχείρ}\end{rotate}}} &

\hsb{\parbox[b]{\cwd}{\begin{rotate}{\ang}\textgreek{Φαμενώθ}\end{rotate}}} &
\hsb{\parbox[b]{\cwd}{\begin{rotate}{\ang}\textgreek{Φαρμουθί}\end{rotate}}} &
\hsb{\parbox[b]{\cwd}{\begin{rotate}{\ang}\textgreek{Παχών}\end{rotate}}} &

\hsb{\parbox[b]{\cwd}{\begin{rotate}{\ang}\textgreek{Παὒνί}\end{rotate}}} &
\hsb{\parbox[b]{\cwd}{\begin{rotate}{\ang}\textgreek{Επιφί}\end{rotate}}} &
\hsb{\parbox[b]{\cwd}{\begin{rotate}{\ang}\textgreek{Μεσορί}\end{rotate}}} &
\hsb{\parbox[b]{\cwd}{\begin{rotate}{\ang}\textgreek{Επαγόμεναι}\end{rotate}}} &

\hsa{\ch{Character}{Character cycli Solis Iuliani}} &
\hsa{\ch{Iulianus}{Cyclus Solis Iulianus intra 228}} &
\hsa{\ch{Iulianus}{Cyclus Solis Iulianus intra 1688}} &
\hsa{\ch{Iulianus}{\medskip Cyclus Solis Iulianus intra 3148}} 
% Force some extra room above this header so that the slanted headers
% don't run over into the table title.
\\
%% Table body
\midrule
%%
 1~ &
4 & 6 & 1 & 3 & 5 & 7 & 2 & 4 & 6 & 1 & 3 & 5 & 7 &
 E  & 19 & 18 & 17 \\
%
 2~ &
5 & 7 & 2 & 4 & 6 & 1 & 3 & 5 & 7 & 2 & 4 & 6 & 1 &
 D  & 20 & 19 & 18 \\
%
 3~ &
6 & 1 & 3 & 5 & 7 & 2 & 4 & 6 & 1 & 3 & 5 & 7 & 2 &
C B & 21 & 20 & 19 \\
%
 4~ &
7 & 2 & 4 & 6 & 1 & 3 & 5 & 7 & 2 & 4 & 6 & 1 & 3 &
 A  & 22 & 21 & 20 \\
%
 5~ &
1 & 3 & 5 & 7 & 2 & 4 & 6 & 1 & 3 & 5 & 7 & 2 & 4 &
 G  & 23 & 22 & 21 \\
%
 6~ &
2 & 4 & 6 & 1 & 3 & 5 & 7 & 2 & 4 & 6 & 1 & 3 & 5 &
 F  & 24 & 23 & 22 \\
%
 7~ &
3 & 5 & 7 & 2 & 4 & 6 & 1 & 3 & 5 & 7 & 2 & 4 & 6 &
E D & 25 & 24 & 23 \\
%
 8~ &
4 & 6 & 1 & 3 & 5 & 7 & 2 & 4 & 6 & 1 & 3 & 5 & 7 &
 C  & 26 & 25 & 24 \\
%
 9~ &
5 & 7 & 2 & 4 & 6 & 1 & 3 & 5 & 7 & 2 & 4 & 6 & 1 &
 B  & 27 & 26 & 25 \\
%
10~ &
6 & 1 & 3 & 5 & 7 & 2 & 4 & 6 & 1 & 3 & 5 & 7 & 2 &
 A  & 28 & 27 & 26 \\
%
11~ &
7 & 2 & 4 & 6 & 1 & 3 & 5 & 7 & 2 & 4 & 6 & 1 & 3 &
G F & ~1 & 28 & 27 \\
%
12~ &
1 & 3 & 5 & 7 & 2 & 4 & 6 & 1 & 3 & 5 & 7 & 2 & 4 &
 E  & ~2 & ~1 & 28 \\
%
13~ &
2 & 4 & 6 & 1 & 3 & 5 & 7 & 2 & 4 & 6 & 1 & 3 & 5 &
 D  & ~3 & ~2 & ~1 \\
%
14~ &
3 & 5 & 7 & 2 & 4 & 6 & 1 & 3 & 5 & 7 & 2 & 4 & 6 &
 C  & ~4 & ~3 & ~2 \\
%
15~ &
4 & 6 & 1 & 3 & 5 & 7 & 2 & 4 & 6 & 1 & 3 & 5 & 7 &
B A & ~5 & ~4 & ~3 \\
%
16~ &
5 & 7 & 2 & 4 & 6 & 1 & 3 & 5 & 7 & 2 & 4 & 6 & 1 &
 G  & ~6 & ~5 & ~4 \\
%
17~ &
6 & 1 & 3 & 5 & 7 & 2 & 4 & 6 & 1 & 3 & 5 & 7 & 2 &
 F  & ~7 & ~6 & ~5 \\
%
18~ &
7 & 2 & 4 & 6 & 1 & 3 & 5 & 7 & 2 & 4 & 6 & 1 & 3 &
 E  & ~8 & ~7 & ~6 \\
%
19~ &
1 & 3 & 5 & 7 & 2 & 4 & 6 & 1 & 3 & 5 & 7 & 2 & 4 &
D C & ~9 & ~8 & ~7 \\
%
20~ &
2 & 4 & 6 & 1 & 3 & 5 & 7 & 2 & 4 & 6 & 1 & 3 & 5 &
 B  & 10 & ~9 & ~8 \\
%
21~ &
3 & 5 & 7 & 2 & 4 & 6 & 1 & 3 & 5 & 7 & 2 & 4 & 6 &
 A  & 11 & 10 & ~9 \\
%
22~ &
4 & 6 & 1 & 3 & 5 & 7 & 2 & 4 & 6 & 1 & 3 & 5 & 7 &
 G  & 12 & 11 & 10 \\
%
23~ &
5 & 7 & 2 & 4 & 6 & 1 & 3 & 5 & 7 & 2 & 4 & 6 & 1 &
F E & 13 & 12 & 11 \\
%
24~ &
6 & 1 & 3 & 5 & 7 & 2 & 4 & 6 & 1 & 3 & 5 & 7 & 2 &
 D  & 14 & 13 & 12 \\
%
25~ &
7 & 2 & 4 & 6 & 1 & 3 & 5 & 7 & 2 & 4 & 6 & 1 & 3 &
 C  & 15 & 14 & 13 \\
%
26~ &
1 & 3 & 5 & 7 & 2 & 4 & 6 & 1 & 3 & 5 & 7 & 2 & 4 &
 B  & 16 & 15 & 14 \\
%
27~ &
2 & 4 & 6 & 1 & 3 & 5 & 7 & 2 & 4 & 6 & 1 & 3 & 5 &
A G & 17 & 16 & 15 \\
%
28~ &
3 & 5 & 7 & 2 & 4 & 6 & 1 & 3 & 5 & 7 & 2 & 4 & 6 &
 F  & 18 & 17 & 16 \\
%
\bottomrule
%%
\end{tabular}
%
\caption{Characteris Mensium in Annis Nabonassari}
\label{tab:p200}
%
\end{tabnums}

\end{table}
%
\lnr{1}Divisis annis
Nabonassari per septem, adiectis tribus, diximus relinqui characterem
Thoth Nabonassari.
\lnr{3}Qui characteri sequentium mensium adiectus,
quos in capite de anno Aegyptiaco posuimus, indicabit feriam omnium
mensium illius anni.
\lnr{5}Sed si vis characterem illum cum cyclo
Solis Iuliano comparare, tunc ingressus faciendus in hunc laterculum.
\lnr{7}Itaque considerabis, an summa annorum propositorum sit intra
228, an vero excedat, et cyclum Solis Iulianum ei congruentem
pro neomenia Thoth assumes.
\lnr{9}Annos igitur Nabonassari per 28 divide.
\lnr{10}Reliquum est annus cycli Solis Nabonassari in latere sinistro.
\lnr{11}E regione illius habes omnes characteres omnium mensium anni propositi:
et a latere dextro cyclum Solis Iulianum ei congruentem.
%
% 201
% {PDF page nr}{source page nr}{line nr}
\plnr{284}{201}{1}Exemplum.
\lnr{1}Volo scire characterem Thoth anni 120 a Nabonassaro,
et cyclum Romanum illi competentem.
\lnr{2}Abiectis 28, remanent 8,
qui est octavus annus cycli Solis Nabonassari.
\lnr{3}A latere sinistro habes
8 annum cycli Nabonassari, feriam Thoth 4, Paophi 6, et cetera.
\lnr{4}Et quia
anni 120 sunt intra 228, in latere dextro cyclus Iulianus ei respondens
est 26.
\lnr{6}Litera dominicalis \textsc{C}.
\lnr{6}Feria ergo 4 erit \textsc{F}. % Newline? Yes. 1598: capital P on Proinde
\lnr{6}Proinde Thoth coepit
litera F, 27 Ianuarii, ut antea explicavimus.
\lnr{7}Dici non potest, quam tutus
et expeditus sit usus huius Laterculi illis, qui Ptolemaeum legunt.
\lnr{9}Quod si numerus annorum Nabonassari excesserit 228, tunc secunda
columna cycli Iuliani assumenda est, ut eius titulus indicat.
\lnr{11}Verbi gratia: in anno 232 erit idem cyclus Nabonassari \rnum{viii}.
\lnr{11}In secunda columna ei respondet cyclus 25 Iulianus.
\lnr{12}Propterea litera Dominicalis erit \textsc{E D}.
\lnr{13}Feria quarta, quae in anno 120 erat in litera \textsc{F}, tunc erit
in litera \textsc{G}, Decembris 30.
\lnr{14}Sed quia reliqui menses Nabonassari pertinent
ad cyclum Solis Iulianum 26, diligenter cavere debent adolescentes,
ne putent semper eundem cyclum per totum annum usurpari.
\lnr{17}Hoc tantum accidit in quatuor annis, nempe 224, 225, 226, 227, in
prima columna.
\lnr{18}In secunda, annis 1684, 1685, 1686, 1687.
\lnr{18}In
tertia culumna, in annis 3144, 3145, 3146, 3147.
\lnr{19}In omnibus reliquis
annis semper bini cycli Iuliani vindicant sibi annum Nabonassari.
\lnr{21}Quare is cyclus Iulianus, qui e regione cycli Nabonassari respondet,
is, inquam, est cyclus Thoth, et si qui sunt alii menses, intra illum annum
cycli.
\lnr{23}Nam constat non totum annum Nabonassari uno cyclo
Iuliani attribui.
\lnr{24}Censorinus scribebat aureolum libellum suum de
die Natali anno Christi Dionysiano 238, cyclo Solis \rnum{xxiii}, Lunae
\rnum{xi}: quod cognoscimus ex anno Iphiti 1014, quo ineunte a diebus
aestivis ea scribebat eximius ille et doctissimus temporum et antiquitatis
vindex.
\lnr{28}Is igitur scribit, eo anno Thoth Nabonassari noningentesimum
octagesimum sextum incurrisse in a.d. \rnum{vii} Kalen. Iulias:
hoc est in \rnum{xxv} Iunii.
\lnr{30}Abiectis 28 de 986, remanet annus cycli Solis
Nabonassari sextus.
\lnr{31}In secunda columna sinistra est cyclus Solis
23, ut proposuimus.
\lnr{32}Feria Thoth secunda.
\lnr{32}Litera Dominicalis cycli
23, est \textsc{G}.
\lnr{33}Ergo secunda feria \textsc{A}.
\lnr{33}Qui est character \rnum{xxv} Iunii.
\lnr{33}Praeterea
quadrans annorum 986, nempe 246, deductis 56, relinquit
190.
\lnr{35}Quae et ipsa de anni Iuliani quantitate deducta relinquunt 175,
ultimam diem anni 985 in 24 Iunii.
\lnr{36}Ergo Thoth 986 in 25.
\lnr{36}Quod
autem scribitur apud eundem Censorinum, Ulpio et Brutio Praesente
\textsc{coss.} anno Nabonassari 886, neomeniam % neomeniā
 Thoth incurrisse in \rnum{xii} Kal.
Augusti, mendum est librarii, non error Censorini.
\lnr{39}Centum enim quadrantes % quadrātes
Aegyptii dant annos Caniculares \rnum{xxv}. % Newline? (same in 1598 ed.)
\lnr{40}Hoc est, dies \rnum{xxv}.
\lnr{40}Additis
25 ad 25 Iunii, incidis in 20 Iulii.
\lnr{41}Periclitemur ex methodo nostra.
\lnr{41}Quadrans
annorum 886 est 165, detractis 56.
%
% 202
% {PDF page nr}{source page nr}{line nr}
\plnr{285}{202}{1}Rursus 165 de 365 detracta
relinquunt 200, nempe 19 Iulii, ultimam diem anni 885.
\lnr{2}Quare
Thoth 886 coepit die Iulii 20.
\lnr{3}Abiectis 28 de 886 relinquitur cyclus
Solis Nabonassari 18.
\lnr{4}Qui in latere sinistro indicat feriam Thoth 7: in
latere dextro, secunda columna cyclum Solis Iulianum \rnum{vii}.
\lnr{5}Feria igitur
septima \textsc{E}, qui est character \rnum{xx} Iulii.
\lnr{6}Sed expeditius, et dicto citius,
addita tria ad annum Nabonassari dabunt feriam neomeniae Thoth,
abiectis omnibus septenariis.
\lnr{8}Nunc ad methodum Lunae in mensibus
Nabonassari veniamus.
\lnr{9}Cui rei maiori ex parte satisfecimus,
cum generaliter \textgreek{εἰκοσιπενταετηρίδα}[?] anna aequabilis explicavimus.
\lnr{10}Ptolemaeus
libro \rnum{v} statuit novilunium primi Thoth Nabonassari
% Odd way of writing: only first group (the day) gets a "Number" bar
% the other two (scrupulis diurnis) do not.
 \textgreek{\gnum{κδ}. μδ'. ιζ''}. % 1598 ed.: iota-zeta
% 20+4. 40+4'. 10+7''
% Newline? Same in 1598 ed.
\lnr{12}Hoc est, Thoth \rnum{xxiiii}. % Newline? No. Sentence runs on.
 scrupulis diurnis 44', 17'', quae ad
rationem Chaldaicam redacta fiunt 24 dies Thoth, horae 17, 770.
\lnr{14}Terminus igitur Thoth primi anni \textgreek{εἰκοσιπενταετηρίδος}[?]
 Nabonassari est
\rnum{xxiiii}.
\lnr{15}Iam diximus initio, epactas descendere; contra Terminos ascendere.
\lnr{16}Si \rnum{xxiiii} est primi anni Terminus, \rnum{xiiii},
 qui proxime antecedit,
erit Terminus secundi, et sic deinceps ascendendo.
\lnr{17}Quibus
terminis per ordinem et contextum suum digestis, facile situs embolismi
in periodo annorum Nabonassari deprehendetur.
\lnr{19}Is enim annus
est \textgreek{ἐμβολιμαῖοσ}[?], quem sequitur Terminus maior.
\lnr{20}Ut, verbi gratia,
annus tertius habet \rnum{iii} pro Termino, quem sequitur \rnum{xxii}, Terminus
nempe maior antecedente.
\lnr{22}Ergo \rnum{iii} est \textgreek{ἐμβολιμαῖος}[?].
\lnr{22}Quibus
animadversis, periodus Nabonassari, et eius embolismi nihil differunt
ab ordine, situ et progressu enneadecaeteridis: quatenus septem
priores embolismi sunt in annis tertio, sexto, octavo, undecimo,
quartodecimo, decimoseptimo, et nonodecimo, ut in Enneadecaeteride. % newline?
% Same in the 1598 edition: Word, period, no space, no capital on next word.
\lnr{27}Reliqui duo sunt principium Enneadecaeteridis.
\lnr{27}Nam embolismus
vicesimus secundus est primus embolismus enneadecaeteridis
in anno tertio.
\lnr{29}Embolismus autem ultimus, est secundus in anno
sexto enneadecaeteridis.
\lnr{30}Quare nova periodus instituenda fuit, cuius
annus primus habet \textgreek{ἐποχὴν}[?] primi novilunii Nabonassari.
\lnr{31}Reliqui anni
sunt vere Chaldaici, quandoquidem pertinent ad enneadecaeterida
Chaldaicam: neque alio differunt, nisi quod epochen % Sic; same in 1598 ed
 Nabonassari
implicitam habent.
%
% Two tables p203
\begin{table}[p]
  % define table height
  \newcommand{\tabh}{\textheight}
%  \setlength{\tabcolsep}{0.0ex}
  \centering
  \begin{tabular}{c @{\hspace{0.06\textwidth}} c}
%  \resizebox{0.45\textwidth}{!}{%
    \begin{minipage}[][\tabh][t]{0.43\textwidth}
      %%% Liber 3 p203a, PDF 286
%% One column version
%% Layout based on Liber 3 p193b
% !TEX root = ../../test-table.tex
%%
%%% Count out columns for fixed-width source font
% 000000011111111112222222222333333333344444444445555555555666666666677777777778
% 345678901234567890123456789012345678901234567890123456789012345678901234567890
%
\begin{tabnums} % Select monospaced numbers
%% Select a general font size (uncomment one from the list)
%\tiny
%\scriptsize
\footnotesize
%\small
%\normalsize
%% Center the whole table left-right
\centering
%% Modify separation between columns
\setlength{\tabcolsep}{1.0ex}
%% Modify distance between rows
\renewcommand{\arraystretch}{0.98}
%
%% Width of a column
\newcommand{\cwd}{3.2em}
%% Define reference symbols
\newcommand{\da}{{\tiny †}}
\newcommand{\db}{{\scriptsize o}}
%% The angle with which to slant
\newcommand{\ang}{90}
%% Header text size: row above bottom row
\newcommand{\hsb}[1]{\footnotesize{#1}}
%% Header text size: bottom row
\newcommand{\hsa}[1]{\tiny{#1}}
%% Generate the column headers
%
\newcommand{\hdrB}{%
  ~ &
  \multicolumn{3}{c}{\hsb{Pars anterior.}} &
%  &
  \multicolumn{4}{c}{\hsb{Pars posterior.}}  
}
%
\newcommand{\hdrA}{%
  \ch{\hsa{Anni per}}{\hsa{Anni per cyclos collecti.}} &
  \ch{\hsa{Feria.}}{\hsa{Feria.}}&
  \ch{\hsa{Hor.}}{\hsa{Hor.}} &
  \ch{\hsa{Scrup.}}{\hsa{Scrup.}} &
%  &
  \ch{\hsa{Dies.}}{\hsa{Dies.}} & & % leave room for correction mark
  \ch{\hsa{Hor.}}{\hsa{Hor.}} &
  \ch{\hsa{Scrup.}}{\hsa{Scrup.}}
}
%
\newcommand{\hdrs}{%
\hdrB \\
\cmidrule(lr){2-4} \cmidrule(lr){5-8}
\hdrA \\
}
%
\begin{tabular}[c]{@{} r rrr r@{}lrr @{}}
\toprule
\multicolumn{8}{c}{\normalsize\textsc{Anni per Cyclos Collecti}} \\
\toprule
\hdrs % Column headers from the above definition
\midrule
%%
  25 & 3 & 22 &  957 & 0&&  1 &  123 \\
  50 & 7 & 21 &  834 & 0&&  2 &  246 \\
  75 & 4 & 20 &  711 & 0&&  3 &  369 \\
 100 & 1 & 19 &  588 & 0&&  4 &  492 \\
 125 & 5 & 18 &  465 & 0&&  5 &  615 \\
 150 & 2 & 17 &  342 & 0&&  6 &  738 \\
 175 & 6 & 16 &  219 & 0&&  7 &  861 \\
 200 & 3 & 15 &   96 & 0&&  8 &  984 \\
 225 & 7 & 13 & 1053 & 0&& 10 &   27 \\
 250 & 4 & 12 &  930 & 0&& 11 &  150 \\
 275 & 1 & 11 &  807 & 0&& 12 &  273 \\
 300 & 5 & 10 &  684 & 0&& 13 &  396 \\
 325 & 2 &  9 &  561 & 0&& 14 &  519 \\
 350 & 6 &  8 &  438 & 0&& 15 &  642 \\
 375 & 3 &  7 &  315 & 0&& 16 &  765 \\
 400 & 7 &  6 &  192 & 0&& 17 &  888 \\
 425 & 4 &  5 &   96 & 0&& 18 & 1011 \\
 450 & 1 &  3 & 1026 & 0&& 20 &   54 \\
 475 & 5 &  2 &  903 & 0&& 21 &  177 \\
 500 & 2 &  1 &  780 & 0&& 22 &  300 \\
 
 525 & 6 &  0 &  657 & 0&& 23 &  423 \\
 550 & 2 & 23 &  534 & 1&&  0 &  546 \\
 575 & 6 & 22 &  411 & 1&&  1 &  669 \\
 600 & 3 & 21 &  288 & 1&&  2 &  792 \\
 625 & 1 & 20 &  165 & 1&&  3 &  915 \\
 650 & 4 & 19 &   42 & 1&&  4 & 1038 \\
 675 & 1 & 17 &  999 & 1&&  6 &   81 \\
 700 & 5 & 16 &  876 & 1&&  7 &  204 \\
 725 & 2 & 15 &  753 & 1&&  8 &  327 \\
 750 & 6 & 14 &  630 & 1&&  9 &  450 \\
 775 & 3 & 13 &  507 & 1&& 10 &  573 \\
 800 & 7 & 12 &  384 & 1&& 11 &  696 \\
 825 & 4 & 11 &  261 & 1&& 12 &  819 \\
 850 & 1 & 10 &  138 & 1&& 13 &  942 \\
 875 & 5 &  9 &   15 & 1&& 14 & 1065 \\
 900 & 2 &  7 &  972 & 1&& 16 &  108 \\
 925 & 6 &  6 &  849 & 1&& 17 &  231 \\
 950 & 3 &  5 &  726 & 1&& 18 &  354 \\
 975 & 7 &  4 &  603 & 1&& 19 &  477 \\
1000 & 4 &  3 &  480 & 1&& 20 &  600 \\
2000 & 1 &  6 &  960 & 3&\super{*}
 & 17 &  120 \\
3000 & 5 & 10 &  360 & 5&\super{*}
 & 13 &  720 \\
\bottomrule
\addlinespace[5pt]
\multicolumn{3}{r}{\footnotesize\super{*}In originalis: 1}
% 1598 edition has this too.
% The corrected values were calculated by following the pattern, where the total
% number of scrupules increases by 48120 for every 1000 anni, and where horae
% are 1080 scrupules, and dies are 24 horae.
% The values in the Hor. and Scrup. columns for rows 2000 and 3000 *do* match
% this pattern in the original.
% This might be an obvious oversight by the writer or the typesetter, simply
% continuing the string of '1' values in that column.
% However, if the "Dies" column is supposed to have modulo 2 values, then
% the value of 1 would be correct after all.
\end{tabular}
\caption{Anni per Cyclos Collecti}
\label{tab:p203a}
\end{tabnums}

    \end{minipage}
%  }
&
%  \resizebox{0.45\textwidth}{!}{%
    \begin{minipage}[][\tabh][t]{0.45\textwidth}
      %%% Liber 3 p203b, PDF 286
%% Layout based on Liber 3 p203a
% !TEX root = ../../test-table.tex
%%
%%% Count out columns for fixed-width source font
% 000000011111111112222222222333333333344444444445555555555666666666677777777778
% 345678901234567890123456789012345678901234567890123456789012345678901234567890
%
\begin{tabnums} % Select monospaced numbers
%% Select a general font size (uncomment one from the list)
%\tiny
%\scriptsize
\footnotesize
%\small
%\normalsize
%% Center the whole table left-right
\centering
%% Modify separation between columns
\setlength{\tabcolsep}{1.0ex}
%% Modify distance between rows
\renewcommand{\arraystretch}{1.072} % Tuned to page length (20 lines)
%
%% Width of a column
\newcommand{\cwd}{3.2em}
%% Define reference symbols
\newcommand{\da}{{\tiny †}}
\newcommand{\db}{{\scriptsize o}}
%% The angle with which to slant
\newcommand{\ang}{90}
%% Header text size: row above bottom row
\newcommand{\hsb}[1]{\footnotesize{#1}}
%% Header text size: bottom row
\newcommand{\hsa}[1]{\tiny{#1}}
%% Generate the column headers
%
\newcommand{\hdrB}{%
  ~ & ~ &
  \multicolumn{3}{c}{\hsb{Pars anterior}} &
%  &
  \multicolumn{3}{c}{\hsb{Pars posterior}}  
}
%
\newcommand{\hdrA}{%
  \ch{\hsa{Men-}}{\hsa{Men\-ses}} &
  \ch{\hsa{men-}}{\hsa{Dies mensium}} &
  \ch{\hsa{Feria}}{\hsa{Feria}}&
  \ch{\hsa{Hor.}}{\hsa{Hor.}} &
  \ch{\hsa{Scrup.}}{\hsa{Scrup.}} &
%  &
  \ch{\hsa{Dies}}{\hsa{Dies}} &
  \ch{\hsa{Hor.}}{\hsa{Hor.}} &
  \ch{\hsa{Scrup.}}{\hsa{Scrup.}}
}
%
\newcommand{\hdrs}{%
\hdrB \\
\cmidrule(lr){3-5} \cmidrule(lr){6-8}
\hdrA \\
}
%
\begin{tabular}[c]{@{} r r rrr rrr @{}}
\toprule
\multicolumn{8}{c}{\large\textsc{Tabella Oppositionis}} \\
\multicolumn{8}{c}{\normalsize\textsc{Luminarium per Menses}} \\
\toprule
\hdrs % Column headers from the above definition
\midrule
%%
  1 &  14 &  7 & 18 &  396 & 15 &  5 &  684 \\
  2 &  44 &  2 &  7 &  109 & 15 & 16 &  971 \\
  3 &  73 &  3 & 19 &  902 & 16 &  4 &  178 \\
  4 & 103 &  5 &  8 &  615 & 16 & 15 &  465 \\
  5 & 132 &  6 & 21 &  328 & 17 &  2 &  752 \\
  6 & 162 &  1 & 10 &   41 & 17 & 13 & 1039 \\
  7 & 191 &  2 & 22 &  834 & 18 &  1 &  246 \\
  8 & 221 &  4 & 11 &  547 & 18 & 12 &  533 \\
  9 & 251 &  6 &  0 &  260 & 18 & 23 &  820 \\
 10 & 280 &  7 & 12 & 1053 & 19 & 11 &   27 \\
 11 & 310 &  2 &  1 &  766 & 19 & 22 &  314 \\
 12 & 339 &  3 & 14 &  479 & 20 &  9 &  601 \\
\bottomrule
\addlinespace[5pt]
\end{tabular}
\caption{Oppositionis Luminarium per Menses}
\label{tab:p203b}
\end{tabnums}

    \end{minipage}
%  }
\\
%    \addlinespace[0.85in] % force a lot of space to make room for the captions
  \end{tabular}
\end{table}
%
% Table "Cyclus Nabonassari per annos expansos" p204
\begin{table}[htbp]
  %%% Liber 3 p204, PDF 287
%% Layout based on Liber 3 p203b
% !TEX root = ../../test-table.tex
%%
%%% Count out columns for fixed-width source font
% 000000011111111112222222222333333333344444444445555555555666666666677777777778
% 345678901234567890123456789012345678901234567890123456789012345678901234567890
%
\begin{tabnums} % Select monospaced numbers
%% Select a general font size (uncomment one from the list)
%\tiny
%\scriptsize
%\footnotesize
%\small
\normalsize
%% Center the whole table left-right
\centering
%% Modify separation between columns
\setlength{\tabcolsep}{1.0ex}
%% Modify distance between rows
\renewcommand{\arraystretch}{1.1} % Not Tuned
%
%% Width of a column
\newcommand{\cwd}{3.2em}
%% Define reference symbols
%\newcommand{\da}{{\scriptsize Emb.\hspace*{2ex}}}
\newcommand{\da}{{\tiny †}}
\newcommand{\db}{{\scriptsize o}}
%% The angle with which to slant
\newcommand{\ang}{90}
%% Header text size: row above bottom row
\newcommand{\hsb}[1]{\small{#1}}
%% Header text size: bottom row
\newcommand{\hsa}[1]{\tiny{#1}}
%% Generate the column headers
%
\newcommand{\hdrB}{%
  ~ &
  \multicolumn{4}{c}{\hsb{Pars anterior}} &
%  &
  \multicolumn{3}{c}{\hsb{Pars posterior}}  
}
%
\newcommand{\hdrAa}{%
  \ch{\hsa{Nabo.}}{\hsa{Anni Nabo.}} &
  \ch{\hsa{novil.}}{\hsa{Ter\-mi\-ni no\-vil.}} &
  \ch{\hsa{Feria}}{\hsa{Feria}}&
  \ch{\hsa{Hor.}}{\hsa{Hor.}} &
  \ch{\hsa{Scrup.}}{\hsa{Scrup.}} &
%  &
  \ch{\hsa{Dies}}{\hsa{Dies}} &
  \ch{\hsa{Hor.}}{\hsa{Hor.}} &
  \ch{\hsa{Scrup.}}{\hsa{Scrup.}}
}
%
\newcommand{\hdrAb}{%
  \ch{\hsa{Nabo.}}{\hsa{Anni Nabo.}} &
  \ch{\hsa{Feria}}{\hsa{Feria}}&
  \ch{\hsa{Hor.}}{\hsa{Hor.}} &
  \ch{\hsa{Scrup.}}{\hsa{Scrup.}} &
%
  \ch{\hsa{Thoth.}}{\hsa{Dies Thoth.}} &
  \ch{XXVIII}{\hsa{Epactae}}
}
%
\newcommand{\hdrs}{%
\hdrB \\
\cmidrule(lr){2-5} \cmidrule(lr){6-8}
\hdrAa & & \hdrAb \\
}
%
\begin{tabular}[c]{@{} r r rrr rrr l @{\hspace{4ex}} r rrr r l @{}}
\toprule
\multicolumn{15}{c}{\Large\textsc{Cyclus Nabonassari per}} \\
\multicolumn{15}{c}{\large\textsc{Annos Expansos}} \\
\toprule
\hdrs % Column headers from the above definition
\midrule
%%
  1& 24& 6& 17& 770&  5&  6&  310&     &  1 & 7 & 12 &  538 & 25 & V \\
  2& 14& 4&  2& 566& 15& 21&  514&     &  2 & 4 & 21 &  334 & 15 & XV \\
  3&  3& 1& 11& 362& 26& 12&  718& \da &  3 & 2 &  6 &  130 &  4 & XXV \\
  4& 22& 7&  8& 951&  7& 15&  129&     &  4 & 1 &  3 &  719 & 23 & VII \\
  5& 11& 4& 17& 747& 18&  6&  333&     &  5 & 5 & 12 &  815 & 12 & XVIII \\
  6&  1& 2&  2& 543& 28& 21&  537& \da &  6 & 2 & 21 &  311 &  2 & XXVIII \\
  7& 20& 1&  0&  52&  9& 23& 1028&     &  7 & 1 & 18 &  896 & 21 & IX \\
  8&  9& 5&  8& 928& 20& 15&  152& \da &  8 & 6 &  3 &  692 & 10 & XX \\
  9& 28& 4&  5& 437&  1& 17&  643&     &  9 & 5 &  1 &  201 & 29 & I \\
 10& 17& 1& 15& 233& 12&  8&  847&     & 10 & 2 &  9 & 1077 & 18 & XII \\
 11&  7& 6&  0&  29& 22& 23& 1051& \da & 11 & 6 & 18 &  873 &  8 & XXII \\
 12& 25& 4& 21& 618&  4&  2&  462&     & 12 & 5 & 16 &  382 & 26 & IIII \\
 13& 15& 2&  6& 414& 14& 17&  666&     & 13 & 5 &  1 &  178 & 16 & XIIII \\
 14&  4& 6& 15& 210& 25&  8&  870& \da & 14 & 7 &  9 & 1074 &  5 & XXV \\
 15& 23& 5& 12& 799&  6& 11&  281&     & 15 & 6 &  7 &  563 & 24 & VI \\
 16& 12& 2& 21& 595& 17&  2&  485&     & 16 & 3 & 16 &  359 & 13 & XVII \\
 17&  2& 7&  6& 391& 27& 17&  689& \da & 17 & 1 &  1 &  155 &  3 & XXVIII \\
 18& 21& 6&  3& 980&  8& 20&  100&     & 18 & 6 & 22 &  744 & 22 & XVIII \\
 19& 10& 3& 12& 779& 19& 11&  304& \da & 19 & 4 &  7 &  540 & 11 & XIX \\
 20& 29& 2& 10& 285&  0& 13&  795&     & 20 & 3 &  5 &   49 & 30 & X \\
 21& 18& 6& 19&  81& 11&  4&  999&     & 21 & 7 & 13 &  925 & 19 & XI \\
 22&  8& 4&  3& 957& 21& 20&  123& \da & 22 & 4 & 22 &  721 &  9 & XXI \\
 23& 27& 3&  1& 466&  2& 22&  614&     & 23 & 3 & 20 &  730 & 28 & II \\
 24& 16& 7& 10& 262& 13& 13&  818&     & 24 & 1 &  5 &   26 & 17 & XIII \\
 25&  5& 4& 19&  58& 24&  4& 1022& \da & 25 & 5 & 13 &  898 &  6 & XXIIII \\
\bottomrule
\addlinespace[5pt]
\multicolumn{3}{r}{\footnotesize\super{\da}Emb.}
\end{tabular}
\caption{Cyclus Nabonassari per Annos Expansos}
\label{tab:p204}
\end{tabnums}

\end{table}
%
\lnr{34}Tabulam autem \textgreek{εἰκοσιπενταετηρίδος}[?] duplicem
fecimus, ut pugillar bipatens, quod vocat Ausonius.
\lnr{35}Prior pars
continet, ut diximus, annos periodi, quorum primus est mera epocha
Nabonassari, reliqui sunt anni communes, aut embolimaei, ut ordo
postulat, cum ipsa epocha impliciti.
\lnr{38}Posterior pars, quae et dextra, habet
illos annos inversos, quo nomine posteriorem partem vocavimus.
\lnr{40}Deducta enim priore parte 24, 17, 770, relinquitur posterior pars,
5, 6, 310.
\lnr{41}Porro prior pars indicat feriam novilunii, posterior diem mensis.
%
% 203
% {PDF page nr}{source page nr}{line nr}
\plnr{286}{203}{1}Utraque horas, et scrupulos.
%
% Two tables
% (included on p202)
%
\lnr{2}Hanc periodum antecedit
Tabella oppositionum
in mensibus Nabonassari. % 'mensib.' Newline? Abbriviation? If so, of what?
% 1598 edition (p194): mensibus
\lnr{4}Quam
in gratiam eorum confecimus,
qui Ptolemaeum legunt: ut sine
magno negotio et labore,
deliquia Lunaria ab antiquis
observata, quae exstant apud illum
scriptorem, reperiant.
\lnr{11}Coniunctiones autem luminarium
petendae sunt ex Tabella
mensium, quam initio huius
libri posuimus, ubi methodum
anni aequabilis tradimus.
\lnr{15}Postremo
anteposita est Tabula
per periodos expansas collecta,
quam ad 3000 annos extendimus.
\lnr{19}Hoc quoque studiosorum
laborem levabit in noviluniis,
et oppositionibus colligendis.
\lnr{21}Methodi summa haec est: Ab
annis Nabonassari propositis abiice semper minorem numerum annorum
in sinistro latere Tabulae maioris, etiam si praecise inveniuntur.
%
% 204
% {PDF page nr}{source page nr}{line nr}
\plnr{287}{204}{1}Ut si propositus sit annus Nabonassari 225: quamius is numerus
praecise in latere Tabellae invenitur, tamen proxime minor 200 abiicitur.
% Table "Cyclus Nabonassari per annos expansos"
% (included on p202)
\lnr{3}Reliqui 25 petendi ex Tabula annorum expansorum.
\lnr{3}Si vis igitur
scire novilunium Thoth illius anni Nabonassari, accipe numeros
partis anterioris annorum 200, quos a 225 abiecisti. % Newline? Abbriviation?
 nempe 3, 15,
96.
\lnr{6}Item numeros anni 25 in Tabula annorum expansorum, 4, 19,
58.
\lnr{7}Compone simul. % Newline? Abbriviation?
% 1598 edition: capital on next word, ergo newline.
\lnr{7}Prodit novilunium Thoth 1, 10, 154.
\lnr{7}Adde tria
annis Nabonassari.
\lnr{8}Abiice 7 omnia.
\lnr{8}Reliqua est feria 4, neomeniae
Thoth.
\lnr{9}Terminus novilunii \rnum{v}, e regione anni 25, indicat novilunium
confici in dicta die quinta Thoth, quae est feria prima, ut recte
collectum est.
\lnr{11}Quod si terminus non indicaret, tamen numeri partis
posterioris rem explicassent.
\lnr{12}Numerus partis posterioris annorum
200, est 0, 8, 984.
\lnr{13}Item numerus eiusdem partis e regione anni 25,
est 24, 4, 1022.
\lnr{14}Compone.
\lnr{14}Prodeunt 24, 13, 926.
\lnr{14}Abiice illa a 30.
%
% 205
% {PDF page nr}{source page nr}{line nr}
\plnr{288}{205}{1}Relinquuntur 5, 10, 154. % Newline?
 dies nempe quinta Thoth, hora 10 post
meridiem, scrup. % abbriviation
 Chald. % abbriviation
% Both same in 1598 ed.
 154.
\lnr{2}Memineris autem, scrupulos a 1080
abiici, horas a 24, feriam a 7, dies a 30.
\lnr{3}Et cum colliguntur dies
partis posterioris, semper triginta abiiciuntur in excessu tricenarii.
\lnr{4}Ut
exploratius tibi constet de fide methodi, periclitare, an eclipsin Lunae,
quae eodem anno 225 contigit, deprehendere possis.
\lnr{6}Ea confecta
est hora 10, 100 post meridiem, 17 Phamenoth, hor. % abbriviation
 1, 280
post mendiam oppositionem.
\lnr{8}Ergo media oppositio fuit hora 8, 988.
\lnr{9}Dies 17 Phamenoth est 197 dies neomenia Thoth.
\lnr{9}Auser igitur
Terminum, hoc est \rnum{v}, a 197.
\lnr{10}Relinquuntur 192.
\lnr{10}Sume oppositionem
numeri proxime minoris 191.
\lnr{11}Numerus anterior 2, 22, 834.
\lnr{12}Iunge cum novilunio Thoth 1, 10, 154.
\lnr{12}Prodit oppositio Phamenoth,
4, 8, 988, ut propositum fuit.
\lnr{13}Annus coepit feria quarta.
\lnr{13}Adde
characterem anni characteri Phamenoth ex laterculo mensium
Aegyptiacorum.
\lnr{15}Neomenia Phamenoth fuit feria 2. % Newline?
\lnr{15}Eadem feria in \rnum{xv}.
\lnr{16}Ergo \rnum{xvii} Phamenoth feria 4, ut indicat epilogismus.
\lnr{16}Quod si nescirem
feriam, tamen per numeros posteriores id collegissem.
\lnr{17}Numerus
posterior oppositionis 191, est 18, 1, 246.
\lnr{18}Iunge cum 24, 13, 926
novilunii Thoth.
\lnr{19}Abiice 30, (quod semper faciendum, si opus est.) % no period in original
% 1598 ed. has period before the closing parenthesis.
\lnr{20}Exeunt, 12, 15, 92.
\lnr{20}Deducta de 30, relinquunt dies 17, 8, 988.
\lnr{20}Ergo
die 17 Phamenoth, feria 4, hora 8, scrup. % abbriv.
 988, fuit media oppositio
luminarium.
\lnr{22}Quare si Termini iidem semper esse possent, non
opus fuisset methodo posterioris et inversae partis.
\lnr{23}Sed in 500 annis
ii mutantur.
\lnr{24}Antevertit enim eos Luna horis 22.
\lnr{24}Quare confugiendum
ad numeros universos.
\lnr{25}Exemplum: Anno Nabonassari 547
defecit Luna, Mesori 16, circiter hor. % abbriv.
 7, post meridiem, horis
duodecim ante mediam oppositionem.
\lnr{27}Ergo media coniunctio non
abfuit ab hora 19 post meridiem.
\lnr{28}Colligo primum numeros 525
annorum proxime minorum, scilicet, 6, 0, 657. % Newline?
\lnr{39}Et in Tabula annorum
expansorum e regione anni 22, numeros 4, 3, 957.
\lnr{30}Conficitur
novilunium Thoth, 3, 4, 534.
\lnr{31}Iam 16 Mesori est 346 dies a neomenia
Thoth.
\lnr{32}Terminus anni 22, nempe 8, deductus de 346, relinquit
oppositionem 338 in mense 12.
\lnr{33}Cuius mensis numerus anterior, 3, 14,
479, cum neomenia Thoth constituit oppositionem Mesori 6, 18,
1013.
\lnr{35}Fuit igitur novilunium feria sexta, Neomenia Thoth feria quarta.
\lnr{36}Quae characteri Mesori adiecta constituit eius neomeniam, et 15
eiusdem mensis, feriam 5.
\lnr{37}Ergo 16 Mesori fuit oppositio media circiter
19 horam, ut propositum erat.
\lnr{38}Iam si secutus fuissem Terminum, superassem
fines oppositionis die solido.
\lnr{39}Nam terminus 8 ad 339 appositus
conficeret 347.
\lnr{40}Quae est 17 dies Mesori, non 16.
\lnr{40}Itaque in annis 547
Termini excedunt fines Lunae die uno.
\lnr{41}At per numeros posteriores
tutissima est methodus.
%
% 206
% {PDF page nr}{source page nr}{line nr}
\plnr{289}{206}{1}Numeri posteriores annorum 525, et 22 adiecti
fiunt 22, 19, 546.
\lnr{2}Quibus adiiciantur numeri oppositionis 339
ex parte posteriori, nempe 20, 9, 601.
\lnr{3}Conflantur 13, 5, 67.
\lnr{3}Deducti
de 30, relinquunt 16, 18, 1013, verum diem oppositionis Mesori.
\lnr{5}Fac periculum omnium defectionum Lunarium, quaecunque exstant
apud Ptolemaeum.
\lnr{6}Semper eandem constantiam experieris.
\lnr{6}Unum
tantum excipio, quod invitus epocham Nabonassari a Ptolemaeo praescriptam
apposui, ne viderer aut arrogans, qui eam contemsissem,
aut supinus, qui non animadvertissem, ut ingenia horum temporum
nihil non in bonis mentiuntur.
\lnr{10}Nam ex vero eclipsium illarum
epilogismo epocha primi novilunii Nabonassari, fuit 24, 18, 423.
\lnr{12}Diferentia illius, et Ptolemaicae, epochae scrup. % Newline?
Chald. % Newline?
733. % Newline?
\lnr{12}Quae in ratione
defectionum non est contemnenda.
\lnr{13}Itaque si quis illa adiiciat
epilogismo Lunari, perveniet semper ad fines mediarum coniunctionum,
ab % à
 quibus tot scrupulis aberit, duce epocha Ptolemaei.
\lnr{15}Haec
interea admonere volui.
\lnr{16}Caetera palam est a nobis recte tradita fuisse.
%
%====
\section{De Nevruz periodico veterum Persarum}
% Capitalisation follows ToC
%
\lnr{17}Omnibus nationibus, quae anno aequabili usae sunt, ab
initio idem caput anni fuit.
\lnr{18}Itaque easdem neomenias mensium
habebant, hoc uno excepto, si quid mutatione \textgreek{ἐπαγομένων}[?]
variabat.
% Table from p207
\begin{table}[htbp]
  %%% Liber III p207, PDF 290
%% Layout based on Liber 3 p198
% !TEX root = ../../test-table.tex
%%
%%% Count out columns for fixed-width source font
% 000000011111111112222222222333333333344444444445555555555666666666677777777778
% 345678901234567890123456789012345678901234567890123456789012345678901234567890
%
\begin{tabnums} % Select monospaced numbers
%% Select a general font size (uncomment one from the list)
%\tiny
%\scriptsize
%\footnotesize
%\small
\normalsize
%% Center the whole table left-right
\centering
%% Modify separation between columns
%\setlength{\tabcolsep}{2.0pt}
%% Modify distance between rows
\renewcommand{\arraystretch}{1.000} % Not Tuned to eliminate Underfull \vbox
%% Size of header text
\newcommand{\hts}{\small}
%% Width of a column
\newcommand{\cwd}{4em}
%
\newcommand{\da}{\scriptsize{†}}
%%
\begin{tabular}{@{} r l c l @{}}
\toprule
  \ch{[Arabic]}{\hts{\textarabic{}[Arabic]}} &
  \ch{[Arabic]}{\hts{\textarabic{}[Arabic]}} &
  \ch{\hts{mensium}}{\hts{Dies mensium}} &
  \ch{Asphandaramaz}{\hts{Nomina dierum mensis}}
\\
\midrule
\textarabic{}[Arabic] & \textarabic{}[Arabic]
 &      I & \textit{Oromazda} \\
\textarabic{}[Arabic] & \textarabic{ب}[?]
 &     II & \textit{Behemen} \\
\textarabic{}[Arabic] & \textarabic{ج}[?]
 &    III & \textit{Adarpahaschth} \\
\textarabic{}[Arabic] & \textarabic{ﺩ}[?]
 &   IIII & \textit{Schahariuz} \\
\textarabic{}[Arabic] & \textarabic{}[Arabic]
 &      V & \textit{Asphandaramaz} \\
\textarabic{}[Arabic] & \textarabic{و}[?]
 &     VI & \textit{Chardad} \\
\textarabic{}[Arabic] & \textarabic{ز}[?]
 &    VII & \textit{Mardad} \\
\textarabic{}[Arabic] & \textarabic{}[Arabic]
 &   VIII & \textit{Dibadar} \\
\textarabic{}[Arabic] & \textarabic{}[Arabic]
 &     IX & \textit{Adhar aban} \\
\textarabic{}[Arabic] & \textarabic{}[Arabic]
 &      X & \textit{Choramah} \\
\textarabic{}[Arabic] & \textarabic{}[Arabic]
 &     XI & \textit{Thirchusch} \\
\textarabic{}[Arabic] & \textarabic{}[Arabic]
 &    XII & \textit{Dehamhar} \\
\textarabic{}[Arabic] & \textarabic{}[Arabic]
 &   XIII & \textit{Maharserusch} \\
\textarabic{}[Arabic] & \textarabic{}[Arabic]
 &  XIIII & \textit{Rasch} \\
\textarabic{}[Arabic] & \textarabic{}[Arabic]
 &     XV & \textit{Phrurdin} \\
\textarabic{}[Arabic] & \textarabic{}[Arabic]
 &    XVI & \textit{Behiram} \\
\textarabic{}[Arabic] & \textarabic{}[Arabic]
 &   XVII & \textit{Ram} \\
\textarabic{}[Arabic] & \textarabic{}[Arabic]
 &  XVIII & \textit{Bad} \\
\textarabic{}[Arabic] & \textarabic{}[Arabic]
 &    XIX & \textit{Debidin} \\
\textarabic{}[Arabic] & \textarabic{}[Arabic]
 &     XX & \textit{Dinarad} \\
\textarabic{}[Arabic] & \textarabic{}[Arabic]
 &    XXI & \textit{Aschnad} \\
\textarabic{}[Arabic] & \textarabic{}[Arabic]
 &   XXII & \textit{Asman} \\
\textarabic{}[Arabic] & \textarabic{}[Arabic]
 &  XXIII & \textit{Ramiad} \\
\textarabic{}[Arabic] & \textarabic{}[Arabic]
 & XXIIII & \textit{Marasphid} \\
\textarabic{}[Arabic] & \textarabic{}[Arabic]
 &    XXV & \textit{Aniran} \\
\textarabic{}[Arabic] & \textarabic{}[Arabic]
 &   XXVI & \textit{Ahnud} \\
\textarabic{}[Arabic] & \textarabic{}[Arabic]
 &  XXVII & \textit{Aschnud} \\
\textarabic{}[Arabic] & \textarabic{}[Arabic]
 & XXVIII & \textit{Aspandemad} \\
\textarabic{}[Arabic] & \textarabic{}[Arabic]
 &   XXIX & \textit{Wahascht} \\
\textarabic{}[Arabic] & \textarabic{}[Arabic]
 &    XXX & \textit{Haschnusch} \\
\bottomrule
\end{tabular}
%
\caption{Nomina dierum mensis}
\label{tab:p207}
%
\end{tabnums}

\end{table}
%
\lnr{20}Nam illae quinque appendices dies aliis aliam sedem habebant.
\lnr{21}Chaldaei, Persae, Armenii, Aegyptii, unum, eundemque annum
% Persae looks like Porsae in original, but unclear by thick ink.
% 1598 edition is clearer; conforms with obvious intended word.
norant.
\lnr{22}Sed collocatio quinque extremarum dierum nonnihil
statum illorum anni inter se dissimilem faciebat.
\lnr{23}Persarum antiquitus
idem annus fuit, qui hodie, tricenariorum mensium duodecim, et
quinque appendicum dierum.
\lnr{25}Q.~Curtius libro tertio: \textit{Magi proximi
patrium carmen canebant.}
\lnr{26}\textit{Magos trecenti et sexaginta quinque
iuvenes sequebantur, puniceis amiculis velati, diebus totius anni pares
numero.}
% Quintus Curtius Rufus: Historiae Alexandri Magni Macedonis Libri Qui Supersunt
% History of Alexander, Book III, chapter 3, section 10
% "Next came the Magi, chanting their traditional hymn.
%  These were followed by three hundred and sixty-five young men clad in
%  purple robes, equal to the number to the days of a whole year.
\lnr{28}Haec Curtius.
\lnr{28}Quemadmodum igitur Aegyptii annum
suum Canicularem vocabant, qui ex quatuor annis simplicibus et
die adventitio constabat: ita etiam Persae veteres suum quadriennium
\textsc{nevruz} vocabant.
% Nevruz/Nowruz; Persian: نوروز‎ ; "new day". Persian new year's day
\lnr{31}\textarabic{نوروز‎}[Arabic] enim illis est initium veris; quod ad
verbum est, \textsc{novus dies}, quasi diceres \textgreek{νεημερίαν}[?].
\lnr{32}Nam ut Aegyptii
ortum caniculae, propter quem Nili incrementa fiebant, observarunt:
ita etiam Persae ad hanc usque diem, principium veris
non solum temporum civilium epocham et titulum statuunt, sed
etiam epulis, et pompa prosequuntur.
\lnr{36}Sane Fernandus Lopez Castagneda
libro Indicarum rerum sexto, Cap. % Abbriv.
 \rnum{xlvi}, scribit Lusitanorum
legatum Baltazarem Personam a praefecto Xeque Ismaelis
Regis in quodam oppido Persidis detentum, quod Rex celebraturus
esset solenne festum, quod lingua Persica \textsc{nevruz} vocari dicit, idque
significare solenne verni temporis.
%
% 207
% {PDF page nr}{source page nr}{line nr}
\setpnrs{290}{207}
% Table p207: Nomina Dierum Mensis
% Included on previous page
%
% 208
% {PDF page nr}{source page nr}{line nr}
\plnr{291}{208}{4}\textit{Baltazarem Personam detinuerat
Praefectus, ut omnem apparatum solennitatis} \textsc{nevruz}, \textit{et
pompam regiam videret.}
\lnr{6}Haec ille.
\lnr{6}Legimus etiam apud antiquos
scriptores, Persis matrimonia legitima non videri, nisi quae tempore
verno constracta essent.
\lnr{8}Hodie illud nomen \textsc{nevruz} non solum
apud Persas plurimum est in ore, sed etiam apud alias nationes, adeo
ut Turcae annum Solarem Nevruz vocent: et cum aetatem alicuius
indicare volunt, tot Nevruz dicunt eum habere, quot annos
natus est.
\lnr{12}Persae igitur annum suum Nevruz vocabant, ut Aegyptii
Canicularem.
\lnr{13}Quin etiam ut Aegyptii \textgreek{ἐνιαυτὸν Θεοῦ}[?], sic illi
 \textarabic{
}[Arabic] Sal Chodai, annum \textsc{dei}, hoc est Solis.
\lnr{14}Unde periodum
annorum Solarium in \textgreek{ἀφετικοῖς}[?] vitae humanae Astrologi vocant
\textsc{salchodai}.
\lnr{16}In Kalendario meo Persico Enneadecaeteris vocatur
\textarabic{}[Arabic] Salmea. % Newline?
\lnr{17}Hoc est periodus Lunaris.
\lnr{17}Sed hoc differebant
Aegyptii a Persis et Chaldaeis. % Newline?
\lnr{18}Quod illi suo Thoth habenas permitterent,
neque intercalatione intra suos fines coercerent.
\lnr{19}Nam quot
erant dies a neomenia Thoth primi, tot quadriennia et annos Caniculares,
sive \textgreek{ἐνιαυτοὺς Θεοῦ}[?] putabant.
\lnr{21}Chaldaei vero et Persae, quia
singuli dies unius mensis appellationibus Heroum, aut regum cognominabantur,
primum quadriennium nomine diei primi vocabant:
secundum quadriennium, nomine secundi.
\lnr{24}Exemplum: Primi anni
magnae periodi, sive \textsc{sal chodai}, caput in \textsc{nevruz}, hoc est,
aequinoctio statuebatur.
\lnr{26}Quatuor primi anni vocabantur Oromazda. % Newline?
\lnr{27}Quia ita vocatur dies primus mensis.
\lnr{27}Secundum quadriennium
appelabant Behemen.
\lnr{28}Is est secundus dies mensis.
\lnr{28}Idque faciebant
addito nomine mensis.
\lnr{29}Ut si diceretur: factum est in Behemen Phrurdin
mensis, intelligebant eum esse secundum quadriennium, quod
inciderat in Phrurdin mensem.
\lnr{31}Sciebant deinde quotus esset mensis
Phrurdin, et a neomenia primi mensis, ad secundam ipsius Phrurdin,
quot dies sunt, tot quadriennia praeteriisse non ignorabant.
\lnr{33}Sed post
annos 120, mensem integrum intercalabant: quia in tot annis, 30
diebus caput periodi antevertit priscam epocham.
\lnr{35}Nam si Thoth
Aegyptius hoc anno coepit incurrere in Kal. Mai, post 120 annos confectos
ipse Thoth deprehendetur in Kal. Aprilis.
\lnr{37}Ut igitur caput periodi
in antiquas sedes summoveretur, unum \textarabic{}[Arabic] Mahe bezurg,
hoc est, magnum mensem fluxisse notabant, qui esset annorum
120. % Newline?
%
% 209
% {PDF page nr}{source page nr}{line nr}
\plnr{292}{209}{1}Et ut labem ac spatium inane replerent, quadrantem eius, hoc
est, 30 dies, intercalabant, hoc modo: Post 120 annos, qui est mensis
magnus Persarum, primus mensis intercalabatur, puta Adar.
\lnr{3}Et
anno































% ==== End of text of Liber Tertius ===

\include{./tex/liber-quartus}
\include{./tex/liber-quintus}
\include{./tex/liber-sextus}
\include{./tex/liber-septimus}

\appendix
\include{./tex/nomenclator}
\include{./tex/index-rerum-et-verborum}
\include{./tex/index-graecarum-vocum}
\include{./tex/index-orientalium-vocum}
\include{./tex/errata}

\backmatter
\include{./tex/operis-finis}
\include{./tex/fragmenta-selecta}

\end{document}
