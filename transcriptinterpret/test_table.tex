%%% Test a table
%
% Set the \input{} to the table you want to test, then typeset this file
%
\documentclass[12pt,twoside,a4paper]{book}
\usepackage[lines=40,showframe]{geometry}
\usepackage{array} % allow row distance change with \arraystretch
%
\usepackage{lipsum}
\usepackage{rotating} % Make it possible to rotate headers
\usepackage[normalsize]{caption} % Keep captions to (long)tables normal sized
\usepackage{longtable}% Allow for split of the table over more than one page
\usepackage{booktabs} % To get nice table layout
%
% To include graphics in tables
\usepackage{graphicx}
%
%% Settings from the main text we also need for the test
\usepackage{fontspec}
\setmainfont{Hoefler Text}[]
%\setmainfont{Times New Roman}[]
\newfontfamily\greekfont{Times New Roman}
\newfontfamily\hebrewfont{Arial Hebrew}
\newfontfamily\arabicfont{Arial Unicode MS}
\newfontfamily\astrofont{Menlo}
\newcommand\astro[1]{{\astrofont #1}}
%%
\usepackage[quiet]{polyglossia}
\setmainlanguage{latin}
\setotherlanguage{greek}
\setotherlanguage{hebrew}
\setotherlanguage{arabic}
%
\newcommand{\rnum}[1]{\textsc{#1}} % Roman numerals
\newcommand{\altsep}{\slash{}} % Separator between alternative values in tables
%
\newlength{\cw} % Column width for \parbox headers
\newcommand{\ch}[2]{\settowidth{\cw}{#1}\parbox[b]{\cw}{\raggedright#2}}
%
% \ruleover{string}
% \ruleover[scale]{string}
\makeatletter
\newlength{\ROver@width} % width of the string
\newlength{\ROver@height} % height of the string
\newlength{\ROver@thick} % thickness of the line
\newcommand{\ruleover}[2][0.75]{%
  \settoheight{\ROver@height}{#2}% Get height of original text
  \settowidth{\ROver@width}{#2}% Get width of original text
  \setlength{\ROver@thick}{0.5333pt}% Set pre-shrink thickness of rule
  \setlength{\ROver@width}{#1\ROver@width}% Calculate shrunken width
  \setlength{\ROver@height}{#1\ROver@height}% Calculate shrunken height
  \makebox[0pt][l]{\scalebox{#1}{#2}}% Print the text with shrinking
  \rule[1.3333\ROver@height]{\ROver@width}{#1\ROver@thick}% Draw bar
}
\makeatother

\newcommand{\gnum}[1]{\ruleover{#1}} % Greek number
\newcommand{\gnums}[2]{\ruleover[#1]{#2}} % Scaled Greek number

\begin{document}

%% Longtable version
%%%% Liber I p43
%%
%%% Count out columns for fixed-width source font
% 000000011111111112222222222333333333344444444445555555555666666666677777777778
% 345678901234567890123456789012345678901234567890123456789012345678901234567890
%
%%
\begingroup
%\tiny
%\scriptsize
\footnotesize
%\small
%\normalsize
%% Modify separation between columns
\setlength{\tabcolsep}{2.4pt}
%% Modify distance between rows
%\renewcommand{\arraystretch}{1.2}
%% Local command to define small header
\newcommand{\sh}[1]{\multicolumn{1}{c}{\tiny{#1}}}
%% Names of the months, to save us a lot of typing
\newcommand{\rabx}{Rabie prior}
\newcommand{\rabz}{Rabie posterior}
\newcommand{\giux}{Giumadi prior}
\newcommand{\giuz}{Giumadi posterior}
\newcommand{\rege}{Regeb}
\newcommand{\saha}{Sahaben}
\newcommand{\rama}{Ramadhan}
\newcommand{\scew}{Scewal}
\newcommand{\dulk}{Dulkaidathi}
\newcommand{\dulc}{Dulchagiathi}
\newcommand{\muha}{Muharam}
\newcommand{\seph}{Sephar}
%% Let longtable process the whole table in one go
\setcounter{LTchunksize}{100}
\begin{longtable}[c]{@{}r  c  c  c  c  r@{~}l l l l l@{}}
\toprule
% Put a reference to the first page of the table in the List of Tables
% Tables in the LoT are listed on the 'section' level
% '\numberline{\thetable}' puts the number of the table before the
% title, correctly indented.
% '\protect' helps pass these commands without error.
\addcontentsline{lot}{section}{%
\protect\numberline{\thetable}Periodi magnae Hagarenorum}
 & \multicolumn{9}{c}{\Large\textsc{Tabula periodi magnae Hagarenorum}}\\
\toprule
~ &
 \sh{Anni} &
 \sh{Cyclus} &
 \sh{Character} &
 \sh{Cyclus} \\
~ &
 \sh{periodi} &
 \sh{Lunae} &
 \sh{anni} &
 \sh{Solis} &
~ & & % 31 Martii
Periodus prima &
Periodus secunda &
Periodus tertia
\\
\midrule
\endfirsthead
\toprule
&
\multicolumn{9}{c}{\Large\textsc{Residuum tabulae periodi magnae Hagarenorum}}\\
\toprule
~ &
 \sh{Anni} &
 \sh{Cyclus} &
 \sh{Character} &
 \sh{Cyclus} \\
~ &
 \sh{periodi} &
 \sh{Lunae} &
 \sh{anni} &
 \sh{Solis} &
~ & & % 31 Martii
Periodus prima &
Periodus secunda &
Periodus tertia
\\
\midrule
\endhead
\bottomrule
  \addlinespace
  & \multicolumn{6}{l}{† \textgreek{ἐμβολ.[?]}}
  & \multicolumn{3}{l}{‡ \textgreek{ὑπερή.[?]}}
\endfoot
\bottomrule
  \addlinespace
  & \multicolumn{6}{l}{† \textgreek{ἐμβολ.[?]}}
  & \multicolumn{3}{l}{‡ \textgreek{ὑπερή.[?]}} \\
  \addlinespace
  % Put the table nr and title below the table, without entry in the LoT
  \caption[]{Periodi magnae Hagarenorum}
  \label{tab:p112}
\endlastfoot
  & ~1 & 14 & 1 & F   & 31&Martii & \seph & \giuz & \scew \\
  & ~2 & 15 & 5 & E   & 20&Martii & \seph & \giuz & \scew & ‡\\
† & ~3 & 16 & 3 & D C &  9&Martii & \seph & \giuz & \scew \\
  & ~4 & 17 & 2 & B   & 28&Martii & \rabx & \rege & \dulk \\
\cmidrule{2-10}
  & ~5 & 18 & 6 & A   & 17&Martii & \rabx & \rege & \dulk \\
† & ~6 & 19 & 3 & G   &  6&Martii & \rabx & \rege & \dulk \\
  & ~7 & ~1 & 2 & F E & 24&Martii & \rabz & \saha & \dulc \\
† & ~8 & ~2 & 6 & D   & 13&Martii & \rabz & \saha & \dulc \\
\cmidrule{2-10}
  & ~9 & ~3 & 5 & C   &  1&April. & \giux & \rama & \muha & ‡\\
  & 10 & ~4 & 3 & B   & 22&Martii & \giux & \rama & \muha \\
† & 11 & ~5 & 7 & A G & 10&Martii & \giux & \rama & \muha \\
  & 12 & ~6 & 6 & F   & 29&Martii & \giuz & \scew & \seph \\
\cmidrule{2-10}
  & 13 & ~7 & 3 & E   & 18&Martii & \giuz & \scew & \seph & ‡\\
† & 14 & ~8 & 1 & D   &  8&Martii & \giuz & \scew & \seph \\
  & 15 & ~9 & 7 & C B & 26&Martii & \rege & \dulk & \rabx \\
† & 16 & 10 & 4 & A   & 15&Martii & \rege & \dulk & \rabx \\
\cmidrule{2-10}
  & 17 & 11 & 3 & G   &  3&April. & \saha & \dulc & \rabz \\
  & 18 & 12 & 7 & F   & 23&Martii & \saha & \dulc & \rabz & ‡\\
† & 19 & 13 & 5 & E D & 12&Martii & \saha & \dulc & \rabz \\
  & 20 & 14 & 4 & C   & 31&Martii & \rama & \muha & \giux \\
\cmidrule{2-10}
  & 21 & 15 & 4 & B   & 20&Martii & \rama & \muha & \giux \\
† & 22 & 16 & 1 & A   &  9&Martii & \rama & \muha & \giux \\
  & 23 & 17 & 5 & G F & 27&Martii & \scew & \seph & \giuz & ‡\\
  & 24 & 18 & 4 & E   & 17&Martii & \scew & \seph & \giuz \\
\cmidrule{2-10}
† & 25 & 19 & 6 & D   &  6&Martii & \scew & \seph & \giuz \\
  & 26 & ~1 & 5 & C   & 25&Martii & \dulk & \rabx & \rege \\
† & 27 & ~2 & 2 & B A & 13&Martii & \dulk & \rabx & \rege \\
  & 28 & ~3 & 1 & G   &  1&April. & \dulc & \rabz & \saha \\
\cmidrule{2-10}
  & 29 & ~4 & 5 & F   & 21&Martii & \dulc & \rabz & \saha & ‡\\
† & 30 & ~5 & 3 & E   & 11&Martii & \dulc & \rabz & \saha \\
  & 31 & ~6 & 2 & D C & 29&Martii & \muha & \giux & \rama \\
  & 32 & ~7 & 6 & B   & 18&Martii & \muha & \giux & \rama \\
\cmidrule{2-10}
† & 33 & ~8 & 3 & A   &  7&Martii & \muha & \giux & \rama \\
  & 34 & ~9 & 3 & G   & 27&Martii & \seph & \giuz & \scew & ‡\\
† & 35 & 10 & 7 & F E & 15&Martii & \seph & \giuz & \scew \\
  & 36 & 11 & 6 & D   &  2&April. & \rabx & \rege & \dulk \\
\cmidrule{2-10}
  & 37 & 12 & 3 & C   & 23&Martii & \rabx & \rege & \dulk \\
† & 38 & 13 & 7 & B   & 12&Martii & \rabx & \rege & \dulk \\
  & 39 & 14 & 6 & A G & 30&Martii & \rabz & \saha & \dulc & ‡\\
  & 40 & 15 & 4 & F   & 20&Martii & \rabz & \saha & \dulc \\
\cmidrule{2-10}
† & 41 & 16 & 1 & E   &  9&Martii & \rabz & \saha & \dulc \\
  & 42 & 17 & 7 & D   & 28&Martii & \giux & \rama & \muha \\
  & 43 & 18 & 4 & C B & 16&Martii & \giux & \rama & \muha & ‡\\
† & 44 & 19 & 2 & A   &  6&Martii & \giux & \rama & \muha \\
\cmidrule{2-10}
  & 45 & ~1 & 1 & G   & 25&Martii & \giuz & \scew & \seph \\
† & 46 & ~2 & 5 & F   & 14&Martii & \giuz & \scew & \seph \\
  & 47 & ~3 & 4 & E D &  1&April. & \rege & \dulk & \rabx \\
  & 48 & ~4 & 1 & C   & 21&Martii & \rege & \dulk & \rabx \\
\cmidrule{2-10}
† & 49 & ~5 & 5 & B   & 10&Martii & \rege & \dulk & \rabx \\
  & 50 & ~6 & 4 & A   & 29&Martii & \saha & \dulc & \rabz & ‡\\
  & 51 & ~7 & 2 & G F & 18&Martii & \saha & \dulc & \rabz \\
† & 52 & ~8 & 6 & E   &  7&Martii & \saha & \dulc & \rabz \\
\cmidrule{2-10}
  & 53 & ~9 & 5 & D   & 26&Martii & \rama & \muha & \giux \\
† & 54 & 10 & 2 & C   & 15&Martii & \rama & \muha & \giux \\
  & 55 & 11 & 1 & B A &  2&April. & \scew & \seph & \giuz & ‡\\
  & 56 & 12 & 6 & G   & 23&Martii & \scew & \seph & \giuz \\
\cmidrule{2-10}
† & 57 & 13 & 3 & F   & 12&Martii & \scew & \seph & \giuz \\
  & 58 & 14 & 2 & E   & 31&Martii & \dulk & \rabx & \rege \\
  & 59 & 15 & 6 & D C & 19&Martii & \dulk & \rabx & \rege & ‡\\
† & 60 & 16 & 4 & B   &  9&Martii & \dulk & \rabx & \rege \\
\cmidrule{2-10}
  & 61 & 17 & 3 & A   & 28&Martii & \dulc & \rabz & \saha \\
  & 62 & 18 & 7 & G   & 17&Martii & \dulc & \rabz & \saha \\
† & 63 & 19 & 4 & F E &  5&Martii & \dulc & \rabz & \saha \\
  & 64 & ~1 & 3 & D   & 24&Martii & \muha & \giux & \rama & ‡\\
\cmidrule{2-10}
† & 65 & ~2 & 1 & C   & 14&Martii & \muha & \giux & \rama \\
  & 66 & ~3 & 7 & B   &  2&April. & \seph & \giuz & \scew \\
  & 67 & ~4 & 4 & A G & 21&Martii & \seph & \giuz & \scew \\
† & 68 & ~5 & 1 & F   & 10&Martii & \seph & \giuz & \scew \\
\cmidrule{2-10}
  & 69 & ~6 & 7 & E   & 29&Martii & \rabx & \rege & \dulk & ‡\\
  & 70 & ~7 & 5 & D   & 18&Martii & \rabx & \rege & \dulk \\
† & 71 & ~8 & 2 & C B &  7&Martii & \rabx & \rege & \dulk \\
  & 72 & ~9 & 1 & A   & 26&Martii & \rabz & \saha & \dulc \\
\cmidrule{2-10}
† & 73 & 10 & 5 & G   & 15&Martii & \rabz & \saha & \dulc \\
  & 74 & 11 & 4 & F   &  3&April. & \giux & \rama & \muha \\
  & 75 & 12 & 1 & E D & 22&Martii & \giux & \rama & \muha \\
† & 76 & 13 & 5 & C   & 11&Martii & \giux & \rama & \muha & ‡\\
\end{longtable}
\endgroup

%% Regular table version
%\begin{table}[htbp]
%  %%% Liber II p117, PDF 200
%%
%%% Count out columns for fixed-width source font
% 000000011111111112222222222333333333344444444445555555555666666666677777777778
% 345678901234567890123456789012345678901234567890123456789012345678901234567890
%
%% Select a general font size (uncomment one from the list)
%\tiny
%\scriptsize
\footnotesize
%\small
%\normalsize
%% Center the whole table left-right
\centering
%% Modify separation between columns
\setlength{\tabcolsep}{2.1pt}
%% Modify distance between rows
%\renewcommand{\arraystretch}{1.3}
%
\newlength{\cw}
%% Parbox column header \ch{sizetext}{fontsize}{text}
\newcommand{\ch}[3]{\settowidth{\cw}{#1}\parbox[b]{\cw}{#2{#3}}}
%%
\begin{tabular}{@{}l c r r c l r@{~}l l r@{~}l c@{}}
\toprule
 \multicolumn{12}{c}{\Large\textsc{Tabella noviluniorum Samaritanorum}}\\
 \multicolumn{12}{c}{\large\textsc{in anno Christi \rnum{mdlxxxiiii}}}
\\
\toprule
 \ch{Giumedi posterior}{\scriptsize}{Menses Lunares} &
 \ch{\scriptsize Feria}{\scriptsize}{Feria} &
 \ch{\scriptsize Horae}{\scriptsize}{Horae} &
 \ch{180}{\tiny}{Scru\-pu\-la\\1800} &
 &
 \ch{Marchesban 28}{\scriptsize}{Menses Samarit. Iuliani} &
 & &
 \ch{Marchesban}{\scriptsize}{Menses Iuliani Samarit.} &
 \multicolumn{2}{l}{\ch{29 Octobr.}{\scriptsize}{Neomenia in mensibus Iul.}} &
 \ch{\tiny eniarum}{\tiny}{Feria Neomeniarum}
\\
\midrule
  Dulchagia &
  2.~3 &
  10. &
  180 &
  D &
  Adar 6 &
  3&Martii &
  Adar &
  26&Feb. &
  5
\\
  Muharram &
  4 &
  3. &
  180 &
  N &
  Pesah Nisan 4 &
  1&April. &
  Nisan &
  29&Martii &
  1
\\
  Sephar &
  6 &
  6. &
  130 &
  N &
  Iiar 4 &
  1&Maii &
  Iiar &
  28&April. &
  3
\\
  Rabie prior &
  7.~1 &
  10. &
  1 &
  D &
  3 Siban &
  31&Maii &
  Siban &
  29&Maii &
  6
\\
  Rabie posterior &
  2 &
  11. &
  2 &
  N &
  Tamuz 2 &
  29&Iunii &
  Tamuz &
  28&Iunii &
  1
\\
  Giumedi prior &
  3.~4 &
  11. &
  152 &
  D &
  Ab 1 &
  28&Iulii &
  Ab &
  29&Iulii &
  4
\\
  Giumedi posterior &
  5 &
  9. &
  10 &
  N &
  Ab 30 &
  27&Aug. &
  Ilul &
  29&Aug. &
  7
\\
  Regeb &
  6.~7 &
  8. &
  152 &
  D &
  Hag Ilul 29 &
  26&Sept. &
  Tisri &
  28&Septem. &
  2
\\
  Sahaben &
  1 &
  5. &
  2 &
  N &
  Tisri 28 &
  25&Octobr. &
  Marchesban &
  29&Octobr. &
  5
\\
  Ramadhan &
  2 &
  5. &
  1 &
  D &
  Marchesban 28 &
  23&Novemb. &
  Caslim &
  28&Nov. &
  7
\\
  Schevval &
  4 &
  5. &
  1 &
  N &
  Caslim 26 &
  23&Decem. &
  Tebith &
  29&Decem. &
  3
\\
  Dulkaida &
  5 &
  3. &
  10 &
  D &
  Teibeth 24 &
  21&Ianuarii &
  Scebat &
  29&Ian. &
  6
\\
\bottomrule
\end{tabular}
%
\caption{Noviluniorum Samaritanorum in anno Christi 1584}
\label{tab:p117}
%
%\end{table}

Banana

\the\bigskipamount

\the\LTpre

%\setlength{\LTpre}{3\bigskipamount plus \bigskipamount minus \bigskipamount}
\the\LTpre


\clearpage
\lipsum

Sed commodo posuere pede. Mauris ut est. Ut quis purus. Sed ac odio.
Sed vehi- cula hendrerit sem. Duis non odio. Morbi ut dui.
Sed accumsan risus eget odio. In hac habitasse platea dictumst.
Pellentesque non elit. Fusce sed justo eu urna porta tincidunt.
Mauris felis odio, sollicitudin sed, volutpat a, ornare ac, erat.
Morbi quis dolor. Donec pellentesque, erat ac sagittis semper,
nunc dui lobortis purus, quis congue purus metus ultricies tellus.
Proin et quam. Class aptent taciti sociosqu ad litora torquent per
conubia nostra, per inceptos hymenaeos.
Praesent sapien turpis, fermentum vel, eleifend faucibus, vehicula eu,
lacus.
Sed commodo posuere pede. Mauris ut est. Ut quis purus. Sed ac odio.
Sed vehi- cula hendrerit sem. Duis non odio. Morbi ut dui.
Sed accumsan risus eget odio. In hac habitasse platea dictumst.
Pellentesque non elit. Fusce sed justo eu urna porta tincidunt.
Sed commodo posuere pede. Mauris ut est. Ut quis purus. Sed ac odio.
Sed vehi- cula hendrerit sem. Duis non odio. Morbi ut dui.
Sed accumsan risus eget odio. In hac habitasse platea dictumst.
Pellentesque non elit. Fusce sed justo eu urna porta tincidunt.
Pellentesque non elit. Fusce sed justo eu urna porta tincidunt.
%AAPellentesque non elit. Fusce sed justo eu urna porta tincidunt.

%\bigskip
\smallskip
%%% Liber II p89-90
%%
%%% Count out columns for fixed-width source font
% 000000011111111112222222222333333333344444444445555555555666666666677777777778
% 345678901234567890123456789012345678901234567890123456789012345678901234567890
%
\begingroup
\tiny
%\scriptsize
%\footnotesize
%\small
%\normalsize
%% Modify separation between columns
\setlength{\tabcolsep}{2.5pt}
%% Modify distance between rows
\renewcommand{\arraystretch}{0.9}
%% Let longtable process the whole table in one go
\setcounter{LTchunksize}{100}
%
%% Define a smaller dagger (unfortunalely tiny is already the smallest)
\newcommand{\da}{{\tiny †}}
%% Command for lines between the rows
\newcommand{\streep}{\cmidrule{2-35}}
%% Command for entries that span 2 columns
\newcommand{\mc}[1]{\multicolumn{2}{c}{#1}}
%%
\begin{longtable}[c]{@{}%
 c c c  r@{~}l r@{~}l r@{~}l r@{~}l r@{~}l r@{~}l
r@{~}l r@{~}l r@{~}l r@{~}l r@{~}l r@{~}l r@{~}l  c c c c r@{~}l
@{}}
\toprule
\multicolumn{35}{c}{\Large\textsc{Tabula neomeniarum periodi Calippicae}}\\
\toprule
% Put entry into List of Tables pointing to the first page of the table
\addcontentsline{lot}{section}{%
\protect\numberline{\thetable}Neomeniarum periodi Calippicae}
\label{tab:p089}
% Read the header description from an external file
% Header for table p89-90
% Version with slanted headers for the names of the months
~ &
\begin{turn}{90}Anni periodi\end{turn} &
\begin{turn}{90}Cyclus Lunae\end{turn} & 

\begin{rotate}{75}\textgreek{Εκατομβαιών}\end{rotate} & &
\begin{rotate}{75}\textgreek{Μεταγειτνιών}\end{rotate} & &
\begin{rotate}{75}\textgreek{Βοηδρομιών}\end{rotate} & &

\begin{rotate}{75}\textgreek{Πυανεψιών}\end{rotate} & &
\begin{rotate}{75}\textgreek{Μαιμακτηριών}\end{rotate} & &
\begin{rotate}{75}\textgreek{Ποσειδεών}\end{rotate} & &

\begin{rotate}{75}\textgreek{Γαμηλιών}\end{rotate} & &
\begin{rotate}{75}\textgreek{Ανθεστηριών}\end{rotate} & &
\begin{rotate}{75}\textgreek{Ελαφηβολιών}\end{rotate} & &

\begin{rotate}{75}\textgreek{Μουνυχιών}\end{rotate} & &
\begin{rotate}{75}\textgreek{Θαργηλιών}\end{rotate} & &
\begin{rotate}{75}\textgreek{Σκιῤῥοφοριών α}\end{rotate} & &
\begin{rotate}{75}\textgreek{Σκιῤῥοφοριών β}\end{rotate} & &

\multicolumn{1}{c}{\begin{turn}{90}Dies collecti\end{turn}} & 
\multicolumn{1}{c}{\begin{turn}{90}Syzygiae collectae\end{turn}} & 
\multicolumn{1}{c}{\begin{turn}{90}Menses cavi[?]\end{turn}} & 
\multicolumn{1}{c}{\begin{turn}{90}Syclus Solis\end{turn}} & 
\multicolumn{1}{r}{\begin{turn}{90}Neomenia\end{turn}} & 
\multicolumn{1}{l}{\begin{turn}{90}Ecatombaeonis\end{turn}}
\\

\midrule
\endfirsthead
%%
\toprule
\multicolumn{35}{c}{%
\large\textsc{Residuum tabulae neomeniarum periodi Calippicae}}\\
\toprule
% Read the header description from an external file
% Header for table p89-90
% Version with slanted headers for the names of the months
~ &
\begin{turn}{90}Anni periodi\end{turn} &
\begin{turn}{90}Cyclus Lunae\end{turn} & 

\begin{rotate}{75}\textgreek{Εκατομβαιών}\end{rotate} & &
\begin{rotate}{75}\textgreek{Μεταγειτνιών}\end{rotate} & &
\begin{rotate}{75}\textgreek{Βοηδρομιών}\end{rotate} & &

\begin{rotate}{75}\textgreek{Πυανεψιών}\end{rotate} & &
\begin{rotate}{75}\textgreek{Μαιμακτηριών}\end{rotate} & &
\begin{rotate}{75}\textgreek{Ποσειδεών}\end{rotate} & &

\begin{rotate}{75}\textgreek{Γαμηλιών}\end{rotate} & &
\begin{rotate}{75}\textgreek{Ανθεστηριών}\end{rotate} & &
\begin{rotate}{75}\textgreek{Ελαφηβολιών}\end{rotate} & &

\begin{rotate}{75}\textgreek{Μουνυχιών}\end{rotate} & &
\begin{rotate}{75}\textgreek{Θαργηλιών}\end{rotate} & &
\begin{rotate}{75}\textgreek{Σκιῤῥοφοριών α}\end{rotate} & &
\begin{rotate}{75}\textgreek{Σκιῤῥοφοριών β}\end{rotate} & &

\multicolumn{1}{c}{\begin{turn}{90}Dies collecti\end{turn}} & 
\multicolumn{1}{c}{\begin{turn}{90}Syzygiae collectae\end{turn}} & 
\multicolumn{1}{c}{\begin{turn}{90}Menses cavi[?]\end{turn}} & 
\multicolumn{1}{c}{\begin{turn}{90}Syclus Solis\end{turn}} & 
\multicolumn{1}{r}{\begin{turn}{90}Neomenia\end{turn}} & 
\multicolumn{1}{l}{\begin{turn}{90}Ecatombaeonis\end{turn}}
\\

\midrule
\endhead
%%
%\bottomrule
\addlinespace[8pt]
& & \multicolumn{29}{l}{\footnotesize \super{†} \textgreek{ἐμβολ. [Abbriv.]}}\\
\endfoot
%%
\bottomrule
\addlinespace[8pt]
& & \multicolumn{29}{l}{\footnotesize \super{†} \textgreek{ἐμβολ. [Abbriv.]}}\\
\addlinespace
% Put the table nr and title below the table, without entry in the LoT
\caption[]{Neomeniarum periodi Calippicae}
\endlastfoot
%%
  &    &    &
     &   &    &   &  4.&5  &    &   &  8.&9  &    &   &
  12.&13 &    &   & 16.&17 &    &   & 20.&21 &    &   &
  24.&25 &
  \\
\nopagebreak
\da &  1 & 14 &
  \mc{7} & \mc{2} & \mc{4} & \mc{5} & \mc{7} & \mc{1} &
  \mc{3} & \mc{4} & \mc{6} & \mc{7} & \mc{2} & \mc{3} &
  \mc{5} &
   384  &  13 &   6 & B & 28&Iun \\
\nopagebreak
%
\streep
  &    &   &
     &   & 28.&29 &    &   &    &   &  2.&3  &    &   &
   6.&7  &    &   & 10.&11 &    &   & 14.&15 &    &   &
     &   &
  \\
\nopagebreak
  &  2 & 15 &
  \mc{6} & \mc{1} & \mc{2} & \mc{4} & \mc{6} & \mc{7} &
  \mc{2} & \mc{3} & \mc{5} & \mc{6} & \mc{1} & \mc{2} &
  \mc{0} &
   739  &  25 &  11 & A G & 16&Iul \\
\nopagebreak
%
\streep
  &    &    &
  18.&19 &    &   & 23.&23\footnote{23.23: Sic.} &    &   & 26.&27 &    &   &
% '23.23' in original
  30.&31 &    &   &    &   &  4.&5  &    &   &  8.&9  &
     &   &
  \\
\nopagebreak
\da &  3 & 16 &
  \mc{4} & \mc{5} & \mc{7} & \mc{1} & \mc{3} & \mc{4} &
  \mc{6} & \mc{7} & \mc{2} & \mc{4} & \mc{5} & \mc{7} &
  \mc{1} &
  1123  &  38 &  17 & F &  6&Iul \\
\nopagebreak
%
\streep
  &    &    &
  12.&13 &    &   & 16.&17 &    &   & 20.&21 &    &   &
  24.&25 &    &   & 27.&28 &    &   &    &   &  1.&2  &
     &   &
  \\
\nopagebreak
  &  4 & 17 &
  \mc{3} & \mc{4} & \mc{6} & \mc{7} & \mc{2} & \mc{3} &
  \mc{5} & \mc{6} & \mc{1} & \mc{2} & \mc{4} & \mc{6} &
  \mc{0} &
  1477  &  50 &  23 & E & 25&Iul \\
\streep
\nopagebreak
%
  &    &    &
     &   &  5.&6  &    &   &  9.&10 &    &   & 13.&14 &
     &   & 17.&18 &    &   & 21.&22 &    &   & 25.&26 &
     &   &
  \\
\nopagebreak
  &  5 & 18 &
  \mc{7} & \mc{2} & \mc{3} & \mc{5} & \mc{6} & \mc{1} &
  \mc{2} & \mc{4} & \mc{5} & \mc{7} & \mc{1} & \mc{3} &
  \mc{0} &
  1831  &  62 &  29 & D & 14&Iul \\
\nopagebreak
%
\streep
  &    &   &
     &   & 29.&30 &    &   &    &   &  3.&4  &    &   &
   7.&8  &    &   & 11.&12 &    &   & 15.&16 &    &   &
  19.&20 &
  \\
\nopagebreak
\da &  6 & 19 &
  \mc{4} & \mc{6} & \mc{7} & \mc{2} & \mc{4} & \mc{5} &
  \mc{7} & \mc{1} & \mc{3} & \mc{4} & \mc{6} & \mc{7} &
  \mc{2} &
  2215  &  75 &  35 & C B &  2&Iul \\
\nopagebreak
%
\streep
  &    &   &
     &   & 23.&24 &    &   & 27.&28 &    &   &    &   &
  11.&12 &    &   &  5.&6  &    &   &  9.&10 &    &   &
     &   &
  \\
\nopagebreak
  &  7 &  1 &
  \mc{3} & \mc{5} & \mc{6} & \mc{1} & \mc{2} & \mc{4} &
  \mc{6} & \mc{7} & \mc{2} & \mc{3} & \mc{5} & \mc{6} &
  \mc{0} &
  2570  &  87 &  40 & A &  21&Iul \\
\nopagebreak
%
\streep
  &    &    &
  13.&14 &    &   & 17.&18 &    &   & 21.&22 &    &   &
  24.&25 &    &   & 28.&29 &    &   &    &   &  2.&3  &
     &   &
  \\
\nopagebreak
  &  8 &  2 &
  \mc{1} & \mc{2} & \mc{4} & \mc{5} & \mc{7} & \mc{1} &
  \mc{3} & \mc{4} & \mc{6} & \mc{7} & \mc{2} & \mc{4} &
  \mc{0} &
  2924  &  99 &  46 & G & 11&Iul \\
%\nopagebreak
%
\streep
  &    &    &
     &   &  6.&7  &    &   & 10.&11 &    &   & 14.&15 &
     &   & 18.&19 &    &   & 22.&23 &    &   & 26.&27 &
     &   &
  \\
\nopagebreak
\da &  9 &  3 &
  \mc{5} & \mc{7} & \mc{1} & \mc{3} & \mc{4} & \mc{6} &
  \mc{7} & \mc{2} & \mc{3} & \mc{5} & \mc{6} & \mc{1} &
  \mc{2} &
  3308  & 112 &  52 & F & 30&Iun \\
\nopagebreak
%
\streep
  &    &    &
  30.&1  &    &   &    &   &  4.&5  &    &   &  8.&9  &
     &   & 12.&13 &    &   & 16.&17 &    &   & 20.&21 &
     &   &
  \\
\nopagebreak
  & 10 &  4 &
  \mc{4} & \mc{5} & \mc{7} & \mc{2} & \mc{3} & \mc{5} &
  \mc{6} & \mc{1} & \mc{2} & \mc{4} & \mc{5} & \mc{7} &
  \mc{0} &
  3662  &  12 &  58 & E D & 18&Iul \\
\nopagebreak
%
\streep
  &    &   &
     &   & 20.&25 &    &   & 28.&29 &    &   &  2.&3  &
     &   &  6.&7  &    &   &    &   & 10.&11 &    &   &
  14.&15 &
  \\
\nopagebreak
\da & 11 &  5 &
  \mc{1} & \mc{3} & \mc{4} & \mc{6} & \mc{7} & \mc{2} &
  \mc{3} & \mc{5} & \mc{6} & \mc{1} & \mc{3} & \mc{4} &
  \mc{6} &
  4046  & 135 &  64 & C &   7&Iul \\
\nopagebreak
%
\streep
  &    &   &
     &   & 18.&19 &    &   & 21.&22 &    &   & 25.&26 &
     &   & 29.&30 &    &   &    &   &  3.&4  &    &   &
     &   &
  \\
\nopagebreak
  & 12 &  6 &
  \mc{7} & \mc{2} & \mc{3} & \mc{5} & \mc{6} & \mc{1} &
  \mc{2} & \mc{4} & \mc{5} & \mc{7} & \mc{2} & \mc{3} &
  \mc{0} &
  4401  & 149 &  69 & B &  26&Iul \\
%\nopagebreak
%
\streep
  &    &    &
   7.&8  &    &   & 11.&12 &    &   & 15.&16 &    &   &
  19.&20 &    &   & 23.&24 &    &   & 27.&28 &    &   &
     &   &
  \\
\nopagebreak
  & 13 &  7 &
  \mc{5} & \mc{6} & \mc{1} & \mc{2} & \mc{4} & \mc{5} &
  \mc{7} & \mc{1} & \mc{3} & \mc{4} & \mc{6} & \mc{7} &
  \mc{0} &
  4755  & 161 &  75 & A & 16&Iul \\
\nopagebreak
%
\streep
  &    &    &
     &   &  1.&2  &    &   &  5.&6  &    &   &  9.&10 &
     &   & 13.&14 &    &   & 17.&18 &    &   & 21.&22 &
     &   &
  \\
\nopagebreak
\da & 14 &  8 &
  \mc{2} & \mc{4} & \mc{5} & \mc{7} & \mc{1} & \mc{3} &
  \mc{4} & \mc{6} & \mc{7} & \mc{2} & \mc{3} & \mc{5} &
  \mc{6} &
  5139  & 174 &  81 & G F &  4&Iul \\
\nopagebreak
%
\streep
  &    &    &
  25.&26 &    &   & 29.&30 &    &   &    &   &  3.&4  &
     &   &  7.&8  &    &   & 11.&12 &    &   & 15.&16 &
     &   &
  \\
\nopagebreak
  & 15 &  9 &
  \mc{1} & \mc{2} & \mc{4} & \mc{5} & \mc{7} & \mc{2} &
  \mc{3} & \mc{5} & \mc{6} & \mc{1} & \mc{2} & \mc{4} &
  \mc{0} &
  5493  & 186 &  87 & E & 23&Iul \\
\nopagebreak
%
\streep
  &    &   &
     &   & 18.&19 &    &   & 22.&23 &    &   & 26.&27 &
     &   & 30.&1  &    &   &    &   &  4.&5  &    &   &
     &   &
  \\
\nopagebreak
  & 16 & 10 &
  \mc{5} & \mc{7} & \mc{1} & \mc{3} & \mc{4} & \mc{6} &
  \mc{7} & \mc{2} & \mc{3} & \mc{5} & \mc{7} & \mc{1} &
  \mc{0} &
  5848  & 198 &  92 & D &  12&Iul \\
%\nopagebreak
%
\streep
  &    &    &
   8.&9  &    &   & 12.&13 &    &   & 16.&17 &    &   &
  20.&21 &    &   & 24.&25 &    &   & 28.&29 &    &   &
     &   &
  \\
\nopagebreak
\da & 17 & 11 &
  \mc{3} & \mc{4} & \mc{6} & \mc{7} & \mc{2} & \mc{3} &
  \mc{5} & \mc{6} & \mc{1} & \mc{2} & \mc{4} & \mc{5} &
  \mc{7} &
  6232  & 211 &  98 & C &  2&Iul \\
\nopagebreak
%
\streep
  &    &    &
   2.&3  &    &   &  6.&7  &    &   & 10.&11 &    &   &
  14.&15 &    &   & 18.&19 &    &   & 22.&23 &    &   &
     &   &
  \\
\nopagebreak
  & 18 & 12 &
  \mc{2} & \mc{3} & \mc{5} & \mc{6} & \mc{1} & \mc{2} &
  \mc{4} & \mc{5} & \mc{7} & \mc{1} & \mc{3} & \mc{4} &
  \mc{0} &
  6586  & 223 & 104 & B A &  20&Iul \\
\nopagebreak
%
\streep
  &    &    &
  26.&27 &    &   & 30.&1  &    &   &    &   &  4.&5  &
     &   &  8.&9  &    &   & 12.&13 &    &   & 15.&16 &
     &   &
  \\
\nopagebreak
  & 19 & 13 &
  \mc{6} & \mc{7} & \mc{2} & \mc{3} & \mc{5} & \mc{7} &
  \mc{1} & \mc{3} & \mc{4} & \mc{6} & \mc{7} & \mc{2} &
  \mc{0} &
  6940  & 235 & 110 & G &  9&Iul \\
\nopagebreak
%
\streep
  &    &   &
     &   & 19.&20 &    &   & 23.&24 &    &   & 27.&28 &
     &   &    &   &  1.&2  &    &   &  5.&6  &    &   &
   9.&10 &
  \\
\nopagebreak
\da & 20 & 14 &
  \mc{3} & \mc{5} & \mc{6} & \mc{1} & \mc{2} & \mc{4} &
  \mc{5} & \mc{7} & \mc{2} & \mc{3} & \mc{5} & \mc{6} &
  \mc{1} &
  7324  & 248 & 116 & F &  28&Iun \\
\nopagebreak
%
\streep
  &    &   &
     &   & 13.&14 &    &   & 17.&18 &    &   & 21.&22 &
     &   & 25.&26 &    &   & 29.&30 &    &   &    &   &
     &   &
  \\
\nopagebreak
  & 21 & 15 &
  \mc{2} & \mc{4} & \mc{5} & \mc{7} & \mc{1} & \mc{3} &
  \mc{4} & \mc{6} & \mc{7} & \mc{2} & \mc{3} & \mc{5} &
  \mc{0} &
  7679  & 260 & 121 & E &  17&Iul \\
\nopagebreak
%
\streep
  &    &    &
   3.&4  &    &   &  7.&8  &    &   & 11.&12 &    &   &
  15.&16 &    &   & 19.&20 &    &   & 23.&24 &    &   &
  27.&28 &
  \\
\nopagebreak
\da & 22 & 16 &
  \mc{7} & \mc{1} & \mc{3} & \mc{4} & \mc{6} & \mc{7} &
  \mc{2} & \mc{3} & \mc{5} & \mc{6} & \mc{1} & \mc{2} &
  \mc{4} &
  8062  & 273 & 128 & D C &   6&Iul \\
\nopagebreak
%
\streep
  &    &    &
     &   &    &   &  1.&2  &    &   &  5.&6  &    &   &
   9.&10 &    &   & 12.&13 &    &   & 16.&17 &    &   &
     &   &
  \\
\nopagebreak
  & 23 & 17 &
  \mc{5} & \mc{7} & \mc{2} & \mc{3} & \mc{5} & \mc{6} &
  \mc{1} & \mc{2} & \mc{4} & \mc{5} & \mc{7} & \mc{1} &
  \mc{0} &
  8417  & 285 & 133 & B &  24&Iul \\
\nopagebreak
%
\streep
  &    &    &
  20.&21 &    &   & 24.&25 &    &   & 28.&29 &    &   &
     &   &  2.&3  &    &   &  6.&7 &    &   & 10.&11 &
     &   &
  \\
\nopagebreak
  & 24 & 18 &
  \mc{3} & \mc{4} & \mc{6} & \mc{7} & \mc{2} & \mc{3} &
  \mc{5} & \mc{7} & \mc{1} & \mc{3} & \mc{4} & \mc{6} &
  \mc{0} &
  8771  & 297 & 139 & A & 14&Iul \\
\nopagebreak
%
\streep
  &    &   &
     &   & 14.&15 &    &   & 18.&19 &    &   & 22.&23 &
     &   & 26.&27 &    &   & 30.&1  &    &   &    &   &
   4.&5  &
  \\
\nopagebreak
\da & 25 & 19 &
  \mc{7} & \mc{2} & \mc{3} & \mc{5} & \mc{6} & \mc{1} &
  \mc{2} & \mc{4} & \mc{5} & \mc{7} & \mc{1} & \mc{3} &
  \mc{5} &
  9155  & 310 & 145 & G &   3&Iul \\
\nopagebreak
%
\streep
  &    &    &
     &   &  8.&9  &    &   & 12.&13 &    &   & 16.&17 &
     &   & 20.&21 &    &   & 24.&25 &    &   & 28.&29 &
     &   &
  \\
\nopagebreak
  & 26 &  1 &
  \mc{6} & \mc{1} & \mc{2} & \mc{4} & \mc{5} & \mc{7} &
  \mc{1} & \mc{3} & \mc{4} & \mc{6} & \mc{7} & \mc{2} &
  \mc{0} &
  9509  & 322 & 151 & F E & 21&Iul \\
\nopagebreak
%
\streep
  &    &    &
     &   &    &   &  2.&3  &    &   &  6.&7  &    &   &
   9.&10 &    &   & 13.&14 &    &   & 17.&18 &    &   &
     &   &
  \\
\nopagebreak
  & 27 &  2 &
  \mc{3} & \mc{5} & \mc{7} & \mc{1} & \mc{3} & \mc{4} &
  \mc{6} & \mc{7} & \mc{2} & \mc{3} & \mc{5} & \mc{6} &
  \mc{0} &
  9864  & 334 & 156 & D &  10&Iul \\
\nopagebreak
%
\streep
  &    &    &
  21.&22 &    &   & 25.&26 &    &   & 29.&30 &    &   &
     &   &  3.&4  &    &   &  7.&8 &    &   & 11.&12 &
     &   &
  \\
\nopagebreak
\da & 28 &  3 &
  \mc{1} & \mc{2} & \mc{4} & \mc{5} & \mc{7} & \mc{1} &
  \mc{3} & \mc{5} & \mc{6} & \mc{1} & \mc{2} & \mc{4} &
  \mc{5} &
 10248  & 347 & 162 & C & 30&Iun \\
\nopagebreak
%
\streep
  &    &    &
  15.&16 &    &   & 19.&20 &    &   & 23.&24 &    &   &
  27.&28 &    &   &    &   &  1.&2  &    &   &  5.&6  &
     &   &
  \\
\nopagebreak
  & 29 &  4 &
  \mc{7} & \mc{1} & \mc{3} & \mc{4} & \mc{6} & \mc{7} &
  \mc{2} & \mc{3} & \mc{5} & \mc{7} & \mc{1} & \mc{3} &
  \mc{0} &
 10602  & 359 & 168 & B & 19&Iul \\
\nopagebreak
%
\streep
  &    &    &
     &   &  9.&10 &    &   & 13.&14 &    &   & 17.&18 &
     &   & 21.&22 &    &   & 25.&26 &    &   & 29.&30 &
     &   &
  \\
\nopagebreak
\da & 30 &  5 &
  \mc{4} & \mc{6} & \mc{7} & \mc{2} & \mc{3} & \mc{5} &
  \mc{6} & \mc{1} & \mc{2} & \mc{4} & \mc{5} & \mc{7} &
  \mc{1} &
 10986  & 372 & 174 & A G &  7&Iul \\
\nopagebreak
%
\streep
  &    &    &
     &   &  3.&4  &    &   &  6.&7  &    &   & 10.&11 &
     &   & 14.&15 &    &   & 18.&19 &    &   & 22.&23 &
     &   &
  \\
\nopagebreak
  & 31 &  6 &
  \mc{3} & \mc{5} & \mc{6} & \mc{1} & \mc{2} & \mc{4} &
  \mc{5} & \mc{7} & \mc{1} & \mc{3} & \mc{4} & \mc{6} &
  \mc{0} &
 11340  & 384 & 180 & F & 26&Iul \\
\nopagebreak
%
\streep
  &    &   &
     &   & 26.&27 &    &   & 30.&1  &    &   &    &   &
   4.&5  &    &   &  8.&9  &    &   & 12.&13 &    &   &
     &   &
  \\
\nopagebreak
  & 32 &  7 &
  \mc{7} & \mc{2} & \mc{3} & \mc{5} & \mc{6} & \mc{1} &
  \mc{3} & \mc{4} & \mc{6} & \mc{7} & \mc{2} & \mc{3} &
  \mc{0} &
 11695  & 396 & 185 & E &  15&Iul \\
\nopagebreak
%
\streep
  &    &    &
  16.&17 &    &   & 20.&21 &    &   & 24.&25 &    &   &
  28.&29 &    &   &    &   &  2.&3  &    &   &  6.&7  &
     &   &
  \\
\nopagebreak
\da & 33 &  8 &
  \mc{5} & \mc{6} & \mc{1} & \mc{2} & \mc{4} & \mc{5} &
  \mc{7} & \mc{1} & \mc{3} & \mc{5} & \mc{6} & \mc{1} &
  \mc{2} &
 12079  & 409 & 191 & D &  5&Iul \\
\nopagebreak
%
\streep
  &    &    &
  10.&11 &    &   & 14.&15 &    &   & 18.&19 &    &   &
  22.&23 &    &   & 26.&27 &    &   & 30.&1  &    &   &
     &   &
  \\
\nopagebreak
  & 34 &  9 &
  \mc{4} & \mc{5} & \mc{7} & \mc{1} & \mc{3} & \mc{4} &
  \mc{6} & \mc{7} & \mc{2} & \mc{3} & \mc{5} & \mc{6} &
  \mc{0} &
 12433  & 421 & 197 & C B &  23&Iul \\
\nopagebreak
%
\streep
  &    &    &
     &   &  3.&4  &    &   &  7.&8  &    &   & 11.&12 &
     &   & 15.&16 &    &   & 19.&20 &    &   & 23.&24 &
     &   &
  \\
\nopagebreak
  & 35 & 10 &
  \mc{1} & \mc{3} & \mc{4} & \mc{6} & \mc{7} & \mc{2} &
  \mc{3} & \mc{5} & \mc{6} & \mc{1} & \mc{2} & \mc{4} &
  \mc{0} &
 12787  & 433 & 203 & A & 12&Iul \\
\nopagebreak
%
\streep
  &    &   &
     &   & 27.&28 &    &   &    &   &  1.&2  &    &   &
   5.&6  &    &   &  9.&10 &    &   & 13.&14 &    &   &
  17.&18 &
  \\
\nopagebreak
\da & 36 & 11 &
  \mc{5} & \mc{7} & \mc{1} & \mc{3} & \mc{5} & \mc{6} &
  \mc{1} & \mc{2} & \mc{4} & \mc{5} & \mc{7} & \mc{1} &
  \mc{3} &
 13171  & 446 & 209 & G & Ka.&Iul \\
\nopagebreak
%
\streep
  &    &    &
     &   & 21.&22 &    &   & 25.&26 &    &   & 29.&30 &
     &   &    &   &  3.&4  &    &   &  7.&8  &    &   &
     &   &
  \\
\nopagebreak
  & 37 & 12 &
  \mc{4} & \mc{6} & \mc{7} & \mc{2} & \mc{3} & \mc{5} &
  \mc{6} & \mc{1} & \mc{3} & \mc{4} & \mc{6} & \mc{7} &
  \mc{0} &
 13526  & 458 & 214 & F & 20&Iul \\
\nopagebreak
%
\streep
  &    &    &
  11.&12 &    &   & 15.&16 &    &   & 19.&20 &    &   &
  23.&24 &    &   & 27.&28 &    &   & 30.&1  &    &   &
     &   &
  \\
\nopagebreak
  & 38 & 13 &
  \mc{2} & \mc{3} & \mc{5} & \mc{6} & \mc{1} & \mc{2} &
  \mc{4} & \mc{5} & \mc{7} & \mc{1} & \mc{3} & \mc{4} &
  \mc{0} &
 13880  & 470 & 220 & E D &   9&Iul \\
\nopagebreak
% page 90
\streep
  &    &    &
     &   &  4.&5  &    &   &  8.&9  &    &   & 12.&13 &
     &   & 16.&17 &    &   & 20.&21 &    &   & 24.&25 &
     &   &
  \\
\nopagebreak
\da & 39 & 14 &
  \mc{6} & \mc{1} & \mc{2} & \mc{4} & \mc{5} & \mc{7} &
  \mc{1} & \mc{3} & \mc{4} & \mc{6} & \mc{7} & \mc{2} &
  \mc{3} &
 14264  & 483 & 226 & C & 28&Iun \\
\nopagebreak
%
\streep
  &    &    &
  28.&29 &    &   &    &   &  2.&3  &    &   &  6.&7&
% '7' not visible in the scan we use. Is visible in other scans and editions
     &   & 10.&11 &    &   & 14.&15 &    &   & 18.&19 &
     &   &
  \\
\nopagebreak
  & 40 & 15 &
  \mc{5} & \mc{6} & \mc{1} & \mc{3} & \mc{4} & \mc{6} &
  \mc{7} & \mc{2} & \mc{3} & \mc{5} & \mc{6} & \mc{1} &
  \mc{0} &
 14618  & 495 & 231 & B & 17&Iul \\
\nopagebreak
%
\streep
  &    &    &
     &   & 22.&23 &    &   & 26.&27 &    &   & 30.&1 &
     &   &    &   &  4.&5  &    &   &  8.&9  &    &   &
  12.&13 &
  \\
\nopagebreak
\da & 41 & 16 &
  \mc{2} & \mc{4} & \mc{5} & \mc{7} & \mc{1} & \mc{3} &
  \mc{4} & \mc{6} & \mc{1} & \mc{2} & \mc{4} & \mc{5} &
  \mc{7} &
 15002  & 508 & 238 & A &  6&Iul \\
\nopagebreak
%
\streep
  &    &    &
     &   & 16.&17 &    &   & 20.&21 &    &   & 24.&25 &
     &   & 27.&28 &    &   &    &   &  1.&2  &    &   &
     &   &
  \\
\nopagebreak
  & 42 & 17 &
  \mc{1} & \mc{3} & \mc{4} & \mc{6} & \mc{7} & \mc{2} &
  \mc{3} & \mc{5} & \mc{6} & \mc{1} & \mc{3} & \mc{4} &
  \mc{0} &
 15357  & 520 & 243 & G F & 24&Iul \\
\nopagebreak
%
\streep
  &    &    &
   5.&6  &    &   &  9.&10 &    &   & 13.&14 &    &   &
  17.&18 &    &   & 21.&22 &    &   & 25.&26 &    &   &
     &   &
  \\
\nopagebreak
  & 43 & 18 &
  \mc{6} & \mc{7} & \mc{2} & \mc{3} & \mc{5} & \mc{6} &
  \mc{1} & \mc{2} & \mc{4} & \mc{5} & \mc{7} & \mc{1} &
  \mc{0} &
 15711  & 532 & 249 & E &  14&Iul \\
\nopagebreak
%
\streep
  &    &    &
  29.&30 &    &   &    &   &  3.&4  &    &   &  7.&8  &
     &   & 11.&12 &    &   & 15.&16 &    &   & 19.&20 &
     &   &
  \\
\nopagebreak
\da & 44 & 19 &
  \mc{3} & \mc{4} & \mc{6} & \mc{1} & \mc{2} & \mc{4} &
  \mc{5} & \mc{7} & \mc{1} & \mc{3} & \mc{4} & \mc{6} &
  \mc{7} &
 16095  & 545 & 255 & D &  3&Iul \\
\nopagebreak
%
\streep
  &    &    &
  23.&24 &    &   & 27.&28 &    &   &    &   &  1.&2  &
     &   &  5.&6  &    &   &  9.&10 &    &   & 13.&14 &
     &   &
  \\
\nopagebreak
  & 45 &  1 &
  \mc{2} & \mc{3} & \mc{5} & \mc{6} & \mc{1} & \mc{3} &
  \mc{4} & \mc{6} & \mc{7} & \mc{2} & \mc{3} & \mc{5} &
  \mc{0} &
 16449  & 557 & 261 & C & 22&Iul \\
\nopagebreak
%
\streep
  &    &    &
     &   & 17.&18 &    &   & 21.&22 &    &   & 24.&25 &
% ".&18" more clear in 1598 edition
     &   & 28.&29 &    &   &    &   &  2.&3  &    &   &
     &   &
  \\
\nopagebreak
  & 46 &  2 &
  \mc{6} & \mc{1} & \mc{2} & \mc{4} & \mc{5} & \mc{7} &
  \mc{1} & \mc{3} & \mc{4} & \mc{6} & \mc{1} & \mc{2} &
  \mc{0} &
 16804  & 569 & 266 & B A & 10&Iul \\
\nopagebreak
%
\streep
  &    &    &
   6.&7  &    &   & 10.&11 &    &   & 14.&15 &    &   &
  18.&19 &    &   & 22.&23 &    &   & 26.&27 &    &   &
     &   &
  \\
\nopagebreak
\da & 47 &  3 &
  \mc{4} & \mc{5} & \mc{7} & \mc{1} & \mc{3} & \mc{4} &
  \mc{6} & \mc{7} & \mc{2} & \mc{3} & \mc{5} & \mc{6} &
  \mc{1} &
 17188  & 582 & 272 & G &  30&Iun \\
\nopagebreak
%
\streep
  &    &    &
   3.&4  &    &   &  4.&5  &    &   &  8.&9  &    &   &
  12.&13 &    &   & 16.&17 &    &   & 20.&21 &    &   &
     &   &
  \\
\nopagebreak
  & 48 &  4 &
  \mc{3} & \mc{4} & \mc{6} & \mc{7} & \mc{1} & \mc{3} &
  \mc{5} & \mc{6} & \mc{1} & \mc{2} & \mc{4} & \mc{5} &
  \mc{0} &
 17542  & 594 & 278 & F &  19&Iul \\
\nopagebreak
%
\streep
  &    &    &
  24.&25 &    &   & 28.&29 &    &   &    &   &  2.&3  &
     &   &  6.&7  &    &   & 10.&11 &    &   & 14.&15 &
     &   &
  \\
\nopagebreak
\da & 49 &  5 &
  \mc{7} & \mc{1} & \mc{3} & \mc{4} & \mc{6} & \mc{1} &
  \mc{2} & \mc{4} & \mc{5} & \mc{7} & \mc{1} & \mc{3} &
  \mc{4} &
 17926  & 607 & 284 & E &  8&Iul \\
\nopagebreak
%
\streep
  &    &    &
  18.&19 &    &   & 21.&22 &    &   & 25.&26 &    &   &
  29.&30 &    &   &    &   &  3.&4  &    &   &  7.&8  &
     &   &
  \\
\nopagebreak
  & 50 &  6 &
  \mc{6} & \mc{7} & \mc{2} & \mc{3} & \mc{5} & \mc{6} &
  \mc{1} & \mc{2} & \mc{4} & \mc{6} & \mc{7} & \mc{2} &
  \mc{0} &
 18280  & 619 & 290 & D C &  26&Iul \\
\nopagebreak
%
\streep
  &    &    &
     &   & 11.&12 &    &   & 15.&16 &    &   & 19.&20 &
     &   & 23.&24 &    &   & 27.&28 &    &   &    &   &
     &   &
  \\
\nopagebreak
  & 51 &  7 &
  \mc{3} & \mc{5} & \mc{6} & \mc{1} & \mc{2} & \mc{4} &
  \mc{5} & \mc{7} & \mc{1} & \mc{3} & \mc{4} & \mc{6} &
  \mc{0} &
 18635  & 631 & 295 & B & 15&Iul \\
\nopagebreak
%
\streep
  &    &    &
   1.&2  &    &   &  5.&6  &    &   &  9.&10 &    &   &
  13.&14 &    &   & 17.&18 &    &   & 21.&22 &    &   &
  25.&26 &
  \\
\nopagebreak
\da & 52 &  8 &
  \mc{1} & \mc{2} & \mc{4} & \mc{5} & \mc{7} & \mc{1} &
  \mc{3} & \mc{4} & \mc{6} & \mc{7} & \mc{2} & \mc{3} &
  \mc{5} &
 19018  & 644 & 302 & A &   5&Iul \\
% '644' clearer in 1598 edition
\nopagebreak
%
\streep
  &    &   &
     &   & 29.&30 &    &   &    &   &  3.&4  &    &   &
   7.&8  &    &   & 11.&12 &    &   & 15.&16 &    &   &
     &   &
  \\
\nopagebreak
  & 53 &  9 &
  \mc{6} & \mc{1} & \mc{2} & \mc{4} & \mc{6} & \mc{7} &
  \mc{2} & \mc{3} & \mc{4} & \mc{6} & \mc{1} & \mc{2} &
  \mc{0} &
 19373  & 656 & 307 & G &  23&Iul \\
\nopagebreak
%
\streep
  &    &    &
  18.&19 &    &   & 22.&23 &    &   & 26.&27 &    &   &
  30.&1  &    &   &    &   &  4.&5  &    &   &  8.&9  &
     &   &
  \\
\nopagebreak
  & 54 & 10 &
  \mc{4} & \mc{5} & \mc{7} & \mc{1} & \mc{3} & \mc{4} &
  \mc{6} & \mc{7} & \mc{2} & \mc{4} & \mc{5} & \mc{7} &
  \mc{0} &
 19727  & 668 & 313 & F E &  12&Iul \\
\nopagebreak
%
\streep
  &    &    &
     &   & 12.&13 &    &   & 12.&13 &    &   & 20.&21 &
     &   & 24.&25 &    &   & 28.&29 &    &   &    &   &
   2.&3  &
  \\
\nopagebreak
\da & 55 & 11 &
  \mc{1} & \mc{3} & \mc{4} & \mc{6} & \mc{7} & \mc{2} &
  \mc{3} & \mc{5} & \mc{6} & \mc{1} & \mc{2} & \mc{4} &
  \mc{6} &
 20111  & 681 & 319 & D & Ka.&Iul \\
\nopagebreak
%
\streep
  &    &    &
     &   &  6.&7  &    &   & 10.&11 &    &   & 14.&15 &
     &   & 18.&19 &    &   & 22.&23 &    &   & 26.&27 &
     &   &
  \\
\nopagebreak
  & 56 & 12 &
  \mc{7} & \mc{2} & \mc{3} & \mc{5} & \mc{6} & \mc{1} &
  \mc{2} & \mc{4} & \mc{5} & \mc{7} & \mc{1} & \mc{3} &
  \mc{0} &
 20465  & 693 & 325 & C &  20&Iul \\
\nopagebreak
%
\streep
  &    &   &
     &   & 30.&1  &    &   &    &   &  4.&5  &    &   &
   8.&9  &    &   & 12.&13 &    &   & 15.&16 &    &   &
     &   &
  \\
\nopagebreak
  & 57 & 13 &
  \mc{4} & \mc{6} & \mc{7} & \mc{2} & \mc{4} & \mc{5} &
  \mc{7} & \mc{1} & \mc{3} & \mc{4} & \mc{6} & \mc{7} &
  \mc{0} &
 20820  & 705 & 330 & B &   9&Iul \\
\nopagebreak
%
\streep
  &    &    &
  19.&20 &    &   & 23.&24 &    &   & 27.&28 &    &   &
     &   &  1.&2  &    &   &  5.&6  &    &   &  9.&10 &
     &   &
  \\
\nopagebreak
\da & 58 & 14 &
  \mc{2} & \mc{3} & \mc{5} & \mc{6} & \mc{1} & \mc{2} &
  \mc{4} & \mc{6} & \mc{7} & \mc{2} & \mc{3} & \mc{5} &
  \mc{6} &
 21204  & 718 & 336 & A G &  28&Iun \\
\nopagebreak
%
\streep
  &    &    &
  13.&14 &    &   & 17.&18 &    &   & 21.&22 &    &   &
  25.&26 &    &   & 29.&30 &    &   &    &   &  3.&4  &
     &   &
  \\
\nopagebreak
  & 59 & 15 &
  \mc{1} & \mc{2} & \mc{4} & \mc{5} & \mc{7} & \mc{1} &
  \mc{3} & \mc{4} & \mc{6} & \mc{7} & \mc{2} & \mc{4} &
  \mc{0} &
 21558  & 730 & 342 & F &  17&Iul \\
\nopagebreak
%
\streep
  &    &    &
     &   &  7.&8  &    &   & 11.&12 &    &   & 15.&16 &
     &   & 19.&20 &    &   & 23.&24 &    &   & 27.&28 &
     &   &
  \\
\nopagebreak
\da & 60 & 16 &
  \mc{5} & \mc{7} & \mc{1} & \mc{3} & \mc{4} & \mc{6} &
  \mc{7} & \mc{2} & \mc{3} & \mc{5} & \mc{6} & \mc{1} &
  \mc{2} &
 21942  & 743 & 348 & E &   6&Iul \\
\nopagebreak
%
\streep
  &    &    &
     &   &  1.&2  &    &   &  5.&6  &    &   &  9.&10 &
     &   & 13.&14 &    &   & 17.&18 &    &   & 21.&22 &
     &   &
  \\
\nopagebreak
  & 61 & 17 &
  \mc{4} & \mc{6} & \mc{7} & \mc{2} & \mc{3} & \mc{5} &
  \mc{6} & \mc{1} & \mc{2} & \mc{4} & \mc{5} & \mc{7} &
  \mc{0} &
 22296  & 755 & 354 & D &  25&Iul \\
\nopagebreak
%
\streep
  &    &    &
     &   & 24.&25 &    &   & 28.&29 &    &   &    &   &
   2.&3  &    &   &  6.&7  &    &   & 10.&11 &    &   &
     &   &
  \\
\nopagebreak
  & 62 & 18 &
  \mc{1} & \mc{3} & \mc{4} & \mc{6} & \mc{7} & \mc{2} &
  \mc{4} & \mc{5} & \mc{7} & \mc{1} & \mc{3} & \mc{4} &
  \mc{0} &
 22631  & 767 & 359 & C B &  13&Iul \\
\nopagebreak
%
\streep
  &    &    &
  14.&15 &    &   & 18.&19 &    &   & 22.&23 &    &   &
  26.&27 &    &   & 30.&1  &    &   &    &   &  4.&5  &
     &   &
  \\
\nopagebreak
\da & 63 & 19 &
  \mc{6} & \mc{7} & \mc{2} & \mc{3} & \mc{5} & \mc{6} &
  \mc{1} & \mc{2} & \mc{4} & \mc{5} & \mc{7} & \mc{2} &
  \mc{3} &
 23035  & 780 & 365 & A &   3&Iul \\
% '365' unclear; better in other scans
\nopagebreak
%
\streep
  &    &    &
   8.&9  &    &   & 12.&13 &    &   & 16.&17 &    &   &
  20.&21 &    &   & 24.&25 &    &   & 28.&29 &    &   &
     &   &
  \\
\nopagebreak
  & 64 &  1 &
  \mc{5} & \mc{6} & \mc{1} & \mc{2} & \mc{4} & \mc{5} &
  \mc{7} & \mc{1} & \mc{3} & \mc{4} & \mc{6} & \mc{7} &
  \mc{0} &
 23389  & 792 & 371 & G &  22&Iul \\
\nopagebreak
%
\streep
  &    &    &
     &   &  2.&3  &    &   &  6.&7  &    &   &  9.&10 &
     &   & 13.&14 &    &   & 17.&18 &    &   & 21.&22 &
     &   &
  \\
\nopagebreak
  & 65 &  2 &
  \mc{2} & \mc{4} & \mc{5} & \mc{7} & \mc{1} & \mc{3} &
  \mc{4} & \mc{6} & \mc{7} & \mc{2} & \mc{3} & \mc{5} &
  \mc{0} &
 23734  & 804 & 377 & F &  11&Iul \\
\nopagebreak
%
\streep
  &    &    &
     &   & 25.&26 &    &   & 29.&30 &    &   &  3.&4  &
     &   &  7.&6  &    &   & 11.&12 &    &   & 15.&16 &
     &   &
  \\
\nopagebreak
\da & 66 &  3 &
  \mc{6} & \mc{1} & \mc{2} & \mc{4} & \mc{5} & \mc{7} &
  \mc{1} & \mc{3} & \mc{4} & \mc{6} & \mc{7} & \mc{2} &
  \mc{3} &
 24127  & 817 & 383 & E D &  29&Iun \\
\nopagebreak
%
\streep
  &    &    &
     &   & 19.&20 &    &   & 23.&24 &    &   & 27.&28 &
     &   &    &   &  1.&2  &    &   &  5.&6  &    &   &
     &   &
  \\
\nopagebreak
  & 67 &  4 &
  \mc{5} & \mc{7} & \mc{1} & \mc{3} & \mc{4} & \mc{6} &
  \mc{7} & \mc{2} & \mc{4} & \mc{5} & \mc{7} & \mc{1} &
  \mc{0} &
 24482  & 829 & 388 & C &  18&Iul \\
\nopagebreak
%
\streep
  &    &    &
   9.&10 &    &   & 13.&14 &    &   & 17.&18 &    &   &
  21.&22 &    &   & 25.&26 &    &   & 29.&30 &    &   &
     &   &
  \\
\nopagebreak
\da & 68 &  5 &
  \mc{3} & \mc{4} & \mc{6} & \mc{7} & \mc{2} & \mc{3} &
  \mc{5} & \mc{6} & \mc{1} & \mc{2} & \mc{4} & \mc{5} &
  \mc{7} &
 24866  & 842 & 394 & B &   8&Iul \\
\nopagebreak
%
\streep
  &    &    &
   3.&4  &    &   &  6.&7  &    &   & 10.&11 &    &   &
  14.&15 &    &   & 18.&19 &    &   & 22.&23 &    &   &
     &   &
  \\
\nopagebreak
  & 69 &  6 &
  \mc{2} & \mc{3} & \mc{5} & \mc{6} & \mc{1} & \mc{2} &
  \mc{4} & \mc{5} & \mc{7} & \mc{1} & \mc{3} & \mc{4} &
  \mc{0} &
 25220  & 854 & 400 & A &  27&Iul \\
% '25220' clearer in other editions and scans
\nopagebreak
%
\streep
  &    &    &
  26.&27 &    &   & 30.&1  &    &   &    &   &  4.&3\footnote{4.3: Sic.}  &
% '4.3' in the original is odd, as in all other entries
% the second number is 1 larger than the first number.
% This is nevertheless the same in other editions, so we assume this is what
% the author wrote, though we suspect it is supposed to be '4.5'.
     &   &  8.&9  &    &   & 12.&13 &    &   & 16.&17 &
     &   &
  \\
\nopagebreak
  & 70 &  7 &
  \mc{6} & \mc{7} & \mc{2} & \mc{3} & \mc{5} & \mc{7} &
  \mc{1} & \mc{3} & \mc{4} & \mc{6} & \mc{7} & \mc{2} &
  \mc{0} &
 25574  & 866 & 406 & G F & 15&Iul \\
\nopagebreak
%
\streep
  &    &    &
     &   & 20.&21 &    &   & 24.&25 &    &   & 28.&29 &
     &   &    &   &  2.&3  &    &   &  6.&7  &    &   &
  10.&11 &
  \\
\nopagebreak
\da & 71 &  8 &
  \mc{3} & \mc{5} & \mc{6} & \mc{1} & \mc{2} & \mc{4} &
  \mc{5} & \mc{7} & \mc{2} & \mc{3} & \mc{5} & \mc{6} &
  \mc{1} &
 25958  & 879 & 412 & E &   4&Iul \\
\nopagebreak
%
\streep
  &    &    &
     &   & 14.&15 &    &   & 18.&19 &    &   & 22.&23 &
     &   & 26.&27 &    &   & 30.&1  &    &   &    &   &
     &   &
  \\
\nopagebreak
  & 72 &  9 &
  \mc{2} & \mc{4} & \mc{5} & \mc{7} & \mc{1} & \mc{3} &
  \mc{4} & \mc{6} & \mc{7} & \mc{2} & \mc{3} & \mc{5} &
  \mc{0} &
 26313  & 891 & 417 & D &  23&Iul \\
\nopagebreak
%
\streep
  &    &    &
   3.&4  &    &   &  7.&8  &    &   & 11.&12 &    &   &
  15.&16 &    &   & 19.&20 &    &   & 23.&24 &    &   &
     &   &
  \\
\nopagebreak
  & 73 & 10 &
  \mc{7} & \mc{1} & \mc{3} & \mc{4} & \mc{6} & \mc{7} &
  \mc{2} & \mc{3} & \mc{5} & \mc{6} & \mc{1} & \mc{2} &
  \mc{0} &
 26667  & 903 & 423 & C &  13&Iul \\
\nopagebreak
%
\streep
  &    &    &
  27.&28 &    &   &    &   &  1.&2  &    &   &  5.&6  &
     &   &  9.&10 &    &   & 13.&14 &    &   & 17.&18 &
     &   &
  \\
\nopagebreak
\da & 74 & 11 &
  \mc{4} & \mc{5} & \mc{7} & \mc{2} & \mc{3} & \mc{5} &
  \mc{6} & \mc{1} & \mc{2} & \mc{4} & \mc{5} & \mc{7} &
  \mc{1} &
 27051  & 916 & 429 & B A &  Ka.&Iul \\
\nopagebreak
%
\streep
  &    &    &
  21.&22 &    &   & 25.&26 &    &   & 29.&30 &    &   &
     &   &  3.&4  &    &   &  7.&8  &    &   & 11.&12 &
     &   &
  \\
\nopagebreak
  & 75 & 12 &
  \mc{3} & \mc{4} & \mc{6} & \mc{7} & \mc{2} & \mc{3} &
  \mc{5} & \mc{7} & \mc{1} & \mc{3} & \mc{4} & \mc{6} &
  \mc{0} &
 27405  & 928 & 435 & G &  20&Iul \\
\nopagebreak
%
\streep
  &    &    &
     &   & 15.&16 &    &   & 19.&20 &    &   & 23.&24 &
     &   & 27.&28 &    &   & 30.&1  &    &   & 30.&1  &
     &   &
  \\
\nopagebreak
  & 76 & 13 &
  \mc{7} & \mc{2} & \mc{3} & \mc{5} & \mc{6} & \mc{1} &
  \mc{2} & \mc{4} & \mc{5} & \mc{7} & \mc{1} & \mc{3} &
  \mc{0} &
 27759  & 940 & 441 & F &   9&Iul \\
%%
\end{longtable}
\endgroup


\section{De Periodo Lunari Calippica ab Autumno}

\lipsum

\lipsum[150]

\listoftables
\end{document}
