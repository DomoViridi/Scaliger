%%% Test a table
%
% Set the \input{} to the table you want to test, then typeset this file
%
\documentclass[12pt,twoside,a4paper]{book}
\usepackage[lines=40,showframe]{geometry}
\usepackage{array} % allow row distance change with \arraystretch
%
\usepackage{lipsum}
\usepackage{rotating} % Make it possible to rotate headers
\usepackage[normalsize]{caption} % Keep captions to (long)tables normal sized
\usepackage{longtable}% Allow for split of the table over more than one page
\usepackage{booktabs} % To get nice table layout
%
% To include graphics in tables
\usepackage{graphicx}
%
%% Settings from the main text we also need for the test
\usepackage{fontspec}
\setmainfont{Hoefler Text}[]
%\setmainfont{Times New Roman}[]
\newfontfamily\greekfont{Times New Roman}
\newfontfamily\hebrewfont{Arial Hebrew}
\newfontfamily\arabicfont{Arial Unicode MS}
\newfontfamily\astrofont{Menlo}
\newcommand\astro[1]{{\astrofont #1}}
%%
\usepackage[quiet]{polyglossia}
\setmainlanguage{latin}
\setotherlanguage{greek}
\setotherlanguage{hebrew}
\setotherlanguage{arabic}
%
\newcommand{\rnum}[1]{\textsc{#1}} % Roman numerals
\newcommand{\altsep}{\slash{}} % Separator between alternative values in tables
%
\newlength{\cw} % Column width for \parbox headers
\newcommand{\ch}[2]{\settowidth{\cw}{#1}\parbox[b]{\cw}{\raggedright#2}}
%
% \ruleover{string}
% \ruleover[scale]{string}
\makeatletter
\newlength{\ROver@width} % width of the string
\newlength{\ROver@height} % height of the string
\newlength{\ROver@thick} % thickness of the line
\newcommand{\ruleover}[2][0.75]{%
  \settoheight{\ROver@height}{#2}% Get height of original text
  \settowidth{\ROver@width}{#2}% Get width of original text
  \setlength{\ROver@thick}{0.5333pt}% Set pre-shrink thickness of rule
  \setlength{\ROver@width}{#1\ROver@width}% Calculate shrunken width
  \setlength{\ROver@height}{#1\ROver@height}% Calculate shrunken height
  \makebox[0pt][l]{\scalebox{#1}{#2}}% Print the text with shrinking
  \rule[1.3333\ROver@height]{\ROver@width}{#1\ROver@thick}% Draw bar
}
\makeatother

\newcommand{\gnum}[1]{\ruleover{#1}} % Greek number
\newcommand{\gnums}[2]{\ruleover[#1]{#2}} % Scaled Greek number

\begin{document}

%% Longtable version
%%%% Liber I p43
%%
%%% Count out columns for fixed-width source font
% 000000011111111112222222222333333333344444444445555555555666666666677777777778
% 345678901234567890123456789012345678901234567890123456789012345678901234567890
%
%%
\begin{tabnums} % Select monospaced numbers
%\tiny
%\scriptsize
\footnotesize
%\small
%\normalsize
%% Modify separation between columns
\setlength{\tabcolsep}{2.4pt}
%% Modify distance between rows
%\renewcommand{\arraystretch}{1.2}
%% Define reference symbols
\newcommand{\dg}{\scriptsize †}
\newcommand{\ddg}{\scriptsize ‡}
%% Local command to define small header
\newcommand{\sh}[1]{\multicolumn{1}{c}{\tiny{#1}}}
%% Names of the months, to save us a lot of typing
\newcommand{\rabx}{Rabie prior}
\newcommand{\rabz}{Rabie posterior}
\newcommand{\giux}{Giumadi prior}
\newcommand{\giuz}{Giumadi posterior}
\newcommand{\rege}{Regeb}
\newcommand{\saha}{Sahaben}
\newcommand{\rama}{Ramadhan}
\newcommand{\scew}{Scewal}
\newcommand{\dulk}{Dulkaidathi}
\newcommand{\dulc}{Dulchagiathi}
\newcommand{\muha}{Muharam}
\newcommand{\seph}{Sephar}
%% Let longtable process the whole table in one go
\setcounter{LTchunksize}{100}
\begin{longtable}[c]{@{}r  c  c  c  c  r@{~}l l l l l@{}}
\toprule
 & \multicolumn{9}{c}{\Large\textsc{Tabula periodi magnae Hagarenorum}}\\
\toprule
% Put a reference to the first page of the table in the List of Tables
% Tables in the LoT are listed on the 'section' level
% '\numberline{\thetable}' puts the number of the table before the
% title, correctly indented.
% '\protect' helps pass these commands without error.
\addcontentsline{lot}{section}{%
\protect\numberline{\thetable}Periodi magnae Hagarenorum}
\label{tab:p112}
%%
~ &
 \sh{Anni} &
 \sh{Cyclus} &
 \sh{Character} &
 \sh{Cyclus} \\
~ &
 \sh{periodi} &
 \sh{Lunae} &
 \sh{anni} &
 \sh{Solis} &
~ & & % 31 Martii
Periodus prima &
Periodus secunda &
Periodus tertia
\\
\midrule
\endfirsthead
\toprule
&
\multicolumn{9}{c}{\Large\textsc{Residuum tabulae periodi magnae Hagarenorum}}\\
\toprule
~ &
 \sh{Anni} &
 \sh{Cyclus} &
 \sh{Character} &
 \sh{Cyclus} \\
~ &
 \sh{periodi} &
 \sh{Lunae} &
 \sh{anni} &
 \sh{Solis} &
~ & & % 31 Martii
Periodus prima &
Periodus secunda &
Periodus tertia
\\
\midrule
\endhead
%\bottomrule
  \addlinespace
  & \multicolumn{6}{l}{\super † \textgreek{ἐμβολ.[?]}}
  & \multicolumn{3}{l}{\super ‡ \textgreek{ὑπερή.[?]}}
\endfoot
\bottomrule
  \addlinespace
  & \multicolumn{6}{l}{\super † \textgreek{ἐμβολ.[?]}}
  & \multicolumn{3}{l}{\super ‡ \textgreek{ὑπερή.[?]}} \\
  \addlinespace
  % Put the table nr and title below the table, without entry in the LoT
  \caption[]{Periodi magnae Hagarenorum}
\endlastfoot
  & ~1 & 14 & 1 & F   & 31&Martii & \seph & \giuz & \scew \\
  & ~2 & 15 & 5 & E   & 20&Martii & \seph & \giuz & \scew & \ddg\\
\dg  
  & ~3 & 16 & 3 & D C &  9&Martii & \seph & \giuz & \scew \\
  & ~4 & 17 & 2 & B   & 28&Martii & \rabx & \rege & \dulk \\
\cmidrule{2-10}
  & ~5 & 18 & 6 & A   & 17&Martii & \rabx & \rege & \dulk \\
\dg
  & ~6 & 19 & 3 & G   &  6&Martii & \rabx & \rege & \dulk \\
  & ~7 & ~1 & 2 & F E & 24&Martii & \rabz & \saha & \dulc \\
\dg
  & ~8 & ~2 & 6 & D   & 13&Martii & \rabz & \saha & \dulc \\
\cmidrule{2-10}
  & ~9 & ~3 & 5 & C   &  1&April. & \giux & \rama & \muha & \ddg\\
  & 10 & ~4 & 3 & B   & 22&Martii & \giux & \rama & \muha \\
\dg
  & 11 & ~5 & 7 & A G & 10&Martii & \giux & \rama & \muha \\
  & 12 & ~6 & 6 & F   & 29&Martii & \giuz & \scew & \seph \\
\cmidrule{2-10}
  & 13 & ~7 & 3 & E   & 18&Martii & \giuz & \scew & \seph & \ddg\\
\dg
  & 14 & ~8 & 1 & D   &  8&Martii & \giuz & \scew & \seph \\
  & 15 & ~9 & 7 & C B & 26&Martii & \rege & \dulk & \rabx \\
\dg
  & 16 & 10 & 4 & A   & 15&Martii & \rege & \dulk & \rabx \\
\cmidrule{2-10}
  & 17 & 11 & 3 & G   &  3&April. & \saha & \dulc & \rabz \\
  & 18 & 12 & 7 & F   & 23&Martii & \saha & \dulc & \rabz & \ddg\\
\dg
  & 19 & 13 & 5 & E D & 12&Martii & \saha & \dulc & \rabz \\
  & 20 & 14 & 4 & C   & 31&Martii & \rama & \muha & \giux \\
\cmidrule{2-10}
  & 21 & 15 & 4 & B   & 20&Martii & \rama & \muha & \giux \\
\dg
  & 22 & 16 & 1 & A   &  9&Martii & \rama & \muha & \giux \\
  & 23 & 17 & 5 & G F & 27&Martii & \scew & \seph & \giuz & \ddg\\
  & 24 & 18 & 4 & E   & 17&Martii & \scew & \seph & \giuz \\
\cmidrule{2-10}
\dg
  & 25 & 19 & 6 & D   &  6&Martii & \scew & \seph & \giuz \\
  & 26 & ~1 & 5 & C   & 25&Martii & \dulk & \rabx & \rege \\
\dg
  & 27 & ~2 & 2 & B A & 13&Martii & \dulk & \rabx & \rege \\
  & 28 & ~3 & 1 & G   &  1&April. & \dulc & \rabz & \saha \\
\cmidrule{2-10}
  & 29 & ~4 & 5 & F   & 21&Martii & \dulc & \rabz & \saha & \ddg\\
\dg
  & 30 & ~5 & 3 & E   & 11&Martii & \dulc & \rabz & \saha \\
  & 31 & ~6 & 2 & D C & 29&Martii & \muha & \giux & \rama \\
  & 32 & ~7 & 6 & B   & 18&Martii & \muha & \giux & \rama \\
\cmidrule{2-10}
\dg
  & 33 & ~8 & 3 & A   &  7&Martii & \muha & \giux & \rama \\
  & 34 & ~9 & 3 & G   & 27&Martii & \seph & \giuz & \scew & \ddg\\
\dg
  & 35 & 10 & 7 & F E & 15&Martii & \seph & \giuz & \scew \\
  & 36 & 11 & 6 & D   &  2&April. & \rabx & \rege & \dulk \\
\cmidrule{2-10}
  & 37 & 12 & 3 & C   & 23&Martii & \rabx & \rege & \dulk \\
\dg
  & 38 & 13 & 7 & B   & 12&Martii & \rabx & \rege & \dulk \\
  & 39 & 14 & 6 & A G & 30&Martii & \rabz & \saha & \dulc & \ddg\\
  & 40 & 15 & 4 & F   & 20&Martii & \rabz & \saha & \dulc \\
\cmidrule{2-10}
\dg
  & 41 & 16 & 1 & E   &  9&Martii & \rabz & \saha & \dulc \\
  & 42 & 17 & 7 & D   & 28&Martii & \giux & \rama & \muha \\
  & 43 & 18 & 4 & C B & 16&Martii & \giux & \rama & \muha & \ddg\\
\dg
  & 44 & 19 & 2 & A   &  6&Martii & \giux & \rama & \muha \\
\cmidrule{2-10}
  & 45 & ~1 & 1 & G   & 25&Martii & \giuz & \scew & \seph \\
\dg
  & 46 & ~2 & 5 & F   & 14&Martii & \giuz & \scew & \seph \\
  & 47 & ~3 & 4 & E D &  1&April. & \rege & \dulk & \rabx \\
  & 48 & ~4 & 1 & C   & 21&Martii & \rege & \dulk & \rabx \\
\cmidrule{2-10}
\dg
  & 49 & ~5 & 5 & B   & 10&Martii & \rege & \dulk & \rabx \\
  & 50 & ~6 & 4 & A   & 29&Martii & \saha & \dulc & \rabz & \ddg\\
  & 51 & ~7 & 2 & G F & 18&Martii & \saha & \dulc & \rabz \\
\dg
  & 52 & ~8 & 6 & E   &  7&Martii & \saha & \dulc & \rabz \\
\cmidrule{2-10}
  & 53 & ~9 & 5 & D   & 26&Martii & \rama & \muha & \giux \\
\dg
  & 54 & 10 & 2 & C   & 15&Martii & \rama & \muha & \giux \\
  & 55 & 11 & 1 & B A &  2&April. & \scew & \seph & \giuz & \ddg\\
  & 56 & 12 & 6 & G   & 23&Martii & \scew & \seph & \giuz \\
\cmidrule{2-10}
\dg
  & 57 & 13 & 3 & F   & 12&Martii & \scew & \seph & \giuz \\
  & 58 & 14 & 2 & E   & 31&Martii & \dulk & \rabx & \rege \\
  & 59 & 15 & 6 & D C & 19&Martii & \dulk & \rabx & \rege & \ddg\\
\dg
  & 60 & 16 & 4 & B   &  9&Martii & \dulk & \rabx & \rege \\
\cmidrule{2-10}
  & 61 & 17 & 3 & A   & 28&Martii & \dulc & \rabz & \saha \\
  & 62 & 18 & 7 & G   & 17&Martii & \dulc & \rabz & \saha \\
\dg
  & 63 & 19 & 4 & F E &  5&Martii & \dulc & \rabz & \saha \\
  & 64 & ~1 & 3 & D   & 24&Martii & \muha & \giux & \rama & \ddg\\
\cmidrule{2-10}
\dg
  & 65 & ~2 & 1 & C   & 14&Martii & \muha & \giux & \rama \\
  & 66 & ~3 & 7 & B   &  2&April. & \seph & \giuz & \scew \\
  & 67 & ~4 & 4 & A G & 21&Martii & \seph & \giuz & \scew \\
\dg
  & 68 & ~5 & 1 & F   & 10&Martii & \seph & \giuz & \scew \\
\cmidrule{2-10}
  & 69 & ~6 & 7 & E   & 29&Martii & \rabx & \rege & \dulk & \ddg\\
  & 70 & ~7 & 5 & D   & 18&Martii & \rabx & \rege & \dulk \\
\dg
  & 71 & ~8 & 2 & C B &  7&Martii & \rabx & \rege & \dulk \\
  & 72 & ~9 & 1 & A   & 26&Martii & \rabz & \saha & \dulc \\
\cmidrule{2-10}
\dg
  & 73 & 10 & 5 & G   & 15&Martii & \rabz & \saha & \dulc \\
  & 74 & 11 & 4 & F   &  3&April. & \giux & \rama & \muha \\
  & 75 & 12 & 1 & E D & 22&Martii & \giux & \rama & \muha \\
\dg
  & 76 & 13 & 5 & C   & 11&Martii & \giux & \rama & \muha & \ddg\\
\end{longtable}
\end{tabnums}

%% Regular table version
%\begin{table}[htbp]
%  %%% Liber II p117, PDF 200
%%
%%% Count out columns for fixed-width source font
% 000000011111111112222222222333333333344444444445555555555666666666677777777778
% 345678901234567890123456789012345678901234567890123456789012345678901234567890
%
%% Select a general font size (uncomment one from the list)
%\tiny
%\scriptsize
\footnotesize
%\small
%\normalsize
%% Center the whole table left-right
\centering
%% Modify separation between columns
\setlength{\tabcolsep}{2.1pt}
%% Modify distance between rows
%\renewcommand{\arraystretch}{1.3}
%
\newlength{\cw}
%% Parbox column header \ch{sizetext}{fontsize}{text}
\newcommand{\ch}[3]{\settowidth{\cw}{#1}\parbox[b]{\cw}{#2{#3}}}
%%
\begin{tabular}{@{}l c r r c l r@{~}l l r@{~}l c@{}}
\toprule
 \multicolumn{12}{c}{\Large\textsc{Tabella noviluniorum Samaritanorum}}\\
 \multicolumn{12}{c}{\large\textsc{in anno Christi \rnum{mdlxxxiiii}}}
\\
\toprule
 \ch{Giumedi posterior}{\scriptsize}{Menses Lunares} &
 \ch{\scriptsize Feria}{\scriptsize}{Feria} &
 \ch{\scriptsize Horae}{\scriptsize}{Horae} &
 \ch{180}{\tiny}{Scru\-pu\-la\\1800} &
 &
 \ch{Marchesban 28}{\scriptsize}{Menses Samarit. Iuliani} &
 & &
 \ch{Marchesban}{\scriptsize}{Menses Iuliani Samarit.} &
 \multicolumn{2}{l}{\ch{29 Octobr.}{\scriptsize}{Neomenia in mensibus Iul.}} &
 \ch{\tiny eniarum}{\tiny}{Feria Neomeniarum}
\\
\midrule
  Dulchagia &
  2.~3 &
  10. &
  180 &
  D &
  Adar 6 &
  3&Martii &
  Adar &
  26&Feb. &
  5
\\
  Muharram &
  4 &
  3. &
  180 &
  N &
  Pesah Nisan 4 &
  1&April. &
  Nisan &
  29&Martii &
  1
\\
  Sephar &
  6 &
  6. &
  130 &
  N &
  Iiar 4 &
  1&Maii &
  Iiar &
  28&April. &
  3
\\
  Rabie prior &
  7.~1 &
  10. &
  1 &
  D &
  3 Siban &
  31&Maii &
  Siban &
  29&Maii &
  6
\\
  Rabie posterior &
  2 &
  11. &
  2 &
  N &
  Tamuz 2 &
  29&Iunii &
  Tamuz &
  28&Iunii &
  1
\\
  Giumedi prior &
  3.~4 &
  11. &
  152 &
  D &
  Ab 1 &
  28&Iulii &
  Ab &
  29&Iulii &
  4
\\
  Giumedi posterior &
  5 &
  9. &
  10 &
  N &
  Ab 30 &
  27&Aug. &
  Ilul &
  29&Aug. &
  7
\\
  Regeb &
  6.~7 &
  8. &
  152 &
  D &
  Hag Ilul 29 &
  26&Sept. &
  Tisri &
  28&Septem. &
  2
\\
  Sahaben &
  1 &
  5. &
  2 &
  N &
  Tisri 28 &
  25&Octobr. &
  Marchesban &
  29&Octobr. &
  5
\\
  Ramadhan &
  2 &
  5. &
  1 &
  D &
  Marchesban 28 &
  23&Novemb. &
  Caslim &
  28&Nov. &
  7
\\
  Schevval &
  4 &
  5. &
  1 &
  N &
  Caslim 26 &
  23&Decem. &
  Tebith &
  29&Decem. &
  3
\\
  Dulkaida &
  5 &
  3. &
  10 &
  D &
  Teibeth 24 &
  21&Ianuarii &
  Scebat &
  29&Ian. &
  6
\\
\bottomrule
\end{tabular}
%
\caption{Noviluniorum Samaritanorum in anno Christi 1584}
\label{tab:p117}
%
%\end{table}

Banana

\the\bigskipamount

\the\LTpre

%\setlength{\LTpre}{3\bigskipamount plus \bigskipamount minus \bigskipamount}
\the\LTpre


\clearpage
\lipsum

Sed commodo posuere pede. Mauris ut est. Ut quis purus. Sed ac odio.
Sed vehi- cula hendrerit sem. Duis non odio. Morbi ut dui.
Sed accumsan risus eget odio. In hac habitasse platea dictumst.
Pellentesque non elit. Fusce sed justo eu urna porta tincidunt.
Mauris felis odio, sollicitudin sed, volutpat a, ornare ac, erat.
Morbi quis dolor. Donec pellentesque, erat ac sagittis semper,
nunc dui lobortis purus, quis congue purus metus ultricies tellus.
Proin et quam. Class aptent taciti sociosqu ad litora torquent per
conubia nostra, per inceptos hymenaeos.
Praesent sapien turpis, fermentum vel, eleifend faucibus, vehicula eu,
lacus.
Sed commodo posuere pede. Mauris ut est. Ut quis purus. Sed ac odio.
Sed vehi- cula hendrerit sem. Duis non odio. Morbi ut dui.
Sed accumsan risus eget odio. In hac habitasse platea dictumst.
Pellentesque non elit. Fusce sed justo eu urna porta tincidunt.
Sed commodo posuere pede. Mauris ut est. Ut quis purus. Sed ac odio.
Sed vehi- cula hendrerit sem. Duis non odio. Morbi ut dui.
Sed accumsan risus eget odio. In hac habitasse platea dictumst.
Pellentesque non elit. Fusce sed justo eu urna porta tincidunt.
Pellentesque non elit. Fusce sed justo eu urna porta tincidunt.
%AAPellentesque non elit. Fusce sed justo eu urna porta tincidunt.

%\bigskip
\smallskip
%%% Liber II p89-90
%%
%%% Count out columns for fixed-width source font
% 000000011111111112222222222333333333344444444445555555555666666666677777777778
% 345678901234567890123456789012345678901234567890123456789012345678901234567890
%
\begingroup
\tiny
%\scriptsize
%\footnotesize
%\small
%\normalsize
%% Modify separation between columns
\setlength{\tabcolsep}{2.5pt}
%% Modify distance between rows
\renewcommand{\arraystretch}{0.9}
%% Let longtable process the whole table in one go
\setcounter{LTchunksize}{100}
\begin{longtable}[c]{@{}%
 c c c  r@{~}l r@{~}l r@{~}l r@{~}l r@{~}l r@{~}l
r@{~}l r@{~}l r@{~}l r@{~}l r@{~}l r@{~}l r@{~}l  c c c c r@{~}l
@{}}
\caption{Tabula neomeniarum periodi Calippicae}\\
\toprule
% Read the header description from an external file
% Header for table p89-90
% Version with slanted headers for the names of the months
~ &
\begin{turn}{90}Anni periodi\end{turn} &
\begin{turn}{90}Cyclus Lunae\end{turn} & 

\begin{rotate}{75}\textgreek{Εκατομβαιών}\end{rotate} & &
\begin{rotate}{75}\textgreek{Μεταγειτνιών}\end{rotate} & &
\begin{rotate}{75}\textgreek{Βοηδρομιών}\end{rotate} & &

\begin{rotate}{75}\textgreek{Πυανεψιών}\end{rotate} & &
\begin{rotate}{75}\textgreek{Μαιμακτηριών}\end{rotate} & &
\begin{rotate}{75}\textgreek{Ποσειδεών}\end{rotate} & &

\begin{rotate}{75}\textgreek{Γαμηλιών}\end{rotate} & &
\begin{rotate}{75}\textgreek{Ανθεστηριών}\end{rotate} & &
\begin{rotate}{75}\textgreek{Ελαφηβολιών}\end{rotate} & &

\begin{rotate}{75}\textgreek{Μουνυχιών}\end{rotate} & &
\begin{rotate}{75}\textgreek{Θαργηλιών}\end{rotate} & &
\begin{rotate}{75}\textgreek{Σκιῤῥοφοριών α}\end{rotate} & &
\begin{rotate}{75}\textgreek{Σκιῤῥοφοριών β}\end{rotate} & &

\multicolumn{1}{c}{\begin{turn}{90}Dies collecti\end{turn}} & 
\multicolumn{1}{c}{\begin{turn}{90}Syzygiae collectae\end{turn}} & 
\multicolumn{1}{c}{\begin{turn}{90}Menses cavi[?]\end{turn}} & 
\multicolumn{1}{c}{\begin{turn}{90}Syclus Solis\end{turn}} & 
\multicolumn{1}{r}{\begin{turn}{90}Neomenia\end{turn}} & 
\multicolumn{1}{l}{\begin{turn}{90}Ecatombaeonis\end{turn}}
\\

\midrule
\endfirsthead
\caption*{Residuum tabula neomeniarum periodi Calippicae}\\
\toprule
% Read the header description from an external file
% Header for table p89-90
% Version with slanted headers for the names of the months
~ &
\begin{turn}{90}Anni periodi\end{turn} &
\begin{turn}{90}Cyclus Lunae\end{turn} & 

\begin{rotate}{75}\textgreek{Εκατομβαιών}\end{rotate} & &
\begin{rotate}{75}\textgreek{Μεταγειτνιών}\end{rotate} & &
\begin{rotate}{75}\textgreek{Βοηδρομιών}\end{rotate} & &

\begin{rotate}{75}\textgreek{Πυανεψιών}\end{rotate} & &
\begin{rotate}{75}\textgreek{Μαιμακτηριών}\end{rotate} & &
\begin{rotate}{75}\textgreek{Ποσειδεών}\end{rotate} & &

\begin{rotate}{75}\textgreek{Γαμηλιών}\end{rotate} & &
\begin{rotate}{75}\textgreek{Ανθεστηριών}\end{rotate} & &
\begin{rotate}{75}\textgreek{Ελαφηβολιών}\end{rotate} & &

\begin{rotate}{75}\textgreek{Μουνυχιών}\end{rotate} & &
\begin{rotate}{75}\textgreek{Θαργηλιών}\end{rotate} & &
\begin{rotate}{75}\textgreek{Σκιῤῥοφοριών α}\end{rotate} & &
\begin{rotate}{75}\textgreek{Σκιῤῥοφοριών β}\end{rotate} & &

\multicolumn{1}{c}{\begin{turn}{90}Dies collecti\end{turn}} & 
\multicolumn{1}{c}{\begin{turn}{90}Syzygiae collectae\end{turn}} & 
\multicolumn{1}{c}{\begin{turn}{90}Menses cavi[?]\end{turn}} & 
\multicolumn{1}{c}{\begin{turn}{90}Syclus Solis\end{turn}} & 
\multicolumn{1}{r}{\begin{turn}{90}Neomenia\end{turn}} & 
\multicolumn{1}{l}{\begin{turn}{90}Ecatombaeonis\end{turn}}
\\

\midrule
\endhead
%\\
\addlinespace[10pt]
& & \multicolumn{29}{l}{\footnotesize \super{†} \textgreek{ἐμβολ. [Abbriv.]}}\\
\endfoot
%%
%\midrule
  &    &    &
     &   &    &   &  4.&5  &    &   &  8.&9  &    &   &
  12.&13 &    &   & 16.&17 &    &   & 20.&21 &    &   &
  24.&25 &
  \\
\nopagebreak
† &  1 & 14 &
 \multicolumn{2}{c}{7} & \multicolumn{2}{c}{2} & \multicolumn{2}{c}{4} &
 \multicolumn{2}{c}{5} & \multicolumn{2}{c}{7} & \multicolumn{2}{c}{1} &
 \multicolumn{2}{c}{3} & \multicolumn{2}{c}{4} & \multicolumn{2}{c}{6} &
 \multicolumn{2}{c}{7} & \multicolumn{2}{c}{2} & \multicolumn{2}{c}{3} &
 \multicolumn{2}{c}{5} &
   384  &  13 &   6 & B & 28&Iun \\
\nopagebreak
%
\midrule
  &    &   &
     &   & 28.&29 &    &   &    &   &  2.&3  &    &   &
   6.&7  &    &   & 10.&11 &    &   & 14.&15 &    &   &
     &   &
  \\
\nopagebreak
  &  2 & 15 &
 \multicolumn{2}{c}{6} & \multicolumn{2}{c}{1} & \multicolumn{2}{c}{2} &
 \multicolumn{2}{c}{4} & \multicolumn{2}{c}{6} & \multicolumn{2}{c}{7} &
 \multicolumn{2}{c}{2} & \multicolumn{2}{c}{3} & \multicolumn{2}{c}{5} &
 \multicolumn{2}{c}{6} & \multicolumn{2}{c}{1} & \multicolumn{2}{c}{2} &
 \multicolumn{2}{c}{0} &
   739  &  25 &  11 & A G & 16&Iul \\
\nopagebreak
%
\midrule
  &    &    &
  18.&19 &    &   & 23.&23 &    &   & 26.&27 &    &   &
  30.&31 &    &   &    &   &  4.&5  &    &   &  8.&9  &
     &   &
  \\
\nopagebreak
† &  3 & 16 &
 \multicolumn{2}{c}{4} & \multicolumn{2}{c}{5} & \multicolumn{2}{c}{7} &
 \multicolumn{2}{c}{1} & \multicolumn{2}{c}{3} & \multicolumn{2}{c}{4} &
 \multicolumn{2}{c}{6} & \multicolumn{2}{c}{7} & \multicolumn{2}{c}{2} &
 \multicolumn{2}{c}{4} & \multicolumn{2}{c}{5} & \multicolumn{2}{c}{7} &
 \multicolumn{2}{c}{1} &
  1123  &  38 &  17 & F &  6&Iul \\
\nopagebreak
%
\midrule
  &    &    &
  12.&13 &    &   & 16.&17 &    &   & 20.&21 &    &   &
  24.&25 &    &   & 27.&28 &    &   &    &   &  1.&2  &
     &   &
  \\
\nopagebreak
  &  4 & 17 &
 \multicolumn{2}{c}{3} & \multicolumn{2}{c}{4} & \multicolumn{2}{c}{6} &
 \multicolumn{2}{c}{7} & \multicolumn{2}{c}{2} & \multicolumn{2}{c}{3} &
 \multicolumn{2}{c}{5} & \multicolumn{2}{c}{6} & \multicolumn{2}{c}{1} &
 \multicolumn{2}{c}{2} & \multicolumn{2}{c}{4} & \multicolumn{2}{c}{6} &
 \multicolumn{2}{c}{0} &
  1477  &  50 &  23 & E & 25&Iul \\
\midrule
\nopagebreak
%
  &    &    &
     &   &  5.&6  &    &   &  9.&10 &    &   & 13.&14 &
     &   & 17.&18 &    &   & 21.&22 &    &   & 25.&26 &
     &   &
  \\
\nopagebreak
  &  5 & 18 &
 \multicolumn{2}{c}{7} & \multicolumn{2}{c}{2} & \multicolumn{2}{c}{3} &
 \multicolumn{2}{c}{5} & \multicolumn{2}{c}{6} & \multicolumn{2}{c}{1} &
 \multicolumn{2}{c}{2} & \multicolumn{2}{c}{4} & \multicolumn{2}{c}{5} &
 \multicolumn{2}{c}{7} & \multicolumn{2}{c}{1} & \multicolumn{2}{c}{3} &
 \multicolumn{2}{c}{0} &
  1831  &  62 &  29 & D & 14&Iul \\
\nopagebreak
%
\midrule
  &    &   &
     &   & 29.&30 &    &   &    &   &  3.&4  &    &   &
   7.&8  &    &   & 11.&12 &    &   & 15.&16 &    &   &
  19.&20 &
  \\
\nopagebreak
† &  6 & 19 &
 \multicolumn{2}{c}{4} & \multicolumn{2}{c}{6} & \multicolumn{2}{c}{7} &
 \multicolumn{2}{c}{2} & \multicolumn{2}{c}{4} & \multicolumn{2}{c}{5} &
 \multicolumn{2}{c}{7} & \multicolumn{2}{c}{1} & \multicolumn{2}{c}{3} &
 \multicolumn{2}{c}{4} & \multicolumn{2}{c}{6} & \multicolumn{2}{c}{7} &
 \multicolumn{2}{c}{2} &
  2215  &  75 &  35 & C B &  2&Iul \\
\nopagebreak
%
\midrule
  &    &   &
     &   & 23.&24 &    &   & 27.&28 &    &   &    &   &
  11.&12 &    &   &  5.&6  &    &   &  9.&10 &    &   &
     &   &
  \\
\nopagebreak
  &  7 &  1 &
 \multicolumn{2}{c}{3} & \multicolumn{2}{c}{5} & \multicolumn{2}{c}{6} &
 \multicolumn{2}{c}{1} & \multicolumn{2}{c}{2} & \multicolumn{2}{c}{4} &
 \multicolumn{2}{c}{6} & \multicolumn{2}{c}{6} & \multicolumn{2}{c}{2} &
 \multicolumn{2}{c}{3} & \multicolumn{2}{c}{3} & \multicolumn{2}{c}{6} &
 \multicolumn{2}{c}{0} &
  2570  &  87 &  40 & A &  21&Iul \\
\nopagebreak
%
\midrule
  &    &    &
  13.&14 &    &   & 17.&18 &    &   & 21.&22 &    &   &
  24.&25 &    &   & 28.&29 &    &   &    &   &  2.&3  &
     &   &
  \\
\nopagebreak
  &  8 &  2 &
 \multicolumn{2}{c}{1} & \multicolumn{2}{c}{2} & \multicolumn{2}{c}{4} &
 \multicolumn{2}{c}{5} & \multicolumn{2}{c}{7} & \multicolumn{2}{c}{1} &
 \multicolumn{2}{c}{3} & \multicolumn{2}{c}{4} & \multicolumn{2}{c}{6} &
 \multicolumn{2}{c}{7} & \multicolumn{2}{c}{2} & \multicolumn{2}{c}{4} &
 \multicolumn{2}{c}{0} &
  2924  &  99 &  46 & G & 11&Iul \\
%\nopagebreak
%
\midrule
  &    &    &
     &   &  6.&7  &    &   & 10.&11 &    &   & 14.&15 &
     &   & 18.&19 &    &   & 22.&23 &    &   & 26.&27 &
     &   &
  \\
\nopagebreak
† &  9 &  3 &
 \multicolumn{2}{c}{5} & \multicolumn{2}{c}{7} & \multicolumn{2}{c}{1} &
 \multicolumn{2}{c}{3} & \multicolumn{2}{c}{4} & \multicolumn{2}{c}{6} &
 \multicolumn{2}{c}{7} & \multicolumn{2}{c}{2} & \multicolumn{2}{c}{3} &
 \multicolumn{2}{c}{5} & \multicolumn{2}{c}{6} & \multicolumn{2}{c}{1} &
 \multicolumn{2}{c}{2} &
  3308  & 112 &  52 & F & 30&Iun \\
\nopagebreak
%
\midrule
  &    &    &
  30.&1  &    &   &    &   &  4.&5  &    &   &  8.&9  &
     &   & 12.&13 &    &   & 16.&17 &    &   & 20.&21 &
     &   &
  \\
\nopagebreak
  & 10 &  4 &
 \multicolumn{2}{c}{4} & \multicolumn{2}{c}{5} & \multicolumn{2}{c}{7} &
 \multicolumn{2}{c}{2} & \multicolumn{2}{c}{3} & \multicolumn{2}{c}{5} &
 \multicolumn{2}{c}{6} & \multicolumn{2}{c}{1} & \multicolumn{2}{c}{2} &
 \multicolumn{2}{c}{4} & \multicolumn{2}{c}{5} & \multicolumn{2}{c}{7} &
 \multicolumn{2}{c}{0} &
  3662  &  12 &  58 & E D & 18&Iul \\
\nopagebreak
%
\midrule
  &    &   &
     &   & 20.&25 &    &   & 28.&29 &    &   &  2.&3  &
     &   &  6.&7  &    &   &    &   & 10.&11 &    &   &
  14.&15 &
  \\
\nopagebreak
† & 11 &  5 &
 \multicolumn{2}{c}{1} & \multicolumn{2}{c}{3} & \multicolumn{2}{c}{4} &
 \multicolumn{2}{c}{6} & \multicolumn{2}{c}{7} & \multicolumn{2}{c}{2} &
 \multicolumn{2}{c}{3} & \multicolumn{2}{c}{5} & \multicolumn{2}{c}{6} &
 \multicolumn{2}{c}{1} & \multicolumn{2}{c}{3} & \multicolumn{2}{c}{4} &
 \multicolumn{2}{c}{6} &
  4046  & 135 &  64 & C &   7&Iul \\
\nopagebreak
%
\midrule
  &    &   &
     &   & 18.&19 &    &   & 21.&22 &    &   & 25.&26 &
     &   & 29.&30 &    &   &    &   &  3.&4  &    &   &
     &   &
  \\
\nopagebreak
  & 12 &  6 &
 \multicolumn{2}{c}{7} & \multicolumn{2}{c}{2} & \multicolumn{2}{c}{3} &
 \multicolumn{2}{c}{5} & \multicolumn{2}{c}{6} & \multicolumn{2}{c}{1} &
 \multicolumn{2}{c}{2} & \multicolumn{2}{c}{4} & \multicolumn{2}{c}{5} &
 \multicolumn{2}{c}{7} & \multicolumn{2}{c}{2} & \multicolumn{2}{c}{3} &
 \multicolumn{2}{c}{0} &
  4401  & 149 &  69 & B &  26&Iul \\
%\nopagebreak
%
\midrule
  &    &    &
   7.&8  &    &   & 11.&12 &    &   & 15.&16 &    &   &
  19.&20 &    &   & 23.&24 &    &   & 27.&28 &    &   &
     &   &
  \\
\nopagebreak
  & 13 &  7 &
 \multicolumn{2}{c}{5} & \multicolumn{2}{c}{6} & \multicolumn{2}{c}{1} &
 \multicolumn{2}{c}{2} & \multicolumn{2}{c}{4} & \multicolumn{2}{c}{5} &
 \multicolumn{2}{c}{7} & \multicolumn{2}{c}{1} & \multicolumn{2}{c}{3} &
 \multicolumn{2}{c}{4} & \multicolumn{2}{c}{6} & \multicolumn{2}{c}{7} &
 \multicolumn{2}{c}{0} &
  4755  & 161 &  75 & A & 16&Iul \\
\nopagebreak
%
\midrule
  &    &    &
     &   &  1.&2  &    &   &  5.&6  &    &   &  9.&10 &
     &   & 13.&14 &    &   & 17.&18 &    &   & 21.&22 &
     &   &
  \\
\nopagebreak
† & 14 &  8 &
 \multicolumn{2}{c}{2} & \multicolumn{2}{c}{4} & \multicolumn{2}{c}{5} &
 \multicolumn{2}{c}{7} & \multicolumn{2}{c}{1} & \multicolumn{2}{c}{3} &
 \multicolumn{2}{c}{4} & \multicolumn{2}{c}{6} & \multicolumn{2}{c}{7} &
 \multicolumn{2}{c}{2} & \multicolumn{2}{c}{3} & \multicolumn{2}{c}{5} &
 \multicolumn{2}{c}{6} &
  5139  & 174 &  81 & G F &  4&Iul \\
\nopagebreak
%
\midrule
  &    &    &
  25.&26 &    &   & 29.&30 &    &   &    &   &  3.&4  &
     &   &  7.&8  &    &   & 11.&12 &    &   & 15.&16 &
     &   &
  \\
\nopagebreak
  & 15 &  9 &
 \multicolumn{2}{c}{1} & \multicolumn{2}{c}{2} & \multicolumn{2}{c}{4} &
 \multicolumn{2}{c}{5} & \multicolumn{2}{c}{7} & \multicolumn{2}{c}{2} &
 \multicolumn{2}{c}{3} & \multicolumn{2}{c}{5} & \multicolumn{2}{c}{6} &
 \multicolumn{2}{c}{1} & \multicolumn{2}{c}{2} & \multicolumn{2}{c}{4} &
 \multicolumn{2}{c}{0} &
  5493  & 186 &  87 & E & 13&Iul \\
\nopagebreak
%
\midrule
  &    &   &
     &   & 18.&19 &    &   & 22.&23 &    &   & 26.&27 &
     &   & 30.&1  &    &   &    &   &  4.&5  &    &   &
     &   &
  \\
\nopagebreak
  & 16 & 10 &
 \multicolumn{2}{c}{5} & \multicolumn{2}{c}{7} & \multicolumn{2}{c}{1} &
 \multicolumn{2}{c}{3} & \multicolumn{2}{c}{4} & \multicolumn{2}{c}{6} &
 \multicolumn{2}{c}{7} & \multicolumn{2}{c}{2} & \multicolumn{2}{c}{3} &
 \multicolumn{2}{c}{5} & \multicolumn{2}{c}{7} & \multicolumn{2}{c}{1} &
 \multicolumn{2}{c}{0} &
  5848  & 198 &  62 & D &  12&Iul \\
%\nopagebreak
%
\midrule
  &    &    &
   8.&9  &    &   & 12.&13 &    &   & 16.&17 &    &   &
  20.&21 &    &   & 24.&25 &    &   & 28.&29 &    &   &
     &   &
  \\
\nopagebreak
† & 17 & 11 &
 \multicolumn{2}{c}{3} & \multicolumn{2}{c}{4} & \multicolumn{2}{c}{6} &
 \multicolumn{2}{c}{7} & \multicolumn{2}{c}{2} & \multicolumn{2}{c}{3} &
 \multicolumn{2}{c}{5} & \multicolumn{2}{c}{6} & \multicolumn{2}{c}{1} &
 \multicolumn{2}{c}{2} & \multicolumn{2}{c}{4} & \multicolumn{2}{c}{5} &
 \multicolumn{2}{c}{7} &
  6232  & 211 &  98 & C &  2&Iul \\
\nopagebreak
%
\midrule
  &    &    &
   2.&3  &    &   &  6.&7  &    &   & 10.&11 &    &   &
  14.&15 &    &   & 18.&19 &    &   & 22.&23 &    &   &
     &   &
  \\
\nopagebreak
  & 18 & 12 &
 \multicolumn{2}{c}{2} & \multicolumn{2}{c}{3} & \multicolumn{2}{c}{5} &
 \multicolumn{2}{c}{6} & \multicolumn{2}{c}{1} & \multicolumn{2}{c}{2} &
 \multicolumn{2}{c}{4} & \multicolumn{2}{c}{5} & \multicolumn{2}{c}{7} &
 \multicolumn{2}{c}{1} & \multicolumn{2}{c}{3} & \multicolumn{2}{c}{4} &
 \multicolumn{2}{c}{0} &
  6586  & 223 & 104 & B A &  20&Iul \\
\nopagebreak
%
\midrule
  &    &    &
  26.&27 &    &   & 30.&1  &    &   &    &   &  4.&5  &
     &   &  8.&9  &    &   & 12.&13 &    &   & 15.&16 &
     &   &
  \\
\nopagebreak
  & 19 & 13 &
 \multicolumn{2}{c}{6} & \multicolumn{2}{c}{7} & \multicolumn{2}{c}{2} &
 \multicolumn{2}{c}{3} & \multicolumn{2}{c}{5} & \multicolumn{2}{c}{7} &
 \multicolumn{2}{c}{1} & \multicolumn{2}{c}{3} & \multicolumn{2}{c}{4} &
 \multicolumn{2}{c}{6} & \multicolumn{2}{c}{7} & \multicolumn{2}{c}{2} &
 \multicolumn{2}{c}{0} &
  6940  & 235 & 110 & G &  9&Iul \\
\nopagebreak
%
\midrule
  &    &   &
     &   & 19.&20 &    &   & 23.&24 &    &   & 27.&28 &
     &   &    &   &  1.&2  &    &   &  5.&6  &    &   &
   9.&10 &
  \\
\nopagebreak
† & 20 & 14 &
 \multicolumn{2}{c}{3} & \multicolumn{2}{c}{5} & \multicolumn{2}{c}{6} &
 \multicolumn{2}{c}{1} & \multicolumn{2}{c}{2} & \multicolumn{2}{c}{4} &
 \multicolumn{2}{c}{5} & \multicolumn{2}{c}{7} & \multicolumn{2}{c}{2} &
 \multicolumn{2}{c}{3} & \multicolumn{2}{c}{5} & \multicolumn{2}{c}{6} &
 \multicolumn{2}{c}{1} &
  7324  & 248 & 116 & F &  28&Iun \\
\nopagebreak
%
\midrule
  &    &   &
     &   & 13.&14 &    &   & 17.&18 &    &   & 21.&22 &
     &   & 25.&26 &    &   & 29.&30 &    &   &    &   &
     &   &
  \\
\nopagebreak
  & 21 & 15 &
 \multicolumn{2}{c}{2} & \multicolumn{2}{c}{4} & \multicolumn{2}{c}{5} &
 \multicolumn{2}{c}{7} & \multicolumn{2}{c}{1} & \multicolumn{2}{c}{3} &
 \multicolumn{2}{c}{4} & \multicolumn{2}{c}{6} & \multicolumn{2}{c}{7} &
 \multicolumn{2}{c}{2} & \multicolumn{2}{c}{3} & \multicolumn{2}{c}{5} &
 \multicolumn{2}{c}{0} &
  7679  & 160 & 121 & E &  17&Iul \\
\nopagebreak
%
\midrule
  &    &    &
   3.&4  &    &   &  7.&8  &    &   & 11.&12 &    &   &
  15.&16 &    &   & 19.&20 &    &   & 23.&24 &    &   &
  27.&28 &
  \\
\nopagebreak
† & 22 & 16 &
 \multicolumn{2}{c}{7} & \multicolumn{2}{c}{1} & \multicolumn{2}{c}{3} &
 \multicolumn{2}{c}{4} & \multicolumn{2}{c}{6} & \multicolumn{2}{c}{7} &
 \multicolumn{2}{c}{2} & \multicolumn{2}{c}{3} & \multicolumn{2}{c}{5} &
 \multicolumn{2}{c}{6} & \multicolumn{2}{c}{1} & \multicolumn{2}{c}{2} &
 \multicolumn{2}{c}{4} &
  8062  & 273 & 128 & D C &   6&Iul \\
\nopagebreak
%
\midrule
  &    &    &
     &   &    &   &  1.&2  &    &   &  5.&6  &    &   &
   9.&10 &    &   & 12.&13 &    &   & 16.&17 &    &   &
     &   &
  \\
\nopagebreak
  & 23 & 17 &
 \multicolumn{2}{c}{5} & \multicolumn{2}{c}{7} & \multicolumn{2}{c}{2} &
 \multicolumn{2}{c}{3} & \multicolumn{2}{c}{5} & \multicolumn{2}{c}{6} &
 \multicolumn{2}{c}{1} & \multicolumn{2}{c}{2} & \multicolumn{2}{c}{4} &
 \multicolumn{2}{c}{5} & \multicolumn{2}{c}{7} & \multicolumn{2}{c}{1} &
 \multicolumn{2}{c}{0} &
  8417  & 285 & 133 & B &  24&Iul \\
\nopagebreak
%
\midrule
  &    &    &
  20.&21 &    &   & 24.&25 &    &   & 28.&29 &    &   &
     &   &  2.&3  &    &   &  6.&7 &    &   & 10.&11 &
     &   &
  \\
\nopagebreak
  & 24 & 18 &
 \multicolumn{2}{c}{3} & \multicolumn{2}{c}{4} & \multicolumn{2}{c}{6} &
 \multicolumn{2}{c}{7} & \multicolumn{2}{c}{2} & \multicolumn{2}{c}{3} &
 \multicolumn{2}{c}{5} & \multicolumn{2}{c}{7} & \multicolumn{2}{c}{1} &
 \multicolumn{2}{c}{3} & \multicolumn{2}{c}{4} & \multicolumn{2}{c}{6} &
 \multicolumn{2}{c}{0} &
  8771  & 297 & 139 & A & 14&Iul \\
\nopagebreak
%
\midrule
  &    &   &
     &   & 14.&15 &    &   & 18.&19 &    &   & 22.&23 &
     &   & 26.&27 &    &   & 30.&1  &    &   &    &   &
   4.&5  &
  \\
\nopagebreak
† & 25 & 19 &
 \multicolumn{2}{c}{7} & \multicolumn{2}{c}{2} & \multicolumn{2}{c}{3} &
 \multicolumn{2}{c}{5} & \multicolumn{2}{c}{6} & \multicolumn{2}{c}{1} &
 \multicolumn{2}{c}{2} & \multicolumn{2}{c}{4} & \multicolumn{2}{c}{5} &
 \multicolumn{2}{c}{7} & \multicolumn{2}{c}{1} & \multicolumn{2}{c}{3} &
 \multicolumn{2}{c}{5} &
  9155  & 310 & 145 & G &   3&Iul \\
\nopagebreak
%
\midrule
  &    &    &
     &   &  8.&9  &    &   & 12.&13 &    &   & 16.&17 &
     &   & 20.&21 &    &   & 24.&25 &    &   & 28.&29 &
     &   &
  \\
\nopagebreak
  & 26 &  1 &
 \multicolumn{2}{c}{6} & \multicolumn{2}{c}{1} & \multicolumn{2}{c}{2} &
 \multicolumn{2}{c}{4} & \multicolumn{2}{c}{5} & \multicolumn{2}{c}{7} &
 \multicolumn{2}{c}{1} & \multicolumn{2}{c}{3} & \multicolumn{2}{c}{4} &
 \multicolumn{2}{c}{6} & \multicolumn{2}{c}{7} & \multicolumn{2}{c}{2} &
 \multicolumn{2}{c}{0} &
  9509  & 322 & 151 & F E & 21&Iul \\
\nopagebreak
%
\midrule
  &    &    &
     &   &    &   &  2.&3  &    &   &  6.&7  &    &   &
   9.&10 &    &   & 13.&14 &    &   & 17.&18 &    &   &
     &   &
  \\
\nopagebreak
  & 27 &  2 &
 \multicolumn{2}{c}{3} & \multicolumn{2}{c}{5} & \multicolumn{2}{c}{7} &
 \multicolumn{2}{c}{1} & \multicolumn{2}{c}{3} & \multicolumn{2}{c}{4} &
 \multicolumn{2}{c}{6} & \multicolumn{2}{c}{7} & \multicolumn{2}{c}{2} &
 \multicolumn{2}{c}{3} & \multicolumn{2}{c}{5} & \multicolumn{2}{c}{6} &
 \multicolumn{2}{c}{0} &
  9864  & 334 & 156 & D &  10&Iul \\
\nopagebreak
%
\midrule
  &    &    &
  21.&22 &    &   & 25.&26 &    &   & 29.&30 &    &   &
     &   &  3.&4  &    &   &  7.&8 &    &   & 11.&12 &
     &   &
  \\
\nopagebreak
† & 28 &  3 &
 \multicolumn{2}{c}{1} & \multicolumn{2}{c}{2} & \multicolumn{2}{c}{4} &
 \multicolumn{2}{c}{5} & \multicolumn{2}{c}{7} & \multicolumn{2}{c}{1} &
 \multicolumn{2}{c}{3} & \multicolumn{2}{c}{5} & \multicolumn{2}{c}{6} &
 \multicolumn{2}{c}{1} & \multicolumn{2}{c}{2} & \multicolumn{2}{c}{4} &
 \multicolumn{2}{c}{5} &
 10248  & 347 & 162 & C & 30&Iun \\
\nopagebreak
%
\midrule
  &    &    &
  15.&16 &    &   & 19.&20 &    &   & 23.&24 &    &   &
  27.&28 &    &   &    &   &  1.&2  &    &   &  5.&6  &
     &   &
  \\
\nopagebreak
  & 29 &  4 &
 \multicolumn{2}{c}{7} & \multicolumn{2}{c}{1} & \multicolumn{2}{c}{3} &
 \multicolumn{2}{c}{4} & \multicolumn{2}{c}{6} & \multicolumn{2}{c}{7} &
 \multicolumn{2}{c}{2} & \multicolumn{2}{c}{3} & \multicolumn{2}{c}{5} &
 \multicolumn{2}{c}{7} & \multicolumn{2}{c}{1} & \multicolumn{2}{c}{3} &
 \multicolumn{2}{c}{0} &
 10602  & 359 & 168 & B & 19&Iul \\
\nopagebreak
%
\midrule
  &    &    &
     &   &  9.&10 &    &   & 13.&14 &    &   & 17.&18 &
     &   & 21.&22 &    &   & 25.&26 &    &   & 29.&30 &
     &   &
  \\
\nopagebreak
† & 30 &  5 &
 \multicolumn{2}{c}{4} & \multicolumn{2}{c}{6} & \multicolumn{2}{c}{7} &
 \multicolumn{2}{c}{2} & \multicolumn{2}{c}{3} & \multicolumn{2}{c}{5} &
 \multicolumn{2}{c}{6} & \multicolumn{2}{c}{1} & \multicolumn{2}{c}{2} &
 \multicolumn{2}{c}{4} & \multicolumn{2}{c}{5} & \multicolumn{2}{c}{7} &
 \multicolumn{2}{c}{1} &
 10985  & 372 & 174 & A G &  7&Iul \\
\nopagebreak
%
\midrule
  &    &    &
     &   &  3.&4  &    &   &  6.&7  &    &   & 10.&11 &
     &   & 14.&15 &    &   & 18.&19 &    &   & 22.&23 &
     &   &
  \\
\nopagebreak
  & 31 &  6 &
 \multicolumn{2}{c}{3} & \multicolumn{2}{c}{5} & \multicolumn{2}{c}{6} &
 \multicolumn{2}{c}{1} & \multicolumn{2}{c}{2} & \multicolumn{2}{c}{4} &
 \multicolumn{2}{c}{5} & \multicolumn{2}{c}{7} & \multicolumn{2}{c}{1} &
 \multicolumn{2}{c}{3} & \multicolumn{2}{c}{4} & \multicolumn{2}{c}{6} &
 \multicolumn{2}{c}{0} &
 11340  & 384 & 180 & F & 26&Iul \\
\nopagebreak
%
\midrule
  &    &   &
     &   & 26.&27 &    &   & 30.&1  &    &   &    &   &
   4.&5  &    &   &  8.&9  &    &   & 12.&13 &    &   &
     &   &
  \\
\nopagebreak
  & 32 &  7 &
 \multicolumn{2}{c}{7} & \multicolumn{2}{c}{2} & \multicolumn{2}{c}{3} &
 \multicolumn{2}{c}{5} & \multicolumn{2}{c}{6} & \multicolumn{2}{c}{1} &
 \multicolumn{2}{c}{3} & \multicolumn{2}{c}{4} & \multicolumn{2}{c}{6} &
 \multicolumn{2}{c}{7} & \multicolumn{2}{c}{2} & \multicolumn{2}{c}{3} &
 \multicolumn{2}{c}{0} &
 11695  & 396 & 185 & E &  15&Iul \\
\nopagebreak
%
\midrule
  &    &    &
  16.&17 &    &   & 20.&21 &    &   & 24.&25 &    &   &
  28.&29 &    &   &    &   &  2.&3  &    &   &  6.&7  &
     &   &
  \\
\nopagebreak
† & 33 &  8 &
 \multicolumn{2}{c}{5} & \multicolumn{2}{c}{6} & \multicolumn{2}{c}{1} &
 \multicolumn{2}{c}{2} & \multicolumn{2}{c}{4} & \multicolumn{2}{c}{5} &
 \multicolumn{2}{c}{7} & \multicolumn{2}{c}{1} & \multicolumn{2}{c}{3} &
 \multicolumn{2}{c}{5} & \multicolumn{2}{c}{6} & \multicolumn{2}{c}{1} &
 \multicolumn{2}{c}{1} &
 12079  & 409 & 191 & D &  5&Iul \\
\nopagebreak
%
\midrule
  &    &    &
  10.&11 &    &   & 14.&15 &    &   & 18.&19 &    &   &
  22.&23 &    &   & 26.&27 &    &   & 30.&1  &    &   &
     &   &
  \\
\nopagebreak
  & 34 &  9 &
 \multicolumn{2}{c}{4} & \multicolumn{2}{c}{5} & \multicolumn{2}{c}{7} &
 \multicolumn{2}{c}{1} & \multicolumn{2}{c}{3} & \multicolumn{2}{c}{4} &
 \multicolumn{2}{c}{6} & \multicolumn{2}{c}{7} & \multicolumn{2}{c}{2} &
 \multicolumn{2}{c}{3} & \multicolumn{2}{c}{5} & \multicolumn{2}{c}{6} &
 \multicolumn{2}{c}{0} &
 12433  & 421 & 197 & C B &  23&Iul \\
\nopagebreak
%
\midrule
  &    &    &
     &   &  3.&4  &    &   &  7.&8  &    &   & 11.&12 &
     &   & 15.&16 &    &   & 19.&20 &    &   & 23.&24 &
     &   &
  \\
\nopagebreak
  & 35 & 10 &
 \multicolumn{2}{c}{1} & \multicolumn{2}{c}{3} & \multicolumn{2}{c}{4} &
 \multicolumn{2}{c}{6} & \multicolumn{2}{c}{7} & \multicolumn{2}{c}{2} &
 \multicolumn{2}{c}{3} & \multicolumn{2}{c}{5} & \multicolumn{2}{c}{6} &
 \multicolumn{2}{c}{1} & \multicolumn{2}{c}{2} & \multicolumn{2}{c}{4} &
 \multicolumn{2}{c}{0} &
 12787  & 433 & 203 & A & 12&Iul \\
\nopagebreak
%
\midrule
  &    &   &
     &   & 27.&28 &    &   &    &   &  1.&2  &    &   &
   5.&6  &    &   &  9.&10 &    &   & 13.&14 &    &   &
  17.&18 &
  \\
\nopagebreak
† & 36 & 11 &
 \multicolumn{2}{c}{5} & \multicolumn{2}{c}{7} & \multicolumn{2}{c}{1} &
 \multicolumn{2}{c}{3} & \multicolumn{2}{c}{5} & \multicolumn{2}{c}{6} &
 \multicolumn{2}{c}{1} & \multicolumn{2}{c}{2} & \multicolumn{2}{c}{4} &
 \multicolumn{2}{c}{5} & \multicolumn{2}{c}{7} & \multicolumn{2}{c}{1} &
 \multicolumn{2}{c}{3} &
 13171  & 446 & 209 & G & Ka.&Iul \\
\nopagebreak
%
\midrule
  &    &    &
     &   & 21.&22 &    &   & 25.&26 &    &   & 29.&30 &
     &   &    &   &  3.&4  &    &   &  7.&8  &    &   &
     &   &
  \\
\nopagebreak
  & 37 & 12 &
 \multicolumn{2}{c}{4} & \multicolumn{2}{c}{6} & \multicolumn{2}{c}{7} &
 \multicolumn{2}{c}{2} & \multicolumn{2}{c}{3} & \multicolumn{2}{c}{5} &
 \multicolumn{2}{c}{6} & \multicolumn{2}{c}{1} & \multicolumn{2}{c}{3} &
 \multicolumn{2}{c}{4} & \multicolumn{2}{c}{6} & \multicolumn{2}{c}{7} &
 \multicolumn{2}{c}{0} &
 13526  & 458 & 214 & F & 20&Iul \\
\nopagebreak
%
\midrule
  &    &    &
  11.&12 &    &   & 15.&16 &    &   & 19.&20 &    &   &
  23.&24 &    &   & 27.&28 &    &   & 30.&1  &    &   &
     &   &
  \\
\nopagebreak
  & 38 & 13 &
 \multicolumn{2}{c}{2} & \multicolumn{2}{c}{3} & \multicolumn{2}{c}{5} &
 \multicolumn{2}{c}{6} & \multicolumn{2}{c}{1} & \multicolumn{2}{c}{2} &
 \multicolumn{2}{c}{4} & \multicolumn{2}{c}{5} & \multicolumn{2}{c}{7} &
 \multicolumn{2}{c}{1} & \multicolumn{2}{c}{3} & \multicolumn{2}{c}{4} &
 \multicolumn{2}{c}{0} &
 13880  & 470 & 220 & E D &   9&Iul \\
\nopagebreak
% page 90
\midrule
  &    &    &
     &   &  4.&5  &    &   &  8.&9  &    &   & 12.&13 &
     &   & 16.&17 &    &   & 20.&21 &    &   & 24.&25 &
     &   &
  \\
\nopagebreak
† & 39 & 14 &
 \multicolumn{2}{c}{6} & \multicolumn{2}{c}{1} & \multicolumn{2}{c}{2} &
 \multicolumn{2}{c}{4} & \multicolumn{2}{c}{5} & \multicolumn{2}{c}{7} &
 \multicolumn{2}{c}{1} & \multicolumn{2}{c}{3} & \multicolumn{2}{c}{4} &
 \multicolumn{2}{c}{6} & \multicolumn{2}{c}{7} & \multicolumn{2}{c}{2} &
 \multicolumn{2}{c}{3} &
 14264  & 483 & 226 & C & 28&Iun \\
\nopagebreak
%
\midrule
  &    &    &
  28.&29 &    &   &    &   &  2.&3  &    &   &  6.&7&
% '7' not visible in the scan we use. Is visible in other scans and editions
     &   & 10.&11 &    &   & 14.&15 &    &   & 18.&19 &
     &   &
  \\
\nopagebreak
  & 40 & 15 &
 \multicolumn{2}{c}{5} & \multicolumn{2}{c}{6} & \multicolumn{2}{c}{1} &
 \multicolumn{2}{c}{3} & \multicolumn{2}{c}{4} & \multicolumn{2}{c}{6} &
 \multicolumn{2}{c}{7} & \multicolumn{2}{c}{2} & \multicolumn{2}{c}{3} &
 \multicolumn{2}{c}{5} & \multicolumn{2}{c}{6} & \multicolumn{2}{c}{1} &
 \multicolumn{2}{c}{0} &
 14618  & 495 & 231 & B & 17&Iul \\
\nopagebreak
%
\midrule
  &    &    &
     &   & 22.&23 &    &   & 26.&27 &    &   & 30.&1 &
     &   &    &   &  4.&5  &    &   &  8.&9  &    &   &
  12.&13 &
  \\
\nopagebreak
† & 41 & 16 &
 \multicolumn{2}{c}{2} & \multicolumn{2}{c}{4} & \multicolumn{2}{c}{5} &
 \multicolumn{2}{c}{7} & \multicolumn{2}{c}{1} & \multicolumn{2}{c}{3} &
 \multicolumn{2}{c}{4} & \multicolumn{2}{c}{6} & \multicolumn{2}{c}{1} &
 \multicolumn{2}{c}{2} & \multicolumn{2}{c}{4} & \multicolumn{2}{c}{5} &
 \multicolumn{2}{c}{7} &
 15002  & 508 & 238 & A &  6&Iul \\
\nopagebreak
%
\midrule
  &    &    &
     &   & 16.&17 &    &   & 20.&21 &    &   & 24.&25 &
     &   & 27.&28 &    &   &    &   &  1.&2  &    &   &
     &   &
  \\
\nopagebreak
  & 42 & 17 &
 \multicolumn{2}{c}{1} & \multicolumn{2}{c}{3} & \multicolumn{2}{c}{4} &
 \multicolumn{2}{c}{6} & \multicolumn{2}{c}{7} & \multicolumn{2}{c}{2} &
 \multicolumn{2}{c}{3} & \multicolumn{2}{c}{5} & \multicolumn{2}{c}{6} &
 \multicolumn{2}{c}{1} & \multicolumn{2}{c}{3} & \multicolumn{2}{c}{4} &
 \multicolumn{2}{c}{0} &
 15357  & 520 & 243 & G F & 24&Iul \\
\nopagebreak
%
\midrule
  &    &    &
   5.&6  &    &   &  9.&10 &    &   & 13.&14 &    &   &
  17.&18 &    &   & 21.&22 &    &   & 25.&26 &    &   &
     &   &
  \\
\nopagebreak
  & 43 & 18 &
 \multicolumn{2}{c}{6} & \multicolumn{2}{c}{7} & \multicolumn{2}{c}{2} &
 \multicolumn{2}{c}{3} & \multicolumn{2}{c}{5} & \multicolumn{2}{c}{6} &
 \multicolumn{2}{c}{1} & \multicolumn{2}{c}{2} & \multicolumn{2}{c}{4} &
 \multicolumn{2}{c}{5} & \multicolumn{2}{c}{7} & \multicolumn{2}{c}{1} &
 \multicolumn{2}{c}{0} &
 15711  & 532 & 249 & E &  14&Iul \\
\nopagebreak
%
\midrule
  &    &    &
  29.&30 &    &   &    &   &  3.&4  &    &   &  7.&8  &
     &   & 11.&12 &    &   & 15.&16 &    &   & 19.&20 &
     &   &
  \\
\nopagebreak
† & 44 & 19 &
 \multicolumn{2}{c}{3} & \multicolumn{2}{c}{4} & \multicolumn{2}{c}{6} &
 \multicolumn{2}{c}{1} & \multicolumn{2}{c}{2} & \multicolumn{2}{c}{4} &
 \multicolumn{2}{c}{5} & \multicolumn{2}{c}{7} & \multicolumn{2}{c}{1} &
 \multicolumn{2}{c}{3} & \multicolumn{2}{c}{4} & \multicolumn{2}{c}{6} &
 \multicolumn{2}{c}{7} &
 16095  & 545 & 255 & D &  3&Iul \\
\nopagebreak
%
\midrule
  &    &    &
  23.&24 &    &   & 27.&28 &    &   &    &   &  1.&2  &
     &   &  5.&6  &    &   &  9.&10 &    &   & 13.&14 &
     &   &
  \\
\nopagebreak
  & 45 &  1 &
 \multicolumn{2}{c}{2} & \multicolumn{2}{c}{3} & \multicolumn{2}{c}{5} &
 \multicolumn{2}{c}{6} & \multicolumn{2}{c}{1} & \multicolumn{2}{c}{3} &
 \multicolumn{2}{c}{4} & \multicolumn{2}{c}{6} & \multicolumn{2}{c}{7} &
 \multicolumn{2}{c}{2} & \multicolumn{2}{c}{3} & \multicolumn{2}{c}{5} &
 \multicolumn{2}{c}{0} &
 16449  & 557 & 261 & C & 22&Iul \\
\nopagebreak
%
\midrule
  &    &    &
     &   & 17.&16 &    &   & 21.&22 &    &   & 24.&25 &
     &   & 28.&29 &    &   &    &   &  2.&3  &    &   &
     &   &
  \\
\nopagebreak
  & 46 &  2 &
 \multicolumn{2}{c}{6} & \multicolumn{2}{c}{1} & \multicolumn{2}{c}{2} &
 \multicolumn{2}{c}{4} & \multicolumn{2}{c}{5} & \multicolumn{2}{c}{7} &
 \multicolumn{2}{c}{1} & \multicolumn{2}{c}{3} & \multicolumn{2}{c}{4} &
 \multicolumn{2}{c}{6} & \multicolumn{2}{c}{1} & \multicolumn{2}{c}{2} &
 \multicolumn{2}{c}{0} &
 16804  & 569 & 266 & B A & 10&Iul \\
\nopagebreak
%
\midrule
  &    &    &
   6.&7  &    &   & 10.&11 &    &   & 14.&15 &    &   &
  18.&19 &    &   & 22.&23 &    &   & 26.&27 &    &   &
     &   &
  \\
\nopagebreak
† & 47 &  3 &
 \multicolumn{2}{c}{4} & \multicolumn{2}{c}{5} & \multicolumn{2}{c}{7} &
 \multicolumn{2}{c}{1} & \multicolumn{2}{c}{3} & \multicolumn{2}{c}{4} &
 \multicolumn{2}{c}{6} & \multicolumn{2}{c}{7} & \multicolumn{2}{c}{2} &
 \multicolumn{2}{c}{3} & \multicolumn{2}{c}{5} & \multicolumn{2}{c}{6} &
 \multicolumn{2}{c}{1} &
 17188  & 582 & 272 & G &  30&Iun \\
\nopagebreak
%
\midrule
  &    &    &
   3.&4  &    &   &  4.&5  &    &   &  8.&9  &    &   &
  12.&13 &    &   & 16.&17 &    &   & 20.&21 &    &   &
     &   &
  \\
\nopagebreak
  & 48 &  4 &
 \multicolumn{2}{c}{3} & \multicolumn{2}{c}{4} & \multicolumn{2}{c}{6} &
 \multicolumn{2}{c}{7} & \multicolumn{2}{c}{1} & \multicolumn{2}{c}{3} &
 \multicolumn{2}{c}{5} & \multicolumn{2}{c}{6} & \multicolumn{2}{c}{1} &
 \multicolumn{2}{c}{2} & \multicolumn{2}{c}{4} & \multicolumn{2}{c}{5} &
 \multicolumn{2}{c}{0} &
 17542  & 594 & 278 & F &  19&Iul \\
\nopagebreak
%
\midrule
  &    &    &
  24.&25 &    &   & 28.&29 &    &   &    &   &  2.&3  &
     &   &  6.&7  &    &   & 10.&11 &    &   & 14.&15 &
     &   &
  \\
\nopagebreak
† & 49 &  5 &
 \multicolumn{2}{c}{7} & \multicolumn{2}{c}{1} & \multicolumn{2}{c}{3} &
 \multicolumn{2}{c}{4} & \multicolumn{2}{c}{6} & \multicolumn{2}{c}{1} &
 \multicolumn{2}{c}{2} & \multicolumn{2}{c}{4} & \multicolumn{2}{c}{5} &
 \multicolumn{2}{c}{7} & \multicolumn{2}{c}{1} & \multicolumn{2}{c}{3} &
 \multicolumn{2}{c}{4} &
 17926  & 607 & 284 & E &  8&Iul \\
\nopagebreak
%
\midrule
  &    &    &
  18.&19 &    &   & 21.&22 &    &   & 25.&26 &    &   &
  29.&30 &    &   &    &   &  3.&4  &    &   &  7.&8  &
     &   &
  \\
\nopagebreak
  & 50 &  6 &
 \multicolumn{2}{c}{6} & \multicolumn{2}{c}{7} & \multicolumn{2}{c}{2} &
 \multicolumn{2}{c}{3} & \multicolumn{2}{c}{5} & \multicolumn{2}{c}{6} &
 \multicolumn{2}{c}{1} & \multicolumn{2}{c}{2} & \multicolumn{2}{c}{4} &
 \multicolumn{2}{c}{6} & \multicolumn{2}{c}{7} & \multicolumn{2}{c}{2} &
 \multicolumn{2}{c}{0} &
 18280  & 619 & 290 & D C &  26&Iul \\
\nopagebreak
%
\midrule
  &    &    &
     &   & 11.&12 &    &   & 15.&16 &    &   & 19.&20 &
     &   & 23.&24 &    &   & 27.&28 &    &   &    &   &
     &   &
  \\
\nopagebreak
  & 51 &  7 &
 \multicolumn{2}{c}{3} & \multicolumn{2}{c}{5} & \multicolumn{2}{c}{6} &
 \multicolumn{2}{c}{1} & \multicolumn{2}{c}{2} & \multicolumn{2}{c}{4} &
 \multicolumn{2}{c}{5} & \multicolumn{2}{c}{7} & \multicolumn{2}{c}{1} &
 \multicolumn{2}{c}{3} & \multicolumn{2}{c}{4} & \multicolumn{2}{c}{6} &
 \multicolumn{2}{c}{0} &
 18635  & 631 & 295 & B & 15&Iul \\
\nopagebreak
%
\midrule
  &    &    &
   1.&2  &    &   &  5.&6  &    &   &  9.&10 &    &   &
  13.&14 &    &   & 17.&18 &    &   & 21.&22 &    &   &
  25.&26 &
  \\
\nopagebreak
† & 52 &  8 &
 \multicolumn{2}{c}{1} & \multicolumn{2}{c}{2} & \multicolumn{2}{c}{4} &
 \multicolumn{2}{c}{5} & \multicolumn{2}{c}{7} & \multicolumn{2}{c}{1} &
 \multicolumn{2}{c}{3} & \multicolumn{2}{c}{4} & \multicolumn{2}{c}{6} &
 \multicolumn{2}{c}{7} & \multicolumn{2}{c}{2} & \multicolumn{2}{c}{3} &
 \multicolumn{2}{c}{5} &
 19018  & 644 & 302 & A &   5&Iul \\
% '644' clearer in 1598 edition
\nopagebreak
%
\midrule
  &    &   &
     &   & 29.&30 &    &   &    &   &  3.&4  &    &   &
   7.&8  &    &   & 11.&12 &    &   & 15.&16 &    &   &
     &   &
  \\
\nopagebreak
  & 53 &  9 &
 \multicolumn{2}{c}{6} & \multicolumn{2}{c}{1} & \multicolumn{2}{c}{2} &
 \multicolumn{2}{c}{4} & \multicolumn{2}{c}{6} & \multicolumn{2}{c}{7} &
 \multicolumn{2}{c}{2} & \multicolumn{2}{c}{3} & \multicolumn{2}{c}{4} &
 \multicolumn{2}{c}{6} & \multicolumn{2}{c}{1} & \multicolumn{2}{c}{2} &
 \multicolumn{2}{c}{0} &
 19373  & 656 & 307 & G &  23&Iul \\
\nopagebreak
%
\midrule
  &    &    &
  18.&19 &    &   & 22.&23 &    &   & 26.&27 &    &   &
  30.&1  &    &   &    &   &  4.&5  &    &   &  8.&9  &
     &   &
  \\
\nopagebreak
  & 54 & 10 &
 \multicolumn{2}{c}{4} & \multicolumn{2}{c}{5} & \multicolumn{2}{c}{7} &
 \multicolumn{2}{c}{1} & \multicolumn{2}{c}{3} & \multicolumn{2}{c}{4} &
 \multicolumn{2}{c}{6} & \multicolumn{2}{c}{7} & \multicolumn{2}{c}{2} &
 \multicolumn{2}{c}{4} & \multicolumn{2}{c}{5} & \multicolumn{2}{c}{7} &
 \multicolumn{2}{c}{0} &
 19727  & 668 & 313 & F E &  12&Iul \\
\nopagebreak
%
\midrule
  &    &    &
     &   & 12.&13 &    &   & 12.&13 &    &   & 20.&21 &
     &   & 24.&25 &    &   & 28.&29 &    &   &    &   &
   2.&3  &
  \\
\nopagebreak
† & 55 & 11 &
 \multicolumn{2}{c}{1} & \multicolumn{2}{c}{3} & \multicolumn{2}{c}{4} &
 \multicolumn{2}{c}{6} & \multicolumn{2}{c}{7} & \multicolumn{2}{c}{2} &
 \multicolumn{2}{c}{3} & \multicolumn{2}{c}{5} & \multicolumn{2}{c}{6} &
 \multicolumn{2}{c}{1} & \multicolumn{2}{c}{2} & \multicolumn{2}{c}{4} &
 \multicolumn{2}{c}{6} &
 20111  & 681 & 319 & D & Ka.&Iul \\
\nopagebreak
%
\midrule
  &    &    &
     &   &  6.&7  &    &   & 10.&11 &    &   & 14.&15 &
     &   & 18.&19 &    &   & 22.&23 &    &   & 26.&27 &
     &   &
  \\
\nopagebreak
  & 56 & 12 &
 \multicolumn{2}{c}{7} & \multicolumn{2}{c}{2} & \multicolumn{2}{c}{3} &
 \multicolumn{2}{c}{5} & \multicolumn{2}{c}{6} & \multicolumn{2}{c}{1} &
 \multicolumn{2}{c}{2} & \multicolumn{2}{c}{4} & \multicolumn{2}{c}{5} &
 \multicolumn{2}{c}{7} & \multicolumn{2}{c}{1} & \multicolumn{2}{c}{3} &
 \multicolumn{2}{c}{0} &
 20465  & 693 & 325 & C &  20&Iul \\
\nopagebreak
%
\midrule
  &    &   &
     &   & 30.&1  &    &   &    &   &  4.&5  &    &   &
   8.&9  &    &   & 12.&13 &    &   & 15.&16 &    &   &
     &   &
  \\
\nopagebreak
  & 57 & 13 &
 \multicolumn{2}{c}{4} & \multicolumn{2}{c}{6} & \multicolumn{2}{c}{7} &
 \multicolumn{2}{c}{4} & \multicolumn{2}{c}{4} & \multicolumn{2}{c}{5} &
 \multicolumn{2}{c}{1} & \multicolumn{2}{c}{1} & \multicolumn{2}{c}{3} &
 \multicolumn{2}{c}{6} & \multicolumn{2}{c}{6} & \multicolumn{2}{c}{7} &
 \multicolumn{2}{c}{0} &
 20820  & 705 & 330 & B &   9&Iul \\
\nopagebreak
%
\midrule
  &    &    &
  19.&20 &    &   & 23.&24 &    &   & 27.&28 &    &   &
     &   &  1.&2  &    &   &  5.&6  &    &   &  9.&10 &
     &   &
  \\
\nopagebreak
† & 58 & 14 &
 \multicolumn{2}{c}{2} & \multicolumn{2}{c}{3} & \multicolumn{2}{c}{5} &
 \multicolumn{2}{c}{6} & \multicolumn{2}{c}{1} & \multicolumn{2}{c}{2} &
 \multicolumn{2}{c}{4} & \multicolumn{2}{c}{6} & \multicolumn{2}{c}{7} &
 \multicolumn{2}{c}{2} & \multicolumn{2}{c}{3} & \multicolumn{2}{c}{5} &
 \multicolumn{2}{c}{6} &
 21204  & 718 & 336 & A G &  28&Iun \\
\nopagebreak
%
\midrule
  &    &    &
  13.&14 &    &   & 17.&18 &    &   & 21.&22 &    &   &
  25.&26 &    &   & 29.&30 &    &   &    &   &  3.&4  &
     &   &
  \\
\nopagebreak
  & 59 & 15 &
 \multicolumn{2}{c}{1} & \multicolumn{2}{c}{2} & \multicolumn{2}{c}{4} &
 \multicolumn{2}{c}{5} & \multicolumn{2}{c}{7} & \multicolumn{2}{c}{1} &
 \multicolumn{2}{c}{3} & \multicolumn{2}{c}{4} & \multicolumn{2}{c}{6} &
 \multicolumn{2}{c}{7} & \multicolumn{2}{c}{2} & \multicolumn{2}{c}{4} &
 \multicolumn{2}{c}{0} &
 21558  & 730 & 342 & F &  17&Iul \\
\nopagebreak
%
\midrule
  &    &    &
     &   &  7.&8  &    &   & 11.&12 &    &   & 15.&16 &
     &   & 19.&20 &    &   & 23.&24 &    &   & 27.&28 &
     &   &
  \\
\nopagebreak
† & 60 & 16 &
 \multicolumn{2}{c}{5} & \multicolumn{2}{c}{7} & \multicolumn{2}{c}{1} &
 \multicolumn{2}{c}{3} & \multicolumn{2}{c}{4} & \multicolumn{2}{c}{6} &
 \multicolumn{2}{c}{7} & \multicolumn{2}{c}{2} & \multicolumn{2}{c}{3} &
 \multicolumn{2}{c}{5} & \multicolumn{2}{c}{6} & \multicolumn{2}{c}{1} &
 \multicolumn{2}{c}{2} &
 21942  & 743 & 348 & E &   6&Iul \\
\nopagebreak
%
\midrule
  &    &    &
     &   &  1.&2  &    &   &  5.&6  &    &   &  9.&10 &
     &   & 13.&14 &    &   & 17.&18 &    &   & 21.&22 &
     &   &
  \\
\nopagebreak
  & 61 & 17 &
 \multicolumn{2}{c}{4} & \multicolumn{2}{c}{6} & \multicolumn{2}{c}{7} &
 \multicolumn{2}{c}{2} & \multicolumn{2}{c}{3} & \multicolumn{2}{c}{5} &
 \multicolumn{2}{c}{6} & \multicolumn{2}{c}{1} & \multicolumn{2}{c}{2} &
 \multicolumn{2}{c}{4} & \multicolumn{2}{c}{5} & \multicolumn{2}{c}{7} &
 \multicolumn{2}{c}{0} &
 22296  & 755 & 354 & D &  25&Iul \\
\nopagebreak
%
\midrule
  &    &    &
     &   & 24.&25 &    &   & 28.&29 &    &   &    &   &
   2.&3  &    &   &  6.&7  &    &   & 10.&11 &    &   &
     &   &
  \\
\nopagebreak
  & 62 & 18 &
 \multicolumn{2}{c}{1} & \multicolumn{2}{c}{3} & \multicolumn{2}{c}{4} &
 \multicolumn{2}{c}{6} & \multicolumn{2}{c}{7} & \multicolumn{2}{c}{2} &
 \multicolumn{2}{c}{4} & \multicolumn{2}{c}{5} & \multicolumn{2}{c}{7} &
 \multicolumn{2}{c}{1} & \multicolumn{2}{c}{3} & \multicolumn{2}{c}{4} &
 \multicolumn{2}{c}{0} &
 22631  & 767 & 359 & C B &  13&Iul \\
\nopagebreak
%
\midrule
  &    &    &
  14.&15 &    &   & 18.&19 &    &   & 22.&23 &    &   &
  26.&27 &    &   & 30.&1  &    &   &    &   &  4.&5  &
     &   &
  \\
\nopagebreak
† & 63 & 19 &
 \multicolumn{2}{c}{6} & \multicolumn{2}{c}{7} & \multicolumn{2}{c}{2} &
 \multicolumn{2}{c}{3} & \multicolumn{2}{c}{5} & \multicolumn{2}{c}{6} &
 \multicolumn{2}{c}{1} & \multicolumn{2}{c}{2} & \multicolumn{2}{c}{4} &
 \multicolumn{2}{c}{5} & \multicolumn{2}{c}{7} & \multicolumn{2}{c}{2} &
 \multicolumn{2}{c}{3} &
 23035  & 780 & 365 & A &   3&Iul \\
% '365' unclear; better in other scans
\nopagebreak
%
\midrule
  &    &    &
   8.&9  &    &   & 12.&13 &    &   & 16.&17 &    &   &
  20.&21 &    &   & 24.&25 &    &   & 28.&29 &    &   &
     &   &
  \\
\nopagebreak
  & 64 &  1 &
 \multicolumn{2}{c}{5} & \multicolumn{2}{c}{6} & \multicolumn{2}{c}{1} &
 \multicolumn{2}{c}{2} & \multicolumn{2}{c}{4} & \multicolumn{2}{c}{5} &
 \multicolumn{2}{c}{7} & \multicolumn{2}{c}{1} & \multicolumn{2}{c}{3} &
 \multicolumn{2}{c}{4} & \multicolumn{2}{c}{6} & \multicolumn{2}{c}{7} &
 \multicolumn{2}{c}{0} &
 23389  & 792 & 371 & G &  22&Iul \\
\nopagebreak
%
\midrule
  &    &    &
     &   &  2.&3  &    &   &  6.&7  &    &   &  9.&10 &
     &   & 13.&14 &    &   & 17.&18 &    &   & 21.&22 &
     &   &
  \\
\nopagebreak
  & 65 &  2 &
 \multicolumn{2}{c}{2} & \multicolumn{2}{c}{4} & \multicolumn{2}{c}{5} &
 \multicolumn{2}{c}{7} & \multicolumn{2}{c}{1} & \multicolumn{2}{c}{3} &
 \multicolumn{2}{c}{4} & \multicolumn{2}{c}{6} & \multicolumn{2}{c}{7} &
 \multicolumn{2}{c}{2} & \multicolumn{2}{c}{3} & \multicolumn{2}{c}{5} &
 \multicolumn{2}{c}{0} &
 23734  & 804 & 377 & F &  11&Iul \\
\nopagebreak
%
\midrule
  &    &    &
     &   & 25.&26 &    &   & 29.&30 &    &   &  3.&4  &
     &   &  7.&6  &    &   & 11.&12 &    &   & 15.&16 &
     &   &
  \\
\nopagebreak
† & 66 &  3 &
 \multicolumn{2}{c}{6} & \multicolumn{2}{c}{1} & \multicolumn{2}{c}{2} &
 \multicolumn{2}{c}{4} & \multicolumn{2}{c}{5} & \multicolumn{2}{c}{7} &
 \multicolumn{2}{c}{1} & \multicolumn{2}{c}{3} & \multicolumn{2}{c}{4} &
 \multicolumn{2}{c}{6} & \multicolumn{2}{c}{7} & \multicolumn{2}{c}{2} &
 \multicolumn{2}{c}{3} &
 24127  & 817 & 383 & E D &  29&Iun \\
\nopagebreak
%
\midrule
  &    &    &
     &   & 19.&20 &    &   & 23.&24 &    &   & 27.&28 &
     &   &    &   &  1.&2  &    &   &  5.&6  &    &   &
     &   &
  \\
\nopagebreak
  & 67 &  4 &
 \multicolumn{2}{c}{5} & \multicolumn{2}{c}{7} & \multicolumn{2}{c}{1} &
 \multicolumn{2}{c}{3} & \multicolumn{2}{c}{4} & \multicolumn{2}{c}{6} &
 \multicolumn{2}{c}{7} & \multicolumn{2}{c}{2} & \multicolumn{2}{c}{4} &
 \multicolumn{2}{c}{5} & \multicolumn{2}{c}{7} & \multicolumn{2}{c}{1} &
 \multicolumn{2}{c}{0} &
 24482  & 829 & 388 & C &  18&Iul \\
\nopagebreak
%
\midrule
  &    &    &
   9.&10 &    &   & 13.&14 &    &   & 17.&18 &    &   &
  21.&22 &    &   & 25.&26 &    &   & 29.&30 &    &   &
     &   &
  \\
\nopagebreak
† & 68 &  5 &
 \multicolumn{2}{c}{3} & \multicolumn{2}{c}{4} & \multicolumn{2}{c}{6} &
 \multicolumn{2}{c}{7} & \multicolumn{2}{c}{2} & \multicolumn{2}{c}{3} &
 \multicolumn{2}{c}{5} & \multicolumn{2}{c}{6} & \multicolumn{2}{c}{1} &
 \multicolumn{2}{c}{2} & \multicolumn{2}{c}{4} & \multicolumn{2}{c}{5} &
 \multicolumn{2}{c}{7} &
 24866  & 842 & 394 & B &   8&Iul \\
\nopagebreak
%
\midrule
  &    &    &
   3.&4  &    &   &  6.&7  &    &   & 10.&11 &    &   &
  14.&15 &    &   & 18.&19 &    &   & 22.&23 &    &   &
     &   &
  \\
\nopagebreak
  & 69 &  6 &
 \multicolumn{2}{c}{2} & \multicolumn{2}{c}{3} & \multicolumn{2}{c}{5} &
 \multicolumn{2}{c}{6} & \multicolumn{2}{c}{1} & \multicolumn{2}{c}{2} &
 \multicolumn{2}{c}{4} & \multicolumn{2}{c}{5} & \multicolumn{2}{c}{7} &
 \multicolumn{2}{c}{1} & \multicolumn{2}{c}{3} & \multicolumn{2}{c}{4} &
 \multicolumn{2}{c}{0} &
 25220  & 854 & 400 & A &  27&Iul \\
\nopagebreak
%
\midrule
  &    &    &
  26.&27 &    &   & 30.&1  &    &   &    &   &  4.&3  &
% '4.3' in the original is odd, as in all other entries
% the second number is 1 larger than the first number.
     &   &  8.&9  &    &   & 12.&13 &    &   & 16.&17 &
     &   &
  \\
\nopagebreak
  & 70 &  7 &
 \multicolumn{2}{c}{6} & \multicolumn{2}{c}{7} & \multicolumn{2}{c}{2} &
 \multicolumn{2}{c}{3} & \multicolumn{2}{c}{5} & \multicolumn{2}{c}{7} &
 \multicolumn{2}{c}{1} & \multicolumn{2}{c}{3} & \multicolumn{2}{c}{4} &
 \multicolumn{2}{c}{6} & \multicolumn{2}{c}{7} & \multicolumn{2}{c}{2} &
 \multicolumn{2}{c}{0} &
 25574  & 866 & 406 & G F & 15&Iul \\
\nopagebreak
%
\midrule
  &    &    &
     &   & 20.&21 &    &   & 24.&25 &    &   & 28.&29 &
     &   &    &   &  2.&3  &    &   &  6.&7  &    &   &
  10.&11 &
  \\
\nopagebreak
† & 71 &  8 &
 \multicolumn{2}{c}{3} & \multicolumn{2}{c}{5} & \multicolumn{2}{c}{6} &
 \multicolumn{2}{c}{1} & \multicolumn{2}{c}{2} & \multicolumn{2}{c}{4} &
 \multicolumn{2}{c}{5} & \multicolumn{2}{c}{7} & \multicolumn{2}{c}{2} &
 \multicolumn{2}{c}{3} & \multicolumn{2}{c}{5} & \multicolumn{2}{c}{6} &
 \multicolumn{2}{c}{1} &
 25958  & 879 & 412 & E &   4&Iul \\
\nopagebreak
%
\midrule
  &    &    &
     &   & 14.&15 &    &   & 18.&19 &    &   & 22.&23 &
     &   & 26.&27 &    &   & 30.&1  &    &   &    &   &
     &   &
  \\
\nopagebreak
  & 72 &  9 &
 \multicolumn{2}{c}{2} & \multicolumn{2}{c}{4} & \multicolumn{2}{c}{5} &
 \multicolumn{2}{c}{7} & \multicolumn{2}{c}{1} & \multicolumn{2}{c}{3} &
 \multicolumn{2}{c}{4} & \multicolumn{2}{c}{6} & \multicolumn{2}{c}{7} &
 \multicolumn{2}{c}{2} & \multicolumn{2}{c}{3} & \multicolumn{2}{c}{5} &
 \multicolumn{2}{c}{0} &
 26313  & 891 & 417 & D &  23&Iul \\
\nopagebreak
%
\midrule
  &    &    &
   3.&4  &    &   &  7.&8  &    &   & 11.&12 &    &   &
  15.&16 &    &   & 19.&20 &    &   & 23.&24 &    &   &
     &   &
  \\
\nopagebreak
  & 73 & 10 &
 \multicolumn{2}{c}{7} & \multicolumn{2}{c}{1} & \multicolumn{2}{c}{3} &
 \multicolumn{2}{c}{4} & \multicolumn{2}{c}{6} & \multicolumn{2}{c}{7} &
 \multicolumn{2}{c}{2} & \multicolumn{2}{c}{3} & \multicolumn{2}{c}{5} &
 \multicolumn{2}{c}{6} & \multicolumn{2}{c}{1} & \multicolumn{2}{c}{2} &
 \multicolumn{2}{c}{0} &
 26667  & 903 & 423 & C &  13&Iul \\
\nopagebreak
%
\midrule
  &    &    &
  27.&28 &    &   &    &   &  1.&2  &    &   &  5.&6  &
     &   &  9.&10 &    &   & 13.&14 &    &   & 17.&18 &
     &   &
  \\
\nopagebreak
† & 74 & 11 &
 \multicolumn{2}{c}{4} & \multicolumn{2}{c}{5} & \multicolumn{2}{c}{7} &
 \multicolumn{2}{c}{2} & \multicolumn{2}{c}{3} & \multicolumn{2}{c}{5} &
 \multicolumn{2}{c}{6} & \multicolumn{2}{c}{1} & \multicolumn{2}{c}{2} &
 \multicolumn{2}{c}{4} & \multicolumn{2}{c}{5} & \multicolumn{2}{c}{7} &
 \multicolumn{2}{c}{1} &
 27051  & 916 & 429 & B A &  Ka.&Iul \\
\nopagebreak
%
\midrule
  &    &    &
  21.&22 &    &   & 25.&26 &    &   & 29.&30 &    &   &
     &   &  3.&4  &    &   &  7.&8  &    &   & 11.&12 &
     &   &
  \\
\nopagebreak
  & 75 & 12 &
 \multicolumn{2}{c}{3} & \multicolumn{2}{c}{4} & \multicolumn{2}{c}{6} &
 \multicolumn{2}{c}{7} & \multicolumn{2}{c}{2} & \multicolumn{2}{c}{3} &
 \multicolumn{2}{c}{5} & \multicolumn{2}{c}{7} & \multicolumn{2}{c}{1} &
 \multicolumn{2}{c}{3} & \multicolumn{2}{c}{4} & \multicolumn{2}{c}{6} &
 \multicolumn{2}{c}{0} &
 27405  & 928 & 435 & G &  20&Iul \\
\nopagebreak
%
\midrule
  &    &    &
     &   & 15.&16 &    &   & 19.&20 &    &   & 23.&24 &
     &   & 27.&28 &    &   & 30.&1  &    &   & 30.&1  &
     &   &
  \\
\nopagebreak
  & 76 & 13 &
 \multicolumn{2}{c}{7} & \multicolumn{2}{c}{2} & \multicolumn{2}{c}{3} &
 \multicolumn{2}{c}{5} & \multicolumn{2}{c}{6} & \multicolumn{2}{c}{1} &
 \multicolumn{2}{c}{2} & \multicolumn{2}{c}{4} & \multicolumn{2}{c}{5} &
 \multicolumn{2}{c}{7} & \multicolumn{2}{c}{1} & \multicolumn{2}{c}{3} &
 \multicolumn{2}{c}{0} &
 27759  & 940 & 441 & F &   9&Iul \\
\nopagebreak
%
\bottomrule
\end{longtable}
\endgroup


\section{De Periodo Lunari Calippica ab Autumno}

\lipsum

\lipsum[150]

\listoftables
\end{document}
