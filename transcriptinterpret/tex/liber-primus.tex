% !TEX TS-program = xelatex
% !TEX encoding = UTF-8 Unicode
% this template is specifically designed to be typeset with XeLaTeX;
% it will not work with other engines, such as pdfLaTeX

%%% Count out columns for fixed-width source font
% 000000011111111112222222222333333333344444444445555555555666666666677777777778
% 345678901234567890123456789012345678901234567890123456789012345678901234567890

\chapter{Liber Primus - De anno aequabili minore}
\begin{center}
{\scshape
IOSEPHI SCALIGERI IULII CAESARIS F.\\
DE EMENDATIONE TEMPORUM\\
LIBER PRIMUS\\
}% scshape
\end{center}
\normalsize

% 1
% {PDF page nr}{source page nr}{line nr}
\plnr{84}{1}{1}Si verum est, quod sciscit Stoicorum
schola, Tempus esse normam rerum, et
custodiam, quia veritatis index atque examen
est, et rerum gestarum memoriam, ac
diuturnitatem posteritati tuetur: ii non vulgari
laude digni sunt, qui temporum rationes
conscribere, atque fugitivam antiquitatem
retrahere conantur.
\lnr{8}Qua in re cum
tam priscis scriptoribus, quam aequalibus
temporum nostrum opera egregie navata sit, dolendum tamen, aut
ferius, quam oportebat, antiquos sese ad id studium contulisse, aut pauciora
ea de re monumenta, quam ab ipsis auctoribus relicta sunt, ad
nos pervenisse.
\lnr{13}Nam ut omnia extent veterum Graecorum scripta, ea
tamen paucorum temporum intervallum complectebantur.
\lnr{14}Graecis
enim ante initia Olymiadum suarum nihil plane exploratum est: et,
quod dolendum est, de illorum scriptis, quae ad Chronologiam spectabant,
nihil nobis praeter desiderium relictum est.
\lnr{17}Nam quae Eusebii exstant,
quamuis ex Graecorum monumentis hausta sunt, et multa egregia
ac cognitu digna nobis conservarunt: tamen dissimulandum non est,
multa in illis reperiri, quae castigatioribus iudiciis non satisfaciant.
\lnr{21}Quod si Thalli, Castoris, Phlegontis,
 Eratosthenis canones exstarent,
perparua, aut nulla potius ratio haberetur librorum quorundam, qui
hodie in penuria meliorum nobis in pretio sunt.
\lnr{23}
Apud Romanos vero,
ea scriptio infeliciter cessit, quod eam cognitionem ferius amplexi sint.
\lnr{25}Nam ante Consulatum Bruti nihil certi apud illos: omnia fabulosa: et,
si rem propius spectemus, ne ipsius quidem Bruti Consulatum, ac tempus
Regifugii satis exploratum habent.
\lnr{27}quamius, ut prodidit Censorinus,
Varro collatis diversarum civitatum temporibus, et intervalla retexens,
verum in lucem protulerit, et viam reperit, qua certus
annorum Urbis conditae numerus iniri posset.

% 2
% {PDF page nr}{source page nr}{line nr}
\plnr{85}{2}{2}Sed, ut suo loco disputabitur,
non magis constabat Varroni de initiis Urbis, quam Graecis de
anno excidii Troiae.
\lnr{4}Nam ea demum est vera demonstratio, quae cogit,
non quae persuadet.
\lnr{5}Soli sacri libri supersunt, ex quorum fontibus
certa temporum ratio hauriri possit.
\lnr{6}Sed omnis temporum cognitio
inutilis est, nisi certa epocha in illis deprehendatur, ad quam omnium
temporum contextus, tam antecedentium, quam consequentium referri
possit.
\lnr{9}Nam, ut praeclare dixit vetus inter Christianos scriptor
Tatianus, apud quos temporum notatio non cohaeret, apud illos neque
veritatis et fidei historicae ratio ulla constare potest.
\lnr{11}Quod si aliquis
sacrae historiae peritissimus, hoc est, qui intervalla rerum gestarum
nobilissima certissimis ratiociniis ex Mose, et
 reliquis sacris Bibliis explorata
habeat, nihil tamen ex illis a certam epocham historiae Graecae,
aut Romanae referre possit: quodnam adiumentum is ex eiusmodi
diligentia adferre potest aut sibi, aut studiosis rerum antiquarum?
\lnr{17}Nam omnis cognitionis finis ad usum aliquem spectat, quem si ex medio
literarum sustuleris, ingratus est omnis labor et opera, quaecunque
in omne studium impenditur.
\lnr{19}Eiusmodi est Iudaeorum scientia, qui
in ratiociniis quidem sacrorum temporum colligendis tantum studio
et diligentia consecuti sunt, ut proxime at veritate abesse dici possint: sed
dum nullam aut saltem depravatam rerum extrarum cognitionem
tenent, multum errant, quod sine externa historia sacram tractare
aggrediuntur.
\lnr{24}Venio ad nostros, recentiores dico, qui hodie summo
cum fructu, sacrae, Graecae, et Romanae historiae tempora digesserunt.
\lnr{26}Ii heroica virtute chronologiam negligentia et contemtu maiorum
intermortuam ac sepultam, ex tenebris et oblivionis silentio quotidie
eruere conantur.
\lnr{28}Certe meum semper iudicium fuit, eam rem maiore
cum laude ab illis restitutam, quam ab antiquis proditam fuisse.
\lnr{29}Nam
non solum pleraque in ratione temporum pristinae integritati reddiderunt,
sed et longe meliora effecerunt.
\lnr{31}In multis tamen iudicium, in quibusdam
etiam diligentiam requiro.
\lnr{32}Neque enim dum verum adepti sunt.
\lnr{33}Argumento suerint omnium, quotquot de his rebus tractarunt,
 dissensiones:
ut inter tot millia Chronologorum vix inter duos de eadem re
conveniat.
\lnr{35}Quanta adhuc contentione de Septimanis Danielis, de initio,
medio, et fine earum velitantur?
\lnr{36}Tamen nihil plane eorum, quae volunt,
assecuti sunt.
\lnr{37}Ab eorum lectione incertior atque indoctior sum,
quam dudum.
\lnr{38}Quis unquam eorum veram epocham Exodi Habraeorum;
quis, quod pudendum est, verum annum natalis Dominici odoratus
est?
\lnr{40}Ecce trita, obvia, vulgaria, ut nobis videtur, ignoramus, et remotiorum
ac reconditiorum indicium promittimus!
\lnr{41}Quis eorum Danielis
Hebdomadas interpretandas suscepit, qui inscitiae suae latebram
non quaesiverit, et reges Persidis, qui nunquam in rerum natura fuerunt,
non commentus sit?

% 3
% {PDF page nr}{source page nr}{line nr}
\plnr{86}{3}{3}Quod si Danielem accuratissime legissent,
eis ad negotium explicandum non aliis regibus Persidis opus fuisset,
quam iis, quos Herodotus, Diodorus, et omnis Graecorum antiquitas
novit.
\lnr{6}Sed quo non progressa est \textgreek{[Greek]}?
\lnr{6}Berosos, Metasthenes, et
nescio quos Catones, ac Philones consulunt, qui ante hos centum annos
ex officina nescio cuius indocti et impudentis prodierunt.
\lnr{8}Et sese
Criticos in temporum notatione profitentur, quibus tam facili genere,
tam pueriliter unus homo otiosus in tanta luce literarum quotidie imoponit.
\lnr{11}Cuius hominis inscitiamam si nihil aliud, certe illud arguere
 possit, quod
Metasthenem pro Megasthene posuit.
\lnr{12}Si Iosephum Graece, aut Strabonem,
aut Athenaeum legisset, is Megasthenem vocari deprehendisset,
quem Metasthenem vocat.
\lnr{14}Si Graece scisset, numquam \textgreek{[Greek]} in illa
lingua reperiri, neque hanc compositionem in eadem probari intellexisset.
\lnr{16}Ut igitur ii resipiscant, qui et novos reges in Perside crearunt,
et Assueros Priscos, Assueros Longimanos, Assueros Pios, duos Cyros,
et nescio quae alia somnia Annii Viterbiensis in medium producunt,
primum uno verbo indicabo fontem erroris eorum: deinde qui medicina
huic morbo fieri possit, docebo.
\lnr{20}Quod igitur in veri investigatione
eos ratio fugerit, duas summas causas reperio: unam, quod veterum
tempora civilia, annorum, mensium formas, status, ac genera ignorarunt:
alteram, quod characterem, et notationem ei anno, quem sibi
proposuerant, non adhibuerunt.
\lnr{24}Ex utraque quidem causa temporum
confusio manavit, sed diverso genere.
\lnr{25}Ex priore causa ignoratus est
annus, mensis et dies multarum nobilium epocharum.
\lnr{26}Huius enim
rei cognitio pertinet ad tempus civile nationum.
\lnr{27}Ex altera causa Palilia
urbis Romae nunc tertio anno Olympiadis, nunc quarto attribuuntur.
\lnr{29}
Item Consulatus Bruti nunc in hunc, nunc in illum annum
Olympiadis confertur.
\lnr{30}
Ut igitur novam rationem emendationis temporum
ineamus, duo illa praecipue nobis discutienda sunt: sed prius
de omnium nationum temporibus civilibus: quam assequi perdifficile
est, nisi prius tempore in sua principia, hoc est ab annis, periodis,
mensibus in ultimum terminum, dies, horas ac scrupula resoluto.
\lnr{35}Nam qui ante nos hanc provinciam aggressi sunt, si modo hanc nostram,
non aliam aggressi sunt, ii satis de tempore, et eius natura
disputarunt.
\lnr{37}Sed hanc disputationem melius interpres \textgreek{[Greek]}
sibi vindicasset.
\lnr{38}Neque vero nos id agimus, ut difiniamus
tempus esse hoc secundum Peripateticos, aut illud seundum Stoicos,
aut Academicos.
\lnr{40}Qui istis definitionibus diu immorati sunt, et hac
sola scientia Chronologiae scribendae modum terminarunt, illi fatis
verborum quiedem, sed rerum nihil definiverunt.

% 4
% {PDF page nr}{source page nr}{line nr}
\plnr{87}{4}{1}Nequid tamen
\textgreek{[Greek]} transigatur, decrevi singularum, vel
 minimarum temporis
partium prius conspectum aliquem dare, quam ad descriptionem
\textgreek{[Greek]} temporum civilium, et eorum methodum aggrediar.
\lnr{4}Incipiam igitur ab ultimo termino, a die scilicet, et eius partibus,
hoc est hora, et scrupulis.
\lnr{6}Ab hora igitur, si libet, principium esto.

\subsection{De Horis et partibus diei reliquis}

\lnr{7}Veteribus statim ab initio has diei partes, quas \textsc{Horas}
vocamus, in usu non fuisse, argumento fuerint priscae locutiones,
quibus dies non in partes secatur, sed actionibus quotidianis
distiguitur: ut cum \textgreek{[Greek]} vesperam vocabant, 
nimirum, ut poëta
% Source: poëta; Rare occurence of a diaeresis. Not sure if it should be
% left there or removed.
inquit, \textit{Demeret emeritis cum iuga Phoebus equis}.
\lnr{11}Item quod tempus
antemeridianum disignantes dicebant \textgreek{[Greek]}
 vel \textgreek{[Greek]},
convenientibus scilicet eo tempore in Comitium viris: ut Hesiodus dicit,
\textgreek{[Greek]}.
\lnr{14}Quod tamen longe aliter interpretes
Graeci illius poëtae exponunt.
% Source: poëtae
\lnr{15}Aiunt enim Hesiodum intellexisse
de tricesima mensis Lunaris: et sensum loci Hesiodei esse perinde
ac si dixisset, Quando homines veram \textgreek{[Greek]} Lunarem agunt, et
non secundum usum politicum, sed secundum motum Lunae.
\lnr{18}Quod
tamen nobis valde coactum videtur: et mentem Hesiodi hanc fuisse dicimus:
\textgreek{[Greek]} esse valde idoneam rebus gerendis ea hora,
 qua homines
ad ius in forum conveniunt.
\lnr{21}
Homerus Odyss. \textgreek{[Greek]}
% Expand "Odyss." in the propper way, with the correct declension.
\begin{quote}
\textgreek{[Greek]}\\
\textgreek{[Greek]}
\end{quote}

\lnr{24}Quae sane interpretatio melior vulgari.
\lnr{24}Sic etiam paulo post dicit,
\textgreek{[Greek]}, loquens de undecima: cuius partem designat,
 cum dicit
\textgreek{[Greek]}.
\lnr{26}Quod nos interpretamur iam adulto die.
\lnr{26}Sic Homerus
meridiem designat, \textgreek{[Greek]}.
\lnr{27}Porro neque
hoc verbum \textgreek{[Greek]} id, quod nunc, valebat.
\lnr{28}Sed tempus actuum quotidianorum
illo notabatur: ut cum dicebant \textgreek{[Greek]}.
\lnr{29}Latinis
vero Tempestas dicebatur.
\lnr{30}In Legibus Decemvirum Atticis fuit:
\textsc{Sol occasus suprema tempestas esto}.
\lnr{31}Neque recte
quidam hinc expungunt \textsc{tempestas}.
\lnr{32}Quod \textsc{suprema} absolute
diceretur, ut apud Plautum.
\lnr{33}Nam plane in legibus Solonis, unde illud
caput traductum, scriptum fuit,
 \textgreek{[Greek]}.
\lnr{34}Stoicus
scriptor apud Stobaeum loquens de Socratis iudicio capitali: 
\textgreek{[Greek], [Greek], [Greek]
ΕΣΧΑΤΗΝ ΩΡΑΝ [Greek], [Greek] ΗΛΙΟΣ ΕΠΙ ΤΩΝ
ΟΡΩΝ, [Greek]}.
\lnr{38}Idem censeas de veteribus Hebraeis,
qui diei nullas alias partes, quam mane, meridiem, et vesperam norant.
et ita dies dividitur Psalmo \textsc{lv}, commate \textsc{xviii}.
% Psalm 55:17 (KJV) "Evening (εσπέρας), and morning (πρωϊ),
% and at noon (μεσημβρίας), will I pray, and cry aloud:
% and he shall hear my voice."

% 5
% {PDF page nr}{source page nr}{line nr}
\plnr{88}{5}{2}Sic Homero,
\textgreek{[Greek]}.
\lnr{3}Sed hic dies intelligitur Lux, exclusa nocte.
\lnr{4}Nam totum \textgreek{[Greek]} Hebraei in quatuor partes
 dividebant, quas vigilias
vocabant.
\lnr{5}Prima vigilia erat at vespere: secunda at media nocte:
tertia at mane: quarta at meridie.
\lnr{6}Alioqui nomen hoc \texthebrew{[Hebrew]} quo hodie
horam designant, ne notum quidem illis erat: atque apud Danielem
aliud significat.
\lnr{8}Posterorum inventum est Horologium, et \textgreek{ηλιοτρόπια},
%[Greek: heliotropia; probably: sundails]
quibus dies per lineas, et intervalla umbrarum distinguebatur.
\lnr{10}
Unde prodiit locutio \textgreek{[Greek]}, pro hora coenae.
\lnr{10}Vel \textgreek{[Greek]}:
quia notis literarum singularium horae distinguebantur.
\lnr{11}Testatur et Epigrammatium de Horologio:
\begin{quote}
\textgreek{[Greek]}\\
\textgreek{[Greek]}
\end{quote}
\lnr{14}Nam ante
\textgreek{Ζ, Η, Θ, Ι,} erat \textgreek{Α, Β, Γ, Δ, Ε, ς.}
\lnr{15}Arabibus, Persis, et reliquis Orientis
gentibus non horologiis, sed
naturalibus matutini, meridiani,
et vespertini temporis
intervallis diem notare,
etiam hodie consuetudo manet.
\lnr{21}Astronomis propria
est divisio diei in sexagesimas
primas, secundas, tertias,
et sic deinceps.
\lnr{24}Artificibus
computi annalis in
horas, puncta, ostena, minuta,
partes.
\lnr{27}Hora est punctorum
4. minutorum 40.
partium 480. momentorum
1760.
\lnr{30}Ostenta autem sunt arbitraria,
quibuslibet aliarum
divisionum in illa resolutis.
\lnr{33}Orientalibus vero Computatoribus
compendiosa horarum
resolutio est.
\lnr{35}
Non
enim in sexagesimas assem
dividunt, sed in 1080 partes
ita ut 18 particulae uni minuto
horario respondeant.
\lnr{40}Hac divisione hodie Iudaei,
Samaritani, Arabes, Persae,
et aliae Orientis nationes utuntur.

% 6
% {PDF page nr}{source page nr}{line nr}
\plnr{89}{6}{1}Quorum in sexagesimas, et
contra, sexagesimarum in haec conuertendarum, Tabellas duas posuimus
 (p. \pageref{tab:convertendi_ostenta}).
\begin{table}
\centering
\setlength{\tabcolsep}{3pt}
\renewcommand{\arraystretch}{1.1}
\begin{tabular}{ |r | r  r  r | c |r | r | r r | }
\multicolumn{4}{c}{\scshape Ostenta in sexagesimas}
& \multicolumn{1}{c}{} &
\multicolumn{4}{c}{\scshape Sexagesimas in ostenta}
\\
\cline{1-4} \cline{6-9}
\itshape\footnotesize Ostenta &
\itshape\footnotesize Sexag. &
\itshape\footnotesize Sexag. &
\itshape\footnotesize Sexag. &
\hspace{10mm} &
\itshape\footnotesize Sexag. &
\itshape\footnotesize Sexag. &
\itshape\footnotesize Ostenta &
\itshape\footnotesize Ostenta
\\
\cline{1-4} \cline{6-9}
   1 &  0' &  3'' & 20''' & &  0' &  1'' &    0' & 324'' \\
\cline{2-4} \cline{8-9}
   2 &  0' &  6'' & 40''' & &  0' &  2'' &    0' & 648'' \\
\cline{2-4} \cline{8-9}
   3 &  0' & 10'' &  0''' & &  0' &  3'' &    0' & 972'' \\
\cline{2-4} \cline{8-9}
   4 &  0' & 13'' & 20''' & &  0' &  4'' &    1' & 210'' \\
\cline{2-4} \cline{8-9}
   5 &  0' & 16'' & 40''' & &  0' &  5'' &    1' & 540'' \\
\cline{2-4} \cline{8-9}
   6 &  0' & 20'' &  0''' & &  0' &  6'' &    1' & 864'' \\
\cline{2-4} \cline{8-9}
   7 &  0' & 23'' & 20''' & &  0' &  7'' &    2' & 108'' \\
\cline{2-4} \cline{8-9}
   8 &  0' & 26'' & 40''' & &  0' &  8'' &    2' & 432'' \\
\cline{2-4} \cline{8-9}
   9 &  0' & 30'' &  0''' & &  0' &  9'' &    2' & 756'' \\
\cline{2-4} \cline{8-9}
  10 &  0' & 33'' & 20''' & &  0' & 10'' &    3' &   0'' \\
\cline{2-4} \cline{8-9}
  20 &  1' &  6'' & 40''' & &  0' & 20'' &    6' &   0'' \\
\cline{2-4} \cline{8-9}
  30 &  1' & 40'' &  0''' & &  0' & 30'' &    9' &   0'' \\
\cline{2-4} \cline{8-9}
  40 &  2' & 13'' & 20''' & &  0' & 40'' &   12' &   0'' \\
\cline{2-4} \cline{8-9}
  50 &  2' & 46'' & 40''' & &  0' & 50'' &   15' &   0'' \\
\cline{2-4} \cline{8-9}
  60 &  3' & 20'' &  0''' & &  1' & 60'' &   18' &   0'' \\
\cline{2-4} \cline{8-9}
  70 &  3' & 53'' & 20''' & &  2' &  0'' &   36' &   0'' \\
\cline{2-4} \cline{8-9}
  80 &  4' & 26'' & 40''' & &  3' &  0'' &   54' &   0'' \\
\cline{2-4} \cline{8-9}
  90 &  5' &  0'' &  0''' & &  4' &  0'' &   72' &   0'' \\
\cline{2-4} \cline{8-9}
 100 &  5' & 33'' & 20''' & &  5' &  0'' &   90' &   0'' \\
\cline{2-4} \cline{8-9}
 200 & 11' &  6'' & 40''' & &  6' &  0'' &  108' &   0'' \\
\cline{2-4} \cline{8-9}
 300 & 16' & 40'' &  0''' & &  7' &  0'' &  126' &   0'' \\
\cline{2-4} \cline{8-9}
 400 & 22' & 13'' & 20''' & &  8' &  0'' &  144' &   0'' \\
\cline{2-4} \cline{8-9}
 500 & 27' & 46'' & 40''' & &  9' &  0'' &  162' &   0'' \\
\cline{2-4} \cline{8-9}
 600 & 33' & 20'' &  0''' & & 10' &  0'' &  180' &   0'' \\
\cline{2-4} \cline{8-9}
 700 & 38' & 53'' & 20''' & & 20' &  0'' &  360' &   0'' \\
\cline{2-4} \cline{8-9}
 800 & 44' & 26'' & 40''' & & 30' &  0'' &  540' &   0'' \\
\cline{2-4} \cline{8-9}
 900 & 50' &  0'' &  0''' & & 40' &  0'' &  720' &   0'' \\
\cline{2-4} \cline{8-9}
1000 & 55' & 33'' & 20''' & & 50' &  0'' &  900' &   0'' \\
\cline{1-4} \cline{8-9}
\multicolumn{4}{c}{}      & & 60' &  0'' & 1080' &   0'' \\
            \cline{6-9}
\end{tabular}
\caption{Tabula convertendi ostenta in sexagesimas et vice versa}
\label{tab:convertendi_ostenta}
\end{table}

% Filename relative to the main tex file.
% Suggested improvements for tables:
% - Split in two tables
% - Bigger letters/numbers
% - Remove Smallcaps latin titles from top. Put as caption to each table

\subsection{De Diebus}

\lnr{4}\textgreek{Το νυχθήμερον},
%[Greek: the day and night, i.e. a full 24 hour cycle]
quod est spatium viginti quatuor horarum, Daniel
eleganter vocat \texthebrew{[Hebrew]} quasi dicas
\textgreek{[Greek]}, initio diei civilis
sumto Iudiace ab eo tempore, quod proxime Solem occasum
sequitur.
\lnr{7}Nam illud intervallum, quatenus vigintiquatuor horarum est,
naturale est: quatenus aliud atque aliud initium habet, dicitur civile,
Atticis et Iudaeis ab occasu Solis: Aegyptiis et Romanis at media nocte:
Chaldaeis Genethliacis ab ortu Solis: Umbris at meridie initium
sumentibus.
\lnr{11}Dierum notationes duplices: aut secundum numerum, et
ordinem: ut prima, secunda, tertia mensis.
\lnr{12}Aut secudum \textgreek{[Greek]},
qua dies alicui rei cognomines.
\lnr{13}Ut dies mensis Persici sunt cognomines
regum priscorum: et dies mensis Mexicanorum, animalium, aut aliarum
rerum: et \textgreek{[Greek]} Aegyptiorum nominibus singulorum Deorum
vocatae.
\lnr{16}Et dies festi, ut quinquatrus, \textgreek{κρόνια},
%[Greek: of Kronos, i.e. Saturn]
\textgreek{ϑαργήλια[Greek]}, Quirinalia.
\lnr{17}Et ab eventu, dies Alliensis, Regifugium.
% - Alliensis: Possibly:
% (Dies) (June 16th, BCE 390), when the Romans were cut to pieces by the
% Gauls near the banks of the river Allia; and ever after held to be a dies
% nefastus , or unlucky day. 
% Or: July 18th of 390 BCE (Dies quartus decimus ante Kalendas Augustas) 
% - Regifugium: Roman feast day, celebrating the eviction of the last king,
% Tarquinius Superbus
\lnr{17}At stellis, dies Septimanae.
\lnr{18}Ecclesia Romana vocat ferias.
\lnr{18}Quia veteris anni Ecclesiastici initium
at Pascha.
\lnr{19}Et Pascha dicebatur annus novus, ut etiam hodie ab Ecclesia
Antiochena: at Constantinopolitana autem \textgreek{[Greek]},
ab eadem mente.
\lnr{21}Illius autem Hebdomadis dies omnes septem erant
feriati, ut testis est Hieronymus, et alii veteres.
\lnr{22}Hinc obtinuit, ut reliquarum
hebdomadum dies etiam Feriae vocarentur, praecipuo quodam
principis septimanae Paschalis auspicio et omine.
\lnr{24}Solon autem
primus omnium \textgreek{[Greek]} vocavit,
 cum antea \textgreek{[Greek]} esset
prima mensis.
\lnr{26}Hesiodus: \textgreek{[Greek]}.
\lnr{27}Diei divisio summa ab actibus quotidianis, in fastos, nefastos, atros,
religiosos, intercisos, iustos: ut Graecis
 \textgreek{[Greek]}, vel, ut alii,
\textgreek{[Greek]},
 \textgreek{[Greek]}.
\lnr{29}Aut ab aequatione annui
temporis, Solaris, et Lunaris, in \textgreek{[Greek]},
 \textgreek{[Greek]}, \textgreek{[Greek]},
\textgreek{[Greek]}, \textgreek{[Greek]},
 \textgreek{[Greek]}, \textgreek{[Greek]}.
\lnr{31}\textgreek{[Greek]} Computatoribus
Graecis dicuntur, quae Latinis Regulares, quae cum Concurentibus,
id est Epactis Solaribus compositae dant characterem Kalendarum,
aut alius diei mensis.
\lnr{34}\textgreek{[Greek]} sunt duplicis generis, Solares, et
Lunares.
\lnr{35}Solares fiunt abiectis septenariis ex cyclo Solari, addito praeterea
die bisextili.
\lnr{36}
Lunares producuntur, excessu Solis, qui est \textsc{xi} dierum,
in numerum aureum ducto, abiectis tricenariis.
\lnr{37}Praeterea utrarumque
Epactarum sua methodus: Solarium ad characterem dierum:
Lunarium ad aetatem Lunae, ut Computatores Latini loquuntur, ut
Graeci autem, \textgreek{[Greek]}.

% 7
% {PDF page nr}{source page nr}{line nr}
\plnr{90}{7}{1}\textgreek{[Greek]} sunt, quae eximuntur de
mense, duplici ex causa: aut ut rationes Solis cum Lunaribus congruant,
ut in anno veteri Graecorum: et in enneadecaeteride Paschali
Saltus Lunae Latinis dictus, Graecis \textgreek{[Greek]}.
\lnr{4}aut ut solennia
festa cum feria Septimanae, ut in anno Iudaico.
\lnr{5}\textgreek{[Greek]}, vel \textgreek{[Greek]}
sunt, quae ex caussa religionis, transferuntur, et dissimulantur per speciem
comperendinationis, ut in anno Iudaico, et olim in prisco Romano.
\lnr{8}In Iudaico enim \textgreek{[Greek]} et comperendinationes
 institutae, ne
feria secunda, quarta, sexta in caput anni incurrat. in Romano prisco
comperendinabantur Nundinae, ut at religiosis diebus summoveientur,
auctore Macrobio.
\lnr{11}\textgreek{Εμβόλίμοι [Greek]} sunt, ut notio verbi declarat, insititii
dies: et erant naturales, aut civiles.
\lnr{12}Naturales, qui ex scrupulis, et
horis appendicibus colliguntur, ut quatro quoque anno exeunte unus
dies ex quadrantibus anni Iuliani, quod \textsc{Bisextum} vocatur: item
in periodo Arabica undecies unus dies intercalatur in fine Dulhagiathi,
qui est ultimus mensis anni Hagareni Mohamedici.
\lnr{16}Civiles sunt,
qui praeter naturalem anni rationem et modum inseruntur, ut unus
dies in fine Marcheschuvan Iudaici, anno qui dicitur superfluus, aut
abundans.
\lnr{19}\textgreek{[Greek]}, quae explendis spatiis anni adiiciuntur potius,
quam inseruntur, ut quinque, quae anno aequabili extra ordinem mensium
adiectae Aegyptiis dicuntur \textsc{nisi}, Persis, et Armeniis
 \textsc{musteraka}: 
\lnr{22}item duae, quae extra modum anni Attici in calce Posideonis
appensae, \textgreek{[Greek]} dicebantur,
 aut \textgreek{[Greek]}, aut \textgreek{[Greek]}.
\lnr{24}At \textgreek{[Greek]} locum habent in anno mobili.
\lnr{24}Est autem intervallum
inter epocham et caput anni, utroque termino excluso.
\lnr{25}Hoc
constat semper in annis, quorum caput nunquam epocham antevertebat.
\lnr{27}
Ut in anno Attico caput Hecatombaeonis nunquam ante Solstitii
veterem epocham statuebatur.
\lnr{28}Itaque quod inter Solstitium, et
propositum Hecatombaeonem interiacet spatii, utroque termino excluso,
dicebantur \textgreek{[Greek]}.
\lnr{30}Idem observabatur in annis magnis
Metonis et Calippi.
\lnr{31}Rursus Romanorum sacri dies Kalendae, Nonae,
Eidus: Graecorum autem \textgreek{[Greek]}.
\lnr{32}Quod ex versu Hesiodi at
nobis adductor constat.
\lnr{33}Sunt praeterea nomina imposita diebus mensium
singulis, ut suo loco referetur.
\lnr{34}Sunt et secundum hebdomadas
ut infra subiecimus.
\begin{table}[h]
%\large
\begin{tabular*}%
{\textwidth}{%
@{\extracolsep{\fill} } r r r @{\hspace{4pt}} || r @{\hspace{4pt}} | @{} l 
}
\multicolumn{3}{c}{\textsc{DIES HEBDOMADIS}} &
\multicolumn{2}{c}{\textsc{ALITER PERSICE.}}
\\
\multicolumn{3}{c}{\textsc{persicae.}} & \multicolumn{2}{c}{}
\\
\hline
\texthebrew{שנב} % some random characters as filler text
& \textarabic{شزذيثب} % some random characters as filler text
& \textarabic{ل}
& 1
& \textit{Ruz iache}
\\
\texthebrew{[Hebrew]}
& \textarabic{[Persian]}
& \textarabic{ب}
& 2
& \textit{Ruz duiemi}
\\
\texthebrew{[Hebrew]}
& \textarabic{[Persian]}
& \textarabic{ج}
& 3
& \textit{Ruz siumi}
\\
\texthebrew{[Hebrew]}
& \textarabic{[Persian]}
& \textarabic{ﺩ}
& 4
& \textit{Ruz tzeharmi}
\\
\texthebrew{[Hebrew]}
& \textarabic{[Persian]}
& \textarabic{م}
& 5
& \textit{Ruz pengemin}
\\
\texthebrew{[Hebrew]}
& \textarabic{[Persian]}
& \textarabic{و}
& 6
& \textit{Ruz schesmin}
\\
\texthebrew{[Hebrew]}
& \textarabic{[Persian]}
& \textarabic{ز}
& 7
& \textit{Ruz haphthemi}
\end{tabular*}

\vspace{\baselineskip}

\begin{tabular*}
{\textwidth}{%
    @{\extracolsep{\fill} } r r @{\hspace{4pt}} || r @{\hspace{4pt}} r c
}
\multicolumn{2}{c}{\textsc{TURCIAE HEBDOMADIS}} & \multicolumn{3}{c}{\textsc{SECUNDUM PLANETAS.}}
\\
\multicolumn{2}{c}{\textsc{dies.}} & \multicolumn{3}{c}{}
\\
\texthebrew{[Hebrew]}
& \textarabic{[Arabic]}
& \texthebrew{[Hebrew]}
& \textarabic{[Arabic]}
& \astro{♄}
\\
\texthebrew{[Hebrew]}
& \textarabic{[Arabic]}
& \texthebrew{[Hebrew]}
& \textarabic{[Arabic]}
& \astro{♃}
\\
\texthebrew{[Hebrew]}
& \textarabic{[Arabic]}
& \texthebrew{[Hebrew]}
& \textarabic{[Arabic]}
& \astro{♂}
\\
\texthebrew{[Hebrew]}
& \textarabic{[Arabic]}
& \texthebrew{[Hebrew]}
& \textarabic{[Arabic]}
& \astro{☉}
\\
\texthebrew{[Hebrew]}
& \textarabic{[Arabic]}
& \texthebrew{[Hebrew]}
& \textarabic{[Arabic]}
& \astro{♀}
\\
\texthebrew{[Hebrew]}
& \textarabic{[Arabic]}
& \texthebrew{[Hebrew]}
& \textarabic{[Arabic]}
& \astro{☿}
\\
\texthebrew{[Hebrew]}
& \textarabic{[Arabic]}
& \texthebrew{[Hebrew]}
& \textarabic{[Arabic]}
& \astro{☾}
\end{tabular*}
\caption{Dies Hebdomadis}
\label{tab:dies_hebdomadis}
\end{table}


% 8
% {PDF page nr}{source page nr}{line nr}
\plnr{91}{8}{1}Cur autem dies cognomines Planetarum non sequuntur ordinem et
situm siderum, quorum cognomines sunt, ut scilicet post diem Saturni
non sequatur dies Iovis, sed dies Solis, haec caussa est.
% Diagram: circle with heptagram, with planets at the points:
% Moon ☾, mercury ☿, venus ♀, sun ☉,
% mars ♂, jupiter ♃, saturn ♄
%\begin{wrapfigure}[9]{r}{10\baselineskip}
\begin{wrapfigure}[9]{R}{9\baselineskip}
  \centering
  \def\svgwidth{9\baselineskip}
  {\astrofont\input{./img/planets.pdf_tex}}
\end{wrapfigure}

\lnr{3}Septem Planetae
per circulum secumdum ordinem suum
dispositae, aequabili intervallo constituunt septem
Triangula isoscele ad peripheriam, quorum
bases sunt latera Heptagoni circulo inscripti,
ut habes in circulo proposito, ad cuius
peripheriam septem errantes sunt secundum
feriem suam sitae, constituentes triangula
isoscele \astro{♄♀♃}, \astro{♃☿♂}, \astro{♂☽☉},
 \astro{☉♄♀}, \astro{♀♃☿}, \astro{☿♂☽}, \astro{☽☉♄}.

\lnr{12}In quibus Triangulis dexter angulus ad basim
est prima stella Trianguli, secunda in angulo ad verticem, tertia angulus
sinister ad basim: ita ut omnis stella anguli dextri habeat oppositam
stellam anguli in vertice, stella autem anguli at vertice stellae
anguli sinistri ad basim sit opposita.

% 9
% {PDF page nr}{source page nr}{line nr}
\plnr{92}{9}{2}Sequentur igitur sese omnes septem
Planetae non per seriem suam, sed per intervalla laterum, quae
verae sunt oppositiones.
\lnr{4}Sit igitur Triangulum \astro{☉☽♂} primum ordine.
\lnr{5}\astro{☉} in angulo basis dextro praeibit. sequetur Luna ei
 opposita in vertice,
eam oppositus Mars in angulo sinistro basis.
\lnr{6}Qui quidem Mars cum in
Tiangulo \astro{☉☽♂}, sinistrum angulum basis occupet,
 in triangulo \astro{♂☿♃} occupabit
dextrum basis angulum, habens oppositum Mercurium,
Mercurius autem oppositum Iovem in angulo sinistro.
\lnr{9}qui Iuppiter
faciet angulum dextrum in Triangulo \astro{♃♀♄}, habens oppositam in vertice
Venerem, ut ea opposita est Saturno in angulo sinistro.
\lnr{11}Sed angulus
ille rursus erit dexter in Triangulo \astro{♄☉☽}.
\lnr{12}Et sic erogati sunt septem
planetae in totidem dies, quas Ecclesia Romana vocat ferias.
\lnr{14}Haec est vera harum appelationum ratio.

\subsection{De Mensibus.}

\lnr{15}Ex diebus fiunt \textgreek{[Greek]}, quae notationes et epochas
temporum constituunt.
\lnr{2}Primum \textgreek{[Greek]} ex diebus dicitur Septimana,
res omnibus quidem Orientis populis ab ultima usque
antiquitate usitata,
 nobis autem Europaeis vix tandem post Christianismum
recepta.
\lnr{19}De ea iam dictum est.
Tum Romanorum \textgreek{[Greek]}: cui
successit hebdomas nostra.
Nam nono quoque die Nundinae erant.
et spatium illud in Kalendario vetere Romano notatum est literis ab
\textsc{a} ad \textsc{h}, ut in nostro Kalendario Hebdomas
 notata est ab \textsc{a} ad \textsc{g}, inclusive,
ut loquuntur.

Mexicanorum \textgreek{[Greek]} sequitur.

Quod
enim spatium nobis septenis diebus, illis finitur ternis denis.

Ita Iudaeorum
est \textgreek{[Greek]} veterum Romanorum \textgreek{[Greek]}, Mexicanorum
\textgreek{[Greek]}.

Proximum ab hoc \textgreek{[Greek]} dierum est Mensis:
qui et naturaliter, et civiliter sumitur.

Naturalis mensis et ipse duplex.

Aut enim Lunaris, aut Solaris.

Rursus Lunaris triplicis generis:
aut quatenus Luna ab eodem puncto Zodiaci profecta, ad idem
revertitur: qui dicitur \textgreek{[Greek]}, item \textgreek{[Greek]}.

quod intervallum
minus est, quam viginti octo dierum: maius quam viginti septem.

Secundum genus est eiusdem sideris at Sole profecti ad eundem
reditus.

Haec dicitur \textgreek{[Greek]}.

Tertii generis mensis est secundus
dies \textgreek{[Greek]}, quae dicitur \textgreek{[Greek]},
 et \textgreek{[Greek]}.

Secundum et tertium genus in temporibus civilibus locum habent.

Nam Athenienses \textgreek{[Greek]} neomenias suas putabant: hodie vero
Hagareni \textgreek{[Greek]}.


p. 10 [pdf 93]

Graecorum enim neomenias ab ipso iugo
Lunae putari solitas testis Vitruvius ex Aristarcho Samio, his verbis,
loquensde Luna:

\textit{Quot mensibus sub rotam Solis radiosque primo die
antequam praeterit, latens obscuratur.}
 \textit{et, cum est sub Sole, nova vocatur.}

\textit{Postero autem die, quo numeratur secunda,
 praeteriens at Sole, visitationem
facit tenuem extremae rotundationis.}
%[Vitruvius, De architectura libri decem, Liber IX, Capitulum II, Sect. 3:
% "Ita quot mensibus sub rotam solis radiosque uno die, antequam praeterit, 
% latens obscuratur. Cum est cum sole, nova vocatur. Postero autem die, quo 
% numeratur secunda, praeteriens ab sole visitationem facit tenuem extremae 
% rotundationis." 
% Transl.: "whence, on the first day of its [the moon's] monthly 
% course, hiding itself under the sun, it is invisible; and when thus in 
% conjunction with the sun, it is called the new moon. The following day, which 
% is called the second, removing a little from the sun, it receives a small 
% portion of light on its disc."]

Ubi etiam dixit visitationem
extremae rotundationis, quam ille Samius sine ullo dubio
 \textgreek{φαίσιν μιωοειδῆ[Greek: phase …]} vocabat.

Sed et Onomacritus, qui sub nomine Orphei \textgreek{τελετὰς[Greek: ceremony]}
scripsit, in opere, quod
 \textgreek{ἡμέρας[Greek: day, hours of daylight]}
 vocavit, mensem Lunarem at iugo Lunae
incipit.

Cuius versus apposui:

\begin{greek}
Παίτ᾽ ἐδάης Μουσαῖε ϑεοφραδἐς. εἰδέ σ᾽ αἰώγει[Greek ?]

ϑυμὸς ἐπωνυμίας μήνης κατὰ μοῖραν ἀκοῦσαι,[Greek ?]

ῤεῖά τοι ἐξερέω, σὺ δ᾽ ἐνὶ φρεσὶ βάλλεο σῆσιν,[Greek]

οἵην τάξιν ἔχοντα κυρεῖ. μάλαν γαρ χρέος ἐστὶν[Greek]

ἴδμεναι, ῶς αὕτη παρέχει κλέος ἄντυγι[?] μηνός.[Greek]

\textgreek{[Greek]}

\textgreek{[Greek]}

\textgreek{[Greek]}

\textgreek{[Greek]}

\textgreek{[Greek]}

\textgreek{[Greek]}
\end{greek}

Sed Neomenia Arabica, excedit modum
 \textgreek{φάσεως[Greek: phase]} ut plurimum. ita ut
civiles neomeniae mensium Lunarium sint non unius generis: Atticae
\textgreek{[Greek]}: Iudaicae saepe \textgreek{[Greek]}.

Arabicae semper \textgreek{[Greek]},
at tertia, inquam, die.

Mensis Solis naturalis est,
qui naturalibus circuli coelestis segmentis definitur, qualis est transitus
Solis at signo ad signum.

Hi, et Lunares, sunt vere coelestes menses.

Mensis civilis Solis est, qui non naturali modo, sed aequaliter tributus
est. ut in anno Aegyptiaco et Graeco omnes aequaliter sunt \textgreek{[Greek]}:
et in Lunari alternis pleni, et cavi. in anno Mexicano \textgreek{[Greek]}
cum ex \textsc{xviii}. mensibus eorum annus constituatur.

Apud Albanos
Martius erat sex et triginta dierum, Maius viginti duum, Sextilis
duodeviginti, September sedecim.

Tusculanorum Quintilis habuit
triginta sex, October triginta duos, Aricinorum October trigintanovem.

At rationes Lunae non patiuntur, ut menses sint alternis
perpetuo pleni, et cavi. sed hoc ad methodum civilis temporis institutum.

Sunt et alii menses ex superfluis diebus collecti, qui Embolimi
dicuntur: iique aut naturales, aut civiles: ambo autem ad aequationem
Solis directi.

Naturales embolimi sunt, qui ex Solis excessu collecti
ad spatia Lunae complenda adhibentur. cuiusmodi est Iudaicus
Adar prior, et Samaritanus Adar alter.

isque mensis est semper tricenum dierum.

Civilis embolimus, qui ex diebus Solis superfluis consurgens
fulciendo anno cavo adiictur.


p. 11 [pdf 94]

Eiusmodi erat Merkendonius
prisci anni Romani alternis binum et vicenum, item trinum et vicenum
dierum.

Eiusmodi et Posideon Atticus.

Neque enim Posideon
naturalis esse potest, quamuis triginta dierum, cum neque Lunaris
esset, quod eius neomenia longe at lunari discederet: neque Solaris,
quod pars esset illius anni, qui ad Solis cursum descriptus non esset.

Idem de Merkedonio dicas, qui neque ad Solarem annum, neque ad
Lunarem pertineret, neque modum eum haberet, qui iusto mensi
competit, cum esset tantum \textsc{xxii}, aut ad summum \textsc{xxiii} dierum.

Mensis divisio Atticis in \textgreek{[Greek]}.
 prima \textgreek{[Greek]} dicebatur \textgreek{[Greek]},
secunda \textgreek{[Greek]}, tertia \textgreek{[Greek]}.

Idque factum, quia
illorum menses omnes erant \textgreek{[Greek]}.

Persae vero in \textgreek{[Greek]},
non solum, quia eorum menses omnes \textgreek{[Greek]}, sed etiam, quia
totus annus constat ex quinariis tribus et septuaginta.

In mense \textgreek{[Greek]}
Athenienses pro \textgreek{[Greek]} dicebant \textgreek{[Greek]}.

Quamuis
enim mensem uno die mutilabant, tamen cum tertia mensis
pro secunda dicebant, non videbantur mensem mutilare, cuius
\textgreek{[Greek]} numerabant.

Meton vero et Calippus eam diem eximunt,
quae post duas syzygias et dies quatuor succedebat.

Mensium nomina
in antiqua Hebraici anni forma nulla fuerunt, neque in hodierna
Sinarum, Iaponensium, et Indorum.

Menses enim illis ab ordine
primi, secundi, tertii dicuntur.

In anno Romano mistae sunt appellationes,
ex cognominibus, et ordine numerario.

Quidam etiam cognomines
imperatorum Romanorum, ut Cypriis \textgreek{[Greek]}.

Romanis ipsus Iulius, Augustus: et temporibus Domitiani
Germanicus pro Septembri, Domitianus pro Octobri.

% Insert: page of latin text "M. AVR. AVG. LIB."
\begin{wrapfigure}[16]{R}{0.5\textwidth}
  \centering
  {M. AVR. AVG. LIB.}
\end{wrapfigure}

Martialis:
 \textit{Dum Ianus hiemes, Domitianus
autumnos}, etc.

Sed Statius omnes
Kalendas vindicat Domitiano,
praeter Iulium, et Augustum,
– \textit{Nondum omnis honorem
Annus habet, cupiuntque decem tua
nomina menses.}

Insania quoque
Commodiidem cōsecuta esset, si
longior vita mōstro illi data fuisset.

Augustum enim Cōmodum,
Septembrem Herculeum, Octobrem
Inuictum, Novembrem
Exuperatorium, Decembrem
Amazonium vocari edicit.


p. 12 [pdf 95]

Extat
quoque lapis Lavinii, in quo mentio
Iduum Commodarum. ubi et
nomenCommodi Senatusconsulto prius derasum, postea alia manu
incusum.

 Quaedam nationes etiam geminos menses cognomines habent.

Annus Syrochaldaicus habet geminum Tisrin, item geminum Conum.

Annus Hagarenus geminum Regiab, et geminum Giumadi.

Annus Sxonicus geminum Giuli, et geminum Lida.

Sed in
anno embolimaeo Lida est tergeminus.

Et tunc annus ille dicebatur
Trilida.

Item, diversarum nationum iidem menses communes.

Nam
Panemus in anno Macedonico fuit, item Corinthiaco, et Thebano.

Artemisius communis fuit Laconum, et Macedonum: Carneus Syracusanis,
et Cyrenensibus usitatus.

Sed differbant situ anni et tempore:
ut suo loco disputabitur.

Sic Martius primus eratRomanorum:
tertius Albanorum, Aricinorum, Formianorum: quartus Forensium,
Pelignorum, Sabinorum: quintus Faliscorum, Laurentum:
sextus Hernicorum: decimus Aequicolorum. Haec in genere
de mensibus.


\subsection{De Anno.}
\setcounter{parcount}{0}

\dropcap{3}{M}{Aximum} \textgreek{[Greek]}
 dierum annus, sed qui multipliciter dictus
sit. \\ \p
Tot enim constitui possunt, quot sunt siderum errantium
periodi.\\ \p
Est enim annus circuitus eius periodi, cuius cognominis
ipse est.

Ut annus Solaris est cognominis circuitus eiussideris,
qui quidem circuitus dupliciter sumitur, aut at Solstitio ad Solstitium,
at bruma ad brumam: et est minor anno Iuliano.

aut at puncto Zodiaci,
ad idem punctum Zodiaci.

qui est maior anno Iuliano.

hoc est maior 365 1/4 diei.

quo ad id puncum Zodiaci redit, unde profectum
erat.

Eadem fere quantitas quae et Soli, attribuitur Veneri et Mercurio.

Saturni periodus est dierum 10747.18'.59''.13'''.

Hoc est annorum
Aegyptiorum 29. dierum 162.

Iovis annus dierum 4330. horarum 17.14'.

Id est annorum Aegyptiorum 11.315.

Martis annus dierum
686. horarum 22.24'.

annorum Aegyptiorum 1.321 dierum.

Lunae,
dierum 29.31'50''.8'''.

Obtinuit tamen vulgo, ut duorum siderum,
Solis et Lunae, labentem coelo qui ducunt annum, ratio intemporibus
civilibus haberetur.

Et Lunae quidem primum unus circuitus
pro anno habebatur, ut apud Aegyptios. deinde tres, ut apud eosdem
Aegyptios et Arcades.

Tandem duodecim periodi Lunares annum
civilem constituerunt dierum 354 cum triente, et paulo plus quam
duum trientum horariorum.

Duodecim quoque segmenta Zodiaci
componunt annum Solarem tantum, quantum diximus.


p. 13 [pdf 96]

Sed ignoratio
motuum utriusque sideris alias atque alias anni formas veteribus
peperit:

quarum vetustissima est ea, quae annum quidem ad cursum
Lunae describebat:

sed incertis neomeniis, quae non produent ex obseruatione
motus Lunae, quales vulgus rusticorum obseruare solet, et
quae proprie civilem mensem constituere non possunt.

Cum igitur
hoc modo incertae essent neomeniae, convenit primum, ut menses omnes
tricenis diebus explicarent, annumque dierum sexaginta et trecentum
constituerent.

quod genus longe desciscebat at modo anni
Lunaris.

Haec diu seruata fuit apud Graecos anni forma.

In Oriente
septuagesima secunda pars illius anni, hoc est quinque dies, accesserunt
anno Graeco: ut anni modus fuerit dierum trecentorum sexaginta quinque:
qua ratione ab anno solari se minimum discedere arbitrati sunt.

Unde duo praecipua genera anni apud veteres suerunt neque Lunaria,
neque Solaria, sed ambigui inter utrumque generis.

Prior forma in Graecia resedit: altera in Oriente.

Graeci vero non una via ad emendationem
suae aggressi sunt.

Difficile erat menses plenos omnes ad
Lunae rationes exigere: et tamen in quibusdam actibus civilibus opus
habebant motu Lunae.

Nam semper Olympias plenilunio, et \textsc{xv}
die mensis celebrabatur.

Ut igitur annus Graecus aequabilis Olympiadem
deprehenderet in \textsc{xv} mensis, hoc difficile non erat.

Ut autem
\textsc{xv} mensis in \textsc{xv}
 Lunae incidat in mensibus aequabilibus, hoc fieri non
potest, nisi post fingula quadriennia, adiectis unicuique anno singulis
biduis, quas \textgreek{[Greek]} vocabant.

Haec Tetraeteris Elidensibus
vocata est Olympias, Delphis Pythias.

eiusque mensis primus duantxat
erat Lunaris: reliquorum ratio claudicabat.
% written as: claudicābat.

Primus Cleostratus
eum annum in Lunarem modum reformare conatus est, excogitata
octaeteride dierum 2922, cuius menses alternis pleni et cavi: anni vero
singuli communes 354 dierum: embolimaei 384. communes quidem
quinque, embolimaeitres.

Syzygiae autem novem et nonginta.

Octaeteridum
vitio deprehenso, Meton enneadecaeterida excogitavit dierum
solidorum 6940.

Cui castigandae periodus Calippica successit dierum
27759, sine ullis scrupulis appendicibus, anno ab editione Metonica
centesimo tertio.

Hanc excepit ultimus, tanquam secutor quidam,
Hipparchus, annis circiter centum octoginta octo ab epocha Calippica,
periodo publicata dierum 111035: quae minor est Calippicis rationibus
die uno, Metonicis autem quinque.

Quare duae castigationes adhibitae
anno aequabili Graeco.

Altera est coniugatio alterna vel interrupta
mensium plenorum et cavorum, ut cum ipsa Luna congruerent, quod
annus Graecus maior esset Lunari.

altera est embolismus mensium, ut
cum sole aequaretur, quod annus Lunaris minor est Solari.

Sed alternatio
plenorum et cavorum mensium aliquando variat: idque fit aut
naturaliter, aut civiliter.


p. 14 [pdf 97]

Naturalis varietas committitur propter embolismum
aut mensis, aut diei.
% oddly, mletter stops the next paragraph from being counted
% or rather: it pushes the number downward; Letter and number are never
% on the same line.

Utroque enim modo duo menses pleni continuantur.

Ut in anno Iudaico cum intercalatur mensis Adar, tunc
Schebat, et Adar embolimus ambo sunt pleni.

In anno vero Arabico
cum accedit dies mensi ultimo, qui Dulhagiathi dicitur, tunc et ipse
Dulhagiathi, et antecedens Dulkaadathi ambo fiunt tricenum dierum.

Sed in Samaritano saepe continuantur tricenarii menses, et in antiquo
Iudaico, ut ex Talmud et Iad Mosis cognoscimus: et menses Harpali,
Metonis, et Calippi non semper alternis continuati sunt.

sed saepe bini
pleni continuati, nunquam autem bini cavi.

Quin etiam cum dies accedit
ultimo mensi Arabico, tres continui menses sunt pleni, Dulkaadathi,
Dulhagiathi, et Muharam sequentis anni.

Isque annus ab Arabibus
dicitur \textarabic{[Arabic]} hoc est embolimaeus.

Sic etiam anno Iudaico pleno
tres menses continui sunt pleni, Tisri, Marchesuvan, Casleu.

Civilis
varietas accidit anno Iudaico tantum, accrescente mensi Marcheschuvan
die uno: et Marchesuvan ex cavo sit plenus.

Rursus et in embolismo
mensium differentia situ, et tempore.

Situ, si aut in medio, aut in calce
intercalatio fiat.

ut in anno Attico ultimus mensis intercalabatur, qui
dicebatur \textgreek{[Greek]}.

In Iudaico sextus mensis intercalatur, et
dicitur Adar prior.

In anno Hagereno mensis embolimus erat desultor,
qui omnes menses anni percurrebat in annis 228, quae sunt enneadecaeterides
duodecim.

qua intercalatione memoria proavorum nostrorum
utebantur Turcae Cilices, donec annum Hegirae simplicern
Muhamedicum usurpare coeperunt.

At in anno prisco Romanorum
situs embolismi longe diversus ab aliis.

non enim is inter duos
menses interiiciebatur, ut alias solet: sed in mensem ipsum, tanquarn
surculus in truncum infindebatur.

Inter \textsc{xxiii} enim, aut \textsc{xxiiii},
aut inter \textsc{xxii}, et \textsc{xxiii} Februarii inserebatur.
neque vero sine caussa.

Hoc enim semper obseruabant, ut mensis proximus Martio semper esset
dierum \textsc{xxviii}.
eratque Februarius ordinarius. at intervallum inter exitum
Ianuarii, et Kalendas Februarii ordinarii imputabatur Merkedonio.
et Kelendae Februarii ordinarii in anno embolimaeo nunc in Regisugium,
nunc in Terminalia, incurrebant.

Neque enim semper inter
Terminalia, et Regisugium intercalabantur, ut vult Censorinus.

quia hoc pacto Februarius ordinarius nunc viginti octo; nunc undetricenum
dierum fuisset.

Quod tamen salsum ex Varrone conuic[ē]tur.

Tempore differt intercalatio, quatenus Iadaei nunquam intercalant,
priusquam \textgreek{[Greek]}, qui sunt dies decem cum horis paulo
magis quam una et viginti, eo rationes Solis deduxerint, ut commode
mensis Lunaris conflari possit.

Quod spatium numquam maius est
triennio, nunquam minus biennio: et in \textsc{xix}. annis semper septies fit.


p. 15 [pdf 98]

At in Calippico et Metonico anno aliquando citius, aliquando ferius
intercalabatur, quam ratiocinia \textgreek{[Greek]} postulare videntur.

quandoquidem hoc unum cavent praecipue Athenienses, ne Hecatombaeonis
neome[?]a Solstitii priscam epocham antevertat: cum in
anno Iudaico ut plurimum neomenia Tisri aequinoctium autumnale,
neomenia vero Nisan aequinoctium veris antiquum, si ratio Iuliani
anni habeatur, antevertat.

Anni Lunaris non unum genus est: sed
summa divisio in duo fastigia discedit: in annos periodicos, et simplices.

Anni periodici dicuntur, qui certo annorum orbe, interuentu
embolismorum, recurrunt.

Huius intervalli modum veteres certo
definire non potuerunt. quippe Cleostratus dierum 2922, Harpalus
2924, Eudoxus plusquam 2922, minus quam 2924: Meton aliter:
et ab omnibus diverse Calippus, et denique ab eo discedens Hipparchus.

Cuius sententia, sed caelestibus rationibus leviter castigata, enneadecaeterida
Lunarem minorem Iuliana statuit, hora una cumscrup. paulo
plus quam viginti septem.

Simplices anni et ipsi quidem sine remedio
intercalationis in pristinam epocham recurrunt, sed longo intervallo,
annorum scilicet Iulianorum 228, qui sunt anni simplices Arabici 235,
scrupuli diurni quinquaginta.

Sunt et in annis Lunaribus cavi, superflui,
aequabiles.

Annus cavus is est, cui competit \textgreek{[Greek]}.

Ideo at nobis \textgreek{[Greek]} vocabitur. ex eo enim eximitur dies
vel propter civile institutum, cuiusmodi est annus Iudaicus,
quem defectivum
Computatores Iudaeorum vocant.

(in eo quippe Casleu, qui natura est plenus, instituto fit cavus.)

vel naturali de caussa: ut anno
decimonono Cycli Paschalis Dionysius diem unum eximit, quem
vocavit Saltum Lunae: Graeci vero Computatores \textgreek{[Greek]}.

quamquam inepte annum ultimum enneadecaeteridis constituit dierum
duntaxat 353, cum eiusmodi annus natura nullus fit.

Superfluus
annus vocetur at nobis \textgreek{[Greek]}.

Accedit enim illi \textgreek{[Greek]}
tam ex caussa civilli, ut in anno Iudaico marcheschuvan naturaliter
cavus, civiliter fit plenus: quam e caussa naturali: ut undecim anni
in Triacontaeteride Arabica augentur singulis diebus ex ratiociniis
Lunae collectis.

Annus aequabilis vocetur \textgreek{[Greek]}.

Iudaeis computatoribus
dicitur annus ordinarius.

Is est, cui nihil accedit, nihil decedit.

Huc usque ad annum Lunarem deduxit nos aequabilis minoris
disputatio.

Nunc de altero aequabili maiore disputandum, quo Aegyptii,
Persae, et Armenii, Mexicani, et Perusiani usi.

Hic antiquitus
Orientis nationibus unus idemque fuit: praeter quam si quando \textgreek{[Greek]}
quinque in alium locum traductae, diversum anni caput constituebant.

qua \textgreek{[Greek]} tralatione utebantur ii, qui post annos 120
aequabiles mensem solidum intercalabant, ut Persae: qui quidem \textgreek{[Greek]}
suas in aequinoctium vernum semper reiiciebant.


p. 16 [pdf 99]

Terminum autem vocabant \textsc{neuruz}.

et habebant mensem desultorem
\textgreek{[Greek]}, omnes menses anni peruagantem, donec in primum
mensem recurreret.

qui orbis non redibat, nisi anno aequabili 1461
vertente, qui sunt anni Iuliani perfecti 1460.

Hic est magnus annus,
cuius menses sunt annorum aequabilium tricenum, quot dierum simplex
mensis.

\textgreek{[Greek]} autem sunt quinquies quatuor annorum, ut
illae simplices quinque dierum.

Quod autem illa anni forma retenta
fit, in caussa fuit non tam ignoratio annis solaris,
 quam facilis, et tractabilis, ac vere popularis eius usus.

Alioqui nulla fere natio fuit, quae
quadrantem anni Solaris ignorarit: sed modum illius dispensandi
nesciebant.

praeterea at mensibus superfluis, qui sunt maiores tricenis
diebus, refugiebant, quos necesse est retincri,
 quadrante illo retento.

Aegyptii singulis quadrienniis exactis diem intercalabant in ortu Caniculae,
et quadriennium illud exactum \textgreek{[Greek]}, \textgreek{[Greek]},
\textgreek{[Greek]}, vocabant.

Attici diem quarto quoque anno exacto intercalabant
inter septimum et octavum diem Ianuarii.

Elidenses inter
octavum, et nonum Iulii.

Syromacedones, Chaldaei, et Iudaei inter
septimum et octavum Octobris.

Eamque diei intercalationem at Seleucidarum
temporibus usque ad imperium Constantini et infra retinuerunt
Iudaei: quam utique simul cum anni Calippici forma at victoribus
Syromacedonibus acceperant.

Romani Atticos secuti brumae
sidere confecto intercalabant; quae ipsis Olympiadum mysteria vocabantur.

Nam et Attici et reliqui omnes Graeci annum Solarem in
quatuor quadrantes dividebant, quae \textgreek{[Greek]}
 vocabant, singulis dies 91.
hor. 7 ½ attribuentes.

quod at temporibus Seleucidarum, ad hanc usque
diem, Iudaei constanter obseruant.

Itaque \textsc{viii} Iulii erant \textgreek{[Greek]},
\textsc{vii} Octobris \textgreek{[Greek]}:
 \textsc{vii} Ianuarii \textgreek{[Greek]}, \textsc{viii}
Aprilis \textgreek{[Greek]}.

Quare cum legis \textgreek{[Greek]}, et \textgreek{[Greek]},
nullas alias intellige, praeter has.

quod et \textgreek{[Greek]} quoque intelligendum.

Haec \textgreek{[Greek]} Iudaei Tekuphoth vocant.

Germani, Celtae,
Saxones inter \textsc{xxv et xxvi} Decembris intercalabant: quam noctem
vocabant \textsc{mudranecht}.

Tartari hodi inter ultimam Ianuarii,
et Kalendas Februarii. quas Kalendas patrio sermone Festum Alborum
vocant. quia albis vestibus eam diem colunt.

Denique quanuis
Lunari anno, aut alio longe diverso at Solari uterentur, tamen tacita
quadam obseruatione post dies 1460 unum diem intercalandum esse
sentiebant.

Neque enim aliter Habraei quatuor Tekuphas suas tueri
potuissent, nisi quadrante post quartum quemque annum rationibus accedente.

Et sane unaquaeque Tekupha est dierum 91, horarum 7 1/2 Unde
quatuor tantae Tekuphae fiunt dies 365 1/4.


p. 17 [pdf 100]

Displicuit tamen haec quadrantis
obseruatio Graecis Astronomis, propter causam admodum futilem
et puerilem, qua Solis quantitatem ad Lunae ratiocinia exigebant,
et cum utriusque sideris exactum modum adhuc non tenerent,
ex Lunae comparatione Solares rationes eliciebant.

Itaque tantam
censuerunt Solis quantitatem, quantam summam dies periodi in annos
periodi distributae relinquebant.

Metonis periodus est dierum
6940.

Divisa per 19 annos relinquit quantitatem anni Solaris Metonici
dierum 365. scrup. diurnorum 15 5/19 Calippi periodus dierum
27759 per 76 annos divisa relinquit modum anni Calippici Solaris
dierum 365 1/4 qualis est annus noster Iulianus.

Periodus Hipparchi
est dierum 111035, annorum 304.

Sed neglectis illis 4,
trecentesima pars diei detrahitur de quantitate anni Calippici Solaris,
ut fiat annus Solaris Himmarchus
 dierum 365. hor. 5. 55.' 15.'' 15/19

Detractis ex quadrante hor. 0. 4.' 44.'' 4/19 quae etiam fuit sententia
Ptolemaei.

Itaque ex sententia Hipparchi et Ptolemaei annus
Tropicus, est annus Iulianus, vel Calippicus nonadecima parte
differentiae enneadecaeteridis Lunaris et Iulianae diminutus: qui
est verus annus Rabbi Ada: de quo alibi.

Philolai Pythagorei magnus
annus dierum 21505 1/2 per 59 annos divisus constituit modum
Solarem dierum 365.

Oenopidae annus magnus dierum 21557
itidem per 59 annos divisus dat modum anni Solaris dierum 365 cum
parte dierum duum et viginti undesexagesima.

Harpali octaeteride per
8 annos divisa remanet modus anni Solaris dierum 365 1/2.

Annus magnus
Democriti dierum 29950 1/2 per 82 annos divisus relinquit annum
Solarem dierum 365, cum quadrante et centesima sexagesimaquatra
parte unius diei.

Denique nullus veterum non patavit rationes
Solis ad Lunam exigendas esse.

Et quotiescunque ex certa collectione
dierum utriusque sideris rationes congruerent, dies illi per to[t]
annos divisi, quot ex illa summa dierum constitui poterant, visi sunt
illis certam anni Solaris quantitatem feninire posse.

Sapientiores vero,
quanuis incomprehensibilem illam existimarent, tamen pro vero quod
proximum putabant amplexi sunt, dies trecentos sexaginta quinque
cum quadrante, qui est modus anni Iuliani.

cui singulis quadrienniis
exactis unus dies accrescit.

sed hic annus comparatione Aegyptiaci
est Solaris: comparatione autem Tropici est aequabilis.

Maior
enim est vera anni ratione scrup. horariis 11.' 6.'' 40.'' secundum
Gelalaeam formam, aut 10.' 48.'' fere, ut Alfonsini docent.

Neque
Prutenicae tabulae multum abludunt, quae constituunt motum
aequalem Solis ab aequinoctio dierum 365. Hor. 5. 49.' 15.'' 46.'''

Itaque hinc nasci possunt aliquot genera anni Solaris.

Aequabilis,
ut Iulianus.

Tropicus, ut Persarum Gelalaeus.

Rursus Tropicus
aut aequabilis, aut caelestis.


p. 18 [pdf 101]

Aequabilis
Tropicus, cuius quantitas
Tropica est, partes autem, hoc est menses, aequales et civiles: ut is,
quem modo dixi, Galelaeus.

Descriptus est enim mensibus aequalibus,
omnibus tricenum dierum, cum epagomenis appendicibus, quae
in communi anno sunt quinque, in embolimaeo sex.

Caelestis Tropicus,
cuius partes in naturalia Zodiaci segmenta tributae sunt.

Rursus
et annus Solis aequabilis in civilem et caelestem dividi potest.

Civilis,
ut Iulianus Romanorum, Syrograecorum, Graecorum Elkupti.

Caelestis,
ut Dionysianus Prolemaei Philadelphi.

Nam et is quoque quadrantem
Canicularem quadriennio exactor accipiebat.

Finis vero
omnis periodi est, ut caput recurrat et revoluatur in idem principium,
quam \textgreek{[Greek]} Graeci vocant: quae quidem pessum iverit tandem,
non seruata veri anni Tropici mensura.

et qua annus Iulianus
suam tueri non potuit, manifestum est Kalendas Ianuarias ab \textsc{viii}
parte Capricorni, in qua statuerat eas Caesar, in vicesimam primam
fere traductas esse hodie.

Sed nihilo commodius epocha in enneadecaeteride
seruari potest.

Nam enneadecaeteris Tropica est velocior
Lunari horis plusquam duabus.

Contra enneadecaeteris Iuliana
maior Lunari hora una, et scrup. plusquam 26.

Cum vero peccatur
utraque ratione, Tropica et Iuliana, Luna, cuius rationes mediae sunt
inter illas duas, fines epochae suae tueri non potest: ut in cyclo Dionysii
Paschali accidit, cuius neque rationes ad enneadecaeterida Luanrem
collectae sunt, neque epocha ad Solis motum castigata: sed eius
forma potius tota mere Calippica est.

ita ut eius statum post trecentos
4. annos variare necesse sit.

Quare ut epochat suas seruarent illi veteres,
immanes periodos excogitaverunt, quales illae Calippi, Philolai, Democriti,
Ocnopidae.

Sunt etiam periodi, quae omnem modum excedebant.

Et cum in omnibus illis orbibus annorum praecipuam
utriusque sideris ationem haberent, tamen nescio quae confidens eos
incessebat opinio, non solum utriusque sideris, sed etiam omnium
\textgreek{[Greek]} illo circuitu fieri.

Sic Harpalus et Eudoxus putarunt
in sua Octaeteride omnes \textgreek{[Greek]}
 et \textgreek{[Greek]} in orbem redire.

Idem etiam censet fieri Aratus in Metonica enneadecaeteride, Eudoxum
suum sectus, qui in fabrica Sphaerae suae eam planetarum et inerrantium
harmoniam in eorum orbibus ostendit esse, ut sequente
restitutione utriusque sideris, necessario et omnium inerrantium reditum
contingere concluderet.

Propterea tot Sphaeras \textgreek{[Greek]} commentus
est, quot narat Aristoteles libro \textsc{xi} \textgreek{[Greek]} quem
consulas licet.

Quin etiam Calippus alios orbes praeter Eudoxum
addidit, ea ratione, ut \textgreek{[Greek]} adstrueret, \textgreek{[Greek]},
ut Aristoteles de ea re scribens pronunciavit.


p. 19 [pdf 102]

Itaque \textgreek{[Greek]} nomine intelligendum ortus,
 et occasus \textgreek{[Greek]},
non autem \textgreek{[Greek]}, hoc est significationes
eorum: quas in orbem redire cum Luna et Sole in enneadecaeteride
Meto quidem, Calippus, et Hipparchus putarunt, et aliis
persuaserunt, donec deprehenso vero anni Tropici modulo vitium
harum periodorum castigatum est.

Cicero quoque apud Macrobium,
sexto de republica, annum illum immanem, quem ex tot millibus
annorum simplicium componit, non aliter in orbem rediturum
cum omnibus errantibus et inerrantibus censet, quam si eadem defectio
Solis in eodem loco, eodem tempore fiat: quanuis defectiones
cyclo enneadecaeterico recurrant non raro.

Et tamen ea eclipsi putat
non tantum Solis et Lunae, sed etiam quinque errantium ad eandem
inter se comparationem, confectis omnium spatiis, reditum fieri, quo
eadem caeli positio, siderumque, quae ab initio maxime fuit, rursus existit.

Quare eclipses ad eam rem notabant veteres, ut etiam \textgreek{[Greek]}
excogitarint \textgreek{[Greek]} vocabant.

Eorum vetustissimus fuit
dierum 6585 1/3, qui sunt anni Arabici 18, syzygiae 7. in genere vero
sunt syzygiae 223.

Quamobrem in secundo libro Plinii perparem legitur
sive culpa ipsius Plinii, sive librarii, defectus luminum ducentis
viginti duobus mensibus redire.

Hipparchus alium \textgreek{[Greek]} longe
maiorem excogitavit dierum 126007, syzygiarum 4267, annorum
Arabicorum 355 cum syzygiis 7: annorum Iulianorum 344 cum
diebus 361.

Quae sunt tolerabiles periodi.

Nam at caussis naturalibus,
nempe at defectionibus luminum proficiscuntur.

quemadmodum
etiam enneadecaeteris Lunaris, et Cyclus Solis: quorum illa Lunam
Soli restituit, hic Solem Septimanae, et praeterea periodus Mexicanorum
constans annis \textsc{lii}, quae restituit \textgreek{[Greek]}, quae ist ipsis
vicem nostrae Hebdomadis.

Neque alia fuit periodus magna Persatum
veterum, quam Salchodai vocabant.

Sunt et aliae, sed civiles, et Indictio;
Aliae inanibus coniecturis insistunt, ut Dodecaeteris Chaldaica
Genethliacorum, item Heracliti, Lini, Orphei, Dionis, et Maorum:
quorum periodus ad modum octavae sphaerae composita est annorum
360000 at conditu Mundi, ut ipsi putant. quorum annorum hic est
centies octagies quater millesimus, sexcentesimus nonagesimus quartus.

Sed longe illa Sinarum prodigiosior, iuxta quam hic annus Christi
1594 est at conditu rerum octigenties octagies quater millesimus,
septingentesimus septuagesimus tertius.

Bonziorum vero Iaponensium
periodus annorum 470 desivit cum anno Christi 1561. et 1562
coepit sequens. eiusque hic est vicesimus currens.

Ea vertente scelera
extirpatum iri: reliquum tempus omnia pacata fore credunt.


p. 20 [pdf 103]

Taceo
diversas Christianorum, Iudaeorum, Samaritanorum de conditu rerum
opiniones: item Romanorum lustrum qunque annorum, saeculum
centum et decem.

Sunt et periodi Computatorum: ut Iudaea
annorum 6916, quae constat cyclis Lunaribus 364, Solaribus 247, periodis
magnis Dionysianis 13.

Habetque tot cyclorum septimanas,
quot dierum septimanae sunt in anno Solari: tot periodos Dionysianas,
quot menses annus embolimaeus: tot cyclos Solares, quot cyclos
Lunares magnus cyclus Iudaicus.

Itaque elegantissima est, et artificiosissima.
eiusque hic agitur annus 5354, anno Christi vulgari 1594.

Et inibit 1595 annus eiusdem proximo autumno, unde omnes epilogismi
neomeniarum Iudaicarum.

Periodus Dionysiana et ipsa ad
annalem computum pertinet, annis constans 532, ducto in sese utroque
cyclo.

Verae quidem periodi magnae caput incurrit in annum
primum utriusque cycli, pertinetque ad methodum Lunae et Solis.

et
locum habet dumtaxat in anno Iuliano, hoc est in eo, cui praeter 365
dies quadrans attibuitur.

Itaque eius initium est at Kal. Ianuariis in
anno Romano: in anno Constantinopolitano at Kal. Septembris. in
Antiocheno at Kal. Octobris. in Alexandrino et Samaritano ab a. d.
\textsc{iiii}. Kal. Septemb.

Periodus vero Dionysii pertinet ad methodum
neomeniae Paschalis, initio sumto ab anno primo natalis Christi, ut
ipse quidem putabat: item ab anno decimo cycli Solis Iuliani, et ab
ea neomenia, cuius quartadecima dies proxime post
 \textsc{xxi}, aut in \textsc{xxii}
Martii conficeretur.

Hactenus at minimis initiis ad summa temporum
incrementa, quam \textgreek{[Greek]} Graeci vocant, Chronologum
perduximus, et eum in conspectu totius antiquitatis collocavimus.

Superest nunc, ut quae carptim et obiter perstrinximus, ea uberius
suis locis expicentur.

Resumamus igitur eos annos, ex quibus tanquam
elementis, ad tot tamque diversa genera annorum progressus
factus est.

Ex anno Graeco, qui est aequabilis minor, omnes anni, Lunaris
formas propagatas esse vidimus: ut ex Aegyptiaco, qui est aequabilis
maior, omnes Solares.

Non igitur confuse, et per saturam haec
tractanda, sed suo quaeque et loco et ordine.

Quatuor igitur libris
quatuor genera anni summa explicare decrevimus.

Primus erit de
anno aequabili minore.

Eo enim omnis Graecia usa tam diversis generibus,
quam multae fuerunt eius terrae nationes, et \textgreek{[Greek]}.

Itaque
ea erit reliqua pars huius libri.

Secundum locum sibi vindicat annus
Lunaris, quia ex illo priore derivatus.

Tertius liber complectetur anni
aequabilis maioris formas, \textgreek{[Greek]}, et differentias.

Quartus illius anni
traduces et propagines persequetur, diversa nempe anni Solaris genera,
et mutationes.

Haec est pars prior, quam initio huius diatribae.

Chronologo promisimus, de annorum et temporum Civilium generibus.


p. 21 [pdf 104]

Altera pars est de charactere, qui necessarius est notandis temporum
intervallis, quae sequentibus libris tractabimus, item diversis
computis nationum annalibus, de quibus librum singularem ad calcem
operis adiiciemus, non tanquam appendicem, sed partem unam
operis nostri.

Quis igitur sit usus characteris temporum, docet nos
Dionysius ex Ephoro, qui cum annum excidii Troiae ex Olympiadum
epocha notare non posset, cum is casus aliquot seculis antiquior
sit prima Olympiade, dixit id accidisse eo anno Attico, quo viginti
\textgreek{[Greek]} annum explebant.

Statim peritis anni Attici subolebat,
quo anno id accidere potuerit.

Sciebant enim quoties in quanto
intervallo annorum id fieri posset.

Exemplo Ephori aut Dionysii
erit nobis character excogitandus, quo animus anceps in trivio constitutus
quaesitum ad fontem manu deducatur.

Erit igitur primum
totius instituti nostri fundamentum annus Iulianus, quem fingimus
ante multa millia annorum fuisse.

Characteres vero illi duos dabimus,
cyclum Lunae Dionysianum, cuius hic est annus \textsc{xviii}.

et cyclum
Solis Iulianum, cuius hodie annus \textsc{vii} currit.

Tertium etiam,
ubi ratio temporum patietur, Indictiones non aspernabimur.

Nam
qui his characteribus semel uti institerint, illi, quae sit constantia, et fides
illius methodi pulcherrimae in ratione temporum, experentur.

Si
quis hoc anno Christi 1594 incertus, quot annos natus sit, tamen et
maiorem se quadraginta novem annorum, et minorem quinquaginta
sex sciat, is imitatur imperitiam Chronologorum Graecorum, qui
circiter illius, et illius regis tempora illud, et illud accidisse dicunt, annum
vero certum non difiniunt.

Sed cum idem adiicit natum se Nonis Augusti, feria quinta, is addit characterem certum et indubitatum, quales sunt viginti \textgreek{[Greek]} Ephori.

Nam feria quinta non
potuit incurrere in Nonas Augusti, nisi cum litera Dominicalis est C.
Ante 49 autem annos id accidit anno Domini 1540, cyclo Solis nono.

Itaque hoc characterismo constantissime affirmanus eo anno hominem
natum, et proximis Nonis Augusti Iulianis illi quinquagesimum
quintum natelem initurum.

Idem usus cycli Lunaris, adhibita
castigatione, ut at prima Olympiade, ad annum Domini 1400, tot
dies neomeniis adhibeas, quoties 304 annos reperies.

Exemplum. hic est annus at prima Olympiade 2370.

In quibus annis septies reperitur
numerus 304. septem igitur dies neomeniis hodiernis adiiciendi.

Verbi gratia. anno primo cycli epactae sunt \textsc{xi}. novilunium
Martii \textsc{xviii}. additis
 \textsc{vii}. diebus, novilunium, vel potius coniunctio
lumanarium erat in \textsc{xxv}.

Martii anno quarto anto primam Olympiadem,
aut quintodecimo post eandem primam Olympiadem, et deinceps
ad 304 annos.

Sed ab hoc saeculo nostro post 150 annos minuendae
erunt neominiae totidem diebus, quoties 304 anni reperientur
post annum Christi 1700. et fortasse citius.


p. 22 [pdf 105]

Sed quia nullam epocham
veterem certiorem Olympiadum capite habemus: illud autem
cum vetustate comparatum novitium esse videtur: inutiles erunt characteres
cyclorum et Indictionis, nisi at quadam remotissima epocha
initium temporum instituamus.

Excogitemus igitur periodum,
quae utrunque cyclum, et Indictionem contineat: quod fiet, si periodum
Dionysii Exigui quindecies multiplicemus: qui fient anni
7980.

Ita periodus illa incipiet ab anno primo tum utriusque cycli,
tum Indictionis: et proinde eiusdem ultimus annus definit in ultimis
utriusque cycli, et Indictionis.

Sed annus Christi, ut vulgo putamus,
3267 desinet in ultimum utriusque cycli, et Indictionis.

Ergo deductis
3267 de 7980 annis, relinquetur epocha anni ante vulgarem
Christi, nempe 4713.

Ita ut 4714 sit primus annus Christi vulgaris cyclo
Solis \textsc{X}, Lunae 2, Indictionis 4, at Kal. Ianuarii: quamuis et Indictio
autumno proxime antecedenti, Cyclus autem Lunae Martio sequenti
caeperit.

Quare annus iste, qui ex errore vulgi putatur 1594, est 6307.
periodi huius, quam Iulianam vocamus, quod ad Iulianam anni formam
accommodata sit.

Ideo 6307 divisis per 28, per 19, per 15 habebimus
huius anni 6307 periodi Iulianae, vel vulgaris Christi 1594, cyclum
Solis septimum at Kal. Ianuarii: Lunae decimumoctavum at
Martio sequente: Indictionis septimum Caesarianae quidem ab ante d.
\textsc{viii} Kal. Octobris antecedentis anni 6306: Pontificiae vero at
Kalendis Ianuarii anni propositi 6307.

Non praedicabo laudes huiusce periodi:
Chronologi et astrologi, qui omnia \textgreek{[Greek]} disputare volunt,
non poterunt eam satis laudare.

Qui igitur eclipses ex Tabul[i]s
Prutenicis putare volent, ex anno periodi Iulianae auferant 2408.

et
cum residuo toto excerperant tempora epochae diluvii.

Exemplum: Eclipsis
Lunaris accidit in Septembri anno Olympiadico 446, qui est annus
periodi Iulianae 4383.

Deductis 2408, remanent 1975.

Excerpo
primum 1900 ex epocha Diluvii: deinde 75, ex filo annorum expansorum.

Postremo menses usque ad Septembrem.

Et reliqua ut ex methodo
Prutenica.

Qui omne dubium ex temporum ratione tollere
volet, uti debet hac periodo, sine qua nihil unquam certi in natione
temporum adferre poterit.
% === End of the part entered as litteral transcript.
