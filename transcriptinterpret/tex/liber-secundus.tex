% !TEX TS-program = xelatex
% !TEX encoding = UTF-8 Unicode
% this template is specifically designed to be typeset with XeLaTeX;
% it will not work with other engines, such as pdfLaTeX

%%% Count out columns for fixed-width source font
% 000000011111111112222222222333333333344444444445555555555666666666677777777778
% 345678901234567890123456789012345678901234567890123456789012345678901234567890

\setheaders{\shorttitle{} Liber II}{\shortauthor{}}
\chapter{Liber Secundus - De anno lunari}
\normalsize

% 61
% {PDF page nr}{source page nr}{line nr}
\plnr{144}{61}{1}Annum Graecum antiquitus Lunarem fuisse,
ut alia temporum et mensium descriptio
in Graecia non fuerit, quam quae Lunae
rationibus congrueret, non solum recentiores
homines scripserunt, sed non paucos
veterum idem in literas retulisse tam
compertum esse puto, quam falso eos sensisse
conuincit ratio tetraeteridum a nobis
libro proximo disputata.
\lnr{9}Praeterea ex
eadem disputatione nostra satis constat naturale anni principium antiquitus
non ab Hecatombaeone, sed a Gamelione, et ex diebus brumlibus
duci solitum.
\lnr{12}Quandiu igitur Athenienses gamelionem et temporibus
auspicandis et rerum actibus principem mensem habuerunt,
tunc semper Comitia magistratibus creandis in calcem Posideonis
reiiciebant, ubi erant \textgreek{ἄναρχοι ἡμέραι δύο[?]},
 extra ordinem mensium tricenariorum
positae, ita ut annus esset dierum non solum 360, propter
menses \textgreek{τριακονθημέρους[?]}, sed et 362,
 propter illas appendices \textgreek{ὑπερβαλλούσας[?]},
quae, quia per illud biduum omnes magistratus annui abdicabantur,
propterea dicebantur \textgreek{α῎ναρχοι ἡμέραι[?]}.
\lnr{19}Praeterea quod in illis
Comitia novorum magistratuum creandorum habebantur, ideo
\textgreek{ἀρχαιρεσίαι[?]}, etiam dicebantur.
\lnr{21}Atque hoc fuit quidem magistratibus
creandis dicatum biduum, donec anni Lunaris formam Astronomi illorum
temporum publicarunt.
\lnr{23}Tunc pro bruma, solstitium: pro Gamelione,
hecatombaeonem vulgus principium anni coepit statuere.
\lnr{24}Et
menses, in quibus singulis Comitia terna agebantur, quas
 \textgreek{κυρίας ἐκκλησίας[?]}
vocabant, pro Tetraeteircis, Lunares: pro solidis, alternis cavi
usurpari coepti.
\lnr{27}Quod ut planius intelligatur, sciendum Athenis
duos summos senatus fuisse, alterum, \textgreek{τῶν ἀρειοπαγιτῶν[?]},
 qui erant iudices
% Greek: also Ἀρεοπαγῑτῶν (pl. gen.)
% A member of the ancient-Athenian conciliary court of the Areopagus.
ut plurimum rerum capitalium et quidem magni momenti: alterum autem
ordinariarum, civilium et bellicarum, et summae denique reipublicae.

% 62
% {PDF page nr}{source page nr}{line nr}
\plnr{145}{62}{2}Sed Areopagitarum consessus perpetuus erat.
\lnr{2}Hic Senatus
quotannis sorte creabatur, olim utique \textgreek{ἐν ὑπερβαλλούσιας ἡμέραις[?]},
postea anno Lunari admisso, in ultimis quatuor diebus anni
Lunaris, hoc est in illis quatuor, qui sunt supra 250.
\lnr{5}Decem
enim Tribus Attica habuit, quales Roma \textsc{xxxv}.
\lnr{6}Ex singuilis Tribubus
quinquaginta magistratus forte creati rebus gerendis admittebantur.
% Bar on quinquaginta?
\lnr{8}Ita ex decem Tribubus quinquageni Senatum Quingentorum
constituebant, qui ab eo \textgreek{οι πεντακόσιοι[?]} decebantur,
 item \textgreek{ἡβουλὴ
τῶν πεντακοσίων[?]}.
\lnr{10}Porro unaquaeque tribus forte unum diem summam
rem gerebat et imperabat.
\lnr{11}Ita cum per 354 dies, quot nimirum
habet annus Lunaris, singuli quinquaginta diem suam per
orbem imperassent, fiebat, ut 35 dies ex toto anno unaquaeque Tribus
rerum potiretur: et, quia decem erant Tribus, sequitur, ut trecentos
quinquaginta dies simul omnes imperarent.
\lnr{15}Reliquae sunt ex anno
Lunari \textgreek{ἄναρχοι ἡμέραι[?]} quatuor.
\lnr{16}Hae igitur quatuor dies vicem illarum
\textgreek{ὑπερβαλλουσῶν[?]} magistratibus creandis reservatae.
\lnr{17}Hoc ita esse, testis Ulpianus
Rhetor, vetus Demosthenis interpres
 \textgreek{εν τω κατα Ανδροτιωνος ἔχειγοῦν[Greek]},
% Demosthenis Orationes ad optimos libros accurate emendatae
inquit,
 \textgreek{υ ενιαιτος κατα τον σεληνιακον δρομον, τριακοσιας πεντηκοι τα τεσσαρας
ημερας[Greek]}.
\lnr{20}\textgreek{και τας μει δ ημερασ εκαλουν οι Αθηναιοι αρχαυρεσιας[Greek]}.
\lnr{20}\textgreek{εν
αις οιυαρχος η Αττικη ην[Greek]}
\lnr{21}\textgreek{εν ταυταις προεβαλλοντο της αρχοντας[Greek]}.
\lnr{21}\textgreek{ηρχον ουν
οι πεντακυσιοι τας τριακοσιας πεντηκοντα ημερας[Greek]}.
\lnr{22}Trecentos igitur et
quinquaginta dies simul imperabant, qui in decem Tribus divisi dant
unicuique dies triginta quinque.
\lnr{24}Nam, exempli gratia, heri \textgreek{ἡ ἀιαντὶς φυλὺ[?]}
imperabat, hodie \textgreek{ἡ κεκροπὶς[?]}, cras \textgreek{ἡ ἀκαμοιυτὶς[?]}.
\lnr{25}Et sic deinceps una quaeque
suam \textgreek{ἐφημερίαν[?]} imperabat, prout sorte[forte?] ducta erat.
\lnr{26}Neque sane quinquaginta
simul imperabant, sed ex singulis Tribubus singuli forte
ducti viri, qui dicebantur \textgreek{πρόεδροι[?]}.
\lnr{28}Illi enim Comitiis habendis praesidebant.
\lnr{29}Nam quingenti simul dicebatur \textgreek{ἡ τῶν πεντακοσίων[?]}, Tribubus
illis decem in unum corpus confusis.
\lnr{30}Quinquaginta autem
per Tribus distincti dicebantur \textgreek{πρυτάνεις[?]}.
\lnr{31}Decem vero vocabantur
\textgreek{πρόεδροι[?]}, qui erant principes quadrainta novem reliquorum, singuli
scilicet in sua Tribu.
\lnr{33}Nam unus \textgreek{πρόεδρος[?]} erat quinquagesimus sui
corporis \textgreek{τῶν πρυτάνεων[?]}: ut in castris Romanis Decurio erat
 decimus
illius decuriae, cuius ipse caput erat.
\lnr{35}Menses igitur Lunares proprii
erant horum et omnium denique magistratuum, et dicebantur
 \textgreek{πρυτανεῖαι[?]}.
\lnr{37}Ut in lege Atheniensium apud Demosthenem, \textgreek{ἐν τῷ κατὰ
 Τιμοκράτοις,
ἐπὶ τὴς πρώτης πρυτανείας τῇ ἑνδεκάτῃ[?]}.
\lnr{38}Hoc est undecima mensis
primi Lunaris, id est, Hecatombaeonis Metonici.
\lnr{39}Dicibatur etiam,
\textgreek{[Greek]}.
\lnr{41}Neque enim est alius Hecatombaeon, quam Lunaris, et eo
die \textgreek{[Greek]} pertinebat ad Tribum \textgreek{[Greek]}.

% 63
% {PDF page nr}{source page nr}{line nr}
\plnr{146}{63}{1}Hoc est, is dies erat \textgreek{[Greek]}.
\lnr{2}Et \textgreek{[Greek]} in omni prytania, sive
mense Lunari, agebantur terstata die mensis, \textgreek{[Greek]},
 \textgreek{[Greek]}, \textgreek{[Greek]}.
\lnr{4}Quare haec anni forma tantum ad magistratus pertinebat.
\lnr{4}Neque
ea unquam populus usus est, a quo menses \textgreek{[Greek]}, et Tetraeterides
extorqueri nunquam potuerunt, ne tunc quidem, cum annus Lunaris
ab Hipparcho emendatior editus esset.
\lnr{7}Primus itaque mensis Hecatombaeon
fuit a solstitio.
\lnr{8}Et quia per undecim dies Hecatombaeon
huius anni antevertebat caput praeteriti, inde fiebat, ut saepe Hecatombaeon
in Scirrhophorionem Tetraeteridis incideret, in quo saepe \textgreek{[Greek]}
et \textgreek{[Greek]} magistratum inibant.
\lnr{11}Demosthenes \textgreek{[Greek]}.
\lnr{11}\textgreek{[Greek]}.
\lnr{12}\textgreek{[Greek]}, et cetera.
\lnr{13}Si Thargelion Tetraeteridis erat ultimus mensis anni
\textgreek{[Greek]}, ergo \textgreek{[Greek]} illud
 \textgreek{[Greek]} incipiebat a Scirrhophorione, in
quem incurrebat mensis Metonicus.
\lnr{15}Et revera Scirrhoporion Metonicus
illius anni, qui erat 62 periodi Atticae, incidebat in 13 Thargelionis
Metonici: Hecatombaeon autem in 12 Scirrhophorionis.
\lnr{17}Idem Orator
\textgreek{[Greek]} ait fuisse in negotio
\textgreek{[Greek]}, ultimam anni, scilices \textgreek{[Greek]}.
\lnr{19}\textgreek{[Greek]}, inquit, \textgreek{[Greek]}, et cetera.
\lnr{21}Quem locum non affequitur interpres.
\lnr{21}Nam orator diserte
indicat, metonicum mensem \textgreek{[Greek]} nunc in Scirrhophorionem
popularem, nunc in Thargelionem incidere, ac consequenter
Hecatombaeonem Metonicum nunc in Schirrhophorionem Tetraeteridis,
nunc in ipsum hecatombaeonem.
\lnr{25}Thucydides: \textgreek{[Greek]}, et cetera.
\lnr{27}Ab initio statim
Veris, hoc est, a Munychione, ait duos adguc tantum superfuisse
menses magistratui Pythodori.
\lnr{29}Definebat igitur ille annus \textgreek{[Greek]}
in Scirrhophorione, et fere conveniebat anno Tetraeterico.
\lnr{30}Porro neomeniae
illae dicebantur \textgreek{[Greek]}, ad differentiam \textgreek{[Greek]}.
\lnr{32}\textgreek{[Greek]}, inquit Thucydides, \textgreek{[Greek]}.
\lnr{33}Alibi:
\textgreek{[Greek]}.
\lnr{35}\textgreek{[Greek]} hic intelliguntur Metonicae, et \textgreek{[Greek]}.
\lnr{36}Cum autem menses Lunares alternis pleni et cavi sint, semper pro
\textgreek{[Greek]} dicebant \textgreek{[Greek]}, et quae in illis mensibus cavis
vocabatur \textgreek{[Greek]}, ea re vera erat \textgreek{[Greek]}.
\lnr{38}Huius rei rationem
reddidimus in Boedromione \textgreek{[Greek]}, item in exemplo
Oeconomicorum Aristotelis supra a nobis producto de Mnemone
Tyranno Lampsaceno.
\lnr{41}In quo diserte ostenditur sex dies de 360 alternis
eximi solitos, item \textgreek{[Greek]} pro
 \textgreek{[Greek]} alternis mensibus
dici solitum fuisse.

% 64
% {PDF page nr}{source page nr}{line nr}
\plnr{147}{64}{2}Quare hic repetenda non sunt.
\lnr{2}Sic Moschopulus
\textgreek{[Greek]} Hesiodi 177.










































% ==== End of text of Liber Secundus ===
