% !TEX TS-program = xelatex
% !TEX encoding = UTF-8 Unicode
% this template is specifically designed to be typeset with XeLaTeX;
% it will not work with other engines, such as pdfLaTeX

%%% Count out columns for fixed-width source font
% 000000011111111112222222222333333333344444444445555555555666666666677777777778
% 345678901234567890123456789012345678901234567890123456789012345678901234567890

\setheaders{\shorttitle{} Liber II}{\shortauthor{}}
\chapter{Liber Secundus - De anno lunari}
\normalsize

% 61
% {PDF page nr}{source page nr}{line nr}
\plnr{144}{61}{1}Annum Graecum antiquitus Lunarem fuisse,
ut alia temporum et mensium descriptio
in Graecia non fuerit, quam quae Lunae
rationibus congrueret, non solum recentiores
homines scripserunt, sed non paucos
veterum idem in literas retulisse tam
compertum esse puto, quam falso eos sensisse
convincit ratio tetraeteridum a nobis
libro proximo disputata.
\lnr{9}Praeterea ex
eadem disputatione nostra satis constat naturale anni principium antiquitus
non ab Hecatombaeone, sed a Gamelione, et ex diebus brumlibus
duci solitum.
\lnr{12}Quandiu igitur Athenienses Gamelionem et temporibus
auspicandis et rerum actibus principem mensem habuerunt,
tunc semper Comitia magistratibus creandis in calcem Posideonis
reiiciebant, ubi erant \textgreek{ἄναρχοι ἡμέραι δύο},
 extra ordinem mensium tricenariorum
positae, ita ut annus esset dierum non solum 360, propter
menses \textgreek{τριακονθημέρους}, sed et 362,
 propter illas appendices \textgreek{ὑπερβαλλούσας[?]},
quae, quia per illud biduum omnes magistratus annui abdicabantur,
propterea dicebantur \textgreek{ἄναρχοι ἡμέραι}.
\lnr{19}Praeterea quod in illis
Comitia novorum magistratuum creandorum habebantur, ideo
\textgreek{ἀρχαιρεσίαι[?]}, etiam dicebantur.
\lnr{21}Atque hoc fuit quidem magistratibus
creandis dicatum biduum, donec anni Lunaris formam Astronomi illorum
temporum publicarunt.
\lnr{23}Tunc pro bruma, solstitium: pro Gamelione,
Hecatombaeonem vulgus principium anni coepit statuere.
\lnr{24}Et
menses, in quibus singulis Comitia terna agebantur, quas
 \textgreek{κυρίας ἐκκλησίας[?]}
vocabant, pro Tetraetericis, Lunares: pro solidis, alternis cavi
usurpari coepti.
\lnr{27}Quod ut planius intelligatur, sciendum Athenis
duos summos senatus fuisse, alterum, \textgreek{τῶν ἀρειοπαγιτῶν[?]},
 qui erant iudices
% Greek: also Ἀρεοπαγῑτῶν (pl. gen.)
% A member of the ancient-Athenian conciliary court of the Areopagus.
ut plurimum rerum capitalium et quidem magni momenti: alterum autem
ordinariarum, civilium et bellicarum, et summae denique reipublicae.

% 62
% {PDF page nr}{source page nr}{line nr}
\plnr{145}{62}{2}Sed Areopagitarum consessus perpetuus erat.
\lnr{2}Hic Senatus
quotannis sorte creabatur, olim utique \textgreek{ἐν ὑπερβαλλούσιας ἡμέραις[?]},
postea anno Lunari admisso, in ultimis quatuor diebus anni
Lunaris, hoc est in illis quatuor, qui sunt supra 350.
\lnr{5}Decem
enim Tribus Attica habuit, quales Roma \textsc{xxxv}.
\lnr{6}Ex singuilis Tribubus
quinquaginta magistratus forte creati rebus gerendis admittebantur.
% Bar on quinquaginta?
\lnr{8}Ita ex decem Tribubus quinquageni Senatum Quingentorum
constituebant, qui ab eo \textgreek{οι πεντακόσιοι[?]} decebantur,
 item \textgreek{ἡβουλὴ
τῶν πεντακοσίων[?]}.
\lnr{10}Porro unaquaeque tribus forte unum diem summam
rem gerebat et imperabat.
\lnr{11}Ita cum per 354 dies, quot nimirum
habet annus Lunaris, singuli quinquaginta diem suam per
orbem imperassent, fiebat, ut 35 dies ex toto anno unaquaeque Tribus
rerum potiretur: et, quia decem erant Tribus, sequitur, ut trecentos
quinquaginta dies simul omnes imperarent.
\lnr{15}Reliquae sunt ex anno
Lunari \textgreek{ἄναρχοι ἡμέραι} quatuor.
\lnr{16}Hae igitur quatuor dies vicem illarum
\textgreek{ὑπερβαλλουσῶν[?]} magistratibus creandis reservatae.
\lnr{17}Hoc ita esse, testis Ulpianus
Rhetor, vetus Demosthenis interpres
 \textgreek{εν τω κατα Ανδροτιωνος ἔχειγοῦν[Greek]},
% Demosthenis Orationes ad optimos libros accurate emendatae
inquit,
 \textgreek{υ ενιαιτος κατα τον σεληνιακον δρομον, τριακοσιας πεντηκοι τα τεσσαρας
ημερας[Greek]}.
\lnr{20}\textgreek{και τας μει δ ημερασ εκαλουν οι Αθηναιοι αρχαυρεσιας[Greek]}.
\lnr{20}\textgreek{εν
αις οιυαρχος η Αττικη ην[Greek]}
\lnr{21}\textgreek{εν ταυταις προεβαλλοντο της αρχοντας[Greek]}.
\lnr{21}\textgreek{ηρχον ουν
οι πεντακυσιοι τας τριακοσιας πεντηκοντα ημερας[Greek]}.
\lnr{22}Trecentos igitur et
quinquaginta dies simul imperabant, qui in decem Tribus divisi dant
unicuique dies triginta quinque.
\lnr{24}Nam, exempli gratia, heri \textgreek{ἡ ἀιαντὶς φυλὺ[?]}
imperabat, hodie \textgreek{ἡ κεκροπὶς[?]}, cras \textgreek{ἡ ἀκαμοιυτὶς[?]}.
\lnr{25}Et sic deinceps una quaeque
suam \textgreek{ἐφημερίαν[?]} imperabat, prout sorte[forte?] ducta erat.
\lnr{26}Neque sane quinquaginta
simul imperabant, sed ex singulis Tribubus singuli forte
ducti viri, qui dicebantur \textgreek{πρόεδροι[?]}.
\lnr{28}Illi enim Comitiis habendis praesidebant.
\lnr{29}Nam quingenti simul dicebatur \textgreek{ἡ τῶν πεντακοσίων[?]}, Tribubus
illis decem in unum corpus confusis.
\lnr{30}Quinquaginta autem
per Tribus distincti dicebantur \textgreek{πρυτάνεις[?]}.
\lnr{31}Decem vero vocabantur
\textgreek{πρόεδροι[?]}, qui erant principes quadraginta novem
 reliquorum, singuli
scilicet in sua Tribu.
\lnr{33}Nam unus \textgreek{πρόεδρος[?]} erat quinquagesimus sui
corporis \textgreek{τῶν πρυτάνεων[?]}: ut in castris Romanis Decurio erat
 decimus
illius decuriae, cuius ipse caput erat.
\lnr{35}Menses igitur Lunares proprii
erant horum et omnium denique magistratuum, et dicebantur
 \textgreek{πρυτανεῖαι[?]}.
\lnr{37}Ut in lege Atheniensium apud Demosthenem, \textgreek{ἐν τῷ κατὰ
 Τιμοκράτοις,
ἐπὶ τὴς πρώτης πρυτανείας τῇ ἑνδεκάτῃ[?]}.
\lnr{38}Hoc est undecima mensis
primi Lunaris, id est, Hecatombaeonis Metonici.
\lnr{39}Dicibatur etiam,
\textgreek{[Greek]}.
\lnr{41}Neque enim est alius Hecatombaeon, quam Lunaris, et eo
die \textgreek{[Greek]} pertinebat ad Tribum \textgreek{[Greek]}.

% 63
% {PDF page nr}{source page nr}{line nr}
\plnr{146}{63}{1}Hoc est, is dies erat \textgreek{[Greek]}.
\lnr{2}Et \textgreek{[Greek]} in omni prytania, sive
mense Lunari, agebantur terstata die mensis, \textgreek{τῇ ἑνδεκάτῃ[?]},
 \textgreek{τῇ εἰκάδι[?]}, \textgreek{τῇ ἔνην καὶ νέαν[?]}.
\lnr{4}Quare haec anni forma tantum ad magistratus pertinebat.
\lnr{4}Neque
ea unquam populus usus est, a quo menses \textgreek{τριακονθήμεροι[?]},
 et Tetraeterides
extorqueri nunquam potuerunt, ne tunc quidem, cum annus Lunaris
ab Hipparcho emendatior editus esset.
\lnr{7}Primus itaque mensis Hecatombaeon
fuit a solstitio.
\lnr{8}Et quia per undecim dies Hecatombaeon
huius anni antevertebat caput praeteriti, inde fiebat, ut saepe Hecatombaeon
in Scirrhophorionem Tetraeteridis incideret, in quo saepe \textgreek{[Greek]}
et \textgreek{[Greek]} magistratum inibant.
\lnr{11}Demosthenes \textgreek{πρὸς Τιμόθεον[?]}.
\lnr{11}\textgreek{και [Greek]}.
\lnr{12}\textgreek{[Greek]}, et cetera.
\lnr{13}Si Thargelion Tetraeteridis erat ultimus mensis anni
\textgreek{πρυτανείας[?]}, ergo \textgreek{ὔστερον[?]} illud
 \textgreek{ἔτος[?]} incipiebat a Scirrhophorione, in
quem incurrebat mensis Metonicus.
\lnr{15}Et revera Scirrhoporion Metonicus
illius anni, qui erat 62 periodi Atticae, incidebat in 13 Thargelionis
Metonici: Hecatombaeon autem in 12 Scirrhophorionis.
\lnr{17}Idem Orator
\textgreek{[Greek]} ait fuisse in negotio
\textgreek{[Greek]}, ultimam anni, scilicet \textgreek{πρυτανείας[?]}.
\lnr{19}\textgreek{Φυλάξας[?]}, inquit, \textgreek{[Greek]}, et cetera.
\lnr{21}Quem locum non affequitur interpres.
\lnr{21}Nam orator diserte
indicat, Metonicum mensem \textgreek{πρυτανείας[?]} nunc in Scirrhophorionem
popularem, nunc in Thargelionem incidere, ac consequenter
Hecatombaeonem Metonicum nunc in Schirrhophorionem Tetraeteridis,
nunc in ipsum hecatombaeonem.
\lnr{25}Thucydides: \textgreek{[Greek]}, et cetera.
\lnr{27}Ab initio statim
Veris, hoc est, a Munychione, ait duos adhuc tantum superfuisse
menses magistratui Pythodori.
\lnr{29}Definebat igitur ille annus \textgreek{πρυτανείας[?]}
in Scirrhophorione, et fere conveniebat anno Tetraeterico.
\lnr{30}Porro neomeniae
illae dicebantur \textgreek{[Greek]}, ad differentiam
 \textgreek{τριακονθημέρων[?]}.
\lnr{32}\textgreek{Του [Greek]}, inquit Thucydides, \textgreek{[Greek]}.
\lnr{33}Alibi:
\textgreek{[Greek]}.
\lnr{35}\textgreek{[Greek]} hic intelliguntur Metonicae, et
 \textgreek{πρυτανεῖαι[?]}.
\lnr{36}Cum autem menses Lunares alternis pleni et cavi sint, semper pro
\textgreek{[Greek]} dicebant \textgreek{[Greek]},
 et quae in illis mensibus cavis
vocabatur \textgreek{[Greek]}, ea re vera erat \textgreek{[Greek]}.
\lnr{38}Huius rei rationem
reddidimus in Boedromione \textgreek{[Greek]}, item in exemplo
Oeconomicorum Aristotelis supra a nobis producto de Mnemone
Tyranno Lampsaceno.
\lnr{41}In quo diserte ostenditur sex dies de 360 alternis
eximi solitos, item \textgreek{[Greek]} pro
 \textgreek{[Greek]} alternis mensibus
dici solitum fuisse.

% 64
% {PDF page nr}{source page nr}{line nr}
\plnr{147}{64}{2}Quare hic repetenda non sunt.
\lnr{2}Sic Moschopulus
\textgreek{εἰς ἡμέρας[?]} Hesiodi 177.
\lnr{3}\textgreek{Αθμιαῖοι[?] τὴν τριακοστὴν φασιν ἐννάτην καὶ ἐικοστήν[?]}.
\lnr{4}Iudaei primum diem mensis cavi dicunt secundum.
\lnr{4}Esto exemplum
de mense Elul.
\lnr{5}Ita scribunt \texthebrew{[Hebrew]}.
\lnr{5}Ita de
omnibus cavis mensibus pronunciant.
\lnr{6}Dies, inquiunt, primus mensis
cavi compensat defectum eius.
\lnr{7}Quia quum pro primo ponimus secundum,
ultimus erit tricesimus, non \textsc{xxix}.
\lnr{8}Et ita omnes menses Iudaeorum
habent \textgreek{τριακάδα[?]}, quemadmodum et Graecorum.
\lnr{9}Diu autem
mensibus Lunaribus usi sunt Graeci, adeo ut Theon Arati interpres
dicat suis temporibus etiamnum aliquot nationes Graecorum eam anni
formam retinuisse.
\lnr{12}\textgreek{[Greek]}.

\subsection{De Octaeteride Cleostrati}

\lnr{14}Primus Solon auctor fuit Atheniensibus
 \textgreek{τὰς ἡμέρασ κατὰ σελήνην ἄγειν},
ut scribit Laertius.
% Diogenes Laertius (3rd century CE) wrote about Solon (c. 638 - c. 558 BCE),
% in "Lives and Opinions of Eminent Philosophers" (Book A, second
% chapter)
% Section 59, near the end: 
% "[ἠξίωσέ τε Ἀθηναίους] τὰς ἡμέρας κατὰ σελήνην ἄγειν." 
% "[He required the Athenians] to adopt a lunar month."
\lnr{15}Quod tamen frustra proposuit, cum
Atheniensibus menses omnes pleni, \textgreek{καὶ τριακονθήμερα} agerentur.
% Greek: and the thirtyday
\lnr{17}Tandem Cleostratus Tenedius Octaeteride Lunari edita, ab Atheniensibus
scripto expressit, quod Solon eloquentia impetrare non potuerat.
\lnr{19}Cum enim iam omnibus persuasum esset, annum Lunarem esse dierum
trecentorum quinquaginta quatuor, Solarem vero Lunari maiorem
esse diebus undecim praecise cum quadrante, Cleostratus animadvertit
octo Lunares annos cum totidem excessibus Solaribus conficere[?] syzygias
% syzygia = conjunction
nonaginta novem, id est dies 2922: quot scilicet diebus octo anni
Solares constant.
% 29 1/2 days per lunar month, 12 lunar months per lunar year = 354 days/l.y.
% 11 1/4 extra days to make a solar year = 365 1/4 days/s.y.
% "8 lunar years plus all the extra days make 99 conjuctions" (lunar months)
% 8 x (354 + 11 1/4) = 2922 days
% 2922 days / 29 1/2 days per lunar month = 99,05085 new moons
\lnr{24}Ergo in octo annis Solaribus totidem syzygias praecise
transigi existimavit: quarum syzygiarum quadraginta octo sint
cavae, reliquae plenae \textgreek{κὰι τριακοντήμερα}.
\lnr{26}Atque hoc intervallo dierum
et syzygiarum putavit \textgreek{τῶν φαινομένων ἀποκατάστασιν[?]} fieri,
 et Lunam
cum Sole, itemque omnium \textgreek{φαινομένων[?]}
 ortus et occasus ad idem punctum
redire, a quo caeperint primum.
\lnr{29}Iste autem Cleostratus (alibi
Leostratum invenio, sed male ut puto) primus Graecorum, et signorum
partes in Zodiaco notavit, et initia Arietis ac Sagittarii, ut ex Plinio
et Hygeno colligimus: idque[?] circa Olympiadem \textsc{lxi}.
% accent on idque: what do we do?
\lnr{32}Instituit
autem caput Octaeteridis a brumalibus diebus necessario, tum quia
eius castigator Harpalus postea idem tempus servavit, tum quia
 \textgreek{ποσειδεῲν δέυτερος[?]}
ostendit Gamelionem hactenus fuisse mensem primum
anni Attici, cum omnis intercalatio debeatur fini anni, Posideon autem
\textgreek{δέυτερος[?]} sit intercalaris.
\lnr{37}Sed Brumam intelligendum est non \textgreek{ἀστρονομικῶς[?]},
sed \textgreek{πολιτικῶς[?]}, quomodo dies civiles agebant Attici.
\lnr{38}Saeculo
enim Iphiti, qui primam Olympiada instauravit, aequinoctium vernum
conficiebatur in \textsc{xxiix} Martii: Solstitium Kalend. Iulii.

% 65
% {PDF page nr}{source page nr}{line nr}
\plnr{148}{65}{1}Sed neomenia
primi mensis Olympici incidit in \textsc{ix} Iulii, hoc est,
 in \textsc{viii}, aut
\textsc{ix} gradum Cancri.
\lnr{3}Neque unquam illam epocham antevertebat.
% Nor was this epoch ever prefered
\lnr{4}Quare primus omnium Cleostratus ostendit Graecis popularibus suis
cardines mundi esse in \textsc{viii} partibus Signorum: idque omnis posteritas
credidit adeo, ut Sosigenes idem Caesari, Caesar posteritati persuaserit.
\lnr{7}Quod si octavus gradus Cancri in \textsc{viii} Iulii, ergo octavus gradus
Capricorni in sexta Ianuarii, cum intervallum sit dierum 182, cum
dimidio.
\lnr{9}Quod si primus annus Olympiadicus non fuisset intercalaris,
neomenia mensis brumalis, qui est Gamelion Atticus, convenisset
in \textsc{vii} Ianuarii, statim post octavum gradum Capricorni.
\lnr{11}Nam a capite
Hecatumbaeonis ad caput Gamelionis, sunt praecise sex menses, et
duae praeterea \textgreek{ἄναρχοι ἡμέραι[?]}:
 qui sunt dies 182, quot scilicet ab \textsc{viii} gradu
Cancri ad \textsc{viii} Capricorni.
% In fact, from the start of Hecatumbaion to the start of Gamelion, there are
% precisely six months, and two extra ἄναρχοι ἡμέραι [anarchic days]:
% this makes 182 days, the number of which is or course from the 8th degree
% of Cancer to the 8th degree of Capricorn.
\lnr{14}Ergo citima neomenia mensis brumalis
est in \textsc{vii} Ianuarii, quos fines nunquam superabit; set embolismi
interventu in Februarium summovebitur.
% Therefore, the nearest new moon in the winter month is on the seventh of
% January, whos ends it will always overflow;
% But embolismic interventions will be withheld until February.
\lnr{16}Omnes igitur menses
alternis sunt pleni et cavi: \textgreek{Ποσειδεὼν[?]}
 etiam \textgreek{δέντερος[?]} plenus.
% All months are therefore alternatively full and short:
% Poseideon [6th month in the Attic calendar] likewise ?? full
%
% Insert table:
% Octaeteris Cleostrati
% -- Placement of the table uncertain. There is no clear indication where it
%    connects to the body text.
% -- There is no column header above "8 Ianua." in the original.
%    In concordance with table
%    038_neomenia_elidensis it might be "Neomenia 1. mensis" or similar.
\begin{table}[htbp]
 \centering
 \renewcommand{\arraystretch}{1.3}
 %%% Liber II p65
%% Wider variation, where the headers are written horizontally
%%
%%% Count out columns for fixed-width source font
% 000000011111111112222222222333333333344444444445555555555666666666677777777778
% 345678901234567890123456789012345678901234567890123456789012345678901234567890
%
%\tiny
%\scriptsize
%\footnotesize
%\small
\normalsize
%% Center the whole table left-right
\centering
%% Modify separation between columns
%\setlength{\tabcolsep}{1.6pt}
%% Modify distance between rows
\renewcommand{\arraystretch}{1.3}
%% Angle to rotate the headers
%\newcommand{\ang}{60}
%%
\begin{tabular}[t]{r c c c c c}
~ & \multicolumn{5}{c}{\Large\textsc{Octaeteris Cleostrati}}\\
\cline{2-6}
~ &
\multicolumn{1}{c}{Anni} &
\multicolumn{1}{c}{Cyclus} &
\multicolumn{1}{c}{Liter} &
~ &
\multicolumn{1}{c}{Dies}
\\
~ &
\multicolumn{1}{c}{octaeteridis} &
\multicolumn{1}{c}{Lunnae} &
\multicolumn{1}{c}{Dominicalis} &
~ &
\multicolumn{1}{c}{collecti}
\\
\cline{2-6}
\scriptsize{†}
  &  1 & 18 &  E &  8 Ianua. &  384 \\
~ &  2 & 19 & DC & 27 Ian.   &  738 \\
\scriptsize{†}
  &  3 &  1 &  B & 15 Ian.   & 1122 \\
~ &  4 &  2 &  A &  3 Febr.  & 1476 \\
~ &  5 &  3 &  G & 23 Ian.   & 1830 \\
\scriptsize{†}
  &  6 &  4 & FE & 12 Ian.   & 2214 \\
~ &  7 &  5 &  D & 30 Ian.   & 2568 \\
~ &  8 &  6 &  C & 19 Ian.   & 2922 \\
\cline{2-6}
\\
~ & \multicolumn{5}{l}{\footnotesize \super{†} \textgreek{Εμβολ.}}\\
\end{tabular}
\caption{Octaeteris Cleostrati}
\label{tab:p065}
%
 \caption{Octaeteris Cleostrati}
 \label{tab:octaeteris_cleostrati}
\end{table}
%
\lnr{17}Scripta est
autem Octaeteris, ut diximus, anno secundo Olympiadis sexagesimae
primae, annis decem post observatam ab Aneximandro obliquitatem
Zodiaci.
% The Octaeteris was however described, as we have said, in the year two
% of the sixty-first Olympiad, ten years after the observation by Aneximander
% of the obliquity of the zodiac.
\lnr{20}Erat cyclus Lunae \textsc{xix},
 Solis \textsc{viii}. anno Iudaico 3227, cuius
Schebat 4.1.76. Ianuarii \textsc{viii}, anno periodi Atticae quinquagesimo
secundo, \textgreek{ποσειδεῶνοσ ἔνῃ καὶ νέᾳ[?]}.
% It was Lunar cycle 19, Solar cycle 8, the Jewish year 3227, being Schebat
% [Jewish month between Tevet and Adar, roughly in Jan-Feb] 4.1.76.
% January 8, the fifty-second year of the Attic period, ποσειδεῶνοσ ἔνῃ καὶ νέᾳ.
% [Poseideon (month 6 in the Attic calendar) Old-and-New (i.e. the 30th day)]
% Leartus about Solon: Πρῶτος δὲ Σόλων τὴν τριακάδα ἔνην καὶ νέαν ἐκάλεσεν.
% "Solon was the first to call the 30th day of the month the Old-and-New day."
% "Lives and opinions..." I.58
\lnr{22}Haec Octaeteris primum fuit initium annorum
et mensium Lunarium \textgreek{πρυτανείας}: quam ipse cum Canone stellarum
Orientium et Occidentium et earum significationibus publicavit.
% And so the first Octaeteris started on the lunar month and year πρυτανείας[??].
% This is when he [Cleostratus] published the Eastern and Western stellar
% Canons and their significance.
\lnr{25}Eum Canonem veteres Graeci
\textgreek{παράπηγμα} vocant.
% This Canon the ancien Greeks called παράπηγμα.
% [Greek: Stall, booth, shed, stand]
\lnr{26}Geminus: \textgreek{Αἱ δὲ
γινόμεναι προῤῥήσεις τῶν ἐπισημασιῶν ἐν τοῖς
παραπήγμασιν οὐκ ἀπό\footnote{No space in De Emendatione} τινων παραγγελμάτων
[ὡρισμένων]\footnote{Word from Geminius missing in De Emendatione} γίνονται}.
% Geminus of Rhodes: Introduction to Phaenomena, ca 1st century BCE
% (Γεμῖνος ὁ Ῥόδιος: Εἰσαγωγὴ εἰς τὰ Φαινόμενα)
% Also simply known as the Isagoge.
% Chapter 16 "Περὶ ἐπισημασιῶν τῶν ἄστρων"
% [About predictions from the stars]
% Second paragraph
% Αἱ δὲ
% γινόμεναι προρρήσεις τῶν ἐπισημασιῶν ἐν τοῖς
% παραπήγμασιν οὐκ ἀπό τινων παραγγελμάτων
% ὡρισμένων γίνονται,
% [οὐδὲ τέχνῃ τινὶ μεθοδεύονται κατηναγκασμένον ἔχουσαι τὸ ἀποτέλεσμα,
% ἀλλ' ἐκ τοῦ ὡς ἐπίπαν γινομένου διὰ τῆς καθ' ἡμέραν παρατηρήσεως τὸ
% σύμφωνον λαμβάνοντες εἰς τὰ παραπήγματα κατεχώρισαν.]
% German translation:
% Die üblichen vorläufigen Angaben der Witterungsanzeichen in den Kalendern
% werden nicht nach bestimmten Regeln gemacht, noch beruhen sie auf einer
% wissenschaftlichen Methode, nach der sie Anspruch auf notwendige Erfülung
% hätten, sondern (die Kalendermacher) haben aus den regelmäsig eintretenden
% Erscheinungen mit Hilfe der täglichen Beobachtung das herausgenommen, was
% ihnen passt, und es in ihre Kalender gesetzt.
% == Carolus Manitius (1898), Gemini Elementa Astronomiae, ad codicum fidem
% recensuit germanica interpretatione et commentariis instruxit.
% From Google Books
\lnr{29}Hic \textgreek{παραπήγματα} vocat \textgreek{τὰς
προῤῥήσεις τῶν ἐπισημασιῶν}, \textit{Tempora quae
messor, quae curvus arator haberet.}
% P. VERGILI MARONIS ECLOGA TERTIA
% Publius Vergillius Maro (Virgil): Ecloga Tertia
% "That they who reap, or stoop behind the plough,
% Might know their several seasons?"
% == Project Gutenberg
\lnr{31}Vitruvius libro nono: \textit{Quorum inventa
secuti, siderum ortus, et occasus, tempestatumque
significatus Eodoxus,
Euctemon, Calippus, Meto, Philippus,
% Accent on Meto
Hipparchus, Aratus, invenerunt,
caeterique ex astrologia, parapegmatorum
disciplinas invenerunt, et
eas posteris explicatas reliquerunt.}
\lnr{39}\textit{Quorum
scientiae sunt hominibus suspiciendae,
quod tanta cura fuerunt, ut etiam videantur divina mente tempestatum
significatus post futuros ante pronunciare.}
% Vitruvius: De Architectura Libri Decem
% Vitruvius: Ten Books on Architecture
% Book 9, chapter VI (Astrology and weather prognostics),
% paragraph 3, 2nd and 3rd sentence
% 3. [When we come to natural philosophy, however, Thales of Miletus,
% Anaxagoras of Clazomenae, Pythagoras of Samos, Xenophanes of Colophon, and
% Democritus of Abdera have in various ways investigated and left us the laws
% and the working of the laws by which nature governs it.]
% In the track of their discoveries, Eudoxus, Euctemon, Callippus, Meto,
% Philippus, Hipparchus, Aratus, and others discovered the risings and settings % of the constellations, as well as weather prognostications from astronomy
% through the study of the calendars, and this study they set forth and left
% to posterity. Their learning deserves the admiration of mankind; for they
% were so solicitous as even to be able to predict, long beforehand, with
% divining mind, the signs of the weather which was to follow in the future.
% [On this subject, therefore, reference must be made to their labours and
% investigations.]

% 66
% {PDF page nr}{source page nr}{line nr}
\plnr{149}{66}{1}Satis clare innuit,
quid, sit \textgreek{παράπηγμα[?]}.
% This clearly indicates that this should be παράπηγμα
\lnr{2}In vulgatis editionibus Vitruvii legitur Eudaemon
Callistus, Melo; pro quo correximus Euctemon, Calippus,
Meto.
% The common edition of Vitruius reads "Eudaemon, Callistus, Melo". We
% corrected this to "Euctemon, Calippus, Meto".

\subsection{De Octaeteride Harpali}

\lnr{5}Octaeteridis Cleostrateae vitium cito deprehensum est,
quod duae Tetraeterides Olympicae cum mense embolimo sint
dierum solidorum 2924, Octaeteris autem Cleostrati dierum
totidem, duobus minus.
\lnr{8}Atqui neomeniae primae Tetraeteridos, et tertiae
ineuntium incidunt in novilunia, ut uberrime a nobis libro priore
disputatum est.
\lnr{10}Quare neomenia Octaeteridis secundae Cleostrateae
incidens in diem penultimum Tetraeteridis secundae, anticipabit novilunium
biduo solido.
\lnr{12}Vitiosa igitur est Octaeteris Cleostrati.
\lnr{12}Cum
igitur intervallum duarum Tetraeteridum inter duo novilunia interiectum
sit, non dubitavit Harpalus, quin illud sit ex iustis syzygiis compositum.
\lnr{15}Omne enim intervallum in idem punctum Lunae desinens,
a quo caeperat, est mere Lunare, hoc est meris mensibus Lunaribus
constans.
\lnr{17}Nam nisi veteres Graeci mensem Lunarem censuissent esse
dierum 29, horarum 12~\myfrac{1}{2}.
\lnr{18}Nunquam spatio dierum 2964[?] iustas syzygias
Lunares fieri posse existimassent.
\lnr{19}Hoc modo annus Lunaris est
dierum 354 horarum 12.
\lnr{20}Qui dies et horae octies multiplicata dant dies
absolutos 2834.
\lnr{21}Qui de 2924 detracti relinquunt 90 dies, qui sunt
menses tres pleni embolimi.
\lnr{22}Quod si dies 2924 per 59 dies dividantur,
habedimus in duabus Tetraeteridibus quadraginta novem paria
mensium alternis plenorum et cavorum, cum diebus praeterea triginta
tribus, hoc est, quinquaginta menses plenos, undequinquaginta
cavos, et tres dies insuper.
\lnr{26}Igitur in Octaeteride, quae constat duabus
Tetraeteridibus Olympicis, sunt syzygiae Lunares nonaginta
novem: quae si essent omnes plenae, fierent omnes dies 2970: de quibus
detractis 2924 diebus duarum Tetraeteridum, remanent 46, differentia
cavorum et plenorum mensium.
\lnr{30}Quadraginta igitur et sex
menses cavi sunt in Octaeteride: et proinde quinquaginta tres erunt
pleni.
\lnr{32}Quae oeconomia mensium immane quantum discrepat a Cleostratea.
\lnr{33}Nam in illa sunt menses alternis pleni, et cavi: in hac tertius
fere mensis est cavus, et aliquando quartus.
\lnr{34}Divisis enim 99 per
46, remanent 2~\myfrac{7}{46} id est duae syzygiae plenae,
 cum \myfrac{7}{46} unius syzygiae.
\lnr{36}Igitur tertius mensis duntaxat est cavus, idque donec ex
 \myfrac{7}{46} consurgat
integra syzygia.
\lnr{37}Tunc enim non iam tertius, sed quartus mensis est cavus,
ut et docet progressus arithmeticus, et potes ex subiecta tabella
animadvertere.

% Table: ΜΗΝΕΣ ΚΟΙΛΟΙ
\begin{table}[htbp]
 \centering
 \footnotesize
 \renewcommand{\arraystretch}{1.3}
 %%% Liber II p67
%%
%%% Count out columns for fixed-width source font
% 000000011111111112222222222333333333344444444445555555555666666666677777777778
% 345678901234567890123456789012345678901234567890123456789012345678901234567890
%
% ΜΗΝΕΣ ΚΟΙΛΟΙ
% "The hollow months" "cavae menses"
%
% This table apparently lists the hollow months that occur in each year
% of the 8 year cycle (octaeteride).
% Each row shows the information for one year in the octaeteride.
% Column 1 indicated if a year is Embolic.
% We have represented the abbreviated word ἐμβολ. by a footnote symbol,
% the same way we did for other tables where it occurs.
% Column 2 uses Byzantine Greek numerals (without a bar above the letters)
% to number the years in the cycle.
% The number 6 is represented by a cursive digamma, rendered here as
% Unicode U03DA.
% The header for the second column is allmost illegible
% ?? ?? ὀκταετερίδος ??
% Columns 3-8 list the 5 or 6 months in the Attic calendar which are hollow.
% Column 9 shows a β. or α. for the ebolic years. Though not explained in the
% table itself, they are probably Byzantine Greek numerals again. The original
% has a bar over the α, but not over the β, indicating they might be numerals.
% Such symbols can be made using the Math features of Tex, e.g.
% $\overbar{\kappa\alpha}$ for '21'.
% Because there is no code for the digamma (representing 6), we need to resort
% to using \varsigma (the end-of-word sigma) for this symbol.
%
%% For testing, uncomment the folowing lines and the lines at the end
%% of the file
%% Test ==>
%\documentclass{book}
%\usepackage{fontspec}
%\setmainfont{Hoefler Text}[]
%%\setmainfont{Times New Roman}[]
%\newfontfamily\greekfont{Times New Roman}
%\usepackage[quiet]{polyglossia}
%\setmainlanguage{latin}
%\setotherlanguage{greek}
%\begin{document}
%% <== Test
%%
\begin{tabular}{ll|lllllll}
\multicolumn{9}{c}{\large{\textgreek{ΜΗΝΕΣ ΚΟΙΛΟΙ}}}
\\
~ & \multicolumn{5}{l}{\textgreek{τ.. τ.. ὀκταετερίδος .τ.}}
\\
\hline
\scriptsize{†} &
\textgreek{α} &
\textgreek{ἐλαφηβολ.} &
\textgreek{θαργηλ.} &
\textgreek{ἑκατομβ.} &
\textgreek{βοηδρομ.} &
\textgreek{μαιμακτ.} &
\textgreek{ποσειδ.} &
\textgreek{β.}
\\
 &
\textgreek{β} &
\textgreek{ἐλαφηβολ.} &
\textgreek{θαργηλ.} &
\textgreek{ἑκατομβ.} &
\textgreek{βονδρομ.} &
\textgreek{μαιμακτ.} &
 &

\\
\scriptsize{†} &
\textgreek{γ} &
\textgreek{γαμηλ.} &
\textgreek{ἐλαφηβολ.} &
\textgreek{σκιῤῥοφ.} &
\textgreek{μεταγείτν.} &
\textgreek{πυανεψ.} &
\textgreek{ποσειδ.} &
\textgreek{α.}
\\
\hline
 &
\textgreek{δ} &
\textgreek{γαμηλ.} &
\textgreek{ἐλαφηβολ.} &
\textgreek{σκιῤῥοφ.} &
\textgreek{βονδρομ.} &
\textgreek{μαιμακτ.} &
 &

\\
 &
\textgreek{ε} &
\textgreek{ανθεστηρ.} &
\textgreek{μουνιχ.} &
\textgreek{σκιῤῥοφ.} &
\textgreek{βονδρομ.} &
\textgreek{μαιμακτ.} &
 &

\\
\scriptsize{†} &
\textgreek{Ϛ} &
\textgreek{γαμηλ.} &
\textgreek{ἐλαφηβολ.} &
\textgreek{θαρυηλ.} &
\textgreek{ἑκατομβ.} &
\textgreek{βονδρομ.} &
\textgreek{ποσειδ.} &
\textgreek{α.}
\\
\hline
 &
\textgreek{ζ} &
\textgreek{γαμηλ.} &
\textgreek{ἐλαφηβολ.} &
\textgreek{θαρυηλ.} &
\textgreek{ἑκατομβ.} &
\textgreek{βονδρομ.} &
\textgreek{ποσειδ.} &

\\
 &
\textgreek{η} &
\textgreek{ανθεστηρ.} &
\textgreek{μουνιχ.} &
\textgreek{σκιῤῥοφ.} &
\textgreek{μεταγείτν.} &
\textgreek{πυανεψ.} &
\textgreek{ποσειδ.} &

\\
\hline
\multicolumn{5}{l}{\footnotesize \super{†} \textgreek{ἐμβολ.}}\\
\end{tabular}
%% Test ==>
%\end{document}

 \caption{\textgreek{Μενες κοιλοι}}
 \label{tab:menes_koiloi}
\end{table}

% Table: ΝΕΟΜΗΝΙΑΙ ΤΗΣ ΟΚΤΑΕΤΗΡΙΔΟΣ
\begin{table}[htbp]
 \centering
 \scriptsize
 %% Modify distance between rows
 \renewcommand{\arraystretch}{1.8}
 %% Modify separation between columns
 \setlength{\tabcolsep}{2.0pt}
 %%% Liber II p67, PDF 150
%%
%%% Count out columns for fixed-width source font
% 000000011111111112222222222333333333344444444445555555555666666666677777777778
% 345678901234567890123456789012345678901234567890123456789012345678901234567890
%
%% Select a general font size (uncomment one from the list)
%\tiny
%\scriptsize
%\footnotesize
%\small
%\normalsize
%% Center the whole table left-right
\centering
%% Modify distance between rows
%\renewcommand{\arraystretch}{1.8}
%% Modify separation between columns
%\setlength{\tabcolsep}{2.0pt}
%%
\begin{tabular}{l llllllll}
\multicolumn{9}{ c }{\large\textgreek{ΝΕΟΜΗΝΙΑΙ ΤΗΣ ΟΚΤΑΕΤΗΡΙΔΟΣ}}\\
\multicolumn{9}{ c }{\normalsize\textgreek{ΚΑΘ' ΕΚΑΣΤΟΝ ΕΤΟΣ}}\\
%\hline
\textgreek{Μηνες} &
\textgreek{ἔτος} &
\textgreek{ἔτος} $\overline{\beta}$ &
\textgreek{ἔτος} $\overline{\gamma}$ &
\textgreek{ἔτος} $\overline{\delta}$ &
\textgreek{ἔτος} $\overline{\epsilon}$ &
\textgreek{ἔτος} $\overline{\varsigma}$ &
\textgreek{ἔτος} $\overline{\zeta}$ &
\textgreek{ἔτος} $\overline{\eta}$
\\
\textgreek{κατὰ σελήνην.} &
~\textgreek{πρῶτον.}
\\
\hline
\textgreek{γαμηλιών.} &
$\overline{\alpha}$          \textgreek{γαμηλ.} &
$\overline{\kappa\epsilon}$  \textgreek{ποσειδ.} &
$\overline{\iota\eta}$       \textgreek{ποσειδ.} &
$\overline{\eta}$            \textgreek{γαμηλ.} &
$\overline{\alpha}$          \textgreek{γαμηλ.} &
$\overline{\kappa\varsigma}$ \textgreek{ποσειδ.} &
$\overline{\iota\varsigma}$  \textgreek{γαμηλ.} &
$\overline{\eta}$            \textgreek{γαμηλ.}
\\
\textgreek{ανθεστηριών.} &
$\overline{\alpha}$          \textgreek{ἀνθεστ.} &
$\overline{\kappa\gamma}$    \textgreek{γαμηλ.} &
$\overline{\iota\epsilon}$   \textgreek{γαμηλ.} &
$\overline{\eta}$            \textgreek{ἀνθεστ.} &
$\overline{\alpha}$          \textgreek{ἀνθεστ.} &
$\overline{\kappa\gamma}$    \textgreek{γαμηλ.} &
$\overline{\iota\epsilon}$   \textgreek{ἀνθεστ.} &
$\overline{\eta}$            \textgreek{ἀνθεστ.}
\\
\textgreek{ἐλαφηβολιών.} &
$\overline{\alpha}$          \textgreek{ἐλαφη.} &
$\overline{\kappa\gamma}$    \textgreek{ἀνθεστ.} &
$\overline{\iota\epsilon}$   \textgreek{ἀνθεστ.} &
$\overline{\zeta}$           \textgreek{ἐλαφη.} &
$\overline{\lambda}$         \textgreek{ἀνθεστ.} &
$\overline{\kappa\gamma}$    \textgreek{ἀνθεστ.} &
$\overline{\iota\epsilon}$   \textgreek{ἐλαφη.} &
$\overline{\zeta}$           \textgreek{ἐλαφη.}
\\
\hline
\textgreek{μυονυχιών.} &
$\overline{\lambda}$         \textgreek{ἐλαφη.} &
$\overline{\kappa\beta}$     \textgreek{ἐλαφη.} &
$\overline{\iota\delta}$     \textgreek{ἐλαφη.} &
$\overline{\zeta}$           \textgreek{μυονυχ.} &
$\overline{\lambda}$         \textgreek{ἐλαφη.} &
$\overline{\kappa\beta}$     \textgreek{ἐλαφη.} &
$\overline{\iota\delta}$     \textgreek{μυονυχ.} &
$\overline{\zeta}$           \textgreek{μυονυχ.}
\\
\textgreek{θαργηλιών.} &
$\overline{\lambda}$         \textgreek{μυονυχ.} &
$\overline{\kappa\beta}$     \textgreek{μυονυχ.} &
$\overline{\iota\delta}$     \textgreek{μυονυχ.} &
$\overline{\varsigma}$       \textgreek{θαργη.} &
$\overline{\kappa\theta}$    \textgreek{μυονυχ.} &
$\overline{\kappa\beta}$     \textgreek{μυονυχ.} &
$\overline{\iota\delta}$     \textgreek{θαργη.} &
$\overline{\varsigma}$       \textgreek{θαργη.}
\\
\textgreek{σκιῤῥοφοριών.} &
$\overline{\kappa\theta}$    \textgreek{θαργη.} &
$\overline{\kappa\alpha}$    \textgreek{θαργη.} &
$\overline{\iota\delta}$     \textgreek{θαργη.} &
$\overline{\varsigma}$       \textgreek{σκιῤῥ.} &
$\overline{\kappa\delta}$    \textgreek{θαργη.} &
$\overline{\kappa\alpha}$    \textgreek{θαργη.} &
$\overline{\iota\gamma}$     \textgreek{σκιῤῥ.} &
$\overline{\varsigma}$       \textgreek{σκιῤῥ.}
\\
\hline
\textgreek{ἑκατομβαιών.} &
$\overline{\kappa\theta}$    \textgreek{σκιῤῥ.} &
$\overline{\kappa\alpha}$    \textgreek{σκιῤῥ.} &
$\overline{\iota\gamma}$     \textgreek{σκιῤῥ.} &
$\overline{\varsigma}$       \textgreek{ἑκατο.} &
$\overline{\kappa\theta}$    \textgreek{σκιῤῥ.} &
$\overline{\kappa\alpha}$    \textgreek{σκιῤῥ.} &
$\overline{\iota\gamma}$     \textgreek{ἑκατο.} &
$\overline{\epsilon}$        \textgreek{ἑκατο.}
\\
\textgreek{μεταγειτνιών.} &
$\overline{\kappa\eta}$      \textgreek{ἑκατο.} &
$\overline{\kappa}$          \textgreek{ἑκατο.} &
$\overline{\iota\gamma}$     \textgreek{ἑκατο.} &
$\overline{\epsilon}$        \textgreek{μεταγ.} &
$\overline{\kappa\eta}$      \textgreek{ἑκατο.} &
$\overline{\kappa}$          \textgreek{ἑκατο.} &
$\overline{\iota\beta}$      \textgreek{μεταγ.} &
$\overline{\epsilon}$        \textgreek{μεταγ.}
\\
\textgreek{βοηδρομιών.} &
$\overline{\kappa\eta}$      \textgreek{μεταγ.} &
$\overline{\kappa}$          \textgreek{μεταγ.} &
$\overline{\iota\beta}$      \textgreek{μεταγ.} &
$\overline{\epsilon}$        \textgreek{βοηδρ.} &
$\overline{\kappa\eta}$      \textgreek{μεταγ.} &
$\overline{\kappa}$          \textgreek{μεταγ.} &
$\overline{\iota\beta}$      \textgreek{βοηδρ.} &
$\overline{\delta}$          \textgreek{βοηδρ.}
\\
\hline
\textgreek{πυανεψιών.} &
$\overline{\kappa\zeta}$     \textgreek{βοηδρ.} &
$\overline{\iota\theta}$     \textgreek{βοηδρ.} &
$\overline{\iota\beta}$      \textgreek{βοηδρ.} &
$\overline{\epsilon}$        \textgreek{πυαν.} &
$\overline{\kappa\zeta}$     \textgreek{βοηδρ.} &
$\overline{\kappa}$          \textgreek{βοηδρ.} &
$\overline{\iota\alpha}$     \textgreek{πυαν.} &
$\overline{\delta}$          \textgreek{πυαν.}
\\
\textgreek{μαιμακτηριών.} &
$\overline{\kappa\zeta}$     \textgreek{πυαν.} &
$\overline{\iota\theta}$     \textgreek{πυαν.} &
$\overline{\iota\alpha}$     \textgreek{πυαν.} &
$\overline{\delta}$          \textgreek{μαιμα.} &
$\overline{\kappa\zeta}$     \textgreek{πυαν.} &
$\overline{\iota\theta}$     \textgreek{πυαν.} &
$\overline{\iota\alpha}$     \textgreek{μαιμα.} &
$\overline{\gamma}$          \textgreek{μαιμα.}
\\
\textgreek{ποσειδεών.} $\overline{\alpha}$&
$\overline{\kappa\varsigma}$ \textgreek{μαιμα.} &
$\overline{\iota\eta}$       \textgreek{μαιμα.} &
$\overline{\iota\alpha}$     \textgreek{μαιμα.} &
$\overline{\gamma}$          \textgreek{ποσειδ.} &
$\overline{\kappa\varsigma}$ \textgreek{μαιμα.} &
$\overline{\iota\theta}$     \textgreek{μαιμα.} &
$\overline{\iota\alpha}$     \textgreek{ποσειδ.} &
$\overline{\gamma}$          \textgreek{ποσειδ.}
\\
\textgreek{ποσειδεών.} $\overline{\beta}$&
$\overline{\kappa\varsigma}$ \textgreek{ποσειδ.} $\overline{\alpha}$ &
    \multicolumn{1}{c}{$\circ$} &
$\overline{\iota}$           \textgreek{ποσειδ.} &
    \multicolumn{1}{c}{$\circ$} &
    \multicolumn{1}{c}{$\circ$} &
$\overline{\iota\eta}$       \textgreek{ποσειδ.} &
    \multicolumn{1}{c}{$\circ$} &
~
\\
\end{tabular}
%
\caption{Neomeniai tes oktaeteridos kath ekaston etos}
 \caption{\textgreek{Νεομηνιαι της Οκταετηριδος}}
 \label{tab:neomeniai_tes_oktaeteridos}
\end{table}

% 67
% {PDF page nr}{source page nr}{line nr}
\plnr{150}{67}{1}Subiecims praeter ea Tabulam neomeniarum Lunarium secundum
menses anni politici aequabilis, id est secundum menses Tetraetericos.
\lnr{3}Haec enim est vera Harpaleae Octaeteridis \textgreek{ψηφοφορία[?]}:
% Greek: vote (politics)
 quippe
quae nigil aliud est, quam mensium Lunarium cum aequabilibus
comparatio.
\lnr{5}Ideo neomeniam Lunaris Gamelionis cum neomenia
Gamelionis aequabilis in primo anno composuimus: non quod
ita fecerit Harpalus: (Nullus enim Gamelion aequabilis eo seculo
Lunaris fuit:) sed quia deprehenso anno primo Octaeteridis Harpali,
non operosum erit divinare quotae diei Gemelionis aequabilis
competat neomenia Gamelionis Lunaris.
\lnr{10}Nam si, verbi gratia,
neomenia primi Gamelionis Lunaris Harpelei incidit in tertiam
diem Gamelionis aequabilis, reliquae omnes neomeniae eundem progressum
servabunt: puta per triduum omnes neomeniae promovendae
erunt.

% 68
% {PDF page nr}{source page nr}{line nr}
\plnr{151}{68}{2}Sed non magis scimus epocham capitis illius Octaeteridis,
quam patriam ipsius Harpali.
\lnr{3}Constat tamen initium fecisse a bruma,
ut est testis Festus Avienus in Arateis:
\begin{quote}
\emph{Non ego nunc longo redeuntia sidera motu\\
In priscas memorem sedes. Habet ista priorum\\
Pagina, et incerta rerum ratione ferentur.\\
Nam quae solem hiberna novem putat aethere volui,\\
Ut Lunae spatium redeat, vetus Harpalus, ipsa\\
Ocius in sedes, momentaque prisca reducit.}
\end{quote}
% Avieni - Aratea, lines 1363-1368
% "Postumius Rufius Festus who is also Avienius"
% He made a somewhat inexact translation of Aratus' didactic poem "Phaenomena",
% the first part of which was in turn a verse setting of Phaenomena by Eudoxus
% of Cnidus.
\lnr{11}Ex quibus cognoscimus et a bruma incepisse,
 et \textgreek{ἐννεαετηρίδα[?]} quoque
vocasse, non, quod annis novem solidis constaret, ut hallucinatur
Festus, sed qua nono quoque anno in orbem rediret: quemadmodum
pentaeteris et trieteris dictae sunt, non a numero annorum, quibus
constabant, sed eorum, quibus ineuntibus \textgreek{ἀποκατάστασιος[?]} fiebat:
quemadmodum quartanam febrem dicimus, non quod intervalla habeat
quaternum dierum, sed quod quarto quoque die in orbem redeat.
\lnr{18}Igitur Harpali Octaeteris admissa fuit cum parapegmate
 et significationibus
stellarum, et haec est periodus secunda \textgreek{πρυτανείας[?]},
 qua Athenienses
usi sunt.
\lnr{20}Plinius quoque significationes siderum et tempestatum
\textgreek{προῤῥήσεις[?]} ex Harpalo citat in indice libri \textsc{xviii}:
 quae non aliunde
petitae, quam ex eius parapegmate.
\lnr{22}Cum autem haec Octaeteris sit
dierum 2924: ipsi dies per octo divisi dant quantitatem anni secundum
Harpalum, dierum 365~\myfrac{1}{2} sive horarum aequinoctialium 12.
\lnr{25}Quare manifestum mendum est in Censorino, ubi differitur de anni
Solaris quantitate secundum Harpalum.
\lnr{26}Nam ibi legimus Harpalum
definisse annum Solis dierum tercentorum sexaginta quinque,
et tredicem horarum praeterea aequinoctialium.
\lnr{28}Omnino enim legendum duodecim horarum.
\lnr{29}Et ne quis dubitet, idem error est infra
apud eundem, ubi dicit Arminon regem Aegypti annum ad tredecim
menses et quinque dies perduxisse.
\lnr{31}Annus Aegyptius non est tredecim,
sed duodecim mensium.
\lnr{32}Alioquin Octaeteridi Harpaleae superessent
horae octo supra 2924.
\lnr{33}Atqui ita nulla fieret aequatio.
\lnr{33}Siquidem
omnis aequationis lex iubet nihil reliquum fieri de ratione horaria,
aut scrupularia.
\lnr{35}Imo tantum abest, ut illarum octo horarum accessione
rationes Solis cum Lunaribus parient, ut etiam fine illis iusto
longior sit Octaeteris, ut alibi demonstratur.
\lnr{37}Hanc Octaeterida exclusit
Enneadecaeteris Euctemonis, et Metonis: De qua proxime dicendum
erat, nisi Eudoxi Octaeteris nos revocaret, quae Metonis quidem
periodo posterior est; propter cognationem autem materiae in
continenti reliquis Octaeteridibus subiiciendam esse arbitrati sumus.

% 69
% {PDF page nr}{source page nr}{line nr}
\plnr{152}{69}{2}De ea igitur prius dicendum.

\subsection{De Octaeteride Eudoxi}
\lnr{3}Eudoxus Cnidius, vir suo saeculo eruditissimus, et Mathematicorum
princeps, in Aegyptum profectus, ibi annum et menses
praeterea integros quatuor, facerdotibus et astrologis operam dedit,
et Octaeterida suam conscripsit, ut docet nos Laertius: \textgreek{[Greek]}.
\lnr{8}Is igitur cum in Aegypto
esset anno tertio Olympiadis \textsc{ciii}, et eclipsium intervalla, quas in
monimentis suis notatas habebant Aegyptii, inter se compararet, deprehendit
syzygiam secundum Harpalum propius abesse a vero, quam
syzygiam secundum Octaeterida Cleostrateam.
\lnr{12}Nam syzygia secundum
Harpalum est dierum 29, horarum aequinoctialium 12~\myfrac{28}{33}.
\lnr{13}At
secundum Cleostratum todidem quidem dierum et horarum, sed et \myfrac{18}{33}
duntaxat unius horae.
\lnr{15}Ratio igitur differentiae Harpaleae ad rationem
differentiae Cleostrateae est dupla quadripertiens tricesimas tertias.
\lnr{16}Quare cum maiuscula sit Harpalea, quam vera sit syzygia;
 putavit Eudoxus
\myfrac{4}{33} unius horae esse excessum Harpaleae sizygiae
 supra veram syzygiam
Lunarem, quam definivit 29 dierum, 12 horarum, \myfrac{24}{33}, vel, quod
idem est, \myfrac{8}{11} unius horae.
\lnr{20}Ideoque ratio differentiae verae syzygiae, ad rationem
Cleostrateae differentiae, est dupla.
\lnr{21}Cum igitur in Octaeteride
sint syzygiae nonaginta novem, si in 29 dies 12~\myfrac{8}{11}
 horae ducantur, habebis
modum unius verae Octaeteridis dies 2923, horas 12, qui est dimidius
dies.
\lnr{24}Ideoque in duabus Octaeteridibus tres dies supererunt supra
rationes Solis: et consequenter in viginti Octaeteridibus, quae sunt
\textgreek{ἑκκαιδεκαετηρίδες[?]} decem, superabunt dies triginta,
 qui est mensis integer.
\lnr{27}Quare Eudoxi periodus magna constat ex Octaeteridibus viginti,
vel Heckaedecaeteridibus decem, quarum ultima deminuenda sit
diebus triginta.
\lnr{29}Et proinde tota periodus Eudoxi est dierum 58440.
\lnr{30}Qui sunt praecise anni Iuliani 160.
\lnr{30}In annis igitur 160 melius putavit
mensem omitti posse, quam diem in \textsc{xix} annis Metonicis.
\lnr{31}Quod tamen
ineptum est.
\lnr{32}Et tam inconsulte ausus est introducere octaeterida
post enneadecaeterida Metonicam, quam iuste enneadecaeteris mendosa
Metonis omni Octaeteridi praelata est.
\lnr{34}Cum igitur scripserit
anno \textsc{iii} Olympiadis \textsc{ciii}, cyclo Lunae decimo sexto,
 Solis octavo, instituerit
vero initium Octaeteridis a solenni Aegyptiaco, quod vocabant
\textgreek{Ισια[?]}, vel \textgreek{Ισίεια[?]}, quae tunc incidebant
 in tempus sideris brumae
confectae, ut odoramur ex Gemino, non difficile erit divinare, quis dies
Iulianus illi tempori competat, si consideremus, cui diei mensis
 \textgreek{ἀιγῶνος[?]}
Eudoxus in parapegmate suo assignavit diem brumae.
% Ισια: Straight, right

% 70
% {PDF page nr}{source page nr}{line nr}
\plnr{153}{70}{2}Nam cum
Meto et Euctemon incipiant annum suum caelestem a \textsc{xxvii} Iunii,
Brumae autem diem centesimum octagesimum secundum numerent
a Solstitio, Eudoxus quartum diem a bruma Euctemonis dixit esse verum
diem brumae, ut notatum est in Parapegmate Attico, cui congruit
dies 28 Decembris.
\lnr{7}Si igitur \textgreek{Ισια[?]} tunc cadebant in brumam, ea necesse
est incidisse in 28 Decembris, quamuis verus dies brumae incidisset
in 27.
\lnr{9}Erat annus Nabonassari 383.
\lnr{9}Neomenia Toth 23 Novembris,
feria prima.
\lnr{10}Neomenia Paophi 23 Decembris.
\lnr{10}Ergo \textgreek{Ισια[?]} in \textsc{vi}
Paophi.
\lnr{11}Sed duo adversantur.
\lnr{11}Quorum alterum est, quod non in mense
Paophi videntur fuisse \textgreek{Ισια[?]}, sed in mense Athyr: qui, posteaquam
fixus fuit, incidit in Novembrem Iulianum.
\lnr{13}Atqui in Novembrem
Iulianum conferuntur Ifia a veteri poeta in descriptione mensium, quam
reperies in Catelectis nostris.
\lnr{15}Ibi enim de Novembri ita scriptum est:
\begin{quote}
\emph{Carbaseo post hunc artus inductus amictu}\\
\emph{\hspace*{1em}Memphios antiqua sacra, Deamque colit.}\\
\emph{A quo vix avidus sistro compescitur anser,}\\
\emph{\hspace*{1em}Devotusque sacris incola Memphidiis.}
\end{quote}

\lnr{20}Manifesto Ifia sunt in Athyr, non in Paophi.
\lnr{20}Alterum est, quod
Geminus reprehendens sui saeculi errorem, qui semper assignabat brumam
Isiis in anno vago Nabonassari, ait ante 120 annos brumam incidisse
in Isia.
\lnr{23}Ergo a tempore Eudoxi ad tempus illud, quo  suum librum
scribebat Geminus, intersunt anni tantum 120.
\lnr{24}Proinde ille
annus a Nabonassaro fuerit 503.
\lnr{25}Ita Geminus fuerit longe antiquior
Hipparcho.
\lnr{26}Quod non puto.
\lnr{26}Longe enim posterior videtur.
\lnr{26}Ad priorem
quidem dubitationem respondere possumus, solenne Memphiticum
Novembris non esse \textgreek{Ισια[?]}, sed \textgreek{εὕρεσιν Οσίριδος[?]}.
\lnr{28}Nam \textgreek{ἀφδυισμός Οσίριδος[?]}
celebrabatur \textsc{xvii} Athyr, teste Plutarcho, hoc est
\textsc{xiii} Novembris: item \textgreek{εὕρεσις Οσίριδος[?]}
 eodem mense celebrabatur,
ut constat ex Kalendario rustico, quod extat Romae, in quo
in mense Novembri cui Athyr respondet, festum \textsc{euresis} notatum
est.
\lnr{33}\textgreek{Τῆς εὑρέσεως[?]} Rutilius Numatianus
 meminit in suo Iternerario:
\begin{quote}
\emph{Et'tum forte hilares per compita rustica pagi}\\
\emph{\hspace*{1em}Mulcebant sacris pectora fessa iocis}\\
\emph{Illo quippe die tandem renovatus Osiris}\\
\emph{\hspace*{1em}Excitat in fruges semina laeta novas.}
\end{quote}
% Claudius Ritillius Namatianus
% "De Reditu Suo, sive De Reditu in Patriam, sive Iter" ca 415 C.E.
% Liber Primus, Line 374-376
% "Et tum forte hilares per compita rustica pagi
%   mulcebant sacris pectora fessa iocis.
% Illo quippe die tandem revocatus Osiris
%   excitat in fruges germina laeta novas."
% Note: renovatus <-> revocatus, semina <-> germina
\lnr{38}Paulo ante indicaverat se soluisse post ortum Pleiados.
\lnr{38}Aperte notat
illam \textgreek{εὕρέσιν[?]} celebrari mense Novembri.
\lnr{39}Et praeterea erant alia
solennia Isidis: \textsc{ut isidis navigium} mense Martio apud idem Kalendarium
et Lactant.
\lnr{41}Et Apul. lib. \textsc{xi} item \textsc{sacrum phariae,
et sarapia} mense Aprili.

% 71
% {PDF page nr}{source page nr}{line nr}
\plnr{154}{71}{1}Ad alteram autem dubitationem nihil
quod respondeamus, habemus.
\lnr{2}Sed utcunque haec fuerint, Eudoxus
sumsit initium Octaeteridis suae a proximo novilunio post 28 Decembris,
hoc est a Scebat Iudaici anni 3396, cuius character 3.23.726, feria
quarta, Decembris \textsc{xxxi}, cum iam tamen neomenia Posideonis embolimi
Metonici moraretur priscam epocham die uno, et tunc esset in
Kal. Ianuarii.
\lnr{7}Igitur inde Octaeterida suam incepit,
 \textgreek{Ποσειδεῶνος δευτέρου ἔνῃ καὶ νέα[?]}:
% Old-and-new of the second Poseidonis
cuius caput post Iulianos 161 praecipitatur usque in
30 Ianuarii.
\lnr{9}Sed \textgreek{ἐξαιρέσει[?]} triginta dierum iterum in ultimam Decembris
reditur: et ita rationes Solis cum Lunaribus aequantur.

\subsection{Elenchus Octaereridis}

\lnr{11}Ne hoc quidem modo constant rationes Lunares.
\lnr{11}Proxime
quidem abest a vero \textgreek{ἑκκαιδεκαετηρὶς[?]} Eudoxi.
\lnr{12}Nam vera \textgreek{ἑκκαιδεκαετηρὶς[?]}
Iudaica est dierum 5847, ut et Eudoxea, et praeterea
horae 1.414, quo excessu superat Eudoxeam.
\lnr{14}Sed syzygia non est praecise
accepta dierum 29, horarum 12, \myfrac{8}{11}: vel, quod idem est,
 dierum 29~\myfrac{1}{2}
et \myfrac{2}{33}.
% 29.5 + 2/33 days = 29 days, 12 hours + 2/33*24 hours
% 2/33*24 = (2/33)*(3*8) = (2*8)/11 = 16/11, which is *not* equal to 8/11
\lnr{16}(Nam tam dies 29 hor. 12~\myfrac{8}{11}, quam dies 29~\myfrac{1}{2}
 \myfrac{2}{3}
 Octaeteridis syzygiam
consummant. Error autem est in Gemino
\myfrac{\textgreek{α}}{\textgreek{λγ}} pro
 \myfrac{\textgreek{β}}{\textgreek{λγ}},
% Alternatively use math mode:
%  $\alpha\over\lambda\gamma$ pro $\beta\over\lambda\gamma$
  ut hoc obiter
moneam.)
% 1/33 pro 2/33
\lnr{18}Syzygia enim praecise est dierum 29, hor. 12~\myfrac{793}{1080}.
% 29d 12 793/1080h = 29d 12.734259h = 29.530d,
% which matches current known value
\lnr{18}Quare
vera \textgreek{ὑπεροχὴ[?]} Solaris anni supra Lunarem erit
 dierum 10, horarum 21.204.
\lnr{20}Quae omnia si octies multiplicentur, fient dies 87, horae 2.472.
\lnr{21}Qui quidem dies non consummant menses tres Lunares.
\lnr{21}Ideo in octo
annis non possunt intercalari menses tres: quod ex enneadecaeteride
probatur.
\lnr{23}Nam in octo enneadecaeteridibus fiunt intercalationes
quinquaginta sex: in \textsc{xix} autem Octaeteridibus fiunt quinquaginta
septem.
\lnr{25}Unus igitur mensis abundat post annos 152; non autem, ut
voluit Eudoxus, post annos 160.
\lnr{26}Et cur illi displicuit Enneadecaeteris
Metonis, non possum conminisci: nisi quia semper sincerum volumus
vas incrustare, et potius reprehendere, quam discere.
\lnr{28}Octaeterida autem
Eudoxi Dositheus Archimedis familiaris correxit, et iterum cum
parapegmate castigatiore edidit.
\lnr{30}Unde quoties veteres \textgreek{ἐν ἐπισημασίαις τῶν φαινομένων[?]}
Dositheum testem producunt, scito esse ex Eudoxi
quidem Octaeteride et parapegmate, sed a Disitheo correctis.
\lnr{32}Parapegma
cum Octaeteride Lucanus vocat fastos, vel, ut ipse loquitur, fastus.
\lnr{34}\emph{Nec meus Eudoxi vinctur fastibus annus.}
\lnr{34}Geminus postquam ex
Dosithei parapegmate quaedam produxit, statim subiicit etiam testimonium
ex Eudoxo.
\lnr{36}Sed utrumque est Eudoxi; prius quidem ex posteriore
editione Dosithei, posterius autem ex priore Eudoxi.
\lnr{37}Quare
multi veterum octaeterida Eudoxi Dositheo attribuunt, teste Censorino.

% 72
% {PDF page nr}{source page nr}{line nr}
\plnr{155}{72}{1}Illustravit etiam eandem Octaeterida commentario et expositione
Eratosthenes Cyrenaeus.
\lnr{2}Porro de Octaeteirde ita legitur apud
Censorinum: \emph{Hunc quoque circuitum vere annum magnum esse pleraque
Graecia existimavit, quod ex annis vertentibus solidis constaret,
ut proprie in anno magno fieri par est.}
\lnr{5}\emph{Nam dies sunt solidi uno minus
centum, annique vertentes solidi octo.}
\lnr{6}Haec a librariis mutilata
ita sunt supplenda: \emph{Nam dies sunt solidi ducenti noningenti viginti
duo, menses uno minus centum, et cetera.}
\lnr{8}Plinius vero de Octaeteride
intelligit, cum scribit libro
 \textsc{xviii}, \textsc{xxv} cap. \emph{Indicandum est et
illud, tempestates ipsas ardores suos habere quadrinis annis, et easdem
non magna differentia reverti ratione solis: octonis vero augeri easdem,
centesima revoluente se Luna, et cetera.}
\lnr{12}Hoc enim ex parapegmatis Eudoxi
et Dosithei hausit, quae quidem erant edita una cum octaeteride
Eudoxea.

\subsection{De Anno Magno Metonis, sive Enneadecaeteride}

\lnr{15}Quis status anni Lunaris Atheniensium esse potuerit, cum ad
Canonas Octaeteridis describeretur, ex comparatione utriusque
octaeteridis Tetraetericae et Lunaris Harpaleae, scire potuisti.
\lnr{18}Quamuis enim initia et tempus institutae Octaeteridis 
 \textgreek{πρυτανείας[?]}
Harpaleae ignoramus, tamen in quadraginta annis magnam turbationem
noviluniorum necessario confecutam esse, potes ex eadem
comparatione colligere.
\lnr{21}Itaque Aristophanes Olympiade 88, Amynia
praetore, Nephelas docens inducit Lunam cum Ahtenionsibus[?] expostulantem,
quod menses ad Lunam non describerent, sed \textgreek{ανω τε και
κάτω κυδοιδοπᾷν[?]}
ait.
\lnr{24}Nam putantes se solennes cultus et ferias ex Lunae
et Octaeteridis observatione obire, eas alieno tempore anni imprudenter
celebrabant.
\lnr{26}Quod plane anomaliae octaeteridis congruit:
donec Metonis anno succedente ea penitus desita est.
\lnr{27}Theophrastus
\textgreek{περὶ σημεῖων ὑδάτῶν καὶ πνευμάτων.}
% Title: περι σημειων υδατων και πνευματων [και χειμωνων και ευδιων] 
\lnr{28}\textgreek{διὸ καὶ ἀγαθοὶ γεγένηται  κατὰ τόπους τινὰς
ἀστρονόμοι ἔνιοι, οἷον Ματρικέτας ἐν Μηθύμνῃ ἀπὸ τοῦ Λεπετύμνου, καὶ Κλεόστρατος
ἐν Τενέδῳ ἀπὸ τῆς Ιδης, καὶ Φαεινὸς Αθήνῃσιν ἀπὸ τοῦ Λυκάμβη τοῦ τὰ
 περὶ τὰς τροπὰς
συνεῖδε.}
\lnr{31}\textgreek{παῤ οὗ Μέτων ἀκούσας τόν τοῦ ἑνὸς δέοντα ἔικοσιν ἐνιαυτὸν
 συνὲταξει.
ἦν δὲ ὁ μὲν Φαεινὸς μέτοικος Αθήνῃσιν ὁ δὲ Μέτων Αθηναῖος. [?]}.
% [καὶ ἄλλοι δὲ τὸν τρόπον τοῦτον ἠστρολόγησαν.]
% Theophrastus
% "De signis"
% Paragraph 4
% "Thus in some parts have been found good astronomers: for instance, Matriketas
% at Methymna observed the solstices from Mount Lepetymnos, Cleostratus in
% Tenedos from Mount Ida, Phaeinos at Athens from Mount Lycabettus: Meton, who
% made the cycle of nineteen years, was the pupil of the last-named. Phaeinos
% was a resident alien at Athens, while Meton was an Athenian.
% [Others also have made astronomical observations in like manner.]"
\lnr{32}Meton igitur Pausaniae filius, ut tunc captus erat Graecorum,
 insignis Mathematicus
floruit ineunte bello Peloponnesiaco, vir non solum peritia motuum
coelestium, sed et aquiliciis, et librationibus nobilis.
\lnr{35}Itaque et
fontes induxit Athenis, ut auctor est
 Phrynichus Comicus \textgreek{μονοτρόπῳ[?]}.
\begin{quote}
\textgreek{Τίς ἐστιν ὁ μετὰ ταῦτα ταύτης σροντιῶν;[?]}\\
\textgreek{Μέτων ὁ Λευκονοιεὺς, ὁ τὰς κρήνας ἄγων.[?]}
\end{quote}
% Phrynichus Comicus, comic poet, ca 430 BCE
% Fragments of his works survive.
% μονοτρόπῳ (Monotropos; The Solitary) is a play by him exhibited in 414 BCE.
% All fragments are collected in Theodor Kock: Comicorum atticorum fragmenta
% (Teubner, 1880). This fragment on page 376
% https://archive.org/stream/comicorumatticor01kockuoft#page/376/mode/1up
% "A. τίς δ᾽ ἔδτιν ὁ μετὰ ταῦτα φροντίζων;
%  B. Μέτων, ὁ Λευκονοιεύς.
%  A. οἶδ᾽, ὁ τὰς κρήνας ἄγων."
% "Leuconoea erat pagus Leontidis tribus, unde oriundus Meton (Aristoph.
% Av. 992 cum interpr. Aelian. V. H. 10, 7, ubu Λευκονοιεύς ex
% Salmasii coniectura pro Λάκων)."

% 73
% {PDF page nr}{source page nr}{line nr}
\plnr{156}{73}{2}Quare huc alludens Manilius noster in Apotelesmatis,
 sub Aquario
Aquilices et Astronomos ait nasci:
\begin{quote}
  \emph{Ille quoque, inflexa fontem qui proiicit urna,\\
  Cognatas tribuit iuvenilis Aquarius artes\\
  Cernere sub terris; undas inducere tectis. et cetera.\\
  Quippe etiam mundi faciem, sedesque movebit\\
  Sidereas, coelumque novum versabit in orbem.}
\end{quote}
% Astronomica (attibuted to Marcus Manilius), written sometime during
% the reign of either Caesar Augustus or Tiberius.
% Only a few copies-of-copies manuscripts survived, and the exact contents
% is, and was, contested.
% Scaliger published two critically edited editions of this work
% (in 1579 and 1599).
% This quote by Scaliger diverges from current publications. 
% Liber IV
% lines 259-261 [et cetera]
% proiicit -> proicit; iuvenilis -> iuvenalis
% sub terris; undas inducere tectis -> sub terris undat, inducere terris.
% and lines 267-268
% coelumque -> caelumque
\lnr{9}Non est locus nobilior in toto Manilio, quamuis olim eum non satis
capiebamus.
\lnr{10}Veteres, quae fuit eorum hac in re imperitia, putabant
omnium motuum coelestium et mundi ipsius integram conversionem
fieri, quando Sol et Luna in idem tempus recurrebant,
in quo antea deprehendebantur, et quamdiu Octaeteridis
fides suspecta non fuit, eam esse mensuram
 \textgreek{ἀποκαταστάσεως τοῦ παντὸς[?]}
non solum vulgis, sed et docti credebant.
\lnr{15}Itaque Festus Avienus ex
Graecorum libris dixit:
\begin{quote}
  \emph{Non ego nunc longo redeuntia sidera motu\\
  In priscas memorem sedes. ---}
\end{quote}
% Avienus - Aratea
% See also page 68, where the same lines are quoted (with slight differences)
% Lines 1363-1369
\lnr{19}Innuit, ut vides, de ortu et occasu
 \textgreek{τῶν μορφώσεων[?]} annuo.
\lnr{19}Nam si qua
est varietas, eius \textgreek{ἀποκαταστασιν[?]},
 quantocunque tempore illa reditura
sit; omittit dicere.
\lnr{21}Subiicit:
\begin{quote}
  \emph{--- Habet ist a priorum\\
  Pagina, et incerta rerum ratione feruntur.}
\end{quote}
% Quote above continues: Lines 1369-1370
Diversa de istis scriptorum et Astronomorum iudicia, et libros exstare
dicit:
\begin{quote}
  \emph{Nam quae Solem hiberna novem putat athere volui,\\
  Ut spatium Lunae redeat, vetus Harpalus, ipsam\\
  Ocius in sedes, momentaque prisca reducit.}
\end{quote}
% Quote above continues: Lines 1371-1373
Putat \textgreek{ἀποκατάστασιν[?]} inerrantium fieri, ita ut sidera,
 ut ipse loquitur, in
priscas sedes longo motu redeant.
\lnr{30}Putat, inquam, hoc accidere, quandocunque
spatium Lunae redit.
\lnr{31}Ipse interpretatur mentem suam.
\lnr{32}Quando, inquit, Luna in sedes et momenta prisca reducitur.
\lnr{32}Quod
intervallum, inquit, Harpalus annorum novem esse decrevit, sed
male, cum ocior sit iusto ista periodus.
\lnr{34}Aperte igitur censet, conversionem
coeli universalem fieri, quando neomenia redit in eandem
diem, et horam, in qua antea fuit.
\lnr{36}Subiicit Festus Avienus, hanc periodum
ob brevitatem fallere, ideoque ei decennium additum a
Metone.

% 74
% {PDF page nr}{source page nr}{line nr}
\plnr{157}{74}{2}Et veram \textgreek{ἀποκατάστασιν τοῦ φαινομένων[?]}
 intra \textgreek{ἐννεαδεκαετηρίδα[Greek]}
fieri.
\lnr{3}De hallucinatione Festi super nomine Enneaeteridos, supra
dictum est.
\lnr{4}Aratus quoque volens ostendere omnium inerrantium
\textgreek{ἐποχὴν[?]} Solem ipsum esse, absurde quidem, sed tamen eius
rei causam confert ad cyclum Metonis.
\lnr{6}Quo intervallo, \textgreek{δύσεων καὶ ἀνατολῶν τοῦ φαινομένων[?]}
fiat restitutio, orbis, et conversio quaedam
mundi universalis.
\lnr{8}Ita enim canit elegantius, quam verius:
\begin{quote}
  \textgreek{Γινώσκεις τάδε καὶ σύ. Τὰ γὰρ συναείδεται ἤδη[?]}\\
  \textgreek{Εννεακαίδεκα κύκλα φαεινοῦ ἠελίοιο[?]}\\
\end{quote}
% Aratus: Phaenomena
% Lines 752-753
% "You too know all these (for by now the nineteen cycles of the shining
% sun are all celebrated by all)"
\lnr{11}Quem locum summum virum Theonem ex toto affecutum non
esse mirum non est, cum Hipparchus, et post eum Astronomiae
Apollo Ptolemaeus, quantitatem anni Tropici ex neomeniarum
restitutione collegerint, quam restitutionem Hipparchus recte
censet post annos 304 statim fieri.
\lnr{15}Sed de his libro quarto amplius.
\lnr{16}Apertius vero Diodorus Siculus docet illos veteres,
 \textgreek{τὰ φαινόμενα ἀποκαθίστασθαι[?]}
illis novemdecim annis vertentibus, putasse.
\lnr{17}Libro
enim duodecimo de illis novemdecim annis loquens ita scribit;
\lnr{19}\textgreek{ἐν δὲ τοῖς ἐιρημένοις ἔτεσι τὰ ἄστρα τὴν ἀποκατάστασιν ποιεῖται,
 καὶ καθάπερ
ἐνιαυτοῦ τινὸς μεγάλου τὸν ἀνακυκλισμὸν λαμβάνει.}
\lnr{20}\textgreek{διὸ καί τινες αὐτὸν Μέτωνος
ἐνιαυτὸν ὀνομάζουσι.}
\lnr{21}\textgreek{δοκεῖ δὲ ὁ ἀνὴρ οὗτος ἐν τῇ προῤῥήσει καὶ προγραφῇ
ταύτῃ θαυμαστῶς ἐπιτετευχέναι.}
\lnr{22}\textgreek{τὰ γὰρ ἄστρα τήν τε κίνησιν, καὶ τὰς ἐπισημασίας
ποιεῖται συμφώνως τῇ γραφῇ.}
\lnr{23}\textgreek{διὸ μέχρι τῶν καθ´ ἡμᾶς χρόνων οἱ πλεῖστοι
τῶν ἑλλένων χρώμενοι τῇ ἐννεα[και]δεκαετηρίδι οὐ διαψεύδονται τῆς ἀληθείας.}
% Diodorus Siculus - Bibliotheca Historica
% Διόδωρος Σικελιώτης - Ἱστορικὴ Βιβλιοθήκη
% Book 12
% [12,36] (second half)
% ἐν δὲ τοῖς εἰρημένοις ἔτεσι τὰ ἄστρα τὴν ἀποκατάστασιν ποιεῖται
% καὶ καθάπερ
% ἐνιαυτοῦ τινος μεγάλου τὸν ἀνακυκλισμὸν λαμβάνει.
% διὸ καί τινες αὐτὸν Μέτωνος
% ἐνιαυτὸν ὀνομάζουσι.
% δοκεῖ δὲ ὁ ἀνὴρ οὗτος ἐν τῇ προῤῥήσει καὶ προγραφῇ
% ταύτῃ θαυμαστῶς ἐπιτετευχέναι.
% τὰ γὰρ ἄστρα τήν τε κίνησιν καὶ τὰς ἐπισημασίας
% ποιεῖται συμφώνως τῇ γραφῇ.
% διὸ μέχρι τῶν καθ´ ἡμᾶς χρόνων οἱ πλεῖστοι
% τῶν Ἑλλήνων χρώμενοι τῇ ἐννεακαιδεκαετηρίδι οὐ διαψεύδονται τῆς ἀληθείας.
\lnr{25}Quid melius potuit versus Arateos interpretari?
\lnr{25}Sed fallitur Diodorus.
\lnr{26}Nam \textgreek{ἐννεαδεκαετηρὶς[?]}, quae illius temporibus obtinebat,
 erat
Calippica, non autem Metonica: cum Metonica plus quam
quinque diebus, Calippica uno fere die illo faeculo iam antevertissit,
quo haec scribebat Diodorus.
\lnr{29}Idem scriptor libro secundo
eadem repetit, loquens de enneadecaeteride gentium Hyperborearum:
\textgreek{λέγεται δὲ καὶ τὸν θεὸν[?]} (Apollinem)
 \textgreek{δἰ ἐτῶν ἐννεακαίδεκα καταντᾷν
εἰσ την νῆσον, ἐν σεσ[?] καὶ αἱ τῶν ἄστρων ἀποκαταστάσεις ἐπὶ τελος ἄγονται.[?]}
\lnr{32}\textgreek{καὶ
διὰ τοῦτο τὸν ἐννεακαιδεκαετῆ χρόνον ὑπὸ τῶν ἑλλήνων μέγαν ἐνιαυτὸν
 ὀνομάζεσθαι[?]}
\lnr{34}Huius meminit et Aelianus libro decimo.
\lnr{34}Nunc quid
velit Manilius per te ipse potes intelligere.
\lnr{35}Ait Metonem coelum
versasse in novum orbem, hoc est, conversionem mundi
 \textgreek{καὶ ἀποκαταστασιν[?]}
novo orbe et nova periodo \textgreek{τὴς ἐννεαδεκαετηρίδος[?]} definivisse, cum
orbis vetus Harpali mendosus, et fallax tempore deprehensus sit.
\lnr{39}Hoc est, quod dicit versare coelum in novum orbem.
\lnr{39}Et novum
orbem periodum Metonicam intelligit, comparatione veteris, quam
Octaeterida Harpali esse ostendit Festus Avienus.

% 75
% {PDF page nr}{source page nr}{line nr}
\plnr{158}{75}{2}Enneadecaeteris
igitur Metonis celeberrima multis quidem nominibus commendatur:
sed eam parum hactenus notam fuisse, argumento sunt ii,
qui eandem penitus cum cyclo Paschali, et nostro vulgari, quem
numerum aureum vocamus, difiniunt.
\lnr{6}Tantum enim septem embolismos,
et novemdecim annos cum nostro numero aureo communes
habet, in reliquis immane quantum differt.
\lnr{8}Nam neque
idem situs embolismorum, neque eadem anni Solaris quantitas:
cum Cyclus Metonis sit absolute dierum 6940, discedens
ab iusta periodo horis 7. 26.' 56.'' 40.'''
% à ->ab
\lnr{11}Unde in annis 76 moratur
Lunae curriculum die uno, horis 5. 47.' 46.'' 40.'''
\lnr{12}In annis
denique 304 Luna antevertit primam epocham Metonicam diebus
solidis quinque.
\lnr{14}Annos autem embolimaeos septem aut eorum
situm, scire non possumus priusquam epocham cycli ipsius
et caput indagemus.
\lnr{16}Veteres Graeci ab bruma, ut cognoscimus
% à ->ab
ex octaeteridibus Cleostrati et Harpali, tempora sua ordiebantur,
et eos fecuti Romani, quod facilius a decrementis umbrae
horas observarent, brumae confecto die, quam ab incrementis,
solstitio.
\lnr{20}Quare omnia horologia Graecorum semper
ad rationes brumae referebantur, ut et Romana: donec primus
omnium Meton noster ab capite anni populari, aut potius ab eius
% à ->ab
epocha, horologium describere instituit, hoc est ab solstitio.
% à ->ab
\lnr{23}Unde
ipsum organon \textgreek{ἡλιοτρόπιον[?]} vocarunt Graeci,
 a solstitii observatione.
\lnr{25}Quod instrumentum nobilissimum, atque priscorum hominum
observationes longe subtilitate vincens, ipse in Comitio
Athenarum dicavit, ut testis est priscus scriptor Philochorus apud
interpretem Aristophanis \textgreek{ὀρνίθων[?]},
 instar \textgreek{ἡλιοτροπίου[?]} Pherecydis, quod
in Scyro insula patria sua auctor dedicavit.
\lnr{29}Itaque videtur non solum
a solstitio caput Enneadecaeteridis suae deduxisse, sed etiam
ab eo tempore, quo Heliotropium suum dicavit.
\lnr{31}Idem
Scholiastes Aristophanis affentitur quidem Philochoro de Heliotropii
positu: sed de tempore refragatur.
\lnr{33}Philochorus enim
censebat Metonem ante Pythodorum Archontem Heliotropium
posuisse.
\lnr{35}Ipse post Pythodori magistratum aut saltem in magistratu
ipso, non autem ante magistratum, positum contendit.
\lnr{36}Verba
Grammatici de Metone haec sunt: \textgreek{ὁ δὲ φιλόχορος ὀν Κολωνῷ μὲν
αὐτὸν οὐδὲν λέγει θεῖναι, φευδῶς δὲ πρὸ Πυθοδώρου ἡλιοτρόπιον ἐν τῇ νηῦ
λεγομένῃ ἐκκλησίᾳ, πρὸς τῶ τείχει τῷ ἐν τῇ πνυκὶ[?]}.
% Possibly: fragment of Phrynichus - Monotropos
% "Fr. 22 PCG, Σ Αν. 997"
\lnr{39}Lege: \textgreek{οὐδὲν λέγει
θεῖναι[?]}.
\lnr{40}\textgreek{ἐπ᾽ Αψεύδοις δὲ τοῦ πρὸ Πυθοδώρου.[?]}
% Same fragment
\lnr{40}Sane verum est adhuc
sub magistratu Apseudis illud Heliotropium dedicasse, et solstitium
observasse 27 Iunii, diebus 36 ante neomeniam sequentis Hecatombaeonis
Tetraeterici.

% 76
% {PDF page nr}{source page nr}{line nr}
\plnr{159}{76}{3}Itaque adhuc erat in magistratu Apseudes:
cui in sequenti anno successit Pythodorus.
\lnr{4}Parum igitur abest,
quin et a Solstitio cyclum suum incipisse, et circa tempora Pythodori
Heliotropium statuisse credamus, optimi scriptoris auctoritate
moti.
\lnr{7}Sed de Solstitio cur dubitem, cum auctorem locupletem
habeam Festum Avienum?
\lnr{8}Qui post eos versiculos a nobis paulo
ante adductos subiicit, loquens de Harpalo:
\begin{quote}
  \lnr{8}\emph{Illius ad numeros prolixa decennia rursum}\\
  \emph{Adiecisse Meton Cecropea dicitur arte,}\\
  \emph{Inseditque animis. Tenuit rem Graecia solers [sic]}\\
  \emph{Protinus, et longos inventam misit in annos.}\\
  \emph{Sed primaeva Meton exordia sumpsit ab anno,}\\
  \emph{Torreret rutilo cum Phoebus sidere Cancrum:}\\
  \emph{Cingula cum veheret pelagus procul Orionis,}\\
  \emph{Et cum caeruleo flagraret Sirius astro.}\\
\end{quote}
% Rufius Festus Avienus: Aratea, lines 1369-1376
% Speaking of Harpalus
% [Edition Alfred Breysig, Lipsiae 1882]
% https://archive.org/details/rufifestiavienia00avieuoft
% "Illius ad numeros prolixa decennia/decentia rursum
% adiecisse Meton Cecropea dicitur arte
% inseuitque/inseditque animis: tenuit rem Graecia sollers
% protinus et longos inuentum misit in annos.
% et/sed primaeua Meton exordia sumpsit ab anno,
% torreret rutilo cum Phoebus sidere cancrum,
% cingula cum ueheret pelagus procul Orionis
% et cum caeruleo flagraret Sirius astro."
% (slashed words are reported by Breysig to be different in various sources)
\lnr{18}A solstitio igitur duxit citimum novilunium.
\lnr{18}Remotissimum vero
statuit ad aestus maximos Caniculae.
\lnr{19}Iam igitur constat de exordio periodi
Metonicae.
\lnr{20}De tempore, hoc est anno observati a Metone solstitii,
item de tempore et epocha ipsius Solstitii, habemus plene apud
Ptolemaeum, libro \textsc{iii},
 qui ait diserte Solstitium a Metone et Euctemone
observatum anno Nabonassari 316, Phamenoth \textsc{xxi}, mane.
\lnr{24}Tempus congruit \textsc{xxvii} Iunii, cyclo Lunae \textsc{vii},
 cyclo Solis \textsc{xxvi}, feria
prima, anno quarto Olympiadis 86 definente, praefecto Athenis
Apseude, T. Verginio, Proculo Geganio Macerino \textsc{coss}.
\lnr{26}Qui erat
tertius annus periodi Atticae.
\lnr{27}Scirrhophorion \textsc{iii} Iulii.
\lnr{27}Ergo Neomenia
Hecatombaeonis Metonici \textgreek{σκιῤῥοφοριωνος τρίτῃ ἐπὶ δέκα}.
\lnr{28}Diodorus
libro duodecimo:
% Diodorus Siculus: Bibliotheca historica (Βιβλιοθήκη ἱστορική),
% Book 12, chapter 36, section 2:
% How Meton of Athens was the first to expound the nineteen-year cycle.
 \textgreek{ἐν δὲ ταῖς Αθήναις Μέτων ὁ Παυσανίου μὲν υἱός,
δεδοξασμένος δὲ ἐν ἀστρολογίᾳ ἐξέθηκε τὴν ὀνομαζομένην ἐννεακαιδεκαετηρίδα,
τὴν ἀρχὴν ποιησάμενος ἀπὸ μηνὸς ἐν Ἀθήναις σκιροφοριῶνος τρισκαιδεκάτης.}
% Translation by C. H. Oldfather (1946)
% ISBN 978-0-674-99413-3
% http://data.perseus.org/citations/urn:cts:greekLit:tlg0060.tlg001.perseus-eng1:12.36.2 
% "In Athens Meton, the son of Pausanias, who had won fame for
% his study of the stars, revealed to the public his nineteen-year cycle,
% as it is called, the beginning of which he fixed on the thirteenth day of
% the Athenian month of Scirophorion."
% [Continues:] In this number of years the stars
% accomplish their return to the same place in the heavens and conclude,
% as it were, the circuit of what may be called a Great Year;
% consequently it is called by some the Year of Meton."
\lnr{32}Proinde \textgreek{θαργηλιῶνος πέμπτῃ φθίνοντος[?]}
 observatum Solstitium
a Metone.
\lnr{33}Et proximo \textgreek{πρυτανείας[?]} Hecatombaeone Pythodorus
inivit magistratum, novem mensibus ante initia belli Peloponesiaci.
\lnr{35}Haec igitur est Epocha cycli Metonici, non autem \textsc{ix} Iulii,
ut solebant vetustiores: neque octava pars Cancri, ut Cleostratus.
\lnr{37}Quare mirari satis non possum, cur Columella dixerit, se, auctore
Metone, solstitium in octava parte Cancri, sicut alia \textgreek{κέντρα[?]}
in octavis partibus signorum suorum, statuere, cum res ipsa eum
satis refellat.

% 77
% {PDF page nr}{source page nr}{line nr}
\plnr{160}{77}{1}Cur enim potius Columellae de Metone, quam
Metoni ipsi credam?
\lnr{2}De modo Enneadecaeteridis Metonicae
scribit Censorinus:
% Censorinus: De Die Natali Liber, chapter 18
 \emph{Praeterea sunt anni magni complures: ut Metonicus,
quem Meton Atheniensis ex annis undeviginti constituit.}
\lnr{4}\emph{Eoque
Enneadecaeteris appellatur: et intercalatur septies: in eoque
anno sunt dierum sex millia, et quadringenti quadraginta.}
% "Praeterea sunt anni magni conplures, ut Metonicus,
% quem Meton Atheniensis ex annis undeviginti constituit,
% eoque enneadecaeteris appellatur et intercalatur septies, inque eo
% anno sunt dierum VI milia et DCCCXL"
% 'sex millia, et quadringenti quadraginta' = 6440
% 'VI milia et DCCCXL' = 6000 + 500 + 300 + 40 = 6840
\lnr{6}Legendum:
\emph{in eoque anno sunt dierum sex millia et noningenti quadraginta.}
% noningenti => nongenti = 900
% sex millia et noningenti quadraginta = 6940
\lnr{8}Nam ex notis vulgaribus fluxit error, dierum sex millia,
 et \textsc{ccccxl}.
\lnr{9}Deest enim \textsc{d}
% Preferably I (Unicode U+2160) plus reversed C (Unicode U+2183): "ⅠↃ"
% But most fonts don't support these.
% Currently (feb 2017) supported on Mac OS X by:
% Baskerville (regular, italic, semibold, semibold italic, bold, bold italic)
% Big Casion Medium
% Courier (regular, oblique, bold, bold oblique)
% Geneva
% Helvetica (regular, oblique, bold, bold oblique)
% Helvetica Neue (regular, italic, bold, bold italic)
% Lucida Grande (regular, bold)
% Trattello
% Notably *not* supported by Society of Biblical Literature (SBL) fonts.
 nota quingentorum.
\lnr{9}Cum 5940 dies comprehenderet
Enneadecaeteris Metonica, ea null modo potuit
congruere cum vera enneadecaeteride Lunari, ut infra demonstrabitur.
\lnr{12}Servata epocha in \textsc{xxvii} Iunii, facile embolismorum
situs et tempora deprehendemus.
\lnr{13}Nulla enim Neomenia Hecatombaeonis
Solstitium antevertebat.
\lnr{14}Quocirca secundus, quintus,
octavus, decimus, tertiusdecimus, sextusdecimus, decimus octavus
anni erant embolimaei; contra quam consent docti homines nostri
temporis.
\lnr{17}Cum autem periodus ipsa 6940 diebus praecise explicaretur,
in illis erant anni Lunares \textsc{xix},
 menses \textgreek{τριακονθήμεροι[?]} septem,
dies quatuor, scrupula nulla.
\lnr{19}Quare propter illos quatuor dies abundantes,
quatuor quoque anni erant \textgreek{ὑπερήμεροι[?]}, dierum scilicet
355, ut postea videbimus.
\lnr{21}Anni autem Lunaris modus, secundum
Metonem, est dierum 354~\myfrac{4}{19}, aut \myfrac{16}{76}.
\lnr{22}Neomenia prima Hecatombaeonis
Metonici fuit Iulii \textsc{xv}.
\lnr{23}Quare cyclus Metonis constat
non ex enneaeteride et dacade, ut voluit Festus Avienus, sed ex Octaeteride
et Hendecaeteride.
\lnr{25}Nam nullae aliae partes sunt, enneadecaeteridis,
quae propius absint a modulo anni Solaris: nec quaemelius
in se cohaerant.
\lnr{27}Quod enim Octaeteridi superest supra rationes
Solis, id deest Hendecaeteridi, et contra.
\lnr{28}Sed cum Meto videret
accurata observatione septimam diem Hecatombaeonis vicesimi
Tetraeterici semper in novilunium concurrere; (verbi gratia, incipiat
primus Hecatombaeon \textsc{ix}. Iulii, ut in principio periodi Atticae;
vicesimus incipiet in Kal. Augusti, et in \textsc{vii} mensis erit novilunium;
id quod me tacente indicat laterculus mensium Tetraetericorum
antea a nobis propositus) cum igitur hoc videret Meto, animadvertit
in hoc intervallo contineri duas Octaeteridas Harpali.
\lnr{35}Id
quod et puero proclive: item dies 1092.
\lnr{36}Sed ii dies est triennium Harpaleum,
vel Cleostrateum, hoc est tres anni Lunares cum uno mense
intercalari pleno.
\lnr{38}Duae autem Octaeterides sive Harpaleae, sive Tetraetericae
sunt dies 5848.
\lnr{39}Quibus si adieceris 1092, confurget summa
dierum 6940.
\lnr{40}igitur iustam periodum confici posse ex duabus
Harpali Octaeteridibus et triennio Lunari existimavit.

% 78
% {PDF page nr}{source page nr}{line nr}
\plnr{161}{78}{1}Rursus in
duabus Octaeteridibus, sex sunt embolismi; in triennio unus.
\lnr{3}Ergo septem embolismi transigentur in ea periodo: et fient omnes
syzygiae 235: quia in duabus Octaeteridibus sunt 198, et in triennio
Lunari 37.
\lnr{5}Atque adeo decemnovem annis tota periodus explicabitur.
\lnr{6}Unde eam \textgreek{ἐννεαδεκαετηρίδα} vocavit.
\lnr{6}Si igitur omnes
menses 235 huius periodi essent \textgreek{τριακονθήμεροι[?]},
 et pleni, ii fierent
dies 7050.
\lnr{8}De quibus si detrahantur dies 6940, quantitas nempe
huius periodi, relinquentur menses cavi 110, qui debentur
huic periodo: et proinde reliqui 125 erunt pleni.
\lnr{10}Longe igitur
maior erit numerus plenorum, quam cavorum: neque erunt
alternis pleni et cavi, ut nec in Harpali Octaeteride erant alternis
pleni et cavi.
\lnr{13}Si igitur 235 in 110 distribuantur, habebimus syzygias
2. dies 4~\myfrac{1}{11}.
\lnr{14}Eae enim in 110 multiplicatae faciunt 235 syzygias
praecise: quas intelligimus omnes plenas.
\lnr{15}Itaque post duas
syzygias et dies 4~\myfrac{1}{11}, utendum erit \textgreek{ἐξαιρέσα[?]},
 ut pro 4 mensis, dicatur
quinta.
\lnr{17}Item eodem modo post quatuor syzygias, pro quintae
syzygiae octava dicetur nona.
\lnr{18}Et ita progrediendo erogabis omnes
\textgreek{ἐξαιρέσας[?]}, donec ultima syzygia sit cava, et pro eius tricesima,
dicatur prima Hecatombaeonis primi secundae periodi.
\lnr{20}Id
quod in conspectu tibi dedimus in duabus sequentibus Tabulis: in
quarum priore omnes syzygiae cum characteribus suis notatae sunt.
\lnr{23}Nam Cella, quae habet tres numeros, ea indicat mensem cavum.
\lnr{24}Puta in primo anno, tertio mense, in cella habes \myfrac{45}{4}.
\lnr{24}Duo
priores numeri significant pro 4 mensis, dicendum 5: vel, ut Graeci
loquuntur pro \textgreek{τετάρτη ἱσταμένου[?]}, \textgreek{πέμπτη ἱσταμένου[?]}.
\lnr{26}Ideo mensis
est cavus.
\lnr{27}Inferior autem numerus est feria, vel character neomeniae.
\lnr{28}Quaecunque autem neomeniae habent unum numerum, eae
sunt plenorum mensium.
\lnr{29}Reliqua facilia sunt.
\lnr{29}Ex quibus vides
\textgreek{ἐξαίρεσιν[?]} non fieri in uno die,
 sed prout coniugatio duarum syzygiarum
postulat.
\lnr{31}Post binas enim syzygias fit \textgreek{ἐξαίρεσις[?]}, donec numerus
ex \myfrac{1}{11} accrescens addat diem prioribus diebus quatuor.
\lnr{32}Quare
in omni undecima syzygia accrescit dies unus.
\lnr{33}Insigniter autem
fallitur Geminus, priscus et eruditus auctor, qui scribit Metonem
divisisse 6940 dies per 110 syzygias: et quia 110 in 6940
continentur sexagesies ter, propterea censet Metonem statim
post 63 dies \textgreek{ἐξαίρεσιν τῶν ἡμερῶν[?]} fecisse.
\lnr{37}Hoc enim ratio ipsa confutat.
\lnr{38}Nam 63 dies sunt syzygiae 2, et dies praeterea 3.
\lnr{38}Quae omnia
in 110 ducta producunt syzygias, 220, dies 330.
\lnr{39}Hoc est syzygias
undecim, quae cum 220 syzygiis compositae dant tantum 231
syzygias plenas.

% 79
% {PDF page nr}{source page nr}{line nr}

\plnr{162}{79}{1}% Tabula Characterismi neomeniarum enneadecaeteridis metonicae
\lnr{1}Itaque calculus desinet in ducentesima prima syzygia.






























% ==== End of text of Liber Secundus ===
