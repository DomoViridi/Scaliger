% !TEX TS-program = xelatex
% !TEX encoding = UTF-8 Unicode
% this template is specifically designed to be typeset with XeLaTeX;
% it will not work with other engines, such as pdfLaTeX

%%% Count out columns for fixed-width source font
% 000000011111111112222222222333333333344444444445555555555666666666677777777778
% 345678901234567890123456789012345678901234567890123456789012345678901234567890

\chapter{Prolegomena}
\sc{Prolegomena in Libros de Emendatione Temporum}

\em{Ad candidum Lectorem.}

\normalfont{}

\setcounter{parcount}{0}
\begin{parnumbers}
\pnr{p. I [pdf 28]}
\plnr{I}{1}Quintusdecimus hic annus agitur, candide
Lector, postquam opus nostrum de
Emendatione Temporum emisimus.
\lnr{3}Persuaseram
mihi, homines studiosos aliquam nobis
gratiam habituros tot rerum, quas et scitu
dignas, et a nobis primum indicatas negare
non poterant.
\lnr{7}Sed longe aliter animatos experti
sumus: atque adeo rem potius invidiosam
atque obtrectationi opportunam, quam illis gratam me suscepisse
intellexi.
\lnr{10}Denique nihil aliud quam significarunt, quiduis potius
se ignorare malle, quam a nobis aliquid discere.
\lnr{11}In quibusdam
candorem, in aliis studium, in omnibus sensum bonarum rerum desideraui.
\lnr{13}Nos vero, qui nihil unquam prius habuimus, quam ut horum
orationes sinamus praeterfluere, modo verum eruere, et inimicos
nostros etiam inuitos iuvare possimus, opus nostrum iterum in
manus sumptum auximus, illustravimus, emendauimus, ut, quanuis
idem sit, aliud tamen a nova cultura videri possit.
\lnr{17}Quae huic editioni
accesserunt, haud promptum est dicere.
\lnr{18}Sed in quibus a priore demutat,
postea intelliges, siquidem instituti nostri rationem aperuero.
\lnr{20}Subiectum operis nostri est ratio Temporum civilium, et eorum,
quae in vetustatis cognitione versantur: finis, Emendatio: quod quidem
me tacente, et Titulus ipse promittit.
\lnr{22}Civilium temporum cognitio,
eorumque historia, vertitur in multiplici diversorum annorum
forma et eorum methodis vulgaribus, quos Computos posterior
aetas vocauit.
\lnr{25}\textgreek{Τα ιςορομγυα[?]} civilium temporum habes in primoribus
tribus libris, et maiore parte quarti: methodum autem in septimo.
\lnr{27}A emendationis duae partes sunt.
%\end{parnumbers}
%%\clearpage
\pnr{p. II [pdf 29]}
%\begin{parnumbers}
\lnr{27}Prior versatur circa epocharum
investigationem, posterior circa verum annum tropicum, 
et periodos Lunares: quam materiam posterior pars libri quarti,
item toti quintus et sextus sibi vindicant.

\plnr{II}{2}Iam quemadmodum Epochae
sunt notationes, et tituli temporum, ita ipsarum epocharum
quaedam debent esse propria \textgreek{γνωρίσματα} et characteres: quorum
characterum alii sunt naturales, alii civiles. 
\lnr{5}Naturales quidem a rationibus
utriusque sideris, unde nati cycli Solaris, et Lunaris: civiles
ab instituto, cuiusmodi indictiones et anni Sabbatici: sine quibus in
harum rerum tractatione omnis conatus irritus. 
\lnr{8}Rursus et eorum
quoque fallax usus est, nisi quaedam annorum ex illis periodus instituatur.
\lnr{10}Sed eae sunt totidem, quot aut formae annorum, aut civilia
initia.
\lnr{11}Nam in anno Aegyptiaco Nabonassari alia opus est, ac in anno
Solari, quia diversa forma: item in anno Actiaco sive Diocletianeo
alia, ac in Iuliano, propter diversa initia.
\lnr{13}In anno Aegyptiaco vago
naturales characteres sunt \textgreek{εἰκοσιπεν σαετηρις[?]} Lunaris, et
\textgreek{έπταετηρις[?]} Solaris:
civilis autem character est quadriennium, quem canicularem
annum minorem vocabant Aegyptii.
\lnr{16}Hi tres characteres in se ducti
producunt periodum magnam annorum 700 Aegyptiacorum: qua
uti debet disputator temporum, siquidem rationes suas ad annos
Nabonassari, Armeniorum, aut Persarum exigit.
\lnr{19}At qui anno Iuliano,
quae omnium formarum temporibus est convenientissima, uti
volet, is cyclo utriusque sideris quindecies ducto componet elegantissimam
periodum annorum 7980, cuius initium in cyclo Solari,
et Indictione Romana, a Kal. Ianuarii, in cyclo Lunari a Martio, in
anno Sabbatico ab autumno.
\lnr{24}Itaque non minus utilis, quam necessaria
est.
\lnr{25}Sine ea nihil agit Chronologus: cum ea tempori, et saeculis
imperat.
\lnr{26}Quam enim lubricum sit retro ab aliqua epocha notare tempora,
quod maior pars doctorum virorum facit, satis nos usus docuit.
\lnr{28}His ita positis, ad singula huius operis membra venio.

%% == Libro primo
\lnr{28}Libro primo
praeter divisionem temporum, et iucundissimam mensium, et
annorum historiam, de antiquissima anni forma disputatur, quae in
menses aequabiles annum describit, qua pleraque omnes Graecia usa
est, et ab ea omnis ratio Olympiadum pendet: nisi potius eam e ratione
Olympiadum propagatam dicas: quod sine cognitione Olympiadum
numquam tam eximium vetustatis et temporum monimentum
in lucem eruissemus. 
\lnr{35}Ex tanta autem Graecorum scriptorum
copia unicus Pindarus nobis facem alluxit, qui solus nos docuit tempus
ludicri Olympici.
\lnr{37}Aliter, quae paucitas est bonorum scriptorum,
nulla erat via ad haec interiora perveniendi.
\lnr{38}Huius anni Graeci
formae doctrina tanto acceptior esse debet, quanto obscurior eius
rei apud maiores nostros scientia fuit: cum ante hos mille quadringentos
plus minus annos eius rei neque volam, neque vestigium
vetustas retinuerit.
%\end{parnumbers}
%%\clearpage
\pnr{p. III [pdf 30]}
%\begin{parnumbers}
\lnr{42}Nam falso veteres multi, ac post eos infamae antiquitatis
scriptores, Macrobius ac Solinus, atque proavorum memoria
summus vir Theodorus Gaza, annum Graecorum statim
ab initio merum Lunarem fuisse prodiderunt.

\plnr{III}{3}Quamuis enim in
Panegyribus suis, ac nobilioribus sacris, quae certo annorum circuitu
redibant, unius Lunae rationem habebant, tamen, ut uno verbo
dicam, eorum anni forma Lunaris non erat.
\lnr{6}Olympicum enim ludicrum
ipsa Lunae plena lampade celebrabatur, ut solus veterum nos
docet Pindarus. 
\lnr{8}Praetera Laconibus ante plenilunium, aut novilunium
aliquid incipere religio erat.
\lnr{9}Unde \textgreek{Λαχώνιχας ςελωοαζ}[?] vicinorum
proverbio iactatas, et contra Arcadibus proverbiali convicio
neglectum religionis obiectum legimus. 
\lnr{11}Quod enim ante novilunium,
aut plenilunium ut plurimim bella aut alia seriora aggrederentur,
ob eam rem a finitimis nationibus \textgreek{[Greek]} vocabantur:
quae convicii caussa ab ipsis Arcadibus interpretatione elusa est,
probrum in laudem conversum ad vetustatem originis suae referentibus,
et antiquiorem sidere gentem suam gloriantibus. 
\lnr{16}Quod igitur
novilunii ac plenilunii tempora Panegyricis ludicris deligebant,
propterea sacra trieterica instituta: cuiusmodi erant orgia Bacchi,
Nemea, Isthmia, alia.
\lnr{19}Ea enim est anni Graecanici forma, ut si, verbi
gratia, novilunium in neomeniam Gamelionis incurrat, plenilunium
in eandem neomeniam incidat anno tertio redeunte.
\lnr{21}Itaque
cum in Tetraeteride orgia Bacchi trieterica celebrabantur, tertio
anno redibant in eum sistum Lunae, qui priorum orgiorum situi oppositus
erat.
\lnr{24}Quare elegantissime Statius trieterida vocat alternam:
quia alternis in novilunium, et plenilunium incurreret.
\lnr{25}At sacra,
quae necessario eodem Lunae tempore obibantur, ea semper erant
tetraeterica: ut in Attica Panathenaica maiuscula, in Elide Olympias,
ut iam tetigimus, plenilunio.
\lnr{}Quod sane fieri non poterat, nisi absoluta
Tetraeteride, et Pentaeteride ineunte.
\lnr{29}Atque ita Tetraeterides
in idem \textgreek{χῆμα} Lunae, non utique in idem tempus Solis redibant.
\lnr{30}Ut
enim in orbem Solis et Lunae redirent, non aliter putabant fieri,
quam octaeteride confecta, eneaeteride ineunte.
\lnr{32}Ex quo quaedam
eneaeterica sacra eo nomine instituta: cuiusmodi ab initio Pythia
fuerunt: et quidem merito.
\lnr{34}Apollini enim, quem eundem cum Sole
faciebant, erant attributa.
\lnr{35}Hinc colligimus, non solum Olymiadis
intervallum annis quatuor solidis explicatum fuisse; sed etiam puerliter
peccare eos, qui annorum quinque solidorum fuisse putant.
\lnr{38}Neque vero quibusdam recentioribus succensendum, qui ita censent,
ita scribunt, sed et Ausonium nostratem culpa liberat Ovidius, scriptor
longe antiquior, et nobilior, qui aetatem suam quinquaginta annorum
decem Olympiadibus definit: quo magis mirum Pausaniam
hominem Graecum in ea haeresi fuisse, ut suo loco a nobis relatum est.
%\end{parnumbers}
%\clearpage
\pnr{p. IV [pdf 31]}
%\begin{parnumbers}
\plnr{IV}{1}Nam minus mirandum de Solino, qui cap. \textsc{xiii} Isthmia vocat
quinquennalia, quae erant tantum triennalia, quod certamen a Cypselo
tyranno intermissum, anno primo Olympiadis 49 instauratum
fuisse dicit.
\lnr{4}Horum igitur omnium caussae ad typum anni Graeci referendae
sunt.
\lnr{5}In quo argumento nihil eorum praetermisimus, quae
ei rei illustrandae faciebant, quanquam pene omnibus praesidiis
destituti.
\lnr{7}Et quidem primum in genere, quod semper solemus, deinde
privatim multarum Graeciae nationum periodos proposuimus,
quae quidem non anni forma, sed situ et capite inter se differunt: in
qua tractatione diu nobis res fuit cum praestantissimo viro Theodoro
Gaza, vel potius cum eius sequacibus, a quibus extorqueri non
potest doctrina et situs mensium, ab illo primum proditus. 
\lnr{12}Quae quidem
velitatio nobilioribus ingeniis, et ab omni invidia remotis, ut
spero, iucunda erit.
\lnr{14}Quid enim est toto libro primo, cuius vel minima
pars, non dicam istis querulis, qui nihil sciunt, sed etiam doctioribus,
hoc saeculo, et ante multa retro saecula oboluerit?
\lnr{16}Quid dicam \textgreek{[Greek]}?
\lnr{17}Quis illarum caussas, et usum sciebat?
\lnr{17}Quis
locum nobilem de illis in Verrina Ciceronis intelligebat?
\lnr{18}Quis locum
\textgreek{Εξαιρέσεως[?]} in secunda Boedromionis?
\lnr{19}Quis Posideonem intercalarem
mensem fuisse?
\lnr{20}Huic materiae accessit \textgreek{ἐποχη χέντςα[?] θερινᾶ}
in \textsc{viii} Iulii, quae in priori editione omissa erat.
\lnr{21}Id erat \textgreek{χάντσον[?]}
populare, quod nomine \textgreek{τςοπων[?] θερινῶν[?]}
 Aristoteles, Theophrastus,
Plutarchus, et omnes veteres intelligunt, non autem ipsum verum
Solstitium: quae rei pulcherrimae notatio nobis viam ad illustriora
praeiuit.
\lnr{25}Quod Solstitiorum, et Aequinoctiorum puncta \textgreek{χέντρα} vocentur,
satis sciunt, qui veterum Graecorum libros legerunt.
\lnr{26}Columella
cardines vocavit.
\lnr{27}In praestantissimo Parapegmate, quod falso
Ptolemaeo attribuitor (est enim antiquius Ptolemaeo) ad \textsc{viii} Kal.
Iulii (quod est Solstitium Sosigenis) annotatum est: \textit{Aestivus cardo,
et momentanea aeris perturbatio}.
\lnr{30}In Graeco (utinam haberemus!)
sine dubio fuit: \textgreek{Θερινὸν χέντρον, χαὶ ςιγμιαία αἔσος Ιασοιχή[?]}.
\lnr{31}Igitur \textgreek{χέντρον
θερινὸν} nihil aliud, quam \textgreek{τροπαὶ θεριναί}.
\lnr{32}Cur \textsc{viii} dies Iulii erat
epocha aestiva in usu civilis anni, non semel caussam reddidimus. 
\lnr{34}Adiecta etiam pernecessaria neomeniarum Atticarum Tabula: quae
non solum priori editioni, sed etiam doctrinae anni Attici deerat.

%% == Liber secundis
\lnr{36}Liber secundus anno Lunari dicatus est ideo, quia is annus ex illo
Graeco aequabili manasse videtur.
\lnr{37}Ibi aperitur omnis antiquitas \textgreek{ἔτοις[?]
πρυτανείας[?]}, Octaeteridum Cleostrati, Harpali, et Eudoxi: quae omnia
hodie nomine tenus nota erant.
\lnr{39}Eudoxea Octaeteris numquam
in usus civiles admissa est.
\lnr{40}Anni vero \textgreek{πρυτανείας[?]} in vetustissimis Psephismasin
Atheniensium primo quidem ex Cleostratea, deinde, illa
abrogata, ex Harpalea petiti sunt.
%\end{parnumbers}
%\clearpage
\pnr{p. V [pdf 32]}
%\begin{parnumbers}
\lnr{42}Sequitur magnus annus Metonicus
ambarum, et Calippicus Metonici castigator.

\plnr{V}{1}Et quidem hi
ambo nomine noti tantum: caussarum autem, et omnium, quae ad
illa pertinent, mira ignoratio hactenus fuit.
\lnr{3}Accesserunt huic editioni
Tabulae operosissimae dispensationum neomeniarum Metonicarum,
et Calippicarum: cuiusmodi etiam in Harpalea Octaeteride
exhibuimus. 
\lnr{6}Quod de Eudoxea Octaeteride diximus, idem de
periodo Chaldaeorum dicendum, eam nunquam ad civilia tempora,
sed ad Genethliacorum themata usurpatam fuisse.
\lnr{8}Id quod tum
multa argumenta, tum unicum certissimum illud est, quod eorum
menses appellationibus Macedonicis, non vero Chaldaicis fuerunt.
\lnr{11}Propterea recte cum illius anni diatriba doctrinam dodecaeteridis
Chaldaicae Genethliacorum coniunximus, cuius nomen quidem
solum notum erat ex Censorino: cognitio autem nobis ex Arabum,
et Orentalium usu repetenda fuit.
\lnr{14}An aliquis Graecorum \textgreek{δωδεχαετρίδος Χαλδαιχῆς}
meminerit, haud promptum est dicere.
\lnr{15}Unum tantum Orpheum sive Onomacritum eius meminisse scimus. 
\textgreek{ὀρφδὶςὀν[?] ταῖς[?] δωδεχαετνρίσιν[?]:}
\begin{quote}
\begin{greek}
\lnr{18}ἔςαι δ᾽ αὖθις ἀνὴρ, ἢ χοίραν[G-circ], ἠὲ τύρανν[G-circ],

\lnr{19}ἢ βαοιλδὶς, ὂς τῆμ[G-circ] ἐς οὐρανὸν ἴξε[right curl] αἰπιιύ [all doubtful].
\end{greek}
\end{quote}
\lnr{20}Est apotelesma cuiusdam Genethliaci consulti super alicuius genesi,
de quo ipse respondit, eum fore magnum regem aut Dynastam, etcetera.
\lnr{22}Citat Tzetzes. 
\lnr{22}Haec multum illustrant doctrinam Dodecaeteridos
genethliacae parum antehac notae.
\lnr{23}Itaque quemadmodum \textgreek{τελετὰς}
ita etiam \textgreek{δωδεκαετνρίδας[?]} scripserat Onomacritus sub nomine
Orphei.
\lnr{25}Qualis fuerit Iudaeorum annus sub Seleucidis, quibus parebant,
multis exemplis testatum reliquimus: in quibus etiam translationis
feriarum in capite anni antiquitatem asseruimus adversus homines
nostrorum temporum, qui nugantur commentum nuperum
Iudaeorum esse.
\lnr{29}In illis Doctor Theologus ingenti commentario
suo in Evangelium secundum Iohannem exultabundus ait illam
translationem confutari ex loco Iosephi, in quo scribit, quo anno
Hyrcanus foedus icit cum Antiocho Sidete, Pentecosten fuisse feria
prima.
\lnr{33}Hunc locum Iosephi nos olim in priore editione produximus,
unde is, aut qui illi indicavit, accepit.
\lnr{34}En, inquit, duo Sabbata continua.
\lnr{35}Si propter continuationem duorum Sabbatorum, feria transfertur,
ergo ubi sunt duo continua Sabbata, non transfertur. 
\lnr{36}In quibus
aperte ostendit se ignorare caussam feriae transferendae, quae fiebat
propter solum Tisri, non autem propter alios menses; propterea
quod ille mesis multa solennia habet, adeo ut si non habeatur
ratio translationis, aliquando non solum duo, sed entiam tria continua
sabbata concurrere necesse sit.
%\end{parnumbers}
%\clearpage
\pnr{p. VI [pdf 33]}
%\begin{parnumbers}
\lnr{41}Si enim feria sexta inciperet
neomenia Tisri, omnino tria sabbata continuarentur, neomenia,
sive clangor tubae, sabbatum ordinarium, et ieiunium Godoliae.

\plnr{VI}{2}Continuantur autem saepernumero in aliquo reliquorum mensium
duo Sabbata: idque fit, quando solenne est aut feria prima, aut feria
sexta.
\lnr{4}Quorum alterutrum quotannis incidere, nisi quando Tisri
incipit feria tertia, Doctor ignoravit.
\lnr{5}In primam feriam incidunt
haec solennia, \textsc{xxv} Casleu, et \textsc{x} Tebeth in anno defectivo tam
communi, quam embolimaeao, quotiescunque Tisri incipit feria secunda:
\textsc{vi} Sivvan; quando Nisan incipit feria septima:
\textsc{xv} Nisan, \textsc{xvii} Tamuz,
\textsc{ix} Ab, quando Nisan incipit feria prima.
\lnr{9}In feriam autem sextam
convenit solenne \textsc{xxv} Casleu et
 \textsc{x} Tebeth, quando Tisri est feria
septima in anno tam communi, quam embolimaeo.
\textsc{xiiii} Adar, quando
Nisan sequens est feria prima: \textsc{vi} Sivvan, quando Nisan feria quinta.
\lnr{13}Vides, quot Sabbata quotannis, nisi quando Tisri incipit
 feria tertia, Iudaei
continuent in aliquo mensium, praeterquam in solo Tisri, cuius
unius gratia illa cautio instituta
% No period at end of sentence
\lnr{15}Itaque doctor tam frustra, quam ridicule
Iosephi testimonium adduxit de sexta Sivvan, id est, Pentecoste
feria prima; cum illo anno neomenia Nisan fuerit Sabbatum.
\lnr{17}Atqui
nihil superesse putavit, quam ut Vaticani montis imago redderet
\textgreek{ἰὴ παιαή[?]}.
\lnr{19}Sed ipse valde ignarus est harum rerum, ut reliqui omnes,
qui contendunt novitium esse Iudaeorum commentum.
\lnr{20}Nos
validissime demonstravimus, et saeculo Christi, et retro sub Seleucidis
translationes in usu fuisse.
\lnr{22}Et sane res peruetusta est.
\lnr{22}Quae tamen
non minus ignorata, quam periodus Calippica, qua Seleucidae, et
Seleucidarum edicto Iudaei usi.
\lnr{24}Quod non solum ex Nisan anni excidii
Hierosolymorum a nobis demonstratum est, sed etiam patet
ex definitione Rabbi Adda.
\lnr{26}Is annum definit dierum \textsc{ccclxv},
horarum 5, 997/1080. 48/76.
\lnr{27}Quid hac definitione aliud vult, quam periodum
Iudaicam fuisse annorum 76?
\lnr{28}Cum Meto definit annum dierum
365. hor. 5. 1/19. ex eo coniiciendum relinquit, se uti periodo annorum
19.
\lnr{30}Utebantur igitur periodo 76 annorum, id est, Calippica:
et tamen in omnibus neomeniis Lunae \textgreek{φάσιν} observabant, non
quod eam ex praescripto periodi non indicerent, sed ideo, ut eam
sanctificarent.
\lnr{33}Nam et hodie quoque observant \textgreek{φάσιν}, non ut ex ea
neomeniam indicant, sed ut eam sanctificent.
\lnr{34}Itaque Luna statim
visa dicunt: \texthebrew{[Hebrew]}.
\lnr{35}\textgreek{ἀγαθὸν τέρας ἔςω ἡμῖν ης[??] παντὶ Ισραήλ.}
\lnr{36}Idem faciunt et Muhammedani, quamuis neomenias ex
scripto indicere soleant.
\lnr{37}Neque aliud intellexit fabulosus quidem,
sed tatem vetus auctor \textgreek{[Greek]} apud Clementem:
\textgreek{[Greek][Lots of Greek]}.
%\end{parnumbers}
%\clearpage
\pnr{p. VII [pdf 34]}
%\begin{parnumbers}
\lnr{42}Praeclara quidem ista: sed nescit, quid dicit.

\plnr{VII}{1}Nam in Iudaeorum potestate
nunquam fuit, ut exspectarent \textgreek{φάσιν}:
% Greek: phase
 quia raro Luna se ostendit,
nisi secundo post coitum die.
\lnr{}Quod si expectandum ipsis esset,
res ridicula accideret, ut Elul, qui semper est cavus mensis, non solum
plenus, sed etiam aliqando unius et triginta dierum esset.
\lnr{}Sine dubio translationem feriae intelligit, cuius caussam ignorat.

\textgreek{πρῶτον σάββατον} vocat
 \texthebrew{רֹאשׁ הַשָּׁנָה‎} (Rosh Hashanah) caput anni.
\lnr{}Nam Sabbatum vocat, quia Festus
dies, \textgreek{κὶα᾽εργός}[?].
\lnr{}Ita etiam vocatur Levitici \textsc{xxiii}, 24.
% Leviticus 23:24: “Tell the people of Isra’el, ‘In the seventh month, the
% first of the month is to be for you a day of complete rest for remembering,
% a holy convocation announced with blasts on the shofar.'"
% λάλησον [Speak] τοις [to the] υιοίς [sons] Ισραήλ [of Israel,] λέγων [saying!]
% του [The] μηνός [(²month] του [] εβδόμου [¹seventh),] μία [day one]
% του [of the] μηνός [month] έσται [will be] υμίν [to you] ανάπαυσις [a rest,]
% μνημόσυνον [a memorial] σαλπίγγων [of trumpets,] κλητή [(²convocation]
% αγία [¹a holy)] τω [to the] κύριος [LORD.]

\textgreek{ἑορτὴν}[?]
% feast
 intellige
\textgreek{κατ᾽ ἐξοχήν τὴν πεντηκοστήν}:
% eminently the Pentekost [?]
 quod ita Hebraice vocetur, nempe \texthebrew{עֲצֶרֶת}[?] ([sh'miní] 'atséret).
% "The eighth [day] of assembly".
 
Vide in Computo Iudaico.
\lnr{}At \textgreek{μεγάλην ἡμέραν}
% great day
 vocat \textgreek{τὴν σκηνοπηγιαν, κατ᾽ ἐξοχήν}
 quoque, id est \texthebrew{הַנ}[?].
\lnr{}Nam aliae erant \textgreek{μεγάλαι ἡμέραι},
% Great Days
proinde ut et \texthebrew{חַנִּים}[?].
\lnr{}Sic Tertullianus magnos dies dixit, quos
Hebraei \texthebrew{[Hebrew]} vel \texthebrew{[Hebrew]}.
\lnr{}Eius verba sunt ex v in Marcionem:
\textit{Dies observatis, et menses, et tempora, et annos, et Sabbata, ut opinor,
et cenas puras, et ieiunia, et dies magnos.}
% Tertullianus: De Adversus Marcionem, Book 5, chapter 4, section 6.
\lnr{}Sed quid Tertullianum
advoco?
\lnr{}Ecce Biblia Graeca ita vertunt ex primo caput Isaiae:
\textgreek{τὰς νουμἠνίας ὑμῶν, καὶ τὰ σάββατα,
 καὶ ἡμέραν μεγάλην οὐκ ἀνέχομαι}[?].
% Isaiah 1:13 ?: I cannot bear your new moons, and your sabbaths,
 and the great day;
\lnr{}Quod Hebraice est \texthebrew{[Hebrew]}, vertunt \textgreek{[Greek]}, quod idem
est quod \texthebrew{[Hebrew]}: et quidem manifesto Sabbata distinguuntur a
magnis diebus. 
\lnr{}Quare perperam quidam \textgreek{[Greek]} interpretantur
Sabbatum apud Iohannem, \textgreek{[Greek]}. de quo infra.
\lnr{}Quin et Tertullianus ipse \textgreek{[Greek]},
quas cenas puras vocat, a diebus magnis, et a ieiuniis, et a
Sabbatis distinguit.
\lnr{}De Cena pura, praeter id quod diximus ad
Festum, ita reperi in veteri et peroptimo Glossario Latinoarabico:
\textit{Parasceue, cena pura, id est, praparatio, que fit prosabbato.}
\lnr{}Conditor Annalium Ecclesiasticorum turbat de cena
pura, et negat esse parasceuen, quia cena pura apud Festum
habeat offam suillam.
\lnr{}Sed ipse, (pace docti viri dixerim) non
aduertit Puram dici, non quia careat carnibus, sed quia religionis
et dicis caussa fit.
\lnr{}Nam et parasceuae Iudaicae habent carnes,
et nihilominus dicuntur cenae purae, quod dicis caussa coquebantur,
coquunturque hodie prosabbato, quia in Sabbato
coqui non liceat.
\lnr{}Non negabis, candide Lector, haec vulgo non intelligi.
\lnr{}Itaque locus ille est nobilissimus. 
\lnr{}Tamen quotus quisque est ex tot Lectoribus, qui non haec aut praeteribit,
aut calumniabitur?
\lnr{}Sequuntur periodi magnae Hagerenorum,
ex quibus ratio anni soluti Indorum, et Muhammedanorum
tota pendet.
\lnr{}Omnia nunc primum ex Arabum scriptis
prodeunt: atque adeo omnis tractatio nostris hominibus
nova est.

%\end{parnumbers}
%\clearpage
\pnr{p. VIII [pdf 35]}
%\begin{parnumbers}
Excipit hanc doctrina anni Iudaici hodierni, res, quod
saepe diximus, artificiosissima, ideoque eximia, quia melior
anni Lunaris forma constitui non potest.
\lnr{}Docemus praeterea, unde
natus sit ille annorum computus, quo utuntur hodie, a \textsc{vii} Octobris:
quem inepte putant a conditu rerum.
\lnr{}Post multarum Periodorum,
Cyclorum, Octaeteridum, Paschalium historias, in locum vltimum
\textgreek{[Greek]} veteris anni Romanorum coniecimus, ideo
quod ea forma proxime abesset a Lunari: ubi de saeculo Romano,
et capite veteris anni Romani, temporibus vltimis C. Iulii Caesaris,
multa accuratissime disputata.
\lnr{}Itaque ex singulis rebus singula capita
confecimus, cum potius singuli libri et quidem ingentes confieri
possent, si, quae hominum hodiernorum est ambitio, eadem nobis
incessisset.
\lnr{}Tertio libro opportune annus aequabilis datus est,
cum annus Solaris Aegyptiacus, adscitis diebus quinque, ex Graeco
propagatus sit: (quemadmodum annus Lunaris ex eodem Graeco
manauit, abiectis ab eo totidem diebus cum horis \textsc{xv}, paulo amplius)
quod, metacente, Plutarchus docuit in libro \textgreek{[Greek]}.
\lnr{}Adeo inter se libri nostri mutuo conspirant, neq; ab eis ratio,
methodus, et ordo abest.
\lnr{}In eo libro de Neuruz antiqui Persarum
periodo annorum \textsc{cxx}, deq; cognominibus dierum Persicorum,
de translatione \textgreek{[Greek]} in enthronismis nouorum Regnum,
item de caussis anni Iezdegird, de annis Armeniorum, et eorum
mensibus, omnia nova protulimus. 
\lnr{}Sed haec non expergefacient animos
hominum, nisi forte ad obtrectandum.
\lnr{}Quartus liber est emendatio
tertii, ut secundus primo erat subsidiarius: qua methodo imperfectus
Lunaris Graecus libro primo disputatur, ut perfectus secundo.
\lnr{}Sic etiam perfectus Solaris, et siqui alii naturam perfecti imitantur,
supplent id, quod aequabili Aegyptiaco, Persico, et Armeniaco
vetuitas detraxerat.
\lnr{}Itaq; in quatuor partes tribuendus fuit.
\lnr{}In
prima continentur anni, quibus quarto anno exeunte dies ex quatuor
quadrantibus conflatus accrescit.
\lnr{}Ex illis nobiliores selegimus,
Iulianum, Actiacum, Antiochenum, Samaritanum, et alios. 
\lnr{}Nam et alios quoq; eius formae habebamus, ut Tyriorum, quorum menses
appelationibus Macedonicis, diuersa initia a Iulianis habent.
% "appelationb." interpreted as abbriviation for "appelationibus"
\lnr{}Sic
etiam Gazensium annus mere Actiacus fuit, appellationibus mensium 
Macedonicis, mensibus tricenariis. 
\lnr{}Marcus Ecclesiae Gazensis Diaconus,
in actis Porphyrii Gazensis Episcopi vocat \textgreek{Διον} Nouembrem
\textgreek{Απελλαιον} Decembrem quae nomina habent a Macedonibus. 
\lnr{}Sed
idem scribit Gazenses celebrasse Theophaniorum diem undecima
Audynaei, quae est sexta Ianuarii Iuliani, se autē redisse Constantinopoli,
Xanthici vicesima tertia, quam ait fuisse decimam octauam
Aprilis secundum Romanos: quibus ostēdit formam illius anni mere
Actiacam fuisse, mensibus tricenariis, appellationibus Macedonicis. 

%\end{parnumbers}
%\clearpage
\pnr{p. IX [pdf 36]}
%\begin{parnumbers}
Secunda pars annos emendatos, eorumque emendandorum
rationem complectitur: tertia periodos multiplices, quarum finis
conciliatio anni civilis cum Solari, cui dies quinto quoque anno
ineunte accrescit.
\lnr{}Quarta pars agit de vera emendatione anni, et
de anno caelesti instituendo, qui pertinet ad methodum epochae
mundi.
\lnr{}Quemadmodum autem nulla Lunaris anni civilis ratio
recta iniri potest, praeter eam, qua Iudaei utuntur: ita nullus annus
caelestis Tropicus recte institui potest, nisi ex forma, quam edidimus,
quam nemo vituperabit, nisi qui ignorauerit; omnis laudabit,
qui intellexerit.
\lnr{}Alioquin scio et malignos et obtrectatores non defuturos. 
\lnr{}Annus tam noster, quam Iudaicus civilis quidem, sed naturalis,
vterq; ad motum quisq; sui sideris descriptus. 
\lnr{}Ideo eius saltem
in scriptis usus esse debet, qualis olim Philadelphi Dionysianus,
Chaldaeorum Calippicus, hodie Persarum Gelaleus. 
\lnr{}Tres igitur libri
primi, et prima pars quarti pertinent ad \textgreek{[Greek]} temporum civilium
cum septimo.
\lnr{}At reliquae tres partes quarti cum duobus libris
sequentibus pertinent ad ipsam emendationem temporum.
\lnr{}Atque
ut a mundi primordiis omnes res deducuntur, ita mundi epocham
primam ordine posuimus: qua in re quam pueriliter hallucinati sint
omnes, non sine admiratione tam imperitiae quam pertinaciae eorum
dicere possum.
\lnr{}Non loquor de iis, qui saeculo uno, aut pluribus altius
originem rei repetunt.
\lnr{}Nam quemadmodum ii nullam rationem
sibi proposuerunt, quam sequerentur, ita nullos lectores nancisci
possunt, nisi imperitos. 
\lnr{}Qui intra saeculum maiorem mundi
epocham faciunt, eorum duo genera reperio.
\lnr{}Prius genus est eorum,
qui solutionem captiuitatis in primum annum Olympiadis \textsc{lv} conferunt:
alterum eorum, qui tempus illud \textsc{xviii} aut \textsc{xix} annis ante
\textsc{lxiiii} Olympiadem definiunt.
\lnr{}In priori haeresi fuerunt et
quidam veterum Ecclesiasticorum, ut alicubi indicauimus. 
\lnr{}Aiunt
Cyrum caepisse imperare primo anno Olymiadis \textsc{lv}, hoc est 217
anno Iphiti, quod verum est: de quibus deductis septuaginta, relinquitur
annus excidii Hierosolymorum, et casus Sedekiae 147 a primo ludicro
Olympico.
\lnr{}Sed puerilis sentētia multis absurditatibus eluditur.
\lnr{}Primo computatione non recta annorum \textsc{lxx} a capto Sedekia.
\lnr{}Deinde quod Cyrum statim initio regni sui Regem Mediae, Persidos,
Susidos, Assyriae, Babyloniae, totius Asiae minoris, Indiae, totius
Syriae constituunt, qui unius Persidos Rex fuerit aliquot annis ante
casum Astyagis[?], et post illud tempus pauculis annis ante obitum
Babylone potitus sit.
% Astyagis or Aftyagis?
\lnr{}Haec sola absurditas facit, ut non solum eorum
nulla ratio habeatur, sed ut ludibrium quoq; debeant.
\lnr{}Tertio 147 annus
Iphiti est 118 Nabonassari: qui erat annus quintus ante initium Nabopollassari
patris Nabuchodonosori.

%\end{parnumbers}
%\clearpage
\pnr{p. X [pdf 37]}
%\begin{parnumbers}
Ergo Nabuchodonosor anno
decimono regni sui templum et Hiersolyma euertit annis quinq;
ante quam pater ipsius, cui ipse successit, regnaret.
\lnr{}Digna profecto
talibus doctoribus sententia.
\lnr{}Tamen tantum abest, ut hac tam insigni
absurditate a sententia desistant, ut animos ab eiusmodi portentis
opinionum sumant.
\lnr{}Postremo ignorant diuersa esse initia Regnum,
ut ipsius Nabuchodonosori, cum patre, et solius: Alexandri,
ab excessu Philippi patris, et ab initio Seleuci: Diocletiani, ab aera
martyrum, et a primo anno imperii.
\lnr{}Sic etiam Cyri, apud Graecos, ab
initio regni Persidis: apud Babylonios, vel a subacto toto Babyloniae
imperio, vel ab aliquo insigni facto, quodcunque illud fuerit, sive
ex edicto ipsius Cyri, sive translatione \textgreek{πδν ἐπα γο υ[?]ων},
ut solebat fieri.
\lnr{}Qui tantam inscitiam sequi noluerunt, non tamen rectam viam
institerunt, quia quindecim aut amplius annis ante \textsc{xlvi} Olymiadem
casum Sedekiae coniiciunt.
\lnr{}Nos ante annum quartum illius
Olympiadis id non potuisse accidere ita demonstramus. 
\lnr{}Ezekias
Rex Iuda, postquam singulari Dei beneficio ab ancipiti morbo conualuisset,
anno \textsc{xiiii} regni sui, accepit Legatos et \textgreek{ςωτήρια[?]}, a
Merodach Rege Chaldaeorum.
\lnr{}Ponamus \textsc{xiiii} annum Ezekiae in
primo anno Merodach, hoc est, in \textsc{xxvii} Nabonassari.
\lnr{}Nam is est
annus primus Merodach apud Ptolemaeum ex Chaldaicis obseruationibus. 
\lnr{}Hoc modo annus primus Ezekiae conuenerit in annum
\textsc{xiiii} Nabonassari. Ab initio Ezekiae, ad excidium templi, anni
sunt absoluti 138.
\lnr{}Hoc est, annus ipsius excidii est 139 labens ab initio
Ezekiae.
\lnr{}quod ita demonstramus. 
\lnr{}Annus primus Sedekiae est
quartus Hebdomadis, teste Ierimia, initio cap. \textsc{xxviii}: et proinde
undecimus, qui et vltimus, est Sabbaticus. 
\lnr{}de quo extat testimonium
apud Ieremiam, et nemo dubitat.
\lnr{}Rursus annus tertius decimus
Ezekiae erat Sabbaticus. 
\lnr{}auctor Isaias \textsc{xxxvii}, 30.
\lnr{}ex quibus
manifesto colligitur, \textsc{xiiii} Ezekiae esse primum Hebdomadis, et primum
Ezekiae esse sextum Hebdomadis. 
\lnr{}Ergo annis ab initio Ezekiae unitas addenda, ad methodum anni Sabbatici.
\lnr{}Addita unitate annis
139, numerus erit septenarius. 
\lnr{}Quare annus labens 139 est verus
annus ab initio Ezekiae.
\lnr{}Quibus additis 13 annis Nabonassari praeteris
(quia posuimus 14 Nabonassari primum Ezekiae) componitur
annus Nabonassari 152, in quo casus Sedekiae ex hac hypothesi
locandus est, hoc est, in anno periodi Iulianę 4118: de quibus deductis
907 absolutis ab Exodo, remanet annus Exodi 3211 in periodo Iuliana.
\lnr{}Porro Nisan Exodi caepit feria quinta, ut toties diximus, et ex
Mose rectissime ante nos Iudęi docuerunt.
\lnr{}At in anno periodi Iulianę
3211 Nisan non caepit feria quinta, sed feria tertia,
 Martii \textsc{xi}, cyclo tam
Solis, quam Lunae \textsc{xix}.

%\end{parnumbers}
%\clearpage
\pnr{p. XI [pdf 38]}
%\begin{parnumbers}
Ergo annus proximus, quo Nisan caepit feria
quinta, is debuit saltē esse annus Exodi: atq; adeo is fuerit annus periodi
Iulianae 3214: in quo sane nisan caepit feria quinta, Aprilis \textsc{vi},
cyclo Solis \textsc{xxii}, Lunae \textsc{iii}.
\lnr{}Additis 907 annis absolutis ab Exodo,
annus 4121 periodi Iulianae suerit is, in quo excidium templi contigit:
qui est quartus Olympiadis 46, ut erat propositum.
\lnr{}Sed et post
Olympiadem 46 ponendum esse casum Sedekiae ita probabimus.
\lnr{}Amasis rex Aegypti, postquam regnasset annos 55, obiit circiter annum
7 Cambysis, anno ante excessum ipsius Cambysis, hoc est, anno
225  Nabonassari.
\lnr{}Nechao intersectus est a Nabuchodonosoro anno
quarto Ioiakim regis Iuda.
\lnr{}Ieremias \textsc{xlvi}, 2.
\lnr{}Post eum regnauit
Psammitichus annos \textsc{vi}.
\lnr{}Cui Aprias, cuius meminit Ieremias
\textsc{xliiii}, 30, succedit.
\lnr{}Is post \textsc{xxv} annos relinquit regnum Amasi.
\lnr{}Summa annorum a caede Nechao ad obitum Amasis anni 86, qui
deducti de 225, relinquunt annum Nabonassari 139, quartum Ioiakim
Regis Iudae, primum Nabuchodonosori.
\lnr{}Ergo Sedekias captus
anno 158 Nabonassari, qui erat tertius 47 Olympiadis.
\lnr{}Idque verum
esse postea validissime demonstrabimus.
\lnr{}Diodorus Siculus,
auctor omnium Graecorum certissimus, attribuit, \textsc{lv} annos Amasidi.
reliquos Apriae et Psammatichi habemus ex Herodoto.
\lnr{}Temere
igitur, et imperite faciunt, qui casum Sedekiae antiquiorem illo
tempore constituunt: neque his cassibus sese explicare poterunt,
quantumuis sua commoueant sacra, ut Plautus loquitur.
\lnr{}His valide
demonstratis, et licentia chronologorum intra aliquos fines summota
quos amplius migrare non possunt, ad originas ipsas penetremus.
\lnr{}Sed prius ut in Mathematicis concessa quaedam, aut quae negari
non possunt, assumuntur, ita et nobis quoque faciendum.
\lnr{}Tempora et initia Regum Babyloniae a Chaldaeis notata in obseruationibus
eclipticis, quae reiicere et damnare extremae impudentiae et
inscitiae est: item, eorundem regum initia et tempora a Beroso Chaldaeo,
qui minus quam tribus saeculis post illos vixit, et qui quae Actis
ac fastis Babyloniorum publicis continebantur, ignorare non potuit,
haec inquam, non tantum tāquam vera haberi postulamus, sed etiam
qui aliter putant, tanquam indignos censeri, qui aut audiri a nobis
mereantur, aut vllas literas attingant, aut aliquem locum inter
doctos habeant.
\lnr{}Tricesimum annum, cuius initio Prophatiae suae
meminit Ezekiel, quique capti Iechoniae quintus erat, Iudaei inepti
deducunt a libro legis reperto, anno \textsc{xviii} Iosiae Regis.
\lnr{}Quis unquam a libro reperto vllam aeram, aut edicto Iosiae institutam, aut a
Prophetis usurpatam legit?
\lnr{}Si tanti erat illa temporis nota, quare
eam non usurpat Ieremias, qui tam accurate annos Regum Iuda Iosiae,
Ioiakim, Iechoniae, Sedekiae notare solet?
\lnr{}Capite \textsc{xxv}, quare dicit
anno quarto Ioiakim, cum dicendum esset vicesimo secundo a libro
inuento?
\lnr{}Esto, cur Ezekiel dixit tricesimo, non tricesimo a libro inuento?
\lnr{}qui tamen dixit anno quinto deportationis Regis Ioachin.

%\end{parnumbers}
%\clearpage
\pnr{p. XII [pdf 39]}
%\begin{parnumbers}
Certe mos est uti epocha, quae omnibus et nota et in usu sut.
\lnr{}Quare
igitur epocham producit, neque plebi notam, neque in usu positam?
\lnr{}Sed quid ea epocha opus in Babylonia, inter deportatos?
\lnr{}Nugae Iudaeorum,
nugae sunt istae, et halluciationes doctorum, qui eos sequuntur.
\lnr{}Quare eruditiores Iudaeorum, huius absurditatis et nugatoriae
caussae conscii, his ineptiis explosis, dicunt, illum annum non a
libro inuento, sed Iubilei fuisse tricesimum.
\lnr{}At hoc est litem lite decidere.
\lnr{}Nam, quomodo Iudaei annos a Iubileo putarent, qui Iubilea
numquam usurparunt?
\lnr{}Annos quidem Hebdomadis notant, utinitio
\textsc{xxviii} Ieremiae mentio anni quarti septimanae: \textit{Initio regni
Sedekia, anno quarto.}
\lnr{}Rursus mentio anni primi, et secundi in annis
\textsc{xiiii}, et \textsc{xv} Ezekiae, apud Isaiam \textsc{xxxvii}, 30.
\lnr{}Sed notationem
per Iubilea, imo ne Iubilei quidem mentionem, nusquam, nisi
in lege, reperies.
\lnr{}Praecepta fuit tantum, non recepta Iubilei obseruatio.
\lnr{}Sed quae haec plumbea Iudaeorum sententia a \textsc{xviii} Iosiae
Iubileum putare?
\lnr{}Iubilea putantur a primo anno hebdomadis, non
a septimo.
\lnr{}At \textsc{xviii} Iosiae suit septimus septimanae, non primus.
\lnr{}Quare, si a Iubileis annos putare mos esset, suerit hic annus non utique
tricesimus, sed undetricesimus Iubilei, a \textsc{xviiii}, non a
\textsc{xviii} Iosiae.
\lnr{}Denique is erat annus 862 ab excessu Mosis, 855 a
diuisione terrae sive \textgreek{[Greek]}.
\lnr{}Ergo suit vicesimus secundus, non
undetricesimus Iubilei.
\lnr{}En quot errores locus praepostere sumptus
nobis peperit.
\lnr{}Cum igitur neque a libro legis inuento, quod est absurdissimum,
neque a Iubileo, quod est falsum dupliciter, ille tricesimus
annus putandus sit; sequitur, quod negari non potest, a
quodam rege tunc imperante putandum esse.
\lnr{}Nam deportati et captiui
inter victores, qua epocha uti possunt, nisi victoris?
\lnr{}In Palaestina,
cum aliqua esset Iudaeorum Respublica, et Ecclesia bene constituta,
Iudae cogebantur uti anno Alexandreo dominorum Seleucidarum:
quanto magis Chaldaeorum, in media Chaldaea, nullis legibus,
nulla Republica, nulla Ecclesia.
\lnr{}Nehemias initio libri sui ita
scribit: \textit{Accidit mense Casleu, anno vicesimo, cùm eßem in castro Susan.}
\lnr{}Si alibi non expressisset se de vicesimo anno Artaxerxis loqui, haud
dubie aliquod Iubileum hic commenti essent inepti Iudaei, et inepti
quidam hominum nostrorum sequuti essent.
\lnr{}Eodem quoq; modo
loquitur Ezekiel \textit{anno tricesimo}, non adiecto regis nomine.
\lnr{}Quid enim opus erat in Chaldaea?
\lnr{}Duo ergo Reges simul imperabant,
Nabuchodonosor, et ille, qui iam tricesimum annum currentem
imperabat.
\lnr{}Quisnam Rex, obsecro, potuit trecesimum annum in regno
agere, cum iam Nabuchodonosor tertium decimum regnaret?
\lnr{}Non alius igitur suerit, praeter Nabopollassarum patrem Nabuchodonosori,
quod verum est.

%\end{parnumbers}
%\clearpage
\pnr{p. XIII [pdf 40]}
%\begin{parnumbers}
Nam \textsc{xxix} solidos annos imperauit,
teste Beroso.
\lnr{}Quod si filius eius anno \textsc{xxx} partis iam duodecimum
absoluerat, profecto imperare caeperit anno partis decimooctauo,
qui erat Nabonassari 140.
\lnr{}Nam primus Nabopollassari est 123 Nabonassari,
testibus Chaldaeis apud Ptolemaeum, ex defectibus Lunaribus
obseruatis.
\lnr{}Et proinde Sedekias captus fuit anno 158 Nabonassari,
tertio autem Olympiadis 47.
\lnr{}Vide locum Berosi apud Iosephum.
\lnr{}Nabopollassarus audita rebellione Aegypti misit filium eo
cum regio imperio, et regio exercitu: a quo tempore consurgit initium
Nabuchodonosori cum patre regnantis.
\lnr{}Mos erat Regum Babyloniae
et Persidis, ut aut prosecturi in expeditionem, filios reges declarent,
aut in expeditionem mitterent cum regio nomine, tanquam
designatos, si contigisset ipsum patrem mori, absente filio, ne
vllus de rege futuro tumultus oriretur.
\lnr{}Exemplum habemus apud
Herodotum de Cyro Cambysen in solium suum collocante in expeditione
in Scythas.
\lnr{}Hinc Ctesias Cambysi attribuit annos 18,
cum tamen solus regnarit octo annos, testibus omnibus veteribus
Graecis, et Chaldaeis ipsis apud Ptolemaeum.
\lnr{}Dario vero Notho annos
idem attribuit 35, cum tantum 19 solus imperarit.
\lnr{}Rursus Berosus
\textsc{xxxxiii} annos ait Nabuchodonosorum imperasse, comprehensis
nimirum 13 annis, quos cum patre communicauit, cum
illis quos solus in imperio transegit.
\lnr{}Quare Nabuchodonosori regnum
dixit non Satrapian, tanquam a patre non ut Satrapes, sed Rex
et socius imperii in rebelles missus.
\lnr{}Verba eius sunt haec: \textgreek{[Greek]}.

\textit{Victo rebelli, eius regionem regno suo subiecit.}
\lnr{}Mox subiicit, Nabopollassari patris morto[?]
audita, qui \textsc{xxix} annos solidos regnauerat, ipsum Babylonem se
contulisse: quod accidit proculdubio aliquot diebus post illud tēpus
ab Ezekiele designatum.
\lnr{}Obiit enim Nabopollassarus anno regni
sui \textsc{xxx}.

\textgreek{[Greek]}.
\lnr{}Pulcherrima haec est obseruatio, quam Beroso vernaculo
Babylonicarum rerum scriptori debemus.
\lnr{}Eadem verba repetit Eusebius
De pręparatione euangelica, ubi plane \textgreek{[Greek]}, quemadmodum
est apud Ptolemęum, nominat, non \textgreek{[Greek]}, ut perperam
est editum in Iosepho: ex quo ineptus quidam duos esse coniecit
Nabulassarum et Nabopollassarum; cum tamen eadē verba sint, ne
una quidem syllaba minus, praeter illud nomen.
\lnr{}Rursus apud Iosephum
lib.\textsc{x} ca.\textsc{ii}.eadem verba Berosi repetuntur.
\lnr{}Sed ubi hic est \textgreek{[Greek]},
ibi est bis \textgreek{[Greek]}, utrobique male pro \textgreek{[Greek]}.
\lnr{}Quam bene haec diuinis scripturis conueniunt?

%\end{parnumbers}
%\clearpage
\pnr{p. XIV [pdf 41]}
%\begin{parnumbers}
Vnde etiam sequitur, mortuo Nabopollassaro, non tricesimum annum Nabuchodonosori
dici caeptum in Chaldaea, sed primum quae res obseruatione
digna.
\lnr{}Iudaei primum annum putarunt ab eo tempore, quo
cum imperio missus est. Sed in Chaldaea primus eius annus consurgit
ab obitu patris.
\lnr{}Itaque Danielis 11, annus secundus Nabuchodonosori
est sine dubio secundus ab obitu Nabopollassari, tricesimus
primus ab initio eiusdem, 152 ab initio Nabonassari, sextus
Sedekiae.
\lnr{}Vnde indubitata eruintur temporis nota illius Capitis secundi
apud Danielem, qui erat quartusdecimus nnus capti Danielis,
et sociorum cum rege Ioiakim, sextus autem regni Sedekiae.
\lnr{}Proinde annus ille erat \textsc{xiiii} Nabuchodonosori in Syria, secundus
autem in Babylonia: non autem \textsc{xxv}, ut coniicit Hieronymus
ex quadam victoria Nabuchodonosori de Syria, et Arabia, cuius
meminerit Borosus.
\lnr{}At Berosus loquitur tantum vsque ad obitum
Nabopollassari, qui erat \textsc{xiii} Nabuchodonosori eius filii.
\lnr{}His tam illustribus demonstrationibus sua somnia praeserant, quibus antiquius
est somniare, quam vera dicere, aut nosse.
\lnr{}Nos ad reliqua
pergamus.
\lnr{}Annus capti Sedekiae est 158 Nabonassari, 4124 in periodo Iuliana.
\lnr{}Deductis annis 907 solidis, relinquitur annus 3217
Exodi, qui est 2264 Iudaici Computi in quo sane Neomenia Nisan
habuit characterem feriam quintam, secunda Aprilis, Cyclo
Solis \textsc{xxv}, Lunaea \textsc{vi}.
\lnr{}Sed quadragesimus annus, et quadragesimus
septimus, hoc est 2303, et 2310 Iudaicus fuit sabbaticus.
\lnr{}Iosuae
\textsc{xiiii}, 7, 10.
\lnr{}Iudaei dicunt septenarios annorum Computi sive aerae
suae esse Sabbaticos.
\lnr{}Atqui 2303, et 2310 sunt septenarii.
\lnr{}Ergo recte
Sabbaticos annos putant Iudaei; ut apud illos post legem nihil
hac obseruatione vetustius sit; res prosecto, quae firmissimum
minimentum futura sit harum rerum investigatoribus.
\lnr{}Neomenia Nisan Exodi conueniebat cum neomenia Krionos.
\lnr{}Ita vere naturalis suit illa neomenia.
\lnr{}Praeterea quadragesimus septimus
annus conuenit sabbatico Iudaico: 902 autem annus est tricesimus
Nabopollassari conueniens cum testimonio Ezekielis.
\lnr{}Deniq; anni 86 a septimo Cambysae retro putati desinunt in anno
caedis Nechao Aegyptii, eodemque 139 Nabonassari: quod conuenit
eidem computationi.
\lnr{}Negari igitur non potest, hanc esse veram
Exodi epocham, quam et verbum diuinum, et usus anni Sabbatici,
et historiae fides penes eximium scriptorem Chaldaeum
Berosum, et naturales neomeniae utriusq; sideris in unum conuenientes
confirmant.
\lnr{}Quid postulamus praeterea?
\lnr{}An ut tam certis,
tam egregiis, tam firmis argumentis somnia Corybantum anteponamus?
\lnr{}Quis unquam ita haec demonstrauit?
\lnr{}Quid demonstrauit?

%\end{parnumbers}
%\clearpage
\pnr{p. XV [pdf 42]}
%\begin{parnumbers}
Quis aliter potest demonstrare?
\lnr{}Iam a conditu rerum, ad exodum,
anni sunt absoluti 2452 cum mensibus sex ab autumno, anni vero
absoluti 2453 a vere.
\lnr{}Sed ante Exodum initium anni putabatur ab
autumno, et eodem initio in tempus veris translato, tekupha tamen,
hoc est, finis anni Solaris mansit in autumno, circa quam tekupham
Deus \textgreek{[Greek]} celebrari praecepit.
\lnr{}Igitur ubi initium anni
ab vltima antiquitate suit, inde et rerum quoq; initium repetendum.
\lnr{}quod quidem a nobis factum, damnata priori sententia, quae
initium rerum statuebat in vere.
\lnr{}Reliqua pete ex capite de conditu
rerum.
\lnr{}Praeterea, quibus annus Lunaris in usu est, illis commodius
initium, et rationibus Tropicis conuenientius ab autumno, quam
a vere, ut Iudaeis propter \textgreek{[Greek]}, et Pascha.
\lnr{}Nam si annum
nostrum caelestem admitterent, et hoc unum cauerent, ut \textgreek{[Greek]}
citima sit in secunda Zygonos, semper citimum Pascha esset in neomenia
Krionos.
\lnr{}quia interuallum a neomenia Zygonos, ad neomeniam
Krionos, est semper 178 dierum, uno die plus, quam a scenopegia
ad Pascha.
\lnr{}Anni Sabbatici caussas iam reddidimus, et verum
annum sabbaticum a Iudaeis hactenus obseruari demonstrauimus,
initio hebdomadum sumpto, non utique a defectu Mannae,
quod fanatici quidam, et veritatis hostes faciunt, sed a 48 anno Exodi,
ex capite \textsc{xiiii} Iosue, et rationibus doctorum Habraeorum, qui
dicunt septem annos \texthebrew{[Hebrew]}, id est, subiugationis terrae,
septem \texthebrew{[Hebrew]}
fuisse, id est, diuisionis.
\lnr{}quod rectissimum est: ideoq; hebdomadem
primam diuisionis, non subiugationis procedere in numerum.
\lnr{}An
potuit annus sabbaticus esse ante agrorum culturam?
\lnr{}Furor est aliter putare.
\lnr{}Tamen non desunt, non deerunt, qui solo contradicendi
studio, ut sapere videantur, aliter statuent: quibus per me non solum
hoc facere, sed etiam nos irridere licet; quandoquidem veritas apud
illos nullo in precio est.
\lnr{}Unde nata sit diversitas epochae excidii Ilii,
cum alii 407 annis, alii 405, eum casum antiquiorem prima Olympiade
statuant, aperuimus ex doctrina anni Attici, cui acceptum
referimus quicquid eximium ex alta obliuione eruimus.
\lnr{}Veram sententiam
Eratosthenis esse deprehendimus, quae illam cladem coniicit
in annum 407 ante caput primae Olymiadis: eiusque veram
diem in anno Iuliano ostendimus.
\lnr{}Primam autem Olympiadem
ex doctrina itidem anni Graeci \textsc{xxiii} die Iulii celebratam fuisse ante
nos aperuerat nemo.
\lnr{}Et tamen quidam Simioli tanquam rem
vulgatam in suis vanidicis Chronologiis retulerunt: cuius rei cognitionem
unus Pindarus, quem illi neq; viderunt, neq; norunt, nos
docuit.


%\end{parnumbers}
%\clearpage
\pnr{p. XVI [pdf 43]}
%\begin{parnumbers}
Quemadmodum autem Olympia, ita etiam Karnia plenilunio
celebrata fuisse, libro primo, capite de periodo Laconum
ostendimus. neque solum plenilunio, sed etiam eodem anno, quo
Olympia.
\lnr{}Itaq; Herodotus libro \textsc{viii} Olympia et Karnia anno primo
Olympiadis 75 celebrata suisse scribit, pag. 307 editionis Henrici
Stephani nostri.
\lnr{}Cum multi eruditissimi viri, et quidem in iis
Onufrius Panuinius Pater historiae, multa accurate de Palilibus Vrbis
disseruerint, ut ei doctrinae nihil ad perfectionem deesse videatur,
tamen et plura deesse ex nostris disputationibus colligi potest.
\lnr{}Monere vero debent Annalium et Fastorum scriptores, qui tempora
sua ad annos Vrbis dirigunt, utra Palilia sequantur, Varroniana,
an Catoniana.
\lnr{}Nam certe Onufrius noster, tametsi Catonem sequitur,
tamen quibusdam imprudens ad Varronem transfugit.
% "transfugit" should not be rendered with a long s
\lnr{}Nisi
haec distinctio adhibeatur, ridicula multa consequi necesse est.
\lnr{}Exemplum habemus in annis Christi per annos Vrbis eruendis,
quod hactenus ab omnibus factitatum.
\lnr{}Christus in annis Varronianus
uno anno maior est apud aliquem, quam in Catonianis apud alium.
\lnr{}Quare, ut dixi, ridicula sunt.
\lnr{}In sequentibus epochis quanuis
non ea occurrit obscuritas, quae in prioribus: tamen semper aliquid
noue demonstratur, praeter superiorum scriptorum consuetudinem:
in quibus sunt quaedam de vero die et anno natalis Alexandri, eiusque
obitus: de Encaeniis Machabaei, de initio Simonis Iudaeorum
Ethnarchae, quem Iudaei Iohannem vocant, de aera Hispanica.
\lnr{}De quibus omnibus pluria nova disseruntur, quam trita et vulgaria.
\lnr{}Iam
excessum Herodis ad suum verum annum ex Iosepho retulimus,
qui ad epocham Actiacam illud tempus diligenter exigit, et praeterea
notationem, cui contradici non possit, adducit, defectum Lunarem,
qui contigit \textsc{ix} Ianuarii, anno 45 Iuliano ineunte, in cuius
anni sequenti Decembri Dionysius Exiguus imperite statuit natalem
Christi, nouem solidis mensibus scilicet post excessum Herodis.
\lnr{}Itaq; diligentissimus \textgreek{[Greek]} omnium scriptorum Iosephus
recte ait decessisse \textsc{xxxv} anno labente regni eius a captis a Sofio[?]
Hirosolymis. in quo tamen interpretatio adhibenda.
% Sofio or Sosio
\lnr{}Nam reuera Herodes
obiit anno tricesimo sexto ex diebus aestiuis noni anni Iuliani.
\lnr{}Ergo tricesimus sextus annus Herodis iniuit ex diebus aestiuis anni
Iuliani \textsc{xliiii}.
\lnr{}Obiit autem initio Nisan.
\lnr{}Igitur sine dubio decessit
anno Iuliano \textsc{xlv}, qui erat tricesimus sextus iniens ex diebus aestiuis,
ut diximus.
\lnr{}Sed ex computatione civili Iudaeorum, nondum
\textsc{xxxvi} annus iniuerat.
\lnr{}Iosephus enim, et Iudaei eo saeculo putabant
omnia tempora a \textsc{xxiii} Iiar, ut albi ostendimus: cuius consuetudinis
ignoratio multos decepit.
\lnr{}Ab Iiar igitur Hyrcani, sive, ut Iudaei
vocant, Iohannis Hasmunai, tricesimus sextus annus Herodis inibat,
qui tamen iam nouem mēsibus ante ex consuetudine Romana iniuisset.

%\end{parnumbers}
%\clearpage
\pnr{p. XVII [pdf 44]}
%\begin{parnumbers}
\lnr{}Itaq; eius decessus confirmatur primum accurata putatione
diligentissimi scriptoris, deinde notatione eclipsis, quae omnem contradictionem
excludit.
\lnr{}At ex epilogismis Eusebii Herodes obierit
anno Iuliano \textsc{lii}, septem annis solidis post illum defectum.
\lnr{}qui stupor non meret castigationem, cum tanquam sorex indicio suo perierit.
\lnr{}Nam statim ab eius decessu tetrarchiam suam Archelaus eius filius
iniuit: quod quidem, si huic oraculo Eusebiano credimus, contigerit
anno Christi Dionysiano septimo labente.
\lnr{}Ergo Christus fuerit
annorum septem, cum ex Aegypto monitu Angeli reuoctus est.
\lnr{}Quod est ridiculum.
\lnr{}Rursus anno decimo regni, aut tetrarchiae suae
Archelaus ab Augusto relegatus est Viennam Allobrogum.
\lnr{}Secundum tempus ab Eusebio determinatum, hoc contigerit anno Iuliano
\textsc{lxi}, qui erat annus Tiberii tertius currens, biennio absoluto
post excessum Augusti.
\lnr{}Hoc modo anno tertio excessus sui Augustus
Archelaum relegauerit.
\lnr{}Vides \textgreek{[Greek]}.
\lnr{}Atqui innumeros videas,
quibus hoc somnium placet.
\lnr{}Nam sane omnes fere Chronologiae
et Annales hoc stigmate inusta sunt.
\lnr{}Atque utinam in illis hominibus
non esset vir eximia doctrina praeditus Dominus Caesar Baronius,
Annalium Ecclesiasticorum scriptor, cuius operis copia nobis
facta est ab amicis, cum haec \textgreek{[Greek]} scriberemus.
\lnr{}Is eruditissimus
vir ex hoc loco Eusebii Iosephum exagitat, tanquam imperitum
temporum: cum Eusebius potius ex Iosepho castigandus fuisset.
\lnr{}Nam absque Iosepho esset, quid certi de Herode haberemus?
\lnr{}Quis haec tractauit, praeter illum?
\lnr{}Qui fieri potuit, ut scriptor, cuius diligentia
et fides in notatione temporum spectatissima, in iis peccauerit,
quae sine illo Eusebius et alii ignorassent?
\lnr{}Sed ipse doctus Annalium
conditor potest iam videre, utri fides de hac re habenda, Iosepho,
cuius ratiocinia cum motibus caelestibus congruunt, an Eusebio,
cuius sententia et historiae, et rationi aduersatur?
\lnr{}Sed de Iosepho
nos hoc audacter dicimus, non solum in rebus Iudaicis, sed etiam
in externis tutius illi credi, quam omnibus Graecis, et Latinis.
\lnr{}Itaque
definat mirari doctus vir, cur tot eruditi, et nos quoq; qui non in illis
eruditis, sed in huius scriptoris lectione peregrini non sumus, tantum
illi deseramus, cuius fides et eruditio in omnibus elucet.
\lnr{}Caeterum de Eusebii anilibus hallucinationibus, praeter hanc, quam
modo protulimus, satis libro sexto differuimus.
\lnr{}Sed ad Epochas
nostras venio: quarum omnium rationem reddere longum esset.
\lnr{}De Epocha Martyrum Diocletianea non possumus tacere, eam hactenus
etiam doctissimis imposuisse, quod eam ab initio Diocletiani
incipere omnes credunt.
\lnr{}Hinc prodigiosi errores, et magna Consulum
confusio in Annales et Fastos deriuata sunt, praesertim in annis.

%\end{parnumbers}
%\clearpage
\pnr{p. XVIII [pdf 45]}
%\begin{parnumbers}
Nam initio Diocletiani perperam sumpto, perperam quoque
persecutionis Epocha initur.
\lnr{}Ea semper antiquitus a solis Aegyptiis
Christianis hactenus usurpata fuit.
\lnr{}Itaque Historici et Chronologi,
qui temporibus Caroli Magni dicunt caeptum putari ab annis
Christi, cum antea mos esset annis Diocletiani uti, errant.
\lnr{}Nam
nullis nationibus in usu fuit.
\lnr{}Vnica autem Ecclesia duntaxat Alexandrina,
et quae illi subditae sunt, hac Epocha vsa est semper, utirurque
hactenus, et vocatur ab Aegyptiis, qui Elkupt dicuntur,
\textarabic{[Arabic]} \textit{Aera Martyrum sanctorum.}
\lnr{}Nam
hallucinatus est ille, qui nuper \textarabic{[Arabic]}
\textit{Captiuitatem} vertit in literis
Alexandrinae Ecclesiae Romam missis, anno Martyrum 1310, qui
erat Christi 1593.
\lnr{}Epocha igitur Martyrum iniuit \textsc{xxix} Augusti,
id est, neomenia Thoth Actiaci, vel Mascaram Habesseni, anno Christi
Dionysiano 284.
\lnr{}Initium autem imperii Diocletiani a Palilibus
anni 287.
\lnr{}Differentia anni duo, menses octo.
\lnr{}Perturbatio, quae est in
Consulibus a temporibus Maximinorum, vsq; ad filios Constantini,
ea utique ab antiquo est.
\lnr{}Sed et non minor confusio in annis persecutionis:
ubi magnae sunt \textgreek{[Greek]} apud Eusebium: quanuis
recte sentit de initio Diocletiani, et primo anno persecutionis.
\lnr{}Tamen
omnium Chronologorum fides hac in parte nutat.
\lnr{}Nam edictum
Diocletiani de tradendis codicibus prius est Ecclesiarum euersione,
euersio Ecclesiarum prior caede Martyrum.
\lnr{}Felix Africanus Episcopus
et socii eius supplicio in Campania affecti ideo, quod codices
Deificos, id est, sacram scripturam tradere noluissent.
\lnr{}Itaque in
Actis illorum scriptum fuit: \textit{Et ductus est ad passionis locum, cum etiam
ipsa Luna in sanguinem conuersa est, die tertio Kalendas Septembris}.
\lnr{}De Eclipsi
Lunari loqui manifestum est, cuius is color fuerit, quem sanguineum
astrologi vocant: cuiusmodi proculdubio accidit anno Christi
301, cyclo lunae 17, annis quatuor solidis ante edictum de euertendis
Ecclesiis, idque \textsc{iii} Nonas Septembris, non autem \textsc{iii} Kal.
Septembris, diebus quatuor post passionem Martyrum.
\lnr{}Itaque perturbatus
est ordo verborum.
\lnr{}Legendum enim videtur: \textit{Et ductus est ad passionis
locum, die tertio Kal. Sept. cum etiam ipsa Luna in sanguinem conversa
est.}
\lnr{}Id est, quo tempore Luna defecit, proximo nimirum novilunio.
\lnr{}Nam cum constet passos \textsc{iii} Kal. Septembris, et ita habeat
Kalendarium, non videtur esse error in notatione temporis.
\lnr{}At Dominus Baronius haec gesta confert in annum 302, tribus annis ante
persecutionem: et tamen putat eum esse secundum annum persecutionis,
qui erat decimus nonus Aerae Martyrum, decimus autem
septimus currens ab imperio Diocletiani.

%\end{parnumbers}
%\clearpage
\pnr{p. XIX [pdf 46]}
%\begin{parnumbers}
Sed \textgreek{[Greek]} illorum
Annalium propagati sunt partim ex erroribus aliorum Chronologorum,
quos auctor sequitur, partim ex annis Christi male ad
suam et veram epocham reductis.
\lnr{}Vnde factum, ut ap initio operis,
ad tempora Nicenae synodi, ne unus quidem annus Christi
verae epochae suae redditus sit.
\lnr{}Itaque triennio aliquando, aliquando
quadriennio, ut plurimum autem biennio erratum est.
\lnr{}Exempli
gratia: Excidium Hierosolymorum contigit anno Christi
Dionysiano \textsc{lxx}, quo neomenia Nisan conueniebat cum neomenia
Xanthici, teste Iosepho.
\lnr{}In Annalibus refertur ad annum
72: qui est error Eusebii, sed alibi ab eodem castigatus.
\lnr{}Certum est, Fructuosum Episcopum, Christi Martyrem, cum fociis
passum anno antequam pax et interspiratio data esset Ecclesiis
sub Marco Aurelio Antonino, et L. Aelio Vero.
\lnr{}quod tempus Eusebius confert in annum quartum Olympiadis \textsc{ccxxxiiii},
id est Christi Dionysianum 160.
\lnr{}Ergo passus est Fructuosus anno Christi
159.
\lnr{}Hoc aliter demonstrabimus.
\lnr{}In Actis agonis Fructuosi et
sociorum legitur: \textit{Producti sunt duodecimo Kalend. Februarii, feria
sexta.}
\lnr{}Ergo litera Dominicalis erat B.
\lnr{}Proinde hoc accidit anno
159, triennio citius, quam notatum in Annalibus.
\lnr{}In Actis Andreae
militis et sociorum scriptum extat, eos necatos fuisse decimoquarto
Kalendas Septembris, Dominico die, hora secunda.
\lnr{}Igitur litera Dominicalis erat G.
\lnr{}Hoc necessario contigit anno 305,
qui erat primus persecutionis a Pascha illius anni antecedente, post
euersas Ecclesias: quod quidem Pascha celebratum 25 Martii, ipso
die termini.
\lnr{}At in Annalibus hoc refertur in annum 301, quadriennio
ante rem gestam.
\lnr{}Rursus in Epistola Vigilii Episcopi Tridentini
de Passione Sanctorum Sisinnii, Martyrii, et Alexandri,
ita legitur: \textit{Die paßionis Sanctorum, quarto Kalendas lunias, feria
sexta, nascente luce.}
\lnr{}Passi ergo sunt anno 403, cyclo Solis \textsc{xx}, quando
\textsc{xxix} Maii erat feria \textsc{vi}.
\lnr{}At in Annalibus dicitur scripta
anno 400 Christi.
\lnr{}Scripta ergo fuisset triennio ante caedem
ipsorum Martyrum.
\lnr{}Cui absurditati ipse non adscribet, certo scio.
\lnr{}In iisdem
Annalibus ex codice Antonii Augustini mentio fit Homiliae
Cyrilli Episcopi dictae in natiuitate Ioannis Baptistae, Pharmuthi
vicesima octaua, indictione prima, sub Theodosio iuniore et Valentiniano.
\lnr{}Ergo dicta fuit Homilia anno Christi 433, April. vicesima
tertia.
\lnr{}At in Annalibus refertur in annum 432, April. 29. S. Benedictus
Monachorum Occidentis Pater, obiit \textsc{xi} Kal. Aprilis, Sabbato
sancto, ut refert Aimoinus monachus ex Actis S. Mauri ipsius
Benedicti discipuli.
\lnr{}Toto illo saeculo hoc non potuit contingere, nisi
anno 536.

%\end{parnumbers}
%\clearpage
\pnr{p. XX [pdf 47]}
%\begin{parnumbers}
Tamē in Annalibus Ecclesiasticis obitus Benedicti confertur
in annum 542, sex annis serius.
\lnr{}Multa igitur peccari necesse est
in Gestis Benedicti, quae in illis Annalibus referuntur.
\lnr{}In Encyclica
epistola Vigilii Papae scriptum fuit: \textit{Piißimus atque clementißimus
Imperator Dominico die, id est, Kalendis Februarii, gloriosos Iudices suos
ad nos destinare dignatus est.}
\lnr{}Anno 554 Kalendis Februarii fuit dies
Dominica.
\lnr{}At in Annalibus hoc confertur in annum 552, duobus
annis citius.
\lnr{}Anno 546 turbatio facta in Pascha, ut ex Cendreno docuimus,
capite de periodo Dionysiana, libro \textsc{iiii}.
\lnr{}In Annalibus referetur
sub anno 545.
\lnr{}Martinus Episcopus Turonensis obiit anno
395, ut accurate a nobis disputatum est.
\lnr{}Auctor Annalium Sigebertum
sequutus coniicit in annum 402.
\lnr{}Ex eo errore multum peccatum
est in temporibus Regum Francorum.
\lnr{}de quibus consulatur vltima
diatriba libri sexti huius operis nostri.
\lnr{}Non semel monuimus magnam
perturbationem esse in initiis Imperatorum, a Maximinis
ad Valentinianum.
\lnr{}Vt alios taceam, Constantini initium ab aliis in
305, ab aliis in 306 annum coniicitur.
\lnr{}At Constantinus iniuit imperium
post obitum patris sui Chlori.
\lnr{}Obiit autem Chlorus in Britannia
anno primo Olympiadis 271, ut inquit Socrates.
\lnr{}Nos ostendimus,
apud Socratem, Hieronymi Supplementum, Ausonium, et alios,
semper Olympiadem sumi pro lustro Iuliano, non pro lustro Olympico
Elidensium, idq; lustrum Iulianum biennio posterius esse Elidensi,
cum incipiat ab anno Iuliano bisextili.
\lnr{}Itaq; is fuit annus bisextilis,
quo obiit Chlorus, et imperium iniuit Constantinus.
\lnr{}Sed duae
cautiones adhibendae.
\lnr{}Prior est, ut scias annum Constantinopolitanum,
sive Nicenum hic intelligi, qui incipiebat a \textsc{xxiiii} Septembris.
\lnr{}Altera, ut prolepsis usurpata intelligatur in anno mortis Chlori.
\lnr{}Nam obiit \textsc{xxv} Iulii, \textsc{lxi} diebus ante
\textsc{xxiiii} Septembris, et
tamen obitus eius ad eundem annum refertur quo iniuit imperium
eius filius, \textgreek{[Greek]}, ut dixi.
\lnr{}Omnino igitur iniuit imperium anno
303, aut 307.
\lnr{}Nam primus annus Olympiadis Iulianae incipit semper
diebus 153 ante bisextum.
\lnr{}Sed nemo concedet Chlorum obiisse
anno 303.
\lnr{}Obiit ergo 307.
\lnr{}Et proinde anno 307 iniuit imperium
Constantinus, ex ante diem \textsc{viii} Kal. Octobr. eiusdem anni 307.
\lnr{}In his prouocamur a docto Annalium scriptore, et rem absurdissimam
prodidisse nos dicit, Constantini imperium iniisse ex anno
308, cum, ut inquit ipse, iniuerit anno primo Olympiadis 271,
Christi vero 306[?].
% 300 or 306 ?
\lnr{}Nos vero negamus vllam culpam aut absurditatem
in nobis admissam.
\lnr{}Nam annus Christi 308 Constantinopolitanus
incipit a Septembri anni 307, ut iam dictum est.
\lnr{}Et proinde ipsum, et alios errare, qui annum Christi 306 a Kalendis Ianuarii
dicunt esse annum labentem Constantini.

%\end{parnumbers}
%\clearpage
\pnr{p. XXI [pdf 48]}
%\begin{parnumbers}
Hoc enim volunt,
cum putant primum 271 Olympiadis Elidensis annum esse primum
Constantini.
\lnr{}Olympias enim illa Iphitea caepit ex diebus aestiuis
anni 305, qui fuit annus primus presecutionis.
\lnr{}Quare in annis
Constantini, ut in aliis, insigniter peccatum est a viro docto.
\lnr{}His
postis, quinquennalia Constantini data sunt anno 312: vicennalia
autem anno 327.
\lnr{}Interuallum inter illas duas celebritates interiectum
haud dubie vocatur Indictio, iniens a datis quinquennalibus,
desinens[?] in vicennalibus, quibus concilium Nicenum dimissum.
% desinens or definens?
\lnr{}Sed neq; hoc placet Domino Baronio: neque caussam appellationis
Indictionum admittit.
\lnr{}At nos dicimus, non minus iniuste nos
hic, quam in initio imperii Constantiniani reprehendi.
\lnr{}An negat
Indictiones in quinquennia indici, et in quinquennalibus Principum
panegyribus remitti?
\lnr{}Si non credit, legat et quae priore, et quae
hac editione ad eam rem collegimus.
\lnr{}Quinquennalia illa dicuntur
\textgreek{[Greek]}, hoc est ad verbum, sparsiones, largitiones, profusiones, in
quibus liberalitas Principis ad remissionem vsq; tributorum, et indictionum,
editiones munerum et spectaculorum, congiaria, et donatiua
extendebatur.
\lnr{}Inde \textgreek{[Greek]} non solum pro illa largitione
sumitur, sed et pro ipsa indictionis temporalis nota.
\lnr{}Nam quod Latini
dicunt, Indictione prima, secunda, tertia hoc factum est, Graeci
dicunt, \textgreek{[Greek]}.
\lnr{}Non ergo nos, sed ipse fallitur.
\lnr{}Quid?
\lnr{}si initium Constantini a nobis ignoraretur, tamē quinquennalia
eius nos manu ad illud deducerent.
\lnr{}Itaque ignorari n n [?]
potest.
% Probable printing error. "n n" should read "non".
\lnr{}Neq; minus errat, cum cladem Maxentii coniicit in annum
312.
\lnr{}Quot modis enim hoc refelli potest?
\lnr{}Sed de eo suo loco.
\lnr{}Nam
Maxientius anno 313, non 312 extinctus est, ut recte Panuinius notat,
sed male inde Indictionum initia et caussas repetit: quod a nobis
olim diligenter discussum fuit.

%%% === Sextus Liber
Sextus liber continet residuum Epocharum,
in quo nobiliores quaestiones de Natali die, et Passione Christi,
de Hebdomadibus Danielis, quae breuibus diatribis explicari
non possunt, presequimur.
\lnr{}Ne autem aut rudiores, aut refractarii auctoritate
veterum scriptorum nobis praescribere possent, pauca de
Eusebii erroribus in antecessum delibauimus, in quibus, praeter frequentes
\textgreek{[Greek]}, puerile illud deliramentum de Effenis confutauimus,
quos Christianos fuisse hoc unico argumento probat, quod
\textgreek{[Greek]} essent, et solitarie viuerent, et monasteria haberent:
\lnr{}quasi Bonzios
Iapanensium Christianos esse censeamus, quia et coenobitae sunt,
et Psalmos quosdam instar monachorum Europaeorum alternis modulantur,
et horas Canonicales eorum exemplo habent.
\lnr{}Eorum Essenorum alii \textgreek{[Greek]}, alii \textgreek{[Greek]} fuerunt.
\lnr{}Sed horum non videtur
secta diuturna fuisse.

%\end{parnumbers}
%\clearpage
\pnr{p. XXII [pdf 49]}
%\begin{parnumbers}
Ast \textgreek{[Greek]}, aut eorum non dissimilium
synagogae fuerunt ad tempora Iustiniani.
\lnr{}Sunt enim ii, qui Caelicolae
vocantur.
\lnr{}Nam et nomen id indicat.
\lnr{}Caelicolae enim sunt
Angeli.
\lnr{}Ita vocari volebant, propter sanctum, et caeleste, ut ipsis videbatur,
vitae institutum.
\lnr{}In perueteri Glossario Latinoarabico \textit{Caelicola}
[Greek][Arabic][?]. id est, Angelus.
\lnr{}Praeterea quia erant \textgreek{[Greek]}, novi
baptismi auctores Donatistis fuerunt.
\lnr{}Princeps eorum vocatur
Maior, ut et aliorum Iudaeorum.
\lnr{}Hoc enim est \texthebrew{[Hebrew]}.
\lnr{}Philo dubitans
quare Esseni illi dicti sint \textgreek{[Greek]},
 utrum quia medicinam profiterentur,
\lnr{}an quia Deum colerent, ex eo coniiciendum relinquit,
eos non dictos esse quasi \texthebrew{[Hebrew]} \textgreek{[Greek]},
 ut volebat quidam Lunaticus
literarum Hebraicarum professor, sed quia \textgreek{[Greek]} vocat, eo ostendit
\texthebrew{[Hebrew]} dictos, hoc est, \textgreek{[Greek]}.
\lnr{}Quod Christiani non essent, sed
mere Esseni, statim initio libri ostendit Philo.
\lnr{}sed et Sabbati summus
cultus, et reliqua, quae a Philone de ipsis narrantur, satis leuitatis
damnant Eusebium, et reliquos veteres, qui Eusebium sequuti,
idem hariolati sunt.
\lnr{}Sed in Annalium tomo primo tacite perstringitur
sententia nostra ab auctore, qui tamen fatetur veros Essenos Iudaeos
fuisse.
\lnr{}Mirati sumus, quomodo ille putauit in unum haec bene
conuenire posse, Iudaismum et Christianismum.
\lnr{}Vt hoc probet, ait
veteres patres idem scribere, quod Eusebium.
\lnr{}Atqui ex Eusebio
hoc desumpserunt, et eius auctoritate contenti Philonem non consuluerunt.
\lnr{}quem si legissent, nunquam tam ridiculae sententiae assensum
accommodassent.
\lnr{}Haec vero puerilia sunt.
\lnr{}Venio nunc ad natalem
Christi, quem vetustas Christianismi ad \textsc{xxviii} annum Actiacum
retulit, recte.
\lnr{}Nam Christus iniens annum unum a tricesimo
aetatis suae accessit ad baptismum, ut omnes vetustissimi Patres ex
Luca retulerunt, et post eos eruditus Annalium scriptor.
\lnr{}Baptizatus est anno \textsc{xv} Tiberii, duobus Geminis \textsc{coss}. anno
Iuliano 74.
\lnr{}Ergo \textsc{xxv} Decembris anni 73 illi inibat annus primus a tricesimo.
\lnr{}Deductis 30 annis absolutis de 73, remanet annus Iulianus
43, in cuius \textsc{xxv} Decembris natus fuerit Dominus, cyclo Lunae
\textsc{xviii}, anno Actiaco \textsc{xxviii},
 ut illi vetustissimi partes crediderunt,
duobus annis solidis ante epocham hodiernam Dionysianam,
anno solido cum diebus aliquot ante excessum Herodis.
\lnr{}Hoc proculdubio
verum est.
\lnr{}Sed in Annalibus peccatur ab auctore in anno
\textsc{xv} Tiberii.
\lnr{}Quem enim putat \textsc{xv}, is est \textsc{xvi}, et magno errore illi
attribuit Consules duos Geminos, quibus Consulibus annus \textsc{xvi}
Tiberii iniit ex \textsc{xix} Augusti, cyclo Lunae undecimo, anno Iuliano
74.
\lnr{}Nisan igitur is, qui proxime sectus est baptismum Christi,
Consulibus duobus Geminis, antecessit annum \textsc{xvi} Tiberii ineuntem,
mensibus quinque.

%\end{parnumbers}
%\clearpage
\pnr{p. XXIII [pdf 50]}
%\begin{parnumbers}
At scriptor Annalium putat duos Geminos
Consulatum gessisse cyclo Lunae \textsc{xvi}: in quo ne sic quidem
sibi constat.
\lnr{}Nam is fuerit annus 75 Iulianus iniens.
\lnr{}Hoc modo Decembri anni 74 Christus iniuerit annum primum a tricesimo: et
deductis 30 absolutis, remanebit annus 44 Iulianus, in quo natus
Christus fuerit, tribus circiter mensibus ante excessum Herodis, anno
solido ante epocham Dionysianam, qua hodie Ecclesia utitur.
\lnr{}quae sane multorum veterum, inque illis Eusebii fuit opinio.
\lnr{}Sed
Christus baptizatus anno 74 Iuliano: passus 78.
\lnr{}Differentia, anni
quatuor solidi, paschata quinque.
\lnr{}Quorum nullum vestigium in illis
Annalibus extat.
\lnr{}Quinetiam auctor, quando numerus annorum
non succedit ex voto, culpam in Iosephum reiicit, mendacem multis
modis arguens: inter alia, quod scripserit \textgreek{[Greek]} factam post
Archelai relegationem, cum, inquit, ea \textgreek{[Greek]} Christo nascente
contigerit, et aperte Eusebius id indicauerit.
\lnr{}Nos hallucinationem
Eusebii loco suo confutauimus, in quo descriptionem patrimonii
Archelai cum descriptione totius orbis Romani confundit more
suo, neq; meminit verbis illis, \textgreek{[Greek]}, designari non
unicam fuisse illam descriptionem, cum \textgreek{[Greek]} mentio fiat.
% Final period not visible in original.
\lnr{}Quare
idem Euangelistes quemadmodum prioris meminit in Euangelio,
ita alterius mentionē facit in Actis.
\lnr{}ut non sit audiendus doctus Annalium
scriptor, qui non solum hac in parte Eusebii auctoritatem
Iosepho opponit, sed etiam adiicit descriptionem illam[?] eandem esse,
de qua Aethicus statim initio libri sui loquitur: cum tamen neque
tempus, neque res conueniat[?] descriptioni nascente Christo factae.
\lnr{}Nam descriptio, de qua intelligit Aethicus, caepit ab anno caedis
Caesaris, desiuit[?] in anno \textsc{xxxiii},
 qui erat tricesimus quartus a primis
Kalendis Ianuariis Iulianis, decem annis absolutis ante verum
natalem Christi, duodecim ante epocham Christi hodiernam Dionysianam.
\lnr{}Res autem eadem non est, imo longe diuersa: atq; adeo
tantem differt[?] descriptio, de qua Aethicus loquitur, a descriptione,
quae facta Christo nascente, quantum decempeda, et tabulae [censuales][?].
\lnr{}Nam illa descriptio Aethici mandata est agrimensoribus, et
Geometris, haec Rationalibus.
\lnr{}Illa orbis mensura, \textgreek{[Greek]},
hac census et facultates in Tabulas relatae.
\lnr{}Sed neq; recte concludit,
Iosephum hallucinatum, quod paulo ante initia belli Iudaici
auditam ex adytis templi vocem scripserit, quae diceret \textsc{hinc
migremvs}: cum, inquit, Eusebius id in passionis Dominicae tempus
referat.
\lnr{}Quomodo Eusebius melius scire potuit ea, quae contigerunt
Christi et belli Iudaici tempore, quam Iosephu? aut unde,
quam ex Iosepho? de illis dico, quae non pertinent ad historiam euangelicam.



%\end{parnumbers}
%\clearpage
\pnr{p. XXIV [pdf 51]}
%\begin{parnumbers}
Sed tam friuolum argumentum eluditur iis, quae aduersus
hanc Eusebii hallucinationem libro sexto decimus.
\lnr{}Denique iniuste
vbique Iosephum reprehendit, omnium scriptorum veracissimum
et religiosissimum, quod quidem ipsius scripta loquuntur.
\lnr{}quem
auctorem si non tam contempsisset, nunquam eos
 \textgreek{ανἀχρονισμους [Greek:anachronism]} commisisset,
quibus totus contextus temporum primi tomi perturbatus
est.
\lnr{}Sed antequam ex hac velitatione facessimus, qua et nos et cognominem
nostrum scriptorem ab animaduersione docti viri vindicamus,
nos homines Aquitani expostulamus cum eo, quod a nobis
tres summos viros abdixit, Paulinum, Phoebadium, et Sulpitium
Seuerum:
\lnr{}qui cum suerint natione, et domo Aquitani, tamen
Paulinum et Sulpitium Romae natos scribit, Phoebadium in Hispania.
\lnr{}Quis illum docuit Paulinum non esse natum Burdigalae, ubi
antiquitus Paulina gens, hodieque quaedam regio vrbis Burdigalensis
Paulino cognominis est?
\lnr{}Phoebadium autem Aginni Nitiobrigum
Episcopum quare in Hispania natum dicit, aut quo auctore?
\lnr{}Apud Hieronymum male excusum est Soebadius, qui error irrepsit
ex Sophronio, ubi legitur \textgreek{[Greek]}.
\lnr{}Sed liber manu scriptus
Sanctae Mariae de Granateria liquido habet Febadium.
\lnr{}Apud Sulpitium
Seuerum deprauatum quoq; est, ubi legitur Fegadius, pro
Febadius, ut quidem librarii scribunt.
\lnr{}nam orthographia est \textgreek{[Greek]},
Phoebadius: satis hodie notus erudita sua in Arrianos Epistola,
quae ante \textsc{xxv} annos primum edita.
\lnr{}Mei municipes Fiarium vocant,
cuius memoriam bis quotannis instaurant, ineunte ieiunio
quadragesimae, et die Marci Euangelistae, mense Aprili, si bene
memini.
\lnr{}Huic successit Gauidius in episcopatu.
\lnr{}Sulpitium Seuerum
nemo hactenus Aquitanum fuisse dubitauit: sed patria ignoratur,
cum tamen ipse Nitiobrigem sese manifesto prodat, cum Seruationem
Tungrorum, Phoebadium autem suum Episcopum fuisse scribit.
\lnr{}Phoebadius autem erat Nitiobrigum Episcopus.
\lnr{}Iste Sulpitius
Ecclasiasticorum purissimus scriptor, post transitum Martini recepit
sese Elusonem, quo tempore ad eum scribebat Paulinus.
\lnr{}Id oppidum est cum arce veteri in finibus Nitiobrigum, qua amni Draguto
a Petrocoriis diuiduntur.
\lnr{}Vulgo \textit{Lausun}.
\lnr{}Sed de hoc satis.
\lnr{}Mei
Nitiobriges pro Sulpitio Supplicium dicunt, quomodo et Bituriges
suum illum vocant, quem eundem cum hoc faciunt perperam,
cum inter transitum Martini, cuius noster Sulpitius discipulus fuit,
et ordinationem Sulpitii Episcopi Bituricensis sub Guntchramno
Rege, intercedant plus minus anni 190.
\lnr{}Non iniuriam facimus
docto viro, si cum bona eius venia doctissimos viros Aquitanos,
et Christianissimos originibus suis vindicamus.

%\end{parnumbers}
%\clearpage
\pnr{p. XXV [pdf 52]}
%\begin{parnumbers}
Sed quemadmodum tribus viris Aquitaniam orbaverat, ita eandem duabus
alienis civitatibus donavit, Reiensi, et Vasensi.
\lnr{}Prosperum non uno
loco dicit Regiensium in Aquitania fuisse Episcopum, cum dicendum
fuerit, Prosperum Aquitanum fuisse Episcopum Reiensium,
aut Regiensium in secunda prouincia Narbonensi.
\lnr{}Hodie \textit{Ries} vocatur.
\lnr{}Nugantur qui eum Regii Lepidi Episcopum et scripserunt,
et in fronte eius sacrorum poematum apponi curarunt: quasi Reienses,
in secunda prouincia Narbonensi, iidem sint cum Regio Lepidi
in Aemilia.
\lnr{}Vasense autem consilium idiotismus illorum temporum
vocauit, quod potius Vasionense dicendum erat.
\lnr{}Vasio Vocontiorum hodie \textit{Vaison} dicitur.
\lnr{}Est Episcopatus Auenioni metropoli
attributus.
\lnr{}Imperite quidam cum foro Vocontiorum confundunt.
\lnr{}Itaque Vasense, vel Vasionense, in Vasatense mutandum non
erat.
\lnr{}quemadmodum in anno Christi 552 perperam Firminum
Vticensem mutat in Venciensem.
\lnr{}Vticenses, vulgo dicuntur \textit{Vsetz}.
\lnr{}Est Episcopatus in prima Narbonensi.
\lnr{}Dicuntur etiam Vcetenses,
et Vcetiae Episcopus.
\lnr{}Apud Gregorium Turonensem libro \textsc{vi},
mentio est Ferreoli Episcopi Vcetensis: ubi vulgo male Vcecensis.
\lnr{}Sed tam imperite vulgus Vticenses deprauauit in Vcetenses, quam
Arausio in Aurasio: Vasensis dixit, pro Vasionensis.
\lnr{}At ciuitas sive
Episcopatus Venciensis, est in secunda Narbonensi. Vulgo S. Paulus
de Venciis.
\lnr{}Scribendum vero per t.[?] Ventiensis, \textgreek{[Greek]} enim dicitur
Ptolemaeo.
\lnr{}Fuitque Nerusiorum in Alpibus Graiis Metropolis.
\lnr{}Sequuntur in sexto libro illa quinque Paschata a baptismo
ad resurrectionem, fuis temporibus, Consulibus, et cyclis notata.
\lnr{}In tertio Paschate quid fit \textgreek{[Greek]}, explicamus,
quae verissima interpretatio adhuc assensum vel mereri, vel
exprimere a doctis hominibus non potuit: quod valde miror,
cum absurdissima sit ea, quam sequuntur ipsi.
\lnr{}Omnes igitur uno
ore putant \textgreek{[Greek]}, pro \textgreek{[Greek]} dictum esse.
\lnr{}Id ad verbum
Hebraice esset \texthebrew{[Hebrew]}:
 aliter \textgreek{[Greek]}, Latine Praeposterum.
\lnr{}quo nihil praeposterius dici potuit.
\lnr{}Nam quid est praeposterum Sabbatum?
\lnr{}Non pudet iocularis interpretationis?
\lnr{}Sed ita est.
\lnr{}Alius fortasse assensum extorsisset.
\lnr{}Sed quia a nobis, ideo
minus acceptum.
\lnr{}Theophylactus post Epiphanium, et alios veteres,
interpretatur \textgreek{[Greek]}.
\lnr{}Itaque verum est, quod diximus, omnes tam veteres, quam
recentiores \textgreek{[Greek]} interpretari
 \textgreek{[Greek]}, id est \textgreek{[Greek]},
praeposterum.
\lnr{}Vt illud probet, idem Theophylactus
subiicit: \textgreek{[Greek]}.

%\end{parnumbers}
%\clearpage
\pnr{p. XXVI [pdf 53]}
%\begin{parnumbers}
Quod falsum est,
propter translationes, quas imperiti negant saeculo Christi usurpatas,
cum tamen longe ante Christum in usu fuisse demonstrauerimus,
ut locus non sit pertinaciae.
\lnr{}Sed valeant Sabbata praepostera.
\lnr{}Igitur \textgreek{[Greek]}, non quod \textgreek{[Greek]}, sed
quod \textgreek{[Greek]}.
\lnr{}Nam \textgreek{[Greek]} inibat
computus \textgreek{[Greek]}.
\lnr{}Hebraei etiam hodie vocant \texthebrew{[Hebrew]}
\textgreek{[Greek]}: id est, Sabbatum, quod est primum a
\textsc{xvi} Nisan, quae est \textgreek{[Greek]}.
\lnr{}Vera, et recta interpretatio sine ulla praeposteritate.
\lnr{}De quinto Paschate verbum non addidissem,
nisi cum haec commentarer, incidissem in Commentarios quorundam,
qui Christum passum volunt ipso solenni Paschatis, \textsc{xv}
Nisan, feria sexta, nempe parasceve Sabbati.
\lnr{}Quanuis auctoritas
Evangelistarum, ratio ipsa, doctrina veteris anni Iudaici, omnia
denique contra illos faciant, tamen potius etiam Evangelistas ipsos
valere iubebunt, quam ut sententiam mutent.
\lnr{}Caussa pertinaciae
verba Evangelistarum, \textgreek{[Greek]}.
\lnr{}Contra obiicitur:
\textgreek{[Greek]}.
\lnr{}Iohan. \textsc{xix}, 14.
\lnr{}Hic miseram latebram quaerunt,
et strenuo mendacio ictum declinant.
\lnr{}Aiunt \textgreek{[Greek]} tantum
dici de Sabbato.
\lnr{}Acuti homines!
\lnr{}Quare dicit \textgreek{[Greek]},
nisi \textgreek{[Greek]}, quasi et sit \textgreek{[Greek]}
 alius rei, quam \textgreek{[Greek]}?
 
Quod quidem verum est.
\lnr{}Nam \textgreek{[Greek]} est genus, cuius species
\textgreek{[Greek]}, \textgreek{[Greek]}.
\lnr{}Utrumque uno verbo Hebraeis est
\texthebrew{[Hebrew]}.
\lnr{}Itaque \texthebrew{[Hebrew]} \textgreek{[Greek]},
 sive \textgreek{[Greek]} dicitur,
\textgreek{[Greek]}, quod sint aliae \textgreek{[Greek]},
 sive \textgreek{[Greek]}, quae a festis
suis appellationem sortiuntur.

\texthebrew{[Hebrew]} \textgreek{[Greek]}, \textgreek{[Greek]}.

\texthebrew{[Hebrew]} \textgreek{[Greek]}, \textgreek{[Greek]}.
\lnr{}Itaque ei parascevae, in qua Christus passus, accidit tum
vt ex consuetudine esset \textgreek{[Greek]}, id est, \textgreek{[Greek]}:
tum, ut casu \textgreek{[Greek]}, id est, \textgreek{[Greek]}.
\lnr{}Sabbatum enim
illud erat \textgreek{[Greek]}.
\lnr{}Quae propterea dicitur \textgreek{[Greek]}
ab Evangelista.

\textgreek{[Greek]}.
\lnr{}Quod
et ipsum quoque \textgreek{[Greek]} dictum, tanquam sit quaedam
 \textgreek{[Greek]},
quae non sit \textgreek{[Greek]}.
\lnr{}Nam quod Hebraice dicitur \texthebrew{[Hebrew]},
id est, Solenne, id \textgreek{[Greek]} Iudaei, et Apostolus vocant
 \textgreek{[Greek]}.
\lnr{}Unde \textgreek{[Greek]} sive \textgreek{[Greek]} solenne \textgreek{[Greek]}
dicitur \textgreek{[Greek]}, ut supra \textgreek{[Greek]} citavimus, nimirum
exemplo Iudaeorum, et Samaritarum, qui \textgreek{[Greek]}
\texthebrew{[Hebrew]} vocant \textgreek{[Greek]}.
\lnr{}Tria autem proprie Hebraice vocantur \texthebrew{[Hebrew]},
aliter \texthebrew{[Hebrew]}, quas \textgreek{[Greek]} quoque vocarunt
 \textgreek{[Greek]}, \textsc{xv} Nisan, id est, \textgreek{[Greek]},
 cum \textsc{xxi} Nisan.

\textsc{vi} Sivvan,
id est, \textgreek{[Greek]}.

\textsc{xv} Tisri, id est, \textgreek{[Greek]}, cum \textsc{xxii} Tisri.

%\end{parnumbers}
%\clearpage
\pnr{p. XXVII [pdf 54]}
%\begin{parnumbers}
\lnr{}Acron quidem in illud \textit{– hodie tricesima sabbata} scholio isto
\textit{quae Neomenias esse dicunt: quoniam per Sabbata Iudei numeros Lunares
accipiunt.}

\textit{Et Sabbatum magnum in renovatione Luna a Iudaeis
hodie celebratur}: videtur omnes neomenias nomine Sabbati
magni indigetare: sed, quid sit Sabbatum magnum, ignorat.
\lnr{}At
Sabbatum ordinarium nunquam dicitur \textgreek{μεγάλη ἡμέρα},
% Great Day
 non magis,
quam \texthebrew{[Hebrew]}.
\lnr{}Sic Philo libro \textgreek{[Greek]} dixit \textgreek{[Greek]},
loquens de Pentecoste.
\lnr{}At ordinarium
Sabbatum nunquam dicitur \textgreek{[Greek]}, aut \textgreek{[Greek]}.
\lnr{}Temere igitur
doctor Theologus Commentario in Iohannem ait \textgreek{[Greek]}
tantum dictam de feria sexta, et omne Sabbatum dici posse \textgreek{[Greek]}.
\lnr{}Ecquae Grammatica haec est, ut \textgreek{[Greek]}
non sit \textgreek{[Greek]}?
\lnr{}Quid potest dici absurtius?
\lnr{}Ac propterea
longe iocularius dicit eodem modo dictum \textgreek{[Greek]},
ut Ioh. \textsc{vii}. 37. \textgreek{[Greek]}.
\lnr{}Nam verum est eodem modo dictum, sed
contra animi eius sententiam: quod nimirum dicta sit \textgreek{[Greek]},
quia octava Tabernaculorum, non quia Sabbatum.
\lnr{}Quae quidem
octava fuit eo anno feria quinta, non Sabbatum, 169 diebus ante
passionem.
\lnr{}Itaque dum hoc effugium parat, suo se gladio iugulat.
\lnr{}Quis unquam tam obstinatos adversus veritatem animos credidisset?
\lnr{}Illam autem obiectionem quam argute eludunt ! [?] \textgreek{[Greek]}.
\lnr{}Aiunt \textgreek{φαγεῖν[?] τὸ πάσχα}, hic non esse
agnum Paschalem comedere, sed alia sacrificia Paschalia.
\lnr{}Argute, docte, eleganter, ut nihil supra: quasi in parasceve comedere
Pascha aliud sit, quam agnum Paschalem comedere.
\lnr{}Imo nos negamus, \textgreek{θύαν κὶ[?] φαγεῖν τὸ πάσχα[Greek]},
 aliud esse, quam agnum Paschalem
immolare, aut manducare.
\lnr{}Exodi \textsc{xii}, 21.
\lnr{}Neque ullus paulo doctior illud sine risu audire potest.
\lnr{}Sed quid ex tot mendaciis consequuntur?
\lnr{}Quid, quam ut se deridendos propinent?
\lnr{}Si quintadecima Nisan, quando Christus passus, fuit feria sexta: ergo Pentecoste
illius anni fuit Sabbatum.
\lnr{}Nam Pentecoste est feria secunda
quintaedecimae Nisan, ut diximus ad Computum Iudaeorum.
\lnr{}Fallitur ergo Ecclesia, et omnis Christianitas, quae ab ultima usque
 antiquitate
credidit illam Pentecosten fuisse diem Dominicam, non
Sabbatum.
\lnr{}Quid igitur?
\lnr{}Quid?

\textit{— Dic aliquem, dic, Quintiliane, colorem.}
\lnr{}Itaque Doctori Theologo non bene procedit commentum.
\lnr{}Rursus urgetur absurditate.
\lnr{}Deus die magno Azymorum districte
vetat opus facere.
\lnr{}Exodi \textsc{xii}, 16.
\lnr{}Levitici \textsc{xxiii}, 7.
\lnr{}Hinc quoque
aliquo insigni facinore elabendum erit [?].
\lnr{}Adducit locum ex libro Iudaico,
cui titulus \texthebrew{[Hebrew]} id est, ligatio Isaaci, ut probetur etiam
Sabbato licuisse[?] opus facere: in quo is, qui locum vertit, imposuit homini
quaestionum, quam Hebraismi peritiori.

%\end{parnumbers}
%\clearpage
\pnr{p. XXVIII [pdf 55]}
%\begin{parnumbers}
\lnr{}Nam qui interrogat,
an Sol occasus sit, item, an sit Sabbatum, eandem rem duabus interrogationibus
significat.
\lnr{}Si enim Sol nondum occidit, est Sabbatum,
in quo non auderet mittere falcem in messem.
\lnr{}Oportet igitur,
ut Sol occiderit prius, et consequenter non erit amplius Sabbatum,
id est, iam praeterierit quintadecima Nisan, quae quacunque feria
inciderit, dicitur Sabbatum, Levitici \textsc{xxiii}, 15.
\lnr{}Igitur interroganti,
an Sol occidit, respondetur, occidisse.
\lnr{}Rursus, an sit Sabbatum, id
est, an \textgreek{[Greek]} nondum praeterierit, si respondetur adhuc
esse, nihil agitur: si respondetur non esse Sabbatum, tunc confidenter
immittit falcem in messem.
\lnr{}Haec profector est mens illius loci,
quanquam libri copia non est.
\lnr{}Sed qui vertit illi haec verba, dicit responderi
esse Sabbatum.
\lnr{}Ergo hoc modo oportebat omnem \textsc{xvi}
Nisan esse Sabbatum, omni anno.
\lnr{}Quod quis non miretur a Doctore
Theologo non animaduersum?
\lnr{}Atque adeo illi commentum
placet.
\lnr{}Denique tot lapides movit, ut tandem concluderetur, Ecclesiam
falso putare diem Pentecostes, quando Spiritus sanctus super
Apostolos descendit, fuisse Dominicam, cum fuerit sabbatum, ex
hypothesibus Doctoris.
\lnr{}Concludimus igitur, quod nemo sani capitis
negaverit, Chistum Pascha comedisse tertia decima Nisan civilis,
quartadecima Luna.
\lnr{}Unde recte Evangelistae: \textgreek{[Greek]},
nempe \textgreek{[Greek]}.
\lnr{}Nam sane quoties fit translatio feriae,
tunc duplex est neomenia, prior quidem \textgreek{[Greek]}, posterior vero
\textgreek{[Greek]}.
\lnr{}Sed, inquiet, alius Evangelistes dicit, \textgreek{[Greek]}.
\lnr{}Ergo omnes \textgreek{[Greek]}.
\lnr{}Non sequitur: quare alius interpretatur, \textgreek{[Greek]}.
\lnr{}Christus \textgreek{[Greek]}.
\lnr{}Christur immolavit Pascha in qua die
oportebat, nempe quartadecima Luna.
\lnr{}Iudaei postridie \textgreek{[Greek]},
in qua non oportebat, nempe quintadecima Luna.
\lnr{}Et ita quoque
hunc nodum soluerunt Monachus Veronensis Hilario, et Paulus
Episcopus Burgensis ex Iudaeo Christianus.
\lnr{}Neque melior solutio
dari potest.
\lnr{}Neque vero illi duo viri docti tam vacui capitis fuerunt,
ut crederent eo anno \textgreek{[Greek]} fuisse feriam sextam.
\lnr{}Sed Doctor melius Latine intelligens, quam Graece, vulgatam tralationem [sic]
sequitur: \textit{In qua necessarium erat immolare}.
\lnr{}Nos negamus
\textgreek{[Greek]} bene traductum, \textit{necessarium erat}.
\lnr{}Atque adeo intererat Logici
scrire, quatenus \textit{Oportere, et Necessarium esse} differunt.
\lnr{}Absurde igitur, imperite, et adversus Evangelistarum mentem, dicitur
Christum crucifixum ipsa die solennis Paschatis.
\lnr{}Nos vero et
hic mutavimus sententiam, cum huic stultae interpretationi haereremus
priore editione: quemadmodum cum Chistum cyclo \textsc{xvi} crucifixum
asserebamus, sequuti Dionysium Exiguum, et alios veteres.

%\end{parnumbers}
%\clearpage
\pnr{p. XXIX [pdf 56]}
%\begin{parnumbers}
\lnr{}Nam Christus passus cyclo \textsc{xv} Lunae, \textsc{xiiii} Solis,
 tertia Aprilis,
anno quarto absoluto a baptismo, quando Sol extra ordinem caligavit,
cuius casus etiam meminit Phlegon, feria sexta, quando
post verum agnum immolatum, immolatus est et typicus, qui tum
primum perperam immolari ceptus, usque ad 70 annum Christi
Dionysianum, quando inclusis in urbem die primo Azymorum
contigit Pascha ultimum immolare.
\lnr{}Atque hactenus quidem de
priore parte sexti libri.
\lnr{}Venio ad alteram, cuius subiectum quo nobilius,
eo plures tractatores habuit: ut nullus non ex plebe scriptorum
ex hoc mustaceo lauream sibi quaesiverit.
\lnr{}Ac quanquam sine
summa doctrina externae historiae, et peritia bonarum literarum
ad ista arcana penetrari non potest, tamen quo quisque imparatior
ab omni copia humaniorum doctrinarum, eo audacius ad hanc
tractationem se contulit.
\lnr{}Quin etiam tantum abest, ut praesidia,
sine quibus hic labor irritus est, isti adhibuerint, ut eos insanire
putent, qui per illa viam sibi ad haec indaganda muniverunt.
\lnr{}Minima quaeque persequi esset horas perdere.
\lnr{}Tria praecipua attingere
satis pro tempore erit, nempe de septuaginta annis captivitatis,
de Regibus Persidis et Babyloniae, de epilogismo Hebdomadum
Danielis.
\lnr{}Septuaginta annorum caput a capto Iechonia sumendum
esse, auctor Ieremias scribens ad eos, quos com Iechonia Babylonem
Rex Nebuchodonosor deportaverat, cap. \textsc{xxix}, post
alia: \textit{Quia Dominus ita dicit: Quando septuaginta anni Babyloni completi
fuerint, ego visitabo vos, et verbum meum bonum super vos
suscitabo, ut vos huc reducam}.
\lnr{}Quid clarius hoc commate?
\lnr{}Vos,
quos cum Iechonia captivos Babylonem taduxit Rex, ego huc
reducam, postquam septuaginta anni completi fuerint Babyloni,
quae vos captivos detinet.
\lnr{}At contra hos 70 annos acuti homines
ineunt a capto Sedekia.
\lnr{}Ergo septuaginta anni sunt octaginta.
\lnr{}Mirum vero Ieremiam nescisse septuaginta esse septuaginta.
\lnr{}Quemadmodum
igitur negando parasceven Pascha esse parasceven
Pascha, res nova et inaudita concluditor, Spiritum sanctum in 
Apostolos descendisse Sabbato, non die Dominico: ita etiam
negando septuaginta annos esse septuaginta annos, haud dubie
aliquid \textgreek{[Greek]}, \textgreek{[Greek]} parturitur.
\lnr{}Audiamus
caussam tam inopinatae interpretationis.
\lnr{}Scriptum est, inquiunt, urbem
Hierosolyma per septuaginta annos sua sabbata requieturam.
\lnr{}Locos, quem designant, est in fine posterioris Chronicorum:
\textit{Ad complendum verbum Dei in ore Ieremia,
 donec terra acquiescat sabbatis
suis. omnes dies desolationis sabbatizavit, usque ad complendum
septuaginta annos}.

%\end{parnumbers}
%\clearpage
\pnr{p. XXX [pdf 57]}
%\begin{parnumbers}
\lnr{}Clare loquitur, omni ambage remota, terram,
quamdiu desolata fuit, sabbatizasse, id est, incultam cessasse, donec
complerentur anni septuaginta ab Ieremia determinati.
\lnr{}Quod
proculdubio verum est.
\lnr{}Nam finis desolationis est septuagesimus
annus: initium vero intra illos, non ab illis.
\lnr{}Nam tametsi terra septem
tantum annos, aut unum annum cessasset, tamen sequeretur,
quod hic dicitur, quandiu desolata suerit, cessasse: et quidem cessase,
usque ad septuagesimum annum ab Ieremia determinatum.
\lnr{}Quid verius, quid simplicius, quid clarius hac interpretatione?
\lnr{}Sane
ex his non colligitur, cessasse septuaginta annos, sed cessasse,
quandiu desolatio duravit: duravit autem usque ad tempus ab Ieremia
definitum, cuius temporis initium aliud est ab initio desolationis.
\lnr{}Huic simile, imo prorsus idem genus loquendi extat eodem capite,
commate 10.
\lnr{}Ioiachin cognomento Iechonias regnavit menses tres,
dies decem.
\lnr{}Tamen ibi dicitur: \textit{Anno vertente, Rex Nabuchodonosor
misit, et deportatus fuit Babylonem}.
\lnr{}Decimum diem mensis quarti
vocat annum vertentem.
\lnr{}Nimirum ille annus aliud habet initium
ab initio regni Iechoniae, cum tamen et anni et Regis Iechoniae
idem finis sit.
\lnr{}Sic initium septuaginta annorum aliud ab initio desolationis,
cum finis septuagesimi anni, et desolationis idem sit.
\lnr{}Neque sane ullus alius sensus hinc elici potest.
\lnr{}Et ita in eodem capite
duo loci similes alter alteri praelucet: quae locorum duorum
collatio mirifice sophisticam obiectionem retundit.
\lnr{}Sed isti boni
interpretes verbum divinum pro pila habent.
\lnr{}Et profecto aliquid
novi, ut dixi, pariet tam portentosum commentum.
\lnr{}Audite
ergo \textgreek{[Greek]}.
\lnr{}Annus capti Iechoniae est \textsc{xxvi} Nabopollassari,
ut supra demonstratum est.
\lnr{}Annus igitur excidii Hierosolymorum
et eversionis templi, est 36 Nabopollassari, qui erat 158 a
Thoth Nabonassari.
\lnr{}Ergo septuagesimus annus ab excidio templi
erit 227 ab eodem Thoth Nabonassari: qui est secundus Darii filii
Hystaspis, testibus observationibus Babyloniorum eclipticis apud
Ptolemaeum, nonus autem a decessu Cyri.
\lnr{}Quare hoc modo
Cyri annus primus, quo soluta captivitas, convenit in annum
nonum post eius obitum, ut volunt acuti homines, praestantissimi
interpretum, et sacrorum Bibliorum hierophantae, qui septuaginta
dicunt esse octaginta.
\lnr{}Tantae molis erat ostendere septuaginta
annos esse octaginta, ut aliquis Rex nono anno post obitum
suum edicta faceret.
\lnr{}Non igitur septuaginta anni a desolatione
ineundi, sed a servitute, et ex quo tempore Iudaei tributarii
Chaldaeis facti.
\lnr{}Itaque recte atque his convenienter apud Ieremiam
\textsc{xxv}, 11.

\textit{Omnis terra erit desolata, et vasta: atque omnes
illa gentes servient regi Babylonis septuaginta annos}.
\lnr{}In his non Iudaeos
tantum comprehendit, sed etiam omnem Syriam.

%\end{parnumbers}
%\clearpage
\pnr{p. XXXI [pdf 58]}
%\begin{parnumbers}
\lnr{}Nam Iudaei
non vocantur \texthebrew{[Hebrew]}, sed omnes alii extra iudaeos.
\lnr{}Itaque non solum
habitantes Hierosolyma, sed etiam omnes finitimas nationes desolatum
iri dicit, commate antecedente, usque ad septuaginta annos:
quorum initium a tempore subiugatae Palaestinae, et capti Iechoniae,
ut etiam clare extat apud Berosum: a cuius fragmento hinc
magna lux affulget.
\lnr{}Sequuntur Reges Babyloniae, et Persidis.
\lnr{}Quis narraverit somnia, hallucinationes, mendacia hominum in horum
Regum Chronologia?
\lnr{}Quid dicam odium, \textgreek{[Greek]} et \textgreek{[Greek]}
eorum in nos, quod praeter eorum expectationem eos Reges non
in \textgreek{[Greek]} Aristophanis, sed apud custodes priscarum Originum
Berosum Chaldaeum, Megasthenem, Herodotum invenerimus?
\lnr{}Quod eos ipsos in Daniele, Zacharia, Esdra, et Nehemia sine
ulla mutatione extare indicaverimus?
\lnr{}Quod ex duobus Regibus
Assuero, et Artaxerxe unum Regem bicorporem non fecerimus?
\lnr{}Berosi igitur et Megasthenis eximiae illae reliquiae apud Iosephum
nobis veritatis fontes recluderunt.
\lnr{}Earum beneficio habemus Reges
Chaldaeorum a Nabuchodonosoro, ad captam a Cyro Babylonem.
\lnr{}Ex quo captae Babylonis tempus iniri potest, cum is fuerit annus
\textsc{lxxxv} a Nabopollassaro patre Nabuchodonosori, ducentesimus
tricesimus sextus Iphiti,
 \textsc{xxi} imperii Cyri.Lib. \textsc{vi}, cap.de duabus
quaest.
\lnr{}Danielis \textsc{vii} annis minus diximus.
\lnr{}Quare corrigatur
numerus.
\lnr{}Sed prius quam ad reges Nabuchodonosori successores aggrediamur,
discutiamus opiniones interpretum Danielis, et Chronologorum.
\lnr{}Maxima pars horum hominum septuaginta annos a
Ieremia determinatos, a capto Sedekia deducunt, in primo anno
Darii Medi terminant, hoc uno argumento moti, quod Daniel primo
anno huius Darii mentionem facit vaticinii Ieremiae de septuaginta
annis, ideoque finem eorum annorum tunc accidisse dicunt.
\lnr{}Sed haec tam actuta sententia facile retunditur.
\lnr{}Nam eodem modo
a primo anno huius Darii ineundae essent Hebdomades, quia eodem
capite earum praecipua mentio fit, imo est unicum subiectum illius
capitis.
\lnr{}Ita sane pueri solent argumentari.
\lnr{}Verba Danielis: \textit{Ego
Daniel inellexi in literis numerum annorum}, etcetera.
\lnr{}Ego, inquit inter
legendum, animadverti septuaginta annos definitos captivitatis.
\lnr{}Non sequitur, illum fuisse septuagesimum captivitatis.
\lnr{}Rursus hunc
Darium volunt esse Astyagem Regem Mediae, filium Cyaxaris,
quem Daniel vocaverit Assuerum.
\lnr{}At quis unquam veterum classicorum
auctorum scripsit Medos Babylone imperasse?
\lnr{}Quis non
illorum dixit Astyagem a Cyro victum, et imperio exutum?

%\end{parnumbers}
%\clearpage
\pnr{p. XXXII [pdf 59]}
%\begin{parnumbers}
\lnr{}Nam Xenophontem tam constat historiam noluisse scribere, sed exemplum
bene educti principis proponere, quam certum est, nihil in tota Cyri
paedia verum esse, praeter sola nomina, et nudam mentionem duorum,
aut trium casuum, ut Babylonis captae, Croesi victi.
\lnr{}Sed tempora,
series gestarum rurum, anni imperii Cyri septem, ut ab illo
scriptore ponuntur, omnia, inquam, illa vera sunt, si vera sunt AEthiopica
Heliodori.
\lnr{}Neq;[?] tam sultus fuit Xenophon, ut crederet se
Graecis haec persuadere posse.
\lnr{}Praesto enim illi scriptis suis pudorem
imposuissent Castor Rheginus, qui sub Cambyse scripsit: Herodotus,
qui tempore Xerxis, et Artaxerxis Longimani floruit: Ctesias, qui in
aula Artaxerxis Memoris consenuit.
\lnr{}Sed fecit, quod \textgreek{[Greek]} aequales
suorum temporum faciebant.
\lnr{}Deligebant personas ex medio priscae
historiae, quas his coloribus pingebant, quibus putabant se animos
lectorum ad virtutem excitare.
\lnr{}Quare stulte faciunt, ne dicam
imperite, qui ex Xenophonte historiae Persicae veritatem petunt.
\lnr{}Qui ita faciunt, eo nomine solum indigni sunt, qui audiantur.
\lnr{}Atquin isti Regem Mediae hunc Darium faciunt, quem Graeci Astyagem
vocarent, proprio autem nomine Darius diceretur.
\lnr{}Obsecro
vos, docti interpretes, unde didicistis, Darium hunc regem fuisse?
\lnr{}Quis illius meminit praeter Danielem?
\lnr{}Consulendus igitur Daniel.
\lnr{}Initio \textsc{vi} capitis ita legimus:
 \textit{Darius Medus accepit regnum}.
\lnr{}Et initio \textsc{xi}: \textit{Primo autem anno Darii Medi}.
\lnr{}Dicite nobis, viri doctissimi, itane
Medus est Rex Mediae?
\lnr{}Quod hoc novum et insolens genus loquendi,
ut Medus sit Rex Mediae?
\lnr{}Si Medus omnis est Rex Mediae,
igitur quot Medi, tot erunt Reges Mediae.
\lnr{}Quare non dicitur Iosias
Iudaeus, pro rege Iuda?
\lnr{}Ioiakim Iudaeus?
\lnr{}Redite ad vos.
\lnr{}Videte, quid
designatis.
\lnr{}Non pudet argumenti?
\lnr{}Ostendite locum in toto Daniele,
in quo aut Nabuchodonosor et Balsasar Babylonius, aut Cyrus
Persa, pro Rege Babylonis, et Rege Persarum dicti sint sine mentione
regni: ut \textsc{vi}, 28.
\lnr{}Ibi enim dicitur regnum Cyri Persae.
\lnr{}Quare eum piguit dicere, \textit{Darius Rex Media accepit Regnum}?
\lnr{}Atqui maxime intererat
tantem rem non taceri.
\lnr{}Iudaeis siquidem scribit, quorum nemo
ignorabat Ioiakim Regem fuisse Iuda.
\lnr{}Quare igitur initio primi
capitis rem notissimam Iudaeis proponit, Ioiakim Regem Iuda, externum
autem regem regis nomine appellare non dignatur?
\lnr{}Quid?
\lnr{}Dicite aliquid.
\lnr{}Si aliud argumentum non habetis, quo nobis persuadeatis,
iam tempus est aut mutare sententiam, aut tacere.
\lnr{}Nam profecto iste Darius homo privatus ad regnum Babylonis accessit.
\lnr{}Scriptum
enim initio Capitis \textsc{ix}.

\textit{Primo anno Darii filii Assueri, de semine
Mediae}.
\lnr{}Latine ita concipiendum erat: \textit{Primo anno Darii filii Oxyaris,
oriundi ex Media}.
\lnr{}Non dicetis unde oriundi sumus, ibi nos natos
esse, sed ibi potius, unde genus deducimus.
\lnr{}Exemplum dandum
non erat rei, de qua non dubitatur.
\lnr{}Sed dabimus tamen.

%\end{parnumbers}
%\clearpage
\pnr{p. XXXIII [pdf 60]}
%\begin{parnumbers}
\lnr{}Stilicho
erat Wandalus, natus in media Wandalia: filius eius Eucherius natus
Romae erat ex semine[?] Wandaliae, ut verbis Danielis utar, non autem
Wandalus natione, quanquam ita vocari potuit, si \textgreek{[Greek]}
ita dictus fuisset eius pater: quemadmodum accidit huic nostro
Dario, qui Medus vocabatur, non quia in Media natus, sed
quia pater eius et Medus erat, et ita vocabatur, ut ab aliis civibus
Babyloniis homo inquilinus distingueretur.
\lnr{}Nam et ipse Assuerus,
sive Oxyares, quisquis is fuit, pater huius Darii, unus, ut apparet,
ex Megistanibus et proceribus Mediae, Babylonem immigravit,
sive \textgreek{[Greek]}, id est seditione aliorum procerum agitatus,
sive proditionis suspectus apud Regem Mediae.
\lnr{}Quaecunque fuerit
illa caussa patriam deserendi, certe homo privatus, non rex
(quod non tacuisset Daniel) Babylone habitavit.
\lnr{}Huius filius Darius,
cognomento Medes, aut Medus, cum aliis proceribus Babylonis
coniuratione in Balsasarem nepotem Nabuchodonosori Regem
Babylonis inita, signifer coniuratorum, Rege interfecto, ipse
in eius locum populi suffragiis vocatus est.
\lnr{}Ita enim Daniel: \textit{Primo
anno Darii filii Oxyaris ex Media oriundi, qui rex supra Chaldaeos
constitutus est}.
\lnr{}At volunt hunc Darium esse Astyagem, qui cum
Cyro obsederit Babylonem, et regem Balsasarem interfecerit: cuius
rei ne minima quidem suspicio extat in Daniele, sed manifesto
ibi dicitur noctu in convivio Balsasarem oppressum fuisse.
\lnr{}Qui
cum mille proceribus tam otiose epularetur, proculdubio necesse
est eum in secura urbe, non autem obsessa fuisse: quanquam simile
nescio quid narrat Herodotus.
\lnr{}Quid certius colligitur ex verbis
Danielis, quam ex condicto, in summa securitate, procul ab omni
metu hostium, interfectum fuisse?
\lnr{}Rex igitur iste Medes creatus
est, ut aperte declarat Daniel verbo
 \texthebrew{[Hebrew]}, quo significatur regem
creatum esse.
\lnr{}Itaque initio capitis \textsc{vi}, statim post caedem Balsasari,
adiicitur: \textit{Darius autem Medes accepit Regnum}.
\lnr{}Verbum,
quo hic utitur Daniel \texthebrew{[Hebrew]}
 significat non simpliciter sumere, sed
alio tradente accipere.
\lnr{}Neque unquam aliter usurpatur.
\lnr{}Unde doctrina,
quam non ex scripto, sed sola traditione percipimus, dicitur
\texthebrew{[Hebrew]}.
\lnr{}Si hoc considerassent, qui Darium, et eius patrem Reges
Mediae faciunt, et verba Hebraica Danielis perpendissent,
nunquam quae a nobis libro \textsc{vi} huius operis tam accurate de hac
re demonstrata sunt, reprehendo, imperitiam suam et \textgreek{[Greek]}
aperuissent.
\lnr{}Vides igitur Darium origine Medum, domo Babylone,
filium Oxyaris Medi, hominis privati, suffragiis populi regem
creatum, teste ipso Daniele.

%\end{parnumbers}
%\clearpage
\pnr{p. XXXIV [pdf 61]}
%\begin{parnumbers}
\lnr{}Apud Berosum igitur, Nabuchodonosoro
post annum regni sui quadragesimum tertium mortuo, filius
eius Hevvilmerodach, anno tertio regni a Neriglissororo, sororis
suae viro, interficitur.
\lnr{}Neriglissororus cum quatuor annos regnasset,
filio suo Laborosoarchodo, Nabuchodonosori ex filia nepoti
regnum reliquit: quo interfecto post menses novem, regnum Nabonido
cuidam, uni ex interfectoribus ipsius, communi consensu
populi traditum est.
\lnr{}Haec Berosus.
\lnr{}Quae omnia sacrae scripturae conveniunt,
si modo adiuventur interpretatione.
\lnr{}Nam profecto nihil
discrepant.
\lnr{}Intefecto enim Hevvilmerodacho, non est dubium
Neriglissororum populo persuasisse, se filio suo Nabuchodonosori
ex filia nepoti regnum conservare, donec adolesceret: ut si quod
desiderium populo esset Hevvilmerodachi propter Nabuchodonosori
memoriam, id omne leniretur, cum viderent ex stirpe Nabuchodonosori
adhuc superesse, qui in eius solio sederent.
\lnr{}Neque dubium est, quanuis re ipsa rex esset, nomine tamen filii regnum
administrasse.
\lnr{}Sane Ieremias cap. \textsc{xxvii}, 7, dixit omnes gentes subiectas
fore imperio Nabuchodonosori, filio eius, et filio filii.
\lnr{}Quanuis ex filia esset nepos, tamen filiorum nomine etiam filiae
 comprehenduntur.
\lnr{}Nam ut Latini liberorum nomine, sic Hebraei nomine
\texthebrew{[Hebrew]} utrumque sexum comprehendunt.
\lnr{}Laborosoarchodus igitur
est Balsasar.
\lnr{}Cui succedit Nabonidus, qui cum annos \textsc{xvii} regnasset,
deditione facta exutus regno, Carmaniam relegatus est a Cyro,
ut ait Berosus.
\lnr{}Megasthenes autem eximius scriptor ita scribit de
Nabuchodonosoro: \textgreek{[Greek](long quote)}.
\lnr{}In
quo utrique scriptori convenit.
\lnr{}Sed paulo antea idem Megasthenes
quendam Medem nomine cives Cyro dedidisse dicit, id est, Babylonem
diripiendam Cyro propinasse, sibi autem fuga salutem peperisse,
cum in munitissimam Borsippenorum arcem se recepisset.
\lnr{}Berosus hoc attribuit Nabonnido.
\lnr{}Nabonnidus igitur vocatus erat
Medes, ultimus Rex Babylonis, suffragiis populi rex creatus.
\lnr{}Apud
Danielem ultimus rex vocatur Medes, et creatus est suffragiis populi.
\lnr{}Darius igitur Medes est Nabonnidus.
\lnr{}Hunc Cyrus anno \textsc{xvii}
ipsius Darii, praelio victum, in arcem Borsippenam compulit.
\lnr{}Interea ipse Cyrus Babylone potitus anno imperii sui \textsc{xix}, aliquod
tempus rebus componendis dedit, et omnia urbis munimenta
quae longa obsidione, et summo labore ceperat, disturbavit, ut
scribit Berosus.
\lnr{}His ita gestis, ad Borsippum expugnandam animum
convertit.
\lnr{}Sed Nabonnidus, arcis deditione facta, et vita
et provincia Carmania donatus est.

%\end{parnumbers}
%\clearpage
\pnr{p. XXXV [pdf 62]}
%\begin{parnumbers}
\lnr{}Interea reliquias belli persequens
Cyrus ad fines usque Indiae profectus est, ubi Capissam urbem
ad Indum fluvium sitam diruit, ut scribit Solinus.
\lnr{}Itaque pacatis Orientis partibus Babylonem rediens, Susianam omnem iugum
detrectantem, imperio iam olim Babyloniae attributam, non
sine magne molimine debellavit, ut ex Strabone patet: quanquam
errat Strabo, dum ait Susianam a Cyro Medis ademptam.
\lnr{}Nam
potius Medis adempta, Chaldaeis adiuncta est: Danielis \textsc{viii}, 2.
\lnr{}Aeschylus enim in Persis arcem Susa a Cyaxare vastatam, ab Astyage
instauratam scribit.
\lnr{}Astyages autem a Cyro debellatus fuit.
\lnr{}Quare sub Astyage Susiana imperio Chaldaeorum adiecta, utique
post Nabonassarum, ut puto, et, si locus coniecturae est, ab illo
Assuero Medo prodita, cum eius fortasse Satrapes esset.
\lnr{}Reliqua,
quae retulimus, non a nobis, ut solent facere interpretes Danielis,
et Chronologi, sed a Daniele, Beroso, Megasthene accepimus.
\lnr{}Reversus igitur Babylonem victor Cyrus, omnibus imperio suo
subactis, tunc iure potuit dicere: \textit{Dominus Deus caeli dedit mihi omnia
regna terra:} quod quidem accidit anno imperii eius undetricesimo,
a Babylone capta nono, a capto Iechonia septuagesimo: ut
olim praedictum erat ab Ieremia; qui scripsit septuagesimum annum
captivitatis Iudaicae finem imperio Chaldaeorum impositurum.
\lnr{}Cap. \textsc{xxv}, 12. \textsc{xxvii}, 7. \textsc{xxix}, 10.
\lnr{}Hoc non potuit contingere,
nisi omni Assyria, et Babylonia subacta: quod tempus,
ut diximus, consurgit ex anno undetricesimo Cyri, qui est primus
annus imperii Babylonici, \textsc{ii} Paralip. \textsc{xxvi}, 22.
\lnr{}Esdrae \textsc{i}, 1.
\textsc{vi}, 3. et tertii anni mentio est Danielis \textsc{x}, 1: qui ei fuit
 ultimus et
 vitae et imperii.
\lnr{}Obiit enim anno uno et tricesimo imperii sui.
\lnr{}Annus primus Cyri caepit Olympiadis 55 primo, quae concurrit cum
anno Nabonassari 188.
\lnr{}Annus primus Cambysis, 219 Nabonassari.
\lnr{}Differentia initii Cyri, et Cambysis, anni solidi 30. a Thoth
189 Nabonassari.
\lnr{}Obiit igitur anno \textsc{xxxi} labente imperii sui Persici,
tertio autem imperii Babylonici, secundo et septuangesimo a
Iechoniae transportatione.
\lnr{}At isti somniant necessario Darium Medum
fuisse regem Persarum et Medorum, propterea quod cap. \textsc{v},
28 dicutur: \textit{Regnum tuum divisum est, et datum Medis et Persis.}
\lnr{}Itaque ex illa divisione coniiciunt, duos Reges designari, alterum
Mediae, alterum Persarum, et proinde necessario Astyagem, et
Cyrum fuisse.
\lnr{}Sed Astyages, ultimus Medorum Rex, ante multos
annos interfectus fuerat anno quatro Cyri.
\lnr{}Solus Cyrus Media et
Perside tunc potiebatur.
\lnr{}Praeterea diversis temporibus regnarunt,
non coniunctim, Darius Medus, et Cyrus. \textsc{vi}, 28.

%\end{parnumbers}
%\clearpage
\pnr{p. XXXVI [pdf 63]}
%\begin{parnumbers}
\lnr{}Itaque neque
duo tunc Reges erant, neque ex verbis Danielis colligitur ulla divisio.

\texthebrew{[Hebrew]} enim verbum, quo utitur Daniel, nihil aliud est, quam
\textgreek{[Greek]}, ut recte Iosephus interpretatur;
 id est, frangere: quo tantum
significatur mutatio regni.
\lnr{}Quandiu enim regnum a patribus
ad liberos iure successorio transfertur, tunc integrum vocari potest:
cum ad exteros transtertur, tunc frangitur.
\lnr{}Sic accipiendum \textsc{i}.
\lnr{}Reg. \textsc{xi}, 31.

\textit{Ecce, ego scindo regnum  de manu Solomonis.}
\lnr{}Nos negamus ullam \textgreek{[Greek]} eo significari.
\lnr{}Translatum igitur imperium
a Balsasaro, ad hominem Medi sanguinis, Nabonidum: a Nabonido
Medo, ad Cyrum Persam.
\lnr{}Itaque recte dicitur: Fractum est
imperium tuum, et datum Mediae, et Persidi.

\textgreek{[Greek]},
\textgreek{[Greek]}.
\lnr{}Hebraismus, vel Chaldaismus
\texthebrew{[Hebrew]}, Mediae, et Persidi, id est, hominibus Mediae et Persidis,
Satis enim est significari, imperium ad exteros transferri.
\lnr{}Itaque his convenienter
dixit Megasthenes, \textgreek{[Greek]}.
\lnr{}Nabonnidum, aut Nabonnidocum creant regem, qui
nullo iure originis aut sanguinis attingebat Balsasarum, ut pote qui
oriundus esset ex Medis.
\lnr{}Et \textgreek{[Greek]}, \texthebrew{[Hebrew]}.
\lnr{}Quo verbo utitur
Daniel \texthebrew{[Hebrew]} \textgreek{[Greek]}, creatus est Rex.
\lnr{}Non dixit \texthebrew{[Hebrew]} \textgreek{[Greek]}.
\lnr{}Huic obiectioni non possunt obluctari.
\lnr{}Sed nondum
finis velitationum.
\lnr{}Qui suffragiis populi a regnum adscitus est,
populum suis legibus vivere finit, neque novum ius sancire potest.
\lnr{}Ecce vero Darius legibus Persarum et Medorum edictum facit.
\lnr{}cap. \textsc{vi}, 8.
\lnr{}Ergo Darius erat Rex Persarum, et Medorum.
\lnr{}Perpendamus
mentem Danielis.
\lnr{}Omnes proceres imperii Darii convenerunt
in unum, ut legem fancirent, ne intra triginta dies ulli
Deo, aut homini, praeterquam Regi Dario, supplicaretur.
\lnr{}Est plane
Senatusconsultum, cui nisi auctor fiat Rex, est, tantum auctoritas
perscripta.
\lnr{}Aditur igitur Rex, ut id confirmet lege Persica et
Medica.
\lnr{}Commate 8.
\lnr{}Nos negamus hinc colligi Darium regem
Persarum, et Medorum fuisse.
\lnr{}Hoc enim ibi actum est, ubi Daniel
erat, nempe in praefectura ipsius Susiana, cuius erat Satrapes.
\lnr{}ca. \textsc{viii}, 2.
\lnr{}Qui enim unus ex tribus Megistanibus regni esset, ut est
\textsc{vi}, 2, non habitasset Susis sine imperio.
\lnr{}In Susiana igitur hoc contigit,
quae profecto nullis legibus Babylonicis tenebatur, sed suis
tantum, id est, Persicis.
\lnr{}Persis enim Suisianam ademit Astyages, ut
diximus ex Strabone.
\lnr{}Ea autem videtur sponte, aut aliquo alio casu,
potius quam belli fortuna Babyloniorum imperio accessisse.
\lnr{}Darius igitur, et superiores Reges Babyloniae, in Suside nihil poterant
sancire nisi iure Persico: quod et hodie in omnibus regnis usurpari
videmus.
\lnr{}Cum enim uni regi multa parent regna; non est concludendum,
illa omnia temperari eisdem legibus.

%\end{parnumbers}
%\clearpage
\pnr{p. XXXVII [pdf 64]}
%\begin{parnumbers}
\lnr{}Imo multa eiuscemodi regna sub uno Principe contrariis legibus videmus regi.
\lnr{}Nostros vero disputatores fugit res pulcherrima, haec contigisse
Susis, non Babylone: Babylonem autem Chaldaico iure, Susianam
Persico seu Medico, quod idem est, usam suisse, ubi proceres Susiani
non Babylonii iudicabant.
\lnr{}Neque locus ullus altercationi.
\lnr{}Nam actum prorsus est de illorum Dario Astyage, cum Astyages diu antea
caesus fuerit: neque meliore conditione in societatem regni cum
Dario divisi Cyrum vocant, cum et Astyages captus sit a Cyro, et
Darius Medus ante Cyrum aliquot annis imperarit, ut ex \textsc{vi}, 28.
non colligitur tantum, sed et convincitur.

\textit{Daniel iste prospere egit in regno Darii, et in regno Cyri Persa.}
\lnr{}Omnino vero tota Susiana a Daniele regebatur.
\lnr{}Sed et arcem munitissimam, quam \textgreek{[Greek]}
Graeci vocant, Ecbatanis Mediae miro opere a Daniele constructam
fuisse testatur Iosephus, eamque tempore ipsius Iosephi adeo
integram mansisse, ut recens videretur, cui semper Iudaeum praefici
ex genere sacerdotum, morem fuisse scribit.
\lnr{}Sed Ecbatana nunquam
Babyloniis parverunt ante tempus Cyri, hoc est, antequam
Cyrus Babylone potitus est.
\lnr{}Nam ipse erat Rex Mediae et Persidis.
\lnr{}Daniel vero sub Cyro adhuc praefectus erat Susidis. \textsc{x}, 4.
\lnr{}Fluvius
enim Tigris, cuius ibi mentio fit, est terminus Susidis ab Occidente.
\lnr{}Fabulam vero Iosephi Iudaicam reiicimus de illa arce Ecbatanorum.
\lnr{}Nam Daniel octagesimo anno captivitatis suae scripsit caput
\textsc{x}, et nondum ea arx condita erat.
\lnr{}Haec igitur nos ex antiquissimis,
et accuratissimis scriptoribus Beroso, et Megasthene eruimus, quorum
fragmenta, quae extant apud Iosephum, omnia obscura in Daniele
de Regibus illustrant.
\lnr{}Quis enim satis ea laudare possit?
\lnr{}Tria verba ex ipsis tot commentatiorum moles, tot somnia interpretationum
eludunt.
\lnr{}Si cui sacrae historiae studioso et veritatis amanti
vacat eas reliquias apud Iosephum, et Eusebium sparsas colligere,
earumque mirum cum sacris libris consensum indicare, habebit
fatis, unde et egreium opus instruat, et posteris aliam viam
muniat.
\lnr{}Chronologiam sacram investigandi, quam hactenus
ab ullo veterum aut recentiorum factum sit.
\lnr{}In Megasthenis eximio fragmento non solum ultimum Regem, qui cives
 \textgreek{[Greek]}, id
est Cyro prodiderit, Medem vocatum fuisse animadvertet, quem
Babylonii mutato nomine \texthebrew{[Hebrew]} Nebonit, Herodotus Lebonit,
aut Labynitum vocaverit, sed etiam aliquid, quasi per caliginem
de Nabuchodonosoro indicari, eum intellectu alienato in deferta
sese abdidisse, ut patet cap. \textsc{iiii}.

%\end{parnumbers}
%\clearpage
\pnr{p. XXXVIII [pdf 65]}
%\begin{parnumbers}
\lnr{}Apud Berosum autem, praeter
regum Babyloniae successionem, inveniet etiam Babylonem moenibus
miro opere, non a Semiramide, ut nugantur Graeci, sed a Nabuchodonosoro
circumdatam fuisse, ut testatur Daniel \textsc{iiii}, 27.
\lnr{}Et
multa praeterea, in quibus illustrandis labor iucundus est, fructus
autem uberrimus.
\lnr{}Sed et non minor utilitas colligitur ex fragmento
Menandri Ephesii apud eundem Iosephum, qui ab Hiram, sive
Iromo Rege, qui ligna cedrina Solomoni suppeditavit, ad Pygmalionem
fratrem Didonis, omnes Reges Sidoniorum et Tyriorum
recenset.
\lnr{}Primus est \textgreek{[Greek]}, \texthebrew{[Hebrew]}
 filius \textgreek{[Greek]} \texthebrew{[Hebrew]}.
\lnr{}Secundus filius Iromi \textgreek{[Greek]}, \texthebrew{[Hebrew]}
 regnavit annos septem.
\lnr{}Huius filius \textgreek{[Greek]}, \texthebrew{[Hebrew]} annos novem.
\lnr{}Quo a filiis nutricis
interfecto, maximus natu filiorum nutricis \textgreek{[Greek]} regnavit
biennium.
\lnr{}Post hunc \textgreek{[Greek]}, \texthebrew{[Hebrew]} annos
duodecim.
\lnr{}Huic succedens frater \textgreek{[Greek]}, \texthebrew{[Hebrew]} annos novem.
\lnr{}Quo interfecto a fratere Phelete, ipse
 \textgreek{[Greek]}, \texthebrew{[Hebrew]}, menses octo.
\lnr{}Hoc interfecto \textgreek{[Greek]} \texthebrew{[Hebrew]}
 Sacerdos Astartes triginta
duos.
\lnr{}Huius successor \textgreek{[Greek]} \texthebrew{[Hebrew]} annos sex.
\lnr{}Huius filius
\textgreek{[Greek]} \texthebrew{[Hebrew]} novem.
\lnr{}Successit \textgreek{[Greek]} \texthebrew{[Hebrew]}, qui regnavit
annos quadraginta septem.
\lnr{}In his videmus Iromum, qui, ut
diximus, cedros Solomoni subministravit, item Ithobaalum, cuius
filiam Iezabelam uxorem duxit Omni Rex Samariae. \textsc{i}. Reg \textsc{xvi},
31.
\lnr{}Ab Iromi obitu, quem duodecimo anno, quo templum aedificatum
est, decessisse scribit Iosephus, ad septimum annum Pygmalionis,
quo Dido Carthaginem condidit, annos fluxisse Iosephus
scribit 155, menses octo.
\lnr{}Sed fugit eum ratio.
\lnr{}Comprehendit
enim duedecim annos Hiromi, qui excluduntur per hypothesin,
et praeterea totos 47 Pygmalionis, cum quadraginta et
unus excipiendi sint.
\lnr{}Ita ab obitu Iromi, ad Carthaginis conditum
fuerint anni tantum 103.
\lnr{}Sed de his omnibus Regibus, et
eorum annis, et quis fructus ex huius fragmenti lectione colligi
possit, alibi plenius dicemus in libello Fragmentorum.
\lnr{}Apud Tertullianum Apologetico, cap. \textsc{xix}.

\textit{Hieronymus Phoenix Rex Tyri.}
\lnr{}Lege \textit{Hiromus Phoenix}.
\lnr{}Sed tamen interea aliquid de Carthagine
dicendum.
\lnr{}Carthago deleta est anno ab urbe condita 606,
periodi Iulianae 4567, annis 737, postquam a Didone condita,
auctore Solino.
\lnr{}Deductis 737 de 4567, relinquitur annus
septimus Ithobaalis 3830 in Periodo Iuliana.
\lnr{}Omri Rex Samariae
gener Ithobaalis caepit regnare anno 578 ab Exodo, id
est 3795 periodi Iulianae.
\lnr{}Ab initio Ithobaalis, ad annum Pygmalionis
septimum, anni sunt absoluti 53.
\lnr{}Quibus deductis de
3830, remanet annus 3777. in periodo Iuliana, primus regni
Ithobaalis.
\lnr{}Inivit igitur regnum \textsc{xviii} annis ante quam gener eius
Omri in Samaria regnaret.

%\end{parnumbers}
%\clearpage
\pnr{p. XXXIX [pdf 66]}
%\begin{parnumbers}
\lnr{}Quanquam igitur haec non sufficiunt
ad stabiliendam regum Samariae epocham, quia in scriptura mentio
non fit quoto anno regni sui Ithobaal filiam suam Regi Omri desponsaverit,
neque quoto anno regni sui illam Omri duxerit uxorem:
tamen certissimum tempus Omri hinc collectum eludit temeritatem
eorum, qui tempora regum Samariae longius summovent
immani intervallo.
\lnr{}Praeterea hinc colligitur intervallum
ab Ilii excidio, ad conditum Carthaginis, nimirum anni 299.
\lnr{}Tot annis AEneas Didonem antecessit.
\lnr{}Qui reges Tyriorum et Sidoniorum
postea regnarint, Iosephus recenset ex Annalibus Phoenicum:
quorum primus \textgreek{[Greek]} \texthebrew{[Hebrew]}
 cum descivisset a Babyloniis,
expugnata Tyro a Nabuchodonosoro, interfectus est.
\lnr{}Reliqui reges, qui postea Tyriis, et Sidoniis imperarunt, tributarii
Chaldaeis facti: quod nec tacuit Ieremias \textsc{xxv}, 11.
\lnr{}Primus
tributariorum \textgreek{[Greek]} \texthebrew{[Hebrew]} regnavit annos decem.
\lnr{}Post hunc Iudex
\textgreek{[Greek]} \texthebrew{[Hebrew]} menses duos.

\textgreek{[Greek]} \texthebrew{[Hebrew]} menses decem.

\textgreek{[Greek]} \texthebrew{[Hebrew]} tres.

\textgreek{[Greek]} \texthebrew{[Hebrew]}
annos sex.
\lnr{}Sed inter illos duos fratres refnavit \textgreek{[Greek]} \texthebrew{[Hebrew]}
annum unum.

\textgreek{[Greek]} arcessitus Babylone \texthebrew{[Hebrew]} annos quatuor.
\lnr{}Frater huius \textgreek{[Greek]} \texthebrew{[Hebrew]} anno viginti.
\lnr{}Huius anno \textsc{xiiii}
incipit regnare Cyrus, referente eodem Iosepho ex iisdem annalibus
Phoenicum, et septimo anno regni Ithobaal, Nabuchodonosor
caepit oppugnare Tyrum.
\lnr{}Auctor idem Iosephus ex iisdem
annalibus.
\lnr{}Summa annorum ab initio oppugnationis Tyri, ad
\textsc{xiiii} Iromi, anni 35, menses tres.
\lnr{}Annus primus Cyri est 187 Nabonassari.
\lnr{}Deductis 35 absolutis, remanet annus captae Tyri 152
Nabonassari, qui erat tertiusdecimus Nabuchodonosori, ultimus
Nabopollassari.
\lnr{}Tot igitur annos impendit oppugnandae munitissimae
urbi.
\lnr{}Quae omnia conveniunt cum illis, quae Berosus de Syria
domita circa captum Iechoniam, et 30 annum Nabopollassari scribit.
\lnr{}Quam bene haec confirmant de Nabopollassaro patre Nabuchodonosori,
et de 30 annis Nabopollassari apud Ezekielem?
\lnr{}Haec Tyri expugnatio ab Ieremia praedicta fuerat cap. \textsc{xlvii}, 4.
\lnr{}Sed iterum
Tyrus rebellavit, et capta ab eodem Nabuchodonosoro post
excidium Templi.
\lnr{}Ezekiel \textsc{xxvi}. cuius annum ignoramus.
\lnr{}Apud
Iosephum mendose colliguntur ab initio Baal, ad \textsc{xiiii} Iromi, anni
54, ut ex dinumeratione constat.
\lnr{}Est error librariorum.
\lnr{}Nam ab expugnata
Tyro, ad initium Cyri, ex hac dinumeratione anni tantum
35, menses 3 colliguntur.
\lnr{}Bearunt nos et hoc Annalium Phoenicum,
et reliqua fragmenta veterum scriptorum apud Iosephum, quae valde
frenant Chronologorum licentiam.

%\end{parnumbers}
%\clearpage
\pnr{p. XL [pdf 67]}
%\begin{parnumbers}
\lnr{}Apposuimus Hebraica
nomina, ut videant studiosi, Sidoniorum, et Tyriorum linguam
Hebraicam fuisse.
\lnr{}Nam et Dido \texthebrew{[Hebrew]} est \textgreek{[Greek]}, quo nutrices
infantibus blandiuntur, cuius masculinum est David \texthebrew{[Hebrew]}.
\lnr{}Elisam
vero \texthebrew{[Hebrew]} satis constat Hebraeum esse.
\lnr{}Praeterea Iosephus refert
ex Theophrasto \textgreek{[Greek]}, Tyriis et Sidoniis nunquam licuisse
iurare nisi \textgreek{[Greek]},
 inter quos ponit \textgreek{[Greek]}, id est, inquit
Iosephus, \textgreek{[Greek]}.
\lnr{}Mediocriter periti Hebraismi sciunt \texthebrew{[Hebrew]} esse
donum sacrum.
\lnr{}Hoc plane demonstrat, Tyriorum et Sidoniorum
dialectum Hebraicam fuisse.
\lnr{}Et sane mirum est illis sanctissimum
fuisse iusiurandum Korban, ut Iudaeis
 \texthebrew{[Hebrew]} vel \texthebrew{[Hebrew]}.
\lnr{}Chaiala,
vel Chaiadonai.
\lnr{}Sed tempore Martialis pronunciabant Chiala.

\textit{iura verpe, per Anchialum.}
\lnr{}Quia Iudaeos audiebat Chiala iurare:
propterea putavit esse Anchialum, quod vox Anchialus sit notior,
quam Chaila.
\lnr{}Tempore autem Plauti, multum Carthaginienses
a puro Tyriorum sermone desciverant, ut patet ex Poenolo:
cuius verba Punica maiori ex parte recte reddere possumus.
\lnr{}Quae propius absunt ab Hebraismo, quam a Syriasmo.
\lnr{}Tamen
si Solinum audimus, Karthago dicta est ab ipsis Poenis Carthada,
\texthebrew{[Hebrew]}.
\lnr{}Quod Syriace esset \texthebrew{[Hebrew]}.
\lnr{}Ita opera in istis fragmentis
bene collocatur.
\lnr{}Quae si prius considerassent qui nos reprehendunt,
ii tam iniuste nos reprehendi, quam se turpiter hallucinari
sensissent.
\lnr{}Alios omitto, quos ne nominatos quidem velim.
\lnr{}Unum Bened.
\lnr{}Pererium convenio, qui in suo in Danielem commentario,
et alios et me in primis reprehendit, quod absurditates
omnes, quas supra confutavimus, quasque nemo sanae mentis
probaverit, non sequor.
\lnr{}Ipse vero postquam omnium adductis sententiis,
pessimas quasque et maxime ridiculas sequitur, quibus verbis
castigandus est!
\lnr{}Is est, qui septuaginta annos a capto Sedekia
deducit, et in primo anno imperii Persici Cyri, id est, in primo
anno Olympiadis \textsc{lv} terminat.
\lnr{}Quod quantum absurdum sit, ex
illis, quae supra a nobis disputata sunt, satis docuimus.
\lnr{}Denique
nihil ram absurdum ab aliis in hac re excogitatum est, quod ipse
melioribus non praetulerit.
\lnr{}Quare in Chronologia Danielis plane
puer est.
\lnr{}Prius discere, quam alios reprehendere debebat.
\lnr{}Quantus liber institui posset in eius erroribus confutandis!
\lnr{}Si mihi excidissent
ea, quae ille scribit, non inultum tulissem, quae est Aristarchorum
meorum insania, et procacitas.
\lnr{}Equidem horum reprehensorum,
qui me doctiorem non reddunt, iudicia non pluris
facio, quam ipsi veritatem, quam eis aperuimus: eam autem illi aut
propter imperitiam non agnoscunt, aut propter malitiam dissimulant.
\lnr{}Sed ad alia propero.
\lnr{}Et haec quidem de Regibus Babylonis.

%\end{parnumbers}
%\clearpage
\pnr{p. XLI [pdf 68]}
%\begin{parnumbers}
\lnr{}De Persarum Regibus, quomodo illi ex Daniele, et Esdra erui possint,
satis libro \textsc{vi} docemus.
\lnr{}Quod Xerxes Assuerus sit apud Esdram
\textsc{iiii}, 6. satis ex contextu sermonis colligitur.
\lnr{}Omni tempore
Cyri, et Darii Regis Persarum, opus templi interpellatum est.
\textsc{iiii}, 5.
\lnr{}Post Darium initio regni Assueri instituerunt accusationem
in Iudaeos.
\lnr{}Commate 6.
\lnr{}Postea imperante Artaxerxe, Mithridates,
et alii hostes Iudaeorum Artaxerxi scribunt in Iudaeos.
\lnr{}Commate 7.
adeo ut ab Artaxerxe illo, ad tempora Darii alius, aedificatio impedita
fuerit.
\lnr{}Commate 24.
\lnr{}Anno illius Darii Haggaeus Propheta
et Zacharias exhortantur Iudaeos ad opus continuandum.

\textsc{v}, 1.
\lnr{}Post hunc Darium sequitur Artaxerxes, qui Esdram septimo anno
imperii sui in Palaestinam misit.
\lnr{}Iam vero Darius nullus est post Cyrum,
praeter Darium filium Hystaspis.
\lnr{}Hoc ne ipsi quidem negant.
\lnr{}Ecce opus disturbatum a primo anno Cyri, ad finem darii.
\lnr{}Ergo sub quatuor Regibus disturbatum, Cyro, Smerdi, Cambyse,
Dario.
\lnr{}Inter Darium Hystaspis, et Artaxerxem, quis esse potest,
praeter Xerxem?
\lnr{}Assuerus est medius inter Darium et Artaxerxem.
\lnr{}Ergo Assuerus, sive, ut pronunciari debet, Oxyares, est Xerxes, ditissimus
regum Persidis.
\lnr{}Daniel \textsc{xi}, 2.
\lnr{}Et eius opibus ac potentiae historiae
Graecorum sidem faciunt.
\lnr{}At maritus Ester vocatur Assuerus.
\lnr{}Et praeterea ditissimus, et latissime imperans, ut est in libro
Ester.
\lnr{}Ergo Assuerus Esdrae et maritus Ester unis idemque est.
\lnr{}Rursus Assuerus Esdrae est Xerxes.
\lnr{}Ergo maritus Ester est Xerxes.
\lnr{}Quid?
\lnr{}Quod Xerxis uxor vocatur Amistris, Persice Am-Ester.
\lnr{}Quid multa alia?
\lnr{}Et tamen en \textgreek{[Greek]}, en succum loliginis.
\lnr{}Quid obiiciunt?
\lnr{}Audire, atque togam iubeo componere.
\lnr{}Si, inquiunt, Affuerus est Xerxes;
Mardochaeus, qua gratia valebat apud Regem, non sivisset opus
templi impediri.
\lnr{}Quot modis haec confutari possunt, si tanti essent!
\lnr{}Adversus tot \textgreek{[Greek]} argumenta quem locum
 habet tam ludicra obiectio?
\lnr{}Quid si oravit regem Mardochaeus, et per Satrapas stetit, quo
minus edicto Regis pereretur, ut sub Cyro contigit?
\lnr{}Nam si Cyro vivente,
eius edictum flocci fecerunt hostes Iudaeorum, quanto magis
gratiam Mardochaei?
\lnr{}Quid si ne Iudaei quidem curarunt ea de re Mardochaeum
compellare, ut non Mardochaei, sed ipsorum Iudaeorum
culpa fuerit?
\lnr{}Ecce \textsc{xxxvii} annis post emissum a Dario Notho edictum
de reficiendis muris Hierosolymorum, Nehemias accipit nuncium,
non solum non refectos muros esse, sed neminem de reficiendis
quidem curare.
\lnr{}Haec negligentia nonne potuit incessere Iudaeis sub
Xerxe, quo imperante accusatio in Iudaeos instituta est, cum sub Artaxerxe
Memore, qui Iudaeis favebat, tanta socordia in illis fuerit?
\lnr{}Quare, ut quidam loquitur, hoc telum stringit, non perforat.

%\end{parnumbers}
%\clearpage
\pnr{p. XLII [pdf 69]}
%\begin{parnumbers}
\lnr{}Quibus tot argumenta satis non sunt ad verum persuadendum, furor est
cum illis rem habere.
\lnr{}A Cyri anno primo, ad ultimum Darii, opus
templi disturbatum.
\lnr{}Esdrae \textsc{iiii}, 5.
% Ezra 4:5 "and hired counselors against them to frustrate their purpose
% all the days of Cyrus king of Persia, even until the reign of Darius
% king of Persia."
\lnr{}A principio Assueri, qui est
Xerxes, toto tempore ipsius Assueri, et Artaxerxis, usque ad secundum
annum Darii, non iam moras obiiciendo, sed calumniis apud
reges Xerxen, et Artaxerxen Iudaeos incessendo, inimici eorum
non solum opus impediverunt, sed Iudaeos in discrimen maximum
adduxerunt.
\lnr{}Esdrae \textsc{iiii}, 6, 7, 11, 12, 13, et sequentibus.
\lnr{}Qua de re
anno secundo Darii apud Zachariam conqueritur Angelus: \textit{Domine
Deus exercituum, quousque non miseraberis Hierosolyma, et
urbes Iuda, quibus iratus es, iam hic agitur septuagesimus annus?}
% Zehariah 1:12
\lnr{}En disturbatio septuaginta annorum.[?] cui loco hostes veritatis contradicere
non possunt.
\lnr{}An dicent illum Darium esse filium Hystaspis?
\lnr{}Annus secundus Darii in periodo Iuliana 4194. Deductis septuaginta
relinquitur annus 4124, id est, tertius Olympiadis 47,
quo captus Sedekias.
\lnr{}Quod est absurdum.
\lnr{}Ergo Darius, sub quo
prophetabat Zacharias, est alius a filio Hystaspis.
\lnr{}Est igitur Darius
Nothus.
\lnr{}Esdras habet a Xerxe, ad secundum annum Darii.

\textsc{iiii}, 24.
% Ezra 4:5 "Then ceased the work of the house of God which is at Jerusalem.
% So it ceased unto the second year of the reign of Darius king of Persia."
\lnr{}Et post illum Darium, Artaxerxen alium.

\textsc{vii}, 1.
%Ezra 7:1 "Now after these things, in the reign of Artaxerxes king of Persia,
% Ezra the son of Seraiah, the son of Azariah, the son of Hilkiah,"
\lnr{}Quis
est ille Artaxerxes post Xerxen, praeter Artaxerxen Longimanum?
\lnr{}Quis Darius post illum Artaxerxen, et ante alium Artaxerxen, praeter
Darium Nothum medium inter duos Artaxerxes, Longimanum,
et Memorem?
\lnr{}Prius confutanda sunt verba Esdrae, antequam
mea sententia oppugnetur.
\lnr{}Prius Esdras mendacii convincendus
est, quam ego.
\lnr{}Aut si ego mendax, cum Esdra ergo mentior.
\lnr{}Negent haec ita extare apud Esdram.
\lnr{}Manus dabo.
\lnr{}Quam inepti
sunt, quam maligni, qui haec negant!
\lnr{}Quanto odio veritatis hoc faciunt!
\lnr{}Oportebat Pererium negare apud Esdram extare, opus disturbatum
fuisse a primo anno Cyri, ad ultimum Darii: post Darium
toto tempore Assueri, et Artaxerxis, Iudaeos apud Regis Persidis
accusatos, usque ad secundum annum Darii, quo quidem anno
Darii secundo templum aedificari caeptum, et post illum Darium,
anno septimo alius Artaxerxis.
\lnr{}Esdras in Palaestinam missus.
\lnr{}Si hoc potest negare, tunc audacter et nos vera dixisse negato.
\lnr{}Sed profecto
in istis confutandis tantum aestuabit, quantum in suo Belo et
Dracone, et in suo Astyage rege Babylonis, iam tum cum Daniel
esset puer: et in aliis, quorum me pudet: in quibus satis ostendit,
quam bene de vera Chronologia meritus est.
\lnr{}Certe, qui tantum in
illis probandis laborat, non mirum si tam confidenter alia negat, ut
tandem Cyrum nono anno post obitum suum edicta facientem inducat.

%\end{parnumbers}
%\clearpage
\pnr{p. XLIII [pdf 70]}
%\begin{parnumbers}
\lnr{}Miratur, quod ex Beroso successionem regum Babylonis, et
ex Herodoto petamus. Malvisset, ut ex Isidoro populari suo illos
reges emendicassem.
\lnr{}Sed satis eum refellunt quae superius, et libro
\textsc{vi} huius operis a nobis dicta sunt.
\lnr{}Darius igitur iste inter duos Artaxerxes
positus, is est Darius Nothus, qui edictum fecit de Hierosolymis
instaurandis.
\lnr{}At a relaxata captivitate, inquiunt, ad secundum
Darii Nothi, anni sunt \textsc{cv}: et adhunc vivunt Iosedek, et Zerubabel.
\lnr{}Valde senes igitur eos oportet fuisse: quod non videtur.
\lnr{}Quare non videtur?
\lnr{}Quare mirum, illos duos accessisse ad aetatem
Levi, Caath, et Amram, imo ad eam aetatem, quam quotidie videmus
in multis?
\lnr{}Hoc argumentum satis suo loco confutatum.
\lnr{}Sed
valde hallucinantur, cum Nehemiam, Esdrae \textsc{ii}, 2, eundem faciunt
cum illo Nehemia, qui muros Hierosolymorum instauravit.
\lnr{}Quod et ipsum alibi confutavimus.
\lnr{}Nam Nehemias Hierosolymorum
instaurator pervenit ad tempora Darii Codomanni;
cuius ipse mentionem facit manifesto Cap. \textsc{xii}. 22.
\lnr{}Tempore Iadduae,
qui sub Alexandro Magno vixit, et Sannaballetis Samaritani,
qui ad Alexandrum in oppugnatione Tyri cum auxiliaribus
copiis venit, et ab eo, ut sibi liceret templum in monte Garizin
construere, impetravit.
\lnr{}Hoc argumentum pertinaciae iugulum
petit: in quo ictu declinando valde laborant.
\lnr{}Postremo
eo impudentiae venerunt, ut dixerint caput \textsc{xii} Nehemiae insititium
esse.
\lnr{}Quid?
\lnr{}Expecto, ut dicant, Esdram, et Nehemiam aut
mentitos esse, aut sua tempora ignorasse.
\lnr{}Sed quid exspectandum
est?
\lnr{}An aliud volunt, cum impudentissime omnia, quae ab Esdra,
et Nehemia dicta sunt, negant?
\lnr{}Cum igitur Nehemias dinumerat
a Iosue, ad suum saeculum, generationes quinque; ipse
Nehemias non potest esse is, qui cum Iosue venit Hierosolyma.
\lnr{}Ecce sex summi Pontifices a soluta captivitate, ad tempora Nehemiae:
Iosue filius Iosedeck, Ioiakim filius eius, Eliasib nepos,
Ioiada pronepos, Ionathan abnepos, Iadua adnepos sub Alexandro,
adhuc vivente Nehemia.
\lnr{}Istius Iaduae patruus duxerat uxorem
filiam Sannaballetis Samaritae.
\lnr{}Nehem. \textsc{xiii}. 28.
\lnr{}Ille vocabatur
Manasses, illa Nicaso.
\lnr{}Ioseph. \textsc{xi}, 7.
\lnr{}Certe adversus haec
nihil potest Sophistice.
\lnr{}Supersunt septuaginta Hebdomades, quarum,
ut omnium fere prophetiarum fuarum, interpres ipse est Daniel.
\lnr{}Nam et principium, et finem digito indicavit: principium ab
edicto de instaurandis Hierosolymis, finem in desolatione Hierosolymorum:
quod sane rationi convenientissimum est, ut scilicet eae
Hebdomades ad initium et finem eiusdem subiecti, nempe Hierosolymorum,
pertineant.

\textit{Septuaginta hebdomades determinatae[?] sunt
super populo tuo, et sancta civitate tua.}

%\end{parnumbers}
%\clearpage
\pnr{p. XLIV [pdf 71]}
%\begin{parnumbers}
\lnr{}An non circumscribit eas
excidio urbis?
\lnr{}Quid enim volunt illa?

\textit{Populus Principis destruet civitatem,
et Sanctuarium?}
\lnr{}Item: \textit{Desolatio erit usque ad consummationem.}
\lnr{}Quis tam caecus, ut in tanta luce nihil videat?
\lnr{}Quis tam ferrea fronte, ut adversus haec hiscere audeat?
\lnr{}En Hebdomades
Danielis Passionem Christi, et Hierosolymorum eversionem complectuntur.
\lnr{}Et tamen alii excidium Hierosolymorum, alii etiam
Passionem hinc excludunt.
\lnr{}Non igitur obscurum est, unde deducantur
hae septimanae, quandoquidem ubi desinant, ipsemet Daniel
ostendit.
\lnr{}Sed et initium ostendit: \textsc{ix}, 25.
\lnr{}Quod quidem initium,
quia controversum est, propter hominum summum in veritatem
odium: tamen septuaginta septimanis a 70 anno Christi
Dionysiano, in quo excidium incidit, retro putatis, pervenientur
ad initia earum.
\lnr{}Sed dimidia septimana, cuius mentio fit commate
ultimo, et nostros et omnium oculos effugerat.
\lnr{}Putant enim,
ut et nos aliquando putavimus, esse partem unius ex septuaginta.
\lnr{}Atqui ex usu Hebraismi nos tandem perceptimus esse extra illas septuaginta:
neque esse dimidium unius ex septuaginta, sed tantum
illos tres annos cum dimidio supra 490 intelligendos esse.
\lnr{}Quae
quidem animadversio ut notabilis, ita in ea non parva pars huius
negotii vertitur.
\lnr{}Summa igitur horum annorum 493 cum dimidio.
\lnr{}Annus captorum Hierosolymorum in periodo Iuliana 4783.
\lnr{}De quibus deductis 493 absolutis, relinquitur annus 4290 in periodo
Iuliana, initium septuaginta Hebdomadum: qui profecto
est annus secundus Darii Nothi, illius, inquam, qui anno secondo
regni sui edictum emisit de non impediendo templi aedificio.
\lnr{}Esdrae \textsc{vi}.
\lnr{}Vidimus ex iis, quae supra ex Esdra adduximus, toto tempore
Assueri, et Artaxerxis, usque ad secundum annum Darii, propter
literas ad Artaxerxem a Iudaeorum hostibus scriptas, aedificationem
interpellatam.
\lnr{}Et tamen extiterunt, qui a vicesimo anno
Artaxerxis, hebdomadum initium deducant.
\lnr{}In quo primum
peccant, quod a temporibus illius regis deducunt, cuius toto tempore
cessavit aedificatio.
\lnr{}Quod plane est dicere, Esdram falli, aut mentiri.
\lnr{}Deinde ostendant quodnam edictum missum est anno vicesimo
Artaxerxis.
\lnr{}Dicant mihi, quare Nehemias non solum miratur,
sed etiam dolet, moenia Hierosolymorum adhuc diruta iacere,
nisi quia putabat iam instaurata esse?
\lnr{}Si putabat, ergo ex
aliquo edicto.
\lnr{}Quodnam aliud edictum fuerit, praeter edictum
Darii?

%\end{parnumbers}
%\clearpage
\pnr{p. XLV [pdf 72]}
%\begin{parnumbers}
\lnr{}Tractorias tantum Regis impetravit Nehemias ad Praefectos
transeuphratensis Satrapiae, ut liber accessus ei permitteretur
in Palaestinam, item literas ad Asaph saltuarium, id est silvae regiae
custodem, ut materia sibi subministraretur ex silva regia, ad
portas reficiendas.
\lnr{}Nehemiae \textsc{i}, 7, 8, 9.
% 7 We have dealt very corruptly against Thee, and have not kept the
% commandments, nor the statutes, nor the judgments which Thou commanded
% Thy servant Moses.
% 8 Remember, I beseech Thee, the word that Thou commanded Thy servant Moses,
% saying, ‘If ye transgress, I will scatter you abroad among the nations;
% 9 but if ye turn unto Me, and keep My commandments and do them, though there
% were some of you cast out unto the uttermost part of the heaven, yet will I
% gather them from thence, and will bring them unto the place that I have
% chosen to set My name there.’
\lnr{}Ex edicto, quod vicesimum
annum Artaxerxis antecissit, moenia, quae a Iudaeis neglecta
iacebant, ex literis autem ad Asaph saltuarium, portas refecit.
\lnr{}Atqui ex edicto reficiendorum Hierosolymorum manifesto deducuntur
hebdomades.
\lnr{}Daniel \textsc{ix}, 25.
% Daniel 9:25 "Know therefore and understand that from the going forth of the
% commandment to restore and to build Jerusalem until the Messiah the Prince,
% shall be seven weeks and threescore and two weeks; the street shall be built
% again, and the wall, even in troublesome times."
\lnr{}Ergo non a \textsc{xx} anno Artaxerxis.
\lnr{}Pererius postquam Lectorem ingenti mole confutationum
obruit, longa exspectatione suspendit, omnes denique, qui
de hac re scripserunt, pueros ostendit, tandem concludit, a vicesimo
anno Artaxerxis putandas esse Septimanas.
\lnr{}Annus Artaxerxis
Longimani vicesimus est 4268 in periodo Iuliana.
\lnr{}Adiectis 490,
componitur annus 4758 in periodo Iuliana.
\lnr{}Deducta Christi hodierna
epocha 4713, remanet annus Christi 45, annis solidos duodecim
post Passionem.
\lnr{}Itaque tertiusdecimus annus currens a Passione,
quintus autem imperii Claudii, est finis Hebdomadum.
\lnr{}Homo eruditus invidit nobis huius mysterii expositionem, quemamodum
multa alia de Belo, et Dracone, de verbis \textgreek{[Greek]},
\textgreek{[Greek]}, et similibus, quae ille profanis apirere noluit.
\lnr{}Quodnam mysterium censet esse in quinto anno Claudii Caesaris?
\lnr{}Cum hoc apparatu Chronologiae reprehendit alios, qui rectam
temporum rationem docent.
\lnr{}Reiecta hac anili et ridicula sententia,
superest illa, non utique vera, sed tamen tolerabilis, quae a
septimo anno eiusdem Artaxerxis Longimani initium facit.
\lnr{}Annus septimus Artaxerxis Longimani 4255 compositus cum 490
definit in anno 4745: qui est annus 32 Christi, ut hodie putamus,
a quo ad Passionem, intercedit annus solidus duntaxat.
\lnr{}Sed
ponamus definere in anno Passionis.
\lnr{}Si Pererio displicuit nostra et
Sulpitii Severi sententia, saltem omissa illa asinina, quae a \textsc{xx} anno
Artaxerxis initium instituit, sectus fuisset istam.
\lnr{}Peccasset quidem,
sed cum ratione quadam.
\lnr{}Peccasset humanitus, non autem
se traduxisset.
\lnr{}Nam \textsc{vii} anno missus Esdras in Palaestinam.
\lnr{}Sed nullum edictum ab Artaxerxe factum. Tantum sub eo continuatum,
quod edicto Darii permissum, Esdrae \textsc{vi}, 14.
% Esra 6:14 "And the elders of the Jews built, and they prospered through the
% prophesying of Haggai the prophet and Zechariah the son of Idd. And they
% built and finished it, according to the commandment of the God of Israel,
% and according to the commandment of Cyrus and Darius and Artaxerxes king
% of Persia."
\lnr{}At Darius
solus permisit Hierosolyma instaurari.
\lnr{}Nam anno secundo Darii
Iudaei incipiunt iacere moenium urbis fundamenta, quod interpellandi
literas Artaxerxi mittunt Iudaeorum hostes, quibus Regi indicant
civitatem rebellem instaurari, eiusque moenia refici.
\lnr{}Esdrae \textsc{iiii}, 12, 13.
% Esra 4:12 "Be it known unto the king that the Jews who came up from thee
% to us have come unto Jerusalem, building the rebellious and the bad city,
% and have set up the walls thereof, and joined the foundations."
% 13 "Be it known now unto the king that, if this city be built and the walls
% set up again, then they will not pay toll, tribute, and custom, and so thou
% shalt bring damage to the revenue of the kings."
\lnr{}Rex reperto vetustissimo Cyri edicto, permittit facere,
quod hostes interpellare conabantur.
\lnr{}An haec negabunt?
\lnr{}Negabunt, scio, et scientes prudentesque mentientur.
\lnr{}Nobis vero satis
est absurditates eorum ostendisse.
\lnr{}De illis triumphare, ut ipsi de
aliis solent, neque nostri moris est, neque praesentis instituti.

%\end{parnumbers}
%\clearpage
\pnr{p. XLVI [pdf 73]}
%\begin{parnumbers}
\lnr{}Itaque
Lector \textsc{vi} librum nostrum amplius consulat.
\lnr{}Si nostras cum horum
rationibus contulerit, satis mirari non poterit, qui factum
sit, ut rem facillimam, quam ipse Daniel interpretatur, contra mentem
Danielis, et obscurissimam, et inexplicabilem fecerint.
\lnr{}Si Porphyrius
viveret, non aliis argumentis magis religionis Christianae
initia oppugnare potuisset, quam horum hominum velitationibus:
quos, quae erat eius eruditio, ex ratione temporum, cuius peritissimus
erat, satis refellere potuisset, et veram Chronologiam
Danielis eos docere.
\lnr{}Nam totam interpretationem obscurissimi
capitis \textsc{xi} illi debemus, ut testis est Hieronymus.
\lnr{}Itaque
absque Porphyrio foret, tota illius capitis historia hactenus alta
silentii oblivione abdita, ridiculas a multis tam veteribus, quam huius
saeculi scriptoribus coniecturas expressisset.
\lnr{}Sed ad Hebdomadas
redeo, quarum initium a secundo Darii Nothi, finis excidium
Hierosolymorum.
\lnr{}Utriusque termini auctor Daniel, quem sequimur.
\lnr{}Invidos, zelotypos, \textgreek{[Greek]}, cum suis paradoxis valere iubemus.
\lnr{}Daniel ita distribuit septimanas, dimidiam, unam, septem,
sexaginta duas.
\lnr{}Summa, septimanae septuaginta cum semisse: anni
493½.
\lnr{}Quae quidem distributio non plus habet mysterii, quam
partitio minae apud Ezekielem \textsc{xlv}, 12.

\textit{Viginti sicli, vigintiquinque
sicli, quindecim sicli vobis erunt Mina.}
\lnr{}Itaque frustra laborant,
qui ea intervalla aliquas epochas continere putant, et propterea
illis explicandis multa coguntur adversus Danielis, et, quod peius
est, animi sui sententiam dicere.
\lnr{}Supererant et aliae epochae, quas
prudentes studiosis reliquimus.
\lnr{}Primordia tantum imperii Francorum
discussimus, et Gregorio Turonensi, propter anachronismos
et hallucinationes intricatissimo scriptori, aliquam lucem attulimus.
\lnr{}Quae omnia historiae Francorum studiosis grata esse optamus.
\lnr{}Optamus, inquam.
\lnr{}Nam in tanta invidia et odio literarum
vix sperare audemus.
\lnr{}Libro septimo in Computo Romano, et alibi
rationem reddidimus, quare dies septimanae feriae vocentur: sed
et libro primo quare cognomines Planetarum.
\lnr{}Nam ea appellatio longe antiquior horis.
\lnr{}Quare non ab horis planetariis nomina diebus
septimanae imposita, sed potius superstitio appellationis dierum
in horas derivata.
\lnr{}Quod autem vetustissima sit appellatio dierum
a planetis etiam apud Graecos, demonstratur vetustissimo oraculo
apud Porphyrium:

% The following is the only on-line trace that could be found of this
% greek quote (10 jan 2016).
% Apparently it is a quote from Porphyrius as given by Eusebius in his
% Praeparatio Evangelica.
% An English translation by E. H. Gifford (1903) of the Praeparatio Evangelica
% by Eusebius has for Book V, Chapter XIV [attributed to Porphyry]:
% ----
% 'IN many cases the gods, by giving signs of their
% statements beforehand, show by their knowledge of
% the arrangement of each man's nativity that they are,
% if we may so say, excellent Magians and perfect
% astrologers. Again he said that in oracular responses
% Apollo spake thus:
%
% "Invoke together Hermes and the Sun
% On the Sun's day, the Moon when her day comes,
% Kronos and Aphrodite in due turn,
% With silent prayers, by chiefest Magian taught,
% Whom all men know lord of the seven-string'd lyre."
%
% 'And when they cried "You mean Ostanes," he added:
% "Call with loud voice seven times each several god." '
% ----
% Sadly the part of the book by Gifford which has the original Greek
% is not included in the on-line PDF.
%
% A representation of the original Greek was found in
% http://latin_latin.enacademic.com/47496/PLANETAE
% "Imo, in vetusto Apollinis Oraculo, legitur, ex Porphyrio apud Eusebium,
% Praep. Euan. l. 5. c. 14.
% [Eusebius, Praeparatio Evangelica, Partis 1, Libro 5, Caput 14]
	\textgreek{Κληίζειν Ερμῆν, ἠδ᾿ ἠέλιον καὶ τὰ ταῦτα}
	\textgreek{Η῾μέρη ἠελίου, μήνην δ᾿ ὅτε τῆς δὲ παρείη}
	\textgreek{Η῾μέρη, ἠδὲ Κρόνον ἠδ᾿ ἑξείης Α᾿φροδίτην},
% 'Ιnvoca Mercurium et pariter Solem
% Die Solis. Lunam itidem, cum eiusdem aderit
% Dies et simili modô Saturnum et Venerem.'"
%
% "Alios. Darii Patrem eundem fuisse, vult Amm. Marcellin. Histor. l. 3. Eodem
% faciunt, quae in Oraculo e Porphyrio modo citato sequuntur, 
	\textgreek{Κλήςεςιν ἀφθέγκτοις ἃς ῟ευρε ὄχ᾿ Μάγων ἄριςτος}

%\end{parnumbers}
%\clearpage
\pnr{p. XLVII [pdf 74]}
%\begin{parnumbers}

	\textgreek{Τῆς ἑπταφθόγγου βαςιλεὺς, ὃν πάντες ἴςαςιν.}
% 'Invocationibus arcanis, quas invenit Magorum praestantissimus
% Septisonae Rex, quem omnes nôrunt.'"
%
% "Nam et ea videtur fuisse in Ostanis disciplina hac Magica: quod elicitur
% ex versu seq. 
	\textgreek{Καὶ σφόδρα καὶ καθ᾿ ἕκαςτον ἀεὶ Θεὸν ἑπτακιφώνην.}
% 'Et valde et singulatim semper Deum septemplici voce.'"	

\textgreek{[Greek]}, (ut id quoque explicem) est Ostanes rex Babylonis.

\textgreek{Βαβυλὼν} enim est \textgreek{[Greek]},
 hoc est, septem literarum.
\lnr{}Vulgus autem credebat Planetas apparere die suo: nempe omni
die Martis, ipsum planetam sui videndi potestatem facere.
\lnr{}Idque plane Orpheus desinavit \textgreek{[Greek]}.

	\textgreek{[Greek]},
	\textgreek{[Greek]}, \textgreek{[Greek]}.
\lnr{}Nam diserte intelligit, si neomenia \textgreek{[Greek]} inciderit in diem
Martis, id est feriam tertiam, abstinendum ab opere.
\lnr{}In reliquorum
Computorum doctrina multa accesserunt praeter superiorem editionem
multa castigata sunt, praesertim in Computo Iudaico.
\lnr{}Nam
caussas Iudaici anni Solaris hodierni docemus, a nobis antea praetermissas,
quia cum omnibus eas ignorabamus.
\lnr{}Praeterea damnavimus sententiam nostram de Tekupharum sede, quas in prisca
mundi epocha collocabamus: unde fiebat, ut aliquando annum
embolimum pro communi acciperemus, qui tamen non uno loco
professi sumus nos non nescire, quae forma anni hodierni esset, et
quis situs embolismorum.
\lnr{}Imo sine doctrina mea ille sciolus qui
epistolam Chronologicam scripsit, nunquam scisset, quid esset
annus Iudaicus hodiernus.
\lnr{}Et tamen quaerit fatuos, quibus persuadeat,
nos ingens flagitium admisisse, qui ita de embolismis Iudaicis
pronunciavimus.
\lnr{}Ego huius non meminissem, nisi scirem hominibus
magis placere, quae veritatem destruunt, quam quae illam docent.
\lnr{}Et rari sunt, quos non magis delectet maledica garrulitas,
quam veritas eliguis.
\lnr{}Longum esset narrare, candide Lector,
quot Marrucinos non solum inscitiae portentosae, sed et morum
improbissimorum editio prior commoverit: ut quis futurus sit furor
eorum ex ista, hinc facile coniicere liceat.
\lnr{}Ecce adhuc prius opus erat sub praelo.
\lnr{}Extitit quidam, non homo, nam ita vocare esset
hominibus iniuriam facere, sed coenum, et barathrum, qui librum
scripsit in me Gallice, conductus frusto panis a postremo bipedum,
clamans se solum peritum Chronologiae, me autem omnia, quae
in opere meo scripserim, a se furatum.
\lnr{}Iste nunquam viderat librum
meum, et tamen fures nos clamavit.
\lnr{}Novo genere et longe ab
aliis diverso nos aggrediebatur.
\lnr{}Nam alii nos oderunt, quia in
nostris libris nihil vident, quod novum non sit: et haec una caussa
est odii, quod praeter exspectationem vident, quae in se potius, quam
in nobis amarent.
\lnr{}Quod si fures essemus alienorum scriptorum, nullus
esset locus invidiae.
\lnr{}Quid?

%\end{parnumbers}
%\clearpage
\pnr{p. XLVIII [pdf 75]}
%\begin{parnumbers}
\lnr{}Nonne anno superiore prodiit Chronologia
\textgreek{[Greek]}, in qua auctor praeter infinita errorum monstra
in ratione temporum, sententias nostras, ipsa verba quoque cavillatur,
tacito nomine nostro, ut civiliter facere videatur?
\lnr{}Iste bonus
Chronologus, septuaginta annis seniorem facit Tara, cum Abraham
genuit, quam Moses scribit: annos autem mundi 74 maiores,
quam patiuntur vera ratiocinia.
\lnr{}Nunquam illi verum excidit, nisi
forte in iis, quae a nobis habet, ut in epocha primae Olympiadis in
\textsc{xxiii} Iulii, quod quidem primi omnium docuimus, et a solo
Pindaro, quem iste nunquam vidit, didicimus: et in eclipsi ante
mortem Herodis, quam primi indicavimus.
\lnr{}Nam simius noster est,
et tamen obrectator.
\lnr{}Opus tam egregium claudit pulcherrima
coronide, nempe forma anni caelestis.
\lnr{}Nam id quoque nos feceramus.
\lnr{}Sed
	\textit{Torquatus nitidas vario de marmore Thermas}
	\textit{Extruxit. cucumam fecit Otacilius.}
Iubet igitur post annos 160 unum bisextum tollere, ut in octingentis
annis bisexta ipse quinque tantum perimat, Alfonsini sex.
\lnr{}Itaque propter tam subtile epichirema \textgreek{[Greek]}, \textgreek{[Greek]}.
\lnr{}Sed de illius stultitia, et aliorum furiis libro singulari dicemus.
\lnr{}Nam non istum solum, quem nunquam laesimus, non ulla privata iniuria,
sed feritas quaedam animi ad maledicendum impulit.
\lnr{}Sunt alii
fanatici, quos idem morbus et rabies exagitat.
\lnr{}Itaque si cui a rebus
seriis vacat, adeat eam Chronologiam.
\lnr{}Habebit et quod indignetur,
et quod rideat.
\lnr{}Quid dicemus de illo severo Censore, qui priorem
editionem nostram nunquam vidit: et tamen de ea edicit basilica
edicta?
\lnr{}Non vidit, et si vidisset, non intellexisset.
\lnr{}Ex aliorum opinione
nostra iudicat.
\lnr{}Haec praecisa confidentia non solum in me,
sed in alios quoque distringitur.
\lnr{}Habet eclogarios, qui illi centones
farciunt;
\lnr{}Locos ex auctoribus excerptos illi subministrant apposito
nomine et cognomine auctoris, sed nomine singulari litera designato.
\lnr{}Hinc fit, ut Martinum vocet, qui est Matthaeus, Petrum, qui est
Philippus.
\lnr{}Me quoque Iohannem vocasset, nisi notior illi essem de
nomine, quam de scriptis, quae ex alieno indicio iudicat.
\lnr{}Nam
ne ipse quidem Eclogarius nostra intellexit satis bene, \textgreek{[Greek]}
\textgreek{[Greek]}, quae sunt doctrinae nostrae.
\lnr{}Ipse profert tanquam vulgaria: quemadmodum ille Chronologus, de quo iam
verba fecimus, putavit rem vulgatam esse, primam Olympiadem
celebratam \textsc{xxiii} Iulii:
 quae res totius anni Graeci arcana complectitur.
\lnr{}De isto igitur severo Censore, et aliis erit alius dicendi locus:
quos nolim in hoc opere nominatos.

%\end{parnumbers}
%\clearpage
\pnr{p. XLIX [pdf 76]}
%\begin{parnumbers}
\lnr{}Nimium enim abutor aequanimitate
tua et patientia, benigne Lector, qui te ingrato otio detineam,
cum has horas melius collocare posses, non tantum horum
hominum caussa, quos melius erat praeterire, quam ut tibi totius instituti
nostri rationem aperirem.
\lnr{}Nos hoc opus non ulla ambitione
modi suscepimus, sed ut Chronologiam a sciolorum deliriis et tyrannide
sophistarum vindicaremus; et nobiliora ingenia ad maiora excitaremus.
\lnr{}Nam magnae huius messis et magna spicilegia sunt.
\lnr{}Ideoque
plures libri adhuc confici poterant.
\lnr{}Nos, qui a veteribus didicimus
\textgreek{[Greek]}, octo priores libros in septem contraximus,
octavi libri censu toto in superiores erogato.
\lnr{}Utinam brevioribus nobis esse per difficultatem rei licuisset!
\lnr{}Nam septem hos alii
in viginti, alii in plures procudiffent.
\lnr{}Ex nostris igitur disputantionibus
videmus, quomodo Chronicon absolutissimum confici possit.
\lnr{}Quis
erit et ab humaniori doctrina paratus, et ab omni inhumana invidia
liber, qui rem sibi gloriosam, ac posteritati fructuosam aggredietur?
\lnr{}Qui laborem adeo laudabilem non gravabitur, primum anno Iuliano
tanquam stabili fundamento opus suum superstruat: periodum Iulianam
verum temporum elenchum adhibeat.
\lnr{}Nam, ut initio diximus,
contextus temporum sine anno Iuliano, annus sine periodo Iuliana,
est ut navis sine velis, remis, et armamentis.
\lnr{}Itaque potest ad
annos periodi Iulianae Fastos suos digerere, si prius in limine operis
omnes epochas praemiserit ad ipsius periodi annos designatas, ut in
libro quinto fecimus.
\lnr{}Ita una periodus Iuliana vicem omnium fuerit.
\lnr{}Nam in Fastis computus multarum epocharum animos lectorum distrahit,
Mundi, Olympiadum, Palilium Urbis, Nabonassari, Christi,
quae omnia fere semper onerant margines Fastorum: cum una periodus
Iuliana omnia complectatur.
\lnr{}Non tamen veto omnes epocharum
computos adiicere, si cui ita videtur.
\lnr{}Prius tamen admonere debet,
utra Palilia sequitur, Varronis, an Catonis.
\lnr{}Qua in parte Chronologorum
prudentiam desidero.
\lnr{}Periculosum est enim ad epocham
controversam annos suos exigere: ut ad Palilia Urbis, ad annos imperii
Augusti, ad quos cum hactenus omnes Historiarum et Annalium
scriptores tempus natalis Christi dirigunt, incertiores amittunt
lectores, quam antea.
\lnr{}His ita constitutis, nullam diem saltem sine
notatione annorum caelestium utiusque sideris, nostri Solaris, et Iudaeorum
Lunaris apponat.
\lnr{}Atque adeo, ubi res postulabit, ne gravetur
etiam diem Nabonassari AEgyptiacam, Hegirae Muhammedanam,
Iezdegird Persicam adhibere, et preastertim Atticam cum mensibus
tam popularibus, quam Prytanias: ita tamen, ut Harpaleis mensibus
ante Metonem, Metonicis ante Calippum, Calippicis usque ad
editionem anni Iuliani utatur.

%\end{parnumbers}
%\clearpage
\pnr{p. L [pdf 77]}
%\begin{parnumbers}
\lnr{}Quare quemadmodum nobiliores epochae,
ita etiam omnes Tabulae neomeniarum Atticarum, et Prytanias,
Iudaicarum, et Muhammedanarum, et harmoniae annorum
in limine operis erunt praeponendae: ne imparatus Lector ad Fastorum
aut Annalium contextum accedat.
\lnr{}Ubi deerunt characteres,
tunc appositione annorum, et intervallis utendum erit, quod est ultimum
subsidium.
\lnr{}Verbi gratia: cum regum Iudae et Samariae nullo
charactere apposito annorum series tantum contexitur, contenti esse
debemus adiectione sola annorum, quia aliter fieri non potest.
\lnr{}Sed
ad intervalla certa anni dirigendi.
\lnr{}Ut ab Exodo, cuius character certissimus
est feria quinta, ad casum Sedekiae, cuius character est annus
Sabbaticus, anni sunt absoluti 907.
\lnr{}A conditu templi, cuius index certus
a Scriptura determinatus in anno Exodi 480, ad casum Sedekiae,
videndum, ne adiectio annorum fraudi sit, propter annos absolutos,
si pro labentibus accipiantur, aut contra.
\lnr{}Ad intervalla igitur
semper operatio erit exigenda.
\lnr{}Exemplum anni absoluti pro labente
habes initio Danielis, annum tertium Ioiakim, qui erat quartus labens
Ieremiae \textsc{xxv}, 1.
\lnr{}Apud Gregorium Turonensem eiusmodi
exempla multa extant, et apud Iosephum, et Plutarchum, et
Xenophontem.
\lnr{}Hoc enim fit, quando excluditur alteruter terminus.
\lnr{}Plutarchus aliquando utrumque terminum excludit.
\lnr{}Exemplum: Sabbatum
posterius est octavus dies a priore sabbato.
\lnr{}Intervallum septem
dies.
\lnr{}At Plutarchus dixisset sex dies.
\lnr{}Nimirum excluso utroque
sabbato.
\lnr{}Hic non est error auctorum, sed idiotismus usurpantium
modum vulgi consuetum.
\lnr{}Sic Matthaei \textsc{xvii}, 1.
% Matthew 17:1 "And after six days Jesus took Peter, James, and John his
% brother, and brought them up onto a high mountain apart."
 Marci \textsc{ix}, 9.
% Marcus 9:9 "And as they came down from the mountain, He charged them that
% they should tell no man what things they had seen, till the Son of Man
% were risen from the dead."
% [This quote seems irrelevant. Compare Mark 9:2 "And after six days Jesus
% took with Him Peter and James and John, and led them up onto a high
% mountain apart by themselves; and He was transfigured before them."]
 sex
dies dicuntur excluso utroque termino, qui Lucae {ix}, 28
% Luke 9:28 "And it came to pass about eight days after these sayings,
% He took Peter and John and James, and went up onto a mountain to pray."
 sunt octo,
utroque termino incluso.
\lnr{}Nullam diem in historiis suo charactere
notatam omittat, etiam si rem certam teneat.
\lnr{}Exemplum: Ex Consulum
ferie, scimus annum Christi 364 esse primum Valentiniani.
\lnr{}At quia Marcellinus anno bisextili creatum, et postridie bisexti imperium
iniisse scribit, ille character omittendus non est.
\lnr{}Ita fiet, ut
hoc initio certo ad finem certum alius intervalli, ubi destituent signa,
hac tanquam Helice in illo mari tuto navigare possis.
\lnr{}Idem dico de eclipsibus, indictionibus, annis Sabbaticis, aera Hispanica,
celebrationibus Paschatis, passim apud historicos Christianos: atque
illa omnia ad cyclos suos revocet, quorum doctrina ut magnis erroribus
occurrit, ita magnos et multos patefacit.
\lnr{}Sed ne temere scriptoribus
de defectibus siderum assentiatur, sine Tabularum astronomicarum
examine.
\lnr{}Fallunt enim auctores non raro, cum defectus
luminum, qui nunquam fuerunt, referunt: ut Dio, et Servius in
caede Caesaris, Aurelius Victor vulgaris in obitu Nervae, Zosimus in
memorabili pugna Theodosii, et Eugenii tyranni: Tarrutius celeberrimus
Mathematicus in conceptu Romuli, et Palilibus Urbis.

%\end{parnumbers}
%\clearpage
\pnr{p. LI [pdf 78]}
%\begin{parnumbers}
\lnr{}Aliquando etiam caliginem Solis defectum putant: qualis caelo ut plurimum
sereno, toto anno caedis Caesaris fuit, et triduo integro, me puero,
sed pallida: at horrenda, et atra in articulo Passionis Dominicae
per aliquot horas: in anno autem Christi 798 per 18 dies ita atra, ut
naves in mari aberrarent: sub Commodo Kal. Ianuariis repentina caligo,
ac tenebrae in circo obortae, teste Lampridio.
\lnr{}Trebellius Pollio in
Gallienis: \textit{Gallieno et Faustino}
 coss. \textit{inter tot bellicas clades etiam terra
motus fuit, et tenebra per multos dies.}
\lnr{}Anno autem 393, aut 394, circa
dies Pentecostes adeo obscuratus est Sol, ut mundi interitus imminere
crederetur.
\lnr{}Hieronymus ad Pammachium: \textit{Nos scindimus Ecclesiam,
qui ante paucos menses, circa dies Pentecostes, cum, obscurato
sole, omnis mundus iam iamque venturum Iudicem formidaret, etcetera.}
\lnr{}Pentecoste Iudaica est semper sexta Sivvan, hoc est, sexta a novilunio.
\lnr{}Itaque Sol nunquam potest deficere in Pentecoste Iudaica: multo minus
deficere potest in Christiana, quae semper Iudaica posterior est, nisi
quando neomenia Nisan est Sabbatum.
\lnr{}Tunc enim Pentecoste Iudaica
incidit in diem Dominicam, et consequenter in Pentecosten Christianam,
ut anno Passionis Dominicae contigit, contra votum Sophistarum,
qui eo anno Pascha in Parasceven conferunt.
\lnr{}Igitur horribilis
illa Solis caligo apud Hieronymum non potuit accidere sexta Sivvan,
aut post sextam Sivvan, secundum naturalem caussam defectionum
Solarium, id est, \textgreek{[Greek]}; cum tale non possit
accidere, nisi \textgreek{[Greek]}.
\lnr{}Sed neque potuit Sol deficere novilunio
Sivvan annis 390, 391, 392, 393, 394, 395, anomalia Lunaris latitudinis
adversante.
\lnr{}Fuit igitur caligo extra ordinem.
\lnr{}Quare memorabilis
est ille locus Hieronymi, ita tamen, ut ad characterem adsumi non
possit.
\lnr{}Similis quoque caligo apud Tertullianum ad Scapulam: \textit{Nam et
Sol ille in conventu Uticensi, extincto p[a]ene lumine, adeo portentum fuit, ut
non potuerit ex ordinario deliquio hoc pati, positus in suo hypsomate, et
domicilio}.
% Tertullian (Quintus Septimius Florens Tertullianus)
% "Ad Scapulam" 3.3
% "Nam et sol ille in conuentu Vticensi extincto paene lumine adeo portentum
% fuit, ut non potuerit ex ordinario deliquio hoc pati, positus in suo
% hypsomate et domicilio."
% Translation by Dalrymple (1790):
% "Again, the sun, with his light almost put out, in the district of Utica,
% was indeed portentous.  That could not have been owing to any eclipse, for
% he was then in his altitude and house. "
% Translation by Thelwall(1869):
% "That sun, too, in the metropolis of Utica, with light all but extinguished,
% was a portent which could not have occurred from an ordinary eclipse,
% situated as the lord of day was in his height and house."
\lnr{}Haec Tertullianus.
\lnr{}Quanquam quod dicit de hypsomate Solis
et domicilio, ego non intelligo.
\lnr{}Scimus quidem quid sint hypsomata
Planetarum et domicilia, sed Solis hypsoma impedire, quo minus
deficiat, ingenue fateor me ignorare.
\lnr{}Rursus consideret Chronologus,
initia Imperatorum Romanorum, post tempora Antoninorum,
usque ad Consulatum Ausonii, confusionis, et tenebrarum plene esse:
\lnr{}Eaque ex methodo nostra, quantum poterit, castiganda esse.
\lnr{}Nam ex disputationibus
nostris, initium Diocletiani male ad initium aerae Martyrum
referri, et ex illo errore infinitos in Fastis propagatos esse, satis
constat, praesertim in historia persecutionis.
\lnr{}Si qua Eclipsium diagrammata,
aut disputationem longam alicuius epochae controversae proponere
velit, caveat eam in contextum Fastorum recipere.

%\end{parnumbers}
%\clearpage
\pnr{p. LII [pdf 79]}
%\begin{parnumbers}
\lnr{}Hoc enim
plane est non solum rerum seriem interrumpere, sed etiam animum
Lectoris in diversa distrahere.
\lnr{}Olympiadem quanque signet sua die,
nempe quintadecima mensis Hyperberetaei Olympici, cum ipsa die
Iuliana, item die mensis Attici, qui nunquam cum mense Olympico
congruit: et quemadmodum in omnibus mensis Caelestis apponendus
est, ita nunquam mensis Caelestis usurpandus, nisi prius praemisso charactere
Zygonos, et die Iuliana, cui neomenia Zygonos competit.
\lnr{}Haec enim omnia naturalia sunt, priscorum civilium temporum, non
autem ficta, neque arbitrii nostri.
\lnr{}Nunquam mensem popularem Atheniensium
sine mense prytanias Metonico, aut Calippico producat.
\lnr{}Cum feriam ab auctoribus notatam referet, diligenter examinet, an
error sit in notatione sive ab auctore, sive a librario.
\lnr{}Quemadmodum
enim falso defectus luminum annotant, ut supra diximus, ita etiam
per \textgreek{[Greek]}, falsum characterem diei aliquando apponunt;
 quod contingit
in lapide Romae Christiano, in quo \textsc{vii} Kal. Novembris dicitur
esse feria sexta Consulatu secundo Stilichonis.
\lnr{}Ex quo sequeretur, Consulatum
secundum Stilichonis fuisse anno Dominico 406, non 405,
contra seriem Consulum, et Chronicon Consulare Marcillini Comitis.
\lnr{}Consulatu Stilichonis primo, 26 Octobris fuit dies Veneris,
non Consulatu secundo.
\lnr{}Cum illi saxo fidem haberem, annum primi
Toletani Consilii dixi fuisse post Consulatum primum Stilichonis.
\lnr{}Quod falsum est: et locus corrigendus in capite de AEra Hispanica.
\lnr{}Eiusmodi falsas annotationes feriarum confutavimus in fragmento
persecutionis Diocletianeae, quod Eusebio assutum est, cum tamen
Eusebii non sit.
\lnr{}Reliqua per se conditor Annalium, aut Fastorum animadvertere
potest.
\lnr{}Neque enim iudiciis eruditioribus diffidimus.
\lnr{}Tabulae nostrae, et methodi omnem laborem minuent.
\lnr{}Parumne est
tam parabili labore tantum gratiae a posteritate inire?
\lnr{}Huiusmodi
opus quid aliud erit, quam, ut Plinius loquitur, aliquod veluti templum
augustum totius antiquitatis Fastos et memoriam complexum?
\lnr{}Quod si mallet ab alio, quam a nobis haec didicisse, cum bona nostra
gratia nomen nostrum dissimulare illi licet, neque vereatur, ne
plagii, aut furti a nobis postuletur.
\lnr{}Unum sibi proponat tantum, de
posteritate sine labore suo bene mereri.
\lnr{}Quare reformidat?
\lnr{}Quare piget tam praeclari incepti?
\lnr{}Plusne apud enum poterit \textgreek{[Greek]}, malitia,
quam amor veri?
\lnr{}Adeone hoc saeculum ab omni humanitate et
morum et literarum exaruit?
\lnr{}Quisquis igitur hoc onus non detrectabis,
scito, ex illo posteritati maximum emolumentum, tibi autem
a posteritate maximam gloriam propositam esse.
\lnr{}Vale candide
Lector.
\lnr{}Lugduni Batavorum.
% I.e. the university city of Leiden in the Republic of the Seven United
% Netherlands, where Scaliger resided.

% End of the Prolegomena
\end{parnumbers}
