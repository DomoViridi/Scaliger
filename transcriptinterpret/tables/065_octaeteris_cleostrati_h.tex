%%% Liber II p65
%% Version with a horizontal layout, with one column per annum
%%
%% For testing, uncomment the folowing lines and the lines at the end
%% of the file
%% Test ==>
\documentclass[draft,a4paper]{book}
\usepackage{rotating}
\usepackage{fontspec}
\setmainfont{Hoefler Text}[]
%\setmainfont{Times New Roman}[]
\newfontfamily\greekfont{Times New Roman}
\usepackage[quiet]{polyglossia}
\setmainlanguage{latin}
\setotherlanguage{greek}
\begin{document}
Itaque deprehenso capite Hecatombaeonis, quot
\textgreek{[Greek]} supersint de Scirrhophorione in anno Iuliano, nobis quidem
deprehendere facile est, qui rationem tantum habemus anni nostri
iuliani.
\begin{table}
%\tiny
%\scriptsize
\small
\centering
\setlength{\tabcolsep}{3pt}
\renewcommand{\arraystretch}{1.1}
%% <== Test
%%
\begin{tabular}{l cccccccc}
\hline
Anni octaeteridis &
1\super{†} & 2 & 3\super{†} & 4 & 5 & 6\super{†} & 7 & 8 \\
Cyclus Lunnae &
18 & 19 & 1 & 2 & 3 & 4 & 5 & 6 \\
Litera Dominicalis &
E & DC & B & A & G & FE & D & C \\
~ &
%Neomenia 1. mensis &
8 Ian. & 27 Ian. & 15 Ian. & 3 Febr. & 23 Ian. & 12 Ian. & 30 Ian. & 19 Ian. \\
Dies collecti &
384 &  738 & 1122 & 1476 & 1830 & 2214 & 2568 & 2922 \\
\hline
\\
\multicolumn{5}{l}{\footnotesize \super{†} \textgreek{Εμβολ.}}\\
\end{tabular}
%% Test ==>
\end{table}
Itaque deprehenso capite Hecatombaeonis, quot
\textgreek{[Greek]} supersint de Scirrhophorione in anno Iuliano, nobis quidem
deprehendere facile est, qui rationem tantum habemus anni nostri
iuliani.
\end{document}
