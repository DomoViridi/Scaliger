%%% Liber II p84, PDF 167
%%
%% Table that gives the names of the month in the Macedonian, Philippeian
%% and Attic calenders.
%%
%% The original table does not have a title
%% In the original, the first column does not have a header.
%%
%% The names of the month are very hard to read in the PDF scan.
%% The table is small and the scan tends to blurr the details
%%
%% From wikipedia, Ancient Greek calendars:
%%
%% Macedonian: 
%%    Dios - Δίος
%%    Apellaios - Ἀπελλαῖος
%%    Audunaios or Audnaios - Αὐδυναῖος or Αὐδναῖος
%% +  Peritios - Περίτιος
%%    Dystros - Δύστρος
%%    Xandikos or Xanthikos - Ξανδικός or Ξανθικός
%%    Artemisios or Artamitios - Ἀρτεμίσιος or Ἀρταμίτιος
%% -  Daisios - Δαίσιος
%%    Panemos or Panamos - Πάνημος or Πάναμος
%%    Loios - Λώιος
%%    Gorpiaios - Γορπιαῖος
%%    Hyperberetaios - Ὑπερβερεταῖος
%%
%% The '-' indicates first month listed in the table in de Macedonici column
%%
%% The Philippei column has the same months listed as the Macedonici column
%% but starts at Peritios (marked with a '+')
%%
%% The Attici column has the same months as the first column, but offset
%% by 4 places.
%%
%%% Count out columns for fixed-width source font
% 000000011111111112222222222333333333344444444445555555555666666666677777777778
% 345678901234567890123456789012345678901234567890123456789012345678901234567890
%
%% Select a general font size (uncomment one from the list)
%\tiny
%\scriptsize
\footnotesize
%\small
%\normalsize
%% Center the whole table left-right
\centering
%% Modify separation between columns
%\setlength{\tabcolsep}{1.6pt}
%% Modify distance between rows
%\renewcommand{\arraystretch}{1.3}
%%
\begin{tabular}{@{}l l  ll@{}}
% Dummy column (c) added to separate cmidrules
\toprule
 &
 \multicolumn{1}{c}{Menses Tetraeterici} &
 \multicolumn{2}{c}{Menses Metonici} \\
\cmidrule(lr){2-2} \cmidrule(lr){3-4}
 ~ & Macedonici &  Philippei & Attici \\
\midrule
 \textgreek{ἑκατομβαιών} &
 \textgreek{δαίσιος} &
 \textgreek{περίτιος} &
 \textgreek{ἐλαφηβολιών}
\\
 \textgreek{μεταγειτνιών} &
 \textgreek{πάνεμοσ} &
 \textgreek{δῦστρος[?]} &
 \textgreek{μυονυχιών}
\\
 \textgreek{βοηδρομιών} &
 \textgreek{λῶος[?]} & % In Emandatione: without the iota 
 \textgreek{ξανθικὸς} &
 \textgreek{θαργηλιών}
\\
\midrule
 \textgreek{πυανεψιών} &
 \textgreek{γορπιαῖος} &
 \textgreek{ἀρτεμίσιος} &
 \textgreek{σκιῤῥοφοριών}
\\
 \textgreek{μαιμακτηριών} &
 \textgreek{ὑπερβερεταῖοσ} &
 \textgreek{δαίσιος} &
 \textgreek{ἑκατομβαιών}
\\
 \textgreek{ποσειδεών} &
 \textgreek{δίοσ} &
 \textgreek{πάνεμοσ} &
 \textgreek{μεταγειτνιών}
\\
\midrule
 \textgreek{γαμηλιών} &
 \textgreek{ἀπελλαῖος} &
 \textgreek{λῶος[?]} &
 \textgreek{βοηδρομιών}
\\
 \textgreek{ανθεστηριών} &
 \textgreek{ἀυδυναῖος} &
 \textgreek{γορπιαῖος} &
 \textgreek{πυανεψιών}
\\
 \textgreek{ἐλαφηβολιών} &
 \textgreek{περίτιος} &
 \textgreek{ὑπερβερεταῖοσ} &
 \textgreek{μαιμακτηριών}
\\
\midrule
 \textgreek{μυονυχιών} &
 \textgreek{δῦστρος[?]} & % Tonal over upsilon: ύ or ῦ ?
 \textgreek{δίοσ} &
 \textgreek{ποσειδεών}
\\
 \textgreek{θαργηλιών} &
 \textgreek{ξανθικός} &
 \textgreek{ἀπελλαῖος} &
 \textgreek{γαμηλιών}
\\
 \textgreek{σκιῤῥοφοριών} &
 \textgreek{ἀρτεμίσιος} &
 \textgreek{ἀυδυναῖος} &
 \textgreek{ἀνθεστηριών}
\\
\bottomrule
\end{tabular}
%
\caption{Menses Tetraeterici et Metonici}
\label{tab:p84a}
