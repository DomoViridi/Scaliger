%%% Laterculum mensium Atticorum secundam anni quadrantes
%%% Liber I p33
%%
%% Table more horizontally spread out to make it look better without
%% text wrapping.
%
%% Names copied from Wikipedia: Attic calendar
%% then modified to match the original
%% - declension of quarter names θέρος -> θΕΡΙΝΟΙ
%% - no accents on the capitals
%% - Autumn: Φθινόπωρον -> ΟΠΩΡΙΝΟΙ
%% - Μουνιχιών -> Μουνυχιών (ι -> υ)
%% - Σκιροφοριών -> Σκιῤῥροφοριών (double ρ)
%% Added numbers to ensure the reader knows the order of the months
%%
%%% Count out columns for fixed-width source font
% 000000011111111112222222222333333333344444444445555555555666666666677777777778
% 345678901234567890123456789012345678901234567890123456789012345678901234567890
%
%% Select a general font size (uncomment one from the list)
%\tiny
%\scriptsize
%\footnotesize
%\small
\normalsize
%% Center the whole table left-right
\centering
%% Modify separation between columns
%\setlength{\tabcolsep}{0.5em}
%% Modify distance between rows
%\renewcommand{\arraystretch}{0.85}
%%
%% Four columns, one for each season
\begin{tabular}{@{}llll@{}}
\toprule
\multicolumn{4}{ c }{\Large\textsc{Laterculum mensium Atticorum}} \\
\multicolumn{4}{ c }{\Large\textsc{secundam anni quadrantes}} \\
\toprule
  \textgreek{Θερινοι}[?] & % Are the tonoi correct?
  \textgreek{Οπωρινοι}[?] &
  \textgreek{Χειμερινοι}[?] &
  \textgreek{Εαρινοι}[?]
\\
  \textgreek{μηνες}[?] &
  \textgreek{μηνες}[?] &
  \textgreek{μηνες}[?] &
  \textgreek{μηνες}[?]
\\
\midrule
%%
  \textgreek{Εκατομβαιών} &
  \textgreek{Πυανεψιών} &
  \textgreek{Γαμηλιών} &
  \textgreek{Μουνυχιών}
\\
  \textgreek{Μεταγειτνιών} &
  \textgreek{Μαιμακτηριών} &
  \textgreek{Ανθεστηριών} &
  \textgreek{Θαργηλιών}
\\
  \textgreek{Βοηδρομιών} &
  \textgreek{Ποσειδεών} &
  \textgreek{Ελαφηβολιών} &
  \textgreek{Σκιῤῥοφοριών}
\\
\bottomrule
\end{tabular}
%
\caption{Laterculum mensium Atticorum secundam anni quadrantes}
\label{tab:p033}
