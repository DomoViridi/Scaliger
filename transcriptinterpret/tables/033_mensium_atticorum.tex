%%% Laterculum mensium Atticorum secundam anni quadrantes
%%% Liber I p33
%%
%% Table more horizontally spread out to make it look better without
%% text wrapping.
%
%% Names copied from Wikipedia: Attic calendar
%% then modified to match the original
%% - declension of quarter names θέρος -> θΕΡΙΝΟΙ
%% - no accents on the capitals
%% - Autumn: Φθινόπωρον -> ΟΠΩΡΙΝΟΙ
%% - Μουνιχιών -> Μουνυχιών (ι -> υ)
%% - Σκιροφοριών -> Σκιῤῥροφοριών (double ρ)
%% Added numbers to ensure the reader knows the order of the months
%%
%%% Count out columns for fixed-width source font
% 000000011111111112222222222333333333344444444445555555555666666666677777777778
% 345678901234567890123456789012345678901234567890123456789012345678901234567890
%
%% Select a general font size (uncomment one from the list)
%\tiny
%\scriptsize
%\footnotesize
%\small
\normalsize
%% Center the whole table left-right
\centering
%% Modify separation between columns
%\setlength{\tabcolsep}{0.5em}
%% Modify distance between rows
%\renewcommand{\arraystretch}{0.85}
%%
%% Nested tables: two tables side-by-side, each containing two seasons
\begin{tabular}{cc}
\begin{tabular}{ c l }
\multicolumn{2}{c}{\textgreek{ΘΕΡΙΝΟΙ ΜΗΝΕΣ}} \\
1. &\textgreek{Εκατομβαιών} \\
2. &\textgreek{Μεταγειτνιών} \\
3. &\textgreek{Βοηδρομιών} \\
%% empty line between the seasons
~ & ~ \\
%%
\multicolumn{2}{c}{\textgreek{ΟΠΩΡΙΝΟΙ ΜΗΝΕΣ}} \\
4. &\textgreek{Πυανεψιών} \\
5. &\textgreek{Μαιμακτηριών} \\
6. &\textgreek{Ποσειδεών} \\
\end{tabular}
%% next column
&
%%
\begin{tabular}{ c l }
\multicolumn{2}{c}{\textgreek{ΧΕΙΜΕΡΙΝΟΙ ΜΗΝΕΣ}} \\
7. &\textgreek{Γαμηλιών} \\
8. &\textgreek{Ανθεστηριών} \\
9. &\textgreek{Ελαφηβολιών} \\
%% empty line between the seasons
~ & ~ \\
%%
\multicolumn{2}{c}{\textgreek{ΕΑΡΙΝΟΙ ΜΗΝΕΣ}} \\
10. &\textgreek{Μουνυχιών} \\
11. &\textgreek{Θαργηλιών} \\
12. &\textgreek{Σκιῤῥοφοριών} \\
\end{tabular}
\end{tabular}
%
\caption{Laterculum mensium Atticorum secundam anni quadrantes}
\label{tab:p033}
%
