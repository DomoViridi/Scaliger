%%% Liber II p67
%%
%%% Count out columns for fixed-width source font
% 000000011111111112222222222333333333344444444445555555555666666666677777777778
% 345678901234567890123456789012345678901234567890123456789012345678901234567890
%
% ΜΗΝΕΣ ΚΟΙΛΟΙ
% "The hollow months" "cavae menses"
%
% This table apparently lists the hollow months that occur in each year
% of the 8 year cycle (octaeteride).
% Each row shows the information for one year in the octaeteride.
% Column 1 indicated if a year is Embolic.
% We have represented the abbreviated word ἐμβολ. by a footnote symbol,
% the same way we did for other tables where it occurs.
% Column 2 uses Byzantine Greek numerals (without a bar above the letters)
% to number the years in the cycle.
% The number 6 is represented by a cursive digamma, rendered here as
% Unicode U03DA.
% The header for the second column is allmost illegible
% ?? ?? ὀκταετερίδος ??
% Columns 3-8 list the 5 or 6 months in the Attic calendar which are hollow.
% Column 9 shows a β. or α. for the ebolic years. Though not explained in the
% table itself, they are probably Byzantine Greek numerals again. The original
% has a bar over the α, but not over the β, indicating they might be numerals.
% Such symbols can be made using the Math features of Tex, e.g.
% $\overbar{\kappa\alpha}$ for '21'.
% Because there is no code for the digamma (representing 6), we need to resort
% to using \varsigma (the end-of-word sigma) for this symbol.
%
%%% Count out columns for fixed-width source font
% 000000011111111112222222222333333333344444444445555555555666666666677777777778
% 345678901234567890123456789012345678901234567890123456789012345678901234567890
%
%\tiny
%\scriptsize
%\footnotesize
\small
%\normalsize
%% Center the whole table left-right
\centering
%% Modify separation between columns
%\setlength{\tabcolsep}{1.6pt}
%% Modify distance between rows
\renewcommand{\arraystretch}{1.3}
%% Angle to rotate the headers
%\newcommand{\ang}{60}
%%
\begin{tabular}{ll llllll}
\toprule
\multicolumn{8}{c}{\large{\textgreek{ΜΗΝΕΣ ΚΟΙΛΟΙ}}}
\\
\toprule
~ & \multicolumn{5}{l}{$\downarrow$ \textgreek{τ.. τ.. ὀκταετερίδος .τ.}}
\\
\midrule
\scriptsize{†} &
\textgreek{\gnum{α}} &
\textgreek{ἐλαφηβολ.} &
\textgreek{θαργηλ.} &
\textgreek{ἑκατομβ.} &
\textgreek{βοηδρομ.} &
\textgreek{μαιμακτ.} &
\textgreek{ποσειδ. \gnum{β}} 
\\
 &
\textgreek{\gnum{β}} &
\textgreek{ἐλαφηβολ.} &
\textgreek{θαργηλ.} &
\textgreek{ἑκατομβ.} &
\textgreek{βονδρομ.} &
\textgreek{μαιμακτ.} &

\\
\scriptsize{†} &
\textgreek{\gnum{γ}} &
\textgreek{γαμηλ.} &
\textgreek{ἐλαφηβολ.} &
\textgreek{σκιῤῥοφ.} &
\textgreek{μεταγείτν.} &
\textgreek{πυανεψ.} &
\textgreek{ποσειδ. \gnum{α}} 
\\
%\midrule
 &
\textgreek{\gnum{δ}} &
\textgreek{γαμηλ.} &
\textgreek{ἐλαφηβολ.} &
\textgreek{σκιῤῥοφ.} &
\textgreek{βονδρομ.} &
\textgreek{μαιμακτ.} &

\\
 &
\textgreek{\gnum{ε}} &
\textgreek{ανθεστηρ.} &
\textgreek{μουνιχ.} &
\textgreek{σκιῤῥοφ.} &
\textgreek{βονδρομ.} &
\textgreek{μαιμακτ.} &

\\
\scriptsize{†} &
\textgreek{\gnum{ϛ}} &
\textgreek{γαμηλ.} &
\textgreek{ἐλαφηβολ.} &
\textgreek{θαρυηλ.} &
\textgreek{ἑκατομβ.} &
\textgreek{βονδρομ.} &
\textgreek{ποσειδ. \gnum{α}}
\\
%\midrule
 &
\textgreek{\gnum{ζ}} &
\textgreek{γαμηλ.} &
\textgreek{ἐλαφηβολ.} &
\textgreek{θαρυηλ.} &
\textgreek{ἑκατομβ.} &
\textgreek{βονδρομ.} &
\textgreek{ποσειδ.}

\\
 &
\textgreek{\gnum{η}} &
\textgreek{ανθεστηρ.} &
\textgreek{μουνιχ.} &
\textgreek{σκιῤῥοφ.} &
\textgreek{μεταγείτν.} &
\textgreek{πυανεψ.} &
\textgreek{ποσειδ.}

\\
\bottomrule
\addlinespace
~ & ~ & \multicolumn{3}{l}{\footnotesize \super{†} \textgreek{ἐμβολ.}}\\
\end{tabular}
%
\caption{\textgreek{Μηνες κοιλοι}}
\label{table:p067a}
