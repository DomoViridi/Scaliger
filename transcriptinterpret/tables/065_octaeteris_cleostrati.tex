%%% Liber II p65
%%
%% For testing, uncomment the folowing lines and the lines at the end
%% of the file
%% Test ==>
%\documentclass[draft,a4paper]{book}
%\usepackage{rotating}
%\usepackage{fontspec}
%\setmainfont{Hoefler Text}[]
%%\setmainfont{Times New Roman}[]
%\newfontfamily\greekfont{Times New Roman}
%\usepackage[quiet]{polyglossia}
%\setmainlanguage{latin}
%\setotherlanguage{greek}
%\begin{document}
%Itaque deprehenso capite Hecatombaeonis, quot
%\textgreek{[Greek]} supersint de Scirrhophorione in anno Iuliano, nobis quidem
%deprehendere facile est, qui rationem tantum habemus anni nostri
%iuliani.
%\begin{table}
%%\tiny
%%\scriptsize
%\centering
%%\setlength{\tabcolsep}{3pt}
%\renewcommand{\arraystretch}{1.1}
%% <== Test
%%
%\begin{tabular}{lc}
\begin{tabular}[t]{r rrccr}
~ & \multicolumn{5}{c}{\textsc{Octaeteris Cleostrati}}\\
\\ % Clumsily make room for the rotated headers
\\
\\
\\
\\
~ &
\begin{rotate}{60}Anni octaeteridis\end{rotate} &
\begin{rotate}{60}Cyclus Lunnae\end{rotate} &
\begin{rotate}{60}Litera Dominicalis\end{rotate} &
~ &
%\begin{rotate}{60}\hspace{5pt}\parbox[t]{3.5cm}{Neomenia\\1. mensis}\end{rotate} &
\multicolumn{1}{l}{\begin{rotate}{60}Dies collecti\end{rotate}}
\\
\cline{2-6}
\scriptsize{†}
  &  1 & 18 &  E &  8 Ianua. &  384 \\
~ &  2 & 19 & DC & 27 Ian.   &  738 \\
\footnotesize{†}
  &  3 &  1 &  B & 15 Ian.   & 1122 \\
~ &  4 &  2 &  A &  3 Febr.  & 1476 \\
%\cline{2-6}
~ &  5 &  3 &  G & 23 Ian.   & 1830 \\
\footnotesize{†}
  &  6 &  4 & FE & 12 Ian.   & 2214 \\
~ &  7 &  5 &  D & 30 Ian.   & 2568 \\
~ &  8 &  6 &  C & 19 Ian.   & 2922 \\
%\cline{2-6}
\\
~ & \multicolumn{5}{l}{\footnotesize \super{†} \textgreek{Εμβολ.}}\\
\end{tabular}
%% Test ==>
%\end{table}
%Itaque deprehenso capite Hecatombaeonis, quot
%\textgreek{[Greek]} supersint de Scirrhophorione in anno Iuliano, nobis quidem
%deprehendere facile est, qui rationem tantum habemus anni nostri
%iuliani.
%\end{document}
