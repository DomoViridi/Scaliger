%%% Liber II p67, PDF 150
%%
%% Dates of the new moon for each year in the octaeterida
%% Translated version with Arabic numerals instead of Greek numerals,
%% and Roman numerals representing the months
%%
%% For testing, uncomment the folowing lines and the lines at the end of the file
%% Test ==>
\documentclass{book}
\usepackage{fontspec}
\setmainfont{Hoefler Text}[]
\newfontfamily\greekfont{Arial Unicode MS}
\usepackage[quiet]{polyglossia}
\setmainlanguage{latin}
\setotherlanguage{greek}
\usepackage{booktabs} % To get nice table layout
\begin{document}
Just some text on top of the table.
Ideo neomeniam Lunaris Gamelionis cum neomenia
Gamelionis aequabilis in primo anno composuimus: non quod
ita fecerit Harpalus: (Nullus enim Gamelion aequabilis eo seculo
Lunaris fuit:) sed quia deprehenso anno primo Octaeteridis Harpali,
non operosum erit divinare quotae diei Gemelionis aequabilis
competat neomenia Gamelionis Lunaris.

\begin{table}[htbp]
 \centering
%% <== Test
%%
\normalsize
%\small
%\footnotesize
%\scriptsize
%\tiny
%% Modify distance between rows
\renewcommand{\arraystretch}{1.2}
%% Modify separation between columns
\setlength{\tabcolsep}{2.0pt}
%
\begin{tabular}{@{}cl llllllll@{}}
\toprule
\multicolumn{2}{ c }{~} &
\multicolumn{8}{ c }{annum}
\\
\cmidrule{3-10}
\multicolumn{2}{ c }{Mensis lunaris} &
\multicolumn{1}{c}{1} &
\multicolumn{1}{c}{2} &
\multicolumn{1}{c}{3} &
\multicolumn{1}{c}{4} &
\multicolumn{1}{c}{5} &
\multicolumn{1}{c}{6} &
\multicolumn{1}{c}{7} &
\multicolumn{1}{c}{8}
\\
\midrule
\textsc{vii} & \textgreek{γαμηλιών} &
 1.\textsc{vii} &
25.\textsc{vi} &
18.\textsc{vi} &
 8.\textsc{vii} &
 1.\textsc{vii} &
26.\textsc{vi} &
16.\textsc{vii} &
 8.\textsc{vii}
\\
\textsc{viii} & \textgreek{ανθεστηριών} &
 1.\textsc{viii} &
23.\textsc{vii} &
15.\textsc{vii} &
 8.\textsc{viii} &
 1.\textsc{viii} &
23.\textsc{vii} &
15.\textsc{viii} &
 8.\textsc{viii}
\\
\textsc{ix} & \textgreek{ἐλαφηβολιών} &
 1.\textsc{ix} &
23.\textsc{viii} &
15.\textsc{viii} &
 7.\textsc{ix} &
30.\textsc{viii} &
23.\textsc{viii} &
15.\textsc{ix} &
 7.\textsc{ix}
\\
\midrule
\textsc{x} & \textgreek{μυονυχιών} &
30.\textsc{ix} &
22.\textsc{ix} &
14.\textsc{ix} &
 7.\textsc{x} &
30.\textsc{ix} &
22.\textsc{ix} &
14.\textsc{x} &
 7.\textsc{x}
\\
\textsc{xi} & \textgreek{θαργηλιών} &
30.\textsc{x} &
22.\textsc{x} &
14.\textsc{x} &
 6.\textsc{xi} &
29.\textsc{x} &
22.\textsc{x} &
14.\textsc{xi} &
 6.\textsc{xi}
\\
\textsc{xii} & \textgreek{σκιῤῥοφοριών} &
29.\textsc{xi} &
21.\textsc{xi} &
14.\textsc{xi} &
 6.\textsc{xii} &
24.\textsc{xi} &
21.\textsc{xi} &
13.\textsc{xii} &
 6.\textsc{xii}
\\
\midrule
\textsc{i} & \textgreek{ἑκατομβαιών} &
29.\textsc{xii} &
21.\textsc{xii} &
13.\textsc{xii} &
 6.\textsc{i} &
29.\textsc{xii} &
21.\textsc{xii} &
13.\textsc{i} &
 5.\textsc{i}
\\
\textsc{ii} & \textgreek{μεταγειτνιών} &
28.\textsc{i} &
20.\textsc{i} &
13.\textsc{i} &
 5.\textsc{ii} &
28.\textsc{i} &
20.\textsc{i} &
12.\textsc{ii} &
 5.\textsc{ii}
\\
\textsc{iii} & \textgreek{βοηδρομιών} &
28.\textsc{ii} &
20.\textsc{ii} &
12.\textsc{ii} &
 5.\textsc{iii} &
28.\textsc{ii} &
20.\textsc{ii} &
12.\textsc{iii} &
 4.\textsc{iii}
\\
\midrule
\textsc{iv} & \textgreek{πυανεψιών} &
27.\textsc{iii} &
19.\textsc{iii} &
12.\textsc{iii} &
 5.\textsc{iv} &
27.\textsc{iii} &
20.\textsc{iii} &
11.\textsc{iv} &
 4.\textsc{iv}
\\
\textsc{v} & \textgreek{μαιμακτηριών} &
27.\textsc{iv} &
19.\textsc{iv} &
11.\textsc{iv} &
 4.\textsc{v} &
27.\textsc{iv} &
19.\textsc{iv} &
11.\textsc{v} &
 3.\textsc{v}
\\
\textsc{vi} & \textgreek{ποσειδεών} $\overline{\alpha}$&
26.\textsc{v} &
18.\textsc{v} &
11.\textsc{v} &
 3.\textsc{vi} &
26.\textsc{v} &
19.\textsc{v} &
11.\textsc{vi} &
 3.\textsc{vi}
\\
\textsc{vi} & \textgreek{ποσειδεών} $\overline{\beta}$&
26.\textsc{vi} $\overline{\alpha}$ &
    \multicolumn{1}{c}{$\circ$} &
10.\textsc{vi} &
    \multicolumn{1}{c}{$\circ$} &
    \multicolumn{1}{c}{$\circ$} &
18.\textsc{vi} &
    \multicolumn{1}{c}{$\circ$} &
\multicolumn{1}{c}{~}\\
\bottomrule
\end{tabular}
%% Test ==>
\caption{Neomeniae octaeteridai Harpali per annum}
\end{table}

Just some text below the table.
Ideo neomeniam Lunaris Gamelionis cum neomenia
Gamelionis aequabilis in primo anno composuimus: non quod
ita fecerit Harpalus: (Nullus enim Gamelion aequabilis eo seculo
Lunaris fuit:) sed quia deprehenso anno primo Octaeteridis Harpali,
non operosum erit divinare quotae diei Gemelionis aequabilis
competat neomenia Gamelionis Lunaris.

\listoftables
\end{document}
