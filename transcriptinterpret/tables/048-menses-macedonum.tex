%%% Liber I p48
%%
%% Menses Macedonum names copied from Wikipedia: Ancient Macedonian calendar
%% then modified to match the original
%% Note that ὑπερβερεταῖος is normally considered the last month in the Macedonian
%% calendar, but appears first in this table.
%% - πάνεμος is written with an ε in the book, not an η or an α as given
%%   on Wikipedia
%% - Wikipedia gives Λώιος while the book appears to have λῶος (with a
%%   circumflex rather than an acute accent over the omega and without the iota)
%%
%% Menses Atheniensium names copied from Wikipedia: Attic calendar
%% then modified to match the original
%% - Μουνιχιών -> μουνυχιών (ι -> υ)
%% - Σκιροφοριών -> σκιῤῥοφοριών (double ρ)
%% Added numbers to ensure the reader knows the order of the months
%%
%%% Count out columns for fixed-width source font
% 000000011111111112222222222333333333344444444445555555555666666666677777777778
% 345678901234567890123456789012345678901234567890123456789012345678901234567890
%
%% Select a general font size (uncomment one from the list)
%\tiny
%\scriptsize
%\footnotesize
%\small
\normalsize
%% Center the whole table left-right
\centering
%% Modify separation between columns
%\setlength{\tabcolsep}{0.5em}
%% Modify distance between rows
%\renewcommand{\arraystretch}{0.85}
%%
\begin{tabular}{ l  l }
\toprule
\parbox[b]{6em}{Menses \\ Macedonum} &
\parbox[b]{6em}{Menses \\ Atheniensium} \\
\midrule
\textgreek{ὑπερβερεταῖος}  &\textgreek{ἑκατομβαιών} \\
\textgreek{δίος}           &\textgreek{μεταγειτνιών} \\
\textgreek{ἀπελλαῖος}      &\textgreek{βοηδρομιών} \\
%
\textgreek{αὐδυναῖος}      &\textgreek{πυανεψιών} \\
\textgreek{περίτιος}        &\textgreek{μαιμακτηριών} \\
\textgreek{δύστρος}        &\textgreek{ποσειδεών} \\
%
\textgreek{ξανθικός}       &\textgreek{γαμηλιών} \\
\textgreek{ἀρτεμίσιος}     &\textgreek{ανθεστηριών} \\
\textgreek{δαίσιος}        &\textgreek{ἐλαφηβολιών} \\
%
\textgreek{πάνεμος}        &\textgreek{μουνυχιών} \\
\textgreek{λῶος}          &\textgreek{θαργηλιών} \\
\textgreek{γορπιαῖος}      &\textgreek{σκιῤῥοφοριών} \\
\bottomrule
\end{tabular}
%
\caption{Menses Macedonum et Atheniensium}
\label{tab:p048}
