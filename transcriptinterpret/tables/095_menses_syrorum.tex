%%% Liber II p95, PDF 178
%%
%% Table that gives the names of the month in the Syrian Attic calenders.
%%
%% This table contians the Syrian months in (what I believe to be) Arabic
%%
%% As an experiment, we copy and paste the names from
%% https://en.wikipedia.org/wiki/Arabic_names_of_calendar_months#Levant
%%
%%% Count out columns for fixed-width source font
% 000000011111111112222222222333333333344444444445555555555666666666677777777778
% 345678901234567890123456789012345678901234567890123456789012345678901234567890
%
%% Select a general font size (uncomment one from the list)
%\tiny
%\scriptsize
%\footnotesize
\small
%\normalsize
%% Center the whole table left-right
\centering
%% Modify separation between columns
%\setlength{\tabcolsep}{1.6pt}
%% Modify distance between rows
%\renewcommand{\arraystretch}{1.3}
%%
\begin{tabular}{@{}r l@{}}
% Dummy column (c) added to separate cmidrules
\toprule
 \multicolumn{2}{c}{\Large\textsc{Menses Syrorum}}\\
\toprule
 \textarabic{تشرين الأول} &
 Tisrin prior
\\
 \textarabic{تشرين الثاني} &
 Tisrin posterior
\\
 \textarabic{كانون الأول} &
 Canun prior
\\
\midrule
 \textarabic{كانون الثاني} &
 Canun posterior
\\
 \textarabic{شباط} &
 Shebat
\\
 \textarabic{آذار} &
 Adar\super †
\\
\midrule
 \textarabic{نيسان} &
 Nisan
\\
 \textarabic{أيار} &
 Iijar
\\
 \textarabic{حزيران} &
 Haziran
\\
\midrule
 \textarabic{تموز} &
 Tamuz
\\
 \textarabic{آب} &
 Ab
\\
 \textarabic{أيلول} &
 Ilul
\\
\bottomrule
 \multicolumn{2}{l}{\super † Mensis embolismus}
\end{tabular}
%
\caption{Menses Syrorum}
\label{tab:p95a}
