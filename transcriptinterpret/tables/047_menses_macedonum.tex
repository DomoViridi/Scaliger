%%% Liber I p47
%%
%% Menses Macedonum names copied from Wikipedia: Ancient Macedonian calendar
%% then modified to match the original
%% Note that ὑπερβερεταῖος is normally considered the last month in the Macedonian
%% calendar, but appears first in this table.
%% - πάνεμος is written with an ε in the book, not an η or an α as given
%%   on Wikipedia
%% - Wikipedia gives Λώιος while the book appears to have λῶος (with a
%%   circumflex rather than an acute accent over the omega and without the iota)
%%
%% Menses Atheniensium names copied from Wikipedia: Attic calendar
%% then modified to match the original
%% - Μουνιχιών -> μουνυχιών (ι -> υ)
%% - Σκιροφοριών -> σκιῤῥοφοριών (double ρ)
%% Added numbers to ensure the reader knows the order of the months
%%
%% For testing, uncomment the folowing lines and the lines at the end of the file
%% Test ==>
%\documentclass{book}
%\usepackage{fontspec}
%\setmainfont{Hoefler Text}[]
%\newfontfamily\greekfont{Arial Unicode MS}
%\usepackage[quiet]{polyglossia}
%\setmainlanguage{latin}
%\setotherlanguage{greek}
%\begin{document}
%% <== Test
%%
\begin{tabular}{ l  l }
Menses                    & Menses\\
Macedonum                 & Atheniensium \\
\hline
\textgreek{ὑπερβερεταῖος}  &\textgreek{ἑκατομβαιών} \\
\textgreek{δίος}           &\textgreek{μεταγειτνιών} \\
\textgreek{ἀπελλαῖος}      &\textgreek{βοηδρομιών} \\
%
\textgreek{αὐδυναῖος}      &\textgreek{πυανεψιών} \\
\textgreek{περίτιος}        &\textgreek{μαιμακτηριών} \\
\textgreek{δύστρος}        &\textgreek{ποσειδεών} \\
%
\textgreek{ξανθικός}       &\textgreek{γαμηλιών} \\
\textgreek{ἀρτεμίσιος}     &\textgreek{ανθεστηριών} \\
\textgreek{δαίσιος}        &\textgreek{ἐλαφηβολιών} \\
%
\textgreek{πάνεμος}        &\textgreek{μουνυχιών} \\
\textgreek{λῶος}          &\textgreek{θαργηλιών} \\
\textgreek{γορπιαῖος}      &\textgreek{σκιῤῥοφοριών} \\
\end{tabular}
%% Test ==>
%\end{document}
