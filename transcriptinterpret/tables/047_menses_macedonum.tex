%%% Liber I p47
%%
%% Menses Macedonum names copied from Wikipedia: Ancient Macedonian calendar
%% then modified to match the original
%% Note that Ὑπερβερεταῖος is normally considered the last month in the Macedonian
%% calendar, but appears first in this table.
%% - Initial capitals on all month names, though the original uses minusculs
%% - Πάνεμος is written with an ε in the book, not an η or an α as given
%%   on Wikipedia
%% - Wikipedia gives Λώιος while the book appears to have Λῶος (with a
%%   circumflex rather than an acute accent over the omega and without the iota)
%%
%% Menses Atheniensium names copied from Wikipedia: Attic calendar
%% then modified to match the original
%% - Μουνιχιών -> Μουνυχιών (ι -> υ)
%% - Σκιροφοριών -> Σκιῤῥροφοριών (double ρ)
%% Added numbers to ensure the reader knows the order of the months
%%
%% For testing, uncomment the folowing lines and the lines at the end of the file
%% Test ==>
%\documentclass{book}
%\usepackage{fontspec}
%\setmainfont{Hoefler Text}[]
%\newfontfamily\greekfont{Arial Unicode MS}
%\usepackage[quiet]{polyglossia}
%\setmainlanguage{latin}
%\setotherlanguage{greek}
%\begin{document}
%% <== Test
%%
\begin{tabular}{ l  l }
Menses                    & Menses\\
Macedonum                 & Atheniensium \\
\hline
\textgreek{Ὑπερβερεταῖος } &\textgreek{Ἑκατομβαιών} \\
\textgreek{Δίος}           &\textgreek{Μεταγειτνιών} \\
\textgreek{Ἀπελλαῖος}      &\textgreek{Βοηδρομιών} \\
%
\textgreek{Αὐδυναῖος}      &\textgreek{Πυανεψιών} \\
\textgreek{Περίτιος}        &\textgreek{Μαιμακτηριών} \\
\textgreek{Δύστρος}        &\textgreek{Ποσειδεών} \\
%
\textgreek{Ξανθικός}       &\textgreek{Γαμηλιών} \\
\textgreek{Ἀρτεμίσιος}     &\textgreek{Ανθεστηριών} \\
\textgreek{Δαίσιος}        &\textgreek{Ελαφηβολιών} \\
%
\textgreek{Πάνεμος}        &\textgreek{Μουνυχιών} \\
\textgreek{Λῶος}          &\textgreek{Θαργηλιών} \\
\textgreek{Γορπιαῖος}      &\textgreek{Σκιῤῥοφοριών} \\
\end{tabular}
%% Test ==>
%\end{document}
