% !TEX TS-program = xelatex
% !TEX encoding = UTF-8 Unicode
% this template is specifically designed to be typeset with XeLaTeX;
% it will not work with other engines, such as pdfLaTeX

%%% Count out columns for fixed-width source font
% 000000011111111112222222222333333333344444444445555555555666666666677777777778
% 345678901234567890123456789012345678901234567890123456789012345678901234567890

\chapter[Prolegomena]{}
\begin{center} \vspace{-18mm}
{\scshape
\head{3.0}{35}{PROLEGOMENA}\\ \vspace{6mm}
\head{1.5}{60}{IN}\\ \vspace{5mm}
\head{2.0}{50}{LIBROS}\\ \vspace{9mm}
\head{1.5}{60}{DE}\\ \vspace{6mm}
\head{3.3}{25}{EMENDATIONE}\\ \vspace{7mm}
\head{2.0}{40}{TEMPORVM}\\ \vspace{7mm}
} % scshape
\em{Ad candidum Lectorem.}
\end{center}
\normalsize

\setcounter{parcount}{0}
\begin{parnumbers}
\dropcapil{8}{Q}{Vintusdecimus} hic annus agitur, candide
Lector, postquam opus nostrum de
Emendatione Temporum emisimus.
\\ \p
Persuaseram
mihi, homines studiosos aliquam nobis
gratiam habituros tot rerum, quas \& scitu
dignas, \& a nobis primum indicatas negare
non poterant.
\\ \p
Sed longe aliter animatos experti
sumus: atque adeo rem potius inuidiosam
atque obtrectationi opportunam, quam illis gratam me suscepisse
intellexi.

Denique nihil aliud quam significarunt, quiduis potius
se ignorare malle, quam a nobis aliquid discere.

In quibusdam
candorem, in aliis studium, in omnibus sensum bonarum rerum desideraui.

Nos vero, qui nihil unquam prius habuimus, quam vt horum
orationes sinamus praeterfluere, modo verum eruere, \& inimicos
nostros etiam inuitos iuuare possimus, opus nostrum iterum in
manus sumptum auximus, illustrauimus, emendauimus, vt, quanuis
idem sit, aliud tamen a noua cultura videri possit.

Quæ huic editioni
accesserunt, haud promptum est dicere.

Sed in quibus a priore demutat,
postea intelliges, siquidem instituti nostri rationem aperuero.

Subiectum operis nostri est ratio Temporum civilium, \& eorum,
quæ in vetustatis cognitione versantur: finis, Emendatio: quod quidem
me tacente, \& Titulus ipse promittit.

Ciuilium temporum cognitio,
eorumque historia, vertitur in multiplici diversorum annorum
forma \& eorum methodis vulgaribus, quos Computos posterior
ætas vocauit.

\textgreek{Τα ιςορομγυα[?]} civilium temporum habes in primoribus
tribus libris, \& maiore parte quarti: methodum autem in septimo.

A emendationis duæ partes sunt.

%\end{parnumbers}
\clearpage
p. II [pdf 29]
%\begin{parnumbers}
Prior versatur circa epocharum
inuestigationem, posterior circa verum annum tropicum, 
\& periodos Lunares: quam materiam posterior pars libri quarti,
item toti quintus \& sextus sibi vindicant.

Iam quemadmodum Epochæ
sunt notationes, \& tituli temporum, ita ipsarum epocharum
quædam debent esse propria \textgreek{γνωρίσματα} \& characteres: quorum
characterum alij sunt naturales, alij ciuiles. 

Naturales quidem a rationibus
utriusque sideris, unde nati cycli Solaris, \& Lunaris: civiles
ab instituto, cuiusmodi indictiones \& anni Sabbatici: sine quibus in
harum rerum tractatione omnis conatus irritus. 

Rursus \& eorum
quoque fallax vsus est, nisi quædam annorum ex illis periodus instituatur.

Sed eæ sunt totidem, quot aut formæ annorum, aut civilia
initia.

Nam in anno Ægyptiaco Nabonassari alia opus est, ac in anno
Solari, quia diversa forma: item in anno Actiaco siue Diocletianeo
alia, ac in Iuliano, propter diversa initia.

In anno Ægyptiaco vago
naturales characteres sunt \textgreek{εἰκοσιπεν σαετηρις[?]} Lunaris, \&
\textgreek{έπταετηρις[?]} Solaris:
civilis autem character est quadriennium, quem canicularem
annum minorem vocabant Ægyptij.

Hi tres characteres in se ducti
producunt periodum magnam annorum 700 Ægyptiacorum: qua
vti debet disputator temporum, siquidem rationes suas ad annos
Nabonassari, Armeniorum, aut Persarum exigit.

At qui anno Iuliano,
quæ omnium formarum temporibus est conuenientissima, vti
volet, is cyclo vtriusque fideris quindecies ducto componet elegantissimam
periodum annorum 7980, cuius initium in cyclo Solari,
\& Indictione Romana, a Kal. Ianuarij, in cyclo Lunari a Martio, in
anno Sabbatico ab autumno.

Itaque non minus utilis, quam necessaria
est.

Sine ea nihil agit Chronologus: cum ea tempori, \& sæculis
imperat.

Quam enim lubricum sit retro ab aliqua epocha notare tēpora,
quod maior pars doctorum virorū facit, satis nos vsus docuit.

His ita positis, ad singula huius operis membra venio.

%% == Libro primo
Libro primo
præter divisionem temporum, \& iucundissimam mensium, \&
annorum historiam, de antiquissima anni forma disputatur, quæ in
menses æquabiles annum describit, qua pleraque omnes Græcia vsa
est, \& ab ea omnis ratio Olympiadum pendet: nisi potius eam e ratione
Olympiadum propagatā dicas: quod sine cognitione Olympiadum
numquam tam eximium vetustatis \& temporum monimentum
in lucem eruissemus. 

Ex tanta autem Græcorū scriptorum
copia vnicus Pindarus nobis facē alluxit, qui solus nos docuit tempus
ludicri Olympici.

Aliter, quæ paucitas est bonorum scriptorum,
nulla erat via ad hæc interiora perueniendi.

Huius anni Græci
formæ doctrina tanto acceptior esse debet, quanto obscurior eius
rei apud maiores nostros scientia fuit: cum ante hos mille quadringentos
plus minus annos eius rei neque volam, neque vestigium
vetustas retinuerit.

%\end{parnumbers}
\clearpage
p. III [pdf 30]
%\begin{parnumbers}
Nam falso veteres multi, ac post eos infamæ antiquitatis
scriptores, Macrobius ac Solinus, atque proauorum memoria
summus vir Theodorus Gaza, annum Græcorum statim
ab initio merum Lunarem fuisse prodiderunt.

Quamuis enim in
Panegyribus suis, ac nobilioribus sacris, quæ certo annorum circuitu
redibant, vnius Lunæ rationem habebant, tamen, vt vno verbo
dicam, eorum anni forma Lunaris non erat.

Olympicum enim ludicrum
ipsa Lunæ plena lampade celebrabatur, vt solus veterum nos
docet Pindarus. 

Prætera Laconibus ante plenilunium, aut nouilunium
aliquid incipere religio erat.

Vnde \textgreek{Λαχώνιχας ςελωοαζ}[?] vicinorum
prouerbio iactatas, \& contra Arcadibus prouerbiali conuicio
neglectum religionis obiectum legimus. 

Quod enim ante nouilunium,
aut plenilunium vt plurimim bella aut alia seriora aggrederentur,
ob eam rem a finitimis nationibus \textgreek{[Greek]} vocabātur:
quæ conuicij caussa ab ipsis Arcadibus interpretatione elusa est,
probrum in laudem conuersum ad vetustatem originis suæ referentibus,
\& antiquiorem sidere gētem suam gloriantibus. 

Quod igitur
nouilunij ac plenilunij tempora Panegyricis ludicris deligebant,
propterea sacra trieterica instituta: cuiusmodi erant orgia Bacchi,
Nemea, Isthmia, alia.

Ea enim est anni Græcanici forma, vt si, verbi
gratia, nouilunium in neomeniam Gamelionis incurrat, plenilunium
in eandem neomeniam incidat anno tertio redeunte.

Itaque
cum in Tetraeteride orgia Bacchi trieterica celebrabantur, tertio
anno redibant in eum sistum Lunæ, qui priorum orgiorum situi oppositus
erat.

Quare elegantissime Statius trieterida vocat alternam:
quia alternis in nouilunium, \& plenilunium incurreret.

At sacra,
quæ necessario eodem Lunæ tempore obibantur, ea semper erant
tetraeterica: vt in Attica Panathenaica maiuscula, in Elide Olympias,
vt iam tetigimus, plenilunio.

quod sane fieri non poterat, nisi absoluta
Tetraeteride, \& Pentaeteride ineunte.

Atq; ita Tetraeterides
in idem \textgreek{χῆμα} Lunæ, non vtiq; in idem tempus Solis redibant.

Vt
enim in orbem Solis \& Lunæ redirent, non aliter putabant fieri,
quam octaeteride confecta, eneaeteride ineunte.

Ex quo quædam
eneaeterica sacra eo nomine instituta: cuiusmodi ab initio Pythia
fuerunt: \& quidem merito.

Apollini enim, quem eundem cum Sole
faciebant, erant attributa.

Hinc colligimus, non solum Olymiadis
interuallum annis quatuor solidis explicatum fuisse; sed etiam puerliter
peccare eos, qui annorum quinque solidorum fuisse putant.

Neq; vero quibusdam recētioribus succensendum, qui ita censent,
ita scribunt, sed \& Ausonium nostratem culpa liberat Ouidius, scriptor
longe antiquior, \& nobilior, qui ætatem suam quinquaginta annorum
decem Olympiadibus definit: quo magis mirum Pausaniam
hominem Græcum in ea hæresi fuisse, vt suo loco a nobis relatum est.

%\end{parnumbers}
\clearpage
p. IV [pdf 31]
%\begin{parnumbers}
Nam minus mirandum de Solino, qui cap. \textsc{xiii} Isthmia vocat
quinquennalia, quæ erant tantum triennalia, quod certamen a Cypselo
tyrāno intermissum, anno primo Olympiadis 49 instauratum
fuisse dicit.

Horum igitur omnium caussæ ad typum anni Græci referendæ
sunt.

In quo argumento nihil eorum prætermisimus, quæ
ei rei illustrandæ faciebant, quanquam pene omnibus præsidiis
destituti.

Et quidem primum in genere, quod semper solemus, deinde
priuatim multarum Græciæ nationum periodos proposuimus,
quæ quidem non anni forma, sed situ \& capite inter se differunt: in
qua tractatione diu nobis res fuit cum præstantissimo viro Theodoro
Gaza, vel potius cum eius sequacibus, a quibus extorqueri non
potest doctrina \& situs mensium, ab illo primum proditus. 

quæ quidem
velitatio nobilioribus ingeniis, \& ab omni inuidia remotis, vt
spero, iucunda erit.

Quid enim est toto libro primo, cuius vel minima
pars, nō dicam istis querulis, qui nihil sciunt, sed etiam doctioribus,
hoc sæculo, \& ante multa retro sæcula oboluerit?

Quid dicam \textgreek{[Greek]}?

quis illarū caussas, \& vsum sciebat?

quis
locum nobilem de illis in Verrina Ciceronis intelligebat?

quis locum
\textgreek{Εξαιρέσεως[?]} in secunda Boedromionis?

quis Posideonem intercalarem
mensem fuisse?

Huic materiæ accessit \textgreek{ἐποχη χέντςα[?] θερινᾶ}
in \textsc{viii} Iulij, quæ in priori editione omissa erat.

Id erat \textgreek{χάντσον[?]}
populare, quod nomine \textgreek{τςοπων[?] θερινῶν[?]} Aristoteles, Thophrastus,
Plutarchus, \& omnes veteres intelligunt, non autem ipsum verum
Solstitium: quæ rei pulcherrimæ notatio nobis viam ad illustriora
præiuit.

Quod Solstitiorum, \& Æquinoctiorum puncta \textgreek{χέντρα} vocentur,
satis sciunt, qui veterum Græcorum libros legerunt.

Columella
cardines vocauit.

In præstantissimo Parapegmate, quod falso
Ptolemæo attribuitor (est enim antiquius Ptolemæo) ad \textsc{viii} Kal.
Iulij (quod est Solstitium Sosigenis) annotatum est: \textit{Æstiuus cardo,
\& momentanea aeris perturbatio}.

In Græco (vtinam haberemus!)
sine dubio fuit: \textgreek{Θερινὸν χέντρον, χαὶ ςιγμιαία αἔσος Ιασοιχή[?]}.

Igitur \textgreek{χέντρον
θερινὸν} nihil aliud, quam \textgreek{τροπαὶ θεριναί}.

Cur \textsc{viii} dies Iulij erat
epocha æestiua in vsu ciuilis anni, non semel caussam reddidimus. 

Adiecta etiam pernecessaria neomeniarū Atticarum Tabula: quæ
non solum priori editioni, sed etiam doctrinæ anni Attici deerat.

Liber secundus anno Lunari dicatus est ideo, quia is annus ex illo
Græco æquabili manasse videtur.

Ibi aperitur omnis antiquitas \textgreek{ἔτοις[?]
πρυτανείας[?]}, Octaeteridum Cleostrati, Harpali, \& Eudoxi: quæ omnia
hodie nomine tenus nota erant.

Eudoxea Octaeteris numquam
in vsus ciuiles admissa est.

Anni vero \textgreek{πρυτανείας[?]} in vetustissimis Psephismasin
Atheniensium primo quidem ex Cleostratea, deinde, illa
abrogata, ex Harpalea petiti sunt.

%\end{parnumbers}
\clearpage
p. V [pdf 32]
%\begin{parnumbers}
Sequitur magnus annus Metonicus
ambarum, \& Calippicus Metonici castigator.

Et quidem hi
ambo nomine noti tantum: caussarum autem, \& omnium, quæ ad
illa pertinent, mira ignoratio hactenus fuit.

Accesserunt huic editioni
Tabulæ operosissimæ dispensationum neomeniarum Metonicarum,
\& Calippicarum: cuiusmodi etiam in Harpalea Octaeteride
exhibuimus. 

Quod de Eudoxea Octaeteride diximus, idem de
periodo Chaldæorum dicendum, eam nunquam ad ciuilia tempora,
sed ad Genethliacorum themata vsurpatam fuisse.

Id quod tum
multa argumenta, tum vnicum certissimum illud est, quod eorum
menses appellationibus Macedonicis, nō vero Chaldaicis fuerunt.

Propterea recte cum illius anni diatriba doctrinam dodecaeteridis
Chaldaicæ Genethliacorum coniunximus, cuius nomen quidem
solum notum erat ex Censorino: cognitio autem nobis ex Arabum,
\& Orentalium vsu repetenda fuit.

An aliquis Græcorum \textgreek{δωδεχαετρίδος Χαλδαιχῆς}
meminerit, haud promptum est dicere.

Vnum tantum Orpheum siue Onomacritum eius meminisse scimus. 

\textgreek{ὀρφδὶςὀν[?] ταῖς[?] δωδεχαετνρίσιν[?]:}

\begin{greek}
ἔςαι δ᾽ αὖθις ἀνὴρ, ἢ χοίραν[G-circ], ἠὲ τύρανν[G-circ],

ἢ βαοιλδὶς, ὂς τῆμ[G-circ] ἐς οὐρανὸν ἴξε[right curl] αἰπιιύ [all doubtful].
\end{greek}

Est apotelesma cuiusdā Genethliaci consulti super alicuius genesi,
de quo ipse respondit, eum fore magnum regem aut Dynastam, \&c.

Citat Tzetzes. 

Hæc multum illustrant doctrinam Dodecaeteridos
genethliacæ parum antehac notæ.

Itaque quemadmodum \textgreek{τελετὰς}
ita etiam \textgreek{δωδεκαετνρίδας[?]} scripserat Onomacritus sub nomine
Orphei.

Qualis fuerit Iudæorum annus sub Seleucidis, quibus parebant,
multis exemplis testatum reliquimus: in quibus etiam translationis
feriarum in capite anni antiquitatem asseruimus aduersus homines
nostrorum temporum, qui nugantur commentum nuperum
Iudæorum esse.

In illis Doctor Theologus ingenti commentario
suo in Euangelium secundum Iohannem exultabundus ait illam
translationem confutari ex loco Iosephi, in quo scribit, quo anno
Hyrcanus fœdus icit cum Antiocho Sidete, Pentecosten fuisse feria
prima.

Hunc locū Iosephi nos olim in priore editione produximus,
vnde is, aut qui illi indicauit, accepit.

En, inquit, duo Sabbata continua.

Si propter continuationem duorum Sabbatorum, feria transfertur,
ergo vbi sunt duo continua Sabbata, non transfertur. 

In quibus aperte ostendit se ignorare caussam feriæ transferendæ, quæ fiebat
propter solum Tisri, non autem propter alios menses; propterea
quod ille mesis multa solennia habet, adeo vt si non habeatur
ratio translationis, aliquando non solum duo, sed entiam tria continua
sabbata concurrere necesse sit.

%\end{parnumbers}
\clearpage
p. VI [pdf 33]
%\begin{parnumbers}
Si enim feria sexta inciperet
neomenia Tisri, omnino tria sabbata continuarentur, neomenia,
siue clangor tubæ, sabbatum ordinarium, \& ieiunium Godoliæ.

Continuantur autem sæpernumero in aliquo reliquorum mensium
duo Sabbata: idque fit, quando solenne est aut feria prima, aut feria
sexta.

quorum alterutrum quotannis incidere, nisi quando Tisri
incipit feria tertia, Doctor ignorauit.

In primam feriam incidunt
hæc solennia, \textsc{xxv} Casleu, \& \textsc{x} Tebeth in anno defectiuo tam
cōmuni, quam embolimæao, quotiescunq; Tisri incipit feria secunda:
\textsc{vi} Sivvan; quando Nisan incipit feria septima:
\textsc{xv} Nisan, \textsc{xvii} Tamuz,
\textsc{ix} Ab, quando Nisan incipit feria prima.

In feriā autem sextā
conuenit solenne \textsc{xxv} Casleu \& \textsc{x} Tebeth, quādo Tisri est feria
septima in anno tā communi, quā embolimæo.

\textsc{xiiii} Adar, quando
Nisan sequens est feria prima: \textsc{vi} Sivvan, quādo Nisan feria quinta.

Vides, quot Sabbata quotānis, nisi quādo Tisri incipit feria tertia, Iudæi
continuent in aliquo mensium, præterquam in solo Tisri, cuius
vnius gratia illa cautio instituta
% No period at end of sentence

Itaq; doctor tā frustra, quam ridicule
Iosephi testimonium adduxit de sexta Sivvan, id est, Pentecoste
feria prima; cum illo anno neomenia Nisan fuerit Sabbatum.

Atqui
nihil superesse putauit, quam vt Vaticani montis imago redderet
\textgreek{ἰὴ παιαή[?]}.

Sed ipse valde ignarus est harum rerum, vt reliqui omnes,
qui contendunt nouitium esse Iudæorum commentum.

Nos
validissime demonstrauimus, \& sæculo Christi, \& retro sub Seleucidis
translationes in vsu fuisse.

\& sane res peruetusta est.

quæ tamen
non minus ignorata, quam periodus Calippica, qua Seleucidæ, \&
Seleucidarum edicto Iudæi vsi.

quod non solum ex Nisan anni excidij
Hierosolymorum a nobis demonstratum est, sed etiam patet
ex definitione Rabbi Adda.

Is annum definit dierum \textsc{ccclxv},
horarum 5, 997/1080. 48/76.

Quid hac definitione aliud vult, quam periodum
Iudaicam fuisse annorum 76?

Cum Meto definit annum dierum
365. hor. 5. 1/19. ex eo coniiciendum relinquit, se vti periodo annorum
19.

Vtebantur igitur periodo 76 annorum, id est, Calippica:
\& tamen in omnibus neomeniis Lunæ \textgreek{φάσιν} obseruabant, non
quod eam ex præscripto periodi non indicerent, sed ideo, vt eam
sanctificarent.

nam \& hodie quoque obseruant \textgreek{φάσιν}, non vt ex ea
neomeniam indicant, sed vt eam sanctificent.

Itaque Luna statim
visa dicunt: \texthebrew{[Hebrew]}.

\textgreek{ἀγαθὸν τέρας ἔςω ἡμῖν ης[??] παντὶ Ισραήλ.}

Idem faciunt \& Muhammedani, quamuis neomenias ex
scripto indicere soleant.

Neque aliud intellexit fabulosus quidem,
sed tatem vetus auctor \textgreek{[Greek]} apud Clementem:
\textgreek{[Greek][Lots of Greek]}.

%\end{parnumbers}
\clearpage
p. VII [pdf 34]
%\begin{parnumbers}
Præclara quidem ista: sed nescit, quid dicit.

Nam in Iudæorum potestate
nunquam fuit, vt exspectarent \textgreek{φάσιν}: quia raro Luna se ostendit,
nisi secundo post coitum die.
% Greek: phase

Quod si expectandum ipsis esset,
res ridicula accideret, vt Elul, qui semper est cauus mensis, non solum
plenus, sed etiam aliqando vnius \& triginta dierum esset.

Sine dubio translationem feriæ intelligit, cuius caussam ignorat.

\textgreek{[Greek]} vocat \texthebrew{[Hebrew]} caput anni.

Nam Sabbatū vocat, quia Festus
dies, \textgreek{[Greek]}.

Ita etiā vocatur Leuitici \textsc{xxiii}, 24.

\textgreek{[Greek]} intellige
\textgreek{[Greek]}: quod ita Hebraice vocetur, nēpe \texthebrew{[Hebrew]}.

Vide in Computo Iudaico.

At \textgreek{[Greek]} vocat \textgreek{[Greek],
[Greek]} quoq;, id est \texthebrew{[Hebrew]}.

Nam aliæ erant \textgreek{[Greek]},
proinde vt \& \texthebrew{[Hebrew]}.

Sic Tertullianus magnos dies dixit, quos
Hebræi \texthebrew{[Hebrew]} vel \texthebrew{[Hebrew]}.

Eius verba sunt ex v in Marcionem:
\textit{Dies obseruatis, \& menses, \& tempora, \& annos, \& Sabbata, vt opinor,
\& cenas puras, \& ieiunia, \& dies magnos.}

Sed quid Tertullianum
aduoco?

Ecce Biblia Græca ita vertunt ex primo cap.

Isaiæ:
\textgreek{[Greek], [Greek], [Greek]}.

Quod Hebraice est \texthebrew{[Hebrew]}, vertunt \textgreek{[Greek]}, quod idem
est quod \texthebrew{[Hebrew]}: \& quidem manifesto Sabbata distinguuntur a
magnis diebus. 

Quare perperam quidam \textgreek{[Greek]} interpretantur
Sabbatum apud Iohannem, \textgreek{[Greek]}. de quo infra.

Quin \& Tertullianus ipse \textgreek{[Greek]},
quas cenas puras vocat, a diebus magnis, \& a ieiuniis, \& a
Sabbatis distinguit.

De Cena pura, præter id quod diximus ad
Festum, ita reperi in veteri \& peroptimo Glossario Latinoarabico:
\textit{Parasceue, cena pura, id est, praparatio, que fit prosabbato.}

Conditor Annalium Ecclesiasticorum turbat de cena
pura, \& negat esse parasceuen, quia cena pura apud Festum
habeat offam suillam.

Sed ipse, (pace docti viri dixerim) non
aduertit Puram dici, non quia careat carnibus, sed quia religionis
\& dicis caussa fit.

Nam \& parasceuæ Iudaicæ habent carnes,
\& nihilominus dicuntur cenæ puræ, quod dicis caussa coquebantur,
coquunturque hodie prosabbato, quia in Sabbato
coqui non liceat.

Non negabis, candide Lector, hæc vulgo non intelligi.

Itaque locus ille est nobilissimus. 

Tamen quotus quisque est ex tot Lectoribus, qui non hæc aut præteribit,
aut calumniabitur?

Sequuntur periodi magnæ Hagerenorum,
ex quibus ratio anni soluti Indorum, \& Muhammedanorum
tota pendet.

Omnia nunc primum ex Arabum scriptis
prodeunt: atque adeo omnis tractatio nostris hominibus
noua est.

%\end{parnumbers}
\clearpage
p. VIII [pdf 35]
%\begin{parnumbers}
Excipit hanc doctrina anni Iudaici hodierni, res, quod
sæpe diximus, artificiosissima, ideoque eximia, quia melior
anni Lunaris forma constitui non potest.

Docemus præterea, vnde
natus sit ille annorum computus, quo vtuntur hodie, a \textsc{vii} Octobris:
quem inepte putant a conditu rerum.

Post multarum Periodorum,
Cyclorum, Octaeteridum, Paschalium historias, in locum vltimum
\textgreek{[Greek]} veteris anni Romanorum coniecimus, ideo
quod ea forma proxime abesset a Lunari: vbi de sæculo Romano,
\& capite veteris anni Romani, temporibus vltimis C. Iulij Cæsaris,
multa accuratissime disputata.

Itaque ex singulis rebus singula capita
confecimus, cum potius singuli libri \& quidem ingentes confieri
possent, si, quæ hominum hodiernorum est ambitio, eadem nobis
incessisset.

Tertio libro opportune annus æquabilis datus est,
cum annus Solaris Ægyptiacus, adscitis diebus quinque, ex Græco
propagatus sit: (quemadmodum annus Lunaris ex eodem Græco
manauit, abiectis ab eo totidem diebus cum horis \textsc{xv}, paulo amplius)
quod, metacente, Plutarchus docuit in libro \textgreek{[Greek]}.

Adeo inter se libri nostri mutuo conspirant, neq; ab eis ratio,
methodus, \& ordo abest.

In eo libro de Neuruz antiqui Persarum
periodo annorum \textsc{cxx}, deq; cognominibus dierum Persicorum,
de translatione \textgreek{[Greek]} in enthronismis nouorum Regnum,
item de caussis anni Iezdegird, de annis Armeniorum, \& eorum
mensibus, omnia noua protulimus. 

Sed hæc non expergefacient animos
hominum, nisi forte ad obtrectandum.

Quartus liber est emendatio
tertij, vt secundus primo erat subsidiarius: qua methodo imperfectus
Lunaris Græcus libro primo disputatur, vt perfectus secundo.

Sic etiam perfectus Solaris, \& siqui alij naturam perfecti immitantur,
supplent id, quod æquabili Ægyptiaco, Persico, \& Armeniaco
vetuitas detraxerat.

Itaq; in quatuor partes tribuendus fuit.

In
prima continentur anni, quibus quarto anno exeunte dies ex quatuor
quadrantibus conflatus accrescit.

Ex illis nobiliores selegimus,
Iulianum, Actiacum, Antiochenum, Samaritanum, \& alios. 

Nam \& alios quoq; eius formæ habebamus, vt Tyriorum, quorum menses
appelationib. Macedonicis, diuersa initia a Iulianis habent.
% "appelationb." interpreted as abbriviation for "appelationibus"

Sic
etiam Gazensium annus mere Actiacus fuit, appellationibus mensiū 
Macedonicis, mensibus tricenariis. 

Marcus Ecclesiæ Gazensis Diaconus,
in actis Porphyrij Gazensis Episcopi vocat \textgreek{Διον} Nouembrem
\textgreek{Απελλαιον} Decembrem quæ nomina habent a Macedonibus. 

Sed
idem scribit Gazenses celebrasse Theophaniorum diem vndecima
Audynæi, quæ est sexta Ianuarij Iuliani, se autē redisse Constantinopoli,
Xanthici vicesima tertia, quam ait fuisse decimam octauam
Aprilis secundum Romanos: quibus ostēdit formam illius anni mere
Actiacam fuisse, mensibus tricenariis, appellationibus Macedonicis. 

%\end{parnumbers}
\clearpage
p. IX [pdf 36]
%\begin{parnumbers}
Secunda pars annos emendatos, eorumque emendandorum
rationem complectitur: tertia periodos multiplices, quarum finis
conciliatio anni ciuilis cum Solari, cui dies quinto quoque anno
ineunte accrescit.

Quarta pars agit de vera emendatione anni, \&
de anno cælesti instituendo, qui pertinet ad methodum epochæ
mundi.

Quemadmodum autem nulla Lunaris anni ciuilis ratio
recta iniri potest, præter eam, qua Iudæi vtuntur: ita nullus annus
cælestis Tropicus recte institui potest, nisi ex forma, quam edidimus,
quam nemo vituperabit, nisi qui ignorauerit; omnis laudabit,
qui intellexerit.

Alioquin scio \& malignos \& obtrectatores non defuturos. 

Annus tam noster, quam Iudaicus ciuilis quidem, sed naturalis,
vterq; ad motum quisq; sui sideris descriptus. 

Ideo eius saltem
in scriptis vsus esse debet, qualis olim Philadelphi Dionysianus,
Chaldæorū Calippicus, hodie Persarum Gelaleus. 

Tres igitur libri
primi, \& prima pars quarti pertinent ad \textgreek{[Greek]} temporum ciuilium
cum septimo.

At reliquæ tres partes quarti cum duobus libris
sequentibus pertinent ad ipsam emendationem temporum.

Atque
vt a mundi primordiis omnes res deducuntur, ita mundi epocham
primā ordine posuimus: qua in re quam pueriliter hallucinati sint
omnes, nō sine admiratione tam imperitiæ quam pertinaciæ eorum
dicere possum.

Non loquor de iis, qui sæculo vno, aut pluribus altius
originem rei repetunt.

Nam quemadmodum ij nullam rationem
sibi proposuerunt, quam sequerentur, ita nullos lectores nancisci
possunt, nisi imperitos. 

Qui intra sæculum maiorem mundi
epocham faciunt, eorū duo genera reperio.

Prius genus est eorum,
qui solutionem captiuitatis in primum annum Olympiadis \textsc{lv} conferunt:
alterum eorum, qui tempus illud \textsc{xviii} aut \textsc{xix} annis ante
\textsc{lxiiii} Olympiadem definiunt.

In priori hæresi fuerunt \&
quidam veterum Ecclesiasticorum, vt alicubi indicauimus. 

Aiunt
Cyrum cæpisse imperare primo anno Olymiadis \textsc{lv}, hoc est 217
anno Iphiti, quod verū est: de quibus deductis septuaginta, relinquitur
annus excidij Hierosolymorū, \& casus Sedekiæ 147 a primo ludicro
Olympico.

Sed puerilis sentētia multis absurditatibus eluditur.

Primo computatione non recta annorum \textsc{lxx} a capto Sedekia.

Deinde quod Cyrum statim initio regni sui Regem Mediæ, Persidos,
Susidos, Assyriæ, Babyloniæ, totius Asiæ minoris, Indiæ, totius
Syriæ constituunt, qui vnius Persidos Rex fuerit aliquot annis ante
casum Astyagis[?], \& post illud tempus pauculis annis ante obitum
Babylone potitus sit.
% Astyagis or Aftyagis?

Hæc sola absurditas facit, vt non solum eorum
nulla ratio habeatur, sed vt ludibriū quoq; debeāt.

Tertio 147 annus
Iphiti est 118 Nabonassari: qui erat annus quintus ante initiū Nabopollassari
patris Nabuchodonosori.

%\end{parnumbers}
\clearpage
p. X [pdf 37]
%\begin{parnumbers}
Ergo Nabuchodonosor anno
decimono regni sui templum \& Hiersolyma euertit annis quinq;
ante quam pater ipsius, cui ipse successit, regnaret.

Digna profecto
talibus doctoribus sententia.

Tamen tantum abest, vt hac tam insigni
absurditate a sententia desistant, vt animos ab eiusmodi portentis
opinionum sumant.

Postremo ignorant diuersa esse initia Regnum,
vt ipsius Nabuchodonosori, cum patre, \& solius: Alexandri,
ab excessu Philippi patris, \& ab initio Seleuci: Diocletiani, ab æra
martyrum, \& a primo anno imperij.

Sic etiam Cyri, apud Græcos, ab
initio regni Persidis: apud Babylonios, vel a subacto toto Babyloniæ
imperio, vel ab aliquo insigni facto, quodcunque illud fuerit, siue
ex edicto ipsius Cyri, siue translatione \textgreek{πδν ἐπα γο υ[?]ων},
vt solebat fieri.

Qui tantam inscitiam sequi noluerunt, non tamen rectam viam
institerunt, quia quindecim aut amplius annis ante \textsc{xlvi} Olymiadem
casum Sedekiæ coniiciunt.

Nos ante annum quartum illius
Olympiadis id non potuisse accidere ita demonstramus. 

Ezekias
Rex Iuda, postquam singulari Dei beneficio ab ancipiti morbo conualuisset,
anno \textsc{xiiii} regni sui, accepit Legatos \& \textgreek{ςωτήρια[?]}, a
Merodach Rege Chaldæorum.

Ponamus \textsc{xiiii} annum Ezekiæ in
primo anno Merodach, hoc est, in \textsc{xxvii} Nabonassari.

Nam is est
annus primus Merodach apud Ptolemæum ex Chaldaicis obseruationibus. 

Hoc modo annus primus Ezekiæ conuenerit in annum
\textsc{xiiii} Nabonassari. Ab initio Ezekiæ, ad excidium templi, anni
sunt absoluti 138.

Hoc est, annus ipsius excidij est 139 labens ab initio
Ezekiæ.

quod ita demonstramus. 

Annus primus Sedekiæ est
quartus Hebdomadis, teste Ierimia, initio cap. \textsc{xxviii}: \& proinde
vndecimus, qui \& vltimus, est Sabbaticus. 

de quo extat testimonium
apud Ieremiam, \& nemo dubitat.

Rursus annus tertius decimus
Ezekiæ erat Sabbaticus. 

auctor Isaias \textsc{xxxvii}, 30.

ex quibus
manifesto colligitur, \textsc{xiiii} Ezekiæ esse primū Hebdomadis, \& primum
Ezekiæ esse sextum Hebdomadis. 

Ergo annis ab initio Ezekiæ vnitas addenda, ad methodum anni Sabbatici.

Addita vnitate annis
139, numerus erit septenarius. 

Quare annus labens 139 est verus
annus ab initio Ezekiæ.

Quibus additis 13 annis Nabonassari præteris
(quia posuimus 14 Nabonassari primum Ezekiæ) componitur
annus Nabonassari 152, in quo casus Sedekiæ ex hac hypothesi
locādus est, hoc est, in anno periodi Iulianę 4118: de quibus deductis
907 absolutis ab Exodo, remanet annus Exodi 3211 in periodo Iuliana.

Porro Nisan Exodi cæpit feria quinta, vt toties diximus, \& ex
Mose rectissime ante nos Iudęi docuerūt.

At in anno periodi Iulianę
3211 Nisan nō cæpit feria quinta, sed feria tertia, Martij \textsc{xi}, cyclo tā
Solis, quā Lunæ \textsc{xix}.

%\end{parnumbers}
\clearpage
p. XI [pdf 38]
%\begin{parnumbers}
Ergo annus proximus, quo Nisan cæpit feria
quinta, is debuit saltē esse annus Exodi: atq; adeo is fuerit annus periodi
Iulianæ 3214: in quo sane nisan cæpit feria quinta, Aprilis \textsc{vi},
cyclo Solis \textsc{xxii}, Lunæ \textsc{iii}.

Additis 907 annis absolutis ab Exodo,
annus 4121 periodi Iulianæ suerit is, in quo excidium templi contigit:
qui est quartus Olympiadis 46, vt erat propositum.

Sed \& post
Olympiadem 46 ponendum esse casum Sedekiæ ita probabimus.

Amasis rex Ægypti, postquam regnasset annos 55, obiit circiter annum
7 Cambysis, anno ante excessum ipsius Cambysis, hoc est, anno
225  Nabonassari.

Nechao intersectus est a Nabuchodonosoro anno
quarto Ioiakim regis Iuda.

Ieremias \textsc{xlvi}, 2.

Post eum regnauit
Psammitichus annos \textsc{vi}.

Cui Aprias, cuius meminit Ieremias
\textsc{xliiii}, 30, succedit.

Is post \textsc{xxv} annos relinquit regnum Amasi.

Summa annorum a cæde Nechao ad obitum Amasis anni 86, qui
deducti de 225, relinquunt annum Nabonassari 139, quartum Ioiakim
Regis Iudæ, primum Nabuchodonosori.

Ergo Sedekias captus
anno 158 Nabonassari, qui erat tertius 47 Olympiadis.

Idque verum
esse postea validissime demonstrabimus.

Diodorus Siculus,
auctor omnium Græcorum certissimus, attribuit, \textsc{lv} annos Amasidi.
reliquos Apriæ \& Psammatichi habemus ex Herodoto.

Temere
igitur, \& imperite faciunt, qui casum Sedekiæ antiquiorem illo
tempore constituunt: neque his cassibus sese explicare poterunt,
quantumuis sua commoueant sacra, vt Plautus loquitur.

His valide
demonstratis, \& licentia chronologorum intra aliquos fines summota
quos amplius migrare non possunt, ad originas ipsas penetremus.

Sed prius vt in Mathematicis cōcessa quædam, aut quæ negari
non possunt, assumuntur, ita \& nobis quoque faciendum.

Tempora \& initia Regum Babyloniæ a Chaldæis notata in obseruationibus
eclipticis, quæ reiicere \& damnare extremæ impudentiæ \&
inscitiæ est: item, eorundem regum initia \& tempora a Beroso Chaldæo,
qui minus quam tribus sæculis post illos vixit, \& qui quæ Actis
ac fastis Babyloniorum publicis continebantur, ignorare nō potuit,
hæc inquam, non tantum tāquam vera haberi postulamus, sed etiā
qui aliter putant, tanquam indignos censeri, qui aut audiri a nobis
mereantur, aut vllas literas attingant, aut aliquem locum inter
doctos habeant.

Tricesimum annum, cuius initio Prophatiæ suæ
meminit Ezekiel, quique capti Iechoniæ quintus erat, Iudæi inepti
deducunt a libro legis reperto, anno \textsc{xviii} Iosiæ Regis.

Quis vnquam a libro reperto vllam æram, aut edicto Iosiæ institutam, aut a
Prophetis vsurpatam legit?

Si tanti erat illa temporis nota, quare
eam nō vsurpat Ieremias, qui tā accurate annos Regum Iuda Iosiæ,
Ioiakim, Iechoniæ, Sedekiæ notare solet?

Capite \textsc{xxv}, quare dicit
anno quarto Ioiakim, cum dicendū esset vicesimo secundo a libro
inuento?

Esto, cur Ezekiel dixit tricesimo, non tricesimo a libro inuento?

qui tamen dixit anno quinto deportationis Regis Ioachin.

%\end{parnumbers}
\clearpage
p. XII [pdf 39]
%\begin{parnumbers}
Certe mos est vti epocha, quæ omnibus \& nota \& in vsu sut.

Quare
igitur epocham producit, neque plebi notam, neque in vsu positam?

Sed quid ea epocha opus in Babylonia, inter deportatos?

Nugæ Iudæorum,
nugæ sunt istæ, \& halluciationes doctorum, qui eos sequūtur.

Quare eruditiores Iudæorum, huius absurditatis \& nugatoriæ
caussæ conscij, his ineptiis explosis, dicunt, illum annum non a
libro inuento, sed Iubilei fuisse tricesimum.

At hoc est litem lite decidere.

Nam, quomodo Iudæi annos a Iubileo putarent, qui Iubilea
numquam vsurparunt?

Annos quidem Hebdomadis notant, vtinitio
\textsc{xxviii} Ieremiæ mentio anni quarti septimanæ: \textit{Initio regni
Sedekia, anno quarto.}

Rursus mentio anni primi, \& secundi in annis
\textsc{xiiii}, \& \textsc{xv} Ezekiæ, apud Isaiam \textsc{xxxvii}, 30.

Sed notationem
per Iubilea, imo ne Iubilei quidem mentionem, nusquam, nisi
in lege, reperies.

Præcepta fuit tantum, non recepta Iubilei obseruatio.

Sed quæ hæc plumbea Iudæorum sententia a \textsc{xviii} Iosiæ
Iubileum putare?

Iubilea putantur a primo anno hebdomadis, non
a septimo.

At \textsc{xviii} Iosiæ suit septimus septimanæ, non primus.

Quare, si a Iubileis annos putare mos esset, suerit hic annus non vtique
tricesimus, sed vndetricesimus Iubilei, a \textsc{xviiii}, non a
\textsc{xviii} Iosiæ.

Denique is erat annus 862 ab excessu Mosis, 855 a
diuisione terræ siue \textgreek{[Greek]}.

Ergo suit vicesimus secundus, non
vndetricesimus Iubilei.

En quot errores locus præpostere sumptus
nobis peperit.

Cum igitur neque a libro legis inuento, quod est absurdissimum,
neque a Iubileo, quod est falsum dupliciter, ille tricesimus
annus putandus sit; sequitur, quod negari non potest, a
quodam rege tunc imperante putandum esse.

Nam deportati \& captiui
inter victores, qua epocha vti possunt, nisi victoris?

In Palæstina,
cum aliqua esset Iudæorum Respublica, \& Ecclesia bene constituta,
Iudæ cogebantur vti anno Alexandreo dominorum Seleucidarum:
quanto magis Chaldæorum, in media Chaldæa, nullis legibus,
nulla Republica, nulla Ecclesia.

Nehemias initio libri sui ita
scribit: \textit{Accidit mense Casleu, anno vicesimo, cùm eßem in castro Susan.}

Si alibi non expressisset se de vicesimo anno Artaxerxis loqui, haud
dubie aliquod Iubileum hic commenti essent inepti Iudæi, \& inepti
quidam hominum nostrorum sequuti essent.

Eodem quoq; modo
loquitur Ezekiel \textit{anno tricesimo}, non adiecto regis nomine.

Quid enim opus erat in Chaldæa?

Duo ergo Reges simul imperabant,
Nabuchodonosor, \& ille, qui iam tricesimum annum currentem
imperabat.

Quisnam Rex, obsecro, potuit trecesimum annum in regno
agere, cum iam Nabuchodonosor tertium decimum regnaret?

Non alius igitur suerit, præter Nabopollassarum patrem Nabuchodonosori,
quod verum est.

%\end{parnumbers}
\clearpage
p. XIII [pdf 40]
%\begin{parnumbers}
Nam \textsc{xxix} solidos annos imperauit,
teste Beroso.

Quod si filius eius anno \textsc{xxx} partis iam duodecimum
absoluerat, profecto imperare cæperit anno partis decimooctauo,
qui erat Nabonassari 140.

Nam primus Nabopollassari est 123 Nabonassari,
testibus Chaldæis apud Ptolemæum, ex defectibus Lunaribus
obseruatis.

Et proinde Sedekias captus fuit anno 158 Nabonassari,
tertio autem Olympiadis 47.

Vide locū Berosi apud Iosephum.

Nabopollassarus audita rebellione Ægypti misit filium eo
cum regio imperio, \& regio exercitu: a quo tempore consurgit initium
Nabuchodonosori cum patre regnātis.

Mos erat Regum Babyloniæ
\& Persidis, vt aut prosecturi in expeditionem, filios reges declarent,
aut in expeditionem mitterent cum regio nomine, tanquam
designatos, si contigisset ipsum patrem mori, absente filio, ne
vllus de rege futuro tumultus oriretur.

Exemplum habemus apud
Herodotum de Cyro Cambysen in solium suum collocante in expeditione
in Scythas.

Hinc Ctesias Cambysi attribuit annos 18,
cum tamen solus regnarit octo annos, testibus omnibus veteribus
Græcis, \& Chaldæis ipsis apud Ptolemæum.

Dario vero Notho annos
idem attribuit 35, cum tantum 19 solus imperarit.

Rursus Berosus
\textsc{xxxxiii} annos ait Nabuchodonosorum imperasse, comprehensis
nimirum 13 annis, quos cum patre communicauit, cum
illis quos solus in imperio transegit.

Quare Nabuchodonosori regnum
dixit non Satrapian, tanquam a patre non vt Satrapes, sed Rex
\& socius imperij in rebelles missus.

Verba eius sunt hæc: \textgreek{[Greek]}.

\textit{Victo rebelli, eius regionem regno suo subiecit.}

Mox subiicit, Nabopollassari patris morto[?]
audita, qui \textsc{xxix} annos solidos regnauerat, ipsum Babylonem se
contulisse: quod accidit proculdubio aliquot diebus post illud tēpus
ab Ezekiele designatum.

Obiit enim Nabopollassarus anno regni
sui \textsc{xxx}.

\textgreek{[Greek]}.

Pulcherrima hæc est obseruatio, quam Beroso vernaculo
Babylonicarū rerum scriptori debemus.

Eadem verba repetit Eusebius
De pręparatione euangelica, vbi plane \textgreek{[Greek]}, quemadmodum
est apud Ptolemęum, nominat, nō \textgreek{[Greek]}, vt perperam
est editū in Iosepho: ex quo ineptus quidā duos esse coniecit
Nabulassarum \& Nabopollassarum; cum tamen eadē verba sint, ne
vna quidem syllaba minus, præter illud nomen.

Rursus apud Iosephum
lib.\textsc{x} ca.\textsc{ii}.eadem verba Berosi repetuntur.

Sed vbi hic est \textgreek{[Greek]},
ibi est bis \textgreek{[Greek]}, vtrobique male pro \textgreek{[Greek]}.

Quā bene hæc diuinis scripturis conueniunt?

%\end{parnumbers}
\clearpage
p. XIV [pdf 41]
%\begin{parnumbers}
Vnde etiam sequitur, mortuo Nabopollassaro, non tricesimum annum Nabuchodonosori
dici cæptum in Chaldæa, sed primum quæ res obseruatione
digna.

Iudæi primum annum putarunt ab eo tempore, quo
cum imperio missus est. Sed in Chaldæa primus eius annus consurgit
ab obitu patris.

Itaque Danielis 11, annus secundus Nabuchodonosori
est sine dubio secundus ab obitu Nabopollassari, tricesimus
primus ab initio eiusdem, 152 ab initio Nabonassari, sextus
Sedekiæ.

Vnde indubitata eruintur temporis nota illius Capitis secundi
apud Danielem, qui erat quartusdecimus nnus capti Danielis,
\& sociorum cum rege Ioiakim, sextus autem regni Sedekiæ.

Proinde annus ille erat \textsc{xiiii} Nabuchodonosori in Syria, secundus
autem in Babylonia: non autem \textsc{xxv}, vt coniicit Hieronymus
ex quadam victoria Nabuchodonosori de Syria, \& Arabia, cuius
meminerit Borosus.

At Berosus loquitur tantum vsque ad obitum
Nabopollassari, qui erat \textsc{xiii} Nabuchodonosori eius filij.

His tam illustribus demonstrationibus sua somnia præserant, quibus antiquius
est somniare, quam vera dicere, aut nosse.

Nos ad reliqua
pergamus.

Annus capti Sedekiæ est 158 Nabonassari, 4124 in periodo Iuliana.

Deductis annis 907 solidis, relinquitur annus 3217
Exodi, qui est 2264 Iudaici Computi in quo sane Neomenia Nisan
habuit characterem feriam quintam, secunda Aprilis, Cyclo
Solis \textsc{xxv}, Lunæa \textsc{vi}.

Sed quadragesimus annus, \& quadragesimus
septimus, hoc est 2303, \& 2310 Iudaicus fuit sabbaticus.

Iosuæ
\textsc{xiiii}, 7, 10.

Iudæi dicunt septenarios annorum Computi siue æræ
suæ esse Sabbaticos.

Atqui 2303, \& 2310 sunt septenarij.

Ergo recte
Sabbaticos annos putant Iudæi; vt apud illos post legem nihil
hac obseruatione vetustius sit; res prosecto, quæ firmissimum
minimentum futura sit harum rerum inuestigatoribus.

Neomenia Nisan Exodi conueniebat cum neomenia Krionos.

Ita vere naturalis suit illa neomenia.

Præterea quadragesimus septimus
annus conuenit sabbatico Iudaico: 902 autem annus est tricesimus
Nabopollassari conueniens cum testimonio Ezekielis.

Deniq; anni 86 a septimo Cambysæ retro putati desinunt in anno
cædis Nechao Ægyptij, eodemque 139 Nabonassari: quod conuenit
eidem computationi.

Negari igitur non potest, hanc esse veram
Exodi epocham, quam \& verbum diuinum, \& vsus anni Sabbatici,
\& historiæ fides penes eximium scriptorem Chaldæum
Berosum, \& naturales neomeniæ vtriusq; sideris in vnum conuenientes
confirmant.

Quid postulamus præterea?

An vt tam certis,
tam egregiis, tam firmis argumentis somnia Corybantum anteponamus?

Quis vnquam ita hæc demonstrauit?

Quid demonstrauit?

%\end{parnumbers}
\clearpage
p. XV [pdf 42]
%\begin{parnumbers}
Quis aliter potest demonstrare?

Iam a conditu rerum, ad exodum,
anni sunt absoluti 2452 cum mensibus sex ab autumno, anni vero
absoluti 2453 a vere.

Sed ante Exodum initium anni putabatur ab
autumno, \& eodem initio in tempus veris translato, tekupha tamen,
hoc est, finis anni Solaris mansit in autumno, circa quam tekupham
Deus \textgreek{[Greek]} celebrari præcepit.

Igitur vbi initium anni
ab vltima antiquitate suit, inde \& rerum quoq; initium repetendum.

quod quidem a nobis factum, damnata priori sententia, quæ
initium rerum statuebat in vere.

Reliqua pete ex capite de conditu
rerum.

Præterea, quibus annus Lunaris in vsu est, illis commodius
initium, \& rationibus Tropicis conuenientius ab autumno, quam
a vere, vt Iudæis propter \textgreek{[Greek]}, \& Pascha.

Nam si annum
nostrum cælestem admitterent, \& hoc vnum cauerent, vt \textgreek{[Greek]}
citima sit in secunda Zygonos, semper citimum Pascha esset in neomenia
Krionos.

quia interuallum a neomenia Zygonos, ad neomeniam
Krionos, est semper 178 dierum, vno die plus, quam a scenopegia
ad Pascha.

Anni Sabbatici caussas iam reddidimus, \& verum
annum sabbaticum a Iudæis hactenus obseruari demonstrauimus,
initio hebdomadum sumpto, non vtique a defectu Mannæ,
quod fanatici quidam, \& veritatis hostes faciūt, sed a 48 anno Exodi,
ex capite \textsc{xiiii} Iosue, \& rationibus doctorum Habræorum, qui
dicunt septem annos \texthebrew{[Hebrew]}, id est, subiugationis terræ,
septem \texthebrew{[Hebrew]}
fuisse, id est, diuisionis.

quod rectissimum est: ideoq; hebdomadem
primam diuisionis, non subiugationis procedere in numerum.

An
potuit annus sabbaticus esse ante agrorum culturam?

Furor est aliter putare.

Tamen non desunt, non deerunt, qui solo contradicendi
studio, vt sapere videantur, aliter staduent: quibus per me non solum
hoc facere, sed etiam nos irridere licet; quandoquidem veritas apud
illos nullo in precio est.

Vnde nata sit diuersitas epochæ excidij Ilij,
cum alij 407 annis, alij 405, eum casum antiquiorem prima Olympiade
statuant, aperuimus ex doctrina anni Attici, cui acceptum
referimus quicquid eximium ex alta obliuione eruimus.

Veram sententiam
Eratosthenis esse deprehendimus, quæ illam cladem coniicit
in annum 407 ante caput primæ Olymiadis: eiusque veram
diem in anno Iuliano ostendimus.

Primam autem Olympiadem
ex doctrina itidem anni Græci \textsc{xxiii} die Iulij celebratam fuisse ante
nos aperuerat nemo.

Et tamen quidam Simioli tanquam rem
vulgatam in suis vanidicis Chronologiis retulerunt: cuius rei cognitionem
vnus Pindarus, quem illi neq; viderunt, neq; norunt, nos
docuit.


%\end{parnumbers}
\clearpage
p. XVI [pdf 43]
%\begin{parnumbers}
Quemadmodum autem Olympia, ita etiam Karnia plenilunio
celebrata fuisse, libro primo, capite de periodo Laconum
ostendimus. neque solum plenilunio, sed etiam eodem anno, quo
Olympia.

Itaq; Herodotus libro \textsc{viii} Olympia \& Karnia anno primo
Olympiadis 75 celebrata suisse scribit, pag. 307 editionis Henrici
Stephani nostri.

Cum multi eruditissimi viri, \& quidem in iis
Onufrius Panuinius Pater historiæ, multa accurate de Palilibus Vrbis
disseruerint, vt ei doctrinæ nihil ad perfectionem deesse videatur,
tamen \& plura deesse ex nostris disputationibus colligi potest.

Monere vero debent Annalium \& Fastorum scriptores, qui tempora
sua ad annos Vrbis dirigunt, vtra Palilia sequantur, Varroniana,
an Catoniana.

Nam certe Onufrius noster, tametsi Catonem sequitur,
tamen quibusdam imprudens ad Varronem transfugit.
% "transfugit" should not be rendered with a long s

Nisi
hæc distinctio adhibeatur, ridicula multa consequi necesse est.

Exemplum habemus in annis Christi per annos Vrbis eruendis,
quod hactenus ab omnibus factitatum.

Christus in annis Varronianus
vno anno maior est apud aliquem, quam in Catonianis apud alium.

Quare, vt dixi, ridicula sunt.

In sequentibus epochis quanuis
non ea occurrit obscuritas, quæ in prioribus: tamen semper aliquid
noue demonstratur, præter superiorum scriptorum consuetudinem:
in quibus sunt quædam de vero die \& anno natalis Alexandri, eiusque
obitus: de Encæniis Machabæi, de initio Simonis Iudæorum
Ethnarchæ, quem Iudæi Iohannem vocant, de æra Hispanica.

De quibus omnibus pluria noua disseruntur, quam trita \& vulgaria.

Iam
excessum Herodis ad suum verum annum ex Iosepho retulimus,
qui ad epocham Actiacam illud tempus diligenter exigit, \& præterea
notationem, cui contradici non possit, adducit, defectum Lunarem,
qui contigit \textsc{ix} Ianuarij, anno 45 Iuliano ineunte, in cuius
anni sequenti Decembri Dionysius Exiguus imperite statuit natalem
Christi, nouem solidis mensibus scilicet post excessum Herodis.

Itaq; diligentissimus \textgreek{[Greek]} omnium scriptorum Iosephus
recte ait decessisse \textsc{xxxv} anno labente regni eius a captis a Sofio[?]
Hirosolymis. in quo tamen interpretatio adhibenda.
% Sofio or Sosio

Nam reuera Herodes
obiit anno tricesimo sexto ex diebus æstiuis noni anni Iuliani.

Ergo tricesimus sextus annus Herodis iniuit ex diebus æstiuis anni
Iuliani \textsc{xliiii}.

Obiit autem initio Nisan.

Igitur sine dubio decessit
anno Iuliano \textsc{xlv}, qui erat tricesimus sextus iniens ex diebus æstiuis,
vt diximus.

Sed ex computatione ciuili Iudæorum, nondum
\textsc{xxxvi} annus iniuerat.

Iosephus enim, \& Iudæi eo sæculo putabant
omnia tempora a \textsc{xxiii} Ijar, vt albi ostendimus: cuius consuetudinis
ignoratio multos decepit.

Ab Ijar igitur Hyrcani, siue, vt Iudæi
vocant, Iohannis Hasmunai, tricesimus sextus annus Herodis inibat,
qui tamen iam nouem mēsibus ante ex consuetudine Romana iniuisset.

%\end{parnumbers}
\clearpage
p. XVII [pdf 44]
%\begin{parnumbers}

Itaq; eius decessus confirmatur primum accurata putatione
diligentissimi scriptoris, deinde notatione eclipsis, quæ omnem contradictionem
excludit.

At ex epilogismis Eusebij Herodes obierit
anno Iuliano \textsc{lii}, septem annis solidis post illum defectum.

qui stupor non meret castigationem, cum tanquam sorex indicio suo perierit.

Nam statim ab eius decessu tetrarchiam suam Archelaus eius filius
iniuit: quod quidem, si huic oraculo Eusebiano credimus, contigerit
anno Christi Dionysiano septimo labente.

Ergo Christus fuerit
annorum septem, cum ex Ægypto monitu Angeli reuoctus est.

Quod est ridiculum.

Rursus anno decimo regni, aut tetrarchiæ suæ
Archelaus ab Augusto relegatus est Viennam Allobrogum.

Secundum tempus ab Eusebio determinatum, hoc contigerit anno Iuliano
\textsc{lxi}, qui erat annus Tiberij tertius currens, biennio absoluto
post excessum Augusti.

Hoc modo anno tertio excessus sui Augustus
Archelaum relegauerit.

Vides \textgreek{[Greek]}.

Atqui innumeros videas,
quibus hoc somnium placet.

Nam sane omnes fere Chronologiæ
\& Annales hoc stigmate inusta sunt.

Atque vtinam in illis hominibus
non esset vir eximia doctrina præditus Dominus Cæsar Baronius,
Annalium Ecclesiasticorum scriptor, cuius operis copia nobis
facta est ab amicis, cum hæc \textgreek{[Greek]} scriberemus.

Is eruditissimus
vir ex hoc loco Eusebij Iosephum exagitat, tanquam imperitum
temporum: cum Eusebius potius ex Iosepho castigādus fuisset.

Nam absque Iosepho esset, quid certi de Herode haberemus?

Quis hæc tractauit, præter illum?

Qui fieri potuit, vt scriptor, cuius diligentia
\& fides in notatione temporum spectatissima, in iis peccauerit,
quæ sine illo Eusebius \& alij ignorassent?

Sed ipse doctus Annalium
conditor potest iam videre, vtri fides de hac re habenda, Iosepho,
cuius ratiocinia cum motibus cælestibus congruunt, an Eusebio,
cuius sententia \& historiæ, \& rationi aduersatur?

Sed de Iosepho
nos hoc audacter dicimus, non solum in rebus Iudaicis, sed etiam
in externis tutius illi credi, quam omnibus Græcis, \& Latinis.

Itaque
definat mirari doctus vir, cur tot eruditi, \& nos quoq; qui non in illis
eruditis, sed in huius scriptoris lectione peregrini non sumus, tantum
illi deseramus, cuius fides \& eruditio in omnibus elucet.

Cæterum de Eusebij anilibus hallucinationibus, præter hanc, quam
modo protulimus, satis libro sexto differuimus.

Sed ad Epochas
nostras venio: quarum omnium rationem reddere longum esset.

De Epocha Martyrum Diocletianea non possumus tacere, eam hactenus
etiam doctissimis imposuisse, quod eam ab initio Diocletiani
incipere omnes credunt.

Hinc prodigiosi errores, \& magna Consulum
confusio in Annales \& Fastos deriuata sunt, præsertim in annis.

%\end{parnumbers}
\clearpage
p. XVIII [pdf 45]
%\begin{parnumbers}
Nam initio Diocletiani perperam sumpto, perperam quoque
persecutionis Epocha initur.

Ea semper antiquitus a solis Ægyptiis
Christianis hactenus vsurpata fuit.

Itaque Historici \& Chronologi,
qui temporibus Caroli Magni dicunt cæptum putari ab annis
Christi, cum antea mos esset annis Diocletiani vti, errant.

Nam
nullis nationibus in vsu fuit.

Vnica autem Ecclesia duntaxat Alexandrina,
\& quæ illi subditæ sunt, hac Epocha vsa est semper, vtirurque
hactenus, \& vocatur ab Ægyptiis, qui Elkupt dicuntur,
\textarabic{[Arabic]} \textit{Æra Martyrum sanctorum.}

Nam
hallucinatus est ille, qui nuper \textarabic{[Arabic]}
\textit{Captiuitatem} vertit in literis
Alexandrinæ Ecclesiæ Romam missis, anno Martyrum 1310, qui
erat Christi 1593.

Epocha igitur Martyrum iniuit \textsc{xxix} Augusti,
id est, neomenia Thoth Actiaci, vel Mascarā Habesseni, anno Christi
Dionysiano 284.

Initium autem imperij Diocletiani a Palilibus
anni 287.

Differentia anni duo, menses octo.

Perturbatio, quæ est in
Consulibus a temporibus Maximinorum, vsq; ad filios Constantini,
ea vtique ab antiquo est.

Sed \& non minor confusio in annis persecutionis:
vbi magnæ sunt \textgreek{[Greek]} apud Eusebium: quanuis
recte sentit de initio Diocletiani, \& primo anno persecutionis.

Tamē
omnium Chronologorum fides hac in parte nutat.

Nam edictū
Diocletiani de tradendis codicibus prius est Ecclesiarum euersione,
euersio Ecclesiarum prior cæde Martyrum.

Felix Africanus Episcopus
\& socij eius supplicio in Campania affecti ideo, quod codices
Deificos, id est, sacram scripturam tradere noluissent.

Itaque in
Actis illorum scriptū fuit: \textit{Et ductus est ad paßionis locum, cum etiam
ipsa Luna in sanguinem conuersa est, die tertio Kalen. Sept.}

De Eclipsi
Lunari loqui manifestum est, cuius is color fuerit, quem sanguineum
astrologi vocant: cuiusmodi proculdubio accidit anno Christi
301, cyclo lunæ 17, annis quatuor solidis ante edictum de euertendis
Ecclesiis, idque \textsc{iii} Nonas Septembris, non autem \textsc{iii} Kal.
Sept. diebus quatuor post passionem Martyrum.

Itaq; perturbatus
est ordo verborum.

Legendum enim videtur: \textit{Et ductus est ad paßionis
locum, die tertio Kal. Sept. cum etiam ipsa Luna in sanguinem conuersa
est.}

id est, quo tempore Luna defecit, proximo nimirum nouilunio.

Nam cum constet passos \textsc{iii} Kal. Septembris, \& ita habeat
Kalendarium, non videtur esse error in notatione temporis.

At Dominus Baronius hæc gesta confert in annum 302, tribus annis ante
persecutionem: \& tamen putat eum esse secundum annum persecutionis,
qui erat decimus nonus Æræ Martyrum, decimus autem
septimus currens ab imperio Diocletiani.

%\end{parnumbers}
\clearpage
p. XIX [pdf 46]
%\begin{parnumbers}
Sed \textgreek{[Greek]} illorum
Annalium propagati sunt partim ex erroribus aliorum Chronologorum,
quos auctor sequitur, partim ex annis Christi male ad
suam \& veram epocham reductis.

Vnde factum, vt ap initio operis,
ad tempora Nicenæ synodi, ne vnus quidem annus Christi
veræ epochæ suæ redditus sit.

Itaque triennio aliquando, aliquando
quadriennio, vt plurimum autem biennio erratum est.

Exempli
gratia: Excidium Hierosolymorum contigit anno Christi
Dionysiano \textsc{lxx}, quo neomenia Nisan conueniebat cum neomenia
Xanthici, teste Iosepho.

In Annalibus refertur ad annum
72: qui est error Eusebij, sed alibi ab eodem castigatus.

Certum est, Fructuosum Episcopum, Christi Martyrem, cum fociis
passum anno antequam pax \& interspiratio data esset Ecclesiis
sub Marco Aurelio Antonino, \& L. Ælio Vero.

quod tempus Eusebius confert in annum quartum Olympiadis \textsc{ccxxxiiii},
id est Christi Dionysianum 160.

Ergo passus est Fructuosus anno Christi
159.

Hoc aliter demonstrabimus.

In Actis agonis Fructuosi \&
sociorum legitur: \textit{Producti sunt duodecimo Kalend. Februarii, feria
sexta.}

Ergo litera Dominicalis erat B.

Proinde hoc accidit anno
159, triennio citius, quam notatum in Annalibus.

In Actis Andreæ
militis \& sociorum scriptum extat, eos necatos fuisse decimoquarto
Kalendas Septembris, Dominico die, hora secunda.

Igitur litera Dominicalis erat G.

Hoc necessario contigit anno 305,
qui erat primus persecutionis a Pascha illius anni antecedente, post
euersas Ecclesias: quod quidem Pascha celebratum 25 Martij, ipso
die termini.

At in Annalibus hoc refertur in annum 301, quadriennio
ante rem gestam.

Rursus in Epistola Vigilij Episcopi Tridentini
de Passione Sanctorum Sisinnij, Martyrij, \& Alexandri,
ita legitur: \textit{Die paßionis Sanctorum, quarto Kalendas lunias, feria
sexta, nascente luce.}

Passi ergo sunt anno 403, cyclo Solis \textsc{xx}, quando
\textsc{xxix} Maij erat feria \textsc{vi}.

At in Annalibus dicitur scripta
anno 400 Christi.

Scripta ergo fuisset triennio ante cædem
ipsorum Martyrum.

Cui absurditati ipse non adscribet, certo scio.

In iisdem
Annalibus ex codice Antonij Augustini mentio fit Homiliæ
Cyrilli Episcopi dictæ in natiuitate Ioannis Baptistæ, Pharmuthi
vicesima octaua, indictione prima, sub Theodosio iuniore \& Valentiniano.

Ergo dicta fuit Homilia anno Christi 433, April. vicesima
tertia.

At in Annalibus refertur in annum 432, April. 29. S. Benedictus
Monachorum Occidentis Pater, obiit \textsc{xi} Kal. Aprilis, Sabbato
sancto, vt refert Aimoinus monachus ex Actis S. Mauri ipsius
Benedicti discipuli.

Toto illo sæculo hoc non potuit contingere, nisi
anno 536.

%\end{parnumbers}
\clearpage
p. XX [pdf 47]
%\begin{parnumbers}
Tamē in Annalibus Ecclesiasticis obitus Benedicti cōfertur
in annum 542, sex annis serius.

Multa igitur peccari necesse est
in Gestis Benedicti, quæ in illis Annalibus referuntur.

In Encyclica
epistola Vigilij Papæ scriptum fuit: \textit{Piißimus atque clementißimus
Imperator Dominico die, id est, Kalendis Februarij, gloriosos Iudices suos
ad nos destinare dignatus est.}

Anno 554 Kalendis Februarij fuit dies
Dominica.

At in Annalibus hoc confertur in annum 552, duobus
annis citius.

Anno 546 turbatio facta in Pascha, vt ex Cendreno docuimus,
capite de periodo Dionysiana, libro \textsc{iiii}.

In Annalibus referetur
sub anno 545.

Martinus Episcopus Turonensis obiit anno
395, vt accurate a nobis disputatum est.

Auctor Annalium Sigebertum
sequutus coniicit in annum 402.

Ex eo errore multū peccatum
est in temporibus Regum Francorum.

de quibus consulatur vltima
diatriba libri sexti huius operis nostri.

Non semel monuimus magnam
perturbationem esse in initiis Imperatorum, a Maximinis
ad Valentinianum.

Vt alios taceam, Constantini initium ab aliis in
305, ab aliis in 306 annum coniicitur.

At Constantinus iniuit imperium
post obitum patris sui Chlori.

Obiit autem Chlorus in Britannia
anno primo Olympiadis 271, vt inquit Socrates.

Nos ostendimus,
apud Socratem, Hieronymi Supplementū, Ausonium, \& alios,
semper Olympiadem sumi pro lustro Iuliano, non pro lustro Olympico
Elidensium, idq; lustrum Iulianum biennio posterius esse Elidensi,
cum incipiat ab anno Iuliano bisextili.

Itaq; is fuit annus bisextilis,
quo obiit Chlorus, \& imperium iniuit Constantinus.

Sed duæ
cautiones adhibendæ.

Prior est, vt scias annum Constantinopolitanum,
siue Nicenum hic intelligi, qui incipiebat a \textsc{xxiiii} Septembris.

Altera, vt prolepsis vsurpata intelligatur in anno mortis Chlori.

Nam obiit \textsc{xxv} Iulij, \textsc{lxi} diebus ante
\textsc{xxiiii} Septembris, \&
tamen obitus eius ad eundem annum refertur quo iniuit imperium
eius filius, \textgreek{[Greek]}, vt dixi.

Omnino igitur iniuit imperium anno
303, aut 307.

Nam primus annus Olympiadis Iulianæ incipit semper
diebus 153 ante bisextum.

Sed nemo concedet Chlorum obiisse
anno 303.

Obiit ergo 307.

Et proinde anno 307 iniuit imperium
Constantinus, ex ante diem \textsc{viii} Kal. Octobr. eiusdem anni 307.

In his prouocamur a docto Annalium scriptore, \& rem absurdissimam
prodidisse nos dicit, Constantini imperium iniisse ex anno
308, cum, vt inquit ipse, iniuerit anno primo Olympiadis 271,
Christi vero 306[?].
% 300 or 306 ?

Nos vero negamus vllam culpam aut absurditatem
in nobis admissam.

Nam annus Christi 308 Constantinopolitanus
incipit a Septembri anni 307, vt iam dictum est.

Et proinde ipsum, \& alios errare, qui annum Christi 306 a Kalendis Ianuarij
dicunt esse annum labentem Constantini.

%\end{parnumbers}
\clearpage
p. XXI [pdf 48]
%\begin{parnumbers}
Hoc enim volunt,
cum putant primum 271 Olympiadis Elidensis annum esse primum
Constantini.

Olympias enim illa Iphitea cæpit ex diebus æstiuis
anni 305, qui fuit annus primus presecutionis.

Quare in annis
Constantini, vt in aliis, insigniter peccatum est a viro docto.

His
postis, quinquennalia Constantini data sunt anno 312: vicennalia
autem anno 327.

Interuallum inter illas duas celebritates interiectum
haud dubie vocatur Indictio, iniens a datis quinquennalibus,
desinens[?] in vicennalibus, quibus concilium Nicenum dimissum.
% desinens or definens?

Sed neq; hoc placet Domino Baronio: neque caussam appellationis
Indictionum admittit.

At nos dicimus, non minus iniuste nos
hic, quam in initio imperij Constantiniani reprehendi.

An negat
Indictiones in quinquennia indici, \& in quinquennalibus Principum
panegyribus remitti?

Si non credit, legat \& quæ priore, \& quæ
hac editione ad eam rem collegimus.

Quinquennalia illa dicuntur
\textgreek{[Greek]}, hoc est ad verbum, sparsiones, largitiones, profusiones, in
quibus liberalitas Principis ad remissionem vsq; tributorum, \& indictionum,
editiones munerum \& spectaculorum, congiaria, \& donatiua
extendebatur.

Inde \textgreek{[Greek]} non solum pro illa largitione
sumitur, sed \& pro ipsa indictionis temporalis nota.

Nam quod Latini
dicunt, Indictione prima, secunda, tertia hoc factum est, Græci
dicunt, \textgreek{[Greek]}.

Non ergo nos, sed ipse fallitur.

Quid?

si initium Constantini a nobis ignoraretur, tamē quinquennalia
eius nos manu ad illud deducerent.

Itaque ignorari n n [?]
potest.
% Probable printing error. "n n" should read "non".

Neq; minus errat, cum cladem Maxentij coniicit in annum
312.

Quot modis enim hoc refelli potest?

Sed de eo suo loco.

Nam
Maxientius anno 313, non 312 extinctus est, vt recte Panuinius notat,
sed male inde Indictionum initia \& caussas repetit: quod a nobis
olim diligenter discussum fuit.

%%% === Sextus Liber
Sextus liber continet residuum Epocharū,
in quo nobiliores quæstiones de Natali die, \& Passione Christi,
de Hebdomadibus Danielis, quæ breuibus diatribis explicari
non possunt, presequimur.

Ne autem aut rudiores, aut refractarij auctoritate
veterum scriptorum nobis præscribere possent, pauca de
Eusebij erroribus in antecessum delibauimus, in quibus, præter frequentes
\textgreek{[Greek]}, puerile illud deliramentū de Effenis confutauimus,
quos Christianos fuisse hoc vnico argumento probat, quod
\textgreek{[Greek]} essent, \& solitarie viuerent, \& monasteria haberent:

quasi Bonzios
Iapanensiū Christianos esse censeamus, quia \& cœnobitæ sunt,
\& Psalmos quosdam instar monachorum Europæorum alternis modulantur,
\& horas Canonicales eorum exemplo habent.

Eorum Essenorum alij \textgreek{[Greek]}, alij \textgreek{[Greek]} fuerunt.

Sed horum non videtur
secta diuturna fuisse.

%\end{parnumbers}
\clearpage
p. XXII [pdf 49]
%\begin{parnumbers}
Ast \textgreek{[Greek]}, aut eorum non dissimilium
synagogæ fuerunt ad tempora Iustiniani.

Sunt enim ij, qui Cælicolæ
vocantur.

Nam \& nomen id indicat.

Cælicolæ enim sunt
Angeli.

Ita vocari volebant, propter sanctum, \& cæleste, vt ipsis videbatur,
vitæ institutum.

In perueteri Glossario Latinoarabico \textit{Cælicola}
[Greek][Arabic][?]. id est, Angelus.

Præterea quia erant \textgreek{[Greek]}, noui
baptismi auctores Donatistis fuerunt.

Princeps eorum vocatur
Maior, vt \& aliorum Iudæorum.

Hoc enim est \texthebrew{[Hebrew]}.

Philo dubitans
quare Esseni illi dicti sint \textgreek{[Greek]},
 vtrum quia medicinam profiterentur,

an quia Deum colerent, ex eo coniiciendum relinquit,
eos non dictos esse quasi \texthebrew{[Hebrew]} \textgreek{[Greek]},
 vt volebat quidam Lunaticus
literarum Hebraicarum professor, sed quia \textgreek{[Greek]} vocat, eo ostendit
\texthebrew{[Hebrew]} dictos, hoc est, \textgreek{[Greek]}.

Quod Christiani non essent, sed
mere Esseni, statim initio libri ostendit Philo.

sed \& Sabbati summus
cultus, \& reliqua, quæ a Philone de ipsis narrantur, satis leuitatis
damnant Eusebium, \& reliquos veteres, qui Eusebium sequuti,
idem hariolati sunt.

Sed in Annalium tomo primo tacite perstringitur
sententia nostra ab auctore, qui tamen fatetur veros Essenos Iudæos
fuisse.

Mirati sumus, quomodo ille putauit in vnum hæc bene
conuenire posse, Iudaismum \& Christianismum.

Vt hoc probet, ait
veteres patres idem scribere, quod Eusebium.

Atqui ex Eusebio
hoc desumpserunt, \& eius auctoritate contenti Philonem non consuluerunt.

quem si legissent, nunquam tam ridiculæ sententiæ assensum
accommodassent.

Hæc vero puerilia sunt.

Venio nunc ad natalem
Christi, quem vetustas Christianismi ad \textsc{xxviii} annum Actiacum
retulit, recte.

Nam Christus iniens annum vnum a tricesimo
ætatis suæ accessit ad baptismum, vt omnes vetustissimi Patres ex
Luca retulerunt, \& post eos eruditus Annalium scriptor.

Baptizatus est anno \textsc{xv} Tiberij, duobus Geminis \textsc{coss}. anno
Iuliano 74.

Ergo \textsc{xxv} Decembris anni 73 illi inibat annus primus a tricesimo.

Deductis 30 annis absolutis de 73, remanet annus Iulianus
43, in cuius \textsc{xxv} Decembris natus fuerit Dominus, cyclo Lunæ
\textsc{xviii}, anno Actiaco \textsc{xxviii},
 vt illi vetustissimi partes crediderunt,
duobus annis solidis ante epocham hodiernam Dionysianam,
anno solido cum diebus aliquot ante excessum Herodis.

Hoc proculdubio
verum est.

Sed in Annalibus peccatur ab auctore in anno
\textsc{xv} Tiberij.

Quem enim putat \textsc{xv}, is est \textsc{xvi}, \& magno errore illi
attribuit Consules duos Geminos, quibus Consulibus annus \textsc{xvi}
Tiberij iniit ex \textsc{xix} Augusti, cyclo Lunæ vndecimo, anno Iuliano
74.

Nisan igitur is, qui proxime sectus est baptismum Christi,
Consulibus duobus Geminis, antecessit annum \textsc{xvi} Tiberij ineuntem,
mensibus quinque.

%\end{parnumbers}
\clearpage
p. XXIII [pdf 50]
%\begin{parnumbers}
At scriptor Annalium putat duos Geminos
Consulatum gessisse cyclo Lunæ \textsc{xvi}: in quo ne sic quidem
sibi constat.

Nam is fuerit annus 75 Iulianus iniens.

Hoc modo Decembri anni 74 Christus iniuerit annum primum a tricesimo: \&
deductis 30 absolutis, remanebit annus 44 Iulianus, in quo natus
Christus fuerit, tribus circiter mensibus ante excessum Herodis, anno
solido ante epocham Dionysianam, qua hodie Ecclesia vtitur.

quæ sane multorum veterum, inque illis Eusebij fuit opinio.

Sed
Christus baptizatus anno 74 Iuliano: passus 78.

Differentia, anni
quatuor solidi, paschata quinque.

Quorum nullum vestigium in illis
Annalibus extat.

Quinetiam auctor, quando numerus annorum
non succedit ex voto, culpam in Iosephum reiicit, mendacem multis
modis arguens: inter alia, quod scripserit \textgreek{[Greek]} factam post
Archelai relegationem, cum, inquit, ea \textgreek{[Greek]} Christo nascente
contigerit, \& aperte Eusebius id indicauerit.

Nos hallucinationem
Eusebij loco suo confutauimus, in quo descriptionem patrimonij
Archelai cum descriptione totius orbis Romani confundit more
suo, neq; meminit verbis illis, \textgreek{[Greek]}, designari non
vnicam fuisse illam descriptionem, cum \textgreek{[Greek]} mentio fiat.
% Final period not visible in original.

Quare
idem Euangelistes quemadmodum prioris meminit in Euangelio,
ita alterius mentionē facit in Actis.

vt nō sit audiendus doctus Annalium
scriptor, qui non solum hac in parte Eusebij auctoritatem
Iosepho opponit, sed etiam adiicit descriptionem illam[?] eandem esse,
de qua Æthicus statim initio libri sui loquitur: cum tamen neque
tempus, neque res conueniat[?] descriptioni nascente Christo factæ.

Nam descriptio, de qua intelligit Æthicus, cæpit ab anno cædis
Cæsaris, desiuit[?] in anno \textsc{xxxiii},
 qui erat tricesimus quartus a primis
Kalendis Ianuariis Iulianis, decem annis absolutis ante verum
natalem Christi, duodecim ante epocham Christi hodiernam Dionysianam.

Res autem eadem non est, imo longe diuersa: atq; adeo
tantem differt[?] descriptio, de qua Æthicus loquitur, a descriptione,
quæ facta Christo nascente, quantum decempeda, \& tabulæ [censuales][?].

Nam illa descriptio Æthici mandata est agrimensoribus, \&
Geometris, hæc Rationalibus.

Illa orbis mensura, \textgreek{[Greek]},
hac census \& facultates in Tabulas relatæ.

Sed neq; recte concludit,
Iosephum hallucinatum, quod paulo ante initia belli Iudaici
auditam ex adytis templi vocem scripserit, quæ diceret \textsc{hinc
migremvs}: cum, inquit, Eusebius id in passionis Dominicæ tempus
referat.

Quomodo Eusebius melius scire potuit ea, quæ contigerunt
Christi \& belli Iudaici tempore, quam Iosephu? aut vnde,
quam ex Iosepho? de illis dico, quæ non pertinent ad historiam euangelicam.



%\end{parnumbers}
\clearpage
p. XXIV [pdf 51]
%\begin{parnumbers}
Sed tam friuolum argumentum eluditur iis, quæ aduersus
hanc Eusebij hallucinationem libro sexto decimus.

Denique iniuste
vbique Iosephum reprehendit, omnium scriptorum veracissimum
\& religiosissimum, quod quidem ipsius scripta loquuntur.

quem
auctorem si non tam contempsisset, nunquā eos
 \textgreek{ανἀχρονισμους [Greek:anachronism]} commisisset,
quibus totus contextus temporum primi tomi perturbatus
est.

Sed antequam ex hac velitatione facessimus, qua \& nos \& cognominem
nostrum scriptorem ab animaduersione docti viri vindicamus,
nos homines Aquitani expostulamus cum eo, quod a nobis
tres summos viros abdixit, Paulinum, Phœbadium, \& Sulpitium
Seuerum:

qui cum suerint natione, \& domo Aquitani, tamen
Paulinum \& Sulpitium Romæ natos scribit, Phœbadium in Hispania.

Quis illum docuit Paulinum non esse natum Burdigalæ, vbi
antiquitus Paulina gens, hodieque quædam regio vrbis Burdigalensis
Paulino cognominis est?

Phœbadium autem Aginni Nitiobrigum
Episcopum quare in Hispania natum dicit, aut quo auctore?

Apud Hieronymum male excusum est Sœbadius, qui error irrepsit
ex Sophronio, vbi legitur \textgreek{[Greek]}.

Sed liber manu scriptus
Sanctæ Mariæ de Granateria liquido habet Febadium.

Apud Sulpitium
Seuerum deprauatum quoq; est, vbi legitur Fegadius, pro
Febadius, vt quidem librarij scribunt.

nam orthographia est \textgreek{[Greek]},
Phœbadius: satis hodie notus erudita sua in Arrianos Epistola,
quæ ante \textsc{xxv} annos primum edita.

Mei municipes Fiarium vocant,
cuius memoriam bis quotannis instaurant, ineunte ieiunio
quadragesimæ, \& die Marci Euangelistæ, mense Aprili, si bene
memini.

Huic successit Gauidius in episcopatu.

Sulpitium Seuerum
nemo hactenus Aquitanum fuisse dubitauit: sed patria ignoratur,
cum tamen ipse Nitiobrigem sese manifesto prodat, cum Seruationem
Tungrorum, Phœbadium autem suum Episcopum fuisse scribit.

Phœbadius autem erat Nitiobrigum Episcopus.

Iste Sulpitius
Ecclasiasticorum purissimus scriptor, post transitum Martini recepit
sese Elusonem, quo tempore ad eum scribebat Paulinus.

Id oppidum est cum arce veteri in finibus Nitiobrigum, qua amni Draguto
a Petrocoriis diuiduntur.

Vulgo \textit{Lausun}.

Sed de hoc satis.

Mei
Nitiobriges pro Sulpitio Supplicium dicunt, quomodo \& Bituriges
suum illum vocant, quem eundem cum hoc faciunt perperam,
cum inter transitum Martini, cuius noster Sulpitius discipulus fuit,
\& ordinationem Sulpitij Episcopi Bituricensis sub Guntchramno
Rege, intercedant plus minus anni 190.

Non iniuriam facimus
docto viro, si cum bona eius venia doctissimos viros Aquitanos,
\& Christianissimos originibus suis vindicamus.

%\end{parnumbers}
\clearpage
p. XXV [pdf 52]
%\begin{parnumbers}
Sed quemadmodum tribus viris Aquitaniam orbauerat, ita eandem duabus
alienis ciuitatibus donauit, Reiensi, \& Vasensi.

Prosperum non vno
loco dicit Regiensium in Aquitania fuisse Episcopum, cum dicendum
fuerit, Prosperum Aquitanum fuisse Episcopum Reiensium,
aut Regiensium in secunda prouincia Narbonensi.

Hodie \textit{Ries} vocatur.

Nugantur qui eum Regij Lepidi Episcopum \& scripserunt,
\& in fronte eius sacrorum poematum apponi curarunt: quasi Reienses,
in secunda prouincia Narbonensi, iidem sint cum Regio Lepidi
in Æmilia.

Vasense autem consilium idiotismus illorum temporum
vocauit, quod potius Vasionense dicendum erat.

Vasio Vocontiorum hodie \textit{Vaison} dicitur.

Est Episcopatus Auenioni metropoli
attributus.

Imperite quidam cum foro Vocontiorum confundunt.

Itaque Vasense, vel Vasionense, in Vasatense mutandum non
erat.

quemadmodum in anno Christi 552 perperam Firminum
Vticensem mutat in Venciensem.

Vticenses, vulgo dicuntur \textit{Vsetz}.

Est Episcopatus in prima Narbonensi.

Dicuntur etiam Vcetenses,
\& Vcetiæ Episcopus.

Apud Gregorium Turonensem libro \textsc{vi},
mentio est Ferreoli Episcopi Vcetensis: vbi vulgo male Vcecensis.

Sed tam imperite vulgus Vticenses deprauauit in Vcetenses, quam
Arausio in Aurasio: Vasensis dixit, pro Vasionensis.

At ciuitas siue
Episcopatus Venciensis, est in secunda Narbonensi. Vulgo S. Paulus
de Venciis.

Scribendum vero per t.[?] Ventiensis, \textgreek{[Greek]} enim dicitur
Ptolemæo.

Fuitque Nerusiorum in Alpibus Graiis Metropolis.

Sequuntur in sexto libro illa quinque Paschata a baptismo
ad resurrectionem, fuis temporibus, Consulibus, \& cyclis notata.

In tertio Paschate quid fit \textgreek{[Greek]}, explicamus,
quæ verissima interpretatio adhuc assensum vel mereri, vel
exprimere a doctis hominibus non potuit: quod valde miror,
cum absurdissima sit ea, quam sequuntur ipsi.

Omnes igitur vno
ore putant \textgreek{[Greek]}, pro \textgreek{[Greek]} dictum esse.

Id ad verbum
Hebraice esset \texthebrew{[Hebrew]}:
 aliter \textgreek{[Greek]}, Latine Præposterum.

quo nihil præposterius dici potuit.

Nam quid est præposterum Sabbatum?

Non pudet iocularis interpretationis?

Sed ita est.

Alius fortasse assensum extorsisset.

Sed quia a nobis, ideo
minus acceptum.

Theophylactus post Epiphanium, \& alios veteres,
interpretatur \textgreek{[Greek]}.

Itaque verum est, quod diximus, omnes tam veteres, quam
recentiores \textgreek{[Greek]} interpretari
 \textgreek{[Greek]}, id est \textgreek{[Greek]},
præposterum.

Vt illud probet, idem Theophylactus
subiicit: \textgreek{[Greek]}.

%\end{parnumbers}
\clearpage
p. XXVI [pdf 53]
%\begin{parnumbers}
quod falsum est,
propter translationes, quas imperiti negant sæculo Christi vsurpatas,
cum tamen longe ante Christum in vsu fuisse demonstrauerimus,
vt locus non sit pertinaciæ.

[Prolegomena continues up to page LII]

\end{parnumbers}
