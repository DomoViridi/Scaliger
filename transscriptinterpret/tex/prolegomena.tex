% !TEX TS-program = xelatex
% !TEX encoding = UTF-8 Unicode
% this template is specifically designed to be typeset with XeLaTeX;
% it will not work with other engines, such as pdfLaTeX

%%% Count out columns for fixed-width source font
% 000000011111111112222222222333333333344444444445555555555666666666677777777778
% 345678901234567890123456789012345678901234567890123456789012345678901234567890

\chapter{Prolegomena}
\sc{Prolegomena in Libros de Emendatione Temporum}

\em{Ad candidum Lectorem.}

\normalfont{}

\setcounter{parcount}{0}
\begin{parnumbers}
Quintusdecimus hic annus agitur, candide
Lector, postquam opus nostrum de
Emendatione Temporum emisimus.
\\ \p
Persuaseram
mihi, homines studiosos aliquam nobis
gratiam habituros tot rerum, quas et scitu
dignas, et a nobis primum indicatas negare
non poterant.
\\ \p
Sed longe aliter animatos experti
sumus: atque adeo rem potius invidiosam
atque obtrectationi opportunam, quam illis gratam me suscepisse
intellexi.

Denique nihil aliud quam significarunt, quiduis potius
se ignorare malle, quam a nobis aliquid discere.

In quibusdam
candorem, in aliis studium, in omnibus sensum bonarum rerum desideraui.

Nos vero, qui nihil unquam prius habuimus, quam ut horum
orationes sinamus praeterfluere, modo verum eruere, et inimicos
nostros etiam inuitos iuvare possimus, opus nostrum iterum in
manus sumptum auximus, illustravimus, emendauimus, ut, quanuis
idem sit, aliud tamen a nova cultura videri possit.

Quae huic editioni
accesserunt, haud promptum est dicere.

Sed in quibus a priore demutat,
postea intelliges, siquidem instituti nostri rationem aperuero.

Subiectum operis nostri est ratio Temporum civilium, et eorum,
quae in vetustatis cognitione versantur: finis, Emendatio: quod quidem
me tacente, et Titulus ipse promittit.

Civilium temporum cognitio,
eorumque historia, vertitur in multiplici diversorum annorum
forma et eorum methodis vulgaribus, quos Computos posterior
aetas vocauit.

\textgreek{Τα ιςορομγυα[?]} civilium temporum habes in primoribus
tribus libris, et maiore parte quarti: methodum autem in septimo.

A emendationis duae partes sunt.

%\end{parnumbers}
\clearpage
p. II [pdf 29]
%\begin{parnumbers}
Prior versatur circa epocharum
investigationem, posterior circa verum annum tropicum, 
et periodos Lunares: quam materiam posterior pars libri quarti,
item toti quintus et sextus sibi vindicant.

Iam quemadmodum Epochae
sunt notationes, et tituli temporum, ita ipsarum epocharum
quaedam debent esse propria \textgreek{γνωρίσματα} et characteres: quorum
characterum alii sunt naturales, alii civiles. 

Naturales quidem a rationibus
utriusque sideris, unde nati cycli Solaris, et Lunaris: civiles
ab instituto, cuiusmodi indictiones et anni Sabbatici: sine quibus in
harum rerum tractatione omnis conatus irritus. 

Rursus et eorum
quoque fallax usus est, nisi quaedam annorum ex illis periodus instituatur.

Sed eae sunt totidem, quot aut formae annorum, aut civilia
initia.

Nam in anno Aegyptiaco Nabonassari alia opus est, ac in anno
Solari, quia diversa forma: item in anno Actiaco sive Diocletianeo
alia, ac in Iuliano, propter diversa initia.

In anno Aegyptiaco vago
naturales characteres sunt \textgreek{εἰκοσιπεν σαετηρις[?]} Lunaris, et
\textgreek{έπταετηρις[?]} Solaris:
civilis autem character est quadriennium, quem canicularem
annum minorem vocabant Aegyptii.

Hi tres characteres in se ducti
producunt periodum magnam annorum 700 Aegyptiacorum: qua
uti debet disputator temporum, siquidem rationes suas ad annos
Nabonassari, Armeniorum, aut Persarum exigit.

At qui anno Iuliano,
quae omnium formarum temporibus est convenientissima, uti
volet, is cyclo utriusque sideris quindecies ducto componet elegantissimam
periodum annorum 7980, cuius initium in cyclo Solari,
et Indictione Romana, a Kal. Ianuarii, in cyclo Lunari a Martio, in
anno Sabbatico ab autumno.

Itaque non minus utilis, quam necessaria
est.

Sine ea nihil agit Chronologus: cum ea tempori, et saeculis
imperat.

Quam enim lubricum sit retro ab aliqua epocha notare tempora,
quod maior pars doctorum virorum facit, satis nos usus docuit.

His ita positis, ad singula huius operis membra venio.

%% == Libro primo
Libro primo
praeter divisionem temporum, et iucundissimam mensium, et
annorum historiam, de antiquissima anni forma disputatur, quae in
menses aequabiles annum describit, qua pleraque omnes Graecia usa
est, et ab ea omnis ratio Olympiadum pendet: nisi potius eam e ratione
Olympiadum propagatam dicas: quod sine cognitione Olympiadum
numquam tam eximium vetustatis et temporum monimentum
in lucem eruissemus. 

Ex tanta autem Graecorum scriptorum
copia unicus Pindarus nobis facem alluxit, qui solus nos docuit tempus
ludicri Olympici.

Aliter, quae paucitas est bonorum scriptorum,
nulla erat via ad haec interiora perveniendi.

Huius anni Graeci
formae doctrina tanto acceptior esse debet, quanto obscurior eius
rei apud maiores nostros scientia fuit: cum ante hos mille quadringentos
plus minus annos eius rei neque volam, neque vestigium
vetustas retinuerit.

%\end{parnumbers}
\clearpage
p. III [pdf 30]
%\begin{parnumbers}
Nam falso veteres multi, ac post eos infamae antiquitatis
scriptores, Macrobius ac Solinus, atque proavorum memoria
summus vir Theodorus Gaza, annum Graecorum statim
ab initio merum Lunarem fuisse prodiderunt.

Quamuis enim in
Panegyribus suis, ac nobilioribus sacris, quae certo annorum circuitu
redibant, unius Lunae rationem habebant, tamen, ut uno verbo
dicam, eorum anni forma Lunaris non erat.

Olympicum enim ludicrum
ipsa Lunae plena lampade celebrabatur, ut solus veterum nos
docet Pindarus. 

Praetera Laconibus ante plenilunium, aut novilunium
aliquid incipere religio erat.

Unde \textgreek{Λαχώνιχας ςελωοαζ}[?] vicinorum
proverbio iactatas, et contra Arcadibus proverbiali convicio
neglectum religionis obiectum legimus. 

Quod enim ante novilunium,
aut plenilunium ut plurimim bella aut alia seriora aggrederentur,
ob eam rem a finitimis nationibus \textgreek{[Greek]} vocabantur:
quae convicii caussa ab ipsis Arcadibus interpretatione elusa est,
probrum in laudem conversum ad vetustatem originis suae referentibus,
et antiquiorem sidere gentem suam gloriantibus. 

Quod igitur
novilunii ac plenilunii tempora Panegyricis ludicris deligebant,
propterea sacra trieterica instituta: cuiusmodi erant orgia Bacchi,
Nemea, Isthmia, alia.

Ea enim est anni Graecanici forma, ut si, verbi
gratia, novilunium in neomeniam Gamelionis incurrat, plenilunium
in eandem neomeniam incidat anno tertio redeunte.

Itaque
cum in Tetraeteride orgia Bacchi trieterica celebrabantur, tertio
anno redibant in eum sistum Lunae, qui priorum orgiorum situi oppositus
erat.

Quare elegantissime Statius trieterida vocat alternam:
quia alternis in novilunium, et plenilunium incurreret.

At sacra,
quae necessario eodem Lunae tempore obibantur, ea semper erant
tetraeterica: ut in Attica Panathenaica maiuscula, in Elide Olympias,
ut iam tetigimus, plenilunio.

Quod sane fieri non poterat, nisi absoluta
Tetraeteride, et Pentaeteride ineunte.

Atque ita Tetraeterides
in idem \textgreek{χῆμα} Lunae, non utique in idem tempus Solis redibant.

Ut
enim in orbem Solis et Lunae redirent, non aliter putabant fieri,
quam octaeteride confecta, eneaeteride ineunte.

Ex quo quaedam
eneaeterica sacra eo nomine instituta: cuiusmodi ab initio Pythia
fuerunt: et quidem merito.

Apollini enim, quem eundem cum Sole
faciebant, erant attributa.

Hinc colligimus, non solum Olymiadis
intervallum annis quatuor solidis explicatum fuisse; sed etiam puerliter
peccare eos, qui annorum quinque solidorum fuisse putant.

Neque vero quibusdam recentioribus succensendum, qui ita censent,
ita scribunt, sed et Ausonium nostratem culpa liberat Ovidius, scriptor
longe antiquior, et nobilior, qui aetatem suam quinquaginta annorum
decem Olympiadibus definit: quo magis mirum Pausaniam
hominem Graecum in ea haeresi fuisse, ut suo loco a nobis relatum est.

%\end{parnumbers}
\clearpage
p. IV [pdf 31]
%\begin{parnumbers}
Nam minus mirandum de Solino, qui cap. \textsc{xiii} Isthmia vocat
quinquennalia, quae erant tantum triennalia, quod certamen a Cypselo
tyranno intermissum, anno primo Olympiadis 49 instauratum
fuisse dicit.

Horum igitur omnium caussae ad typum anni Graeci referendae
sunt.

In quo argumento nihil eorum praetermisimus, quae
ei rei illustrandae faciebant, quanquam pene omnibus praesidiis
destituti.

Et quidem primum in genere, quod semper solemus, deinde
privatim multarum Graeciae nationum periodos proposuimus,
quae quidem non anni forma, sed situ et capite inter se differunt: in
qua tractatione diu nobis res fuit cum praestantissimo viro Theodoro
Gaza, vel potius cum eius sequacibus, a quibus extorqueri non
potest doctrina et situs mensium, ab illo primum proditus. 

Quae quidem
velitatio nobilioribus ingeniis, et ab omni invidia remotis, ut
spero, iucunda erit.

Quid enim est toto libro primo, cuius vel minima
pars, non dicam istis querulis, qui nihil sciunt, sed etiam doctioribus,
hoc saeculo, et ante multa retro saecula oboluerit?

Quid dicam \textgreek{[Greek]}?

Quis illarum caussas, et usum sciebat?

Quis
locum nobilem de illis in Verrina Ciceronis intelligebat?

Quis locum
\textgreek{Εξαιρέσεως[?]} in secunda Boedromionis?

Quis Posideonem intercalarem
mensem fuisse?

Huic materiae accessit \textgreek{ἐποχη χέντςα[?] θερινᾶ}
in \textsc{viii} Iulii, quae in priori editione omissa erat.

Id erat \textgreek{χάντσον[?]}
populare, quod nomine \textgreek{τςοπων[?] θερινῶν[?]} Aristoteles, Theophrastus,
Plutarchus, et omnes veteres intelligunt, non autem ipsum verum
Solstitium: quae rei pulcherrimae notatio nobis viam ad illustriora
praeiuit.

Quod Solstitiorum, et Aequinoctiorum puncta \textgreek{χέντρα} vocentur,
satis sciunt, qui veterum Graecorum libros legerunt.

Columella
cardines vocavit.

In praestantissimo Parapegmate, quod falso
Ptolemaeo attribuitor (est enim antiquius Ptolemaeo) ad \textsc{viii} Kal.
Iulii (quod est Solstitium Sosigenis) annotatum est: \textit{Aestivus cardo,
et momentanea aeris perturbatio}.

In Graeco (utinam haberemus!)
sine dubio fuit: \textgreek{Θερινὸν χέντρον, χαὶ ςιγμιαία αἔσος Ιασοιχή[?]}.

Igitur \textgreek{χέντρον
θερινὸν} nihil aliud, quam \textgreek{τροπαὶ θεριναί}.

Cur \textsc{viii} dies Iulii erat
epocha aestiva in usu civilis anni, non semel caussam reddidimus. 

Adiecta etiam pernecessaria neomeniarum Atticarum Tabula: quae
non solum priori editioni, sed etiam doctrinae anni Attici deerat.

Liber secundus anno Lunari dicatus est ideo, quia is annus ex illo
Graeco aequabili manasse videtur.

Ibi aperitur omnis antiquitas \textgreek{ἔτοις[?]
πρυτανείας[?]}, Octaeteridum Cleostrati, Harpali, et Eudoxi: quae omnia
hodie nomine tenus nota erant.

Eudoxea Octaeteris numquam
in usus civiles admissa est.

Anni vero \textgreek{πρυτανείας[?]} in vetustissimis Psephismasin
Atheniensium primo quidem ex Cleostratea, deinde, illa
abrogata, ex Harpalea petiti sunt.

%\end{parnumbers}
\clearpage
p. V [pdf 32]
%\begin{parnumbers}
Sequitur magnus annus Metonicus
ambarum, et Calippicus Metonici castigator.

Et quidem hi
ambo nomine noti tantum: caussarum autem, et omnium, quae ad
illa pertinent, mira ignoratio hactenus fuit.

Accesserunt huic editioni
Tabulae operosissimae dispensationum neomeniarum Metonicarum,
et Calippicarum: cuiusmodi etiam in Harpalea Octaeteride
exhibuimus. 

Quod de Eudoxea Octaeteride diximus, idem de
periodo Chaldaeorum dicendum, eam nunquam ad civilia tempora,
sed ad Genethliacorum themata usurpatam fuisse.

Id quod tum
multa argumenta, tum unicum certissimum illud est, quod eorum
menses appellationibus Macedonicis, non vero Chaldaicis fuerunt.

Propterea recte cum illius anni diatriba doctrinam dodecaeteridis
Chaldaicae Genethliacorum coniunximus, cuius nomen quidem
solum notum erat ex Censorino: cognitio autem nobis ex Arabum,
et Orentalium usu repetenda fuit.

An aliquis Graecorum \textgreek{δωδεχαετρίδος Χαλδαιχῆς}
meminerit, haud promptum est dicere.

Unum tantum Orpheum sive Onomacritum eius meminisse scimus. 

\textgreek{ὀρφδὶςὀν[?] ταῖς[?] δωδεχαετνρίσιν[?]:}

\begin{greek}
ἔςαι δ᾽ αὖθις ἀνὴρ, ἢ χοίραν[G-circ], ἠὲ τύρανν[G-circ],

ἢ βαοιλδὶς, ὂς τῆμ[G-circ] ἐς οὐρανὸν ἴξε[right curl] αἰπιιύ [all doubtful].
\end{greek}

Est apotelesma cuiusdam Genethliaci consulti super alicuius genesi,
de quo ipse respondit, eum fore magnum regem aut Dynastam, etcetera.

Citat Tzetzes. 

Haec multum illustrant doctrinam Dodecaeteridos
genethliacae parum antehac notae.

Itaque quemadmodum \textgreek{τελετὰς}
ita etiam \textgreek{δωδεκαετνρίδας[?]} scripserat Onomacritus sub nomine
Orphei.

Qualis fuerit Iudaeorum annus sub Seleucidis, quibus parebant,
multis exemplis testatum reliquimus: in quibus etiam translationis
feriarum in capite anni antiquitatem asseruimus adversus homines
nostrorum temporum, qui nugantur commentum nuperum
Iudaeorum esse.

In illis Doctor Theologus ingenti commentario
suo in Evangelium secundum Iohannem exultabundus ait illam
translationem confutari ex loco Iosephi, in quo scribit, quo anno
Hyrcanus foedus icit cum Antiocho Sidete, Pentecosten fuisse feria
prima.

Hunc locum Iosephi nos olim in priore editione produximus,
unde is, aut qui illi indicavit, accepit.

En, inquit, duo Sabbata continua.

Si propter continuationem duorum Sabbatorum, feria transfertur,
ergo ubi sunt duo continua Sabbata, non transfertur. 

In quibus aperte ostendit se ignorare caussam feriae transferendae, quae fiebat
propter solum Tisri, non autem propter alios menses; propterea
quod ille mesis multa solennia habet, adeo ut si non habeatur
ratio translationis, aliquando non solum duo, sed entiam tria continua
sabbata concurrere necesse sit.

%\end{parnumbers}
\clearpage
p. VI [pdf 33]
%\begin{parnumbers}
Si enim feria sexta inciperet
neomenia Tisri, omnino tria sabbata continuarentur, neomenia,
sive clangor tubae, sabbatum ordinarium, et ieiunium Godoliae.

Continuantur autem saepernumero in aliquo reliquorum mensium
duo Sabbata: idque fit, quando solenne est aut feria prima, aut feria
sexta.

Quorum alterutrum quotannis incidere, nisi quando Tisri
incipit feria tertia, Doctor ignoravit.

In primam feriam incidunt
haec solennia, \textsc{xxv} Casleu, et \textsc{x} Tebeth in anno defectivo tam
communi, quam embolimaeao, quotiescunque Tisri incipit feria secunda:
\textsc{vi} Sivvan; quando Nisan incipit feria septima:
\textsc{xv} Nisan, \textsc{xvii} Tamuz,
\textsc{ix} Ab, quando Nisan incipit feria prima.

In feriam autem sextam
convenit solenne \textsc{xxv} Casleu et
 \textsc{x} Tebeth, quando Tisri est feria
septima in anno tam communi, quam embolimaeo.

\textsc{xiiii} Adar, quando
Nisan sequens est feria prima: \textsc{vi} Sivvan, quando Nisan feria quinta.

Vides, quot Sabbata quotannis, nisi quando Tisri incipit feria tertia, Iudaei
continuent in aliquo mensium, praeterquam in solo Tisri, cuius
unius gratia illa cautio instituta
% No period at end of sentence

Itaque doctor tam frustra, quam ridicule
Iosephi testimonium adduxit de sexta Sivvan, id est, Pentecoste
feria prima; cum illo anno neomenia Nisan fuerit Sabbatum.

Atqui
nihil superesse putavit, quam ut Vaticani montis imago redderet
\textgreek{ἰὴ παιαή[?]}.

Sed ipse valde ignarus est harum rerum, ut reliqui omnes,
qui contendunt novitium esse Iudaeorum commentum.

Nos
validissime demonstravimus, et saeculo Christi, et retro sub Seleucidis
translationes in usu fuisse.

Et sane res peruetusta est.

Quae tamen
non minus ignorata, quam periodus Calippica, qua Seleucidae, et
Seleucidarum edicto Iudaei usi.

Quod non solum ex Nisan anni excidii
Hierosolymorum a nobis demonstratum est, sed etiam patet
ex definitione Rabbi Adda.

Is annum definit dierum \textsc{ccclxv},
horarum 5, 997/1080. 48/76.

Quid hac definitione aliud vult, quam periodum
Iudaicam fuisse annorum 76?

Cum Meto definit annum dierum
365. hor. 5. 1/19. ex eo coniiciendum relinquit, se uti periodo annorum
19.

Utebantur igitur periodo 76 annorum, id est, Calippica:
et tamen in omnibus neomeniis Lunae \textgreek{φάσιν} observabant, non
quod eam ex praescripto periodi non indicerent, sed ideo, ut eam
sanctificarent.

Nam et hodie quoque observant \textgreek{φάσιν}, non ut ex ea
neomeniam indicant, sed ut eam sanctificent.

Itaque Luna statim
visa dicunt: \texthebrew{[Hebrew]}.

\textgreek{ἀγαθὸν τέρας ἔςω ἡμῖν ης[??] παντὶ Ισραήλ.}

Idem faciunt et Muhammedani, quamuis neomenias ex
scripto indicere soleant.

Neque aliud intellexit fabulosus quidem,
sed tatem vetus auctor \textgreek{[Greek]} apud Clementem:
\textgreek{[Greek][Lots of Greek]}.

%\end{parnumbers}
\clearpage
p. VII [pdf 34]
%\begin{parnumbers}
Praeclara quidem ista: sed nescit, quid dicit.

Nam in Iudaeorum potestate
nunquam fuit, ut exspectarent \textgreek{φάσιν}:
% Greek: phase
 quia raro Luna se ostendit,
nisi secundo post coitum die.

Quod si expectandum ipsis esset,
res ridicula accideret, ut Elul, qui semper est cavus mensis, non solum
plenus, sed etiam aliqando unius et triginta dierum esset.

Sine dubio translationem feriae intelligit, cuius caussam ignorat.

\textgreek{πρῶτον σάββατον} vocat \texthebrew{רֹאשׁ הַשָּׁנָה‎} (Rosh Hashanah) caput anni.

Nam Sabbatum vocat, quia Festus
dies, \textgreek{κὶα᾽εργός}[?].

Ita etiam vocatur Levitici \textsc{xxiii}, 24.
% Leviticus 23:24: “Tell the people of Isra’el, ‘In the seventh month, the first of the month is to be for you a day of complete rest for remembering, a holy convocation announced with blasts on the shofar.'"
% λάλησον [Speak] τοις [to the] υιοίς [sons] Ισραήλ [of Israel,] λέγων [saying!]
% του [The] μηνός [(²month] του [] εβδόμου [¹seventh),] μία [day one]
% του [of the] μηνός [month] έσται [will be] υμίν [to you] ανάπαυσις [a rest,]
% μνημόσυνον [a memorial] σαλπίγγων [of trumpets,] κλητή [(²convocation]
% αγία [¹a holy)] τω [to the] κύριος [LORD.]

\textgreek{ἑορτὴν}[?]
% feast
 intellige
\textgreek{κατ᾽ ἐξοχήν τὴν πεντηκοστήν}:
% eminently the Pentekost [?]
 quod ita Hebraice vocetur, nempe \texthebrew{עֲצֶרֶת}[?] ([sh'miní] 'atséret).
% "The eighth [day] of assembly".
 
Vide in Computo Iudaico.

At \textgreek{μεγάλην ἡμέραν}
% great day
 vocat \textgreek{τὴν σκηνοπηγιαν, κατ᾽ ἐξοχήν}
 quoque, id est \texthebrew{הַנ}[?].

Nam aliae erant \textgreek{μεγάλαι ἡμέραι},
% Great Days
proinde ut et \texthebrew{חַנִּים}[?].

Sic Tertullianus magnos dies dixit, quos
Hebraei \texthebrew{[Hebrew]} vel \texthebrew{[Hebrew]}.

Eius verba sunt ex v in Marcionem:
\textit{Dies observatis, et menses, et tempora, et annos, et Sabbata, ut opinor,
et cenas puras, et ieiunia, et dies magnos.}
% Tertullianus: De Adversus Marcionem, Book 5, chapter 4, section 6.

Sed quid Tertullianum
advoco?

Ecce Biblia Graeca ita vertunt ex primo caput Isaiae:
\textgreek{τὰς νουμἠνίας ὑμῶν, καὶ τὰ σάββατα,
 καὶ ἡμέραν μεγάλην οὐκ ἀνέχομαι}[?].
% Isaiah 1:13 ?: I cannot bear your new moons, and your sabbaths,
 and the great day;

Quod Hebraice est \texthebrew{[Hebrew]}, vertunt \textgreek{[Greek]}, quod idem
est quod \texthebrew{[Hebrew]}: et quidem manifesto Sabbata distinguuntur a
magnis diebus. 

Quare perperam quidam \textgreek{[Greek]} interpretantur
Sabbatum apud Iohannem, \textgreek{[Greek]}. de quo infra.

Quin et Tertullianus ipse \textgreek{[Greek]},
quas cenas puras vocat, a diebus magnis, et a ieiuniis, et a
Sabbatis distinguit.

De Cena pura, praeter id quod diximus ad
Festum, ita reperi in veteri et peroptimo Glossario Latinoarabico:
\textit{Parasceue, cena pura, id est, praparatio, que fit prosabbato.}

Conditor Annalium Ecclesiasticorum turbat de cena
pura, et negat esse parasceuen, quia cena pura apud Festum
habeat offam suillam.

Sed ipse, (pace docti viri dixerim) non
aduertit Puram dici, non quia careat carnibus, sed quia religionis
et dicis caussa fit.

Nam et parasceuae Iudaicae habent carnes,
et nihilominus dicuntur cenae purae, quod dicis caussa coquebantur,
coquunturque hodie prosabbato, quia in Sabbato
coqui non liceat.

Non negabis, candide Lector, haec vulgo non intelligi.

Itaque locus ille est nobilissimus. 

Tamen quotus quisque est ex tot Lectoribus, qui non haec aut praeteribit,
aut calumniabitur?

Sequuntur periodi magnae Hagerenorum,
ex quibus ratio anni soluti Indorum, et Muhammedanorum
tota pendet.

Omnia nunc primum ex Arabum scriptis
prodeunt: atque adeo omnis tractatio nostris hominibus
nova est.

%\end{parnumbers}
\clearpage
p. VIII [pdf 35]
%\begin{parnumbers}
Excipit hanc doctrina anni Iudaici hodierni, res, quod
saepe diximus, artificiosissima, ideoque eximia, quia melior
anni Lunaris forma constitui non potest.

Docemus praeterea, unde
natus sit ille annorum computus, quo utuntur hodie, a \textsc{vii} Octobris:
quem inepte putant a conditu rerum.

Post multarum Periodorum,
Cyclorum, Octaeteridum, Paschalium historias, in locum vltimum
\textgreek{[Greek]} veteris anni Romanorum coniecimus, ideo
quod ea forma proxime abesset a Lunari: ubi de saeculo Romano,
et capite veteris anni Romani, temporibus vltimis C. Iulii Caesaris,
multa accuratissime disputata.

Itaque ex singulis rebus singula capita
confecimus, cum potius singuli libri et quidem ingentes confieri
possent, si, quae hominum hodiernorum est ambitio, eadem nobis
incessisset.

Tertio libro opportune annus aequabilis datus est,
cum annus Solaris Aegyptiacus, adscitis diebus quinque, ex Graeco
propagatus sit: (quemadmodum annus Lunaris ex eodem Graeco
manauit, abiectis ab eo totidem diebus cum horis \textsc{xv}, paulo amplius)
quod, metacente, Plutarchus docuit in libro \textgreek{[Greek]}.

Adeo inter se libri nostri mutuo conspirant, neq; ab eis ratio,
methodus, et ordo abest.

In eo libro de Neuruz antiqui Persarum
periodo annorum \textsc{cxx}, deq; cognominibus dierum Persicorum,
de translatione \textgreek{[Greek]} in enthronismis nouorum Regnum,
item de caussis anni Iezdegird, de annis Armeniorum, et eorum
mensibus, omnia nova protulimus. 

Sed haec non expergefacient animos
hominum, nisi forte ad obtrectandum.

Quartus liber est emendatio
tertii, ut secundus primo erat subsidiarius: qua methodo imperfectus
Lunaris Graecus libro primo disputatur, ut perfectus secundo.

Sic etiam perfectus Solaris, et siqui alii naturam perfecti imitantur,
supplent id, quod aequabili Aegyptiaco, Persico, et Armeniaco
vetuitas detraxerat.

Itaq; in quatuor partes tribuendus fuit.

In
prima continentur anni, quibus quarto anno exeunte dies ex quatuor
quadrantibus conflatus accrescit.

Ex illis nobiliores selegimus,
Iulianum, Actiacum, Antiochenum, Samaritanum, et alios. 

Nam et alios quoq; eius formae habebamus, ut Tyriorum, quorum menses
appelationibus Macedonicis, diuersa initia a Iulianis habent.
% "appelationb." interpreted as abbriviation for "appelationibus"

Sic
etiam Gazensium annus mere Actiacus fuit, appellationibus mensium 
Macedonicis, mensibus tricenariis. 

Marcus Ecclesiae Gazensis Diaconus,
in actis Porphyrii Gazensis Episcopi vocat \textgreek{Διον} Nouembrem
\textgreek{Απελλαιον} Decembrem quae nomina habent a Macedonibus. 

Sed
idem scribit Gazenses celebrasse Theophaniorum diem undecima
Audynaei, quae est sexta Ianuarii Iuliani, se autē redisse Constantinopoli,
Xanthici vicesima tertia, quam ait fuisse decimam octauam
Aprilis secundum Romanos: quibus ostēdit formam illius anni mere
Actiacam fuisse, mensibus tricenariis, appellationibus Macedonicis. 

%\end{parnumbers}
\clearpage
p. IX [pdf 36]
%\begin{parnumbers}
Secunda pars annos emendatos, eorumque emendandorum
rationem complectitur: tertia periodos multiplices, quarum finis
conciliatio anni civilis cum Solari, cui dies quinto quoque anno
ineunte accrescit.

Quarta pars agit de vera emendatione anni, et
de anno caelesti instituendo, qui pertinet ad methodum epochae
mundi.

Quemadmodum autem nulla Lunaris anni civilis ratio
recta iniri potest, praeter eam, qua Iudaei utuntur: ita nullus annus
caelestis Tropicus recte institui potest, nisi ex forma, quam edidimus,
quam nemo vituperabit, nisi qui ignorauerit; omnis laudabit,
qui intellexerit.

Alioquin scio et malignos et obtrectatores non defuturos. 

Annus tam noster, quam Iudaicus civilis quidem, sed naturalis,
vterq; ad motum quisq; sui sideris descriptus. 

Ideo eius saltem
in scriptis usus esse debet, qualis olim Philadelphi Dionysianus,
Chaldaeorum Calippicus, hodie Persarum Gelaleus. 

Tres igitur libri
primi, et prima pars quarti pertinent ad \textgreek{[Greek]} temporum civilium
cum septimo.

At reliquae tres partes quarti cum duobus libris
sequentibus pertinent ad ipsam emendationem temporum.

Atque
ut a mundi primordiis omnes res deducuntur, ita mundi epocham
primam ordine posuimus: qua in re quam pueriliter hallucinati sint
omnes, non sine admiratione tam imperitiae quam pertinaciae eorum
dicere possum.

Non loquor de iis, qui saeculo uno, aut pluribus altius
originem rei repetunt.

Nam quemadmodum ii nullam rationem
sibi proposuerunt, quam sequerentur, ita nullos lectores nancisci
possunt, nisi imperitos. 

Qui intra saeculum maiorem mundi
epocham faciunt, eorum duo genera reperio.

Prius genus est eorum,
qui solutionem captiuitatis in primum annum Olympiadis \textsc{lv} conferunt:
alterum eorum, qui tempus illud \textsc{xviii} aut \textsc{xix} annis ante
\textsc{lxiiii} Olympiadem definiunt.

In priori haeresi fuerunt et
quidam veterum Ecclesiasticorum, ut alicubi indicauimus. 

Aiunt
Cyrum caepisse imperare primo anno Olymiadis \textsc{lv}, hoc est 217
anno Iphiti, quod verum est: de quibus deductis septuaginta, relinquitur
annus excidii Hierosolymorum, et casus Sedekiae 147 a primo ludicro
Olympico.

Sed puerilis sentētia multis absurditatibus eluditur.

Primo computatione non recta annorum \textsc{lxx} a capto Sedekia.

Deinde quod Cyrum statim initio regni sui Regem Mediae, Persidos,
Susidos, Assyriae, Babyloniae, totius Asiae minoris, Indiae, totius
Syriae constituunt, qui unius Persidos Rex fuerit aliquot annis ante
casum Astyagis[?], et post illud tempus pauculis annis ante obitum
Babylone potitus sit.
% Astyagis or Aftyagis?

Haec sola absurditas facit, ut non solum eorum
nulla ratio habeatur, sed ut ludibrium quoq; debeant.

Tertio 147 annus
Iphiti est 118 Nabonassari: qui erat annus quintus ante initium Nabopollassari
patris Nabuchodonosori.

%\end{parnumbers}
\clearpage
p. X [pdf 37]
%\begin{parnumbers}
Ergo Nabuchodonosor anno
decimono regni sui templum et Hiersolyma euertit annis quinq;
ante quam pater ipsius, cui ipse successit, regnaret.

Digna profecto
talibus doctoribus sententia.

Tamen tantum abest, ut hac tam insigni
absurditate a sententia desistant, ut animos ab eiusmodi portentis
opinionum sumant.

Postremo ignorant diuersa esse initia Regnum,
ut ipsius Nabuchodonosori, cum patre, et solius: Alexandri,
ab excessu Philippi patris, et ab initio Seleuci: Diocletiani, ab aera
martyrum, et a primo anno imperii.

Sic etiam Cyri, apud Graecos, ab
initio regni Persidis: apud Babylonios, vel a subacto toto Babyloniae
imperio, vel ab aliquo insigni facto, quodcunque illud fuerit, sive
ex edicto ipsius Cyri, sive translatione \textgreek{πδν ἐπα γο υ[?]ων},
ut solebat fieri.

Qui tantam inscitiam sequi noluerunt, non tamen rectam viam
institerunt, quia quindecim aut amplius annis ante \textsc{xlvi} Olymiadem
casum Sedekiae coniiciunt.

Nos ante annum quartum illius
Olympiadis id non potuisse accidere ita demonstramus. 

Ezekias
Rex Iuda, postquam singulari Dei beneficio ab ancipiti morbo conualuisset,
anno \textsc{xiiii} regni sui, accepit Legatos et \textgreek{ςωτήρια[?]}, a
Merodach Rege Chaldaeorum.

Ponamus \textsc{xiiii} annum Ezekiae in
primo anno Merodach, hoc est, in \textsc{xxvii} Nabonassari.

Nam is est
annus primus Merodach apud Ptolemaeum ex Chaldaicis obseruationibus. 

Hoc modo annus primus Ezekiae conuenerit in annum
\textsc{xiiii} Nabonassari. Ab initio Ezekiae, ad excidium templi, anni
sunt absoluti 138.

Hoc est, annus ipsius excidii est 139 labens ab initio
Ezekiae.

quod ita demonstramus. 

Annus primus Sedekiae est
quartus Hebdomadis, teste Ierimia, initio cap. \textsc{xxviii}: et proinde
undecimus, qui et vltimus, est Sabbaticus. 

de quo extat testimonium
apud Ieremiam, et nemo dubitat.

Rursus annus tertius decimus
Ezekiae erat Sabbaticus. 

auctor Isaias \textsc{xxxvii}, 30.

ex quibus
manifesto colligitur, \textsc{xiiii} Ezekiae esse primum Hebdomadis, et primum
Ezekiae esse sextum Hebdomadis. 

Ergo annis ab initio Ezekiae unitas addenda, ad methodum anni Sabbatici.

Addita unitate annis
139, numerus erit septenarius. 

Quare annus labens 139 est verus
annus ab initio Ezekiae.

Quibus additis 13 annis Nabonassari praeteris
(quia posuimus 14 Nabonassari primum Ezekiae) componitur
annus Nabonassari 152, in quo casus Sedekiae ex hac hypothesi
locandus est, hoc est, in anno periodi Iulianę 4118: de quibus deductis
907 absolutis ab Exodo, remanet annus Exodi 3211 in periodo Iuliana.

Porro Nisan Exodi caepit feria quinta, ut toties diximus, et ex
Mose rectissime ante nos Iudęi docuerunt.

At in anno periodi Iulianę
3211 Nisan non caepit feria quinta, sed feria tertia, Martii \textsc{xi}, cyclo tam
Solis, quam Lunae \textsc{xix}.

%\end{parnumbers}
\clearpage
p. XI [pdf 38]
%\begin{parnumbers}
Ergo annus proximus, quo Nisan caepit feria
quinta, is debuit saltē esse annus Exodi: atq; adeo is fuerit annus periodi
Iulianae 3214: in quo sane nisan caepit feria quinta, Aprilis \textsc{vi},
cyclo Solis \textsc{xxii}, Lunae \textsc{iii}.

Additis 907 annis absolutis ab Exodo,
annus 4121 periodi Iulianae suerit is, in quo excidium templi contigit:
qui est quartus Olympiadis 46, ut erat propositum.

Sed et post
Olympiadem 46 ponendum esse casum Sedekiae ita probabimus.

Amasis rex Aegypti, postquam regnasset annos 55, obiit circiter annum
7 Cambysis, anno ante excessum ipsius Cambysis, hoc est, anno
225  Nabonassari.

Nechao intersectus est a Nabuchodonosoro anno
quarto Ioiakim regis Iuda.

Ieremias \textsc{xlvi}, 2.

Post eum regnauit
Psammitichus annos \textsc{vi}.

Cui Aprias, cuius meminit Ieremias
\textsc{xliiii}, 30, succedit.

Is post \textsc{xxv} annos relinquit regnum Amasi.

Summa annorum a caede Nechao ad obitum Amasis anni 86, qui
deducti de 225, relinquunt annum Nabonassari 139, quartum Ioiakim
Regis Iudae, primum Nabuchodonosori.

Ergo Sedekias captus
anno 158 Nabonassari, qui erat tertius 47 Olympiadis.

Idque verum
esse postea validissime demonstrabimus.

Diodorus Siculus,
auctor omnium Graecorum certissimus, attribuit, \textsc{lv} annos Amasidi.
reliquos Apriae et Psammatichi habemus ex Herodoto.

Temere
igitur, et imperite faciunt, qui casum Sedekiae antiquiorem illo
tempore constituunt: neque his cassibus sese explicare poterunt,
quantumuis sua commoueant sacra, ut Plautus loquitur.

His valide
demonstratis, et licentia chronologorum intra aliquos fines summota
quos amplius migrare non possunt, ad originas ipsas penetremus.

Sed prius ut in Mathematicis concessa quaedam, aut quae negari
non possunt, assumuntur, ita et nobis quoque faciendum.

Tempora et initia Regum Babyloniae a Chaldaeis notata in obseruationibus
eclipticis, quae reiicere et damnare extremae impudentiae et
inscitiae est: item, eorundem regum initia et tempora a Beroso Chaldaeo,
qui minus quam tribus saeculis post illos vixit, et qui quae Actis
ac fastis Babyloniorum publicis continebantur, ignorare non potuit,
haec inquam, non tantum tāquam vera haberi postulamus, sed etiam
qui aliter putant, tanquam indignos censeri, qui aut audiri a nobis
mereantur, aut vllas literas attingant, aut aliquem locum inter
doctos habeant.

Tricesimum annum, cuius initio Prophatiae suae
meminit Ezekiel, quique capti Iechoniae quintus erat, Iudaei inepti
deducunt a libro legis reperto, anno \textsc{xviii} Iosiae Regis.

Quis unquam a libro reperto vllam aeram, aut edicto Iosiae institutam, aut a
Prophetis usurpatam legit?

Si tanti erat illa temporis nota, quare
eam non usurpat Ieremias, qui tam accurate annos Regum Iuda Iosiae,
Ioiakim, Iechoniae, Sedekiae notare solet?

Capite \textsc{xxv}, quare dicit
anno quarto Ioiakim, cum dicendum esset vicesimo secundo a libro
inuento?

Esto, cur Ezekiel dixit tricesimo, non tricesimo a libro inuento?

qui tamen dixit anno quinto deportationis Regis Ioachin.

%\end{parnumbers}
\clearpage
p. XII [pdf 39]
%\begin{parnumbers}
Certe mos est uti epocha, quae omnibus et nota et in usu sut.

Quare
igitur epocham producit, neque plebi notam, neque in usu positam?

Sed quid ea epocha opus in Babylonia, inter deportatos?

Nugae Iudaeorum,
nugae sunt istae, et halluciationes doctorum, qui eos sequuntur.

Quare eruditiores Iudaeorum, huius absurditatis et nugatoriae
caussae conscii, his ineptiis explosis, dicunt, illum annum non a
libro inuento, sed Iubilei fuisse tricesimum.

At hoc est litem lite decidere.

Nam, quomodo Iudaei annos a Iubileo putarent, qui Iubilea
numquam usurparunt?

Annos quidem Hebdomadis notant, utinitio
\textsc{xxviii} Ieremiae mentio anni quarti septimanae: \textit{Initio regni
Sedekia, anno quarto.}

Rursus mentio anni primi, et secundi in annis
\textsc{xiiii}, et \textsc{xv} Ezekiae, apud Isaiam \textsc{xxxvii}, 30.

Sed notationem
per Iubilea, imo ne Iubilei quidem mentionem, nusquam, nisi
in lege, reperies.

Praecepta fuit tantum, non recepta Iubilei obseruatio.

Sed quae haec plumbea Iudaeorum sententia a \textsc{xviii} Iosiae
Iubileum putare?

Iubilea putantur a primo anno hebdomadis, non
a septimo.

At \textsc{xviii} Iosiae suit septimus septimanae, non primus.

Quare, si a Iubileis annos putare mos esset, suerit hic annus non utique
tricesimus, sed undetricesimus Iubilei, a \textsc{xviiii}, non a
\textsc{xviii} Iosiae.

Denique is erat annus 862 ab excessu Mosis, 855 a
diuisione terrae sive \textgreek{[Greek]}.

Ergo suit vicesimus secundus, non
undetricesimus Iubilei.

En quot errores locus praepostere sumptus
nobis peperit.

Cum igitur neque a libro legis inuento, quod est absurdissimum,
neque a Iubileo, quod est falsum dupliciter, ille tricesimus
annus putandus sit; sequitur, quod negari non potest, a
quodam rege tunc imperante putandum esse.

Nam deportati et captiui
inter victores, qua epocha uti possunt, nisi victoris?

In Palaestina,
cum aliqua esset Iudaeorum Respublica, et Ecclesia bene constituta,
Iudae cogebantur uti anno Alexandreo dominorum Seleucidarum:
quanto magis Chaldaeorum, in media Chaldaea, nullis legibus,
nulla Republica, nulla Ecclesia.

Nehemias initio libri sui ita
scribit: \textit{Accidit mense Casleu, anno vicesimo, cùm eßem in castro Susan.}

Si alibi non expressisset se de vicesimo anno Artaxerxis loqui, haud
dubie aliquod Iubileum hic commenti essent inepti Iudaei, et inepti
quidam hominum nostrorum sequuti essent.

Eodem quoq; modo
loquitur Ezekiel \textit{anno tricesimo}, non adiecto regis nomine.

Quid enim opus erat in Chaldaea?

Duo ergo Reges simul imperabant,
Nabuchodonosor, et ille, qui iam tricesimum annum currentem
imperabat.

Quisnam Rex, obsecro, potuit trecesimum annum in regno
agere, cum iam Nabuchodonosor tertium decimum regnaret?

Non alius igitur suerit, praeter Nabopollassarum patrem Nabuchodonosori,
quod verum est.

%\end{parnumbers}
\clearpage
p. XIII [pdf 40]
%\begin{parnumbers}
Nam \textsc{xxix} solidos annos imperauit,
teste Beroso.

Quod si filius eius anno \textsc{xxx} partis iam duodecimum
absoluerat, profecto imperare caeperit anno partis decimooctauo,
qui erat Nabonassari 140.

Nam primus Nabopollassari est 123 Nabonassari,
testibus Chaldaeis apud Ptolemaeum, ex defectibus Lunaribus
obseruatis.

Et proinde Sedekias captus fuit anno 158 Nabonassari,
tertio autem Olympiadis 47.

Vide locum Berosi apud Iosephum.

Nabopollassarus audita rebellione Aegypti misit filium eo
cum regio imperio, et regio exercitu: a quo tempore consurgit initium
Nabuchodonosori cum patre regnantis.

Mos erat Regum Babyloniae
et Persidis, ut aut prosecturi in expeditionem, filios reges declarent,
aut in expeditionem mitterent cum regio nomine, tanquam
designatos, si contigisset ipsum patrem mori, absente filio, ne
vllus de rege futuro tumultus oriretur.

Exemplum habemus apud
Herodotum de Cyro Cambysen in solium suum collocante in expeditione
in Scythas.

Hinc Ctesias Cambysi attribuit annos 18,
cum tamen solus regnarit octo annos, testibus omnibus veteribus
Graecis, et Chaldaeis ipsis apud Ptolemaeum.

Dario vero Notho annos
idem attribuit 35, cum tantum 19 solus imperarit.

Rursus Berosus
\textsc{xxxxiii} annos ait Nabuchodonosorum imperasse, comprehensis
nimirum 13 annis, quos cum patre communicauit, cum
illis quos solus in imperio transegit.

Quare Nabuchodonosori regnum
dixit non Satrapian, tanquam a patre non ut Satrapes, sed Rex
et socius imperii in rebelles missus.

Verba eius sunt haec: \textgreek{[Greek]}.

\textit{Victo rebelli, eius regionem regno suo subiecit.}

Mox subiicit, Nabopollassari patris morto[?]
audita, qui \textsc{xxix} annos solidos regnauerat, ipsum Babylonem se
contulisse: quod accidit proculdubio aliquot diebus post illud tēpus
ab Ezekiele designatum.

Obiit enim Nabopollassarus anno regni
sui \textsc{xxx}.

\textgreek{[Greek]}.

Pulcherrima haec est obseruatio, quam Beroso vernaculo
Babylonicarum rerum scriptori debemus.

Eadem verba repetit Eusebius
De pręparatione euangelica, ubi plane \textgreek{[Greek]}, quemadmodum
est apud Ptolemęum, nominat, non \textgreek{[Greek]}, ut perperam
est editum in Iosepho: ex quo ineptus quidam duos esse coniecit
Nabulassarum et Nabopollassarum; cum tamen eadē verba sint, ne
una quidem syllaba minus, praeter illud nomen.

Rursus apud Iosephum
lib.\textsc{x} ca.\textsc{ii}.eadem verba Berosi repetuntur.

Sed ubi hic est \textgreek{[Greek]},
ibi est bis \textgreek{[Greek]}, utrobique male pro \textgreek{[Greek]}.

Quam bene haec diuinis scripturis conueniunt?

%\end{parnumbers}
\clearpage
p. XIV [pdf 41]
%\begin{parnumbers}
Vnde etiam sequitur, mortuo Nabopollassaro, non tricesimum annum Nabuchodonosori
dici caeptum in Chaldaea, sed primum quae res obseruatione
digna.

Iudaei primum annum putarunt ab eo tempore, quo
cum imperio missus est. Sed in Chaldaea primus eius annus consurgit
ab obitu patris.

Itaque Danielis 11, annus secundus Nabuchodonosori
est sine dubio secundus ab obitu Nabopollassari, tricesimus
primus ab initio eiusdem, 152 ab initio Nabonassari, sextus
Sedekiae.

Vnde indubitata eruintur temporis nota illius Capitis secundi
apud Danielem, qui erat quartusdecimus nnus capti Danielis,
et sociorum cum rege Ioiakim, sextus autem regni Sedekiae.

Proinde annus ille erat \textsc{xiiii} Nabuchodonosori in Syria, secundus
autem in Babylonia: non autem \textsc{xxv}, ut coniicit Hieronymus
ex quadam victoria Nabuchodonosori de Syria, et Arabia, cuius
meminerit Borosus.

At Berosus loquitur tantum vsque ad obitum
Nabopollassari, qui erat \textsc{xiii} Nabuchodonosori eius filii.

His tam illustribus demonstrationibus sua somnia praeserant, quibus antiquius
est somniare, quam vera dicere, aut nosse.

Nos ad reliqua
pergamus.

Annus capti Sedekiae est 158 Nabonassari, 4124 in periodo Iuliana.

Deductis annis 907 solidis, relinquitur annus 3217
Exodi, qui est 2264 Iudaici Computi in quo sane Neomenia Nisan
habuit characterem feriam quintam, secunda Aprilis, Cyclo
Solis \textsc{xxv}, Lunaea \textsc{vi}.

Sed quadragesimus annus, et quadragesimus
septimus, hoc est 2303, et 2310 Iudaicus fuit sabbaticus.

Iosuae
\textsc{xiiii}, 7, 10.

Iudaei dicunt septenarios annorum Computi sive aerae
suae esse Sabbaticos.

Atqui 2303, et 2310 sunt septenarii.

Ergo recte
Sabbaticos annos putant Iudaei; ut apud illos post legem nihil
hac obseruatione vetustius sit; res prosecto, quae firmissimum
minimentum futura sit harum rerum investigatoribus.

Neomenia Nisan Exodi conueniebat cum neomenia Krionos.

Ita vere naturalis suit illa neomenia.

Praeterea quadragesimus septimus
annus conuenit sabbatico Iudaico: 902 autem annus est tricesimus
Nabopollassari conueniens cum testimonio Ezekielis.

Deniq; anni 86 a septimo Cambysae retro putati desinunt in anno
caedis Nechao Aegyptii, eodemque 139 Nabonassari: quod conuenit
eidem computationi.

Negari igitur non potest, hanc esse veram
Exodi epocham, quam et verbum diuinum, et usus anni Sabbatici,
et historiae fides penes eximium scriptorem Chaldaeum
Berosum, et naturales neomeniae utriusq; sideris in unum conuenientes
confirmant.

Quid postulamus praeterea?

An ut tam certis,
tam egregiis, tam firmis argumentis somnia Corybantum anteponamus?

Quis unquam ita haec demonstrauit?

Quid demonstrauit?

%\end{parnumbers}
\clearpage
p. XV [pdf 42]
%\begin{parnumbers}
Quis aliter potest demonstrare?

Iam a conditu rerum, ad exodum,
anni sunt absoluti 2452 cum mensibus sex ab autumno, anni vero
absoluti 2453 a vere.

Sed ante Exodum initium anni putabatur ab
autumno, et eodem initio in tempus veris translato, tekupha tamen,
hoc est, finis anni Solaris mansit in autumno, circa quam tekupham
Deus \textgreek{[Greek]} celebrari praecepit.

Igitur ubi initium anni
ab vltima antiquitate suit, inde et rerum quoq; initium repetendum.

quod quidem a nobis factum, damnata priori sententia, quae
initium rerum statuebat in vere.

Reliqua pete ex capite de conditu
rerum.

Praeterea, quibus annus Lunaris in usu est, illis commodius
initium, et rationibus Tropicis conuenientius ab autumno, quam
a vere, ut Iudaeis propter \textgreek{[Greek]}, et Pascha.

Nam si annum
nostrum caelestem admitterent, et hoc unum cauerent, ut \textgreek{[Greek]}
citima sit in secunda Zygonos, semper citimum Pascha esset in neomenia
Krionos.

quia interuallum a neomenia Zygonos, ad neomeniam
Krionos, est semper 178 dierum, uno die plus, quam a scenopegia
ad Pascha.

Anni Sabbatici caussas iam reddidimus, et verum
annum sabbaticum a Iudaeis hactenus obseruari demonstrauimus,
initio hebdomadum sumpto, non utique a defectu Mannae,
quod fanatici quidam, et veritatis hostes faciunt, sed a 48 anno Exodi,
ex capite \textsc{xiiii} Iosue, et rationibus doctorum Habraeorum, qui
dicunt septem annos \texthebrew{[Hebrew]}, id est, subiugationis terrae,
septem \texthebrew{[Hebrew]}
fuisse, id est, diuisionis.

quod rectissimum est: ideoq; hebdomadem
primam diuisionis, non subiugationis procedere in numerum.

An
potuit annus sabbaticus esse ante agrorum culturam?

Furor est aliter putare.

Tamen non desunt, non deerunt, qui solo contradicendi
studio, ut sapere videantur, aliter statuent: quibus per me non solum
hoc facere, sed etiam nos irridere licet; quandoquidem veritas apud
illos nullo in precio est.

Unde nata sit diversitas epochae excidii Ilii,
cum alii 407 annis, alii 405, eum casum antiquiorem prima Olympiade
statuant, aperuimus ex doctrina anni Attici, cui acceptum
referimus quicquid eximium ex alta obliuione eruimus.

Veram sententiam
Eratosthenis esse deprehendimus, quae illam cladem coniicit
in annum 407 ante caput primae Olymiadis: eiusque veram
diem in anno Iuliano ostendimus.

Primam autem Olympiadem
ex doctrina itidem anni Graeci \textsc{xxiii} die Iulii celebratam fuisse ante
nos aperuerat nemo.

Et tamen quidam Simioli tanquam rem
vulgatam in suis vanidicis Chronologiis retulerunt: cuius rei cognitionem
unus Pindarus, quem illi neq; viderunt, neq; norunt, nos
docuit.


%\end{parnumbers}
\clearpage
p. XVI [pdf 43]
%\begin{parnumbers}
Quemadmodum autem Olympia, ita etiam Karnia plenilunio
celebrata fuisse, libro primo, capite de periodo Laconum
ostendimus. neque solum plenilunio, sed etiam eodem anno, quo
Olympia.

Itaq; Herodotus libro \textsc{viii} Olympia et Karnia anno primo
Olympiadis 75 celebrata suisse scribit, pag. 307 editionis Henrici
Stephani nostri.

Cum multi eruditissimi viri, et quidem in iis
Onufrius Panuinius Pater historiae, multa accurate de Palilibus Vrbis
disseruerint, ut ei doctrinae nihil ad perfectionem deesse videatur,
tamen et plura deesse ex nostris disputationibus colligi potest.

Monere vero debent Annalium et Fastorum scriptores, qui tempora
sua ad annos Vrbis dirigunt, utra Palilia sequantur, Varroniana,
an Catoniana.

Nam certe Onufrius noster, tametsi Catonem sequitur,
tamen quibusdam imprudens ad Varronem transfugit.
% "transfugit" should not be rendered with a long s

Nisi
haec distinctio adhibeatur, ridicula multa consequi necesse est.

Exemplum habemus in annis Christi per annos Vrbis eruendis,
quod hactenus ab omnibus factitatum.

Christus in annis Varronianus
uno anno maior est apud aliquem, quam in Catonianis apud alium.

Quare, ut dixi, ridicula sunt.

In sequentibus epochis quanuis
non ea occurrit obscuritas, quae in prioribus: tamen semper aliquid
noue demonstratur, praeter superiorum scriptorum consuetudinem:
in quibus sunt quaedam de vero die et anno natalis Alexandri, eiusque
obitus: de Encaeniis Machabaei, de initio Simonis Iudaeorum
Ethnarchae, quem Iudaei Iohannem vocant, de aera Hispanica.

De quibus omnibus pluria nova disseruntur, quam trita et vulgaria.

Iam
excessum Herodis ad suum verum annum ex Iosepho retulimus,
qui ad epocham Actiacam illud tempus diligenter exigit, et praeterea
notationem, cui contradici non possit, adducit, defectum Lunarem,
qui contigit \textsc{ix} Ianuarii, anno 45 Iuliano ineunte, in cuius
anni sequenti Decembri Dionysius Exiguus imperite statuit natalem
Christi, nouem solidis mensibus scilicet post excessum Herodis.

Itaq; diligentissimus \textgreek{[Greek]} omnium scriptorum Iosephus
recte ait decessisse \textsc{xxxv} anno labente regni eius a captis a Sofio[?]
Hirosolymis. in quo tamen interpretatio adhibenda.
% Sofio or Sosio

Nam reuera Herodes
obiit anno tricesimo sexto ex diebus aestiuis noni anni Iuliani.

Ergo tricesimus sextus annus Herodis iniuit ex diebus aestiuis anni
Iuliani \textsc{xliiii}.

Obiit autem initio Nisan.

Igitur sine dubio decessit
anno Iuliano \textsc{xlv}, qui erat tricesimus sextus iniens ex diebus aestiuis,
ut diximus.

Sed ex computatione civili Iudaeorum, nondum
\textsc{xxxvi} annus iniuerat.

Iosephus enim, et Iudaei eo saeculo putabant
omnia tempora a \textsc{xxiii} Ijar, ut albi ostendimus: cuius consuetudinis
ignoratio multos decepit.

Ab Ijar igitur Hyrcani, sive, ut Iudaei
vocant, Iohannis Hasmunai, tricesimus sextus annus Herodis inibat,
qui tamen iam nouem mēsibus ante ex consuetudine Romana iniuisset.

%\end{parnumbers}
\clearpage
p. XVII [pdf 44]
%\begin{parnumbers}

Itaq; eius decessus confirmatur primum accurata putatione
diligentissimi scriptoris, deinde notatione eclipsis, quae omnem contradictionem
excludit.

At ex epilogismis Eusebii Herodes obierit
anno Iuliano \textsc{lii}, septem annis solidis post illum defectum.

qui stupor non meret castigationem, cum tanquam sorex indicio suo perierit.

Nam statim ab eius decessu tetrarchiam suam Archelaus eius filius
iniuit: quod quidem, si huic oraculo Eusebiano credimus, contigerit
anno Christi Dionysiano septimo labente.

Ergo Christus fuerit
annorum septem, cum ex Aegypto monitu Angeli reuoctus est.

Quod est ridiculum.

Rursus anno decimo regni, aut tetrarchiae suae
Archelaus ab Augusto relegatus est Viennam Allobrogum.

Secundum tempus ab Eusebio determinatum, hoc contigerit anno Iuliano
\textsc{lxi}, qui erat annus Tiberii tertius currens, biennio absoluto
post excessum Augusti.

Hoc modo anno tertio excessus sui Augustus
Archelaum relegauerit.

Vides \textgreek{[Greek]}.

Atqui innumeros videas,
quibus hoc somnium placet.

Nam sane omnes fere Chronologiae
et Annales hoc stigmate inusta sunt.

Atque utinam in illis hominibus
non esset vir eximia doctrina praeditus Dominus Caesar Baronius,
Annalium Ecclesiasticorum scriptor, cuius operis copia nobis
facta est ab amicis, cum haec \textgreek{[Greek]} scriberemus.

Is eruditissimus
vir ex hoc loco Eusebii Iosephum exagitat, tanquam imperitum
temporum: cum Eusebius potius ex Iosepho castigandus fuisset.

Nam absque Iosepho esset, quid certi de Herode haberemus?

Quis haec tractauit, praeter illum?

Qui fieri potuit, ut scriptor, cuius diligentia
et fides in notatione temporum spectatissima, in iis peccauerit,
quae sine illo Eusebius et alii ignorassent?

Sed ipse doctus Annalium
conditor potest iam videre, utri fides de hac re habenda, Iosepho,
cuius ratiocinia cum motibus caelestibus congruunt, an Eusebio,
cuius sententia et historiae, et rationi aduersatur?

Sed de Iosepho
nos hoc audacter dicimus, non solum in rebus Iudaicis, sed etiam
in externis tutius illi credi, quam omnibus Graecis, et Latinis.

Itaque
definat mirari doctus vir, cur tot eruditi, et nos quoq; qui non in illis
eruditis, sed in huius scriptoris lectione peregrini non sumus, tantum
illi deseramus, cuius fides et eruditio in omnibus elucet.

Caeterum de Eusebii anilibus hallucinationibus, praeter hanc, quam
modo protulimus, satis libro sexto differuimus.

Sed ad Epochas
nostras venio: quarum omnium rationem reddere longum esset.

De Epocha Martyrum Diocletianea non possumus tacere, eam hactenus
etiam doctissimis imposuisse, quod eam ab initio Diocletiani
incipere omnes credunt.

Hinc prodigiosi errores, et magna Consulum
confusio in Annales et Fastos deriuata sunt, praesertim in annis.

%\end{parnumbers}
\clearpage
p. XVIII [pdf 45]
%\begin{parnumbers}
Nam initio Diocletiani perperam sumpto, perperam quoque
persecutionis Epocha initur.

Ea semper antiquitus a solis Aegyptiis
Christianis hactenus usurpata fuit.

Itaque Historici et Chronologi,
qui temporibus Caroli Magni dicunt caeptum putari ab annis
Christi, cum antea mos esset annis Diocletiani uti, errant.

Nam
nullis nationibus in usu fuit.

Vnica autem Ecclesia duntaxat Alexandrina,
et quae illi subditae sunt, hac Epocha vsa est semper, utirurque
hactenus, et vocatur ab Aegyptiis, qui Elkupt dicuntur,
\textarabic{[Arabic]} \textit{Aera Martyrum sanctorum.}

Nam
hallucinatus est ille, qui nuper \textarabic{[Arabic]}
\textit{Captiuitatem} vertit in literis
Alexandrinae Ecclesiae Romam missis, anno Martyrum 1310, qui
erat Christi 1593.

Epocha igitur Martyrum iniuit \textsc{xxix} Augusti,
id est, neomenia Thoth Actiaci, vel Mascaram Habesseni, anno Christi
Dionysiano 284.

Initium autem imperii Diocletiani a Palilibus
anni 287.

Differentia anni duo, menses octo.

Perturbatio, quae est in
Consulibus a temporibus Maximinorum, vsq; ad filios Constantini,
ea utique ab antiquo est.

Sed et non minor confusio in annis persecutionis:
ubi magnae sunt \textgreek{[Greek]} apud Eusebium: quanuis
recte sentit de initio Diocletiani, et primo anno persecutionis.

Tamen
omnium Chronologorum fides hac in parte nutat.

Nam edictum
Diocletiani de tradendis codicibus prius est Ecclesiarum euersione,
euersio Ecclesiarum prior caede Martyrum.

Felix Africanus Episcopus
et socii eius supplicio in Campania affecti ideo, quod codices
Deificos, id est, sacram scripturam tradere noluissent.

Itaque in
Actis illorum scriptum fuit: \textit{Et ductus est ad passionis locum, cum etiam
ipsa Luna in sanguinem conuersa est, die tertio Kalendas Septembris}.

De Eclipsi
Lunari loqui manifestum est, cuius is color fuerit, quem sanguineum
astrologi vocant: cuiusmodi proculdubio accidit anno Christi
301, cyclo lunae 17, annis quatuor solidis ante edictum de euertendis
Ecclesiis, idque \textsc{iii} Nonas Septembris, non autem \textsc{iii} Kal.
Septembris, diebus quatuor post passionem Martyrum.

Itaque perturbatus
est ordo verborum.

Legendum enim videtur: \textit{Et ductus est ad passionis
locum, die tertio Kal. Sept. cum etiam ipsa Luna in sanguinem conversa
est.}

Id est, quo tempore Luna defecit, proximo nimirum novilunio.

Nam cum constet passos \textsc{iii} Kal. Septembris, et ita habeat
Kalendarium, non videtur esse error in notatione temporis.

At Dominus Baronius haec gesta confert in annum 302, tribus annis ante
persecutionem: et tamen putat eum esse secundum annum persecutionis,
qui erat decimus nonus Aerae Martyrum, decimus autem
septimus currens ab imperio Diocletiani.

%\end{parnumbers}
\clearpage
p. XIX [pdf 46]
%\begin{parnumbers}
Sed \textgreek{[Greek]} illorum
Annalium propagati sunt partim ex erroribus aliorum Chronologorum,
quos auctor sequitur, partim ex annis Christi male ad
suam et veram epocham reductis.

Vnde factum, ut ap initio operis,
ad tempora Nicenae synodi, ne unus quidem annus Christi
verae epochae suae redditus sit.

Itaque triennio aliquando, aliquando
quadriennio, ut plurimum autem biennio erratum est.

Exempli
gratia: Excidium Hierosolymorum contigit anno Christi
Dionysiano \textsc{lxx}, quo neomenia Nisan conueniebat cum neomenia
Xanthici, teste Iosepho.

In Annalibus refertur ad annum
72: qui est error Eusebii, sed alibi ab eodem castigatus.

Certum est, Fructuosum Episcopum, Christi Martyrem, cum fociis
passum anno antequam pax et interspiratio data esset Ecclesiis
sub Marco Aurelio Antonino, et L. Aelio Vero.

quod tempus Eusebius confert in annum quartum Olympiadis \textsc{ccxxxiiii},
id est Christi Dionysianum 160.

Ergo passus est Fructuosus anno Christi
159.

Hoc aliter demonstrabimus.

In Actis agonis Fructuosi et
sociorum legitur: \textit{Producti sunt duodecimo Kalend. Februarii, feria
sexta.}

Ergo litera Dominicalis erat B.

Proinde hoc accidit anno
159, triennio citius, quam notatum in Annalibus.

In Actis Andreae
militis et sociorum scriptum extat, eos necatos fuisse decimoquarto
Kalendas Septembris, Dominico die, hora secunda.

Igitur litera Dominicalis erat G.

Hoc necessario contigit anno 305,
qui erat primus persecutionis a Pascha illius anni antecedente, post
euersas Ecclesias: quod quidem Pascha celebratum 25 Martii, ipso
die termini.

At in Annalibus hoc refertur in annum 301, quadriennio
ante rem gestam.

Rursus in Epistola Vigilii Episcopi Tridentini
de Passione Sanctorum Sisinnii, Martyrii, et Alexandri,
ita legitur: \textit{Die paßionis Sanctorum, quarto Kalendas lunias, feria
sexta, nascente luce.}

Passi ergo sunt anno 403, cyclo Solis \textsc{xx}, quando
\textsc{xxix} Maii erat feria \textsc{vi}.

At in Annalibus dicitur scripta
anno 400 Christi.

Scripta ergo fuisset triennio ante caedem
ipsorum Martyrum.

Cui absurditati ipse non adscribet, certo scio.

In iisdem
Annalibus ex codice Antonii Augustini mentio fit Homiliae
Cyrilli Episcopi dictae in natiuitate Ioannis Baptistae, Pharmuthi
vicesima octaua, indictione prima, sub Theodosio iuniore et Valentiniano.

Ergo dicta fuit Homilia anno Christi 433, April. vicesima
tertia.

At in Annalibus refertur in annum 432, April. 29. S. Benedictus
Monachorum Occidentis Pater, obiit \textsc{xi} Kal. Aprilis, Sabbato
sancto, ut refert Aimoinus monachus ex Actis S. Mauri ipsius
Benedicti discipuli.

Toto illo saeculo hoc non potuit contingere, nisi
anno 536.

%\end{parnumbers}
\clearpage
p. XX [pdf 47]
%\begin{parnumbers}
Tamē in Annalibus Ecclesiasticis obitus Benedicti confertur
in annum 542, sex annis serius.

Multa igitur peccari necesse est
in Gestis Benedicti, quae in illis Annalibus referuntur.

In Encyclica
epistola Vigilii Papae scriptum fuit: \textit{Piißimus atque clementißimus
Imperator Dominico die, id est, Kalendis Februarii, gloriosos Iudices suos
ad nos destinare dignatus est.}

Anno 554 Kalendis Februarii fuit dies
Dominica.

At in Annalibus hoc confertur in annum 552, duobus
annis citius.

Anno 546 turbatio facta in Pascha, ut ex Cendreno docuimus,
capite de periodo Dionysiana, libro \textsc{iiii}.

In Annalibus referetur
sub anno 545.

Martinus Episcopus Turonensis obiit anno
395, ut accurate a nobis disputatum est.

Auctor Annalium Sigebertum
sequutus coniicit in annum 402.

Ex eo errore multum peccatum
est in temporibus Regum Francorum.

de quibus consulatur vltima
diatriba libri sexti huius operis nostri.

Non semel monuimus magnam
perturbationem esse in initiis Imperatorum, a Maximinis
ad Valentinianum.

Vt alios taceam, Constantini initium ab aliis in
305, ab aliis in 306 annum coniicitur.

At Constantinus iniuit imperium
post obitum patris sui Chlori.

Obiit autem Chlorus in Britannia
anno primo Olympiadis 271, ut inquit Socrates.

Nos ostendimus,
apud Socratem, Hieronymi Supplementum, Ausonium, et alios,
semper Olympiadem sumi pro lustro Iuliano, non pro lustro Olympico
Elidensium, idq; lustrum Iulianum biennio posterius esse Elidensi,
cum incipiat ab anno Iuliano bisextili.

Itaq; is fuit annus bisextilis,
quo obiit Chlorus, et imperium iniuit Constantinus.

Sed duae
cautiones adhibendae.

Prior est, ut scias annum Constantinopolitanum,
sive Nicenum hic intelligi, qui incipiebat a \textsc{xxiiii} Septembris.

Altera, ut prolepsis usurpata intelligatur in anno mortis Chlori.

Nam obiit \textsc{xxv} Iulii, \textsc{lxi} diebus ante
\textsc{xxiiii} Septembris, et
tamen obitus eius ad eundem annum refertur quo iniuit imperium
eius filius, \textgreek{[Greek]}, ut dixi.

Omnino igitur iniuit imperium anno
303, aut 307.

Nam primus annus Olympiadis Iulianae incipit semper
diebus 153 ante bisextum.

Sed nemo concedet Chlorum obiisse
anno 303.

Obiit ergo 307.

Et proinde anno 307 iniuit imperium
Constantinus, ex ante diem \textsc{viii} Kal. Octobr. eiusdem anni 307.

In his prouocamur a docto Annalium scriptore, et rem absurdissimam
prodidisse nos dicit, Constantini imperium iniisse ex anno
308, cum, ut inquit ipse, iniuerit anno primo Olympiadis 271,
Christi vero 306[?].
% 300 or 306 ?

Nos vero negamus vllam culpam aut absurditatem
in nobis admissam.

Nam annus Christi 308 Constantinopolitanus
incipit a Septembri anni 307, ut iam dictum est.

Et proinde ipsum, et alios errare, qui annum Christi 306 a Kalendis Ianuarii
dicunt esse annum labentem Constantini.

%\end{parnumbers}
\clearpage
p. XXI [pdf 48]
%\begin{parnumbers}
Hoc enim volunt,
cum putant primum 271 Olympiadis Elidensis annum esse primum
Constantini.

Olympias enim illa Iphitea caepit ex diebus aestiuis
anni 305, qui fuit annus primus presecutionis.

Quare in annis
Constantini, ut in aliis, insigniter peccatum est a viro docto.

His
postis, quinquennalia Constantini data sunt anno 312: vicennalia
autem anno 327.

Interuallum inter illas duas celebritates interiectum
haud dubie vocatur Indictio, iniens a datis quinquennalibus,
desinens[?] in vicennalibus, quibus concilium Nicenum dimissum.
% desinens or definens?

Sed neq; hoc placet Domino Baronio: neque caussam appellationis
Indictionum admittit.

At nos dicimus, non minus iniuste nos
hic, quam in initio imperii Constantiniani reprehendi.

An negat
Indictiones in quinquennia indici, et in quinquennalibus Principum
panegyribus remitti?

Si non credit, legat et quae priore, et quae
hac editione ad eam rem collegimus.

Quinquennalia illa dicuntur
\textgreek{[Greek]}, hoc est ad verbum, sparsiones, largitiones, profusiones, in
quibus liberalitas Principis ad remissionem vsq; tributorum, et indictionum,
editiones munerum et spectaculorum, congiaria, et donatiua
extendebatur.

Inde \textgreek{[Greek]} non solum pro illa largitione
sumitur, sed et pro ipsa indictionis temporalis nota.

Nam quod Latini
dicunt, Indictione prima, secunda, tertia hoc factum est, Graeci
dicunt, \textgreek{[Greek]}.

Non ergo nos, sed ipse fallitur.

Quid?

si initium Constantini a nobis ignoraretur, tamē quinquennalia
eius nos manu ad illud deducerent.

Itaque ignorari n n [?]
potest.
% Probable printing error. "n n" should read "non".

Neq; minus errat, cum cladem Maxentii coniicit in annum
312.

Quot modis enim hoc refelli potest?

Sed de eo suo loco.

Nam
Maxientius anno 313, non 312 extinctus est, ut recte Panuinius notat,
sed male inde Indictionum initia et caussas repetit: quod a nobis
olim diligenter discussum fuit.

%%% === Sextus Liber
Sextus liber continet residuum Epocharum,
in quo nobiliores quaestiones de Natali die, et Passione Christi,
de Hebdomadibus Danielis, quae breuibus diatribis explicari
non possunt, presequimur.

Ne autem aut rudiores, aut refractarii auctoritate
veterum scriptorum nobis praescribere possent, pauca de
Eusebii erroribus in antecessum delibauimus, in quibus, praeter frequentes
\textgreek{[Greek]}, puerile illud deliramentum de Effenis confutauimus,
quos Christianos fuisse hoc unico argumento probat, quod
\textgreek{[Greek]} essent, et solitarie viuerent, et monasteria haberent:

quasi Bonzios
Iapanensium Christianos esse censeamus, quia et coenobitae sunt,
et Psalmos quosdam instar monachorum Europaeorum alternis modulantur,
et horas Canonicales eorum exemplo habent.

Eorum Essenorum alii \textgreek{[Greek]}, alii \textgreek{[Greek]} fuerunt.

Sed horum non videtur
secta diuturna fuisse.

%\end{parnumbers}
\clearpage
p. XXII [pdf 49]
%\begin{parnumbers}
Ast \textgreek{[Greek]}, aut eorum non dissimilium
synagogae fuerunt ad tempora Iustiniani.

Sunt enim ii, qui Caelicolae
vocantur.

Nam et nomen id indicat.

Caelicolae enim sunt
Angeli.

Ita vocari volebant, propter sanctum, et caeleste, ut ipsis videbatur,
vitae institutum.

In perueteri Glossario Latinoarabico \textit{Caelicola}
[Greek][Arabic][?]. id est, Angelus.

Praeterea quia erant \textgreek{[Greek]}, novi
baptismi auctores Donatistis fuerunt.

Princeps eorum vocatur
Maior, ut et aliorum Iudaeorum.

Hoc enim est \texthebrew{[Hebrew]}.

Philo dubitans
quare Esseni illi dicti sint \textgreek{[Greek]},
 utrum quia medicinam profiterentur,

an quia Deum colerent, ex eo coniiciendum relinquit,
eos non dictos esse quasi \texthebrew{[Hebrew]} \textgreek{[Greek]},
 ut volebat quidam Lunaticus
literarum Hebraicarum professor, sed quia \textgreek{[Greek]} vocat, eo ostendit
\texthebrew{[Hebrew]} dictos, hoc est, \textgreek{[Greek]}.

Quod Christiani non essent, sed
mere Esseni, statim initio libri ostendit Philo.

sed et Sabbati summus
cultus, et reliqua, quae a Philone de ipsis narrantur, satis leuitatis
damnant Eusebium, et reliquos veteres, qui Eusebium sequuti,
idem hariolati sunt.

Sed in Annalium tomo primo tacite perstringitur
sententia nostra ab auctore, qui tamen fatetur veros Essenos Iudaeos
fuisse.

Mirati sumus, quomodo ille putauit in unum haec bene
conuenire posse, Iudaismum et Christianismum.

Vt hoc probet, ait
veteres patres idem scribere, quod Eusebium.

Atqui ex Eusebio
hoc desumpserunt, et eius auctoritate contenti Philonem non consuluerunt.

quem si legissent, nunquam tam ridiculae sententiae assensum
accommodassent.

Haec vero puerilia sunt.

Venio nunc ad natalem
Christi, quem vetustas Christianismi ad \textsc{xxviii} annum Actiacum
retulit, recte.

Nam Christus iniens annum unum a tricesimo
aetatis suae accessit ad baptismum, ut omnes vetustissimi Patres ex
Luca retulerunt, et post eos eruditus Annalium scriptor.

Baptizatus est anno \textsc{xv} Tiberii, duobus Geminis \textsc{coss}. anno
Iuliano 74.

Ergo \textsc{xxv} Decembris anni 73 illi inibat annus primus a tricesimo.

Deductis 30 annis absolutis de 73, remanet annus Iulianus
43, in cuius \textsc{xxv} Decembris natus fuerit Dominus, cyclo Lunae
\textsc{xviii}, anno Actiaco \textsc{xxviii},
 ut illi vetustissimi partes crediderunt,
duobus annis solidis ante epocham hodiernam Dionysianam,
anno solido cum diebus aliquot ante excessum Herodis.

Hoc proculdubio
verum est.

Sed in Annalibus peccatur ab auctore in anno
\textsc{xv} Tiberii.

Quem enim putat \textsc{xv}, is est \textsc{xvi}, et magno errore illi
attribuit Consules duos Geminos, quibus Consulibus annus \textsc{xvi}
Tiberii iniit ex \textsc{xix} Augusti, cyclo Lunae undecimo, anno Iuliano
74.

Nisan igitur is, qui proxime sectus est baptismum Christi,
Consulibus duobus Geminis, antecessit annum \textsc{xvi} Tiberii ineuntem,
mensibus quinque.

%\end{parnumbers}
\clearpage
p. XXIII [pdf 50]
%\begin{parnumbers}
At scriptor Annalium putat duos Geminos
Consulatum gessisse cyclo Lunae \textsc{xvi}: in quo ne sic quidem
sibi constat.

Nam is fuerit annus 75 Iulianus iniens.

Hoc modo Decembri anni 74 Christus iniuerit annum primum a tricesimo: et
deductis 30 absolutis, remanebit annus 44 Iulianus, in quo natus
Christus fuerit, tribus circiter mensibus ante excessum Herodis, anno
solido ante epocham Dionysianam, qua hodie Ecclesia utitur.

quae sane multorum veterum, inque illis Eusebii fuit opinio.

Sed
Christus baptizatus anno 74 Iuliano: passus 78.

Differentia, anni
quatuor solidi, paschata quinque.

Quorum nullum vestigium in illis
Annalibus extat.

Quinetiam auctor, quando numerus annorum
non succedit ex voto, culpam in Iosephum reiicit, mendacem multis
modis arguens: inter alia, quod scripserit \textgreek{[Greek]} factam post
Archelai relegationem, cum, inquit, ea \textgreek{[Greek]} Christo nascente
contigerit, et aperte Eusebius id indicauerit.

Nos hallucinationem
Eusebii loco suo confutauimus, in quo descriptionem patrimonii
Archelai cum descriptione totius orbis Romani confundit more
suo, neq; meminit verbis illis, \textgreek{[Greek]}, designari non
unicam fuisse illam descriptionem, cum \textgreek{[Greek]} mentio fiat.
% Final period not visible in original.

Quare
idem Euangelistes quemadmodum prioris meminit in Euangelio,
ita alterius mentionē facit in Actis.

ut non sit audiendus doctus Annalium
scriptor, qui non solum hac in parte Eusebii auctoritatem
Iosepho opponit, sed etiam adiicit descriptionem illam[?] eandem esse,
de qua Aethicus statim initio libri sui loquitur: cum tamen neque
tempus, neque res conueniat[?] descriptioni nascente Christo factae.

Nam descriptio, de qua intelligit Aethicus, caepit ab anno caedis
Caesaris, desiuit[?] in anno \textsc{xxxiii},
 qui erat tricesimus quartus a primis
Kalendis Ianuariis Iulianis, decem annis absolutis ante verum
natalem Christi, duodecim ante epocham Christi hodiernam Dionysianam.

Res autem eadem non est, imo longe diuersa: atq; adeo
tantem differt[?] descriptio, de qua Aethicus loquitur, a descriptione,
quae facta Christo nascente, quantum decempeda, et tabulae [censuales][?].

Nam illa descriptio Aethici mandata est agrimensoribus, et
Geometris, haec Rationalibus.

Illa orbis mensura, \textgreek{[Greek]},
hac census et facultates in Tabulas relatae.

Sed neq; recte concludit,
Iosephum hallucinatum, quod paulo ante initia belli Iudaici
auditam ex adytis templi vocem scripserit, quae diceret \textsc{hinc
migremvs}: cum, inquit, Eusebius id in passionis Dominicae tempus
referat.

Quomodo Eusebius melius scire potuit ea, quae contigerunt
Christi et belli Iudaici tempore, quam Iosephu? aut unde,
quam ex Iosepho? de illis dico, quae non pertinent ad historiam euangelicam.



%\end{parnumbers}
\clearpage
p. XXIV [pdf 51]
%\begin{parnumbers}
Sed tam friuolum argumentum eluditur iis, quae aduersus
hanc Eusebii hallucinationem libro sexto decimus.

Denique iniuste
vbique Iosephum reprehendit, omnium scriptorum veracissimum
et religiosissimum, quod quidem ipsius scripta loquuntur.

quem
auctorem si non tam contempsisset, nunquam eos
 \textgreek{ανἀχρονισμους [Greek:anachronism]} commisisset,
quibus totus contextus temporum primi tomi perturbatus
est.

Sed antequam ex hac velitatione facessimus, qua et nos et cognominem
nostrum scriptorem ab animaduersione docti viri vindicamus,
nos homines Aquitani expostulamus cum eo, quod a nobis
tres summos viros abdixit, Paulinum, Phoebadium, et Sulpitium
Seuerum:

qui cum suerint natione, et domo Aquitani, tamen
Paulinum et Sulpitium Romae natos scribit, Phoebadium in Hispania.

Quis illum docuit Paulinum non esse natum Burdigalae, ubi
antiquitus Paulina gens, hodieque quaedam regio vrbis Burdigalensis
Paulino cognominis est?

Phoebadium autem Aginni Nitiobrigum
Episcopum quare in Hispania natum dicit, aut quo auctore?

Apud Hieronymum male excusum est Soebadius, qui error irrepsit
ex Sophronio, ubi legitur \textgreek{[Greek]}.

Sed liber manu scriptus
Sanctae Mariae de Granateria liquido habet Febadium.

Apud Sulpitium
Seuerum deprauatum quoq; est, ubi legitur Fegadius, pro
Febadius, ut quidem librarii scribunt.

nam orthographia est \textgreek{[Greek]},
Phoebadius: satis hodie notus erudita sua in Arrianos Epistola,
quae ante \textsc{xxv} annos primum edita.

Mei municipes Fiarium vocant,
cuius memoriam bis quotannis instaurant, ineunte ieiunio
quadragesimae, et die Marci Euangelistae, mense Aprili, si bene
memini.

Huic successit Gauidius in episcopatu.

Sulpitium Seuerum
nemo hactenus Aquitanum fuisse dubitauit: sed patria ignoratur,
cum tamen ipse Nitiobrigem sese manifesto prodat, cum Seruationem
Tungrorum, Phoebadium autem suum Episcopum fuisse scribit.

Phoebadius autem erat Nitiobrigum Episcopus.

Iste Sulpitius
Ecclasiasticorum purissimus scriptor, post transitum Martini recepit
sese Elusonem, quo tempore ad eum scribebat Paulinus.

Id oppidum est cum arce veteri in finibus Nitiobrigum, qua amni Draguto
a Petrocoriis diuiduntur.

Vulgo \textit{Lausun}.

Sed de hoc satis.

Mei
Nitiobriges pro Sulpitio Supplicium dicunt, quomodo et Bituriges
suum illum vocant, quem eundem cum hoc faciunt perperam,
cum inter transitum Martini, cuius noster Sulpitius discipulus fuit,
et ordinationem Sulpitii Episcopi Bituricensis sub Guntchramno
Rege, intercedant plus minus anni 190.

Non iniuriam facimus
docto viro, si cum bona eius venia doctissimos viros Aquitanos,
et Christianissimos originibus suis vindicamus.

%\end{parnumbers}
\clearpage
p. XXV [pdf 52]
%\begin{parnumbers}
Sed quemadmodum tribus viris Aquitaniam orbaverat, ita eandem duabus
alienis civitatibus donavit, Reiensi, et Vasensi.

Prosperum non uno
loco dicit Regiensium in Aquitania fuisse Episcopum, cum dicendum
fuerit, Prosperum Aquitanum fuisse Episcopum Reiensium,
aut Regiensium in secunda prouincia Narbonensi.

Hodie \textit{Ries} vocatur.

Nugantur qui eum Regii Lepidi Episcopum et scripserunt,
et in fronte eius sacrorum poematum apponi curarunt: quasi Reienses,
in secunda prouincia Narbonensi, iidem sint cum Regio Lepidi
in Aemilia.

Vasense autem consilium idiotismus illorum temporum
vocauit, quod potius Vasionense dicendum erat.

Vasio Vocontiorum hodie \textit{Vaison} dicitur.

Est Episcopatus Auenioni metropoli
attributus.

Imperite quidam cum foro Vocontiorum confundunt.

Itaque Vasense, vel Vasionense, in Vasatense mutandum non
erat.

quemadmodum in anno Christi 552 perperam Firminum
Vticensem mutat in Venciensem.

Vticenses, vulgo dicuntur \textit{Vsetz}.

Est Episcopatus in prima Narbonensi.

Dicuntur etiam Vcetenses,
et Vcetiae Episcopus.

Apud Gregorium Turonensem libro \textsc{vi},
mentio est Ferreoli Episcopi Vcetensis: ubi vulgo male Vcecensis.

Sed tam imperite vulgus Vticenses deprauauit in Vcetenses, quam
Arausio in Aurasio: Vasensis dixit, pro Vasionensis.

At ciuitas sive
Episcopatus Venciensis, est in secunda Narbonensi. Vulgo S. Paulus
de Venciis.

Scribendum vero per t.[?] Ventiensis, \textgreek{[Greek]} enim dicitur
Ptolemaeo.

Fuitque Nerusiorum in Alpibus Graiis Metropolis.

Sequuntur in sexto libro illa quinque Paschata a baptismo
ad resurrectionem, fuis temporibus, Consulibus, et cyclis notata.

In tertio Paschate quid fit \textgreek{[Greek]}, explicamus,
quae verissima interpretatio adhuc assensum vel mereri, vel
exprimere a doctis hominibus non potuit: quod valde miror,
cum absurdissima sit ea, quam sequuntur ipsi.

Omnes igitur uno
ore putant \textgreek{[Greek]}, pro \textgreek{[Greek]} dictum esse.

Id ad verbum
Hebraice esset \texthebrew{[Hebrew]}:
 aliter \textgreek{[Greek]}, Latine Praeposterum.

quo nihil praeposterius dici potuit.

Nam quid est praeposterum Sabbatum?

Non pudet iocularis interpretationis?

Sed ita est.

Alius fortasse assensum extorsisset.

Sed quia a nobis, ideo
minus acceptum.

Theophylactus post Epiphanium, et alios veteres,
interpretatur \textgreek{[Greek]}.

Itaque verum est, quod diximus, omnes tam veteres, quam
recentiores \textgreek{[Greek]} interpretari
 \textgreek{[Greek]}, id est \textgreek{[Greek]},
praeposterum.

Vt illud probet, idem Theophylactus
subiicit: \textgreek{[Greek]}.

%\end{parnumbers}
\clearpage
p. XXVI [pdf 53]
%\begin{parnumbers}
Quod falsum est,
propter translationes, quas imperiti negant saeculo Christi usurpatas,
cum tamen longe ante Christum in usu fuisse demonstrauerimus,
ut locus non sit pertinaciae.

Sed valeant Sabbata praepostera.

Igitur \textgreek{[Greek]}, non quod \textgreek{[Greek]}, sed
quod \textgreek{[Greek]}.

Nam \textgreek{[Greek]} inibat
computus \textgreek{[Greek]}.

Hebraei etiam hodie vocant \texthebrew{[Hebrew]}
\textgreek{[Greek]}: id est, Sabbatum, quod est primum a
\textsc{xvi} Nisan, quae est \textgreek{[Greek]}.

Vera, et recta interpretatio sine ulla praeposteritate.

De quinto Paschate verbum non addidissem,
nisi cum haec commentarer, incidissem in Commentarios quorundam,
qui Christum passum volunt ipso solenni Paschatis, \textsc{xv}
Nisan, feria sexta, nempe parasceve Sabbati.

Quanuis auctoritas
Evangelistarum, ratio ipsa, doctrina veteris anni Iudaici, omnia
denique contra illos faciant, tamen potius etiam Evangelistas ipsos
valere iubebunt, quam ut sententiam mutent.

Caussa pertinaciae
verba Evangelistarum, \textgreek{[Greek]}.

Contra obiicitur:
\textgreek{[Greek]}.

Iohan. \textsc{xix}, 14.

Hic miseram latebram quaerunt,
et strenuo mendacio ictum declinant.

Aiunt \textgreek{[Greek]} tantum
dici de Sabbato.

Acuti homines!

Quare dicit \textgreek{[Greek]},
nisi \textgreek{[Greek]}, quasi et sit \textgreek{[Greek]}
 alius rei, quam \textgreek{[Greek]}?
 
Quod quidem verum est.

Nam \textgreek{[Greek]} est genus, cuius species
\textgreek{[Greek]}, \textgreek{[Greek]}.

Utrumque uno verbo Hebraeis est
\texthebrew{[Hebrew]}.

Itaque \texthebrew{[Hebrew]} \textgreek{[Greek]},
 sive \textgreek{[Greek]} dicitur,
\textgreek{[Greek]}, quod sint aliae \textgreek{[Greek]},
 sive \textgreek{[Greek]}, quae a festis
suis appellationem sortiuntur.

\texthebrew{[Hebrew]} \textgreek{[Greek]}, \textgreek{[Greek]}.

\texthebrew{[Hebrew]} \textgreek{[Greek]}, \textgreek{[Greek]}.

Itaque ei parascevae, in qua Christus passus, accidit tum
vt ex consuetudine esset \textgreek{[Greek]}, id est, \textgreek{[Greek]}:
tum, ut casu \textgreek{[Greek]}, id est, \textgreek{[Greek]}.

Sabbatum enim
illud erat \textgreek{[Greek]}.

Quae propterea dicitur \textgreek{[Greek]}
ab Evangelista.

\textgreek{[Greek]}.

Quod
et ipsum quoque \textgreek{[Greek]} dictum, tanquam sit quaedam
 \textgreek{[Greek]},
quae non sit \textgreek{[Greek]}.

Nam quod Hebraice dicitur \texthebrew{[Hebrew]},
id est, Solenne, id \textgreek{[Greek]} Iudaei, et Apostolus vocant
 \textgreek{[Greek]}.

Unde \textgreek{[Greek]} sive \textgreek{[Greek]} solenne \textgreek{[Greek]}
dicitur \textgreek{[Greek]}, ut supra \textgreek{[Greek]} citavimus, nimirum
exemplo Iudaeorum, et Samaritarum, qui \textgreek{[Greek]}
\texthebrew{[Hebrew]} vocant \textgreek{[Greek]}.

Tria autem proprie Hebraice vocantur \texthebrew{[Hebrew]},
aliter \texthebrew{[Hebrew]}, quas \textgreek{[Greek]} quoque vocarunt
 \textgreek{[Greek]}, \textsc{xv} Nisan, id est, \textgreek{[Greek]},
 cum \textsc{xxi} Nisan.

\textsc{vi} Sivvan,
id est, \textgreek{[Greek]}.

\textsc{xv} Tisri, id est, \textgreek{[Greek]}, cum \textsc{xxii} Tisri.

%\end{parnumbers}
\clearpage
p. XXVII [pdf 54]
%\begin{parnumbers}

Acron quidem in illud \textit{– hodie tricesima sabbata} scholio isto
\textit{quae Neomenias esse dicunt: quoniam per Sabbata Iudei numeros Lunares
accipiunt.}

\textit{Et Sabbatum magnum in renovatione Luna a Iudaeis
hodie celebratur}: videtur omnes neomenias nomine Sabbati
magni indigetare: sed, quid sit Sabbatum magnum, ignorat.

At
Sabbatum ordinarium nunquam dicitur \textgreek{μεγάλη ἡμέρα},
% Great Day
 non magis,
quam \texthebrew{[Hebrew]}.

Sic Philo libro \textgreek{[Greek]} dixit \textgreek{[Greek]},
loquens de Pentecoste.

At ordinarium
Sabbatum nunquam dicitur \textgreek{[Greek]}, aut \textgreek{[Greek]}.

Temere igitur
doctor Theologus Commentario in Iohannem ait \textgreek{[Greek]}
tantum dictam de feria sexta, et omne Sabbatum dici posse \textgreek{[Greek]}.

Ecquae Grammatica haec est, ut \textgreek{[Greek]}
non sit \textgreek{[Greek]}?

Quid potest dici absurtius?

Ac propterea
longe iocularius dicit eodem modo dictum \textgreek{[Greek]},
ut Ioh. \textsc{vii}. 37. \textgreek{[Greek]}.

Nam verum est eodem modo dictum, sed
contra animi eius sententiam: quod nimirum dicta sit \textgreek{[Greek]},
quia octava Tabernaculorum, non quia Sabbatum.

Quae quidem
octava fuit eo anno feria quinta, non Sabbatum, 169 diebus ante
passionem.

Itaque dum hoc effugium parat, suo se gladio iugulat.

Quis unquam tam obstinatos adversus veritatem animos credidisset?

Illam autem obiectionem quam argute eludunt ! [?] \textgreek{[Greek]}.

Aiunt \textgreek{φαγεῖν[?] τὸ πάσχα}, hic non esse
agnum Paschalem comedere, sed alia sacrificia Paschalia.

Argute, docte, eleganter, ut nihil supra: quasi in parasceve comedere
Pascha aliud sit, quam agnum Paschalem comedere.

Imo nos negamus, \textgreek{θύαν κὶ[?] φαγεῖν τὸ πάσχα[Greek]},
 aliud esse, quam agnum Paschalem
immolare, aut manducare.

Exodi \textsc{xii}, 21.

Neque ullus paulo doctior illud sine risu audire potest.

Sed quid ex tot mendaciis consequuntur?

Quid, quam ut se deridendos propinent?

Si quintadecima Nisan, quando Christus passus, fuit feria sexta: ergo Pentecoste
illius anni fuit Sabbatum.

Nam Pentecoste est feria secunda
quintaedecimae Nisan, ut diximus ad Computum Iudaeorum.

Fallitur ergo Ecclesia, et omnis Christianitas, quae ab ultima usque
 antiquitate
credidit illam Pentecosten fuisse diem Dominicam, non
Sabbatum.

Quid igitur?

Quid?

\textit{— Dic aliquem, dic, Quintiliane, colorem.}

Itaque Doctori Theologo non bene procedit commentum.

Rursus urgetur absurditate.

Deus die magno Azymorum districte
vetat opus facere.

Exodi \textsc{xii}, 16.

Levitici \textsc{xxiii}, 7.

Hinc quoque
aliquo insigni facinore elabendum erit [?].

Adducit locum ex libro Iudaico,
cui titulus \texthebrew{[Hebrew]} id est, ligatio Isaaci, ut probetur etiam
Sabbato licuisse[?] opus facere: in quo is, qui locum vertit, imposuit homini
quaestionum, quam Hebraismi peritiori.

%\end{parnumbers}
\clearpage
p. XXVIII [pdf 55]
%\begin{parnumbers}

Nam qui interrogat,
an Sol occasus sit, item, an sit Sabbatum, eandem rem duabus interrogationibus
significat.

Si enim Sol nondum occidit, est Sabbatum,
in quo non auderet mittere falcem in messem.

Oportet igitur,
ut Sol occiderit prius, et consequenter non erit amplius Sabbatum,
id est, iam praeterierit quintadecima Nisan, quae quacunque feria
inciderit, dicitur Sabbatum, Levitici \textsc{xxiii}, 15.

Igitur interroganti,
an Sol occidit, respondetur, occidisse.

Rursus, an sit Sabbatum, id
est, an \textgreek{[Greek]} nondum praeterierit, si respondetur adhuc
esse, nihil agitur: si respondetur non esse Sabbatum, tunc confidenter
immittit falcem in messem.

Haec profector est mens illius loci,
quanquam libri copia non est.

Sed qui vertit illi haec verba, dicit responderi
esse Sabbatum.

Ergo hoc modo oportebat omnem \textsc{xvi}
Nisan esse Sabbatum, omni anno.

Quod quis non miretur a Doctore
Theologo non animaduersum?

Atque adeo illi commentum
placet.

Denique tot lapides movit, ut tandem concluderetur, Ecclesiam
falso putare diem Pentecostes, quando Spiritus sanctus super
Apostolos descendit, fuisse Dominicam, cum fuerit sabbatum, ex
hypothesibus Doctoris.

Concludimus igitur, quod nemo sani capitis
negaverit, Chistum Pascha comedisse tertia decima Nisan civilis,
quartadecima Luna.

Unde recte Evangelistae: \textgreek{[Greek]},
nempe \textgreek{[Greek]}.

Nam sane quoties fit translatio feriae,
tunc duplex est neomenia, prior quidem \textgreek{[Greek]}, posterior vero
\textgreek{[Greek]}.

Sed, inquiet, alius Evangelistes dicit, \textgreek{[Greek]}.

Ergo omnes \textgreek{[Greek]}.

Non sequitur: quare alius interpretatur, \textgreek{[Greek]}.

Christus \textgreek{[Greek]}.

Christur immolavit Pascha in qua die
oportebat, nempe quartadecima Luna.

Iudaei postridie \textgreek{[Greek]},
in qua non oportebat, nempe quintadecima Luna.

Et ita quoque
hunc nodum soluerunt Monachus Veronensis Hilario, et Paulus
Episcopus Burgensis ex Iudaeo Christianus.

Neque melior solutio
dari potest.

Neque vero illi duo viri docti tam vacui capitis fuerunt,
ut crederent eo anno \textgreek{[Greek]} fuisse feriam sextam.

Sed Doctor melius Latine intelligens, quam Graece, vulgatam tralationem [sic]
sequitur: \textit{In qua necessarium erat immolare}.

Nos negamus
\textgreek{[Greek]} bene traductum, \textit{necessarium erat}.

Atque adeo intererat Logici
scrire, quatenus \textit{Oportere, et Necessarium esse} differunt.

Absurde igitur, imperite, et adversus Evangelistarum mentem, dicitur
Christum crucifixum ipsa die solennis Paschatis.

Nos vero et
hic mutavimus sententiam, cum huic stultae interpretationi haereremus
priore editione: quemadmodum cum Chistum cyclo \textsc{xvi} crucifixum
asserebamus, sequuti Dionysium Exiguum, et alios veteres.

%\end{parnumbers}
\clearpage
p. XXIX [pdf 56]
%\begin{parnumbers}

Nam Christus passus cyclo \textsc{xv} Lunae, \textsc{xiiii} Solis,
 tertia Aprilis,
anno quarto absoluto a baptismo, quando Sol extra ordinem caligavit,
cuius casus etiam meminit Phlegon, feria sexta, quando
post verum agnum immolatum, immolatus est et typicus, qui tum
primum perperam immolari ceptus, usque ad 70 annum Christi
Dionysianum, quando inclusis in urbem die primo Azymorum
contigit Pascha ultimum immolare.

Atque hactenus quidem de
priore parte sexti libri.

Venio ad alteram, cuius subiectum quo nobilius,
eo plures tractatores habuit: ut nullus non ex plebe scriptorum
ex hoc mustaceo lauream sibi quaesiverit.

Ac quanquam sine
summa doctrina externae historiae, et peritia bonarum literarum
ad ista arcana penetrari non potest, tamen quo quisque imparatior
ab omni copia humaniorum doctrinarum, eo audacius ad hanc
tractationem se contulit.

Quin etiam tantum abest, ut praesidia,
sine quibus hic labor irritus est, isti adhibuerint, ut eos insanire
putent, qui per illa viam sibi ad haec indaganda muniverunt.

Minima quaeque persequi esset horas perdere.

Tria praecipua attingere
satis pro tempore erit, nempe de septuaginta annis captivitatis,
de Regibus Persidis et Babyloniae, de epilogismo Hebdomadum
Danielis.

Septuaginta annorum caput a capto Iechonia sumendum
esse, auctor Ieremias scribens ad eos, quos com Iechonia Babylonem
Rex Nebuchodonosor deportaverat, cap. \textsc{xxix}, post
alia: \textit{Quia Dominus ita dicit: Quando septuaginta anni Babyloni completi
fuerint, ego visitabo vos, et verbum meum bonum super vos
suscitabo, ut vos huc reducam}.

Quid clarius hoc commate?

Vos,
quos cum Iechonia captivos Babylonem taduxit Rex, ego huc
reducam, postquam septuaginta anni completi fuerint Babyloni,
quae vos captivos detinet.

At contra hos 70 annos acuti homines
ineunt a capto Sedekia.

Ergo septuaginta anni sunt octaginta.

Mirum vero Ieremiam nescisse septuaginta esse septuaginta.

Quemadmodum
igitur negando parasceven Pascha esse parasceven
Pascha, res nova et inaudita concluditor, Spiritum sanctum in 
Apostolos descendisse Sabbato, non die Dominico: ita etiam
negando septuaginta annos esse septuaginta annos, haud dubie
aliquid \textgreek{[Greek]}, \textgreek{[Greek]} parturitur.

Audiamus
caussam tam inopinatae interpretationis.

Scriptum est, inquiunt, urbem
Hierosolyma per septuaginta annos sua sabbata requieturam.

Locos, quem designant, est in fine posterioris Chronicorum:
\textit{Ad complendum verbum Dei in ore Ieremia,
 donec terra acquiescat sabbatis
suis. omnes dies desolationis sabbatizavit, usque ad complendum
septuaginta annos}.

%\end{parnumbers}
\clearpage
p. XXX [pdf 57]
%\begin{parnumbers}

Clare loquitur, omni ambage remota, terram,
quamdiu desolata fuit, sabbatizasse, id est, incultam cessasse, donec
complerentur anni septuaginta ab Ieremia determinati.

Quod
proculdubio verum est.

Nam finis desolationis est septuagesimus
annus: initium vero intra illos, non ab illis.

Nam tametsi terra septem
tantum annos, aut unum annum cessasset, tamen sequeretur,
quod hic dicitur, quandiu desolata suerit, cessasse: et quidem cessase,
usque ad septuagesimum annum ab Ieremia determinatum.

Quid verius, quid simplicius, quid clarius hac interpretatione?

Sane
ex his non colligitur, cessasse septuaginta annos, sed cessasse,
quandiu desolatio duravit: duravit autem usque ad tempus ab Ieremia
definitum, cuius temporis initium aliud est ab initio desolationis.

Huic simile, imo prorsus idem genus loquendi extat eodem capite,
comate [?] 10.

Ioiachin cognomento Iechonias regnavit menses tres,
dies decem.

Tamen ibi dicitur: \textit{Anno vertente, Rex Nabuchodonosor
misit, et deportatus fuit Babylonem}.

Decimum diem mensis quarti
vocat annum vertentem.

Nimirum ille annus aliud habet initium
ab initio regni Iechoniae, cum tamen et anni et Regis Iechoniae
idem finis sit.

Sic initium septuaginta annorum aliud ab initio desolationis,
cum finis septuagesimi anni, et desolationis idem sit.

Neque sane ullus alius sensus hinc elici potest.

Et ita in eodem capite
duo loci similes alter alteri praelucet: quae locorum duorum
collatio mirifice sophisticam obiectionem retundit.

Sed isti boni
interpretes verbum divinum pro pila habent.

Et profecto aliquid
novi, ut dixi, pariet tam portentosum commentum.

Audite
ergo \textgreek{[Greek]}.

Annus capti Iechoniae est \textsc{xxvi} Nabopollassari,
ut supra demonstratum est.

Annus igitur excidii Hierosolymorum
et eversionis templi, est 36 Nabopollassari, qui erat 158 a
Thoth Nabonassari.

Ergo septuagesimus annus ab excidio templi
erit 227 ab eodem Thoth Nabonassari: qui est secundus Darii filii
Hystaspis, testibus observationibus Babyloniorum eclipticis apud
Ptolemaeum, nonus autem a decessu Cyri.

Quare hoc modo
Cyri annus primus, quo soluta captivitas, conuenit in annum
nonum post eius obitum, ut volunt acuti homines, praestantissimi
interpretum, et facrorum Bibliorum hierophantae, qui septuaginta
dicunt esse octaginta.

Tantae molis erat ostendere septuaginta
annos esse octaginta, ut aliquis Rex nono anno post obitum
suum edicta faceret.

Non igitur septuaginta anni a desolatione
ineundi, sed a servitute, et ex quo tempore Iudaei tributarii
Chaldaeis facti.

Itaque recte atque his convenienter apud Ieremiam
\textsc{xxv}, 11.

\textit{Omnis terra erit desolata, et vasta: atque omnes
illa gentes servient regi Babylonis septuaginta annos}.

In his non Iudaeos
tantum comprehendit, sed etiam omnem Syriam.

%\end{parnumbers}
\clearpage
p. XXXI [pdf 58]
%\begin{parnumbers}

Nam Iudaei
non vocantur \texthebrew{[Hebrew]}, sed omnes alii extra iudaeos.

Itaque non solum
habitantes Hierosolyma, sed etiam omnes finitimas nationes desolatum
iri dicit, commate antecedente, usque ad septuaginta annos:
quorum initium a tempore subiugatae Palaestinae, et capti Iechoniae,
ut etiam clare extat apud Berosum: a cuius fragmento hinc
magna lux affulget.

Sequuntur Reges Babyloniae, et Persidis.

Quis narraverit somnia, hallucinationes, mendacia hominum in horum
Regum Chronologia?

Quid dicam odium, \textgreek{[Greek]} et \textgreek{[Greek]}
eorum in nos, quod praeter eorum expectationem eos Reges non
in \textgreek{[Greek]} Aristophanis, sed apud custodes priscarum Originum
Berosum Chaldaeum, Megasthenem, Herodotum invenerimus?

Quod eos ipsos in Daniele, Zacharia, Esdra, et Nehemia sine
ulla mutatione extare indicaverimus?

Quod ex duobus Regibus
Assuero, et Artaxerxe unum Regem bicorporem non fecerimus?

Berosi igitur et Megasthenis eximiae illae reliquiae apud Iosephum
nobis veritatis fontes recluserunt.

Earum beneficio habemus Reges
Chaldaeorum a Nabuchodonosoro, ad captam a Cyro Babylonem.

Ex quo captae Babylonis tempus iniri potest, cum is fuerit annus
\textsc{lxxxv} a Nabopollassaro patre Nabuchodonosori, ducentesimus
tricesimus sextus Iphiti,
 \textsc{xxi} imperii Cyri.Lib. \textsc{vi}, cap.de duabus
quaest.

Danielis \textsc{vii} annis minus diximus.

Quare corrigatur
numerus.

Sed prius quam ad reges Nabuchodonosori successores aggrediamur,
discutiamus opiniones interpretum Danielis, et Chronologorum.

Maxima pars horum hominum septuaginta annos a
Ieremia determinatos, a capto Sedekia deducunt, in primo anno
Darii Medi terminant, hoc uno argumento moti, quod Daniel primo
anno huius Darii mentionem facit vaticinii Ieremiae de septuaginta
annis, ideoque finem eorum annorum tunc accidisse dicunt.

Sed haec tam actuta sententia facile retunditur.

Nam eodem modo
a primo anno huius Darii ineundae essent Hebdomades, quia eodem
capite earum praecipua mentio fit, imo est unicum subiectum illius
capitis.

Ita sane pueri solent argumentari.

Verba Danielis: \textit{Ego
Daniel inellexi in literis numerum annorum}, etcetera.

Ego, inquit inter
legendum, animadverti septuaginta annos definitos captivitatis.

Non sequitur, illum fuisse septuagesimum captivitatis.

Rursus hunc
Darium volunt esse Astyagem Regem Mediae, filium Cyaxaris,
quem Daniel vocaverit Assuerum.

At quis unquam veterum calssicorum
auctorum scripsit Medos Babylone imperasse?

Quis non
illorum dixit Astyagem a Cyro victum, et imperio exutum?

%\end{parnumbers}
\clearpage
p. XXXII [pdf 59]
%\begin{parnumbers}

Nam Xenophontem tam constat historiam noluisse scribere, sed exemplum
bene educti principis proponere, quam certum est, nihil in tota Cyri
paedia verum esse, praeter sola nomina, et nudam mentionem duorum,
aut trium casuum, ut Babylonis captae, Croesi victi.

[Prolegomena continues up to page LII]

\end{parnumbers}
