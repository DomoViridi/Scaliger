% !TEX root = ../de-emendatione-temporum-1629.tex
% !TEX TS-program = xelatex
% !TEX encoding = UTF-8 Unicode
% this template is specifically designed to be typeset with XeLaTeX;
% it will not work with other engines, such as pdfLaTeX

%%% Count out columns for fixed-width source font
% 000000011111111112222222222333333333344444444445555555555666666666677777777778
% 345678901234567890123456789012345678901234567890123456789012345678901234567890

\setheaders{\shorttitle{} Liber III}{\shortauthor{}}
\chapter{De Anno Aequabili Maiore}
%
% 188
% {PDF page nr}{source page nr}{line nr}
\plnr{271}{188}{1}In Astronomicis \textgreek{ὁμαλαὶ κινήσεις} vocantur
eae, quibus ex Anomaliarum Canonibus
competentes \textgreek{προσθαφαιρέσεις} adhibitae non
sunt.
\lnr{4}Ideo Illis motibus aut deest semper,
aut superest aliquid.
\lnr{5}Sic in ratione anni
eum annum aequabilem vocabus, cui propter
commodiorem usum aut abest, aut superest aliquid.
\lnr{8}Exempli gratia: Persico anno
quadrans de Solis ratiociniis deest.
\lnr{9}At
Graeco supra Lunae ratiocinia dies fere sex supersunt.
\lnr{10}Utrique hoc contigit
propter et commodiorem mensium in triginta dies tributionem,
et aequabilem dierum descriptionem: qua quidem aequabilitate mensium
Graeci volebant Lunae motum assequi. % ss/ff ?
\lnr{13}Sed hoc erat ultra fines
iaculum expedire.
\lnr{14}Plus enim dierum ad eam rem assumebatur, quam
modus anni postulabat.
\lnr{15}Contraria ratione Orientis nationes, cum
eodem anno antea uterentur, atque ad vitandum taedium embolismorum
eum castigare vellent, primum illum a Solis propius, quam
a Lunae rationibus abesse animaduerterunt: quod videlicet populares
temporum errores magis in Luna, propter illius sideris celeritatem,
deprehenduntur, quam in Sole: deinde convenit inter
eos, ut ex modulo anni sui, qui 360 dierum tantum erat,
partem 72 decerperent, atque eam anno appenderent.
\lnr{22}Divisa itaque
anni quantitate in 72, excerptae sunt dies \rnum{v}, et in calcem anni
post 360 dies reiectae: quas \textgreek{ἐπαγομήνας}[?] Graeci vocant, Aegyptii
prisca linqua \textsc{nesi}: Persae et Arabes \textarabic{}[?]
 \textsc{musterakath}.
\lnr{26}Aethiopes corrupto Graeco vocubulo etiamnum hodie \textsc{pagomen}
appellant.
\lnr{27}Cum ita annum 365 dierum constituissent, non dubitarunt
eum Solarem appellare: cum tamen non ignorarent, illi anno ad
perfectionem quadrantem diurni temporis deesse.
%
% 189
% {PDF page nr}{source page nr}{line nr}
\plnr{272}{189}{1}Quo neglecto in
120 annis per mensem integrum recedit caput anni ab epocha Solari.
\lnr{3}Verbi gratia: Caput anni, quod hodie fuerit in Kal. Mai Iuliani, id
post 120 annos, incidet in Kal. Apr. % End of line?
\lnr{4}Quare duodecim magnis mensibus
vertentibus disceditur a prima epocha, dies 360.
\lnr{5}Quod tempus
est annorum duodecies 120, hoc est, 1440.
\lnr{6}Post viginti annos receditur
dies 365.
\lnr{7}Et summa annorum fit 1460 anni aequabiles.
\lnr{7}Quo
intervallo caput anni huius totam ordinationis Iulianae feriem pervagatur.
\lnr{9}Id spatium annum magnum vocabant.
\lnr{9}Constat enim ex duodecim
magnis mensibus, qui singuli sunt annorum 120, vel decies duodecim,
et quinque magnis \textgreek{ἐπαγομήναις}[?], vel quinquies quater annis.
\lnr{12}Sed quod Cansorinus huiusmodi magnum annum \textgreek{κυνικὸν}[?]
 vocari ait,
if verum est, si spectes neomeniam Thoth.
\lnr{13}Quoties enim neomenia
Thoth in ortum caniculae incidit, is dicitur recte
 \textgreek{κυνικὸς ἐνιαυτός}[?].
\lnr{15}Idque iterum ut accidat, non paucis annorum centuriis opus est.
\lnr{16}Sed quod illud intervallum, quo Thoth redit ad ortum Caniculae,
unde profectus erat initio, constet annis 1460, id vero perabsurdum
est.
\lnr{18}Nam quem Thoth Canicularem dicit fuisse, Ulpio et
Brutio Praesente \textsc{coss.} is post 1460 annos Canicularis esse non
poterit, nimirum anno Christi 1599.
\lnr{20}Qui erit annus aequabilis 1460
absolutus ab illo Thoth Caniculari Censorini.
\lnr{21}Ut enim Solstitia et
aequinoctia intra illud tempus antevertunt circiter dies \rnum{xi}, ita ortus
siderum tardius oriuntur: quia annus sidereus; % Semicolon or comma?
 ut vulgus astronomorum
loquitur, tardior est anno Iuliano, Iulianus anno Tropico,
\lnr{24}Quare
si circa annum Christi 138 Caniculae ortus incidit in \rnum{xx} Iulii, anno
Christi 1598 incidet in \rnum{xxix} Iulii, si quidem Thebithio anni siderei
auctori credimus.
\lnr{27}Sed, inquies, nondum veteres illi \textgreek{τὴν ἀλήθειαν
ἐξηκριβώσανυτο}[?].
\lnr{28}Itaque magnus illorum annus, qui constabat ex 1460
Iulianis, erat potius magnus orbis anni Aegyptiaci aequabilis, in epocham
Iulianamrestituti, quam Caniculae ad Thith vagum redeuntis.
\lnr{31}Sed et Dio Cassius, hunc magnum annum in animo habens,
iniecta mentione intercalationis Besexti Iuliani, pueriliter admodum
de ea re pronuntiavit.
\lnr{33}Putat in illo intervallo 1460 annorum,
unum diem adiiciendum esse, praeter illum ordinarium, quod \textsc{Bisextum}
dicitur.
\lnr{35}Quod tantum abest, ut verum sit, ut in totidem
annis Iulianis undecim potius dies excreverint, quam ut unus desit.
\lnr{37}Verba Dionis:
 \textgreek{τὴν μέντοι μίαν τὴν ἐκ τῶν τεταρτημορίων συμπληρουμένην
διὰ τεσσάρων καὶ αὐτὸς ἐτῶν ἐισήγαγεν, ὤστε μηδὲν ἔτι τὰς ὤρας αὐτῶν,
πλὴν ἐλαχίστου, παραλλάττειν.}[?]
\lnr{39}\textgreek{ἐν γοῦν χιλίοις καὶ τετρακοσίοις, καὶ ἑξήκοντα,
καὶ ἑνὶ ἔτει μιᾶς ἄλλης ἡμέρας ἐμβολίμου δέονται.}[?]
% Cassius Dio: Historiae Romanae XLIII.26.3
% [ἑνὸς μηνὸς ἀφεῖλεν, ἐνήρμοσε.]
% τὴν μέντοι μίαν τὴν ἐκ τῶν τεταρτημορίων συμπληρουμένην
% διὰ πέμπτων[!] καὶ αὐτὸς ἐτῶν ἐσήγαγεν[!] ὥστε μηδὲν ἔτι τὰς ὥρας αὐτῶν
% πλὴν ἐλαχίστου παραλλάττειν:
% ἐν γοῦν χιλίοις καὶ τετρακοσίοις καὶ ἑξήκοντα
% καὶ ἑνὶ ἔτει μιᾶς ἄλλης ἡμέρας ἐμβολίμου δέονται.
% Translation by LacusCurtius:
% "The one day, however, which results from the fourths he introduced
% into every fourth year, so as to make the annual seasons no longer differ
% at all except in the slightest degree; at any rate in fourteen hundred and
% sixty-one years there is need of only one additional intercalary day.
\lnr{40}Vide quot
errores.
\lnr{41}Ait \textgreek{διὰ τεσσάρων ἐτῶν}[?] embolismum bisexti fieri.
\lnr{41}Quod Graeco
proprie est, tertio anno exacto.
%
% 190
% {PDF page nr}{source page nr}{line nr}
\plnr{273}{190}{1}Sic Herodotus significans Graecos singulis
bienniis exactis intercalare, dicit, \textgreek{ἕλληνες διὰ τρίτου ἔτεος ἐμβόλιμον
ἐπεμβάλλουσι}[?].
\lnr{3}Hoc est inter secundum annum exactum, et tertium
ineuntem.
\lnr{4}Deinde arbitratur praeter quadrantem diei aliquid
superesse.
\lnr{5}Postremo post annos mille quadringentos sexaginta unum,
celebrandam unius diei intercalationem.
\lnr{6}Nam si id, quod ille male
intellexit, verum erat, unus annus supra 1460, ad intercalationem
non pertinet.
\lnr{8}Sed post 1460 Iulianos exactos, anno proxime
ineunte, Thoth Aegyptiacus redit iterum in eam diem mensis Iuliani,
in qua ante 1461 annos aequabiles fuerat.
\lnr{10}Praeterea anni aequabiles
1461 sunt 1460 absoluti Iuliani.
\lnr{11}Ego sane in hoc loco Dionis
immorandum amplius esse non censuissem, nisi is locus magnum
virum Gazam in eundem errorem impulisset: qui eum, quanquam
non iisdem verbis, in suum libellum de Mensibus traduxit.
\lnr{14}Cuius
verba necessario apponenda esse iudicavi.
\lnr{15}\textgreek{δπλοῦσιν ὁι Αἰγύπτιοι[?] πρῶτοι
μησὶ χρήσασθαι καθ᾽ ἥλιον τριακονθημέροις δώδεκα, καὶ πέντε ἡμέρας ἐπαγαγέιν[?]
κατ᾽ ἐνιαυτὸν ἕκαστον, τό τε ἐπιτρέχον μόριον τὴς ἡμέρας ἐις ἐκπλήρωσιν
τοῦ ὅλου ὲνιαυτοῦ ἐκ περιόδων ἐτῶν ἀπολαβόντες συνθέσται μίαν ἡμέραν}[?].
\lnr{18}Aperte
ex Dione expiscatus est, ut vides.
\lnr{19}Et \textgreek{περίοδον πλειόνων ἐτῶν}[?] intelligit
mille quadrigentos sexaginta annos.
\lnr{20}Error sane in Philosopho potius,
quam historico castigandus.
\lnr{21}Non ergo \textgreek{μίαν ἡμέραν σηνθέσθαι}[?], sed
\textgreek{ἓν ἔτος}[?].
\lnr{22}Sed quid Iulio Firmico facias, qui ita in prooemio % procemio?
 operis sui
scribit?
\lnr{23}\emph{Quantis etiam cinversionibus maior ille, quem ferunt,
 persiceretur
annus, qui quinque has stellas, Lunam etiam, ac solem locis
suis, originibusque restituit, qui mille quadringentorum, et sexaginta
unius annorum circuitu terminatur.}
\lnr{26}An non plane \textgreek{ἀποκατάστασιν}[?] omnium
planetarum illo circuitu anni Canicularis fieri putat?
\lnr{27}Quid
imperitius potuit dici?
\lnr{28}Haec sunt, quae de anno aequabili Aegyptiaco
homines eruditi tradiderunt.
\lnr{29}Superest nunc, ut methodum Lunae in
hoc anno reperiamus, si qua periodus proxime cursum eius sideris
cum anno Aegyptiaco exaequare possit.
\lnr{31}Quae sane deprehensa est esse
viginti quinque annorum, ut quemanmodum enneadecaeteris in anno
Iuliano, ita \textgreek{εἰκοσιπενταετηρὶς}[?]
 in anno aequabili proxime ad praecisam
aequationem accdat.
\lnr{34}Enneadecaeteris Iuliana maiuscula est Lunari
hor.1,485.
\lnr{35}\textgreek{εἰκοσιπενταετηρὶς}[?]
 autem Aegyptica relinquit supra ratiocinia
Lunae, hor.1,123.
\lnr{36}Tanto praecisior est \textgreek{εἰκοσιπενταετηρὶς}[?] Enneadecaeteride.
\lnr{37}Modus anni Aegyptiaci, sive aequabilis, dies 365.
\lnr{37}Annus Lunaris
354, 8, 876.
\lnr{38}Differentia, dies 10, 15, 204.
\lnr{38}Duc igitur vicesies
quinquies hanc \textgreek{υπεροχὴν}[?].
\lnr{39}Prodeunt dies 265, hor. 19, scr. 780.
\lnr{39}Mensis
Lunaris 29, 12, 793.
\lnr{40}Quos multiplica novies.
\lnr{40}Quia tot embolismi sunt
in hac periodo.
\lnr{41}Producuntur dies 265, 18, 657.
\lnr{41}Deducantur de illo excessu
anni Aegyptiaci.
%
% 191
% {PDF page nr}{source page nr}{line nr}
\plnr{274}{191}{1}Remanent dies 0, 1, 123.
\lnr{1}Nunc videamus,
quomodo et quando intercalandum.
\lnr{2}





































% ==== End of text of Liber Tertius ===
