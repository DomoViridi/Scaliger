% !TEX root = ../de-emendatione-temporum-1629.tex
% !TEX TS-program = xelatex
% !TEX encoding = UTF-8 Unicode
% this template is specifically designed to be typeset with XeLaTeX;
% it will not work with other engines, such as pdfLaTeX

%%% Count out columns for fixed-width source font
% 000000011111111112222222222333333333344444444445555555555666666666677777777778
% 345678901234567890123456789012345678901234567890123456789012345678901234567890

\setheaders{\shorttitle{} Liber III}{\shortauthor{}}
\chapter{De Anno Aequabili Maiore}
%
% 188
% {PDF page nr}{source page nr}{line nr}
\plnr{271}{188}{1}In Astronomicis \textgreek{ὁμαλαὶ κινήσεις} vocantur
eae, quibus ex Anomaliarum Canonibus
competentes \textgreek{προσθαφαιρέσεις} adhibitae non
sunt.
\lnr{4}Ideo Illis motibus aut deest semper,
aut superest aliquid.
\lnr{5}Sic in ratione anni
eum annum aequabilem vocamus, cui propter
commodiorem usum aut abest, aut superest aliquid.
\lnr{8}Exempli gratia: Persico anno
quadrans de Solis ratiociniis deest.
\lnr{9}At
Graeco supra Lunae ratiocinia dies fere sex supersunt.
\lnr{10}Utrique hoc contigit
propter et commodiorem mensium in triginta dies tributionem,
et aequabilem dierum descriptionem: qua quidem aequabilitate mensium
Graeci volebant Lunae motum assequi.
\lnr{13}Sed hoc erat ultra fines
iaculum expedire.
\lnr{14}Plus enim dierum ad eam rem assumebatur, quam
modus anni postulabat.
\lnr{15}Contraria ratione Orientis nationes, cum
eodem anno antea uterentur, atque ad vitandum taedium embolismorum
eum castigare vellent, primum illum a Solis propius, quam
a Lunae rationibus abesse animaduerterunt: quod videlicet populares
temporum errores magis in Luna, propter illius sideris celeritatem,
deprehenduntur, quam in Sole: deinde convenit inter
eos, ut ex modulo anni sui, qui 360 dierum tantum erat,
partem 72 decerperent, atque eam anno appenderent.
\lnr{22}Divisa itaque
anni quantitate in 72, excerptae sunt dies \rnum{v}, et in calcem anni
post 360 dies reiectae: quas \textgreek{ἐπαγομήνας}[?] Graeci vocant, Aegyptii
prisca linqua \textsc{nesi}: Persae et Arabes \textarabic{}[?]
 \textsc{musterakath}.
\lnr{26}Aethiopes corrupto Graeco vocabulo etiamnum hodie \textsc{pagomen}
appellant.
\lnr{27}Cum ita annum 365 dierum constituissent, non dubitarunt
eum Solarem appellare: cum tamen non ignorarent, illi anno ad
perfectionem quadrantem diurni temporis deesse.
%
% 189
% {PDF page nr}{source page nr}{line nr}
\plnr{272}{189}{1}Quo neglecto in
120 annis per mensem integrum recedit caput anni ab epocha Solari.
\lnr{3}Verbi gratia: Caput anni, quod hodie fuerit in Kal. Mai Iuliani, id
post 120 annos, incidet in Kal. Apr.
% End of line? 1598 edition p180: yes.
% Lots more space between "Apr." and "Quare" in that edition.
\lnr{4}Quare duodecim magnis mensibus
vertentibus disceditur a prima epocha, dies 360.
\lnr{5}Quod tempus
est annorum duodecies 120, hoc est, 1440.
\lnr{6}Post viginti annos receditur
dies 365.
\lnr{7}Et summa annorum fit 1460 anni aequabiles.
\lnr{7}Quo
intervallo caput anni huius totam ordinationis Iulianae feriem pervagatur.
\lnr{9}Id spatium annum magnum vocabant.
\lnr{9}Constat enim ex duodecim
magnis mensibus, qui singuli sunt annorum 120, vel decies duodecim,
et quinque magnis \textgreek{ἐπαγομήναις}[?], vel quinquies quater annis.
\lnr{12}Sed quod Censorinus huiusmodi magnum annum \textgreek{κυνικὸν}[?]
 vocari ait,
id verum est, si spectes neomeniam Thoth.
\lnr{13}Quoties enim neomenia
Thoth in ortum caniculae incidit, is dicitur recte
 \textgreek{κυνικὸς ἐνιαυτός}[?].
\lnr{15}Idque iterum ut accidat, non paucis annorum centuriis opus est.
\lnr{16}Sed quod illud intervallum, quo Thoth redit ad ortum Caniculae,
unde profectus erat initio, constet annis 1460, id vero perabsurdum
est.
\lnr{18}Nam quem Thoth Canicularem dicit fuisse, Ulpio et
Brutio Praesente \textsc{coss.} is post 1460 annos Canicularis esse non
poterit, nimirum anno Christi 1599.
\lnr{20}Qui erit annus aequabilis 1460
absolutus ab illo Thoth Caniculari Censorini.
\lnr{21}Ut enim Solstitia et
aequinoctia intra illud tempus antevertunt circiter dies \rnum{xi}, ita ortus
siderum tardius oriuntur: quia annus sidereus,
% Semicolon or comma? 1598 edition: comma.
 ut vulgus astronomorum
loquitur, tardior est anno Iuliano, Iulianus anno Tropico,
\lnr{24}Quare
si circa annum Christi 138 Caniculae ortus incidit in \rnum{xx} Iulii, anno
Christi 1598 incidet in \rnum{xxix} Iulii, si quidem Thebithio anni siderei
auctori credimus.
\lnr{27}Sed, inquies, nondum veteres illi \textgreek{τὴν ἀλήθειαν
ἐξηκριβώσανυτο}[?].
\lnr{28}Itaque magnus illorum annus, qui constabat ex 1460
Iulianis, erat potius magnus orbis anni Aegyptiaci aequabilis, in epocham
Iulianam restituti, quam Caniculae ad Thoth vagum redeuntis.
\lnr{31}Sed et Dio Cassius, hunc magnum annum in animo habens,
iniecta mentione intercalationis Bisexti Iuliani, pueriliter admodum
de ea re pronuntiavit.
\lnr{33}Putat in illo intervallo 1460 annorum,
unum diem adiiciendum esse, praeter illum ordinarium, quod \textsc{Bisextum}
dicitur.
\lnr{35}Quod tantum abest, ut verum sit, ut in totidem
annis Iulianis undecim potius dies excreverint, quam ut unus desit.
\lnr{37}Verba Dionis:
 \textgreek{τὴν μέντοι μίαν τὴν ἐκ τῶν τεταρτημορίων συμπληρουμένην
διὰ τεσσάρων καὶ αὐτὸς ἐτῶν ἐισήγαγεν, ὤστε μηδὲν ἔτι τὰς ὤρας αὐτῶν,
πλὴν ἐλαχίστου, παραλλάττειν.}[?]
\lnr{39}\textgreek{ἐν γοῦν χιλίοις καὶ τετρακοσίοις, καὶ ἑξήκοντα,
καὶ ἑνὶ ἔτει μιᾶς ἄλλης ἡμέρας ἐμβολίμου δέονται.}[?]
% Cassius Dio: Historiae Romanae XLIII.26.3
% [ἑνὸς μηνὸς ἀφεῖλεν, ἐνήρμοσε.]
% τὴν μέντοι μίαν τὴν ἐκ τῶν τεταρτημορίων συμπληρουμένην
% διὰ πέμπτων[!] καὶ αὐτὸς ἐτῶν ἐσήγαγεν[!] ὥστε μηδὲν ἔτι τὰς ὥρας αὐτῶν
% πλὴν ἐλαχίστου παραλλάττειν:
% ἐν γοῦν χιλίοις καὶ τετρακοσίοις καὶ ἑξήκοντα
% καὶ ἑνὶ ἔτει μιᾶς ἄλλης ἡμέρας ἐμβολίμου δέονται.
% Translation by LacusCurtius:
% "The one day, however, which results from the fourths he introduced
% into every fourth year, so as to make the annual seasons no longer differ
% at all except in the slightest degree; at any rate in fourteen hundred and
% sixty-one years there is need of only one additional intercalary day.
\lnr{40}Vide quot
errores.
\lnr{41}Ait \textgreek{διὰ τεσσάρων ἐτῶν}[?] embolismum bisexti fieri.
\lnr{41}Quod Graeco
proprie est, tertio anno exacto.
%
% 190
% {PDF page nr}{source page nr}{line nr}
\plnr{273}{190}{1}Sic Herodotus significans Graecos singulis
bienniis exactis intercalare, dicit, \textgreek{ἕλληνες διὰ τρίτου ἔτεος ἐμβόλιμον
ἐπεμβάλλουσι}[?].
\lnr{3}Hoc est inter secundum annum exactum, et tertium
ineuntem.
\lnr{4}Deinde arbitratur praeter quadrantem diei aliquid
superesse.
\lnr{5}Postremo post annos mille quadringentos sexaginta unum,
celebrandam unius diei intercalationem.
\lnr{6}Nam si id, quod ille male
intellexit, verum erat, unus annus supra 1460, ad intercalationem
non pertinet.
\lnr{8}Sed post 1460 Iulianos exactos, anno proxime
ineunte, Thoth Aegyptiacus redit iterum in eam diem mensis Iuliani,
in qua ante 1461 annos aequabiles fuerat.
\lnr{10}Praeterea anni aequabiles
1461 sunt 1460 absoluti Iuliani.
\lnr{11}Ego sane in hoc loco Dionis
immorandum amplius esse non censuissem, nisi is locus magnum
virum Gazam in eundem errorem impulisset: qui eum, quanquam
non iisdem verbis, in suum libellum de Mensibus traduxit.
\lnr{14}Cuius
verba necessario apponenda esse iudicavi.
\lnr{15}\textgreek{δοκοῦσιν ὁν Αἰγύπτιοι πρῶτοι
μησὶ χρήσασθαι καθ᾽ ἥλιον τριακονθημέροις δώδεκα, καὶ πέντε ἡμέρας ἐπαγαγέιν
κατ᾽ ἐνιαυτὸν ἕκαστον, τό, τε ἐπιτρέχον μόριον τὴς ἡμέρας ἐις ἐκπλήρωσιν
τοῦ ὅλου ὲνιαυτοῦ ἐκ περιόδων ἐτῶν ἀπολαβόντες συνθέσται μίαν ἡμέραν}[?].
\lnr{18}Aperte
ex Dione expiscatus est, ut vides.
\lnr{19}Et \textgreek{περίοδον πλειόνων ἐτῶν}[?] intelligit
mille quadrigentos sexaginta annos.
\lnr{20}Error sane in Philosopho potius,
quam historico castigandus.
\lnr{21}Non ergo \textgreek{μίαν ἡμέραν σηνθέσθαι}[?], sed
\textgreek{ἓν ἔτος}[?].
\lnr{22}Sed quid Iulio Firmico facias, qui ita in prooemio
% procemio? 1598 ed: "oe" ligature
 operis sui
scribit?
\lnr{23}\textit{Quantis etiam conversionibus maior ille, quem ferunt,
 persiceretur
annus, qui quinque has stellas, Lunam etiam, ac solem locis
suis, originibusque restituit, qui mille quadringentorum, et sexaginta
unius annorum circuitu terminatur.}
\lnr{26}An non plane \textgreek{ἀποκατάστασιν}[?] omnium
planetarum illo circuitu anni Canicularis fieri putat?
\lnr{27}Quid
imperitius potuit dici?
\lnr{28}Haec sunt, quae de anno aequabili Aegyptiaco
homines eruditi tradiderunt.
\lnr{29}Superest nunc, ut methodum Lunae in
hoc anno reperiamus, si qua periodus proxime cursum eius sideris
cum anno Aegyptiaco exaequare possit.
\lnr{31}Quae sane deprehensa est esse
viginti quinque annorum, ut quemanmodum enneadecaeteris in anno
Iuliano, ita \textgreek{εἰκοσιπενταετηρὶς}[?]
 in anno aequabili proxime ad praecisam
aequationem accdat.
\lnr{34}Enneadecaeteris Iuliana maiuscula est Lunari
hor. 1,485.
\lnr{35}\textgreek{εἰκοσιπενταετηρὶς}[?]
 autem Aegyptica relinquit supra ratiocinia
Lunae, hor. 1,123.
\lnr{36}Tanto praecisior est \textgreek{εἰκοσιπενταετηρὶς}[?] Enneadecaeteride.
\lnr{37}Modus anni Aegyptiaci, sive aequabilis, dies 365.
\lnr{37}Annus Lunaris
354, 8, 876.
\lnr{38}Differentia, dies 10, 15, 204.
\lnr{38}Duc igitur vicesies
quinquies hanc \textgreek{ὑπεροχὴν}[?].
\lnr{39}Prodeunt dies 265, hor. 19, scr. 780.
\lnr{39}Mensis
Lunaris 29, 12, 793.
\lnr{40}Quos multiplica novies.
\lnr{40}Quia tot embolismi sunt
in hac periodo.
\lnr{41}Producuntur dies 265, 18, 657.
\lnr{41}Deducantur de illo excessu
anni Aegyptiaci.
%
% 191
% {PDF page nr}{source page nr}{line nr}
\plnr{274}{191}{1}Remanent dies 0, 1, 123.
\lnr{1}Nunc videamus,
quomodo et quando intercalandum.
\lnr{2}Tres \textgreek{ὑπεροχαὶ}[?] anni Aegyptiaci
fiunt dies 31, 21, 612.
\lnr{3}Deducto mense Lunari, remanent epactae anni
tertii, 2, 8, 899.
\lnr{4}Quae cum triplicato iterum excessu, abiecto mense
Lunari, relinquunt epactas anni sexti, 4, 17, 718.
\lnr{5}Tertio triplicatur
excessus: qui cum epactis anni sexti, deducto mense Lunari,
relinquit epactas anni noni, 7, 2, 537.
\lnr{7}Quas si cum triplicato
excessu iunxero, deducto mense Lunari, epactae, quae hinc prodibunt,
nimium quantum excedent.
\lnr{9}Contenti simus igitur duplicato
excessu, 21, 6, 408.
\lnr{10}Cum epactis proxime praecedentis embolismi
fiunt dies 28, 8, 945.
\lnr{11}En Lunae rationes maiores anno Aegyptiaco.
\lnr{12}Nam mensis Lunaris excedit illam summam, ut scis.
\lnr{12}Deducatur
igitur nunc summa anni Aegyptiaci, de summa Lunari.
\lnr{13}Relinquuntur
epactae undecimi embolismi.
\lnr{14}Item triplietur excessus, neque iungatur
superioribus epactis: quia iam non anni Aegyptiaci excessus est,
sed Lunaris.
\lnr{16}Ex triplicato excessu Solis prodeunt epactae 2, 8, 899.
\lnr{17}Et quia hae epactae anni Aegyptiaci maiores sunt proximis epactis
 Lunaribus,
aufer Lunares 1, 3, 928, ab Aegyptiacis 2, 8, 899.
\lnr{18}Exeunt
1, 4, 1051, epactae 14 embolismi.
\lnr{19}Triplica iterum excessum.
\lnr{19}Iunge
epactas proximas.
\lnr{20}Abiice mensem Lunarem.
\lnr{20}Habes 3, 13, 870,
epactas anni 17.
\lnr{21}Triplica excessum.
\lnr{21}Iunge epactas.
\lnr{21}Abiice mensem.
\lnr{22}Prodeunt epactae vicesimi embolismi, 5, 22, 689.
\lnr{22}Iunge triplicato
excessui.
\lnr{23}Abiecto mense Lunari, remanent epactae vicesimi tertii
embolismi, 8, 7, 508.
\lnr{24}Quae cum duplicato excessu, deducta syzygiae
unius quantitate, relinquunt in vicesimo quinto anno differentiam
anni Aegyptiaci, et Lunaris, dies 0, 1, 123.
\lnr{26}Novies igitur intercalatur:
Ternis quidem annis, tertio, sexto, nono, quartodecimo, decimo
septimo, vicesimo, vicesimo tertio: Binis autem, undecimo,
et ultimo.
\lnr{29}Rursus quemadmodum ex Solis excessu supra Lunam in
cyclo enneadecaeterico epactae formantur ad indicandam aetatem
Lunae: ita etiam in hac periodo ex anni Aegyptiaci excessu Epactae
produci possunt.
\lnr{32}Epactae igitur primi anni erit excessus ipse anni
Aegyptiaci, 10, 15, 204.
\lnr{33}Secundi anni, duplum illarum 21, 6, 408.
\lnr{34}Tertii triplum.
\lnr{34}Et sic deinceps.
\lnr{34}Hae epactae in anno Aegyptiaco eum
usum habent, quem illae nostrae in anno Iuliano ad aetatem Lunae, ut
Computatores nostri loquuntur, ad \textgreek{ποστιαίαν}[?],
 ut Graeci, indicandam:
nempe ut Kalendarum Regulares cum die proposita mensis, et cum
epactis simul, abiectis tricenariis, cum opus erit,
 ostendant \textgreek{ποστιαίαυ τῆς
σελήνης}[?].
\lnr{39}Sed hoc differunt Regulares Aegyptiaci a regularibus Iulianis:
quod omnium Kalendarum Regulares Iuliani in methodo epactarum
adhibeantur: ut, verbi gratia, in Februario, qui est duodecimus
mensis Paschalis, cum investigatur aetas Lunae, duodecim regulares
mensium adhibentur: tot nimirum, quot sunt Kalendae a
Martio.
%
% 192
% {PDF page nr}{source page nr}{line nr}
\plnr{275}{192}{3}At in anno aequabili alterni regulares adiiciuntur.
\lnr{3}Nam omnium
mensium paris numeri nulli sunt regulares.
\lnr{4}Sed illorum vicem
antecedentium mensium regulares funguntur.
\lnr{5}Exempli gratia: Paophi
est secundus mensis, ideo paris numeri.
\lnr{6}Proinde regularis unus
erit Thoth praecedentis, et Paophi sequentis.
\lnr{7}Volo novilunium illi
competens investigare in anno, in quo epactae Lunares erunt \rnum{v}.
\lnr{8}Non
binos Regulares, quad binae Kalendae fluxerint, sed unos tantum
assumo: et invenio novilunium in \rnum{xxiiii} Paophi.
\lnr{10}Ratio huius haec
est: Novilunium in mensibus \textgreek{τριακονθημέροις}[?]
 binis in eadem die semper
deprehenditur, quamquam semisse die prius praeveritur a sequenti.
\lnr{13}Primus itaque et secundus mensis habent novilunium in eadem
die: tertius item et quartus.
\lnr{14}Et sic deinceps bini menses in
eodem die conficiunt novilunia.
\lnr{15}Quod non \textgreek{ἀκριβῶς}[?], sed \textgreek{πλατικῶς}[?]
dictum velim, quatenus patitur popularium Epactarum ratio.
\lnr{16}Nam
hac in re in anno Iuliano peccatur.
\lnr{17}Sunto epactae Lunares 25.
% Sentence starts with roman numeral.
\lnr{17}\rnum{xxix}dies
Aprilis erit \rnum{xxvi} Lunae.
\lnr{18}Et \rnum{xxx}dies eiusdem erit \rnum{xxvii} Lunae.
\lnr{19}Quare Kal. Mai Luna erit \rnum{xxviii}, ut docent Epactae.
\lnr{19}At assumptis
regularibus trinarum Kalendarum, prima dies Mai erit \rnum{xxix}.
\lnr{20}Fallit
ergo regula epactarum in mensibus illis, quos praecedunt menses
\textgreek{τριακονθήμεροι}[?].
\lnr{22}Propter regulares igitur fit \textgreek{ὑπέρβατοσ}[?] una dies.
\lnr{22}Cui
errori occurrendum est.
\lnr{23}Nam antecedente mense tricenario non videbantur
assumendi regulares sequentis.
\lnr{24}In anno igitur Aegyptiaco
menses, qui sunt pari numero, nullos regulares habent.
\lnr{25}In mensibus
vero imparibus, quot neomeniae fluxerint imparium, tot regulares
assumendi.
% No capital on 'ita'
\lnr{27}Ita ut numerus mensium dimidiandus sit, et productum
sint Regulares.
\lnr{28}Hic est usus epactarum, quarum diagramma
infra subiecimus.
\lnr{29}Cuius duplex usus.
\lnr{29}Nam epactae eatenus nomen
hoc retinent quandiu descendunt.
\lnr{30}Ascendentes autem sunt termini
noviluniorum.
\lnr{31}Exemplum: Epactae primi anni \textgreek{εἰκοστπενταετηρίδος}[?] in annis
Nabonassari sunt, 5, 22, 689.
\lnr{32}Quas reperies in 20 anno diagrammatis.
\lnr{33}Sequentis anni epactae sunt in 21 anno diagrammatis, nempe,
16, 13, 893, et ita deinceps.
\lnr{34}Contra terminus noviluniorum primi anni
Nabonassari est in eo anno, qui ascendens proxime praecedit annum
primum Epactarum, nempe in anno 19 Diagrammatis: in
quo notatus est terminus 24, 20, 198.
\lnr{37}Dies igitur \rnum{v} Epactarum,
cum uno regulari Thoth constituit novilunium in 24 die Thoth,
qui est Terminus novilunii.
\lnr{39}Eadem methodus in aliis.
\lnr{39}Quae facilima
est, dummodo Epactas scias descendere, Terminos autem ascendere:
item proximum annum ascendentem a primis Epactis, esse primum
Terminorum.
%
% 193
% {PDF page nr}{source page nr}{line nr}
\plnr{276}{193}{1}Haec ratio expedita erat, si epactae praecise essent
dierum certorum, neque horas et scrupulos appendices haberent.
\lnr{2}Quo fit, ut dies tantum % tantū; 1598 ed.: tantum
 terminorum, exclusis horis, et scrupulis progressu
temporis novilunii fines non attingant.
\lnr{4}Nam in 23 periodis Luna
antevertit cyclum anni aequabilis die uno.
\lnr{5}Itaque castigatio adhibenda,
quoties summa annorum Aegyptiorum, aut Armeniorum,
aut Persicorum excedit 500 annos, aut paulo amplius.
\lnr{7}Melius igitur
per characterem et feriam diei methodum
Lunae tractabis.
%
\begin{table}[p]
  % define table height
  \newcommand{\tabh}{\textheight}
%  \setlength{\tabcolsep}{0.0ex}
%  \centering
  \begin{tabular}{r @{\hspace{0.02\textwidth}} r}
%  \resizebox{0.45\textwidth}{!}{%
    \begin{minipage}[][\tabh][t]{0.45\textwidth}
      %%% Liber 3 p193, PDF 276
%%
%%% Count out columns for fixed-width source font
% 000000011111111112222222222333333333344444444445555555555666666666677777777778
% 345678901234567890123456789012345678901234567890123456789012345678901234567890
%
\begin{tabnums} % Select monospaced numbers
%% Select a general font size (uncomment one from the list)
%\tiny
%\scriptsize
%\footnotesize
%\small
\normalsize
%% Center the whole table left-right
\centering
%% Modify separation between columns
\setlength{\tabcolsep}{1.0ex}
%% Modify distance between rows
%\renewcommand{\arraystretch}{1.2}
%
%% Width of a column
\newcommand{\cwd}{3.2em}
%% Define reference symbols
\newcommand{\da}{{\tiny †}}
\newcommand{\db}{{\scriptsize o}}
%% The angle with which to slant
\newcommand{\ang}{90}
%% Header text size: row above row above bottom row
\newcommand{\hsc}[1]{\small{#1}}
%% Header text size: row above bottom row
\newcommand{\hsb}[1]{\scriptsize{#1}}
%% Header text size: bottom row
\newcommand{\hsa}[1]{\tiny{#1}}
%% Generate the column headers
\newcommand{\hdrC}{%
  \multicolumn{6}{c}{\hsc{Periodus prior}} &
  &
  \multicolumn{6}{c}{\hsc{Periodus altera}}  
}
%
\newcommand{\hdrB}{%
  \multicolumn{4}{c}{\hsb{Pars Prior.}} &
  &
  \multicolumn{3}{c}{\hsb{Pars Poster.}}  
}
%
\newcommand{\hdrA}{%
  \ch{888}{\hsa{Dies collecti}} &
  \ch{81}{\hsa{Feria}}&
  \ch{88}{\hsa{Horae}} &
  \ch{1888}{\hsa{Scrup.}} &
  &
  \ch{81}{\hsa{Dies}} &
  \ch{88}{\hsa{Horae}} &
  \ch{1888}{\hsa{Scrup.}}
}
%
\newcommand{\hdrs}{%
 ~ & \hdrB \\
\cmidrule(lr){2-5} \cmidrule(lr){7-9}
 ~ & \hdrA \\
}
%
\begin{tabular}[c]{@{} r rrrr c rrr @{}}
\toprule
\multicolumn{9}{c}{\Large\textsc{Tabella Mensium}} \\
\toprule
\hdrs % Column headers from the above definition
\midrule
%%
 1 &  29 & 1 & 12 &  793 && 0 & 11 & 287 \\
 2 &  59 & 3 &  1 &  506 && 0 & 22 & 574 \\
 3 &  88 & 4 & 14 &  219 && 1 &  9 & 861 \\
 4 & 118 & 6 &  2 & 1012 && 1 & 21 &  68 \\
 5 & 147 & 7 & 15 &  725 && 2 &  8 & 355 \\
 6 & 177 & 2 &  4 &  438 && 2 & 19 & 642 \\
 7 & 206 & 3 & 17 &  151 && 3 &  6 & 929 \\
 8 & 236 & 5 &  5 &  944 && 3 & 18 & 136 \\
 9 & 265 & 6 & 18 &  657 && 4 &  5 & 423 \\
10 & 295 & 1 &  7 &  370 && 4 & 16 & 710 \\
11 & 324 & 2 & 20 &   83 && 5 &  3 & 997 \\
12 & 354 & 4 &  8 &  876 && 5 & 15 & 204 \\
13 & 383 & 5 & 21 &  589 && 6 &  2 & 491 \\
\bottomrule
\end{tabular}
\caption{Tabella Mensium}
\label{tab:p193}
\end{tabnums}

      \bigskip
      %%% Liber 3 p193, PDF 276
%%
%%% Count out columns for fixed-width source font
% 000000011111111112222222222333333333344444444445555555555666666666677777777778
% 345678901234567890123456789012345678901234567890123456789012345678901234567890
%
\begin{tabnums} % Select monospaced numbers
%% Select a general font size (uncomment one from the list)
%\tiny
%\scriptsize
\footnotesize
%\small
%\normalsize
%% Center the whole table left-right
\centering
%% Modify separation between columns
\setlength{\tabcolsep}{1.0ex}
%% Modify distance between rows
%\renewcommand{\arraystretch}{1.2}
%
%% Width of a column
\newcommand{\cwd}{3.2em}
%% Define reference symbols
\newcommand{\da}{{\tiny †}}
\newcommand{\db}{{\scriptsize o}}
%% The angle with which to slant
\newcommand{\ang}{90}
%% Header text size: row above row above bottom row
\newcommand{\hsc}[1]{\small{#1}}
%% Header text size: row above bottom row
\newcommand{\hsb}[1]{\scriptsize{#1}}
%% Header text size: bottom row
\newcommand{\hsa}[1]{\tiny{#1}}
%% Generate the column headers
%
\newcommand{\hdrA}{%
  \ch{888}{\hsa{Anni collecti}} &
  \ch{888}{\hsa{Dies.}}&
  \ch{888}{\hsa{Hor.}} &
  \ch{1888}{\hsa{Scrup.}} &
  &
  \ch{888}{\hsa{Feria.}} &
  \ch{888}{\hsa{Hor.}} &
  \ch{1888}{\hsa{Scrup.}}
}
%
\newcommand{\hdrs}{%
\hdrA \\
}
%
\begin{tabular}[c]{@{} r rrr c rrr @{}}
\toprule
\multicolumn{8}{c}{\Large\textsc{Tabella Annorum}} \\
\multicolumn{8}{c}{\large\textsc{Collectorum}} \\
\toprule
\hdrs % Column headers from the above definition
\midrule
%%
  25 & 0 &  1 &  123 && 3 & 22 &  957 \\
  50 & 0 &  2 &  246 && 7 & 21 &  834 \\
  75 & 0 &  3 &  369 && 4 & 20 &  711 \\
 100 & 0 &  4 &  492 && 1 & 19 &  588 \\
 125 & 0 &  5 &  615 && 5 & 18 &  465 \\
 150 & 0 &  6 &  738 && 2 & 17 &  342 \\
 175 & 0 &  7 &  861 && 6 & 16 &  219 \\
 200 & 0 &  8 &  984 && 3 & 15 &   96 \\
 225 & 0 & 10 &   27 && 7 & 13 & 1053 \\
 250 & 0 & 11 &  150 && 4 & 12 &  930 \\
 500 & 0 & 22 &  300 && 2 &  1 &  780 \\
 750 & 1 &  9 &  450 && 6 & 14 &  630 \\
1000 & 1 & 20 &  600 && 4 &  3 &  480 \\
1250 & 2 &  7 &  750 && 1 & 16 &  330 \\
1500 & 2 & 18 &  900 && 6 &  5 &  180 \\
1750 & 3 &  5 & 1050 && 3 & 18 &   30 \\
2000 & 3 & 17 &  120 && 1 &  6 &  960 \\
2250 & 4 &  4 &  270 && 5 & 19 &  810 \\
2500 & 4 & 15 &  420 && 3 &  8 &  660 \\
5000 & 5 &  2 &  570 && 6 & 17 &  240 \\
\bottomrule
\end{tabular}
\caption{Annorum Collectorum}
\label{tab:p193c}
\end{tabnums}

    \end{minipage}
%  }
&
%  \resizebox{0.45\textwidth}{!}{%
    \begin{minipage}[][\tabh][t]{0.53\textwidth}
      %%% Liber 3 p193, PDF 276
%%
%%% Count out columns for fixed-width source font
% 000000011111111112222222222333333333344444444445555555555666666666677777777778
% 345678901234567890123456789012345678901234567890123456789012345678901234567890
%
\begin{tabnums} % Select monospaced numbers
%% Select a general font size (uncomment one from the list)
%\tiny
%\scriptsize
%\footnotesize
%\small
\normalsize
%% Center the whole table left-right
\centering
%% Modify separation between columns
\setlength{\tabcolsep}{1.0ex}
%% Modify distance between rows
%\renewcommand{\arraystretch}{1.2}
%
%% Width of a column
\newcommand{\cwd}{3.2em}
%% Define reference symbols
\newcommand{\da}{{\tiny †}}
\newcommand{\db}{{\scriptsize o}}
%% The angle with which to slant
\newcommand{\ang}{90}
%% Header text size: row above row above bottom row
\newcommand{\hsc}[1]{\small{#1}}
%% Header text size: row above bottom row
\newcommand{\hsb}[1]{\scriptsize{#1}}
%% Header text size: bottom row
\newcommand{\hsa}[1]{\tiny{#1}}
%% Generate the column headers
%
\newcommand{\hdrB}{%
  ~ &
  \multicolumn{3}{c}{\hsb{Epactae.}} &
  &
  \multicolumn{3}{c}{\hsb{Novilunia.}}  
}
%
\newcommand{\hdrA}{%
  \ch{888}{\hsa{Anni expansi}} &
  \ch{81}{\hsa{Epact.}}&
  \ch{88}{\hsa{Hor.}} &
  \ch{1888}{\hsa{Scrup.}} &
  &
  \ch{81}{\hsa{Feria.}} &
  \ch{88}{\hsa{Hor.}} &
  \ch{1888}{\hsa{Scrup.}}
}
%
\newcommand{\hdrs}{%
\hdrB \\
\cmidrule(lr){2-4} \cmidrule(lr){6-8}
\hdrA \\
}
%
\begin{tabular}[c]{@{} r rrr c rrr l@{}}
\toprule
\multicolumn{9}{c}{\Large\textsc{Tabella Annorum}} \\
\multicolumn{9}{c}{\large\textsc{Expansorum}} \\
\toprule
\hdrs % Column headers from the above definition
\midrule
%%
 1 & 10 & 15 &  204 && 4 &  8 &  876 & ~\\
 2 & 21 &  6 &  408 && 1 & 17 &  672 & ~\\
 3 &  2 &  8 &  899 && 7 & 15 &  181 & \da \\
 4 & 13 &  0 &   23 && 4 & 23 & 1057 & ~\\
 5 & 23 & 15 &  227 && 2 &  8 &  853 & ~\\
 6 &  4 & 17 &  718 && 1 &  6 &  362 & \da \\
 7 & 15 &  8 &  922 && 5 & 15 &  158 & ~\\
 8 & 26 &  0 &   46 && 2 & 23 & 1034 & ~\\
 9 &  7 &  2 &  537 && 1 & 21 &  543 & \da \\
10 & 17 & 17 &  741 && 6 &  6 &  339 & ~\\
11 & 28 &  8 &  945 && 5 &  3 &  928 & \da \\
12 &  9 & 11 &  356 && 2 & 12 &  724 & ~\\
13 & 20 &  2 &  560 && 6 & 21 &  520 & ~\\
14 &  1 &  4 & 1051 && 5 & 19 &   29 & \da \\
15 & 11 & 20 &  175 && 3 &  3 &  905 & ~\\
16 & 22 & 11 &  379 && 7 & 12 &  701 & ~\\
17 &  3 & 13 &  870 && 6 & 10 &  210 & \da \\
18 & 14 &  4 & 1074 && 3 & 19 &    6 & ~\\
19 & 24 & 20 &  198 && 1 &  3 &  882 & ~\\
20 &  5 & 22 &  689 && 7 &  1 &  391 & \da \\
21 & 16 & 13 &  893 && 4 & 10 &  187 & ~\\
22 & 27 &  5 &   17 && 1 & 18 & 1063 & ~\\
23 &  8 &  7 &  508 && 7 & 16 &  572 & \da \\
24 & 18 & 22 &  712 && 5 &  1 &  368 & ~\\
25 &  0 &  1 &  123 && 3 & 22 &  957 & \da \\
\bottomrule
\addlinespace[5pt]
 & \multicolumn{3}{l}{\footnotesize\super{†}Emb.}
\end{tabular}
\caption{Tabella Mensium}
\label{tab:p193}
\end{tabnums}

    \end{minipage}
%  }
\\
%    \addlinespace[0.85in] % force a lot of space to make room for the captions
  \end{tabular}
\end{table}
%
\lnr{9}Cuius rei
gratia Tabulas duplices tibi confecimus
in mensibus, annis expansis
et collectis per periodos suas.\super{p.~\pageref{tab:p193a}}
\lnr{13}Cuius usus duplex, ut et ipsa duplex.
\lnr{14}Aut enim per partem anteriorem,
aut per posteriorem, operari
potes.
%
% 194
% {PDF page nr}{source page nr}{line nr}
\plnr{277}{194}{1}Et pars quidem anterior, cum collecta fuerit, auferenda
est de 30 diebus, ut suo loco dicetur.
\lnr{2}Residuum erit exactissimum
novilunium, in horis, et scrupulis: ut Abacus Astronomicus certius
daturus non sit.
\lnr{4}Contraria methodus in posteriore parte.
\lnr{4}Nam
per adiectionem, non per detractionem res tractatur.
\lnr{5}Quod suo quidque
loco explicandum relinquimus.
%
%====
\section{De Anno Aegyptiaco}
%
\lnr{7}Antiquissima, et simplicissima anni forma, ac popularibus
temporibus accommodatissima, est ea, quae in tricenarios
numeros tribuitur.
\lnr{9}Nam sane antiquitus, praesertim apud
Aegyptios, annus constabat tantum diebus trecentis sexaginta.
\lnr{10}Qui numerus
propter aequabilitatem divisionis est aptissimus.
\lnr{11}Aegyptii vero
Hierophantae, cum scirent anno populari ad perfectionem deesse dies
integros quinque ac quadrantem, eum annum perfectum siquando
significare vellent, Serpentem in orbem ac circulum convolutum,
cauda ore admorsa pingebant in suis Hieroglyphicis monumentis,
tanquam Sol perfectum et absolutum Zodiaci curriculum non conficiat,
si quinque illae dies appendices desint.
\lnr{17}(Aliter apud Eusebium
libro primo \textgreek{προπαρασκευῆς}[?];
% Greek: "Preparation"
\textgreek{ἔτι μὴν οἱ Αἰγύπτιοι ἀπὸ τὴς αὐτῆς ἐννοίας
τὸν κόσμον γράφοντες, περιφερῆ κύκλον ἀεροειδῆ καὶ πυρωπὸν χαράσσουσιν, καὶ μέσον
τεταμένον ὄφιν ἱερακόμορφον. καὶ ἔστι τὸ πᾶν σχῆμα ὡς τὸ παρ᾽ ἡμῖν Θῆτα.
τὸν μὲν κύκλον, κόσμον μηνύοντες. τὸν δὲ μέσον ὄφιν συνεκτικὸν τούτου ἀγαθὸν
δαίμονα σημαίνοντες.}[?])
% Eusebius of Caesarea: Preparation for the Gospel (Εὐαγγελικὴ προπαρασκευή)
% (Praeparatio evangelica). Book 1; Chapter 10: The theology of the Phoenecians.
% 4th paragraph from the end.
% Ἔτι μὴν οἱ Αἰγύπτιοι ἀπὸ τῆς αὐτῆς ἐννοίας
% τὸν κόσμον γράφοντες περιφερῆ κύκλον ἀεροειδῆ καὶ πυρωπὸν χαράσσουσιν καὶ μέσα
% τεταμένον ὄφιν ἱερακόμορφον (καὶ ἔστι τὸ πᾶν σχῆμα ὡς τὸ παρ' ἡμῖν Θῆτα),
% τὸν μὲν κύκλον κόσμον μηνύοντες, τὸν δὲ μέσον ὄφιν συνεκτικὸν τούτου Ἀγαθὸν
% Δαίμονα σημαίνοντες.
% Translation by E. H. Gifford (1903)
% "Moreover the Egyptians, describing the world from the same idea, engrave the
% circumference of a circle, of the colour of the sky and of fire, and a hawk-
% shaped serpent stretched across the middle of it, and the whole shape is like 
% our Theta (θ), representing the circle as the world, and signifying by the
% serpent which connects it in the middle the good daemon."
\lnr{22}Eum Serpentem \textgreek{νεισὶ}[?] vocabant: quodmodo hodie
Coptitae, et Aegyptii vetustissimi Christiani
 \textgreek{τὰς ὲπαγομένας}[?] vocare
solent lingua prisca Aegyptiaca, qua sacros utriusque Testamenti
libros conscriptos habent, et sacra itidem in templis obeunt.
\lnr{26}Inuenio enim illas dies \textarabic{}[?] ab illis, Arabice denotari.
\lnr{26}Nam Horus
Apollo Serpentem illum \textgreek{κοσμοειδῶς ἐσχηματισμείον}[?],
 quo annum, mundum,
Regem, et alia significabant, scribit vocari \textgreek{Μεισὶ}[?].
\lnr{28}Sed mendum
esse librarii puto.
\lnr{29}Nam, ut dixi, finem anni Solaris etiamnum
hodie Aegyptii Christiani Nesi vocant.
\lnr{30}Eum autem \textsc{nesi} sive \textsc{Nisi}
his elegantissimis versibus describit Claudianus Panegyrico
 \ruleover{\rnum{ii}} in
Stiliconem.
\begin{verse}
\textit{Est ignota procul, nostraeque impervia menti\\
  Vix adeunda Deis, annorum squalida mater,\\
  Immensi spelunca \textsc{aevi}, quae tempora vasto\\
  Suppeditat, revocatque sinu. Complectitur antrum,\\
  Omnia qui placido consumit numine \textsc{serpens}:\\
  Perpetuumque viret squamis, caudamque reducto\\
  Ore vorat, tacito relegens exordia lapsu.
}
% Claudius Claudianus: De Consulatu Stilichonis
% Claudian: On Stilicho's Consulship
% Lines 424-430
% Immensi/Immensis/Inmensi
% reducto -> reductam
\end{verse}
%
% 195
% {PDF page nr}{source page nr}{line nr}
\plnr{278}{195}{1}Porro nomen \textgreek{ἐπαγομένων}[?] dictum est
 \textgreek{κατὰ στέρησιν}[?], tanquam olim annus
360 dierum simpliciter fuisset.
\lnr{2}Quod, ut dixi, Hierophantae
ipsi non solum in libris suis scriptum habebant, sed et fabulam origini
et causae \textgreek{τῶν ἐπαγομένων}[?] adiiciebant: Mercurium scilicet alea
cum Luna ludentem vicisse, et septuagesimam secundam partem
anni ab ea extorsisse, quam postea 360 diebus, qui erat modus anni
prisci, adiecerit.
\lnr{7}Plutarchus ita rem narrat: \textgreek{Τῆς Ρέας, φασὶ, κρύφα
τῷ Κρόνῳ συγγενομένης, αἰσθόμενον ὲπαράσασθαι τὸν Ηλιον αὐτῇ, μήτε μηνὶ,
μήτε ἐνιαυτῷ τεκεῖν.}[?]
\lnr{9}\textgreek{ἐρῶντα δὲ τὸν Ερμῆν τῆς θεοῦ συνελθεῖν.}[?]
\lnr{9}\textgreek{εἶτα παίξαντα
πεττία πρὸς τὴν σελήνην, καὶ ἀφελόντα τῶν φώτων ἑκάστου ἑβδομηκοστὸν δεύτερον,
ἐκ πάντων ἡμέρας πέντε συνελεῖν, καὶ ταῖς ἑξήκοντα καὶ τριακοσίαις
ἐπαγαγεῖν, ἃς νῦν ἐπαγομένας
Αἰγύπτιοι καλοῦσι, καὶ τῶν θεῶν γενεθλίους
ἄγουσι.}[?]
% Plutarch: Moralia;Book 5: Isis and Osiris [12]
% Τῆς Ῥέας φασὶ κρύφα
% τῷ Κρόνῳ συγγενομένης αἰσθόμενον ἐπαράσασθαι τὸν Ἥλιον αὐτῇ μήτε μηνὶ
% μήτ᾿ ἐνιαυτῷ τεκεῖν·
% ἐρῶντα δὲ τὸν Ἑρμῆν τῆς θεοῦ συνελθεῖν,
% εἶτα παίξαντα
% πεττία πρὸς τὴν σελήνην καὶ ἀφελόντα τῶν φώτων ἑκάστου [τὸ] ἑβδομηκοστὸν
% ἐκ πάντων ἡμέρας πέντε συνελεῖν καὶ ταῖς ἑξήκοντα καὶ τριακοσίαις
% ἐπαγαγεῖν, ἃς νῦν ἐπαγομένας
% Αἰγύπτιοι καλοῦσι καὶ τῶν θεῶν γενεθλίους
% ἄγουσι.
% Translation:
% They say that the Sun, when he became aware of Rhea’s intercourse with
% Cronus, invoked a curse upon her that she should not give birth to a child
% in any month or any year; but Hermes, being enamoured of the goddess,
% consorted with her. Later, playing at draughts with the moon, he won from her
% the seventieth part of each of her periods of illumination, and from all the
% winnings he composed five days, and intercalated them as an addition to the
% three hundred and sixty days. The Egyptians even now call these five days
% intercalated and celebrate them as the birthdays of the gods.
\lnr{14}Ergo ab initio \textgreek{ἐπαγόμεναι}[?]
nullae erant.
\lnr{15}Et solo
verbo indicatur privatio earum
antiquitus.
\lnr{17}Quod nimirum antea
non fuisse videantur, eo
quod adiectae sint.
%
% Small text table
%
\begin{table}[h]
  %%% Liber 3 p195, PDF 278
%%
%%% Count out columns for fixed-width source font
% 000000011111111112222222222333333333344444444445555555555666666666677777777778
% 345678901234567890123456789012345678901234567890123456789012345678901234567890
%
%% Center the whole table left-right
\centering
%
% Contents is not a real table, more a set of text lines
% Implemented as a minipage
\begin{minipage}{0.6\textwidth}
  \begin{center}
    \textgreek{ΕΠΑΓΟΜΕΝΑΙ ἤτοι ΝΕΙΣΙ.}[?]\\
  \end{center}
  \textgreek{Πρώτη, ΟΣΙΡΙΣ.} \textit{Frater Isidis.}\\
  \textgreek{Δευτέρα, ΑΡΟΥΗΡΙΣ.} \textit{Putamus esse anubim.}[?]\\
  \textgreek{Τρίτη ΤΥΦΩΕΥΣ}[?], \textit{vir Isidis.}\\
  \textgreek{Τετάρτη ΙΣΙΣ.}[?]\\
  \textgreek{Πέμητη ΝΕΦΘΗ, ἤτοι ΑΠΟΦΡΑΣ.}[?]
    \textit{Firmico de errore profanarum religionum dicitur}
    \textgreek{ΝΕΦΘΟΥΝΗ}[?], \textit{Soror Isidis.}
\end{minipage}
\caption[\textgreek{Επαγομεναι ἤτοι Νεισι}]{}
\label{tab:p195}

\end{table}
\lnr{19}\textgreek{Επαγομένων}[?]
autem Aegyptiacarum haec cognomina erant, ut infra subiecta sunt.
\lnr{21}Annus ergo Aegyptiacus fuit dierum trecentum sexaginta quinque,
sine quadrantis adiectione.
\lnr{22}Cuius Thoth primus necessario coepit ab
ortu Caniculae, Sole in Leonem transeunte, novilunio.
\lnr{23}Quia
observatio anni et temporum in Caniculae ortu ab Aegyptiis statuebatur:
eaque erat ipsis cursus Solaris epocha.
\lnr{25}Et quamuis eorum Thoth
laxis habenis in anteriora fugeret, tamen quatro quoque anno diem
intercalabant, ut testatur Diodorus Siculus libro primo, cum de
Thebanis Aegypti loquitur: \textgreek{τὰς γὰρ ἡμέρας}[?], inquit,
 \textgreek{οὐκ ἄγουσι κατὰ σελήνην,
ἀλλὰ κατὰ τὸν ἥλιον, τριακονθημέρους μὲν τιθέμὲνοι τοὺς μῆνας, πέντε
δὲ ἡμέρας, καὶ τέταρτον τοῖς δώδεκα μησὶν ἐπάγουσι.}[?]
\lnr{30}\textgreek{καὶ τούτῳ τῷ τρόπῳ τὸν ἐνιαύσιον κύκλον ἀναπληροῦσιν.}[?]
% Diodorus Siculus, Bibliotheca Historica, Libro primo, caput L, phrase 3
% Τὰς γὰρ ἡμέρας
% [, inquit,]
%  οὐκ ἄγουσι κατὰ σελήνην,
% ἀλλὰ κατὰ τὸν ἥλιον, τριακονθημέρους μὲν τιθέμενοι τοὺς μῆνας, πέντε
% δ´ ἡμέρας καὶ τέταρτον τοῖς δώδεκα μησὶν ἐπάγουσι,
% καὶ τούτῳ τῷ τρόπῳ τὸν ἐνιαύσιον κύκλον ἀναπληροῦσιν.
% Translation: "For they do not reckon the days by the moon, but by the sun,
% making their month of thirty days, and they add five and a quarter days
% to the twelve months and in this way fill out the cycle of the year."
\lnr{31}Id autem fiebat hoc modo: Esto neomenia
Thoth, Kalendis Augusti, Canicula oriente: post quartum annum
Thoth superatis Kal. Augusti ascendet in \rnum{xxxi} Iulii.
\lnr{33}Quare Aegyptiorum
Hierophantae quinto anno ineunte, annum suum \textgreek{ἱερογλυφικὸν}[?]
a secunda die Thoth auspicabantur, et spatium illud quadriennii
\textgreek{κυνικὸν ἐνιαυτὸν}[?] vocabant, item \textgreek{ἡλιακὸν ἔτος}[?],
 item \textgreek{ἔτος Θεοῦ}[?], id
est Solis.
\lnr{37}Sic secundo quadriennio exacto, noni anni principium a
tertia Thoth putabant.
\lnr{38}Et quot quadirennia praeterierant, tot \textgreek{ἐνιαυτοὺς
κυνικοὺς}[?] putabant; quadriennium quidem exactum
 \textgreek{ἔτος Θεοῦ}[?] vocantes,
annum autem aequabilem labentem, \textsc{quadrantem}.
\lnr{40}Verbi
gratia: Proponatur vicesima sexta dies Paophi, qui est secundus
mensis.
%
% 196
% {PDF page nr}{source page nr}{line nr}
\plnr{279}{196}{1}A neomenia Thoth, ad vicesimam sextam Paophi, fluxerunt
dies quinquaginta quinque solidi.
\lnr{2}Qui per quatuor divisi
dant tredecim \textgreek{ἐνιαυτοὺς θεού}[?] cum tribus quadrantibus, et vicesima
sexta dies est quartus quadrans anni \rnum{xiiii} Canicularis.
\lnr{4}Itaque hoc
modo in literis subsignabant, et sacris libris: \textsc{actum anni
dei quarti decimi quadrante quarto}.
\lnr{6}Et quia
annum, ut dixi, aequabilem vocabant \textsc{quadrantem}, propterea
illum \textgreek{ἱερογλυφικῶς}[?] designare volentes,
 quartam partem modi arvalis,
seu iugeri pingebant.
\lnr{9}Horus Apollo: \textgreek{ἔτοσ τὸ ἐνιστάμηνον γράφοντες,
τέταρτον ἀρούρας γράφουσιν.}[?]
\lnr{10}\textgreek{ἔστι δὲ μέτρον γῆς ἡ ἄρουρα, πηχῶν
ἑκατόν.}[?]
\lnr{11}\textgreek{βουλόμηνοί τε ἔτος εἰπεῖν, ΤΕΤΑΡΤΟΝ λέγουσιν.}[?]
\lnr{11}\textgreek{ἐπειδή,
φασι, κατὰ τὴν ἀνατολὴν τοῦ ἄστρου τῆς Σώθεως, μέχρι τῆς ἄλλης ἀνατολῆς,
τέταρτον ἡμέρας προστίθεται, ὡς εἶναι τὸ ἔτος τοῦ ΘΕΟΥ τριακοσίων
ἑξήκοντα πέντε ἡμερῶν.}[?]
\lnr{14}\textgreek{ὅθεν καὶ διὰ τετραετηρίδος περισσὴν ἡμέραν ἀριθμοῦσιν
Αἰγύπτιοι.}[?]
\lnr{15}\textgreek{τὰ τὴς τέσσαρα τέταρτα ἡμέραν ἀπαρτίζει.}[?]
\lnr{15}Idem lib.
\rnum{II}. \rnum{lxxxix}.
\lnr{16}\textgreek{τὸ δ᾽ ἔτος κατ᾽ Αἰγυπτίους τεσσάρων ἐνιαυτῶν.}[?]
\lnr{16}Dilucide et
plane non solum ex quadriennio Aegyptiaco, et quadrante \textgreek{κυνικὸν
ἐνιαυτὸν}[?] constare dicit, sed etiam simplicem annum \textsc{quadrantem}
vocari.
\lnr{19}Sed minus proprie, imo falso dixit, \textgreek{διὰ
τετραετηρίδος περισσὴν ἡμέραν}[?] numerare, imo \textgreek{διὰ πενταετηρίδος}[?], hoc
est anno quarto absoluto, quinto ineunte.
\lnr{21}Videntur vero manca scriptoris
Graeci verba, et legendum, \textgreek{ῶς εἶναι τὸ ἔτος τοῦ ΘΕΟΥ τριακοσίων
ἑξήκοιτα πέντε ἡμερῶν, καὶ ἔτι πρὸς τεταρτημορίου}[?].
\lnr{23}Quare errat Censorinus
dupliciter, cum de hoc anno loquitur: primum, quod \textgreek{κυνικὸν
ἐνιαθτὸν}[?] vocari tantum dicit, eum qui ex multis centuriis annorum
constet.
\lnr{26}Sed etiam, qui putet 1460 anno Iuliano vertente Thoth in
eundem Solis revolui locum, unde ante tot annos profectus fuerat.
\lnr{28}Nam ut paulo ante disputavimus, Thoth profectus ab ortu Caniculae
Kal. Augusti, post 1460 annos aberit ab ipso ortu Caniculae in consequentia
diebus plus minus novem.
\lnr{30}Quod autem quarto quoque
anno diem intercalerent Aegyptii, et in ipso ortu Caniculae, satis
constat ex observatione Eudoxi, qui periodos et ambitus tempestatum
putabat confici quarto quoque anno.
\lnr{33}Eoque fine ille lustrum
suum circumscripsit, anno intercalari.
\lnr{34}Plinius libro \ruleover{\rnum{ii}}: \textit{Omnium
quidem, si libeat observare minimos ambitus, redire easdem vices quadriennio
exacto Eudoxus putat, non ventorum modo, verum et reliquarum
tempestatum magna ex parte.}
\lnr{37}\textit{Est principium lustri eius semper intercalari
anno, Caniculae ortu.}
% Gaius Plinius Secundus, Naturalis Historia, liber 2
% 130.1-4
\lnr{38}Hactenus Plinius.
\lnr{38}Qui et libro \rnum{xviii} ait tempestates
ipsas quadrinis annis suos ardores habere, et easdem non magna
differentia reverti ratione Solis: octonis vero augeri easdem, centesima
revoluente Luna.
\lnr{41}Quod nihil aliud est, quam quatuor annis Iulianis
fieri \textgreek{ἀποκατάστασιν τῶν ἐπισημασιῶν}[?], eamque geminari bis totidem
annis.
%
% 197
% {PDF page nr}{source page nr}{line nr}
\plnr{280}{197}{2}Non solum autem in Aegyptum profectum fuisse Eudoxum
ex Laertio et aliis constat, sed etiam diei quarto quoque anno
exacto intercalandi methodum Graecis prodidisse auctor est Strabo.
\lnr{5}Quanquam multo antea Graecis usurpatam scio.
\lnr{5}Quomodo enim
epocham primi mensis Iphitei servare potuissent?
\lnr{6}Quare vel hinc
clarum est, lustrum Eudoxi nihil aliud, quam quadriennium Caniculare
Aegyptiorum fuisse.
\lnr{8}Idque ipsum Eudoxum a sacerditibus
Aegyptiis didicisse, cum eorum gratia in Aegyptum profectus esset,
ibique suam Octaeterida conscripsisset.
\lnr{10}Sed omne dubium tollit Canicularis
ortus adiecta mentio.
\lnr{11}Ut autem maximus annus Canicularis
revertatur, fateor equidem, opus esse, ut Thoth in ortum Caniculae
incurrat, non autem, ut anni Aegyptiaci solidi 1460 elabantur.
\lnr{14}Hoc accidisse Ulpio, et Brutio Praesente \textsc{coss}.
 testis est idem Censorinus.
\lnr{15}Quo tempore Thoth incidit in \rnum{xx} Iulii, Sole in Leonem transitum
faciente, et Caniculae ortu iam imminente, anno ab % à
 Nabonassaro
886, Christi 138.
\lnr{17}Verus igitur \textgreek{ἐνιαυτὸς ΘΕΟΥ}[?], sive \textgreek{κηνικὸς}[?],
constat quatuor annis Aegyptiacis, et die ex quatuor quadrantibus
diurni temporis conflato.
\lnr{19}Magna autem periodus, quem \textgreek{κηνικὸν
ἐνιαυτὸν}[?] frustra vocat ab ortu Caniculae Censorinus,
 constat annis Canicularibus
solidis 365, quot nempe dies sunt in anno aequabili.
\lnr{22}Et hic est vere annus magnus Canicularis,
 dictus quia ex tot Canicularibus
annis constet, quot dies habet aequabilis annus absolutus:
non autem quod eius Thoth incidat in ortum Caniculae.
\lnr{24}Nam
unde libuerit, Periodum illam magnam constituere potes.
\lnr{25}Triplex
igitur fuerit annus Canicularis.
\lnr{26}Magnus constans quatuor annis et
quadrante: ita dictus, quod quadrans eius observatus sit in ortu Caniculae.
\lnr{28}Item maior Canicularis ex 365 magnis compositus.
\lnr{28}Et maximus, mense Thoth in ortum Caniculae incidente.
\lnr{29}Quod si annus
ille, qui incidit in Consulatum Ulpii, et Brutii Praesentis, fuisset verus
annus vertens periodi magnae Canicularis, iam nihil obstaret
nobis, quominus possemus caput verae periodi Aegyptiae investigare.
\lnr{33}Nunc cum Censorinus referat caussam anni Canicularis
magni ad ortum Caniculae, in quem Thoth forte eo anno inciderat,
manifestum est, eam non esse periodum, quam ipse intelligit.
\lnr{36}Quamuis non dubium est, primum Thoth ab ortu \textgreek{Σώθιος}[?] sive
Caniculae, repetendum esse.
\lnr{37}Apud Herodotum in Euterpe, de anni
Aegyptiaci antiquitate haec exstant: Temporibus ipsius Herodoti,
Aegyptios a mundi conditu putare annos 11340.
\lnr{39}Eosque dicere intra
illud tempus Solem bis ortum, et occasum mutasse.
\lnr{40}Quod quamuis
prima fronte fabulosum videtur, habet tamen implicitam speciem
veri.
%
% 198
% {PDF page nr}{source page nr}{line nr}
\plnr{281}{198}{1}Nam in una magna periodo Sol mutat sedem semel in mensibus
Aegyptiacis, ut qui principio in Thoth Solstitium ingrederetur,
post 730 annos in brumam incideret in aliqua parte eius mensis.
\lnr{3}Sed
hoc non fuerit occasum et orientem mutari.
\lnr{5}Missa igitur illa mendacia et somnia
Aegyptiorum faciamus.
\lnr{6}Necesse est periodum
illorum definere in eundem diem Iulianum,
unde profecta erat, non autem in
ortum Caniculae.
\lnr{9}Et contra principium
periodi fuisse ab ortu Caniculae, quandoquidem
quadrantes in ortu Caniculae intercalabantur,
et ab eodem ortu putabantur.
\lnr{13}Quamuis, ut diximus, anni sideri ratio
hoc non patitur.
\lnr{14}Ex his apparet, in
illa magna periodo quatuor annos pro
una die maximi anni Canicularis accipi,
et triginta eiusmodi dies, sive annos Caniculares,
esse mensem illius maximi anni, qui constaret annis aequabilibus
120.
\lnr{19}Sed de his amplius in anno Persico disputabitur.
\lnr{19}Menses autem Aegyptiorum, cum characteribus suis in hunc laterculum
coniecimus.
% Table
%
\begin{table}[tb]
  %%% Liber III p198, PDF 281
%%
%%% Count out columns for fixed-width source font
% 000000011111111112222222222333333333344444444445555555555666666666677777777778
% 345678901234567890123456789012345678901234567890123456789012345678901234567890
%
\begin{tabnums} % Select monospaced numbers
%% Select a general font size (uncomment one from the list)
%\tiny
%\scriptsize
%\footnotesize
%\small
\normalsize
%% Center the whole table left-right
\centering
%% Modify separation between columns
%\setlength{\tabcolsep}{2.0pt}
%% Modify distance between rows
\renewcommand{\arraystretch}{1.000} % Tuned to eliminate Underfull \vbox
% Just lucky? No stretch required :-)
%% Size of header text
\newcommand{\hts}{\scriptsize}
%% Width of a column
\newcommand{\cwd}{4em}
%
\newcommand{\da}{\scriptsize{†}}
%%
\begin{tabular}{@{} l r r @{}}
\toprule
  \ch{Aegyptiorum}{Menses Aegyptiorum} &
  \ch{\hts{Dies men-}}{\hts{Dies mensium collecti}} &
  \ch{\hts{Character}}{\hts{Character mensium}}
\\
\midrule
\textgreek{θώθ}[?]
 &  30 & 0 \\
\textgreek{παοφί}[?]
 &  60 & 2 \\
\textgreek{ἀθύρ}[?]
 &  90 & 4 \\
\textgreek{χοιάκ}[?]
 & 120 & 6 \\
\textgreek{τυβί}[?]
 & 150 & 1 \\
\textgreek{μεχείρ}[?]
 & 180 & 3 \\
\textgreek{φαμενώθ}[?]
 & 210 & 5 \\
\textgreek{φαρμουθί}[?]
 & 240 & 7 \\
\textgreek{παχών}[?]
 & 270 & 2 \\
\textgreek{παὒνί}[?]
 & 300 & 4 \\
\textgreek{ἐπιφί}[?]
 & 330 & 6 \\
\textgreek{μεσορί}[?]
 & 360 & 1 \\
\textgreek{ἐπαγόμεναι}[?]
 & 365 & 3 \\
\bottomrule
\end{tabular}
%
\caption{Menses Aegyptiorum}
\label{tab:p198}
%
\end{tabnums}

\end{table}
%
%====
\section{De Annis Nabonassari Aegyptiacis}
%
\lnr{22}Veteres Aegyptii innovarunt intervalla annorum suorum,
quoties a novis regibus et victoribus legem acciepiebant.
\lnr{23}Itaque
eos id saepe factitasse dubium non est.
\lnr{24}Antiquissimae eorum
epochae, quae quidem ad memoriam nostram pervenerunt,
sunt hae: Nabonassari, Philippi, vel mortis Alexandri, et Philadelphi.
\lnr{27}Mox anno vago fixo accepta est epocha Augusti Actiaca: et
postrema omnium Diocletiani, quam etiamnum hodie retinent
Aegyptii Christiani \textarabic{}[Arabic] Elkupt.
\lnr{29}Primus Thoth Nabonassari feria
quarta, Februarii 26, anno periodi Iulianae 3967.
\lnr{30}Ideo cum
annos et dies Iulianos in Aegyptiacos Nabonassari rediges, semper
aufer 56 de diebus, et bisextis Iulianis.
\lnr{32}Exemplum: Annus vulgaris
Christi, scilicet a natali, est 4713 periodi Iulianae.
\lnr{33}Deductis
3967 de 4713, remanent 747 anni Iuliani, a fine Decembris anni
3967, ad finem Decembris anni 4713: set ratione epochae Nabonassari,
a 26 Februarii 3967, ad 26 Februarii 4713.
\lnr{36}Detractis
igitur 56 diebus, remanent anni solidi Iuliani 746, dies 309
Bisexta Iuliana annorum 746, id est dies 186, componantur cum
diebus 309.
\lnr{2}Fiunt dies 495, id est annus unus Aegyptiacus cum diebus
130.
%
% 199
% {PDF page nr}{source page nr}{line nr}
\plnr{282}{199}{3}Ergo a primo Thoth Nabonassari, ad finem anni 4713 periodi
Iulianae, sunt anni Aegyptiaci absoluti 747: diesque 130 praeterea
de anno 748 inchoato.
\lnr{5}Deduc 130 de 365, nempe de anno
integro.
\lnr{6}Remanent dies 235 a Kal. Ianuarii definentes in 23 Augusti
inclusive.
\lnr{7}Ergo Thoth 748 Nabonassari incidit in illud tempus.
\lnr{8}Semper adiice 3 ad annos propositus Nabonassari.
\lnr{8}Habebis feriam
Thoth, abiectis scilicet omnibus septenariis.
\lnr{9}Ergo Thoth 748
caepit feria secunda, cyclo Solis \rnum{ix}, Augusti 23.
\lnr{10}Proinde a Thoth
Nabonassari, ad finem Decembris anni 4713 periodi Iulianae, anni
sunt Aegyptiaci, ut diximus, absoluti 747, dies 130: qui fiunt
simul dies 272786, ut volunt quoque Alfonsini.
\lnr{13}Rursus ut annos
Aegyptiacos in Iulianos convertas, bisexta, quae tot annis Iulianis
competunt, quot anni Aegyptiaci propositi sunt, deducantur de annis
et diebus propositis.
\lnr{16}Reliquum erunt anni et dies Iuliani.
\lnr{16}Exemplum:
Proponantur anni Aegyptiaci 747 absoluti, dies 130, in annos
Iulianos et dies redigendi.
\lnr{18}Bisexta, quae competunt annis 747,
id est dies 186, deducantur de annis Aegyptiacis 747, diebus 130:
vel de annis 746, diebus 495.
\lnr{20}Remanent anni Iuliani absoluti 746,
dies 309, ut antea.
\lnr{21}Rursus, cum solos annos Nabonassari investigas,
ut locum Thoth in anno Iuliano invenias, quadrantem annorum
Nabonassari abiice a 56, siquidem quadrans minor sit.
\lnr{23}Reliquum
sunt dies Iuliani a Kal. Ianuarii.
\lnr{24}Volo scire locum neomeniae Thoth
in anno Nabonassari 120.
\lnr{25}Quadrans 30 deductus de 56 relinquit
27 Ianuarii, diem ultimam anni 119.
\lnr{26}Ergo 27 Ianuarii erit neomenia
Thoth.
\lnr{27}Si quadrans excesserit 56, aufer 56 a quadrante.
\lnr{27}Reliquum
sunt dies retro numerandi ab ultima Decembris: vel, quod
idem est, deducendi de quantitate anni Iuliani 365.
\lnr{29}Relinquetur ultima
dies anni praeteriti.
\lnr{30}Alexander decessit anno Nabonassari 425.
\lnr{31}Nam a Thoth Nabonassari, ad Thoth Philippi, sive mortis Alexandri,
Ptolomaeus ponit intervallum, annos 424 praecisos.
\lnr{32}Quadrans
106, deductione facta, relinquit 50.
\lnr{33}Ea porro 50 a 365
diebus Iulianis detracta relinquunt ultimam diem anni 424
in \rnum{xi} Novembris.
\lnr{35}Itaque Thoth annorum Philippi debetur
duodecimae Novembris, feria prima.
\lnr{36}Rursus annis Nabonassari
adiice 18,
% Newline? No; 1598 ed (p190.B) has comma
siquidem non excesserint 228.
\lnr{37}Reliquum per 28
divisum relinquit cyclum Solis Iulianum.
\lnr{38}Si excedunt 228, adiice
17.
\lnr{39}Denique si excedunt 1688, adiice 16.
\lnr{39}Sed nequis error
incautis iuvenibus obrepat, praestat Laterculum characteris omnium
mensium in annis Nabonassari proponere cum cyclo Solis
Iuliano.
%
% 200
% {PDF page nr}{source page nr}{line nr}
\plnr{283}{200}{1}Eius Laterculi exemplar infra subiecimus.
% Table "Laterculus characteris mensium in annis Nabonassari"
%
\begin{table}[tbp]
  %%% Liber III p200
%% Layout based on Liber II p89
% !TEX root = ../../test-table.tex
%%
%%% Count out columns for fixed-width source font
% 000000011111111112222222222333333333344444444445555555555666666666677777777778
% 345678901234567890123456789012345678901234567890123456789012345678901234567890
%
\begin{tabnums} % Select monospaced numbers
%\tiny
%\scriptsize
%\footnotesize
%\small
\normalsize
%% Center the whole table left-right
\centering
%% Modify separation between columns
\setlength{\tabcolsep}{3.0pt}
%% Modify distance between rows
\renewcommand{\arraystretch}{1.0}
%
%% Define a smaller dagger (unfortunalely tiny is already the smallest)
\newcommand{\da}{{\tiny †}}
%% The angle with which to slant
\newcommand{\ang}{75}
%% Header text size: slanted text
\newcommand{\hsb}[1]{\small{#1}}
%% Header text size: bottom row
\newcommand{\hsa}[1]{\scriptsize{#1}}
%% Width of a column
\newcommand{\cwd}{1.0em}
%%
%%
\begin{tabular}[c]{@{} r  c c c c c c c c c c c c c  c c c c @{}}
%%
\toprule
%% Table title
\multicolumn{18}{c}{\Large\textsc{Laterculus Characteris Mensium in Annis}}\\
\multicolumn{18}{c}{\large\textsc{Nabonassari, per Cyclum Solis Iulianum}}\\
\toprule
%% Table header
\hsa{\ch{Cyclus sol-}{Cyclus so\-lis Na\-bo\-nas\-sa\-ri}} &

\hsb{\parbox[b]{\cwd}{\begin{rotate}{\ang}\textgreek{Θώθ}\end{rotate}}} &
\hsb{\parbox[b]{\cwd}{\begin{rotate}{\ang}\textgreek{Παωφί}\end{rotate}}} &
\hsb{\parbox[b]{\cwd}{\begin{rotate}{\ang}\textgreek{Αθύρ}\end{rotate}}} &

\hsb{\parbox[b]{\cwd}{\begin{rotate}{\ang}\textgreek{Χοιάκ}\end{rotate}}} &
\hsb{\parbox[b]{\cwd}{\begin{rotate}{\ang}\textgreek{Τυβί}\end{rotate}}} &
\hsb{\parbox[b]{\cwd}{\begin{rotate}{\ang}\textgreek{Μεχείρ}\end{rotate}}} &

\hsb{\parbox[b]{\cwd}{\begin{rotate}{\ang}\textgreek{Φαμενώθ}\end{rotate}}} &
\hsb{\parbox[b]{\cwd}{\begin{rotate}{\ang}\textgreek{Φαρμουθί}\end{rotate}}} &
\hsb{\parbox[b]{\cwd}{\begin{rotate}{\ang}\textgreek{Παχών}\end{rotate}}} &

\hsb{\parbox[b]{\cwd}{\begin{rotate}{\ang}\textgreek{Παὒνί}\end{rotate}}} &
\hsb{\parbox[b]{\cwd}{\begin{rotate}{\ang}\textgreek{Επιφί}\end{rotate}}} &
\hsb{\parbox[b]{\cwd}{\begin{rotate}{\ang}\textgreek{Μεσορί}\end{rotate}}} &
\hsb{\parbox[b]{\cwd}{\begin{rotate}{\ang}\textgreek{Επαγόμεναι}\end{rotate}}} &

\hsa{\ch{Character}{Character cycli Solis Iuliani}} &
\hsa{\ch{Iulianus}{Cyclus Solis Iulianus intra 228}} &
\hsa{\ch{Iulianus}{Cyclus Solis Iulianus intra 1688}} &
\hsa{\ch{Iulianus}{\medskip Cyclus Solis Iulianus intra 3148}} 
% Force some extra room above this header so that the slanted headers
% don't run over into the table title.
\\
%% Table body
\midrule
%%
 1~ &
4 & 6 & 1 & 3 & 5 & 7 & 2 & 4 & 6 & 1 & 3 & 5 & 7 &
 E  & 19 & 18 & 17 \\
%
 2~ &
5 & 7 & 2 & 4 & 6 & 1 & 3 & 5 & 7 & 2 & 4 & 6 & 1 &
 D  & 20 & 19 & 18 \\
%
 3~ &
6 & 1 & 3 & 5 & 7 & 2 & 4 & 6 & 1 & 3 & 5 & 7 & 2 &
C B & 21 & 20 & 19 \\
%
 4~ &
7 & 2 & 4 & 6 & 1 & 3 & 5 & 7 & 2 & 4 & 6 & 1 & 3 &
 A  & 22 & 21 & 20 \\
%
 5~ &
1 & 3 & 5 & 7 & 2 & 4 & 6 & 1 & 3 & 5 & 7 & 2 & 4 &
 G  & 23 & 22 & 21 \\
%
 6~ &
2 & 4 & 6 & 1 & 3 & 5 & 7 & 2 & 4 & 6 & 1 & 3 & 5 &
 F  & 24 & 23 & 22 \\
%
 7~ &
3 & 5 & 7 & 2 & 4 & 6 & 1 & 3 & 5 & 7 & 2 & 4 & 6 &
E D & 25 & 24 & 23 \\
%
 8~ &
4 & 6 & 1 & 3 & 5 & 7 & 2 & 4 & 6 & 1 & 3 & 5 & 7 &
 C  & 26 & 25 & 24 \\
%
 9~ &
5 & 7 & 2 & 4 & 6 & 1 & 3 & 5 & 7 & 2 & 4 & 6 & 1 &
 B  & 27 & 26 & 25 \\
%
10~ &
6 & 1 & 3 & 5 & 7 & 2 & 4 & 6 & 1 & 3 & 5 & 7 & 2 &
 A  & 28 & 27 & 26 \\
%
11~ &
7 & 2 & 4 & 6 & 1 & 3 & 5 & 7 & 2 & 4 & 6 & 1 & 3 &
G F & ~1 & 28 & 27 \\
%
12~ &
1 & 3 & 5 & 7 & 2 & 4 & 6 & 1 & 3 & 5 & 7 & 2 & 4 &
 E  & ~2 & ~1 & 28 \\
%
13~ &
2 & 4 & 6 & 1 & 3 & 5 & 7 & 2 & 4 & 6 & 1 & 3 & 5 &
 D  & ~3 & ~2 & ~1 \\
%
14~ &
3 & 5 & 7 & 2 & 4 & 6 & 1 & 3 & 5 & 7 & 2 & 4 & 6 &
 C  & ~4 & ~3 & ~2 \\
%
15~ &
4 & 6 & 1 & 3 & 5 & 7 & 2 & 4 & 6 & 1 & 3 & 5 & 7 &
B A & ~5 & ~4 & ~3 \\
%
16~ &
5 & 7 & 2 & 4 & 6 & 1 & 3 & 5 & 7 & 2 & 4 & 6 & 1 &
 G  & ~6 & ~5 & ~4 \\
%
17~ &
6 & 1 & 3 & 5 & 7 & 2 & 4 & 6 & 1 & 3 & 5 & 7 & 2 &
 F  & ~7 & ~6 & ~5 \\
%
18~ &
7 & 2 & 4 & 6 & 1 & 3 & 5 & 7 & 2 & 4 & 6 & 1 & 3 &
 E  & ~8 & ~7 & ~6 \\
%
19~ &
1 & 3 & 5 & 7 & 2 & 4 & 6 & 1 & 3 & 5 & 7 & 2 & 4 &
D C & ~9 & ~8 & ~7 \\
%
20~ &
2 & 4 & 6 & 1 & 3 & 5 & 7 & 2 & 4 & 6 & 1 & 3 & 5 &
 B  & 10 & ~9 & ~8 \\
%
21~ &
3 & 5 & 7 & 2 & 4 & 6 & 1 & 3 & 5 & 7 & 2 & 4 & 6 &
 A  & 11 & 10 & ~9 \\
%
22~ &
4 & 6 & 1 & 3 & 5 & 7 & 2 & 4 & 6 & 1 & 3 & 5 & 7 &
 G  & 12 & 11 & 10 \\
%
23~ &
5 & 7 & 2 & 4 & 6 & 1 & 3 & 5 & 7 & 2 & 4 & 6 & 1 &
F E & 13 & 12 & 11 \\
%
24~ &
6 & 1 & 3 & 5 & 7 & 2 & 4 & 6 & 1 & 3 & 5 & 7 & 2 &
 D  & 14 & 13 & 12 \\
%
25~ &
7 & 2 & 4 & 6 & 1 & 3 & 5 & 7 & 2 & 4 & 6 & 1 & 3 &
 C  & 15 & 14 & 13 \\
%
26~ &
1 & 3 & 5 & 7 & 2 & 4 & 6 & 1 & 3 & 5 & 7 & 2 & 4 &
 B  & 16 & 15 & 14 \\
%
27~ &
2 & 4 & 6 & 1 & 3 & 5 & 7 & 2 & 4 & 6 & 1 & 3 & 5 &
A G & 17 & 16 & 15 \\
%
28~ &
3 & 5 & 7 & 2 & 4 & 6 & 1 & 3 & 5 & 7 & 2 & 4 & 6 &
 F  & 18 & 17 & 16 \\
%
\bottomrule
%%
\end{tabular}
%
\caption{Characteris Mensium in Annis Nabonassari}
\label{tab:p200}
%
\end{tabnums}

\end{table}
%
\lnr{1}Divisis annis
Nabonassari per septem, adiectis tribus, diximus relinqui characterem
Thoth Nabonassari.
\lnr{3}Qui characteri sequentium mensium adiectus,
quos in capite de anno Aegyptiaco posuimus, indicabit feriam omnium
mensium illius anni.
\lnr{5}Sed si vis characterem illum cum cyclo
Solis Iuliano comparare, tunc ingressus faciendus in hunc laterculum.
\lnr{7}Itaque considerabis, an summa annorum propositorum sit intra
228, an vero excedat, et cyclum Solis Iulianum ei congruentem
pro neomenia Thoth assumes.
\lnr{9}Annos igitur Nabonassari per 28 divide.
\lnr{10}Reliquum est annus cycli Solis Nabonassari in latere sinistro.
\lnr{11}E regione illius habes omnes characteres omnium mensium anni propositi:
et a latere dextro cyclum Solis Iulianum ei congruentem.
%
% 201
% {PDF page nr}{source page nr}{line nr}
\plnr{284}{201}{1}Exemplum.
\lnr{1}Volo scire characterem Thoth anni 120 a Nabonassaro,
et cyclum Romanum illi competentem.
\lnr{2}Abiectis 28, remanent 8,
qui est octavus annus cycli Solis Nabonassari.
\lnr{3}A latere sinistro habes
8 annum cycli Nabonassari, feriam Thoth 4, Paophi 6, et cetera.
\lnr{4}Et quia
anni 120 sunt intra 228, in latere dextro cyclus Iulianus ei respondens
est 26.
\lnr{6}Litera dominicalis \textsc{C}.
\lnr{6}Feria ergo 4 erit \textsc{F}. % Newline? Yes. 1598: capital P on Proinde
\lnr{6}Proinde Thoth coepit
litera F, 27 Ianuarii, ut antea explicavimus.
\lnr{7}Dici non potest, quam tutus
et expeditus sit usus huius Laterculi illis, qui Ptolemaeum legunt.
\lnr{9}Quod si numerus annorum Nabonassari excesserit 228, tunc secunda
columna cycli Iuliani assumenda est, ut eius titulus indicat.
\lnr{11}Verbi gratia: in anno 232 erit idem cyclus Nabonassari \rnum{viii}.
\lnr{11}In secunda columna ei respondet cyclus 25 Iulianus.
\lnr{12}Propterea litera Dominicalis erit \textsc{E D}.
\lnr{13}Feria quarta, quae in anno 120 erat in litera \textsc{F}, tunc erit
in litera \textsc{G}, Decembris 30.
\lnr{14}Sed quia reliqui menses Nabonassari pertinent
ad cyclum Solis Iulianum 26, diligenter cavere debent adolescentes,
ne putent semper eundem cyclum per totum annum usurpari.
\lnr{17}Hoc tantum accidit in quatuor annis, nempe 224, 225, 226, 227, in
prima columna.
\lnr{18}In secunda, annis 1684, 1685, 1686, 1687.
\lnr{18}In
tertia culumna, in annis 3144, 3145, 3146, 3147.
\lnr{19}In omnibus reliquis
annis semper bini cycli Iuliani vindicant sibi annum Nabonassari.
\lnr{21}Quare is cyclus Iulianus, qui e regione cycli Nabonassari respondet,
is, inquam, est cyclus Thoth, et si qui sunt alii menses, intra illum annum
cycli.
\lnr{23}Nam constat non totum annum Nabonassari uno cyclo
Iuliani attribui.
\lnr{24}Censorinus scribebat aureolum libellum suum de
die Natali anno Christi Dionysiano 238, cyclo Solis \rnum{xxiii}, Lunae
\rnum{xi}: quod cognoscimus ex anno Iphiti 1014, quo ineunte a diebus
aestivis ea scribebat eximius ille et doctissimus temporum et antiquitatis
vindex.
\lnr{28}Is igitur scribit, eo anno Thoth Nabonassari noningentesimum
octagesimum sextum incurrisse in a.d. \rnum{vii} Kalen. Iulias:
hoc est in \rnum{xxv} Iunii.
\lnr{30}Abiectis 28 de 986, remanet annus cycli Solis
Nabonassari sextus.
\lnr{31}In secunda columna sinistra est cyclus Solis
23, ut proposuimus.
\lnr{32}Feria Thoth secunda.
\lnr{32}Litera Dominicalis cycli
23, est \textsc{G}.
\lnr{33}Ergo secunda feria \textsc{A}.
\lnr{33}Qui est character \rnum{xxv} Iunii.
\lnr{33}Praeterea
quadrans annorum 986, nempe 246, deductis 56, relinquit
190.
\lnr{35}Quae et ipsa de anni Iuliani quantitate deducta relinquunt 175,
ultimam diem anni 985 in 24 Iunii.
\lnr{36}Ergo Thoth 986 in 25.
\lnr{36}Quod
autem scribitur apud eundem Censorinum, Ulpio et Brutio Praesente
\textsc{coss.} anno Nabonassari 886, neomeniam % neomeniā
 Thoth incurrisse in \rnum{xii} Kal.
Augusti, mendum est librarii, non error Censorini.
\lnr{39}Centum enim quadrantes % quadrātes
Aegyptii dant annos Caniculares \rnum{xxv}. % Newline? (same in 1598 ed.)
\lnr{40}Hoc est, dies \rnum{xxv}.
\lnr{40}Additis
25 ad 25 Iunii, incidis in 20 Iulii.
\lnr{41}Periclitemur ex methodo nostra.
\lnr{41}Quadrans
annorum 886 est 165, detractis 56.
%
% 202
% {PDF page nr}{source page nr}{line nr}
\plnr{285}{202}{1}Rursus 165 de 365 detracta
relinquunt 200, nempe 19 Iulii, ultimam diem anni 885.
\lnr{2}Quare
Thoth 886 coepit die Iulii 20.
\lnr{3}Abiectis 28 de 886 relinquitur cyclus
Solis Nabonassari 18.
\lnr{4}Qui in latere sinistro indicat feriam Thoth 7: in
latere dextro, secunda columna cyclum Solis Iulianum \rnum{vii}.
\lnr{5}Feria igitur
septima \textsc{E}, qui est character \rnum{xx} Iulii.
\lnr{6}Sed expeditius, et dicto citius,
addita tria ad annum Nabonassari dabunt feriam neomeniae Thoth,
abiectis omnibus septenariis.
\lnr{8}Nunc ad methodum Lunae in mensibus
Nabonassari veniamus.
\lnr{9}Cui rei maiori ex parte satisfecimus,
cum generaliter \textgreek{εἰκοσιπενταετηρίδα}[?] anna aequabilis explicavimus.
\lnr{10}Ptolemaeus
libro \rnum{v} statuit novilunium primi Thoth Nabonassari
% Odd way of writing: only first group (the day) gets a "Number" bar
% the other two (scrupulis diurnis) do not.
 \textgreek{\gnum{κδ}. μδ'. ιζ''}. % 1598 ed.: iota-zeta
% 20+4. 40+4'. 10+7''
% Newline? Same in 1598 ed.
\lnr{12}Hoc est, Thoth \rnum{xxiiii}. % Newline? No. Sentence runs on.
 scrupulis diurnis 44', 17'', quae ad
rationem Chaldaicam redacta fiunt 24 dies Thoth, horae 17, 770.
\lnr{14}Terminus igitur Thoth primi anni \textgreek{εἰκοσιπενταετηρίδος}[?]
 Nabonassari est
\rnum{xxiiii}.
\lnr{15}Iam diximus initio, epactas descendere; contra Terminos ascendere.
\lnr{16}Si \rnum{xxiiii} est primi anni Terminus, \rnum{xiiii},
 qui proxime antecedit,
erit Terminus secundi, et sic deinceps ascendendo.
\lnr{17}Quibus
terminis per ordinem et contextum suum digestis, facile situs embolismi
in periodo annorum Nabonassari deprehendetur.
\lnr{19}Is enim annus
est \textgreek{ἐμβολιμαῖοσ}[?], quem sequitur Terminus maior.
\lnr{20}Ut, verbi gratia,
annus tertius habet \rnum{iii} pro Termino, quem sequitur \rnum{xxii}, Terminus
nempe maior antecedente.
\lnr{22}Ergo \rnum{iii} est \textgreek{ἐμβολιμαῖος}[?].
\lnr{22}Quibus
animadversis, periodus Nabonassari, et eius embolismi nihil differunt
ab ordine, situ et progressu enneadecaeteridis: quatenus septem
priores embolismi sunt in annis tertio, sexto, octavo, undecimo,
quartodecimo, decimoseptimo, et nonodecimo, ut in Enneadecaeteride. % newline?
% Same in the 1598 edition: Word, period, no space, no capital on next word.
\lnr{27}Reliqui duo sunt principium Enneadecaeteridis.
\lnr{27}Nam embolismus
vicesimus secundus est primus embolismus enneadecaeteridis
in anno tertio.
\lnr{29}Embolismus autem ultimus, est secundus in anno
sexto enneadecaeteridis.
\lnr{30}Quare nova periodus instituenda fuit, cuius
annus primus habet \textgreek{ἐποχὴν}[?] primi novilunii Nabonassari.
\lnr{31}Reliqui anni
sunt vere Chaldaici, quandoquidem pertinent ad enneadecaeterida
Chaldaicam: neque alio differunt, nisi quod epochen % Sic; same in 1598 ed
 Nabonassari
implicitam habent.
%
% Two tables p203
%
\lnr{34}Tabulam autem \textgreek{εἰκοσιπενταετηρίδος}[?] duplicem
fecimus, ut pugillar bipatens, quod vocat Ausonius.
\lnr{35}Prior pars
continet, ut diximus, annos periodi, quorum primus est mera epocha
Nabonassari, reliqui sunt anni communes, aut embolimaei, ut ordo
postulat, cum ipsa epocha impliciti.
\lnr{38}Posterior pars, quae et dextra, habet
illos annos inversos, quo nomine posteriorem partem vocavimus.
\lnr{40}Deducta enim priore parte 24, 17, 770, relinquitur posterior pars,
5, 6, 310.
\lnr{41}Porro prior pars indicat feriam novilunii, posterior diem mensis.
%
% 203
% {PDF page nr}{source page nr}{line nr}
\plnr{286}{203}{1}Utraque horas, et scrupulos.
%
% Two tables
%
\lnr{2}Hanc periodum antecedit
Tabella oppositionum
in mensibus Nabonassari. % 'mensib.' Newline? Abbriviation? If so, of what?
% 1598 edition (p194): mensibus
\lnr{4}Quam
in gratiam eorum confecimus,
qui Ptolemaeum legunt: ut sine
magno negotio et labore,
deliquia Lunaria ab antiquis
observata, quae exstant apud illum
scriptorem, reperiant.
\lnr{11}Coniunctiones autem luminarium
petendae sunt ex Tabella
mensium, quam initio huius
libri posuimus, ubi methodum
anni aequabilis tradimus.
\lnr{15}Postremo
anteposita est Tabula
per periodos expansas collecta,
quam ad 3000 annos extendimus.
\lnr{19}Hoc quoque studiosorum
laborem levabit in noviluniis,
et oppositionibus colligendis.
\lnr{21}Methodi summa haec est: Ab
annis Nabonassari propositis abiice semper minorem numerum annorum
in sinistro latere Tabulae maioris, etiam si praecise inveniuntur.
%
% 204
% {PDF page nr}{source page nr}{line nr}
\plnr{287}{204}{1}Ut si propositus sit annus Nabonassari 225: quamius is numerus
praecise in latere Tabellae invenitur, tamen proxime minor 200 abiicitur.
% Table "Cyclus Nabonassari per annos expansos"
\lnr{3}Reliqui 25 petendi ex Tabula annorum expansorum.
\lnr{3}Si vis igitur
scire novilunium Thoth illius anni Nabonassari, accipe numeros
partis anterioris annorum 200, quos a 225 abiecisti. % Newline? Abbriviation?
 nempe 3, 15,
96.
\lnr{6}Item numeros anni 25 in Tabula annorum expansorum, 4, 19,
58.
\lnr{7}Compone simul. % Newline? Abbriviation?
% 1598 edition: capital on next word, ergo newline.
\lnr{7}Prodit novilunium Thoth 1, 10, 154.
\lnr{7}Adde tria
annis Nabonassari.
\lnr{8}Abiice 7 omnia.
\lnr{8}Reliqua est feria 4, neomeniae
Thoth.
\lnr{9}Terminus novilunii \rnum{v}, e regione anni 25, indicat novilunium
confici in dicta die quinta Thoth, quae est feria prima, ut recte
collectum est.
\lnr{11}Quod si terminus non indicaret, tamen numeri partis
posterioris rem explicassent.
\lnr{12}Numerus partis posterioris annorum
200, est 0, 8, 984.
\lnr{13}Item numerus eiusdem partis e regione anni 25,
est 24, 4, 1022.
\lnr{14}Compone.
\lnr{14}Prodeunt 24, 13, 926.
\lnr{14}Abiice illa a 30.
%
% 205
% {PDF page nr}{source page nr}{line nr}
\plnr{288}{205}{1}Relinquuntur 5, 10, 154. % Newline?
 dies nempe quinta Thoth, hora 10 post
meridiem, scrup. % abbriviation
 Chald. % abbriviation
% Both same in 1598 ed.
 154.
\lnr{2}Memineris autem, scrupulos a 1080
abiici, horas a 24, feriam a 7, dies a 30.
\lnr{3}Et cum colliguntur dies
partis posterioris, semper triginta abiiciuntur in excessu tricenarii.
\lnr{4}Ut
exploratius tibi constet de fide methodi, periclitare, an eclipsin Lunae,
quae eodem anno 225 contigit, deprehendere possis.
\lnr{6}Ea confecta
est hora 10, 100 post meridiem, 17 Phamenoth, hor. % abbriviation
 1, 280
post mendiam oppositionem.
\lnr{8}Ergo media oppositio fuit hora 8, 988.
\lnr{9}Dies 17 Phamenoth est 197 dies neomenia Thoth.
\lnr{9}Auser igitur
Terminum, hoc est \rnum{v}, a 197.
\lnr{10}Relinquuntur 192.
\lnr{10}Sume oppositionem
numeri proxime minoris 191.
\lnr{11}Numerus anterior 2, 22, 834.
\lnr{12}Iunge cum novilunio Thoth 1, 10, 154.
\lnr{12}Prodit oppositio Phamenoth,
4, 8, 988, ut propositum fuit.
\lnr{13}Annus coepit feria quarta.
\lnr{13}Adde
characterem anni characteri Phamenoth ex laterculo mensium
Aegyptiacorum.
\lnr{15}Neomenia Phamenoth fuit feria 2. % Newline?
\lnr{15}Eadem feria in \rnum{xv}.
\lnr{16}Ergo \rnum{xvii} Phamenoth feria 4, ut indicat epilogismus.
\lnr{16}Quod si nescirem
feriam, tamen per numeros posteriores id collegissem.
\lnr{17}Numerus
posterior oppositionis 191, est 18, 1, 246.
\lnr{18}Iunge cum 24, 13, 926
novilunii Thoth.
\lnr{19}Abiice 30, (quod semper faciendum, si opus est.) % no period in original
% 1598 ed. has period before the closing parenthesis.
\lnr{20}Exeunt, 12, 15, 92.
\lnr{20}Deducta de 30, relinquunt dies 17, 8, 988.
\lnr{20}Ergo
die 17 Phamenoth, feria 4, hora 8, scrup. % abbriv.
 988, fuit media oppositio
luminarium.
\lnr{22}Quare si Termini iidem semper esse possent, non
opus fuisset methodo posterioris et inversae partis.
\lnr{23}Sed in 500 annis
ii mutantur.
\lnr{24}Antevertit enim eos Luna horis 22.
\lnr{24}Quare confugiendum
ad numeros universos.
\lnr{25}Exemplum: Anno Nabonassari 547
defecit Luna, Mesori 16, circiter hor. % abbriv.
 7, post meridiem, horis
duodecim ante mediam oppositionem.
\lnr{27}Ergo media coniunctio non
abfuit ab hora 19 post meridiem.
\lnr{28}Colligo primum numeros 525
annorum proxime minorum, scilicet, 6, 0, 657. % Newline?
\lnr{39}Et in Tabula annorum
expansorum e regione anni 22, numeros 4, 3, 957.
\lnr{30}Conficitur
novilunium Thoth, 3, 4, 534.
\lnr{31}Iam 16 Mesori est 346 dies a neomenia
Thoth.
\lnr{32}Terminus anni 22, nempe 8, deductus de 346, relinquit
oppositionem 338 in mense 12.
\lnr{33}Cuius mensis numerus anterior, 3, 14,
479, cum neomenia Thoth constituit oppositionem Mesori 6, 18,
1013.
\lnr{35}Fuit igitur novilunium feria sexta, Neomenia Thoth feria quarta.
\lnr{36}Quae characteri Mesori adiecta constituit eius neomeniam, et 15
eiusdem mensis, feriam 5.
\lnr{37}Ergo 16 Mesori fuit oppositio media circiter
19 horam, ut propositum erat.
\lnr{38}Iam si secutus fuissem Terminum, superassem
fines oppositionis die solido.
\lnr{39}Nam terminus 8 ad 339 appositus
conficeret 347.
\lnr{40}Quae est 17 dies Mesori, non 16.
\lnr{40}Itaque in annis 547
Termini excedunt fines Lunae die uno.
\lnr{41}At per numeros posteriores
tutissima est methodus.
%
% 206
% {PDF page nr}{source page nr}{line nr}
\plnr{289}{206}{1}Numeri posteriores annorum 535, et 22 adiecti
fiunt 22, 19, 546.
\lnr{2}Quibus
































% ==== End of text of Liber Tertius ===
