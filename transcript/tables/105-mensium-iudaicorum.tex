%%% Liber II p101, PDF 184
%%
%% Table in the section "De periodo Chaldaeorum Alexandrea"
%% Title is given in the original.
%% The body of the table is filled mostly with Arabic, Turkish, Persian and
%% "Chatai", which all look arabic. There is also a column of Syrian,
%% same as what is seen in the table on p99.
%%
%% In the 1593 edition, p78, the body of this table is filled with Hebrew,
%% and the column of numbers is headed "Anni Schaichun."
%%
%% There is currently (nov 2017) no information on the Chaldean calendar
%% on Wikipedia
%%
%%% Count out columns for fixed-width source font
% 000000011111111112222222222333333333344444444445555555555666666666677777777778
% 345678901234567890123456789012345678901234567890123456789012345678901234567890
%
%% Select a general font size (uncomment one from the list)
%\tiny
%\scriptsize
%\footnotesize
%\small
\normalsize
%
%% Center the whole table left-right
\centering
%
%% Modify separation between columns
%\setlength{\tabcolsep}{1.6pt}
%
%% Modify distance between rows
\renewcommand{\arraystretch}{0.99} % Tuned to eliminate Underfull \vbox
%%
% Header text size: bottom row
\newcommand{\hsb}[1]{\scriptsize{#1}} 
%
\begin{tabular}{@{}l c c c c c c@{}}
\toprule
 \multicolumn{7}{c}{\Large\textsc{Tabula characterismi}}\\
 \multicolumn{7}{c}{\large\textsc{mensium Iudaicorum}}
\\
\toprule
 
 ~ &
 \multicolumn{3}{c}{Communis} &
 \multicolumn{3}{c}{Embolimaeus}
\\
\cmidrule(lr){2-4}
\cmidrule(lr){5-7} 
 ~ &
 \multicolumn{1}{c}{\hsb Defectivus} &
 \multicolumn{1}{c}{\hsb Ordinarius} &
 \multicolumn{1}{c}{\hsb Abundans} &
 \multicolumn{1}{c}{\hsb Defectivus} &
 \multicolumn{1}{c}{\hsb Ordinarius} &
 \multicolumn{1}{c}{\hsb Abundans}
\\
\midrule
 \textsc{Tisri} &
 0 &
 0 &
 0 &
 0 &
 0 &
 0
 \\
 \textsc{Marcheswan} &
 2 &
 2 &
 2 &
 2 &
 2 &
 2
\\
 \textsc{Chaslev} &
 3 &
 3 &
 4 &
 3 &
 3 &
 4
\\
\midrule
 \textsc{Tebeth} &
 4 &
 5 &
 6 &
 4 &
 5 &
 6
\\
 \textsc{Schebat} &
 5 &
 6 &
 7 &
 5 &
 6 &
 7
\\
 \textsc{Adar prior.} &
 0 &
 0 &
 0 &
 7 &
 1 &
 2
\\
 \textsc{Adar poste.} &
 7 &
 1 &
 2 &
 2 &
 3 &
 4
\\
\midrule
 \textsc{Nisan} &
 1 &
 2 &
 3 &
 3 &
 4 &
 5
\\
 \textsc{Iiar} &
 3 &
 4 &
 5 &
 5 &
 6 &
 7
\\
 \textsc{Siwan} &
 4 &
 5 &
 6 &
 6 &
 7 &
 1
\\
\midrule
 \textsc{Thamuz} &
 6 &
 7 &
 1 &
 1 &
 2 &
 3
\\
 \textsc{Ab} &
 7 &
 1 &
 2 &
 2 &
 3 &
 4
\\
 \textsc{Elul} &
 2 &
 3 &
 4 &
 4 &
 5 &
 6
\\
\bottomrule
\end{tabular}
%
\caption{Characterismi mensium Iudaicorum}
\label{tab:p105}
