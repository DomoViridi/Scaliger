%%% Liber II p110, PDF 193
%%
%% Table in the section "De periodo Arabum Hagarenorum"
%% No title is given in the original.
%% The second column is in Arabic
%% Modern names of the months as given on the "Islamic calendar" page of
%% Wikipedia were used to fill this column. The English transcriptions 
%% given there are added here as comments.
%%
%%% Count out columns for fixed-width source font
% 000000011111111112222222222333333333344444444445555555555666666666677777777778
% 345678901234567890123456789012345678901234567890123456789012345678901234567890
%
%% Select a general font size (uncomment one from the list)
%\tiny
%\scriptsize
%\footnotesize
%\small
\normalsize
%
%% Center the whole table left-right
\centering
%
%% Modify separation between columns
%\setlength{\tabcolsep}{1.6pt}
%
%% Modify distance between rows
%\renewcommand{\arraystretch}{1.3}
%%
\begin{tabular}{@{}r r l@{}}
\toprule
 0 &
 % Rabī‘ al-awwal (the first spring)
 \textarabic{رَبيع الأوّل}[?] &
 \emph{Rabiu prior}
\\
 2 &
 % Rabī‘ ath-thānī (the second spring)
 \textarabic{رَبيع الثاني}[?] &
 \emph{Rabiu posterior}
\\
 3 &
 % Jumādá al-ūlá (the first of parched land)
 \textarabic{جُمادى الأولى}[?] &
 \emph{Giumediiu prior}
\\
\midrule
 5 &
 % Jumādá al-ākhirah (the last of parched land)
 \textarabic{جُمادى الآخرة}[?] &
 \emph{Giumediiu posterior}
\\
 6 &
 % Rajab (respect, honour)
 \textarabic{رَجَب}[?] &
 \emph{Regebu}
\\
 1 &
 % Sha‘bān (scattered)
 \textarabic{شَعْبان}[?] &
 \emph{Saabenu}
\\
\midrule
 2 &
 % Ramaḍān (burning heat)
 \textarabic{رَمَضان}[?] &
 \emph{Ramadhanu}
\\
 4 &
 % Shawwāl (raised)
 \textarabic{شَوّال}[?] &
 \emph{Schevvalu}
\\
 5 &
 % Dhū al-Qa‘dah (the one of truce/sitting)
 \textarabic{ذو القعدة}[?] &
 \emph{Dulkaidathi}
\\
\midrule
 7 &
 % Dhū al-Ḥijjah (the one of pilgrimage)
 \textarabic{ذو الحجة}[?] &
 \emph{Dulhagaiathi}
\\
 1 &
 % Muḥarram (forbidden)
 \textarabic{مُحَرَّم}[?] &
 \emph{Muharramu}
\\
 3 &
 % Ṣafar (void)
 \textarabic{صَفَر}[?] &
 \emph{Tzepharu}
\\
\bottomrule
\end{tabular}
%
\caption{Periodo Hagarena}
\label{tab:p110}
