%%% Liber II p138, PDF 221
%%
%% Arabic names of the month copied from Wikipedia "Islamic calendar"
%% We could also use the \HijriMonthArabic{nr} command from the
%% polyglossia/hijrical.sty package which is already loaded
%% (probably when \otherlanguage{arabic} is set for polyglossia).
%% But that looks slightly different, and slightly less than the original :-(
%%
%%% Count out columns for fixed-width source font
% 000000011111111112222222222333333333344444444445555555555666666666677777777778
% 345678901234567890123456789012345678901234567890123456789012345678901234567890
%
\begin{tabnums} % Select monospaced numbers
%% Select a general font size (uncomment one from the list)
%\tiny
%\scriptsize
%\footnotesize
%\small
\normalsize
%% Center the whole table left-right
\centering
%% Modify separation between columns
%\setlength{\tabcolsep}{2.1pt}
%% Modify distance between rows
\renewcommand{\arraystretch}{1.015} % Tuned to eliminate Underfull \vbox
%% Size of header text
\newcommand{\hts}{\footnotesize}
%
%% Define names of the months
\newcommand{\Tis}{Tisri}
\newcommand{\Mar}{Marcheswan}
%%
\begin{tabular}{@{} r l c @{}}
\toprule
\multicolumn{3}{c}{\large\textsc{Menses Hegirae Muhamedicae}} \\
\multicolumn{3}{c}{\large\textsc{soluto cyclo hagareno}} \\
\toprule
  ~ &
  ~ & 
  \hts{\ch{Character}{Cha\-rac\-ter mensium}}
  \\
\midrule
 \textarabic{مُحَرَّم}[?]        & Machurramu          & 0 \\
 \textarabic{صَفَر}[?]         & Tzepharu            & 2 \\
 \textarabic{رَبيع الأوّل}[?]   & Rabie prior         & 3 \\
 \textarabic{رَبيع الثاني}[?] & Rabie posterior     & 5 \\
 \textarabic{جُمادى الأولى}[?] & Giumadiun prior     & 6 \\
 \textarabic{جُمادى الآخرة}[?] & Giumadiun posterior & 1 \\
 \textarabic{رَجَب}[?]         & Regiabu             & 2 \\
 \textarabic{شَعْبان}[?]       & Sahabenu            & 4 \\
 \textarabic{رَمَضان}[?]       & Ramadhanu           & 5 \\
 \textarabic{شَوّال}[?]        & Schevvalu           & 7 \\
 \textarabic{ذو القعدة}[?]   & Dulkaidathi         & 1 \\
 \textarabic{ذو الحجة}[?]    & Dulchagiathi        & 3 \\
% \textarabic{\HijriMonthArabic{ 1}} & Machurramu          & 0 \\
% \textarabic{\HijriMonthArabic{ 2}} & Tzepharu            & 2 \\
% \textarabic{\HijriMonthArabic{ 3}} & Rabie prior         & 3 \\
% \textarabic{\HijriMonthArabic{ 4}} & Rabie posterior     & 5 \\
% \textarabic{\HijriMonthArabic{ 5}} & Giumadiun prior     & 6 \\
% \textarabic{\HijriMonthArabic{ 6}} & Giumadiun posterior & 1 \\
% \textarabic{\HijriMonthArabic{ 7}} & Regiabu             & 2 \\
% \textarabic{\HijriMonthArabic{ 8}} & Sahabenu            & 4 \\
% \textarabic{\HijriMonthArabic{ 9}} & Ramadhanu           & 5 \\
% \textarabic{\HijriMonthArabic{10}} & Schevvalu           & 7 \\
% \textarabic{\HijriMonthArabic{11}} & Dulkaidathi         & 1 \\
% \textarabic{\HijriMonthArabic{12}} & Dulchagiathi        & 3 \\
 \bottomrule
\end{tabular}
%
\caption{Menses Hegirae Muhammedicae soluto cyclo hagareno}
\label{tab:p138}
%
\end{tabnums}
