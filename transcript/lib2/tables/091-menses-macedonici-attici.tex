%%% Liber II p91, PDF 174
%%
%% Table that gives the names of the month in the Macedonian
%% and Attic calenders.
%%
%% The original table does not have a title
%%
%% The names of the month are very hard to read in the PDF scan.
%% The table is small and the scan tends to blurr the details
%%
%% From wikipedia, Ancient Greek calendars:
%%
%% Macedonian: 
%%    Dios - Δίος
%%    Apellaios - Ἀπελλαῖος
%%    Audunaios or Audnaios - Αὐδυναῖος or Αὐδναῖος
%%    Peritios - Περίτιος
%%    Dystros - Δύστρος
%%    Xandikos or Xanthikos - Ξανδικός or Ξανθικός
%%    Artemisios or Artamitios - Ἀρτεμίσιος or Ἀρταμίτιος
%%    Daisios - Δαίσιος
%%    Panemos or Panamos - Πάνημος or Πάναμος
%%    Loios - Λώιος
%%    Gorpiaios - Γορπιαῖος
%% -  Hyperberetaios - Ὑπερβερεταῖος
%%
%% The '-' indicates first month listed in the table in de Macedonici column
%%
%% This table is in essence a rehash of 084_menses_tetraeterici
%% using the first column from that table as the Attic
%% and with a double line for δαίσιος/σκιῤῥοφοριών
%%
%%% Count out columns for fixed-width source font
% 000000011111111112222222222333333333344444444445555555555666666666677777777778
% 345678901234567890123456789012345678901234567890123456789012345678901234567890
%
%% Select a general font size (uncomment one from the list)
%\tiny
%\scriptsize
\footnotesize
%\small
%\normalsize
%% Center the whole table left-right
\centering
%% Modify separation between columns
%\setlength{\tabcolsep}{1.6pt}
%% Modify distance between rows
%\renewcommand{\arraystretch}{1.3}
%%
\begin{tabular}{@{}l l@{}}
\toprule
 Menses Macedonici    & Menses Attici \\
\midrule
 \textgreek{ὑπερβερεταῖοσ} &
 \textgreek{πυανεψιών}
\\
 \textgreek{δίοσ} &
 \textgreek{μαιμακτηριών}
\\
 \textgreek{ἀπελλαῖος} &
 \textgreek{ποσειδεών}
\\
\midrule
 \textgreek{ἀυδυναῖος} &
 \textgreek{γαμηλιών}
\\
 \textgreek{περίτιος} &
 \textgreek{ἀνθεστηριών}
\\
 \textgreek{δῦστρος} &
 \textgreek{ἐλαφηβολιών}
\\
\midrule
 \textgreek{ξανθικός} &
 \textgreek{μυονυχιών}
\\
 \textgreek{ἀρτεμίσιος} &
 \textgreek{θαργηλιών}
\\
 \textgreek{δαίσιος πρότερος} &
 \textgreek{σκιῤῥοφοριών πρότερος}
\\
 \textgreek{δαίσιος δευτερος} &
 \textgreek{σκιῤῥοφοριών δευτερος}
\\
\midrule
 \textgreek{πάνεμοσ} &
 \textgreek{ἑκατομβαιών}
\\
 \textgreek{λῶος} & 
 \textgreek{μεταγειτνιών}
\\
 \textgreek{γορπιαῖος} &
 \textgreek{βοηδρομιών}
\\
\bottomrule
\end{tabular}
%
\caption{Menses Macedonici et Attici}
%