\chapter{Primus= Liber - De anno aequabili minore}

\mletter{A} \begin{parnumbers}
\dropcapil{9}{S}{i vervm}
est, quod sciscit Stoicorum schola, Tempus= esse normam rerum, \& custodiam, quia veritatis= index atque examen est, \& rerum gestarum memoriam, ac diuturnitatem posteritati tuetur: ij non vulgari laude digni sunt, qui temporum rationes= conscribere, atque fugitiuam antiquitatem retrahere conantur.
\\ \p
Qua in re cum tam priscis= scriptoribus=, quam æqualibus= temporum nostrum opera egregie nauata sit, dolendum tamen, aut
\mletter{B}
ferius=, quam oportebat, antiquos= sese ad id studium contulisse, aut pauciora ea de re monumenta, quam ab ipsis= auctoribus= relicta sunt, ad nos= peruenisse.

Nam vt omnia extent veterum Græcorum scripta, ea tamen paucorum temporum interuallum complectebantur.

Græcis= enim ante initia Olymiadum suarum nihil plane exploratum est: \&, quod dolendum est, de illorum scriptis=, quæ ad Chronologiam spectabant, nihil nobis= præter desiderium relictum est.

Nam quæ Eusebij exstant, quamuis= è Græcorum monumentis= hausta sunt, \& multa egregia ac cognitu digna nobis= conseruarunt: tamen dissimulandum non est, multa in illis= reperiri, quæ castigatioribus= iudiciis= non satis=faciant.

\mletter{C}
Quod si Thalli, Castoris=, Phlegontis=, Eratosthenis= canones= exstarent, perparua, aut nulla potius= ratio haberetur librorum quorundam, qui hodi in penuria meliorum nobis= in pretio sunt.

Apud Romanos= vero, ea scriptio infeliciter cessit, quod eam cognitionem ferius= amplexi sint.

Nam ante Consulatum Bruti nihil certi apud illos=: omnia fabulosa: \&, si rem propius= spectemus=, ne ipsius= quidem Bruti Consulatum, ac tempus= Regifugij satis= exploratum habent.

\end{parnumbers}
\clearpage
p. 2 [pdf 85]

\begin{parnumbers}

quamius=, vt prodidit Censorinus=, Varro collatis= diuersarum ciuitatum temporibus=, \& interualla retexens=, verum in lucem protulerit, \& viam reperit, qua certus=
\mletter{A}
annorum Vrbis= conditæ numerus= iniri posset.

Sed, vt suo loco disputabitur, non magis= constabat Varroni de initiis= Vrbis=, quam Græcis= de anno excidij Troiæ.

Nam ea demum est vera demonstratio, quæ cogit, non quæ persuadet.

Soli sacri libri supersunt, ex quorum fontibus= certa temporum ratio hauriri possit.

Sed omnis= temporum cognitio inutilis= est, nisi certa epocha in illis= deprehendatur, ad quam omnium temporum contextus=, tam antecedentium, quam consequentium referri possit.

Nam, vt præclare dixit vetus= inter Christianos= scriptor Tatianus=, apud quos= temporum notatio non cohæret, apud illos= neque veritatis= \& fidei historicæ ratio vlla constare potest.

Quod si aliquis= sacræ historiæ peritissimus=, hoc est, qui interualla rerum gestarum
\mletter{B}
nobilissima certissimis= ratiociniis= ex Mose, \& reliquis= sacris= Bibliis= explorata habeat, nihil tamen ex illis= a certam epocham historiæ Græcæ, aut Romanæ referre possit: quodnam adiumentum is= ex eiusmodi diligentia adferre potest aut sibi, aut studiosis= rerum antiquarum?

Nam omnis= cognitionis= finis= ad vsum aliquem spectat, quem si ex medio literarum sustuleris=, ingratus= est omnis= labor \& opera, quæcunque in omne studium impenditur.

Eiusmodi est Iudæorum scientia, qui in ratiociniis= quidem sacrorum temporum colligendis= tantum studio \& diligentia consecuti sunt, vt proxime à veritate abesse dici possint: sed dum nullam aut saltem deprauatam rerum extrarum cognitionem tenent, multum errant, quod sine externa historia sacram tractare
\mletter{C}
aggrediuntur.

Venio ad nostros=, recentiores= dico, qui hodie summo cum fructu, sacræ, Græcæ, \& Romanæ historiæ tempora digesserunt.

Ij heroica virtute chronologiam negligentia \& contemtu maiorum intermortuam ac sepultam, è tenebris= \& obliuionis= silentio quotidie eruere conantur.

Certe meum semper iudicium fuit, eam rem maiore cum laude ab illis= restitutam, quam ab antiquis= proditam fuisse.

Nam non solum pleraque in ratione temporum pristinæ integritati reddiderunt, sed \& longe meliora effecerunt.

In multis= tamen iudicium, in quibusdam etiam diligentiam requiro.

neq; enim dum verum adepti sunt.

Argumento suerint omnium, quotquot de his= rebus= tractarunt, dissensiones=: vt inter tot millia Chronologorum vix inter duos= de eadem re
\mletter{D}
conueniat.

Quanta adhuc contentione de Septimanis= Danielis=, de initio, medio, \& fine earum velitantur?

Tamen nihil plane eorum, quæ volunt, assecuti sunt.

Ab eorum lectione incertior atque indoctior sum, quam dudum.

Quis= vnquam eorum veram epocham Exodi Habræorum; quis=, quod pudendum est, verum annum natalis= Dominici odoratus= est?

Ecce trita, obuia, vulgaria, vt nobis= videtur, ignoramus=, \& remotiorum ac reconditiorum indicium promittimus=!

\end{parnumbers}
\clearpage
p. 3 [pdf 86]

\begin{parnumbers}

Quis= eorum Danielis= \mletter{A} Hebdomadas= interpretandas= suscepit, qui inscitiæ suæ latebram non quæsiuerit, \& reges= Persidis=, qui nunquam in rerum natura fuerunt, non commentus= sit?

Quod si Danielem accuratissime legissent, eis= ad negotium explicandum non aliis= regibus= Persidis= opus= fuisset, quam iis=, quos= Herodotus=, Diodorus=, \& omnis= Græcorum antiquitas= nouit.

Sed quo non progressa est [Greek]?

Berosos=, Metasthenes=, \& nescio quos= Catones=, ac Philones= consulunt, qui ante hos= centum annos= ex officina nescio cuius= indocti \& impudentis= prodierunt.

Et sese Criticos= in temporum notatione profitentur, quibus= tam facili genere, tam pueriliter vnus= homo otiosus= in tanta luce literarū quotidie imoponit.

\mletter{B} Cuius= hominis= inscitiā si nihil aliud, certe illud arguere possit, quod Metasthenem pro Megasthene posuit. Si Iosephum Græce, aut Strabonem, aut Athenæum legisset, is= Megasthenem vocari deprehendisset, quem Metasthenem vocat.

Si Græce scisset, numquam [Greek] in illa lingua reperiri, neque hanc compositionem in eadem probari intellexisset.

Vt igitur ij resipiscant, qui \& nouos= reges= in Perside crearunt, \& Assueros= Priscos=, Assueros= Longimanos=, Assueros= Pios=, duos= Cyros=, \& nescio quæ alia somnia Annij Viterbiensis= in medium producunt, primum vno verbo indicabo fontem erroris= eorum: deinde qui medicina huic morbo fieri possit, docebo.

Quod igitur in veri inuestigatione \mletter{C} eos= ratio fugerit, duas= summas= causas= reperio: vnam, quod veterum tempora ciuilia, annorum, mensium formas=, status=, ac genera ignorarunt: alteram, quod characterem, \& notationem ei anno, quem sibi proposuerant, non adhibuerunt.

Ex vtraque quidem causa temporum confusio manauit, sed diuerso genere.

Ex priore causa ignoratus= est annus=, mensis= \& dies= multarum nobilium epocharum.

Huius= enim rei cognitio pertinet ad tempus= ciuile nationum.

Ex altera causa Palilia vrbis= Romæ nunc tertio anno Olympiadis=, nunc quarto attribuuntur.

Item Consulatus= Bruti nunc in hunc, nunc in illum annum Olympiadis= confertur.

Vt igitur nouam rationem emendationis= temporum ineamus=, duo illa præcipue nobis= discutienda sunt: sed 
prius= \mletter{D} de omnium nationum temporibus= ciuilibus=: quam assequi perdifficile est, nisi prius= tempore in sua principia, hoc est ab annis=, periodis=, mensibus= in vltimum terminum, dies=, horas= ac scrupula resoluto.

Nam qui ante nos= hanc prouinciam aggressi sunt, si modo hanc nostram, non aliam aggressi sunt, ij satis= de tempore, \& eius= natura disputarunt.

Sed hanc disputationem melius= interpres= [Greek] sibi vindicasset.

Neque vero nos= id agimus=, vt difiniamus= tempus= esse hoc secundum Peripateticos=, aut illud seundum Stoicos=, aut Academicos=.

\end{parnumbers}
\clearpage
p. 4 [pdf 87]

\begin{parnumbers}

Qui istis= definitionibus= diu immorati sunt, \& hac sola scientia Chronologiæ scribendæ modum terminarunt, illi fatis= \mletter{A} verborum quiedem, sed rerum nihil definiuerunt.

Nequid tamen [Greek] transigatur, decreui singularum, vel minimarum temporis= partium prius= conspectum aliquem dare, quam ad descriptionem [Greek] temporum ciuilium, \& eorum methodum aggrediar.

Incipiam igitur ab vltimo termino, a die scilicet, \& eius= partibus=, hoc est hora, \& scrupulis=.

Ab hora igitur, si libet, principium esto.
\end{parnumbers}

\subsection{De Horis= et partibvs= diei reliqvis=.}
\setcounter{parcount}{0}
\begin{parnumbers}

\dropcap{3}{V}{eteribus} statim ab initio has= diei partes=, quas= H O R A S vocamus=, in vsu non fuisse, argumento fuerint priscæ locutiones=, \mletter{B} quibus= dies= non in partes= secatur, sed actionibus= quotidianis= distiguitur: vt cum [Greek] vesperam vocabant, nimirum, vt poëta inquit, Demeret emeritis= cum iuga Phœbus= equis=.

Item quod tempus= antemeridianum disignantes= dicebant [Greek] vel [Greek], conuenientibus= scilicet eo tempore in Comitium viris=: vt Hesiodus= dicit, [Greek].

Quod tamen longe aliter interpretes= Græci illius= poëtæ exponunt.

Aiunt enim Hesiodum intellexisse de tricesima mensis= Lunaris=: \& sensum loci Hesiodei esse perinde ac si dixisset, Quando homines= veram [Greek] Lunarem agunt, \& non secundum vsum politicum, sed secundum motum Lunæ.

Quod \mletter{C} tamen nobis= valde coactum videtur: \& mentem Hesiodi hanc fuisse dicimus=: [Greek] esse valde idoneam rebus= gerendis= ea hora, qua homines= ad ius= in forum conueniunt.

Homerus= Odyss. [Greek]

[Greek]

[Greek]

Quæ sane interpretatio melior vulgari.

Sic etiam paulo post dicit, [Greek], loquens= de vndecima: cuius= partem designat, cum dicit [Greek].

Quod nos= interpretamur iam adulto die.

Sic Homerus= meridiem designat, [Greek].

Porro neq; hoc verbum [Greek] id, quod nunc, valebat.

Sed tempus= actuum quotidianorum illo notabatur: vt cum dicebant [Greek].

\mletter{D} Latinis= vero Tempestas= dicebatur.

In Legibus= Decemuirum Atticis= fuit: SOL OCCASVS SVPREMA TEMPESTAS ESTO.

Neque recte quidam hinc expungunt TEMPESTAS. quod SVPREMA absolute diceretur, vt apud Plautum.

Nam plane in legibus= Solonis=, vnde illud caput traductum, scriptum fuit, [Greek].

Stoicus= scriptor apud Stobæum loquens= de Socratis= iudicio capitali: [Greek Greek Greek].

Idem censeas= de veteribus= Hebræis=, \mletter{A} qui diei nullas= alias= partes=, quam mane, meridiem, \& vesperam norant. \& ita dies= diuiditur Psalmo L V, commate X V I I I.

\end{parnumbers}
