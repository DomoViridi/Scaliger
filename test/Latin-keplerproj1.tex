\documentclass{book}
\setcounter{secnumdepth}{6}
\setcounter{tocdepth}{3}

\usepackage[utf8]{inputenc}
\usepackage[LGR,T1]{fontenc}
%\usepackage{textcomp}
%  \usepackage{newtxtext,newtxmath}
%  \usepackage{substitutefont}
%  \substitutefont{LGR}{\rmdefault}{artemisia}

% Kepler Fonts: Write with long s and other old-style typesetting
% from http://tex.stackexchange.com/questions/9495/latex-font-for-18th-century-english
% also http://ctan.org/pkg/kpfonts
\usepackage[veryoldstyle]{kpfonts}

% load alphabeta after math setup and encoding setup!
\usepackage{alphabeta}

\usepackage[greek,latin,english]{babel}

% Make no-break space known to LaTeX
\DeclareUnicodeCharacter{00A0}{ }

% Trick to number paragraphs with no chapter/section/subsection numbers
\usepackage{mparhack}   % get marginpars to always show up on the correct side (need to compile twice)
\usepackage{lipsum}     % for dummy text

\setlength\parindent{0cm}

\newcommand{\parnum}{\arabic{parcount}).}

% Special paragraph counter which resets on each new page
\newcounter{parcount}[page]
\newcommand\p{%
    \refstepcounter{parcount}%
    \parnum \hspace{1em}%
}

\newenvironment{parnumbers}{%
   \par%
   \everypar{\noindent \stepcounter{parcount}\parnum \hspace{1em}}%
}{}


\begin{document}
\part{The first book}

\chapter{The first chapter}

\section{First section}

\begin{parnumbers}
\lipsum[1-17]

\end{parnumbers}

% Alternative way to get numbered paragraphs, using the same trick
\p \lipsum[100]

\paragraph{Yes=, we have a first paragraph.}
\subparagraph{A subparagraph looks like this.}

\paragraph{A second paragraph}

\paragraph{And a third one, with several long esses=.}

A line of text that is= not a paragraph.

Inserting a blank line marks= a new paragraph in {LaTeX}, but it is= not marked
as= a paragraph.

And a third line separated with a blank line.

\paragraph
But what happens= if we don't pass= a parameter to the paragraph command, and start typing on the next line.
\paragraph Or even start typing after the paragraph command.
\paragraph It looks= like it tries= to make the first letter of the line a special one.
\paragraph  So we try to put an extra space between the command and the line.
\paragraph{} That does= not work, so let's try an empty parameter to the command. Succes=!

Thanks= to the kpfonts= package we can have old-style typesetting and
long esses= while typing the input as= simple flat text. Quick!

This= package also allows= us= to write \textsc{small caps} text, such as=
roman numerals=:\\
Regular:	XXXVII\\
small caps (2x) \textsc{XXVII x x v i i}\\
Lower case: abcdefg xxvii.

Original Kepler Fonts demo text:\\
Where Fractions= are wanting, a Division serves= to distinguish the
Numerator from the Denominator, by putting it thus=; viz.\ 3-8, 12-63, 
16-50, though some other symbol might serve better for the purpose; and
therefore we propose one that is= similar to an Italic~\textit{l}
inverted, and whose figure takes= in the whole depth of its= body.
Quite an achievement, don't you think?

\section{Greek}
Greek letters= using UTF-8 input encoding:\\
ΑΒΓΔΕΖΗΘΙΚΛΜΝΞΟΠΡΣΤΥΦΧΨΩ\\
αβγδεζηθικλμνξοπρςςτυφχψω\\
Alpha, beta, gamma, delta using macros=:\\
\alpha \beta \gamma \delta.

\end{document}
