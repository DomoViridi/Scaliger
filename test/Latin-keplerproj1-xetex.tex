\documentclass{book}

\usepackage{xltxtra} % Extra customizations for XeLaTeX;
% xltxtra automatically loads fontspec and xunicode, both of which you need

% FONTS

%\setmainfont{Junicode}

% Kepler Fonts: Write with long s and other old-style typesetting
% from http://tex.stackexchange.com/questions/9495/latex-font-for-18th-century-english
% also http://ctan.org/pkg/kpfonts
\usepackage[veryoldstyle,noamsmath]{kpfonts}

% load alphabeta after math setup and encoding setup!
%\usepackage{alphabeta}

%\setmainfont{Arial Unicode MS}
%\setmainfont{Kepler}
%\setmainfont{Kp-Regular}
\setmainfont{jkpmn8r.pfb}
%\setmainfont{jkpvosbmi.vf}
%\setmainfont{jkpvosbmi.tfm}
%\setsansfont{jkpssvos}

\usepackage{polyglossia}
\setdefaultlanguage{latin}
\setotherlanguages{greek,hebrew,english}

\newfontfamily\greekfont[Script=Greek]{Arial Unicode MS}

%%% Commands to number paragraphs
%% http://tex.stackexchange.com/questions/10513/automatically-assign-a-number-to-every-paragraph
%\usepackage{mparhack}   % get marginpars to always show up on the correct side (need to compile twice)
%\usepackage{lipsum}     % for dummy text

%\setlength\parindent{0cm}

%% Command \parnum to format the paragraph counter number (in bold, arabic numbers)
%\newcommand\parnum{%
%    \bfseries\arabic{parcount}%
%}

%% Special paragraph counter which resets with every 'subject' (represented as a subsection)
%\newcounter{parcount}[subsection]

%% Command for an isolated paragraph counter mark
%% Useful e.g. for big, multi-paragraph spanning Initials
%\newcommand\p{%
%    \stepcounter{parcount}%
%    \leavevmode\marginpar[\hfill\parnum]{\parnum}%
%}

%% Define an environment to use in the source.
%\newenvironment{parnumbers}{%
%    \par%
%    \everypar{%
%        \stepcounter{parcount}%
%        \leavevmode\marginpar[\hfill\parnum]{\parnum}%
%    }%
%}{}

% DOCUMENT
\begin{document}

\chapter{The first chapter}

\subsection{First subsection}

%\begin{parnumbers}
%\lipsum[1-17]

%\end{parnumbers}

% Alternative way to get numbered paragraphs, using the same trick
%\p \lipsum[100]

%\begin{parnumbers}
A line of text that is= not a paragraph.

Inserting a blank line marks= a new paragraph in XeTeX, but it is= not marked
as= a paragraph.

And a third line separated with a blank line.

Thanks= to the kpfonts= package we can have old-style typesetting and
long esses= while typing the input as= simple flat text. Quick!

This= package also allows= us= to write \textsc{small caps} text, such as=
roman numerals=:\\
Regular:	XXXVII\\
small caps (2x) \textsc{XXVII x x v i i}\\
Lower case: abcdefg xxvii.

Original Kepler Fonts demo text:\\
Where Fractions= are wanting, a Division serves= to distinguish the
Numerator from the Denominator, by putting it thus=; viz.\ 3-8, 12-63, 
16-50, though some other symbol might serve better for the purpose; and
therefore we propose one that is= similar to an Italic~\textit{l}
inverted, and whose figure takes= in the whole depth of its= body.
Quite an achievement, don't you think?

\section{Greek}
Greek letters= using UTF-8 input encoding:\\
%\begin{greek}[variant=ancient]
αβγδεζηθικλμνξοπρςςτυφχψω\\
ΑΒΓΔΕΖΗΘΙΚΛΜΝΞΟΠΡΣΤΥΦΧΨΩ\\
αβγδεζηθικλμνξοπρςςτυφχψω\\
%\end{greek}
%Alpha, beta, gamma, delta using macros=:\\
%\alpha \beta \gamma \delta.

Astrological symbols:\\
Moon Sun Mercury Venus Mars Jupiter Saturn\\
☾☉☿♀♂♃♄

Nice, eh?

%\end{parnumbers}
\end{document}
