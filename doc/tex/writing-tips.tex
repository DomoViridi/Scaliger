% XeLaTeX can use any Mac OS X font. See the setromanfont command below.
% Input to XeLaTeX is full Unicode, so Unicode characters can be typed directly into the source.

% The next lines tell TeXShop to typeset with xelatex, and to open and save the source with Unicode encoding.

%!TEX TS-program = xelatex
%!TEX encoding = UTF-8 Unicode

%%% Count out columns for fixed-width source font
% 000000011111111112222222222333333333344444444445555555555666666666677777777778
% 345678901234567890123456789012345678901234567890123456789012345678901234567890

\documentclass{report}

\usepackage{geometry}
%% See geometry.pdf to learn the layout options. There are lots.

\geometry{a4paper}         %% ... or a4paper or a5paper or ... 
%\geometry{landscape}      %% Activate for rotated page geometry

%\usepackage[parfill]{parskip}
%% Activate to begin paragraphs with an empty line rather than an indent
%\usepackage{graphicx}
%\usepackage{amssymb}

% Will Robertson's fontspec.sty can be used to simplify font choices.
% To experiment, open /Applications/Font Book to examine the fonts provided on Mac OS X,
% and change "Hoefler Text" to any of these choices.

\usepackage{fontspec,xltxtra,xunicode}
%\defaultfontfeatures{Mapping=tex-text}
%\setromanfont[Mapping=tex-text]{Hoefler Text}
%\setsansfont[Scale=MatchLowercase,Mapping=tex-text]{Gill Sans}
%\setmonofont[Scale=MatchLowercase]{Andale Mono}

\title{Writing tips}
\author{Hops Splurt}
\date{\today}                                           % Activate to display a given date or no date

\begin{document}
\maketitle

\tableofcontents{}

% For many users, the previous commands will be enough.
% If you want to directly input Unicode, add an Input Menu or Keyboard to the menu bar 
% using the International Panel in System Preferences.
% Unicode must be typeset using a font containing the appropriate characters.
% Remove the comment signs below for examples.

% \newfontfamily{\A}{Geeza Pro}
% \newfontfamily{\H}[Scale=0.9]{Lucida Grande}
% \newfontfamily{\J}[Scale=0.85]{Osaka}

% Here are some multilingual Unicode fonts: this is Arabic text: {\A السلام عليكم}, this is Hebrew: {\H שלום}, 
% and here's some Japanese: {\J 今日は}.

%%=====
\chapter{Project setup}
The aim of this project is to produce a modern version of
De Emendatione Temporum by Scaligeri.
At the very least we want an easily readable transliteration
in the original Latin and all the other languages used.
At best we produce
a full translation into English (and possibly other languages).

%---
\section{Interpreted Transcription}
In this first stage, we intend to transcribe the scanned PDF original(s)
into a computer file which can be searched, rendered
and modified.

Originally we wanted to stay as close as possible to the original
in the first stage.
This turned out to be a lot more work than we bargained for.
As a result the priority has shifted to producing
an interpreted transcription first.

In this stage we create a document which has the same text as the original,
but in a modern script.
This document should be easy to read for the modern reader who knows
Latin and Greek.

\begin{itemize}
\item All ligatures and diacriticals are expanded (e.g. æ to ae).
\item Long s is replaced with modern s.
\item Abbreviated words (e.g. \verb+quoq;+) are written in full (\verb+quoque+).
\item Interpunction is done in modern style, and more consistent.
\item Often sentences starting with a \verb+q+ won't have this q capitalised.
We will start each sentence with a capitalised word.
\item All Greek phrases will be rendered using modern polytonic Greek
script with Unicode encoding.
\end{itemize}

\subsection{Line number references}
To correlate this transcript to the original, the lines
in the original PDF will be numbered.
The numbers will start with 1 on each page, and this and every 5th line
will be marked with its number in the margin.

In the transcript each beginning of a line will be marked with that number
in parentheses. The first sentence on each page will also give the page
number in that mark.

As a result each line in the transcript can easily be found in the original,
and vice versa.

\subsection{Extra features}
Features will be added that are not in the original text.
\begin{itemize}
\item The text will be divided into paragraphs to the best of our insights.
\item Headers and footers will be generated in a modern style.
\item Table of Contents, Index, Glossary and Bibliography will be generated
automatically.
\item List of Tables and List of Figures will be added.
\item Lots of annotations will be added to clarify the text.
For example: Scaligeri habitually quotes classics without giving a reference.
We will add:
    \begin{itemize}
    \item The exact reference (Author, title of the work, section/page/verse)
    \item If possible a link to an on-line version.
    \item If misquoted, the actual quote.
    \item An explanation why this quote is appropriate.
    \end{itemize}
\end{itemize}

Note that before we get to the fancy stuff, the initial big work will be
 the actual transcribing,
i.e. reading from the scanned original and typing in the Latin text.

%---
\section{Litteral Transcription}

To make references to the original easier, we plan to produce a version that
resembles the original as closely as possible.

This includes:
\begin{itemize}
\item Using a font that has the old-style ligatures (such as 'st' and 'ct'),
the long s, and preferably a Q with a swish.
These features should be available with as little special encoding as possible.
This will require the selection of a suitable font.
\item A page in the transcription should match a page in the original and
vice versa.
If necessary we will insert an
extra page in the transcription to hold the bit of text that does not fit
on one page.
Preferably we prevent this problem
by making sure a transcription page can hold as least as much text as an
original page, by adjusting the margins
and the font size.
\item Though it is very hard to match the newlines in the original, an effort
should be made to maintain a close relation with the original.
This way the line numbering we intend to add to the PDF scan can be maintained
in this transcription.
An effort should be made to find a way to force XeLaTeX to put a linebreak
in a word.
\item The letters in the margin of the original should also be in the
transcription, for easy cross-reference.
\item All other languages (Greek, Hebrew, Arabic, Persian?, more?)
and symbols (planet symbols) must be rendered
as closely as possible to the original.
\item The original Greek uses many, many, many ligatures.
Preferably the transcription shows the same ligatures.
This will require a search for (and possibly implementation of) a specialised
font for this purpose.
\end{itemize}

%---
\section{Translation}
A translation of all the text.
A reader should be able to read and understand everything without knowledge
of ancient languages.
The layout will be in a similar modern style as the transliteration.
In addition there will be many notes 
(footnotes?) giving clarifications and annotations to the vagaries of the
translation process.

In principle each phrase in the original will correspond to a phrase in the
translation, but this is not a strict rule.
We may even change our mind and go for a smooth running translation.

%%=====
\chapter{Tools}
\XeLaTeX{} part of \TeX{} package. Editing source with TeXShop. Can also
be done with any other editor of flat text. 
Compilation must be done with \texttt{xelatex}. Can be done from TeXShop.
Note: use the Typeset button on the root document
(which has the \verb;\documentclass{}; command in it).
Sources kept on \texttt{GitHub}.

%%=====
\chapter{Interpreted transcription}
\section{Textual cleanup}
\begin{itemize}
\item All ligatures are expanded
	\begin{enumerate}
	\item æ to ae
	\item Æ to Ae (e.g. Aegyptica)
	\end{enumerate}
\item Long s is replaced with modern s. Tricky, as the long s looks almost
exactly like an 'f'.
\item Diacriticals will be expanded
	\begin{itemize}
	\item ā to am (sometimes an)
	\item ō to on (sometimes om)
	\item ē to en (sometimes em)
	\item à (as a whole word) to ab
	\item ë in poëta will be unchanged for now
	\end{itemize}
\item Abbreviated words (e.g. \verb+quoq;+) are written in full (\verb+quoque+).
\item Interpunction is done in modern style, and more consistent.
\item Often sentences starting with a \verb+q+ won't have this q capitalised.
We will start each sentence with a capitalised word. Sometimes other initial
letters won't have been capitalised either.
\item All Greek phrases will be rendered using modern polytonic Greek
script with Unicode encoding.
This has about 49 base glyphs (24 upper case, 24 normal lower case
plus the word-final sigma) and about 233 extra glyps for
the polytonic diacriticals and iota subscripts.
The original uses a Byzantine Minuskel script which has about 1500 glyphs.
\end{itemize}

\section{Special things to pay attention to}
\begin{itemize}
\item Numbers with fractions: Put a non-break space \verb+~+ between the number
and the fraction, e.g. \verb+7~1/2+ to give 7~1/2.
\item Reading the interpunction can be very tricky where numbers are involved.
Often there will be a period after an arabic or roman number, while the
construction of the phrases suggest that it is not intended as the end of 
the sentence. E.g. Liber Primus, p 17, line 10 and further.
\end{itemize}

\section{Extra features to be added}
\begin{itemize}
\item Text will be divided into paragraphs to the best of our insights.
\item Headers and footers will be generated in a modern style.
\item Table of Contents, Index, Glossary and Bibliography are generated
automatically.
\item List of Tables and List of Figures added.
\item Lots of annotations to clarify the text.
For example: Scaligeri habitually quotes classics without giving a reference.
We will add:
    \begin{itemize}
    \item The exact reference (Author, title of the work, section/page/verse)
    \item If possible a link to an on-line version.
    \item If misquoted, the actual quote.
    \item An explanation why this quote is appropriate.
    \end{itemize}
\end{itemize}

\section{Line number references}
To correlate this transcript to the original,  the lines
in the PDF of the original will be numbered.
The numbers will start with 1 on each page, and every 5th line will be marked
with its number in the margin.
In the transcript each beginning of a line will be marked with that number
in parentheses.

Study of how lines are commonly numbered in long transcribed documents
will be required.

The line numbers in the PDF will have the form of annotations. To save work
a program has been developed to automaticaly generate the numbers in roughly
the correct position for sets of pages. Note that the original has over 900
pages of text, and each page requires 9 notes (sometimes less), requiring
about 8100 notes to be generated.

Doing this by hand in a PDF reading program (like Acrobat Reader or OSX Preview)
would be tedious, especialy since these programs become extremely slow when
trying to handle annotations.

\subsection{Commands for line numbers}
Commands were created to add the line- and pagenumbers:
\begin{itemize}
\item \verb+\lnr{<line number>}+ to add the line number to the beginning of
a sentence.
\item \verb+\plnr{<PDF page>}{<page nr>}{<line nr>}+ to use at the beginning of the first sentence on a new page in the source. Parameters are:
\begin{enumerate}
\item The PDF page number of the page in this particular scan of this particular
copy of this particular edition of the book.
This value is only used internally for now. Intended to be used to
make links to the original PDF file.
\item The page number as shown on the page. This will be printed in the output.
\item The line number in the original where de sentence starts.
\end{enumerate}
\item \verb+\Rplnr{<PDF page>}{<page nr>}{<line nr>}+. This is the same as the
 \verb+\plnr+ command, except that the page number will be printed as a roman
 numeral rather than an arabic number. Used for chapters that use roman numerals
 as page numbers, such as the Prolegomena.
\end{itemize}

\section{General guidelines for the \XeLaTeX code}
The following guidelines are intended to make it easier to compare the
tex file to the original.
\begin{itemize}
\item A new sentence in the original should start at a new line in the tex file.
\item The first word on a line in the original should be at the start of a line
in the tex file.
\item No line in the tex file should be longer than 80 characters.
\item If the above rules are followed, sometimes a tex line will come out
longer than 80 characters. In that case, split the line in two (or more),
where each extra line is indented by one space.
\end{itemize}
Note that \LaTeX treats a new line as if it is a space. This means that
breaking up a line in the tex file will not affect the outcome.
A new paragraph is marked by a double newline (i.e. an empty line) in the
tex file.

\subsection{Wrapping text around figures}
Attempts to use \verb+wrapfigure+ to make the text flow around figures were
more trouble then its worth. In particular, the text would not start to
flow until a new paragraph is started. Because we seldom start a new paragraph,
we would have to start a paragraph on purpose (causing an indent) just for the
wrapping. We will simply put the figures on their own, without wrapping.

\section{Specific issues per chapter}
\subsection{Prolegomena}
\begin{itemize}
\item
The pages are numbered with roman numerals rather than arabic numbers.
The \verb+\Rplnr{}{}{}+ command
was created to print the page numbers in roman numerals in the text.
\item
The Prolegomena in the original is basicaly a fifty page wall of text. Some attempts have been made to find places where the text changes the subject and insert a new paragraph there, but further work is needed. We may want to
introduce sections or subsections. See Issue \#12.
\end{itemize}

\subsection{Liber Primus}
\begin{itemize}
\item
There is frequent use of "à" as a word. After some consideration this
is transcribed as the word "ab". Each occurence is marked in comment.
\item
The word "scrupulus" (Lit: "small pebble"; also: 1/24 of an ounce)
is often abbreviated. To expand these, the proper
declination must be determined in each case. See Issue \#2 on GitHub.
\item
Wrapping text around figures turned out to be impractical. We now will
simply use a floating object for figures and tables.
\item
The tables imported from the litteral transcription can be simplified
a lot, because we no longer need to immitate the original. E.g. the headers
for the Ostenta/Sexagesima conversion tables (Table 1.1) can now be written
as a simple line of text.
\item
For the same tables: the text in it can be a lot bigger, as the tables
will now appear on a separate page anyway.
\item
To fine-tune the placement of tables and figures it turned out to be
neccesary to move the location where they are defined slightly foreward or
backward in the tex file. To make this easier, we put the code for tables
 (and some figures) in separate files.
This way we don't have to cut and paste large pieces of code if we want to
move the definition of the table or figure. We now only have to move the
\verb+\include{}+ command.
\item
All tables and figures will be given a caption and a label, so they can
appear in a List of Tables/Figures, and can be referred to from within the
text.
\item
The \verb+\begin{blockquote}...\end{blockquote}+ command has shown to be
very useful to represent multi-line Greek quotes, e.g. p. 4, line 21.
\item
The word "poëta/poëtae" is used on page 4, lines 7 and 14. We are not sure
what to do with the trema.
\item
The image on page 11 (Figure 1.1: Lapis Levinii). We made a cleaned-up copy
from an enlarged PDF page by editing nearly at the pixel level. It looks like
Scaliger had an representation made for his book using typeset letters.
E.g. the text "A latere dextro saxi" is obviously not on the stone itself.
There is no indication
of how he knew what the original looked like.
\end{itemize}

%%=====
\chapter{Litteral transcription}
\section{Entering text}
\subsection{Flat text}
Enter the text as simple flat text.
Most of the text will consist of Latin,
with bits of Greek (words, short phrases, sometimes longer quotes) and words
in Hebrew, Arabic or Persian.
In the Latin body text there are several special characters:
\begin{itemize}
\item The long s: 'ſ'. This can simply be entered as an 's'. The right glyph
will appear due to the selected font features.
\item The ampersand: '\&'. This must be escaped in the source with a backslash,
like so: \verb;\&;
\item Ligatures 'æ' and 'œ': These can be entered as Unicode characters.
On a Mac~OS keyboard use \verb;Alt-'; for æ and \verb;Alt-q; for œ.
\item The double-s character 'ß', sometimes seen in italic text (e.g. PDF p.45).
Enter this with Alt-s.
\item The 'ct' ligature, where a little loop is added that connects the tops
of the c and the t.
This is handled by a font feature.
You can simply enter 'ct' in the source.
Note that the original does not contain any 'st' ligatures in the same style
as the 'ct' ligature, but many fonts will try to render them anyway.
\item Other ligatures (ffi, ſt, etc): these are handled by the font and need
not be entered as special codes in the source.
Simply use the sequence of normal characters (\verb;ffi;, \verb;st;, etc).
\item  Capital Q with swash: this is handled by the font features.
Simply type a Q in the source.
\item Vowel with bar ā, ē, ī, ō, ū: These can be entered as Unicode characters.
The diacritical bar (macron) can be entered on Mac~OS (U.S. Extended or
ABC Extended keyboard must be
selected\footnote{The U.S. Extended keyboard layout was renamed to
ABC Extended with the introduction of Mac OS X 10.11 El Capitan})
with Alt-a followed by the vowel.
\item Vowel with ogonek ę: This can be entered as a Unicode character.
Can be entered on Mac~OS (U.S. Extended or ABC Extended keyboard must be
selected\footnotemark[\value{footnote}]) with Alt-m followed by the vowel.
\item It appears that sometimes a final 's' in a word is printed as what looks like a small question mark. E.g. Prolegomena p.XXIII (PDF p.50) has "Ioſephu?" and "Ioſepho?" near the bottom. For the time being, try to use a question mark here.
\end{itemize}
Some special text must be surrounded by commands:
\begin{itemize}
\item Roman numerals: \textsc{vii}.
Enter them as lower case characters, surrounded by the \verb;\textsc{};
command to make them small caps.
The extra spacing between the characters is handled with a font feature.
\item Greek, Hebrew, Arabic and Persian: First enter these as a placeholder,
like so:
\begin{itemize}
\item\verb;\greektext{[Greek]};
\item\verb;\hebrewtext{[Hebrew]};
\item\verb;\arabicfont{[Arabic]};
\item\verb;\arabicfont{[Persian]};
\end{itemize}
There is no specific command for Persian, as it uses the same font
and settings as Arabic.
\\
The commands will handle the selection of a suitable font for that language. In
later phase the placeholders will be filled in with the actual text, which can
be entered as Unicode characters.
\\
For the Greek ligatures we will probably 
have to resort to a special coding scheme, such as \verb+2<epsilon><iota>B;+ for
"a 2 character ligature representing <epsilon><iota>,
using the second ('B') of the available ligatures".
\item Astrological symbols: For the astrological symbols used to represent the
sun, moon and planets there is the \verb;\astro{}; command, which will select
a suitable font. The symbols themselves can be entered in Unicode.
\begin{table}[h]
\begin{center}
\begin{tabular}{c | c}
\begin{tabular}{r l}
Sun & U+2609 \\
Moon & U+263E \\
Mercury & U+263F \\
Venus & U+2640
\end{tabular}
&
\begin{tabular}{r l}
Mars & U+2642 \\
Jupiter & U+2643 \\
Saturn & U+2644 \\
\parbox[c]{1cm}{}
\end{tabular}
\end{tabular}
\end{center}
\end{table}%

\end{itemize}

%
\subsection{Chapter titles}
The multi-line chapter titles can be immitated using the \verb;\head;
command. It has three parameters:
\begin{enumerate}
\item {\ttfamily\bfseries scalesize}: A factor by which the size of
the font is multiplied. Usual values: 1.5, 2.0, 3.0, 3.3.
\item {\ttfamily\bfseries spacing}: Influenzes the spacing between letters.
The value is passed to the LetterSpacing fontfacility.
Usual values: 25, 35, 40, 50, 60.
\item {\ttfamily\bfseries text}: The text to put on the line.
\end{enumerate}
Steps to make the lines match the original:
\begin{enumerate}
\item Set up a frame for each line of the form
 \verb;\vspace{0mm}\head{1.0}{0}{}\\;
\item Surround the block with a command for small caps
(\verb;{scshape … };) and for centering (\verb;\begin{center} … \end{center};)
\item Fill in the text for all lines in the 3rd parameter of the \verb;\head;
commands.
\item For each line, adjust the scale so the height of the result matches 
the height of the original.
\item For each line, adjust the spacing so the width of the result matches
the width of the original.
\item Working from top to bottom, adjust values of the \verb;\vspace; commands
to match the vertical position of each line to the original. For the first line
use a negative value (usually -18mm) to move the line up.
\end{enumerate}
Spaces in these title lines will not come out well, because the LetterSpacing
does not make the space itself wider. This gives the impression that as the
spacing between the characters gets bigger, the space character disappears.
Replace spaces with \verb;\hspace{1.0em};.


%
\subsection{Drop caps}
Extra large initials in the first line of a chapter or section can be made
with one of two commands: \verb;\dropcap; for normal drop caps,
and \verb;\dropcapil; for illuminated drop caps.

Both have the same three parameters:
\begin{enumerate}
\item \texttt{lines}: The desired hight of the dropcap expressed as
a number of lines.
Count the lines in the original to find this value.
\item \texttt{initial}: The initial letter.
\item \texttt{caps}: The short bit of text after the initial which will be
set as small caps.
The first of these is usually upper case, the rest lower case.
\end{enumerate}
After setting up the drop cap you may find several lines of the text running
into the drop cap. You can solve this by making artificial paragraphs by
using \verb;\\ \p; instead of a blank line between the paragraphs. Do this
for enough lines to clear the drop cap.\footnote{This problem might be solved
by cooking our own lettrine command. This should be easy since we are already
using the \texttt{wrapfig} package.}

%
\subsection{New page}
A new page in the original should be reflected in the transcription by a
\verb;\clearpage; command. This puts a page break in the transcription.
Also put a line with the page number and PDF source page number in the form:\\
\indent\texttt{p. 2 [pdf 85]}\\
Make sure there are no blank lines around this command and this line. Otherwise
the paragraph counter will be incremented. You may want to put a comment line
here, which will show up as a marker in a different colour in a suitable
source editor.

%---
\section{Common error messages}
\begin{description}
\item[Undefined control sequence.] \hfill \\
A common or normal command such as \verb;\chapter;
is not recognised. You attempted to compile a source file meant for inclusion. Compile the root file (the one that has the \verb;\documentclass{}; command in it) instead. Often happens by pressing the TeXShop \texttt{Typeset} button in the window you are entering text in. Open the window with the root document and hit \texttt{Typeset} there.

\item[Misplaced alignment tab character \&.] \hfill \\
You forgot to escape the ampersand \& with a backslash, like so: \verb;\&;. The program expects the ampersand to be part of a table definition.
\end{description}

%%=====
\chapter{Implemented features}
\section{Unicode}
After using Mac OS TextEdit with RTF, which did not give enough control over formatting, and \LaTeX, which
has bad support for Unicode, we switched to \XeLaTeX. This directly accepts UTF-8 as source input and allows
putting pieces of Greek, Hebrew, Arabic and other languages in their original form in the source. The breaking point
came when \LaTeX  made it impossible to produce the astrological signs for the planets.

\texttt{fontspec}

%---
\section{Latin ligatures and long s}
These are handled by choosing a suitable font and selecting the right font features. The text can than be entered
as simple text with no extra codes and the font will automatically put in the 'st' and 'ct' ligatures, the long s (where
appropriate) and the 'Q' with a swish.

Other ligatures and special characters must be entered manually. Select the U. S. Extended keyboard on Mac OS
to enter these with an Alt-key combination. Use the 'Show Keyboard Viewer' feature to find out what Alt key combinations
to use.
\begin{itemize}
\item 'æ' and 'Æ' (Alt-' and Shift-Alt-', i.e. the single quote key next to the semicolon key).
\item 'œ' and 'Œ' (Alt-q and Shift-Alt-q)
\item ā, ē, ō, ū: vowels with a bar (macron) over them. The original sometimes uses these at the end of words
as an abbreviation. Use the 'dead' key Alt-a first, followed by the vowel.
\item The occasional 'regular' accented letter, e.g. á, à, ä, can be made with common Alt dead key combinations.
\item Occasionally there is an abbreviation in the source that looks like a cedilla under the letter. E.g. "Iulianę" for "Iulianae".
Use an Ogonek (looks like a mirrored cedilla) there, because our favourite font does not have an 'e' with a cedilla, but
does haven an 'e' with ogonek. (Dead key: Alt-m)
\end{itemize}

The only special code required is for the ampersand character "\&", which is frequently used in the original. Because
this character has special meaning in \XeTeX, it must be escaped with a backslash, like so: \verb;\&;.

%---
\section{Bigger paper, smaller margins}
In order to get at least the same amount of text on a page in the litteral
transcription as in the original, we would like to have matching paper size
and margins.

The paper size of the original is 217.6 x 338.3 mm. The closest available 
standard paper size is legal, at 215.9 x 355.6 mm. This is 1.7 mm narrower
(so almost the same width),
and 17.3 mm longer. The \texttt{legalpaper} size is chosen with an option
to the \verb;\documentclass{}; command.

The standard margins of \texttt{legalpaper} are rather wide compared to the original. This has been adjusted with the \texttt{geometry} package, by setting new margin sizes explicitly in the options. These sizes have been measured from
the original. The bottom margin has been enlarged to compensate for the above mentioned difference of 17.3 mm in the length of the paper.

%---
\section{Wrapping around tables and figures}
The original contains some tables (e.g. PDF p. 88) and figures (e.g. PDF p. 91)
where the text continues on the side of the table/figure, or "wraps" around it.
We can get this effect by using the \texttt{wrapfig} package. This gives us the
\texttt{wraptable} and \texttt{wrapfigure} commands. These replace the
\texttt{tabular} and \texttt{figure} commands respectively.

%---
\section{Phrase numbering}
We start every phrase from the source on a new line, and give every phrase a number. Numbering starts at 1 for
a new chapter or section and can run over several pages. These numbers will serve to connect the transliterated
and translated versions back to the original.

It is unfeasible to try to match the layout of the original line-by-line, going to a new line on the exact same word
or hyphenation as the original. This is because the font will not have identical properties to the original. Also,
the original has some very odd spacing, especially around interpunction. Once we let go of this requirement, we can
choose a layout which suits our needs.

We could emulate the original and let the text run for pages on end without any paragraphs, but this has
several disadvantages.
\begin{itemize}
\item The transcription will also be a wall of text. This makes it hard to compare the transcription to the original, because
we need to search for the start of a phrase in both versions.
\item This makes it hard to correlate the transliteration with the transcription. As the transcription generally expands
a lot of characters, words and punctuation (e.g. \&) a 'free-hand' layout in the transliteration will not match the transcription
line-by-line.
\end{itemize}
We chose for a solution where even the transcription has each phrase starting on a new line. Advantages:
\begin{itemize}
\item We can add numbers to identify each phrase. These numbers can then be used to easily compare the transliteration
to the transcription.
\item The translation can use these phrase numbers to refer to the original. Having each phrase start on a new line
in the transcription start on a new line makes it easier to find the phrase being referred to.
\end{itemize}

\subsection{Problems}
\begin{itemize}
\item Punctuation in the original is very sloppy.
\item Some parts are very long, without any subdivision. E.g. the Prolegomena runs for
50 pages with an estimated 900 phrases.
\item As we start counting from 1 with each new section, the phrase number alone does not identify the phase. This will
also require the chapter/section name or number.
\end{itemize}


\texttt{mparhack}, with special commands defined

Counter \texttt{parcount}

Command \verb;\p;: put an isolated paragraph counter mark in the margin.

Environment \texttt{parnumbers}: Automatically put paragraph counter marks in the margin

%---
\section{Spaced small caps}
In the original, all occurrences of small caps (in chapter/section titles, in roman numerals) have extra spacing between
the characters. We can probably do this by setting a font feature for small caps fonts.
Maybe use the \texttt{LetterSpace} font teature (\texttt{fontspec.pdf} sect 11.2)

\chapter{Desired Features}
\section{Fractions}
See \texttt{fontspec.pdf} section 12.6

%---
\section{Margin letters on the same side as in the original}
Due to limitations of the \verb;\marginpar{}; command, the margin letters must go on the same side as the
paragraph numbers. We would like to have a system where we have better control over text in the margins.

%---
\section{Formatting chapter/section headers}
We would like the headers at the start of each chapter (Pars) and each subsection (subject) to imitate the original.
Scaligeri has a rather idiosyncratic way of titling his chapters, e.g. introducing his Liber Primus with:
\begin{itemize}
\item His full name, on three lines
\item The title of the book: De Emendatione Temporum
\item "Liber Primus." rather small below that.
\end{itemize}
All this is written in centered, spaced out capitals in various sizes.

The standard \LaTeX chapter titles for the book style are placed with lots of white space above and below them. Scaligeri has no space above the chapter titles.

Scaligeri's subjects have multi-line headers in centered, spaced out capitals. The first line has a bigger font size than
subsequent lines (see e.g. pdf page 87)

The \texttt{titlesec} package may work here.

%---
\section{Putting the right titles in the ToC}

%---
\section{A mechanism to mark and reference pages in the original PDF}

%---
\section{Page headers}
Scaligeri uses as a header in the main body text:
\begin{description}
\item[On the left page] \hfill
    \begin{description}
    \item[Left] Page number (on the outside)
    \item[Centre] His name (centered, small caps) \\
      \textsc{Iosephi Scaligeri}
    \item[Right] <nothing>
    \end{description}
\item[On the right page] \hfill
    \begin{description}
    \item[Left] <nothing>
    \item[Centre] Book title + Chapter title (centered, small caps, abbreviated) \\
       \textsc{De Emendat. Temporvm Lib. I}
    \item[Right] Page number (on the outside)
    \end{description}
\end{description}

In the Prolegomena it is similar, but with roman numerals for the page numbers and only the chapter title
\textsc{(Prolegomena.)} on both pages. 

The Dedication has no headers.

The only package that gives sufficiently detailed control of the headers (and footers) is probably \texttt{titlesec}.

%---
\section{Footers}
In the original, the footer has a link word on the right on each page. A link word is the same as the first word on the
next page, which helped the setter and printer to make sure the pages appear in the right order.

Most right-hand pages also have a Greek or Latin letter + sheet number before the link word.
These were probably used to identify the gathering (a binding unit).

The count goes:
G, G 2, G 3, G 4, blank, blank, H, H 2, H 3, H 4, blank, blank, I, I 2, etcetera.
This suggests that there are 3 sheets (bifolios) per gathering,
with G, G 2 and G 3 on the front and G 4 on the back of the bifolio that has G 3 on the front.

The count starts on the title page (PDF p. 16) with alpha. As the title page does not have a footer,
the first actual mark is alpha 2 on PDF p. 18.
This continues with
beta (PDF p. 28, start of the Prolegomena), gamma, delta and epsilon.
It ends with zeta (PDF p. 76), zeta 2 and finally zeta 3 (PDF p. 80, start of the ToC). PDF p. 82 does not have a zeta 4, against expectations. Also missing are the two blank pages in the zeta group. This leads to the conclusion that the zeta gathering only has two sheets, that is 8 pages
(namely PDF p. 76-83).

Liber I starts on PDF p. 84 with a chapter header, and has the Latin letter A as a mark. This continues with A 2, A 3,
A 4, blank, blank, B, B 2, B 3, etcetera.

After Z 4 (PDF p. 354) the sequence continues with Aa (PDF p. 360), Aa 2 (PDF p. 362), Aa 3, Aa 4, Bb (PDF p. 372),
Bb 2, etc.

After Zz 4 (PDF p. 630) it continues with Aaa (PDF p. 636), Aaa 2, etc.

After Zzz 4 (PDF p. 912, page 7 of Index Orientalium vocum) we get PDF p. 918: the title page of "Veterum Graecorum Fragmenta Selecta", no footer, and then 'a 2' (PDF p. 920), 'a 3', 'a 4', blank, blank, 'b' (PDF p. 930), 'b 2', 'b 3', 'b 4', blank, blank, 'c', 'c 1', 'c 2', 'c 3', 'c 4' (PDF p. 972), which is the last mark in the book, which ends on PDF p. 976.

Policy: because the footers in the original only contain information for the printer, we will leave this information out.
We might use the footer space for our own information, e.g. for a PDF file page reference, a title of our transcription,
and/or a version number of the transcription.

%---
\section{Indents for multi-line Greek quotes}

%---
\section{Exact ligature transcription for Greek}

\texttt{RGreekl2 ?}

\end{document}
 