% XeLaTeX can use any Mac OS X font. See the setromanfont command below.
% Input to XeLaTeX is full Unicode, so Unicode characters can be typed directly into the source.

% The next lines tell TeXShop to typeset with xelatex, and to open and save the source with Unicode encoding.

%!TEX TS-program = xelatex
%!TEX encoding = UTF-8 Unicode

%%% Count out columns for fixed-width source font
% 000000011111111112222222222333333333344444444445555555555666666666677777777778
% 345678901234567890123456789012345678901234567890123456789012345678901234567890

\documentclass{report}

\usepackage{geometry}
%% See geometry.pdf to learn the layout options. There are lots.

\geometry{a4paper}         %% ... or a4paper or a5paper or ... 
%\geometry{landscape}      %% Activate for rotated page geometry

%\usepackage[parfill]{parskip}
%% Activate to begin paragraphs with an empty line rather than an indent
%\usepackage{graphicx}
%\usepackage{amssymb}

% Will Robertson's fontspec.sty can be used to simplify font choices.
% To experiment, open /Applications/Font Book to examine the fonts provided on Mac OS X,
% and change "Hoefler Text" to any of these choices.

\usepackage{fontspec,xltxtra,xunicode}
%\setmainfont{Hoefler Text}
%\defaultfontfeatures{Mapping=tex-text}
%\setromanfont[Mapping=tex-text]{Hoefler Text}
%\setsansfont[Scale=MatchLowercase,Mapping=tex-text]{Gill Sans}
%\setmonofont[Scale=MatchLowercase]{Andale Mono}

% Allow hypertext links to external documents/websites
\usepackage{hyperref}

% Use color to mark the hyperlinks
\usepackage{color}

%% Command to create a hyperlink to an Issue on GitHub
%% Parameter: the issue number on GitHub
\newcommand{\SeeIssue}[1]{%
See \textcolor{blue}{%
\href{https://github.com/DomoViridi/Scaliger/issues/#1}%
{Issue \#{#1}}%
} on GitHub%
}

\title{Writing tips}
\author{Erik Groenhuis}
\date{\today}                                           % Activate to display a given date or no date

\begin{document}
\maketitle

\tableofcontents{}

% For many users, the previous commands will be enough.
% If you want to directly input Unicode, add an Input Menu or Keyboard to the menu bar 
% using the International Panel in System Preferences.
% Unicode must be typeset using a font containing the appropriate characters.
% Remove the comment signs below for examples.

% \newfontfamily{\A}{Geeza Pro}
% \newfontfamily{\H}[Scale=0.9]{Lucida Grande}
% \newfontfamily{\J}[Scale=0.85]{Osaka}

% Here are some multilingual Unicode fonts: this is Arabic text: {\A السلام عليكم}, this is Hebrew: {\H שלום}, 
% and here's some Japanese: {\J 今日は}.

%%=====
\chapter{Project setup}
The aim of this project is to produce a modern version of
De Emendatione Temporum by Scaligeri.
At the very least we want an easily readable transliteration
in the original Latin and all the other languages used.
At best we produce
a full translation into English (and possibly other languages).

The base document is the most recent copy of the most recent edition:
\begin{quote}
  \verb+1629-Geneva-1-Opus_de_emendatione_temporum_hac_postrem.pdf+
\end{quote}
This can be found on Google Books as:
\begin{quote}
  \url{https://books.google.nl/books?id=vTZBAAAAcAAJ}
\end{quote}



%%---
\section{Interpreted Transcription}
In this first stage, we intend to transcribe the scanned PDF original(s)
into a computer file which can be searched, rendered
and modified.

Originally we wanted to stay as close as possible to the original
in the first stage.
This turned out to be a lot more work than we bargained for.
As a result the priority has shifted to producing
an interpreted transcription.

In this stage we create a document which has the same text as the original,
but in a modern script.
This document should be easy to read for the modern reader who knows
Latin and Greek.

Some examples of what changes are made for the transcription are:

\begin{itemize}
\item All ligatures and diacriticals are expanded (e.g. æ to ae).
\item Long s 'ſ' is replaced with modern s.
\item Usage of 'u' and 'v' is changed to the modern style.
\item Abbreviated words (e.g. \verb+quoq;+) are written in full (\verb+quoque+).
\item Interpunction is done in modern style, and more consistent.
\item All Greek phrases will be rendered using modern polytonic Greek
script with Unicode encoding.
\item The layout of tables will be done in a modern style.
\end{itemize}
More details are given in section \ref{sec:textual_cleanup}.

%--
\subsection{Line number references}
To correlate this transcript to the original, the lines
in the original PDF will be numbered.
The numbers will start with 1 on each page, and this and every 5th line
will be marked with its number in the margin.
These numbers are added to the PDF in the form of PDF annotations.

In the transcript each beginning of a line will be marked with the line number
in parentheses. The first sentence on each page will also give the page
number in that mark.

As a result each line in the transcript can easily be found in the original,
and vice versa.

%--
\subsection{Extra features}
Features will be added that are not in the original text.
\begin{itemize}
\item The text will be divided into paragraphs to the best of our insights.
\item Headers and footers will be generated in a modern style.
\item Table of Contents, Index, Glossary and Bibliography will be generated
automatically.
\item List of Tables and List of Figures will be added.
\item Lots of annotations will be added to clarify the text.
For example: Scaligeri habitually quotes classics without giving a reference.
We will add:
    \begin{itemize}
    \item The exact reference (Author, title of the work, section/page/verse)
    \item If possible a link to an on-line version.
    \item If misquoted, the actual quote.
    \item An explanation why this quote is appropriate.
    \end{itemize}
\end{itemize}

Note that before we get to the fancy stuff, the initial big work will be
 the actual transcribing,
i.e. reading from the scanned original and typing in the Latin text.

%%---
\section{Translation}
A translation of all the text.
A reader should be able to read and understand everything without knowledge
of ancient languages.
The layout will be in a similar modern style as the transliteration.
In addition there will be many notes 
(footnotes?) giving clarifications and annotations to the vagaries of the
translation process.

In principle each phrase in the original will correspond to a phrase in the
translation, but this is not a strict rule.
We may even change our mind and go for a smooth running translation.


%%=====
\chapter{How Scaliger divided his book}
The book is divided into Liberi which are in turn divided into subjects of
 a few pages each. Some books are divided into Partes, some are not.
 Each Liber or Pars begins with an introduction before the first subject starts. 
 Titles of the Liberi and Partes are usually only given in the table of
 contents, not in the body of the text. Though the parts are mentioned in the
 Table of Contents, they are not marked in the body of the text.
 If the Pars has a title, than sometimes that title is in the body of the text,
 just before and slightly bigger than the title of the first subject in that
 Pars, but sometimes it is not in the body. Partes do not start on a new page.
 Liberi usually do. Sometimes (e.g. Liber Quartus) the title of the first
 subject which was mentioned in the ToC is missing from the body text.
~
\\
\\
The structure goes as follows (numbers on the left are PDF file page numbers;
 even numbers are right-hand pages):

\begin{description}
\item[16] Title page (1 page + blank verso)
\item[18] Dedication ("Domino Achilli Harlaeo", 4 pages)
\item[22] Another 2 pages ("Q. Sept. Florens Christianus de Iosephi Scaligeri")
\item[24] Greek quotes (3 pages + blank verso)
\item[28] Prolegomena (52 pages)
\item[80] Table of contents ("Conspectus", 4 pages)
\item[84] Liber Primus
 ("De anno aequabili minore", pp 1-60, 18 subjects, 60 pages)
\item[144] Liber Secundus
 ("De anno Lunari", pp 61-187, 32 subjects, 127 pages)
\item[271] Liber Tertius
 ("De anno aequabili maiore", pp 188-226, 7 subjects, 39 pages)
\item[310] Liber Quartus
 ("De anno solari, tributus in partis quatuor", pp 227-344, 118 pages)
\begin{description}
  \item[310] Pars Prima
   (pp 227-272 \& 353, 16 subjects, 46 pages)
  \item[355] Secunda pars
   (pp 272-293, 5 subjects, 22 pages)
  \item[376] Tertia pars
   ("De periodis solaribus.", pp 293-303, 5 subjects, 11 pages)
  \item[386] Quarta pars
   ("De annis tropicis et lunaribus emendatis",
    pp 303-353, 9 subjects, 51 pages)
  \item[436] Omissa Lib. II \& IV
   (p 353, two additions: to page 141 and to page 272)
\end{description}
\item[437] Liber Quintus
 ("Prior de epochis temporum", pp 354-530, 106 subjects, 117 pages)
\item[614] Liber Sextus
 ("Posterior de epochis temporum, in duas partes tributus", pp 531-624,
  94 pages)
\begin{description}
  \item[624] Prior pars (pp 541-567, 10 subjects, 27 pages)
  \item[657] Altera pars de duabus
   ("Quaestionibus Danielis", pp 574-624, 7 subjects)
\end{description}
\item[708] Liber Septimus
 ("De computis annalibus nationum", pp 625-784, 20 subjects, 160 pages)
\item[870] Nomenclator (Index,  9 pages, no page numbers)
\item[879] Index rerum et verborum memorabilium
 (Index, 21 pages, no page numbers)
\item[900] Index Graecarum vocum (Index, 6 pages, no page numbers)
\item[906] Index Orientalium vocum (Index,  8 pages, no page numbers)
\item[914] Operarum typographicarum; Errata ita corrigito
 (1 page, no page number)
\item[915] Operis de Emendatione Temporum Finis. (1 page, no page number)
\item[916] Two blank pages
\item[918] "Veterum Graecorum Fragmenta Selecta" (pp 1-2, 1 page + blank verso)
\begin{description}
  \item[920] "ΒΗΡΩΣΣΟΣ ΒΑΒΥΛΩΝΙΟΣ ΙΕΡΕΥΣ ΒΗΛΟΥ ..."
     (pp 3-8, 6 pages)
  \item[926] "In fragmentum Berosi Babylonii sacerdotis beli ..."
     (pp 9-49, 41 pages)
  \item[966] "Epochae in vetere instrumento quae certum characterem habent"
     (pp 49-50, 2 pages)
  \item[969] "Characteres vetustatis, qui hodie apud Iudaeos"
     (pp 51-53, 3 pages)
  \item[971] "Characteres temporum in novo testamento"
     (pp 54-59, 6 pages)
\end{description}
\item[977] Blank (6 pages)
\item[983] Back cover
\end{description}
~
\\
For the transcription we can equate the subdivisions as follows:
\begin{itemize}
\item Liberi will be represented by LaTeX Chapters
\item Subjects will be represented by LaTeX Subsections
\item Partes will not be directly represented. Maybe by a special command
 which inserts the Pars text in the body (optional) and the ToC
 (\verb+\addcontentslline+, in the Section style). Or maybe create a special
  kind of Chapter that has a Parts thing between Chapter and Section.
\item The bits before Primus Liber will go in the LaTeX Top matter, and be
 preceded by the \verb+\frontmatter+ command.
\item The Prolegomena is a special Chapter with roman page numbers and
 no entry in the ToC (accomplished with the \verb+\frontmatter+ command).
\item The four indices and the errata page will be represented as Appendices
 (precede by \verb+\appendix+).
\item The "Finis" page and the Veterum Graecorum part will be preceded by
 \verb+\backmatter+.
\end{itemize}


%%---
\section{Footers}
(This information has no effect on the transcription.
It is kept here as general background.)

In the original, the footer has a link word on the right on each page. A link word is the same as the first word on the
next page, which helped the setter and printer to make sure the pages appear in the right order.

Most right-hand pages also have a Greek or Latin letter + sheet number before the link word.
These were probably used to identify the gathering (a binding unit).

The count goes:
G, G 2, G 3, G 4, blank, blank, H, H 2, H 3, H 4, blank, blank, I, I 2, etcetera.
This suggests that there are 3 sheets (bi-folios) per gathering,
with G, G 2 and G 3 on the front and G 4 on the back of the bifolio that has G 3 on the front.

The count starts on the title page (PDF p. 16) with alpha. As the title page does not have a footer,
the first actual mark is alpha 2 on PDF p. 18.
This continues with
beta (PDF p. 28, start of the Prolegomena), gamma, delta and epsilon.
It ends with zeta (PDF p. 76), zeta 2 and finally zeta 3 (PDF p. 80, start of the ToC). PDF p. 82 does not have a zeta 4, against expectations. Also missing are the two blank pages in the zeta group. This leads to the conclusion that the zeta gathering only has two sheets, that is 8 pages
(namely PDF p. 76-83).

Liber I starts on PDF p. 84 with a chapter header, and has the Latin letter A as a mark. This continues with A 2, A 3,
A 4, blank, blank, B, B 2, B 3, etcetera.

After Z 4 (PDF p. 354) the sequence continues with Aa (PDF p. 360), Aa 2 (PDF p. 362), Aa 3, Aa 4, Bb (PDF p. 372),
Bb 2, etc.

After Zz 4 (PDF p. 630) it continues with Aaa (PDF p. 636), Aaa 2, etc.

After Zzz 4 (PDF p. 912, page 7 of Index Orientalium vocum) we get PDF p. 918: the title page of "Veterum Graecorum Fragmenta Selecta", no footer, and then 'a 2' (PDF p. 920), 'a 3', 'a 4', blank, blank, 'b' (PDF p. 930), 'b 2', 'b 3', 'b 4', blank, blank, 'c', 'c 1', 'c 2', 'c 3', 'c 4' (PDF p. 972), which is the last mark in the book, which ends on PDF p. 976.

Policy: because the footers in the original only contain information for the printer, we will leave this information out.
We might use the footer space for our own information, e.g. for a PDF file page reference, a title of our transcription,
and/or a version number of the transcription.


%%=====
\chapter{Tools}
\XeLaTeX{} part of \TeX{} package. Editing source with TeXShop. Can also
be done with any other editor of flat text. 
Compilation must be done with \texttt{xelatex}. Can be done from TeXShop.
Note: use the Typeset button on the root document
(which has the \verb;\documentclass{}; command in it).
Sources kept on \texttt{GitHub}.

%%---
\section{Implemented features}
%--
\subsection{Unicode}
After using Mac OS TextEdit with RTF, which did not give enough control over formatting, and \LaTeX, which
has bad support for Unicode, we switched to \XeLaTeX. This directly accepts UTF-8 as source input and allows
putting pieces of Greek, Hebrew, Arabic and other languages in their original form in the source. The breaking point
came when \LaTeX  made it impossible to produce the astrological signs for the planets.

%--
\subsection{Fractions}
See \texttt{fontspec.pdf} section 12.6.
Solved with \verb+\myfrac{}{}+ based on \verb+\sfrac{}[]{}+
from the \verb+xfrac+ package.

%--
\subsection{A mechanism to mark and reference pages in the original PDF}
This is done by adding line numbers to the PDF using annotations, and
putting commands in the tex source that refer to those
line numbers before every sentence.


%%---
\section{Desired Features}
%--
\subsection{Roman numerals}
Introduce a command like \verb+\rnum{}+ to format roman numerals, to replace
the \verb+\textsc{}+ small caps command. Features:
\begin{itemize}
\item Small caps
\item Possibly wider spacing
\item A suitable font with
\begin{itemize}
\item Serifs
\item Support for small caps
\item Support for U+2183 ROMAN NUMERAL REVERSED ONE HUNDRED, which is used
in Liber Secundus.
\end{itemize}

\end{itemize}


%--
\subsection{Putting the right titles in the ToC}

%%---
\section{Common error messages}
\begin{description}
\item[Undefined control sequence.] \hfill \\
A common or normal command such as \verb;\chapter;
is not recognised. You attempted to compile a source file meant for inclusion. Compile the root file (the one that has the \verb;\documentclass{}; command in it) instead. Often happens by pressing the TeXShop \texttt{Typeset} button in the window you are entering text in. Open the window with the root document and hit \texttt{Typeset} there.

\item[Misplaced alignment tab character \&.] \hfill \\
You forgot to escape the ampersand (\&) with a backslash, like so: \verb;\&;.
The program expects the ampersand to be part of a table definition.
Note that an ampersand should almost always be transliterated as 'et'.
\end{description}


%%=====
\chapter{Transcription}

%%---
\section{Entering text}

Enter the text as simple flat text in the \TeX{} source code.
Most of the text will consist of Latin,
with bits of Greek (words, short phrases, sometimes longer quotes) and words
in Hebrew, Arabic or Persian.

All text must be entered with Unicode UTF-8 encoding.
For the Latin body text the 2x26 characters of the common Latin alphabet,
without diacriticals will suffice, together with the common interpunction
characters.
For the Greek we use the modern 25+24 characters plus the diacriticals
for the polytonic style.
\\
\\
Some special text must be surrounded by commands:
\begin{itemize}
\item Greek, Hebrew, Arabic and Persian: First enter these as a placeholder,
like so:
\begin{itemize}
\item\verb;\greektext{[Greek]};
\item\verb;\hebrewtext{[Hebrew]};
\item\verb;\arabicfont{[Arabic]};
\item\verb;\arabicfont{[Persian]};
\end{itemize}
There is no specific command for Persian, as it uses the same font
and settings as Arabic.
\\
The commands will handle the selection of a suitable font for that language. In
later phase the placeholders will be filled in with the actual text, which can
be entered as Unicode characters.

If you feel so inclined, you can enter the foreign text, replacing the
"\verb+[Greek]+" etcetera. If you have entered some foreign text but are
if it is correct, leave a question mark in brackets, like so: \verb+[?]+.

\item Astrological symbols: For the astrological symbols used to represent the
sun, moon and planets there is the \verb;\astro{}; command, which will select
a suitable font. The symbols themselves can be entered in Unicode.
%\begin{table}[h]
% table command disabled; we do not want to float this table.
\begin{center}
\begin{tabular}{c | c}
\begin{tabular}{r l}
Sun & U+2609 \\
Moon & U+263E \\
Mercury & U+263F \\
Venus & U+2640
\end{tabular}
&
\begin{tabular}{r l}
Mars & U+2642 \\
Jupiter & U+2643 \\
Saturn & U+2644 \\
\parbox[c]{1cm}{}
\end{tabular}
\end{tabular}
\end{center}
%\end{table}%

\item Roman numerals, such as ''\textsc{vii}''.
Enter them as lower case characters, surrounded by the \verb;\rnum{};
command. This command will show the numeral in small caps,
with the appropriate spacing.
Note: Before the introduction of the \verb;\rnum{}; command,
the \verb;\textsc{}; command was used. Older parts of the \TeX{} source
may still use this command. These instances will be replaced eventually.

\item Numbers with fractions:
\begin{itemize}
\item Use the \verb+\myfrac{numerator}{denominator}+ command to render fractions.
\item If a whole number precedes the fraction,
(e.g. "365 1/4")
put a non-break space tilde code '\verb+~+' between the number
and the fraction: \verb+365~\myfrac{1}{4}+ 
\end{itemize}

\end{itemize}

%%---
\section{Textual cleanup}
\label{sec:textual_cleanup}
\begin{itemize}
\item All ligatures are expanded
	\begin{enumerate}
	\item æ to ae
	\item Æ to Ae (e.g. Aegyptica)
	\end{enumerate}

\item Long s is replaced with modern s. Tricky, as the long s looks almost
exactly like an 'f'.

\item Diacriticals will be expanded
	\begin{itemize}
	\item ā to am (sometimes an)
	\item ō to on (sometimes om)
	\item ē to en (sometimes em)
	\item à (as a whole word) to ab
	\item ë in poëta (p4,7) or Laërtius (p23,36) will be unchanged for now
	\end{itemize}

\item "u" and "v" are used exchangeably in the original.
These will be converted to modern style,
 using a "u" if a vowel, and a "v" if a consonant.

\item The ampersand \verb+&+ will be expanded as "et" in all cases.
 Note that \verb+&+ has a special meaning in \LaTeX. If it ever needs to be
 used literally in the text, it should be escaped with a backslash, as
 \verb+\&+.

\item \verb+&c.+ should be expanded as "et cetera" (two words). Though
 "etcetera" (one word) is a word in English, in Latin it was two words.

\item Abbreviated words are written in full
 (see also Diacriticals above).
	\begin{itemize}
	\item "Quoq" becomes "quoque"
	\item "propagatā" becomes "propagatam".
	\item "nō" becomes "non"
	\item "à" becomes "ab"
	\item "Iulianae" is sometimes written with an 'e' with a cedille underneath
	 it instead of the 'ae'. E.G. Prolegomena p~\textsc{X}.
	 Same for "Iudaei".
	\item Occasionally words are shortened using a period. These should be
	expanded, but for a non-scholar it is hard to find the proper declension
	required.
	Example: "Kal." for Kalendae, Kalendarum, Kalendis, Kalendas or Kalends.
	\end{itemize}

\item Interpunction is done in modern style, and more consistent, e.g.:
No space before a comma, period, colon, semicolon,
 question mark or exclamation mark. One space after them.

\item Often sentences starting with a \verb+q+ won't have this q capitalised.
We will start each sentence with a capitalised word. Sometimes other initial
letters won't have been capitalised either.

\item The Greek in the original was printed in an imitation of the
Byzantine Minuscule style of handwriting which was current at the time.
This involves many, many ligatures and also many variant glyphs for single
characters. All in all (also counting diacriticals) about 1500 glyphs are used.
These can be very hard to read for the modern reader.

Therefore all Greek phrases will be rendered using modern polytonic Greek
script with Unicode encoding.\footnote{Also, it is very hard to find a font
and an encoding to fully render all the minuscule glyphs used.}
This has about 49 base glyphs (24 upper case, 24 normal lower case
plus the word-final sigma) and about 233 extra glyphs for
the polytonic diacriticals and iota subscripts.

\end{itemize}

%%---
\section{Special things to pay attention to}
\begin{itemize}
\item Reading the interpunction can be very tricky where numbers are involved.
Often there will be a period after an Arabic or roman number, while the
construction of the phrases suggest that it is not intended as the end of 
the sentence. E.g. Liber Primus, p 17, line 10 and further.
\item The difference between an 'f' and a 'long s' (ſ) can often be hard to see,
 especially if followed by an 'i'. The printers used two ligatures which are
 almost identical ('fi' and 'ſi'). In the original (as, indeed, in this
 document if all goes well), the letters in the f-i ligature ('fi') are
 connected at the hight of the crossbar, while in the long s-i ligature ('ſi')
 they are not connected there.
\item It appears that sometimes a final 's' in a word is printed as what looks
 like a small question mark. E.g. Prolegomena p.XXIII (PDF p.50) has "Ioſephu?"
 and "Ioſepho?" near the bottom. Transliterate as 's'.
\end{itemize}

%%---
\section{Further tips}
\begin{itemize}
\item To help entering Greek or Hebrew text, modern operating systems
have a way to configure the keyboard for quick typing, or a virtual
keyboard where you can click on the characters on the screen.
If available (e.g. in OS~X) use of the 'Greek Polytonic' keyboard is
preferred over the 'Greek' keyboard.

\item While entering the Latin body text you can postpone entering the other
languages and use a placeholder;
 "\verb+[Greek]+" or
 "\verb+[Hebrew]+".

\item Any extra notes and remarks should go between square brackets

\item Use other editions to compare the transcription. They often give a
 different or no abbreviation, or different spacing.

\end{itemize}


%%---
\section{Extra features to be added}
\begin{itemize}
\item Text will be divided into paragraphs to the best of our insights.
\item Page headers and footers will be generated in a modern style.
\item Table of Contents, Index, Glossary and Bibliography are generated
automatically.
\item List of Tables and List of Figures added.
\item Lots of annotations to clarify the text.
For example: Scaligeri habitually quotes classics without giving a reference.
We will add:
    \begin{itemize}
    \item The exact reference (Author, title of the work, section/page/verse)
    \item If possible a link to an on-line version.
    \item If misquoted, the actual quote.
    \item An explanation why this quote is appropriate.
    \end{itemize}
\item Change the shape of tables to a modern style,
with fewer horizontal lines and no
vertical lines. Use the \verb+booktabs+ package, which has
\verb+\toprule+,
\verb+\midrule+ and
\verb+\bottomrule+ commands.
The documentation of the package also gives information on a good way
to layout tables.
\end{itemize}

%%---
\section{Line number references}
To correlate this transcript to the original,  the lines
in the PDF of the original will be numbered.
The numbers will start with 1 on each page, and every 5th line will be marked
with its number in the margin.

These line numbers will have the form of annotations in the PDF, and will
be added a chapter at the time, as the project advances.

In the transcript each beginning of a sentence will be marked with
the line number of the original
in parentheses.

%--
\subsection{Commands for line numbers}
Commands were created in the TeX source to add the line- and page-numbers:
\begin{itemize}
\item \verb+\lnr{<line number>}+ to add the line number to the beginning of
a sentence.
\item \verb+\plnr{<PDF page>}{<page nr>}{<line nr>}+ to use at the beginning of the first sentence on a new page in the original. Parameters are:
\begin{enumerate}
\item The PDF page number of the page in this particular scan of this particular
copy of this particular edition of the book.
This value is only used internally for now. Intended to be used to
make links to the original PDF file.
\item The page number as shown on the page. This will be printed in the output.
\item The line number in the original where the sentence starts.
\end{enumerate}
\item \verb+\Rplnr{<PDF page>}{<page nr>}{<line nr>}+. This is the same as the
 \verb+\plnr+ command, except that the page number will be printed as a roman
 numeral rather than an Arabic number. Used for chapters that use roman numerals
 as page numbers, such as the Prolegomena.
\end{itemize}

Note: \emph{Do not put spaces or newlines between these commands and the
text that follows.} That would put an extra space in the output after the no-break-space that is programmed into the command. The text should always follow
directly after the command. Correct example:\begin{quote}
\verb+\lnr{23}Lore ipsum dolor savit+
\end{quote}

%%---
\section{General guidelines for the \XeLaTeX code}
The following guidelines are intended to make it easier to compare the
tex file to the original.
Note that \LaTeX treats a single newline as if it is a space. This means that
breaking up a line in the tex file with a newline will not affect the outcome.
Beware that a double newline (leaving a blank line) \emph{does} have a
special meaning: it starts a new paragraph.
\begin{itemize}
\item A new sentence in the original should start at a new line in the tex file,
preceded by one of the commands for line numbers listed in the previous section.
\item The first full word on a line in the original should be at the start
of a line in the tex file.
Insert an extra newline before that word if necessary.
\item No line in the tex file should be longer than 80 characters.
Sometimes if the first two rules are followed, a tex line will come out
longer than 80 characters. In that case, split the line in two or more,
where each extra line is indented by one space.
\end{itemize}
The effect of these guidelines is that a non-indented new line
in the tex file corresponds
to either:
\begin{itemize}
\item The beginning of a sentence (marked with a \verb+\lnr{}+ command or
its siblings).
\item The first word of a line in the source text.
\end{itemize}

%--
\subsection{Wrapping text around figures}
Attempts to use \verb+wrapfigure+ to make the text flow around figures
and tables were
more trouble then their worth. In particular, the text would not start to
flow until a new paragraph is started. Because we seldom start a new paragraph,
we would have to start a paragraph on purpose (causing an indent) just for the
wrapping. We will simply put the figures on their own, without wrapping.

%%====
\chapter{Specific issues per chapter}

%%---
\section{Prolegomena}
\begin{itemize}
\item
The pages are numbered with roman numerals rather than Arabic numbers.
The \verb+\Rplnr{}{}{}+ command
was created to print the page numbers in roman numerals in the text.
\item
The Prolegomena in the original is basically a fifty page wall of text.
Some attempts have been made to find places where the text changes the subject
and insert a new paragraph there, but further work is needed.
We may want to introduce sections or subsections.
\SeeIssue{12}.
\end{itemize}

%%---
\section{Liber Primus}
\begin{itemize}
\item
There is frequent use of "à" as a word. After some consideration this
is transcribed as the word "ab". Each occurrence is marked in comment.
\item
The word "scrupulus" (Lit: "small pebble"; also: 1/24 of an ounce)
is often abbreviated. To expand these, the proper
declination must be determined in each case. \SeeIssue{2}.
\item
Wrapping text around figures turned out to be impractical. We now will
simply use a floating object for figures and tables.
\item
The tables can be simplified
a lot, because there is no urgent need to imitate the original text.
E.g. the headers
for the Ostenta/Sexagesima conversion tables (Table 1.1) can now be written
as a simple line of text,
rather than broken up over three lines with different
fonts and font sizes.
\item
Also for tables: the text in it can be a lot bigger, as the tables
will now take up the whole width of a page anyway,
rather than with the body text flowing around them
\item
To fine-tune the placement of tables and figures it turned out to be
necessary to move the location where they are defined slightly forward or
backward in the tex file. To make this easier, we put the code for tables
 (and some figures) in separate files.
This way we don't have to cut and paste large pieces of code if we want to
move the definition of the table or figure. We now only have to move the
\verb+\include{}+ command.
\item
All tables and figures will be given a caption and a label, so they can
appear in a List of Tables/Figures, and can be referred to from within the
text.
\item
Multi-line Greek and Latin quotes, e.g. p. 4, line 21
can be done with the \verb+quote+ environment.
\begin{verbatim}
\begin{quote}
Lorem ipsum dolor sit amet, consectetur adipiscing elit,\\
sed do eiusmod tempor incididunt ut labore et dolore magna aliqua.\\
Ut enim ad minim veniam, quis nostrud exercitation ullamco\\
laboris nisi ut aliquip ex ea commodo consequat.
\end{quote}
\end{verbatim}
Which gives:
\begin{quote}
Lorem ipsum dolor sit amet, consectetur adipiscing elit,\\
sed do eiusmod tempor incididunt ut labore et dolore magna aliqua.\\
Ut enim ad minim veniam, quis nostrud exercitation ullamco\\
laboris nisi ut aliquip ex ea commodo consequat.
\end{quote}
This is nicely indented. New lines must be forced with \verb+\\+.
Also available are the \verb+quotation+ and \verb+verse+ environments.

\item
The word "poëta/poëtae" is used on page 4, lines 7 and 14. We are not sure
what to do with the trema. For the time being we write it with the trema.

\item
The image on page 11 (Figure 1.1: Lapis Levinii). We made a cleaned-up copy
from an enlarged PDF page by editing nearly at the pixel level. It looks like
Scaliger had a representation made for his book using typeset letters.
E.g. the text "A latere dextro saxi" is obviously not on the stone itself.
There is no indication
of how he knew what the original looked like.

\item
Abbreviations:
\begin{description}
\item[scrup.]See above at "scrupulus".
\item[Odyss.]For some declination of "Odysius" (by Homer). p4. ln 21.
\item[Kal.]As in "Kalends Septembris" or some-such.
\item[\&c.]Presumably this stands for "et cetera". NB: "etcetera" is an English
word. In Latin it is two separate words.
\item[Philostr. v.]p30, line 4.
\item[Olynth.]p 32, line 21.
\item[Oecon.]p 33, line 10.
\end{description}
\SeeIssue{2}.

\item
Some pages have loads of Greek text (several sentences in a row) with many
Byzantine style ligatures and diacritics. It is most impractical to set up
a framework of \verb+\textgreek{}+ commands, so the long bits are denoted
by "Lots of [Greek]" or "Line of [Greek]" inside a single such command.
For example pages 29 and 30.
\item
Page 35, line 21 gives a clear example of how capital letters at the beginning
of sentences are often messed up, either in the manuscript or by the printer.
The 1629 Geneva edition has "..., non alio mense. fiebat enim..." while the
1598 Lugduni edition has (p35, line 4) "..., non alio mense. Fiebat enim...".
The period is still there, but the capital is lost.
\item
The table on page 38 (Tabula neomeniarum primi mensis Elidensis in annis
periodi Olympicae) is problematic in many ways. It will probably not be 
the last one to be so.
\begin{itemize}
  \item{}There are very many entries (76 rows and 5 main columns) making the
  original small and hard to read (especially the Greek bits), and making it
  hard to fit the transcribed table on a page.
  \item{}There are Greek abbreviated words in the margin of the table. It is
  hard to see what they say, and hard to find out what they are an abbreviation
  of, and hard to know what they mean. There appear to be only 3 different
  words. To save space, the were replaced by symbols, with a legend at the
  end of the table.
  \item{}The columns with numbers were manually padded out in the original.
  In the transcription we get wildly varying widths  because the font we use
  has proportional digits. We may want to use a font with monospace digits.
  \item{}Rotated headers were needed to keep the columns narrow.
  \item{}It will probably not be an improvement to change the table to a
  horizontal layout. That will likely take up more space, and even be
  less clear.
\end{itemize}
\end{itemize}

\end{document}
 