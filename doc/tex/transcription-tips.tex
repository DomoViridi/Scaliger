% XeLaTeX can use any Mac OS X font. See the setromanfont command below.
% Input to XeLaTeX is full Unicode, so Unicode characters can be typed
% directly into the source.

% The next lines tell TeXShop to typeset with xelatex, and to open and save
% the source with Unicode encoding.

%!TEX TS-program = xelatex
%!TEX encoding = UTF-8 Unicode

%%% Count out columns for fixed-width source font
% 000000011111111112222222222333333333344444444445555555555666666666677777777778
% 345678901234567890123456789012345678901234567890123456789012345678901234567890

\documentclass{report}

\usepackage{geometry}
%% See geometry.pdf to learn the layout options. There are lots.

\geometry{a4paper}         %% ... or a4paper or a5paper or ... 
%\geometry{landscape}      %% Activate for rotated page geometry

%\usepackage[parfill]{parskip}
%% Activate to begin paragraphs with an empty line rather than an indent
%\usepackage{graphicx}
%\usepackage{amssymb}

% Will Robertson's fontspec.sty can be used to simplify font choices.
% To experiment, open /Applications/Font Book to examine the fonts
% provided on Mac OS X,
% and change "Hoefler Text" to any of these choices.

\usepackage{fontspec,xltxtra,xunicode}
%\setmainfont{Hoefler Text}
%\defaultfontfeatures{Mapping=tex-text}
%\setromanfont[Mapping=tex-text]{Hoefler Text}
%\setsansfont[Scale=MatchLowercase,Mapping=tex-text]{Gill Sans}
%\setmonofont[Scale=MatchLowercase]{Andale Mono}

\title{Writing tips}
\author{Erik Groenhuis}
\date{\today}              % Activate to display a given date or no date

\begin{document}
\maketitle

\tableofcontents{}

% For many users, the previous commands will be enough.
% If you want to directly input Unicode, add an Input Menu or Keyboard
% to the menu bar using the International Panel in System Preferences.
% Unicode must be typeset using a font containing the appropriate characters.
% Remove the comment signs below for examples.

% \newfontfamily{\A}{Geeza Pro}
% \newfontfamily{\H}[Scale=0.9]{Lucida Grande}
% \newfontfamily{\J}[Scale=0.85]{Osaka}

% Here are some multilingual Unicode fonts:
% this is Arabic text: {\A السلام عليكم}, this is Hebrew: {\H שלום}, 
% and here's some Japanese: {\J 今日は}.

%%=====
\chapter{Tips for transcribing and translating Scaliger}

\begin{itemize}
\item Base document is the most recent copy of the most recent edition:
\begin{quote}
  \verb+1629-Geneva-1-Opus_de_emendatione_temporum_hac_postrem.pdf+
\end{quote}
\item A page of the transcription matches a page of the original
\item In the transcription, each sentence is a new paragraph, and is numbered.
\item If a sentence in the original flows over to the next page, all of it will be put at the top of the next page in the transcription.  This is so that the letters in the margin will always start with "A" at the top of the page.
\item The original has page numbers at the top of the page. Pagenumbers in the transcription will be generated by LaTeX. We might change the pre-defined layout to something more resembling the original at some point.
\item Use the "U. S. Extended" keyboard for the Latin (i.e. non-Greek and non-Hebrew) parts of the text. This allows you to enter the abbreviating dash-above-character by pressing Alt-a.
\end{itemize}

There will be two transcriptions:
\begin{enumerate}
\item A litteral transcription. Uses of characters, abbriviations, diacriticals etcetera will follow the original as closely as possible. This will be in the transcriptlitteral directory.
\item An interpreted transcription. Abbriviations are expanded, ligatures expanded, all characters converted to a modern 26 character latin alphabet, etcetera. This will be the transcriptinterpret directory.
\end{enumerate}

%---
\section{Particular tips}
\begin{itemize}
\item "u" and "v" are used exchangeably in the original. In the litteral, always transcribe "u" as "u" and "v" as "v". In the interpreted, convert to "u" if a vowel, and to a "v" if a consonnant.
\item With the Kepler Font package in LaTeX, the character "s" is automatically rendered as the "ſ" (long s). To get a normal 's' (e.g. at the end of words) write \verb"s=" in the LaTeX source.
\item Some words are shortened in the original.  Keep shortened in the litteral, transcribe as full word in the interpreted:
\begin{itemize}
\item "Quoq;" becomes "Quoque"
\item "propagatā" becomes "propagatam". Enter the line above the character using Alt-a (U. S. Extended keyboard required)
\item "nō" becomes "non"
\item "à" becomes "ab"
\item "Iulianae" is sometimes rendered with an 'e' with a cedille underneath it instead of the 'ae' at the end. Use a cedille (Alt-C) in litteral.
\item Occasionaly words are shortened using a period.
\end{itemize}
\item When the original uses "\&" (short for "et"), use "\verb+\&+" in the litteral (backslash is needed for LaTeX) and "et" in the interpreted.
\end{itemize}

\subsection{Further tips}
\begin{itemize}
\item Try to enter the bits of Greek using the 'Greek Polytonic' keyboard.
\item When not transcribing the Greek or Hebrew, replace with "[Greek]" or "[Hebrew]".
\item Any extra notes and remarks should go between square brackets
\item Roman numerals: Make small caps using \verb+\textsc+,
 e.g. \verb+\textsc{xxvii}+
\item Use other editions to compare the transcription. They often give a different or no abbriviation, or different spacing.
\end{itemize}

%---
\section{Obsolete, but informative}
To facilitate entering the 'long s' (ſ) character, a change has been made to the Mac~OS~X configuration. The file:
\begin{quote}
\texttt{/System/Library/Input Methods/PressAndHold.app/Contents/Resources/Keyboard-en.plist}
\end{quote}
has been modified so that when you press and hold 's', the list of accented characters presented to the user now starts with the long s.
\\
\\
To enter the long s either:
\begin{itemize}
\item press-and-hold s, then press 1 (to select the first item on the list)
\item press-and-hold s, then press TAB (to select the next, and in this case the first, item on the list), then press ENTER to confirm the selection.
\end{itemize}
~
\\
To enter a double long s, either:
\begin{itemize}
\item press-and-hold s, then press 2 (to select the second item on the list)
\item press-and-hold s, then press TAB twice, then press ENTER to confirm the selection.
\end{itemize}

%%=====
\chapter{How Scaliger divided his book}
The book is divided into Liberi which are in turn divided into subjects of a few pages each. Some books are divided into Partes, some are not. Each Liber or Pars begins with an introduction before the first subject starts. Titles of the Liberi and Partes are usually only given in the table of contents, not in the body of the text. Though the parts are mentioned in the Table of Contents, they are not marked in the body of the text. If the Pars has a title, than sometimes that title is in the body of the text, just before and slightly bigger than the title of the first subject in that Pars, but sometimes it is not in the body. Partes do not start on a new page. Liberi usualy do. Sometimes (e.g. Liber Quartus) the title of the first subject which was mentioned in the ToC is missing from the body text.

The structure goes as follows (numbers on the left are PDF file page numbers; even numbers are right-hand pages):

\begin{description}
\item[16] Title page
\item[18] Dedication ("Domino Achilli Harlaeo", 4 pages)
\item[22] Another 2 pages ("Q. Sept. Florens Christianus de Iosephi Scaligeri")
\item[24] Greek quotes (3 pages)
\item[28] Prolegomena (52 pages)
\item[80] Table of contents (4 pages)
\item[84] Liber Primus ("De anno aequabili minore", pp 1-60, 18 subjects, 60 pages)
\item[144] Liber Secundus ("De anno Lunari", pp 61-187, 32 subjects, 127 pages)
\item[271] Liber Tertius ("De anno aequabili maiore", pp 188-226, 7 subjects, 39 pages)
\item[310] Liber Quartus ("De anno solari, tributus in partis quatuor", pp 227-344, 118 pages)
\begin{description}
  \item[319] Pars Prima (pp 227-272 \& 353, 16 subjects)
  \item[355] Secunda pars (pp 272-293, 5 subjects)
  \item[376] Tertia pars ("De periodis solaribus.", pp 293-303, 5 subjects)
  \item[386] Quarta pars ("De annis tropicis et lunaribus emendatis.", pp 303-353,  9 subjects)
  \item[436] Omissa Lib. II \& IV (p 353, two additions: to page 141 and to page 272)
\end{description}
\item[437] Liber Quintus ("Prior de epochis temporum", pp 354-530, 106 subjects, 117 pages)
\item[614] Liber Sextus ("Posterior de epochis temporum, in duas partes tributus", pp 531-624, 94 pages)
\begin{description}
  \item[624] Prior pars (pp 541-567, 10 subjects)
  \item[657] Altera pars de duabus ("Quaestionibus Danielis", pp 574-624, 7 subjects)
\end{description}
\item[708] Liber Septimus ("De computis annalibus nationum", pp 625-784, 20 subjects, 160 pages)
\item[870] Nomenclator (Index,  9 pages, no page numbers)
\item[879] Index rerum et verborum memorabilium (Index, 21 pages, no page numbers)
\item[900] Index Graecarum vocum (Index, 6 pages, no page numbers)
\item[906] Index Orientalium vocum (Index,  8 pages, no page numbers)
\item[914] Operarum typographicarum; Errata ita corrigito (1 page, no page number)
\item[915] Operis de Emendatione Temporum Finis. (1 page, no page number)
\item[916] Two blank pages
\item[918] "Veterum Graecorum Fragmenta Selecta" (p 1)
\begin{description}
  \item[919] Blank page (p 2)
  \item[920] "ΒΗΡΩΣΣΟΣ ΒΑΒΥΛΩΝΙΟΣ ΙΕΡΕΥΣ ΒΗΛΟΥ ..." (pp 3-8, 6 pages)
  \item[926] "In fragmentum Berosi Babylonii sacerdotis beli ..." (pp 9-49, 41 pages)
  \item[966] "Epochae in vetere instrumento quae certum characterem habent" (pp 49-51, 2 pages)
  \item[969] "Characteres vetustatis, qui hodie apud Iudaeos" (pp 51-53, 3 pages)
  \item[971] "Characteres temporum in novo testamento" (pp 54-59, 6 pages)
\end{description}
\end{description}

For the transcription we can equate the subdivisions as follows:
\begin{itemize}
\item Liberi will be represented by LaTeX Chapters
\item Subjects will be represented by LaTeX Subsections
\item Partes will not be directly represented. Maybe by a special command which inserts the Pars text in the body (optional) and the ToC (\verb+\addcontentslline+, in the Section style). Or maybe create a special kind of Chapter that has a Parts thing between Chapter and Section.
\item The bits before Primus Liber will go in the LaTeX Top matter, and be preceded by the \verb+\frontmatter+ command.
\item The Prolegomena is a special Chapter with roman page numbers and no entry in the ToC (accomplished with the \verb+\frontmatter+ command).
\item The four indices and the errata page will be represented as Appendices (precede by \verb+\appendix+).
\item The "Finis" page and the Veterum Graecorum part will be preceded by \verb+\backmatter+.
\end{itemize}


\end{document}
 